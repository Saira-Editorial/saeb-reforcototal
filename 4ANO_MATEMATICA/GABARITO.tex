\chapter{Respostas}
\pagestyle{plain}
\footnotesize

\pagecolor{gray!40}

\section*{Módulo 1 – Treino}

\begin{enumerate}
\item
a) Correta. Todas as ordens estão corretamente representadas e somadas entre si.
b) Incorreta. As quantidades indicadas pelos algarismos 5, 7 e 3 estão erradas.
c) Incorreta. As quantidades indicadas pelos algarismos 5 e 7 estão erradas.
d) Incorreta. As quantidades indicadas pelos algarismos 7 e 3 estão erradas.
SAEB: Compor ou decompor números naturais de até 6 ordens na forma aditiva, ou em suas ordens, ou em adições e multiplicações.
BNCC: EF04MA02 -- Mostrar, por decomposição e composição, que todo número natural pode ser escrito
por meio de adições e multiplicações por potências de dez, para compreender o sistema de
numeração decimal e desenvolver estratégias de cálculo.

\item
a)  Incorreta. Faltou considerar o valor do primeiro X.
b)  Correta. O número formado pelas duas últimas letras (IX), 9, deve ser somado ao número representado pela primeira letra (X), 10: 10 + 9 = 19.
c)  Incorreta. Dessa forma, não se leva em consideração a ordem da escrita.
d)  Incorreta. Faltou considerar o valor do segundo X.
SAEB: Escrever números racionais (naturais de até 6 ordens, representação fracionária ou decimal finita até a ordem dos milésimos) em sua representação por algarismos ou em língua materna ou associar o registro numérico ao registro em língua materna.
BNCC: EF04MA01 -- Ler, escrever e ordenar números naturais até a ordem de dezenas de milhar.

\item
a)  Incorreta. A identificação do valor na placa está correta, mas ele deveria estar como 500.
b)  Incorreta. O valor na placa deveria ser como 500, mas ele está como 5.
c)  Correta. Na placa, está representado o número 125 -- em que o algarismo 5 ocupa a ordem das unidades. O número representado deveria ser o 521 -- em que o algarismo 5 ocuparia a ordem das centenas.
d)  Incorreta. As duas identificações estão incorretas.
SAEB: Identificar a ordem ocupada por um algarismo ou seu valor posicional (ou valor relativo) em um número natural de até 6 ordens.
BNCC: EF04MA01 -- Ler, escrever e ordenar números naturais até a ordem de dezenas de milhar.
\end{enumerate}

\section*{Módulo 2 – Treino}

\begin{enumerate}
\item
a) Incorreta. Cometeu-se um erro na segunda etapa da conta.
b) Correta. 417 – 105 = 312.
c) Incorreta. Cometeu-se um erro na segunda etapa da conta.
d) Incorreta. Todo o procedimento foi errado.
SAEB: Calcular o resultado de adições ou subtrações envolvendo números naturais de até 6 ordens.
BNCC: EF04MA07 -- Resolver e elaborar problemas de divisão cujo divisor tenha no máximo dois algarismos,
envolvendo os significados de repartição equitativa e de medida, utilizando estratégias diversas,
como cálculo por estimativa, cálculo mental e algoritmos.

\item
a) Incorreta. Não foram operacionalizadas primeiro as multiplicações.
b) Incorreta. Cometeu-se um erro na ordem das operações.
c) Incorreta. O subtraendo da conta final foi calculado errado.
d) Correta. 200 -- 2 x (1 x 5 + 2 x 7) = 200 -- 2 x (5 + 14) = 200 -- 2 x 19 = 200 -- 38 = 162 peças.
SAEB: Calcular o resultado de multiplicações ou divisões envolvendo números naturais de até 6 ordens.
BNCC: EF04MA07 -- Resolver e elaborar problemas de divisão cujo divisor tenha no máximo dois algarismos,
envolvendo os significados de repartição equitativa e de medida, utilizando estratégias diversas,
como cálculo por estimativa, cálculo mental e algoritmos.

\item
a) Incorreta. A subtração foi realizada incorretamente.
b) Correta. 1.200 –- 540 = 660 vagas. 932 –- 660 = 272 pessoas não conseguirão assistir à sessão.
c) Incorreta. Só foi calculado o número de lugares vazios.
d) Incorreta. Errou-se na primeira subtração.
SAEB: Resolver problemas de adição ou de subtração, envolvendo números naturais de até 6 ordens, com os significados de juntar, acrescentar, separar, retirar, comparar ou completar.
BNCC: EF04MA07 -- Resolver e elaborar problemas de divisão cujo divisor tenha no máximo dois algarismos,
envolvendo os significados de repartição equitativa e de medida, utilizando estratégias diversas,
como cálculo por estimativa, cálculo mental e algoritmos.
\end{enumerate}

\section*{Módulo 3 – Treino}

\begin{enumerate}
\item
a) Incorreta. O aluno não entendeu a lógica da sequência.
b) Incorreta. O aluno não identificou o padrão.
c) Incorreta. O aluno chegou a um número menor que o da resposta correta.
d) Correta. A sequência é esta: (2; 6; 12; 20; 30; 42).
SAEB: Inferir o padrão ou a regularidade de uma sequência de números naturais ordenados, objetos ou figuras.
BNCC: EF04MA11 -- Identificar regularidades em sequências numéricas compostas por múltiplos de um
número natural.

\item
a) Incorreta. O aluno se confundiu com o antecessor do número que falta.
b) Correta. A sequência aumenta de 100 em 100 unidades.
c) Incorreta. O aluno se confundiu com o sucessor do número que falta.
d) Incorreta. O aluno selecionou, incorretamente, o primeiro número da sequência.
SAEB: Inferir os elementos ausentes em uma sequência de números naturais ordenados, objetos ou figuras.
BNCC: EF04MA11 -- Identificar regularidades em sequências numéricas compostas por múltiplos de um
número natural.

\item
a) Incorreta. Esse é o antecessor do antecessor.
b) Incorreta. Esse é o antecessor.
c) Incorreta. Esse é o sucessor.
d) Correta. O sucessor do sucessor de 7081 é 7081 + 1 + 1 = 7083.
SAEB: Inferir ou descrever atributos ou propriedades comuns que os elementos que constituem uma sequência recursiva de números naturais apresentam.
BNCC: EF04MA11 -- Identificar regularidades em sequências numéricas compostas por múltiplos de um
número natural.
\end{enumerate}

\section*{Módulo 4 – Treino}

\begin{enumerate}
\item
a) Incorreta. O aluno somou uma hora e meia a menos.
b) Incorreta. O aluno somou uma hora a menos.
c) Incorreta. O aluno somou meia hora a menos.
d) Correta. 8 + 4,5 = 12,5 (12:30).
SAEB: Determinar o horário de início, o horário de término ou a duração de um acontecimento.
BNCC: EF04MA22 -- Ler e registrar medidas e intervalos de tempo em horas, minutos e segundos em
situações relacionadas ao seu cotidiano, como informar os horários de início e término de realização
de uma tarefa e sua duração.

\item
a) Incorreta. O aluno chegou a um número menor que a metade do valor da resposta.
b) Incorreta. O aluno chegou a um valor que é maior que a metade do valor da resposta, mas ainda menor que a resposta.
c) Incorreta. O aluno se esqueceu de contar os 80 mL que ainda ficariam de fora da conta.
d) Correta. 4 x 8 x 15 = 480 mL. Como cada frasco possui 100 mL, ela deverá
comprar 5 frascos, e haverá uma sobra de xarope.
SAEB: Estimar/inferir medida de comprimento, capacidade ou massa de objetos, utilizando unidades de medida convencionais ou não ou medir comprimento, capacidade ou massa de objetos.
BNCC: EF04MA20 -- Medir e estimar comprimentos (incluindo perímetros), massas e capacidades, utilizando
unidades de medida padronizadas mais usuais, valorizando e respeitando a cultura local.

\item
a) Correta. Se a partida do voo foi às 10 horas e 42 minutos e a chegada ao destino foi às 14 horas e 8 minutos, o tempo de voo foi de 3 horas e 16 minutos, ou seja, 196 minutos, o que equivale a 11.760 segundos.
b)Incorreta. O aluno errou na subtração entre os dois horários.
c)Incorreta. O aluno errou na conversão para segundos.
d) Incorreta. O aluno errou na subtração entre os dois horários e na conversão para segundos.
SAEB: Resolver problemas que envolvam medidas de grandezas (comprimento, massa, tempo e capacidade) em que haja conversões entre as unidades mais usuais.
BNCC: EF04MA22 -- Ler e registrar medidas e intervalos de tempo em horas, minutos e segundos em
situações relacionadas ao seu cotidiano, como informar os horários de início e término de realização
de uma tarefa e sua duração.
\end{enumerate}

\section*{Módulo 5 – Treino}

\begin{enumerate}
\item
a) Incorreta. O aluno contou errado a quantidade de quadradinhos.
b) Correta. Paulo deverá andar 5 lados de quadrado. Como cada lado de quadrado possui medida igual a 2 m, ele deverá andar 10 metros.
c) Incorreta. O aluno contou errado a quantidade de quadradinhos.
d) Incorreta. O aluno contou errado a quantidade de quadradinhos.
SAEB: Medir ou comparar perímetro ou área de figuras planas desenhadas em malha quadriculada.
BNCC: EF04MA20 -- Medir e estimar comprimentos (incluindo perímetros), massas e capacidades, utilizando
unidades de medida padronizadas mais usuais, valorizando e respeitando a cultura local.

\item
a) Incorreta. O aluno contou errado o número de quadradinhos de cada figura.
b) Incorreta. O aluno contou errado o número de quadradinhos de cada figura.
c) Correta. Todas as figuras possuem 6 quadradinhos de área.
d) Incorreta. O aluno contou errado o número de quadradinhos de cada figura.
SAEB: Resolver problemas que envolvam área de figuras planas
BNCC: EF04MA21 -- Medir, comparar e estimar área de figuras planas desenhadas em malha quadriculada,
pela contagem dos quadradinhos ou de metades de quadradinho, reconhecendo que duas figuras
com formatos diferentes podem ter a mesma medida de área.

\item
a) Incorreta. O aluno não soube interpretar o horário no relógio analógico.
b) Incorreta. O aluno trocou o ponteiro das horas com o dos minutos.
c) Incorreta. O aluno trocou o ponteiro das horas com o dos minutos.
d) Correta. O relógio estava marcando 11 horas e 35 minutos. Acrescentando a esse horário 55 minutos, tem-se no relógio 12 horas e 30 minutos.
SAEB: Identificar horas em relógios analógicos ou associar horas em relógios analógicos e digitais.
BNCC: EF04MA22 -- Ler e registrar medidas e intervalos de tempo em horas, minutos e segundos em
situações relacionadas ao seu cotidiano, como informar os horários de início e término de realização
de uma tarefa e sua duração.
\end{enumerate}

\section*{Módulo 6 – Treino}

\begin{enumerate}
\item
a) Correta. (12 x R\$ 0,50) + (8 x R\$ 0,25) = R\$ 6,00 + R\$ 2,00 = R\$ 8,00 = 4 cédulas de R\$ 2,00.
b) Incorreta. O aluno fez a associação errada dos valores com as quantidades.
c) Incorreta. O aluno só calculou o valor que seria trocado.
d) Incorreta. O aluno só calculou a quantidade de moedas sem considerar o valor.
SAEB: Resolver problemas que envolvam moedas e/ou cédulas do sistema monetário brasileiro.
BNCC: EF04MA25 -- Resolver e elaborar problemas que envolvam situações de compra e venda e formas
de pagamento, utilizando termos como troco e desconto, enfatizando o consumo ético, consciente e
responsável.

\item
a) Incorreta. O aluno se esqueceu de contar as moedas.
b) Incorreta. O aluno não contou todas as moedas.
c) Incorreta. O aluno deixou de contar uma moeda de 5 centavos.
d) Correta. Letícia tem R\$ 9,00 em cédulas e R\$ 1,15 em moedas; portanto, no total, ela
encontrou em sua bolsa R\$ 10,15.
SAEB: Resolver problemas que envolvam moedas e/ou cédulas do sistema monetário brasileiro.
BNCC: EF04MA25 -- Resolver e elaborar problemas que envolvam situações de compra e venda e formas
de pagamento, utilizando termos como troco e desconto, enfatizando o consumo ético, consciente e
responsável.

\item
a) Incorreta. O aluno contou apenas os bombons.
b) Incorreta. O aluno errou nos cálculos.
c) Correta. (4 x 5,00) + (2 x 6,00) + (3 x 12,00) = 20,00 + 12,00 + 36,00 = R\$ 68,00.
d) Incorreta. O aluno contou um bombom a mais.
SAEB: Resolver problemas que envolvam moedas e/ou cédulas do sistema monetário brasileiro.
BNCC: EF04MA25 -- Resolver e elaborar problemas que envolvam situações de compra e venda e formas
de pagamento, utilizando termos como troco e desconto, enfatizando o consumo ético, consciente e
responsável.
\end{enumerate}

\section*{Módulo 7 – Treino}

\begin{enumerate}
\item
a) Incorreta. Não podem ser contados o sábado e domingo.
b) Incorreta. A chance é sempre apenas uma, porque se trata somente da quinta-feira.
c) Correta. Entre os cinco dias possíveis, só a quinta-feira seria escolhida.
d) Incorreta. A chance é sempre apenas uma, porque se trata somente da quinta-feira.
SAEB: Identificar, entre eventos aleatórios, aqueles que têm menos, maiores ou iguais chances de ocorrência, sem utilizar frações.
BNCC: EF04MA26 -- Identificar, entre eventos aleatórios cotidianos, aqueles que têm maior chance de
ocorrência, reconhecendo características de resultados mais prováveis, sem utilizar frações.

\item
a) Incorreta. Há 28 clientes, e apenas 7 garçons.
b) Incorreta. É mais provável escolher um cliente, já que os clientes estão em maior número.
c) Correta. Como há mais clientes que garçons no restaurante, é mais provável que seja escolhido um cliente
d) Incorreta. As chances não são iguais, porque há números diferentes de clientes ou garçons.
SAEB: Identificar, entre eventos aleatórios, aqueles que têm menos, maiores ou iguais chances de ocorrência, sem utilizar frações.
BNCC: EF04MA26 -- Identificar, entre eventos aleatórios cotidianos, aqueles que têm maior chance de
ocorrência, reconhecendo características de resultados mais prováveis, sem utilizar frações.

\item
a) Incorreta. Carlos é terceiro com mais chances.
b) Correta. Márcia tem mais cupons que os outros; portanto é a que tem mais chance de ser sorteada.
c) Incorreta. Paulo é o segundo com mais chances.
d) Incorreta. Rogério é a quarta com mais chances.
SAEB: Identificar, entre eventos aleatórios, aqueles que têm menos, maiores ou iguais chances de ocorrência, sem utilizar frações.
BNCC: EF04MA26 -- Identificar, entre eventos aleatórios cotidianos, aqueles que têm maior chance de
ocorrência, reconhecendo características de resultados mais prováveis, sem utilizar frações.
\end{enumerate}

\section*{Módulo 8 – Treino}

\begin{enumerate}
\item
a) Correta. O candidato A teve todas as notas acima de 30 e é o que teve mais notas iguais.
b) Incorreta. O candidato não teve o maior número de notas iguais, apesar de todas serem acima de 30.
c) Incorreta. O candidato tem uma nota abaixo de 30.
d) Incorreta. O candidato tem duas notas abaixo de 30.
SAEB: Ler/identificar ou comparar dados estatísticos expressos em tabelas (simples ou de dupla entrada).
BNCC: EF04MA27 -- Analisar dados apresentados em tabelas simples ou de dupla entrada e em gráficos de
colunas ou pictóricos, com base em informações das diferentes áreas do conhecimento, e produzir
texto com a síntese de sua análise.

\item
a)  Correta. O número total de alunos do 4º ano é dado por esta conta: 32 + 29 + 25 = 86.
b)  Incorreta. O aluno somou 4º e 5º anos da turma A.
c)  Incorreta. O aluno somou os alunos de 5º ano.
d)  Incorreta. O aluno somou a quantidade total de alunos da escola.
SAEB: Ler/identificar ou comparar dados estatísticos expressos em tabelas (simples ou de dupla entrada).
BNCC: EF04MA27 -- Analisar dados apresentados em tabelas simples ou de dupla entrada e em gráficos de
colunas ou pictóricos, com base em informações das diferentes áreas do conhecimento, e produzir
texto com a síntese de sua análise.

\item
a)  Incorreta. O aluno considerou as crianças de 4 a 6 anos.
b)  Incorreta. O aluno considerou as crianças de 7 a 9 anos.
c)  Incorreta. O aluno se confundiu com os cálculos.
d)  Correta. Segundo o gráfico apresentado, 12 + 9 = 21 crianças de 7 a 12 anos visitaram a loja.
SAEB: Ler/identificar ou comparar dados estatísticos expressos em tabelas (simples ou de dupla entrada).
BNCC: EF04MA27 -- Analisar dados apresentados em tabelas simples ou de dupla entrada e em gráficos de
colunas ou pictóricos, com base em informações das diferentes áreas do conhecimento, e produzir
texto com a síntese de sua análise.
\end{enumerate}

\section*{Módulo 9 – Treino}

\begin{enumerate}
\item
a) Incorreta. Não deve ser representado o todo.
b) Incorreta. O aluno não numerou bem nem o total de bombons nem a quantidade de cada tipo.
c) Correta. Cada tipo de bombom tem $\frac{3}{6}$ do total de bombons na caixa, ou seja, $\frac{1}{2}$.
d) Incorreta. São 3, e não 4 bombons de cada tipo.
SAEB: Representar frações menores ou maiores que a unidade (por meio de representações
pictóricas) ou associar frações a representações pictóricas.
BNCC: EF04MA09 -- Reconhecer as frações unitárias mais usuais ($\frac{1}{2}$, $\frac{1}{3}$, $\frac{1}{4}$, $\frac{1}{5}$, $\frac{1}{10}$ e $\frac{1}{100}$) como
unidades de medida menores do que uma unidade, utilizando a reta numérica como recurso.

\item
a)  Incorreta. Não temos divisões iguais em 3 partes.
b)  Incorreta. Não temos divisões iguais em 4 partes.
c)  Correta. Há quatro divisões iguais de um todo.
d)  Incorreta. Não temos divisão em 3 partes iguais.
SAEB: Representar frações menores ou maiores que a unidade (por meio de representações
pictóricas) ou associar frações a representações pictóricas.
BNCC: EF04MA09 -- Reconhecer as frações unitárias mais usuais ($\frac{1}{2}$, $\frac{1}{3}$, $\frac{1}{4}$, $\frac{1}{5}$, $\frac{1}{10}$ e $\frac{1}{100}$) como
unidades de medida menores do que uma unidade, utilizando a reta numérica como recurso.

\item
a)  Incorreta. $\frac{1}{4}$ de 48 não é igual a 8.
b)  Correta. $\frac{1}{4}$ de 48 é igual a 12.
c)  Incorreta. $\frac{1}{4}$ de 48 não é igual a 20.
d)  Incorreta. $\frac{1}{4}$ de 48 não é igual a 28.
SAEB: Representar frações menores ou maiores que a unidade (por meio de representações
pictóricas) ou associar frações a representações pictóricas.
BNCC: EF04MA09 -- Reconhecer as frações unitárias mais usuais ($\frac{1}{2}$, $\frac{1}{3}$, $\frac{1}{4}$, $\frac{1}{5}$, $\frac{1}{10}$ e $\frac{1}{100}$) como
unidades de medida menores do que uma unidade, utilizando a reta numérica como recurso.
\end{enumerate}

\section*{Módulo 10 – Treino}

\begin{enumerate}
\item
a)  Incorreta. Esse é o valor médio de uma única pizza.
b)  Incorreta. Esse já é o valor das duas pizzas.
c)  Incorreta. Os cálculos foram feitos incorretamente.
d)  Correta.  Valor médio de cada pizza: R\$ 81,60 : 2 = R\$ 40,80. Valor médio de 6 pizzas: 6 x 40,80 = R\$ 244,80.
SAEB: Resolver problemas que envolvam variação de proporcionalidade direta entre duas grandezas.
BNCC: EF04MA06 -- Resolver e elaborar problemas envolvendo diferentes significados da multiplicação
(adição de parcelas iguais, organização retangular e proporcionalidade), utilizando estratégias
diversas, como cálculo por estimativa, cálculo mental e algoritmos.

\item
a) Incorreta. A quantidade de café não será três vezes maior apenas.
b) Correta. Para 48 cafezinhos, ela fará 6 receitas. Sendo assim, basta
multiplicar a quantidade de colheres de pó de café para 8 cafezinhos por
6: 3 x 6 = 18 colheres de pó de café.
c) Incorreta. A quantidade de café não será seis vezes maior.
d) Incorreta. A quantidade de café não será dezesseis vezes maior.
SAEB: Resolver problemas que envolvam variação de proporcionalidade direta entre duas grandezas.
BNCC: EF04MA06 -- Resolver e elaborar problemas envolvendo diferentes significados da multiplicação
(adição de parcelas iguais, organização retangular e proporcionalidade), utilizando estratégias
diversas, como cálculo por estimativa, cálculo mental e algoritmos.

\item
a) Incorreta. Será preciso imprimir 3.500 folhas, mas, em um minuto, são impressas apenas 10 folhas.
b) Incorreta. Será preciso imprimir 3.500 folhas, mas, em quinze minutos, são impressas apenas 1.500 folhas.
c) Correta. Quantidade de folhas para 700 jornais: 5 x 700 = 3.500 folhas.
Tempo gasto para a produção de 3.500 folhas = 3.500 : 100 = 35 minutos.
d) Incorreta. Em cinquenta e cinco minutos, serão produzidas mais de 3.500 folhas.
SAEB: Resolver problemas que envolvam variação de proporcionalidade direta entre duas grandezas.
BNCC: EF04MA06 -- Resolver e elaborar problemas envolvendo diferentes significados da multiplicação
(adição de parcelas iguais, organização retangular e proporcionalidade), utilizando estratégias
diversas, como cálculo por estimativa, cálculo mental e algoritmos.
\end{enumerate}

\section*{Módulo 11 – Treino}

\begin{enumerate}
\item
a) Incorreta. Esse é o número de cores sugeridas.
b) Incorreta. Em cada espaço, só uma cor diferente pode ser usada.
c) Correta. Deve-se multiplicar o número 12 (cores disponíveis para um dos espaços) pelo número 11 (cores disponíveis para o outro espaço, sem uma cor, usada no primeiro espaço), o que resulta em 132 composições diversas.
d) Incorreta. Esse número é o resultado do número 12 multiplicado por ele mesmo, mas, no segundo espaço a ser colorido, uma das cores (a já usada no primeiro espaço) não pode ser repetida.
SAEB: Resolver problemas simples de contagem (combinatória).
BNCC: EF04MA08 -- Resolver, com o suporte de imagem e/ou material manipulável, problemas simples
de contagem, como a determinação do número de agrupamentos possíveis ao se combinar cada
elemento de uma coleção com todos os elementos de outra, utilizando estratégias e formas de
registro pessoais.

\item
a) Correta. Multiplica-se o número 4 (opções de cobertura) pelo número 20 (opções de sabor de sorvete), o que resulta em 80 combinações diferentes.
b) Incorreta. O aluno multiplicou o número 4 por seu dobro.
c) Incorreta. O aluno multiplicou o número de opções de cobertura por três apenas.
d) Incorreta. O aluno se confundiu com o número de opções de sabor de sorvete.
SAEB: Resolver problemas simples de contagem (combinatória).
BNCC: EF04MA08 -- Resolver, com o suporte de imagem e/ou material manipulável, problemas simples
de contagem, como a determinação do número de agrupamentos possíveis ao se combinar cada
elemento de uma coleção com todos os elementos de outra, utilizando estratégias e formas de
registro pessoais.

\item
a) Incorreta. O aluno se esqueceu dos caminhos saindo de X e passando por Y antes de chegar a Z.
b) Correta. Caminhos passando só por S: 3 x 2 = 6; caminhos passando por S e Y: 3 x 2 x 2 = 12; caminhos passando só por Y: 1 x 2 = 2; caminhos passando só por R: 3 x 1 = 3; caminhos passando por R e Y: 3 x 3 x 2 = 18. Total: 6 + 12 + 2 + 3 + 18 = 41 caminhos diferentes.
c) Incorreta. O aluno se esqueceu dos caminhos saindo de X e passando por S antes de chegar a Z.
d) Incorreta. O aluno se confundiu na somatória do total.
SAEB: Resolver problemas simples de contagem (combinatória).
BNCC: EF04MA08 -- Resolver, com o suporte de imagem e/ou material manipulável, problemas simples
de contagem, como a determinação do número de agrupamentos possíveis ao se combinar cada
elemento de uma coleção com todos os elementos de outra, utilizando estratégias e formas de
registro pessoais.
\end{enumerate}

\section*{Simulado 1}

\begin{enumerate}
\item
a) Correta. (6 x 100) + (7 x 10) + (5 x 1) = 675.
b) Incorreta. O aluno se esqueceu de contar 250 unidades.
c) Incorreta. O aluno se esqueceu de contar 170 unidades.
d) Incorreta. O aluno se esqueceu de contar 150 unidades.
SAEB: Compor ou decompor números naturais de até 6 ordens na forma aditiva, ou
em suas ordens, ou em adições e multiplicações.
BNCC: EF04MA01 -- Ler, escrever e ordenar números naturais até a ordem de dezenas de milhar.

\item
a) Incorreta. O aluno se confundiu com as unidades.
b) Incorreta. O aluno se esqueceu da dezena de milhar.
c) Incorreta. O aluno se confundiu com os milhares e com a dezena de milhar.
d) Correta. (3 x 10.000) + (2 x 1.000) + (5 x 100) + (7 x 10) = 32.570.
SAEB: Compor ou decompor números naturais de até 6 ordens na forma aditiva, ou
em suas ordens, ou em adições e multiplicações.
BNCC: EF04MA01 -- Ler, escrever e ordenar números naturais até a ordem de dezenas de milhar.

\item
a) Incorreta. Esse é o valor relativo na posição em que ele está na placa errada.
b) Incorreta. O aluno confundiu a centena comum com a dezena comum.
c) Correta. Como o número apresentado no enunciado está com o primeiro e o último
algarismo trocados, conclui-se que o número correto seria 785, que tem o algarimos 7 com o valor relativo de 700 (centena comum).
d) Incorreta. O número apresentado não tem quatro ordens.
SAEB: Identificar a ordem ocupada por um algarismo ou seu valor posicional (ou
valor relativo) em um número natural de até 6 ordens.
BNCC: EF04MA01 -- Ler, escrever e ordenar números naturais até a ordem de dezenas de milhar.

\item
a) Incorreta. O aluno realizou a subtração no lugar da adição.
b) Incoreta. O aluno errou no cálculo da adição.
c) Correta. 1.359 + 1.246 = 2.605.
d) Incorreta. O aluno se confundiu com os termos da adição.
SAEB: Resolver problemas de adição ou de subtração, envolvendo números
naturais de até 6 ordens, com os significados de juntar, acrescentar,
separar, retirar, comparar ou completar.
BNCC: EF04MA03 -- Resolver e elaborar problemas com números naturais envolvendo adição e subtração,
utilizando estratégias diversas, como cálculo, cálculo mental e algoritmos, além de fazer estimativas
do resultado.

\item
a) Incorreta. O aluno errou nos cálculos.
b) Incorreta. O aluno errou na subtração.
c) Correta. Como o aumento foi o mesmo nos dois números, não
precisamos fazer a soma do aumento aos números antigos já que a
diferença entre eles irá se manter: 3.452 -- 1.834 = 1.618.
d) Incorreta. O aluno se confundiu com os termos da subtração.
SAEB: Resolver problemas de adição ou de subtração, envolvendo números
naturais de até 6 ordens, com os significados de juntar, acrescentar,
separar, retirar, comparar ou completar.
BNCC: EF04MA03 -- Resolver e elaborar problemas com números naturais envolvendo adição e subtração,
utilizando estratégias diversas, como cálculo, cálculo mental e algoritmos, além de fazer estimativas
do resultado.

\item
a) Incorreta. O aluno se confundiu com os termos da divisão.
b) Incorreta. O aluno se confundiu com os termos da divisão.
c) Incorreta. O aluno errou no cálculo na divisão.
d) Correta. Foram comprados 18 L = 18.000 mL de refrigerante.
Número máximo de copos que poderão ser servidos: 18.000 : 250 = 72.
SAEB: Resolver problemas que envolvam medidas de grandezas
(comprimento, massa, tempo e capacidade) em que haja conversões entre as
unidades mais usuais.
BNCC: EF04MA20 -- Medir e estimar comprimentos (incluindo perímetros), massas e capacidades, utilizando
unidades de medida padronizadas mais usuais, valorizando e respeitando a cultura local.

\item
a) Incorreta. O aluno contou errado o número de quadradinhos.
b) Incorreta. O aluno contou errado o número de quadradinhos.
c) Incorreta. O aluno se esqueceu de multiplicar por 2 o número total de quadradinhos.
d) Correta. São 29 quadradinhos pintados ao todo: 29 x 2 = 58 centímetros quadrados.
SAEB: Medir ou comparar perímetro ou área de figuras planas
desenhadas em malha quadriculada.
BNCC: EF04MA21 -- Medir, comparar e estimar área de figuras planas desenhadas em malha quadriculada,
pela contagem dos quadradinhos ou de metades de quadradinho, reconhecendo que duas figuras
com formatos diferentes podem ter a mesma medida de área.

\item
a) Incorreta. O aluno errou na contagem dos valores.
b) Incorreta. O aluno errou no cálculo da adição.
c) Correta. Ana Beatriz possui R\$ 87,70; Camila possui R\$ 136,30. Juntas, possuem: 87,70 + 136,30 = R\$ 224,00.
d) Incorreta. O aluno errou no cálculo da adição.
SAEB: Relacionar valores de moedas e/ou cédulas do sistema
monetário brasileiro, com base nas imagens desses objetos.
BNCC: EF04MA25 -- Resolver e elaborar problemas que envolvam situações de compra e venda e formas
de pagamento, utilizando termos como troco e desconto, enfatizando o consumo ético, consciente e
responsável.

\item
a) Incorreta. Obrigatoriamente, nascerá um menino ou uma menina.
b) Incorreta. As chances são iguais.
c) Correta. Como nesse evento só temos duas possibilidades equiprováveis (menino ou
menina), a probabilidade de Isabele ter um irmão é de 50\%.
d) Incorreta. Não se pode ter certeza sobre o sexo da criança antes do exame.
SAEB: Determinar a probabilidade de ocorrência de um
resultado em eventos aleatórios, quando todos os resultados possíveis
têm a mesma chance de ocorrer (equiprováveis).
BNCC: EF04MA26 -- Identificar, entre eventos aleatórios cotidianos, aqueles que têm maior chance de
ocorrência, reconhecendo características de resultados mais prováveis, sem utilizar frações.

\item
a) Correta. A maior taxa foi 8,6\%, que é a de janeiro.
b) Incorreta. A taxa de julho (7,7\%) foi menor que a de janeiro.
c) Incorreta. A taxa de outubro (6,8\%) foi a menor do ano.
d) Incorreta. A taxa de dezembro (8,5\%) foi a segunda maior.
SAEB: Resolver problemas que envolvam dados apresentados
tabelas (simples ou de dupla entrada) ou gráficos estatísticos (barras
simples ou agrupadas, colunas simples ou agrupadas, pictóricos ou de
linhas).
BNCC: EF04MA27 -- Analisar dados apresentados em tabelas simples ou de dupla entrada e em gráficos de
colunas ou pictóricos, com base em informações das diferentes áreas do conhecimento, e produzir
texto com a síntese de sua análise.

\item
a) Incorreta. A reta está dividida em 5 partes de $\frac{1}{5}$.
b) Correta. Como o tamanho total está dividido em 5 partes iguais, a fração entre
duas partes consecutivas será $\frac{1}{5}$. Como teremos duas partes, a fração
será $\frac{2}{5}$.
c) Incorreta. A reta não está divivida em três partes iguais.
d) Incorreta. A resta está dividida em 5 partes de $\frac{1}{5}$, mas devem-se contar duas partes.
SAEB: Representar frações menores ou maiores que a unidade
(por meio de representações pictóricas) ou associar frações a
representações pictóricas.
BNCC: EF04MA09 -- Reconhecer as frações unitárias mais usuais ($\frac{1}{2}$, $\frac{1}{3}$, $\frac{1}{4}$, $\frac{1}{5}$, $\frac{1}{10}$ e $\frac{1}{100}$) como
unidades de medida menores do que uma unidade, utilizando a reta numérica como recurso.

\item
a) Incorreta. O aluno contou uma hora a menos que o valor correto.
b) Incorreta. O aluno se confundiu com a proporcionalidade.
c) Incorreta. O aluno se confundiu com a proporcionalidade.
d) Correta. Como na situação inicial ele percorre 80 km em uma hora, em uma hora e
meia percorrerá 120 km. Sendo assim, andando a 50 km/h, ele percorrerá 50 km em uma hora;
mantendo-se a proporção, conseguirá percorrer 120 km em 2 horas e 24
minutos.
SAEB: Resolver problemas que envolvam variação de
proporcionalidade direta entre duas grandezas.
BNCC: EF04MA22 -- Ler e registrar medidas e intervalos de tempo em horas, minutos e segundos em
situações relacionadas ao seu cotidiano, como informar os horários de início e término de realização
de uma tarefa e sua duração.

\item
a) Incorreta. O aluno não compreendeu o princípio da contagem.
b) Incorreta. O aluno errou no cálculo da multiplicação.
c) Incorreta. O aluno errou no cálculo da multiplicação.
d) Correta. Multiplica-se o número de camisetas pelo número de bermudas: 32 x 15 = 480.
SAEB: Resolver problemas simples de contagem (combinatória).
BNCC: EF04MA08 -- Resolver, com o suporte de imagem e/ou material manipulável, problemas simples
de contagem, como a determinação do número de agrupamentos possíveis ao se combinar cada
elemento de uma coleção com todos os elementos de outra, utilizando estratégias e formas de
registro pessoais.

\item
a) Incorreta. O aluno errou no cálculo da divisão.
b) Incorreta. O aluno errou no cálculo da divisão.
c) Incorreta. O aluno errou no cálculo da divisão.
d) Correta. Quantidade de litros de suco de laranja que serão colocados em cada
tambor: 141,1 : 17 = 8,3 litros, o que corresponde a 8 litros e 300 mililitros.
SAEB: Resolver problemas de multiplicação ou de divisão,
envolvendo números naturais de até 6 ordens, com os significados de
formação de grupos iguais (incluindo repartição equitativa e medida),
proporcionalidade ou disposição retangular.
BNCC: EF04MA07 -- Resolver e elaborar problemas de divisão cujo divisor tenha no máximo dois algarismos,
envolvendo os significados de repartição equitativa e de medida, utilizando estratégias diversas,
como cálculo por estimativa, cálculo mental e algoritmos.

\item
a) Incorreta. O primeiro número é este: quinze mil, seiscentos e dez.
b) Correta. O número 3.456 é escrito assim: três mil, quatrocentos e cinquenta e seis.
c) Incorreta. O terceiro número é este: mil, duzentos e setenta e oito.
d) Incorreta. O quarto número é este: dez mil, trezentos e vinte e um.
SAEB: Escrever números racionais (naturais de até 6 ordens, representação
fracionária ou decimal finita até a ordem dos milésimos) em sua
representação por algarismos ou em língua materna OU associar o registro
numérico ao registro em língua materna.
BNCC: EF04MA01 -- Ler, escrever e ordenar números naturais até a ordem de dezenas de milhar.
\end{enumerate}


\section*{Simulado 2}

\begin{enumerate}
\item
a) Incorreta. O aluno não reconheceu a dezena de milhar.
b) Incorreta. O aluno não reconheceu a dezena de milhar.
c) Incorreta. O aluno contou incorretamente as dezenas de milhar.
d) Correta. (3 x 10.000) + (1 x 1.000) + (3 x 100) + (1 x 10) + (4 x 1) = 31.314.
SAEB: Compor ou decompor números naturais de até 6 ordens na
forma aditiva, ou em suas ordens, ou em adições e multiplicações.
BNCC: EF04MA02 -- Mostrar, por decomposição e composição, que todo número natural pode ser escrito
por meio de adições e multiplicações por potências de dez, para compreender o sistema de
numeração decimal e desenvolver estratégias de cálculo.

\item
a) Incorreta. O número 364 é maior que 200.
b) Incorreta. O número 364 é maior que 300.
c) Correta. Seguindo-se a sequência da reta numérica, conclui-se que o número 364 deverá
ser colocado entre o 350 e o 400.
d) Incorreta. O número 364 é menor que 450.
SAEB: Comparar ou ordenar números
racionais (naturais de até 6 ordens, representação fracionária ou
decimal finita até a ordem dos milésimos), com ou sem suporte da reta
numérica.
BNCC: EF04MA01 -- Ler, escrever e ordenar números naturais até a ordem de dezenas de milhar.

\item
a) Incorreta. O aluno errou no cálculo da divisão.
b) Incorreta. Incorreta. O aluno errou no cálculo da divisão.
c) Correta. Número de caixas que serão produzidas é numericamente igual ao número de
escolas que receberão as caixas. Sendo assim, 26.104 : 52 = 502 escolas.
d) Incorreta. O aluno errou no cálculo da divisão.
SAEB: Resolver problemas de multiplicação ou de divisão,
envolvendo números naturais de até 6 ordens, com os significados de
formação de grupos iguais (incluindo repartição equitativa e medida),
proporcionalidade ou disposição retangular.
BNCC: EF04MA07 -- Resolver e elaborar problemas de divisão cujo divisor tenha no máximo dois algarismos,
envolvendo os significados de repartição equitativa e de medida, utilizando estratégias diversas,
como cálculo por estimativa, cálculo mental e algoritmos.

\item
a) Correta. A sequência diminui de 60 em 60 unidades.
b) Incorreta. Essa é a razão da sequência, e não o próximo termo.
c) Incorreta. O aluno encontrou, erroneamente, o sexto número.
d) Incorreta. O aluno não compreendeu o conceito.
SAEB: Inferir o padrão ou a regularidade de uma sequência de
números naturais ordenados, objetos ou figuras.
BNCC: EF04MA11 -- Identificar regularidades em sequências numéricas compostas por múltiplos de um
número natural.

\item
a) Correta. (7 semanas x 7 dias) + 3 dias = 52 dias.
b) Incorreta. O aluno errou nos cálculos.
c) Incorreta. O aluno errou nos cálculos.
d) Incorreta. O aluno errou nos cálculos.
SAEB: Determinar o horário de início, o horário de término ou
a duração de um acontecimento.
BNCC: EF04MA22 -- Ler e registrar medidas e intervalos de tempo em horas, minutos e segundos em
situações relacionadas ao seu cotidiano, como informar os horários de início e término de realização
de uma tarefa e sua duração.

\item
a) Incorreta. O aluno não soube fazer a multiplicação e a conversão de medidas.
b) Incorreta. O aluno não soube fazer a multiplicação e a conversão de medidas.
c) Correta. 7 x 21 = 147 cm. Portanto aproximadamente 1,5 m.
d) Incorreta. O aluno não soube fazer a multiplicação e a conversão de medidas.
SAEB: Estimar/inferir medida de comprimento, capacidade ou
massa de objetos, utilizando unidades de medida convencionais ou não ou
medir comprimento, capacidade ou massa de objetos.
BNCC: EF04MA20 -- Medir e estimar comprimentos (incluindo perímetros), massas e capacidades, utilizando
unidades de medida padronizadas mais usuais, valorizando e respeitando a cultura local.

\item
a) Correta. Quantidade de lados de quadrados que terão fita: 10. Como cada lado do piso mede 1,20 m, ele precisará de 1,20 metro x 10 quadrados x 3 voltas = 36 metros de fita.
b) Incorreta. O aluno deixou de dar uma volta.
c) Incorreta. O aluno deixou de dar duas voltas.
d) Incorreta. O aluno deu só meia volta.
SAEB: Medir ou comparar perímetro ou área de figuras planas
desenhadas em malha quadriculada.
BNCC: EF04MA20 -- Medir e estimar comprimentos (incluindo perímetros), massas e capacidades, utilizando
unidades de medida padronizadas mais usuais, valorizando e respeitando a cultura local.

\item
a) Incorreta. O aluno errou no cálculo da adição.
b) Incorreta. Incorreta. O aluno errou no cálculo da adição.
c) Correta. A soma é esta: 103,00 + 59,00 = R\$ 162,00.
d) Incorreta. O aluno errou no cálculo da adição.
SAEB: Resolver problemas que envolvam moedas e/ou cédulas do
sistema monetário brasileiro.
BNCC: EF04MA25 -- Resolver e elaborar problemas que envolvam situações de compra e venda e formas
de pagamento, utilizando termos como troco e desconto, enfatizando o consumo ético, consciente e
responsável.

\item
a) Correta. Não há, no dado, número igual a 0 ou menor que 0.
b) Incorreta. Não há probabilidade de esse evento ocorrer.
c) Incorreta. Não há probabilidade de esse evento ocorrer.
d) Incorreta. Não há probabilidade de esse evento ocorrer.
SAEB: Determinar a probabilidade de ocorrência de um
resultado em eventos aleatórios, quando todos os resultados possíveis
têm a mesma chance de ocorrer (equiprováveis).
BNCC: EF04MA26 -- Identificar, entre eventos aleatórios cotidianos, aqueles que têm maior chance de
ocorrência, reconhecendo características de resultados mais prováveis, sem utilizar frações.

\item
a) Incorreta. O aluno X será o primeiro colocado.
b) Incorreta. O aluno Y será o segundo colocado.
c) Correta. Soma das notas do aluno X: 31; soma das notas do aluno Y: 30; soma das notas do aluno Z: 29.
d) Incorreta. Não haverá empate.
SAEB: Resolver problemas que envolvam dados apresentados
tabelas (simples ou de dupla entrada) ou gráficos estatísticos (barras
simples ou agrupadas, colunas simples ou agrupadas, pictóricos ou de
linhas).
BNCC: EF04MA27 -- Analisar dados apresentados em tabelas simples ou de dupla entrada e em gráficos de
colunas ou pictóricos, com base em informações das diferentes áreas do conhecimento, e produzir
texto com a síntese de sua análise.

\item
a) Incorreta. O aluno errou nos cálculos.
b) Correta. Para consumir $\frac{2}{5}$ da barra ele terá de consumir $\frac{2}{5}$ de 20 quadradinhos, ou seja, 8 quadradinhos.
c) Incorreta. O aluno errou nos cálculos.
d) Incorreta. O aluno errou nos cálculos.
SAEB: Representar frações menores ou maiores que a unidade
(por meio de representações pictóricas) ou associar frações a
representações pictóricas.
BNCC: EF04MA09 -- Reconhecer as frações unitárias mais usuais ($\frac{1}{2}$, $\frac{1}{3}$, $\frac{1}{4}$, $\frac{1}{5}$, $\frac{1}{10}$ e $\frac{1}{100}$) como
unidades de medida menores do que uma unidade, utilizando a reta numérica como recurso.

\item
a) Incorreta. A receita não é o dobro.
b) Incorreta. A receita não é o triplo.
c) Incorreta. A receita não é o quádruplo.
d) Correta. Como 1.250 mL é igual a 5 x 250, conclui-se que, pela proporção, serão necessárias 5 x 1 = 5 colheres de pó de café.
SAEB: Resolver problemas que envolvam variação de
proporcionalidade direta entre duas grandezas.
BNCC: EF04MA06 -- Resolver e elaborar problemas envolvendo diferentes significados da multiplicação
(adição de parcelas iguais, organização retangular e proporcionalidade), utilizando estratégias
diversas, como cálculo por estimativa, cálculo mental e algoritmos.

\item
a) Incorreta. O aluno não compreendeu o conceito.
b) Incorreta. O aluno se confundiu com os termos da multiplicação.
c) Correta. 22 (todas as possibilidades) x 21 (todas as possibilidades menos o primeiro lugar) x 20 (todas as possibilidades menos o primeiro e o segundo lugar) = 9.240.
d) Incorreta. O aluno errou nos cálculos da multiplicação.
SAEB: Resolver problemas simples de contagem (combinatória).
BNCC: EF04MA08 -- Resolver, com o suporte de imagem e/ou material manipulável, problemas simples
de contagem, como a determinação do número de agrupamentos possíveis ao se combinar cada
elemento de uma coleção com todos os elementos de outra, utilizando estratégias e formas de
registro pessoais.

\item
a) Incorreta. O aluno errou no cálculo da divisão.
b) Incorreta. O aluno errou no cálculo da divisão.
c) Incorreta. O aluno errou no cálculo da divisão.
d) Correta. Quantidade de litros que Ricardo colocou: R\$ 208,00 : R\$ 4,00 = 52
litros.
SAEB: Resolver problemas de multiplicação ou de divisão,
envolvendo números racionais apenas na representação decimal finita até
a ordem dos milésimos, com os significados de formação de grupos iguais
(incluindo repartição equitativa de medida), proporcionalidade ou
disposição retangular.
BNCC: EF04MA07 -- Resolver e elaborar problemas de divisão cujo divisor tenha no máximo dois algarismos,
envolvendo os significados de repartição equitativa e de medida, utilizando estratégias diversas,
como cálculo por estimativa, cálculo mental e algoritmos.

\item
a) Incorreta. Esse é o número sete mil e cinco.
b) Correta. A representação em algarismos está correta.
c) Incorreta. Esse é o número treze mil, quatrocentos e oitenta.
d) Incorreta. Esse é o número dezesseis mil, oitocentos e quarenta.
SAEB: Escrever números racionais (naturais de até 6 ordens, representação
fracionária ou decimal finita até a ordem dos milésimos) em sua
representação por algarismos ou em língua materna OU associar o registro
numérico ao registro em língua materna.
BNCC: EF04MA01 -- Ler, escrever e ordenar números naturais até a ordem de dezenas de milhar.
\end{enumerate}

\section*{Simulado 3}

\begin{enumerate}
\item
a) Incorreta. O aluno inverteu os números.
b) Incorreta. O aluno se esqueceu de algumas ordens.
c) Incorreta. O aluno se esqueceu de algumas ordens.
d) Correta. 50.000 + 3.000 + 300 + 50 = 53.350.
SAEB: Compor ou decompor números naturais de até 6 ordens na
forma aditiva, ou em suas ordens, ou em adições e multiplicações.
BNCC: EF04MA02 -- Mostrar, por decomposição e composição, que todo número natural pode ser escrito
por meio de adições e multiplicações por potências de dez, para compreender o sistema de
numeração decimal e desenvolver estratégias de cálculo.

\item
a) Incorreta. O aluno não compreendeu o princípio.
b) Correta. Basta colocar os algarismos dados em ordem decrescente.
c) Incorreta. O aluno não compreendeu o princípio.
d) Incorreta. O aluno não compreendeu o princípio.
SAEB: Compor ou decompor números naturais de até 6 ordens na
forma aditiva, ou em suas ordens, ou em adições e multiplicações.
BNCC: EF04MA02 -- Mostrar, por decomposição e composição, que todo número natural pode ser escrito
por meio de adições e multiplicações por potências de dez, para compreender o sistema de
numeração decimal e desenvolver estratégias de cálculo.

\item
a) Incorreta. O número 2 não resolve o problema.
b) Incorreta. O número 6 não resolve o problema.
c) Incorreta. O número 7 não resolve o problema.
d) Correta. O número escondido que torna a conta correta é o 8.
SAEB: Calcular o resultado de multiplicações ou divisões
envolvendo números naturais de até 6 ordens.
BNCC: EF04MA04 -- Utilizar as relações entre adição e subtração, bem como entre multiplicação e divisão,
para ampliar as estratégias de cálculo.

\item
a) Incorreta. O aluno não compreendeu a lei de formação da sequência.
b) Incorreta. O aluno não compreendeu a lei de formação da sequência.
c) Correta. Observa-se, na sequência, que para descobrir um termo basta
multiplicar por 3 seu antecessor.
d) Incorreta. O aluno não compreendeu a lei de formação da sequência.
SAEB: Inferir o padrão ou a regularidade de uma sequência de
números naturais ordenados, objetos ou figuras.
BNCC: EF04MA11 -- Identificar regularidades em sequências numéricas compostas por múltiplos de um
número natural.

\item
a) Incorreta. O aluno se esqueceu dos 15 minutos.
b) Incorreta. O aluno se confundiu com o tempo de abertura do zoológico.
c) Correta. 9,25 + 7,5 = 16,75 = 16 horas e 45 minutos.
d) Incorreta. O aluno se confundiu com o tempo de abertura do zoológico.
SAEB: Determinar o horário de início, o horário de término ou
a duração de um acontecimento.
BNCC: EF04MA22 -- Ler e registrar medidas e intervalos de tempo em horas, minutos e segundos em
situações relacionadas ao seu cotidiano, como informar os horários de início e término de realização
de uma tarefa e sua duração.

\item
a) Incorreta. O aluno não soube operacionalizar os números.
b) Incorreta. O aluno não soube operacionalizar os números.
c) Incorreta. O aluno não soube operacionalizar os números.
d) Correta. A diferença entre 13:55 e 15:39 é igual a 1 hora e 44 minutos.
SAEB: Determinar o horário de início, o horário de término ou
a duração de um acontecimento.
BNCC: EF04MA22 -- Ler e registrar medidas e intervalos de tempo em horas, minutos e segundos em
situações relacionadas ao seu cotidiano, como informar os horários de início e término de realização
de uma tarefa e sua duração.

\item
a) Incorreta. O aluno contou errado os quadradinhos.
b) Incorreta. O aluno contou errado os quadradinhos.
c) Correta. Contando-se o número de quadradinhos que representam o tapete, chega-se a 18. Portanto: 18 quadradinhos x 1 metro quadrado = 18 metros quadrados de carpete.
d) Incorreta. O aluno contou errado os quadradinhos.
SAEB: Resolver problemas que envolvam área de figuras planas.
BNCC: EF04MA21 -- Medir, comparar e estimar área de figuras planas desenhadas em malha quadriculada,
pela contagem dos quadradinhos ou de metades de quadradinho, reconhecendo que duas figuras
com formatos diferentes podem ter a mesma medida de área.

\item
a) Incorreta. Existe 50\% de chance de sair um número par.
b) Correta. Metade dos números de 1 a 50 são pares, enquanto a outra metade é de número ímpares; portanto as chances são iguais.
c) Incorreta. As chances são as mesmas.
d) Incorreta. Pode sair um número ímpar, com a mesma probabilidade.
SAEB: Determinar a probabilidade de ocorrência de um
resultado em eventos aleatórios, quando todos os resultados possíveis
têm a mesma chance de ocorrer (equiprováveis).
BNCC: EF04MA26 -- Identificar, entre eventos aleatórios cotidianos, aqueles que têm maior chance de
ocorrência, reconhecendo características de resultados mais prováveis, sem utilizar frações.

\item
a) Incorreta. Na disciplina I, o estudante ficou com nota maior que 6,00.
b) Incorreta. Na disciplina II, o estudante ficou com nota maior que 6,00.
c) Correta. O estudante ficou com 6,00 na disciplina III.
d) Incorreta. Na disciplina IV, o estudante ficou com nota maior que 6,00.
SAEB: Argumentar ou analisar argumentações/conclusões com
base em dados apresentados em tabelas (simples ou de dupla entrada) ou
gráficos (barras simples ou agrupadas, colunas simples ou agrupadas,
pictóricos ou de linhas).
BNCC: EF04MA27 -- Analisar dados apresentados em tabelas simples ou de dupla entrada e em gráficos de
colunas ou pictóricos, com base em informações das diferentes áreas do conhecimento, e produzir
texto com a síntese de sua análise.

\item
a) Incorreta. Os das raias 1 e 8 não estão entre os mais velozes.
b) Correta. Os mais velozes são o que fizeram a prova em menor tempo, ou seja os das raias 3, 5 e 6.
c) Incorreta. Esses não são os três mais velozes.
d) Incorreta. O da raia não está entre os mais velozes.
SAEB: Resolver problemas que envolvam dados apresentados tabelas (simples ou
de dupla entrada) ou gráficos estatísticos (barras simples ou agrupadas,
colunas simples ou agrupadas, pictóricos ou de linhas).
BNCC: EF04MA27 -- Analisar dados apresentados em tabelas simples ou de dupla entrada e em gráficos de
colunas ou pictóricos, com base em informações das diferentes áreas do conhecimento, e produzir
texto com a síntese de sua análise.

\item
a) Incorreta. Nenhum deles vai receber esse valor.
b) Incorreta. Nenhum deles vai receber esse valor.
c) Correta. O segundo colocado vai receber (1 x 900) : 5 = R\$ 180,00.
d) Incorreta. Nenhum deles vai receber esse valor.
SAEB: Resolver problemas que envolvam fração como resultado
de uma divisão (quociente).
BNCC: EF04MA09 -- Reconhecer as frações unitárias mais usuais ($\frac{1}{2}$, $\frac{1}{3}$, $\frac{1}{4}$, $\frac{1}{5}$, $\frac{1}{10}$ e $\frac{1}{100}$) como
unidades de medida menores do que uma unidade, utilizando a reta numérica como recurso.

\item
a) Correta. Ele acertou em metade das tentativas: $\frac{5}{10}$ = $\frac{1}{2}$.
b) Incorreta. Ele acertou mais que em 25\% das tentativas.
c) Incorreta. Ele acertou menos que em $\frac{2}{3}$ das tentativas.
d) Incorreta. Ele acertou mais que em 10\% das tentativas.
SAEB: Identificar frações equivalentes.
BNCC: EF04MA26 -- Reconhecer as frações unitárias mais usuais ($\frac{1}{2}$, $\frac{1}{3}$, $\frac{1}{4}$, $\frac{1}{5}$, $\frac{1}{10}$ e $\frac{1}{100}$) como
unidades de medida menores do que uma unidade, utilizando a reta numérica como recurso.

\item
a) Incorreta. O aluno, incorretamente, somou os dois quantitativos de posições diferentes.
b) Incorreta. O aluno não compreendeu o princípio.
c) Incorreta. Opções para o assento: 4; opções para o encosto: 6. Portanto: 4 x 6 = 24 maneiras diferentes de se posicionar essa cadeira.
d) Incorreta. O aluno errou nos cálculos.
SAEB: Resolver problemas simples de contagem (combinatória).
BNCC: EF04MA08 -- Resolver, com o suporte de imagem e/ou material manipulável, problemas simples
de contagem, como a determinação do número de agrupamentos possíveis ao se combinar cada
elemento de uma coleção com todos os elementos de outra, utilizando estratégias e formas de
registro pessoais.

\item
a) Incorreta. O aluno não fez a divisão corretamente.
b) Correta. Divide-se o fio de cobre em pedaços de tamanho desejado: 700 : 14 = 50
pedaços. Como ele já possui 24 pedaços, ele precisará de mais 50 -- 24 = 26 pedaços.
c) Incorreta. O aluno errou na divisão de um número pelo outro.
d) Incorreta. O aluno se esqueceu de subtrair o número de pedaços que o personagem já tem.
SAEB: Resolver problemas de multiplicação ou de divisão,
envolvendo números racionais apenas na representação decimal finita até
a ordem dos milésimos, com os significados de formação de grupos iguais
(incluindo repartição equitativa de medida), proporcionalidade ou
disposição retangular.
BNCC: EF04MA07 -- Resolver e elaborar problemas de divisão cujo divisor tenha no máximo dois algarismos,
envolvendo os significados de repartição equitativa e de medida, utilizando estratégias diversas,
como cálculo por estimativa, cálculo mental e algoritmos.

\item
a) Incorreta. Esse é o valor do número multiplicado por 10.
b) Incorreta. Esse é o valor do número multiplicado por 100.
c) Incorreta. Esse é o valor do número multiplicado por 1.000.
d) Correta. 32 x 10.000 = 320.000 (trezentos e vinte mil).
SAEB: Escrever números racionais (naturais de até 6 ordens, representação
fracionária ou decimal finita até a ordem dos milésimos) em sua
representação por algarismos ou em língua materna OU associar o registro
numérico ao registro em língua materna.
BNCC: EF04MA02 -- Mostrar, por decomposição e composição, que todo número natural pode ser escrito
por meio de adições e multiplicações por potências de dez, para compreender o sistema de
numeração decimal e desenvolver estratégias de cálculo.
\end{enumerate}

\section*{Simulado 4}

\begin{enumerate}
\item
a) Correta. As operações totalizam 1.249.
b) Incorreta. Faltaram 200 unidades.
c) Incorreta. Faltaram 99 unidades.
d) Incorreta. Faltaram 9 unidades.
SAEB: Compor ou decompor números naturais de até 6 ordens na
forma aditiva, ou em suas ordens, ou em adições e multiplicações.
BNCC: EF04MA02 -- Mostrar, por decomposição e composição, que todo número natural pode ser escrito
por meio de adições e multiplicações por potências de dez, para compreender o sistema de
numeração decimal e desenvolver estratégias de cálculo.

\item
a) Incorreta. O aluno errou nos cálculos.
b) Incorreta. O aluno errou nos cálculos.
c) Correta. (10 x 10) + (2 x 100) + (8 x 10) + 7 + 3 = 390.
d) Incorreta. O aluno errou nos cálculos.
SAEB: Resolver problemas de adição ou de subtração,
envolvendo números naturais de até 6 ordens, com os significados de
juntar, acrescentar, separar, retirar, comparar ou completar.
BNCC: EF04MA03 -- Resolver e elaborar problemas com números naturais envolvendo adição e subtração,
utilizando estratégias diversas, como cálculo, cálculo mental e algoritmos, além de fazer estimativas
do resultado.

\item
a) Incorreta. O aluno se perdeu nos termos.
b) Incorreta. O aluno se perdeu nos termos.
c) Correta. 4 + 2 x (16 -- 9) + 8 : 2 = 4 + 2 x 7 + 8 : 2 = 4 + 14 + 4 = 22.
d) Incorreta. O aluno se perdeu nos termos.
SAEB: Resolver problemas de multiplicação ou de divisão, envolvendo números
naturais de até 6 ordens, com os significados de formação de grupos
iguais (incluindo repartição equitativa e medida), proporcionalidade ou
disposição retangular.
BNCC: EF04MA03 -- Resolver e elaborar problemas com números naturais envolvendo adição e subtração,
utilizando estratégias diversas, como cálculo, cálculo mental e algoritmos, além de fazer estimativas
do resultado.

\item
a) Incorreta. Para o número, faltam duas unidades ao dobro de 150.
b) Correta. O número que está faltando na sequência é o 302 (antecessor de 303),
pois a lógica embutida é a soma de 100 unidades de um número para o
outro.
c) Incorreta. Para o número, faltam cinquenta unidades ao sucessor de 251.
d) Incorreta. Para o número, faltam cinquenta e duas unidades à metade de 500.
SAEB: Inferir os elementos ausentes em uma sequência de
números naturais ordenados, objetos ou figuras.
BNCC: EF04MA11 -- Identificar regularidades em sequências numéricas compostas por múltiplos de um
número natural.

\item
a) Incorreta. A caneca não tem capacidade para 40 litros.
b) Incorreta. A jarra não tem capacidade para 40 litros.
d) Incorreta. O garrafão não tem capacidade para 40 litros.
d) Correta. Pela análise das imagens, podemos estimar que o tambor será o recipiente que
pode ter pelo menos 40 litos de capacidade.
SAEB: Estimar/inferir medida de comprimento, capacidade ou
massa de objetos, utilizando unidades de medida convencionais ou não ou
medir comprimento, capacidade ou massa de objetos.
BNCC: EF04MA20 -- Medir e estimar comprimentos (incluindo perímetros), massas e capacidades, utilizando
unidades de medida padronizadas mais usuais, valorizando e respeitando a cultura local.

\item
a) Incorreta. O aluno não compreendeu o princípio.
b) Incorreta. O aluno errou nos cálculos.
c) Correta. 6 (número total de cores disponíveis) x 5 (cores sem a repetição de uma cor já usada) = 30 combinações diferentes de cores para a bandeira.
d) Incorreta. O aluno errou nos cálculos.
SAEB: Resolver problemas simples de contagem (combinatória).
BNCC: EF04MA08 -- Resolver, com o suporte de imagem e/ou material manipulável, problemas simples
de contagem, como a determinação do número de agrupamentos possíveis ao se combinar cada
elemento de uma coleção com todos os elementos de outra, utilizando estratégias e formas de
registro pessoais.

\item
a) Correta. Valor gasto por Rafael: 35 + 3 = R\$ 38,00. Valor das cédulas e das moedas descritas: 10 + 25 + 3 = R\$ 38,00.
b) Incorreta. Valor gasto por Rafael: 35 + 3 = R\$ 38,00. Valor das cédulas e das moedas descritas: 10 + 20 + 3 = R\$ 33,00.
c) Incorreta. Valor gasto por Rafael: 35 + 3 = R\$ 38,00. Valor das cédulas e das moedas descritas: 20 + 5 + 3 = R\$ 28,00.
d) Incorreta. Valor gasto por Rafael: 35 + 3 = R\$ 38,00. Valor das cédulas e das moedas descritas: 20 + 10 + 2 = R\$ 32,00.
SAEB: Resolver problemas que envolvam moedas e/ou cédulas do
sistema monetário brasileiro.
BNCC: EF04MA25 -- Resolver e elaborar problemas que envolvam situações de compra e venda e formas
de pagamento, utilizando termos como troco e desconto, enfatizando o consumo ético, consciente e
responsável.

\item
a) Incorreta. O aluno se confundiu com os valores.
b) Incorreta. O aluno se confundiu com os valores.
c) Incorreta. O aluno se confundiu com os valores.
d) Correta. (10 x 0,05) + (5 x 0,50) + (70 x 0,10) + (10 x 1,00) = 0,50 + 2,50 + 7,00 + 10,00 = R\$ 20,00.
SAEB: Resolver problemas que envolvam moedas e/ou cédulas do
sistema monetário brasileiro.
BNCC: EF04MA25 -- Resolver e elaborar problemas que envolvam situações de compra e venda e formas
de pagamento, utilizando termos como troco e desconto, enfatizando o consumo ético, consciente e
responsável.

\item
a) Incorreta. As chances são todas iguais.
b) Incorreta. As chances são todas iguais.
c) Incorreta. As chances são todas iguais.
d) Correta. Como há mudas de cada flor nas mesmas quantidades, as chances são sempre iguais na primeira vez.
SAEB: Determinar a probabilidade de ocorrência de um
resultado em eventos aleatórios, quando todos os resultados possíveis
têm a mesma chance de ocorrer (equiprováveis).
BNCC: EF04MA26 -- Identificar, entre eventos aleatórios cotidianos, aqueles que têm maior chance de
ocorrência, reconhecendo características de resultados mais prováveis, sem utilizar frações.

\item
a) Incorreta. O E não empatou.
b) Correta. A: 18 pontos; B: 18 pontos; C: 18 pontos; D: 19 pontos; E: 20 pontos.
c) Incorreta. O D e o E não empataram.
d) Incorreta. O E não empatou.
Habilidade SAEB: Resolver problemas que envolvam dados apresentados tabelas (simples ou
de dupla entrada) ou gráficos estatísticos (barras simples ou agrupadas,
colunas simples ou agrupadas, pictóricos ou de linhas).
BNCC: EF04MA27 -- Analisar dados apresentados em tabelas simples ou de dupla entrada e em gráficos de
colunas ou pictóricos, com base em informações das diferentes áreas do conhecimento, e produzir
texto com a síntese de sua análise.

\item
a) Incorreta. O aluno se confundiu com o valor da fração.
b) Incorreta. O aluno se confundiu com o valor da fração.
c) Correta. $\frac{4}{7}$ x 77 = 44 voltas. Portanto ainda faltavam 77 -- 44 = 33 voltas.
d) Incorreta. O número 44 deveria ter sido subtraído do total de voltas para se chegar à resposta.
SAEB: Resolver problemas que envolvam fração como resultado
de uma divisão (quociente).
BNCC: EF04MA09 -- Reconhecer as frações unitárias mais usuais ($\frac{1}{2}$, $\frac{1}{3}$, $\frac{1}{4}$, $\frac{1}{5}$, $\frac{1}{10}$ e $\frac{1}{100}$) como
unidades de medida menores do que uma unidade, utilizando a reta numérica como recurso.

\item
a) Correta. $\frac{1}{250}$ x 2.000 = 8.
b) Incorreta. O aluno errou no cálculo da divisão.
c) Incorreta. O aluno errou no cálculo da divisão.
d) Incorreta. O aluno se confundiu com os números.
SAEB: Resolver problemas que envolvam variação de proporcionalidade direta entre duas grandezas.

\item
a) Incorreta. O aluno se confundiu com o princípio da contagem.
b) Incorreta O aluno não compreendeu o princípio.
c) Correta. Para a escolha do primeiro algarismo, temos 9 opções de algarismos. Para
a escolha do segundo algarismo, teremos 8, já que não podemos repetir
algarismos. Para a escolha do terceiro algarismo, teremos 7. Sendo assim: 9 x 8 x 7 = 504 combinações diferentes.
d) Incorreta. O aluno não formou número com apenas três algarismos.
SAEB: Resolver problemas simples de contagem (combinatória).
BNCC: Resolver, com o suporte de imagem e/ou material manipulável, problemas simples
de contagem, como a determinação do número de agrupamentos possíveis ao se combinar cada
elemento de uma coleção com todos os elementos de outra, utilizando estratégias e formas de
registro pessoais.

\item
a) Incorreta. Esse é o valor sem parcelar.
b) Incorreta. O aluno se confundiu no momento de calcular o preço com desconto.
c) Correta. Valor que ele pagará pelo tênis: 322,00 -- 42,00 = R\$ 280,00.
Dividindo o valor em 4 parcelas: 280 : 4 = R\$ 70,00.
d) Incorreta. O aluno se confundiu no momento de fazer a divisão entre as quatro parcelas iguais.
SAEB: Resolver problemas de multiplicação ou de divisão,
envolvendo números racionais apenas na representação decimal finita até
a ordem dos milésimos, com os significados de formação de grupos iguais
(incluindo repartição equitativa de medida), proporcionalidade ou
disposição retangular.
BNCC: EF04MA07 -- Resolver e elaborar problemas de divisão cujo divisor tenha no máximo dois algarismos,
envolvendo os significados de repartição equitativa e de medida, utilizando estratégias diversas,
como cálculo por estimativa, cálculo mental e algoritmos.

\item
a) Incorreta. O aluno errou a ordem das letras.
b) Incorreta. O aluno duplicou a letra errada.
c) Correta. O número 12 em algarismos romanos é representado por XII.
d) Incorreta. O aluno duplicou a letra errada.
SAEB: Escrever números racionais (naturais de até 6 ordens, representação
fracionária ou decimal finita até a ordem dos milésimos) em sua
representação por algarismos ou em língua materna OU associar o registro
numérico ao registro em língua materna.
BNCC: EF04MA01 -- Ler, escrever e ordenar números naturais até a ordem de dezenas de milhar.
\end{enumerate}