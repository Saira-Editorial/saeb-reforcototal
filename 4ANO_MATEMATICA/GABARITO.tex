\chapter{Respostas}
\pagestyle{plain}
\footnotesize

\pagecolor{gray!40}

\colorsec{Matemática – Módulo 1 – Treino}

\begin{enumerate}
\item
a) Correta. Todas as ordens estão corretamente representadas e somadas entre si.
b) Incorreta. As quantidades indicadas pelos algarismos 5, 7 e 3 estão erradas.
c) Incorreta. As quantidades indicadas pelos algarismos 5 e 7 estão erradas.
d) Incorreta. As quantidades indicadas pelos algarismos 7 e 3 estão erradas.
SAEB: Compor ou decompor números naturais de até 6 ordens na forma aditiva, ou em suas ordens, ou em adições e multiplicações.
BNCC: EF04MA02 -- Mostrar, por decomposição e composição, que todo número natural pode ser escrito
por meio de adições e multiplicações por potências de dez, para compreender o sistema de
numeração decimal e desenvolver estratégias de cálculo.

\item
a)  Incorreta. Faltou considerar o valor do primeiro X.
b)  Correta. O número formado pelas duas últimas letras (IX), 9, deve ser somado ao número representado pela primeira letra (X), 10: 10 + 9 = 19.
c)  Incorreta. Dessa forma, não se leva em consideração a ordem da escrita.
d)  Incorreta. Faltou considerar o valor do segundo X.
SAEB: Escrever números racionais (naturais de até 6 ordens, representação fracionária ou decimal finita até a ordem dos milésimos) em sua representação por algarismos ou em língua materna ou associar o registro numérico ao registro em língua materna.
BNCC: EF04MA01 -- Ler, escrever e ordenar números naturais até a ordem de dezenas de milhar.

\item
a)  Incorreta. A identificação do valor na placa está correta, mas ele deveria estar como 500.
b)  Incorreta. O valor na placa deveria ser como 500, mas ele está como 5.
c)  Correta. Na placa, está representado o número 125 -- em que o algarismo 5 ocupa a ordem das unidades. O número representado deveria ser o 521 -- em que o algarismo 5 ocuparia a ordem das centenas.
d)  Incorreta. As duas identificações estão incorretas.
SAEB: Identificar a ordem ocupada por um algarismo ou seu valor posicional (ou valor relativo) em um número natural de até 6 ordens.
BNCC: EF04MA01 -- Ler, escrever e ordenar números naturais até a ordem de dezenas de milhar.
\end{enumerate}

\colorsec{Matemática – Módulo 2 – Treino}

\begin{enumerate}
\item
a) Incorreta. Cometeu-se um erro na segunda etapa da conta.
b) Correta. 417 – 105 = 312.
c) Incorreta. Cometeu-se um erro na segunda etapa da conta.
d) Incorreta. Todo o procedimento foi errado.
SAEB: Calcular o resultado de adições ou subtrações envolvendo números naturais de até 6 ordens.
BNCC: EF04MA07 -- Resolver e elaborar problemas de divisão cujo divisor tenha no máximo dois algarismos,
envolvendo os significados de repartição equitativa e de medida, utilizando estratégias diversas,
como cálculo por estimativa, cálculo mental e algoritmos.

\item
a) Incorreta. Não foram operacionalizadas primeiro as multiplicações.
b) Incorreta. Cometeu-se um erro na ordem das operações.
c) Incorreta. O subtraendo da conta final foi calculado errado.
d) Correta. 200 -- 2 x (1 x 5 + 2 x 7) = 200 -- 2 x (5 + 14) = 200 -- 2 x 19 = 200 -- 38 = 162 peças.
SAEB: Calcular o resultado de multiplicações ou divisões envolvendo números naturais de até 6 ordens.
BNCC: EF04MA07 -- Resolver e elaborar problemas de divisão cujo divisor tenha no máximo dois algarismos,
envolvendo os significados de repartição equitativa e de medida, utilizando estratégias diversas,
como cálculo por estimativa, cálculo mental e algoritmos.

\item
a) Incorreta. A subtração foi realizada incorretamente.
b) Correta. 1.200 –- 540 = 660 vagas. 932 –- 660 = 272 pessoas não conseguirão assistir à sessão.
c) Incorreta. Só foi calculado o número de lugares vazios.
d) Incorreta. Errou-se na primeira subtração.
SAEB: Resolver problemas de adição ou de subtração, envolvendo números naturais de até 6 ordens, com os significados de juntar, acrescentar, separar, retirar, comparar ou completar.
BNCC: EF04MA07 -- Resolver e elaborar problemas de divisão cujo divisor tenha no máximo dois algarismos,
envolvendo os significados de repartição equitativa e de medida, utilizando estratégias diversas,
como cálculo por estimativa, cálculo mental e algoritmos.
\end{enumerate}

\colorsec{Matemática – Módulo 3 – Treino}

\begin{enumerate}
\item
a) Incorreta. O aluno não entendeu a lógica da sequência.
b) Incorreta. O aluno não identificou o padrão.
c) Incorreta. O aluno chegou a um número menor que o da resposta correta.
d) Correta. A sequência é esta: (2; 6; 12; 20; 30; 42).
SAEB: Inferir o padrão ou a regularidade de uma sequência de números naturais ordenados, objetos ou figuras.
BNCC: EF04MA11 -- Identificar regularidades em sequências numéricas compostas por múltiplos de um
número natural.

\item
a) Incorreta. O aluno se confundiu com o antecessor do número que falta.
b) Correta. A sequência aumenta de 100 em 100 unidades.
c) Incorreta. O aluno se confundiu com o sucessor do número que falta.
d) Incorreta. O aluno selecionou, incorretamente, o primeiro número da sequência.
SAEB: Inferir os elementos ausentes em uma sequência de números naturais ordenados, objetos ou figuras.
BNCC: EF04MA11 -- Identificar regularidades em sequências numéricas compostas por múltiplos de um
número natural.

\item
a) Incorreta. Esse é o antecessor do antecessor.
b) Incorreta. Esse é o antecessor.
c) Incorreta. Esse é o sucessor.
d) Correta. O sucessor do sucessor de 7081 é 7081 + 1 + 1 = 7083.
SAEB: Inferir ou descrever atributos ou propriedades comuns que os elementos que constituem uma sequência recursiva de números naturais apresentam.
BNCC: EF04MA11 -- Identificar regularidades em sequências numéricas compostas por múltiplos de um
número natural.
\end{enumerate}

\colorsec{Matemática – Módulo 4 – Treino}

\begin{enumerate}
\item
a) Incorreta. O aluno somou uma hora e meia a menos.
b) Incorreta. O aluno somou uma hora a menos.
c) Incorreta. O aluno somou meia hora a menos.
d) Correta. 8 + 4,5 = 12,5 (12:30).
SAEB: Determinar o horário de início, o horário de término ou a duração de um acontecimento.
BNCC: EF04MA22 -- Ler e registrar medidas e intervalos de tempo em horas, minutos e segundos em
situações relacionadas ao seu cotidiano, como informar os horários de início e término de realização
de uma tarefa e sua duração.

\item
a) Incorreta. O aluno chegou a um número menor que a metade do valor da resposta.
b) Incorreta. O aluno chegou a um valor que é maior que a metade do valor da resposta, mas ainda menor que a resposta.
c) Incorreta. O aluno se esqueceu de contar os 80 mL que ainda ficariam de fora da conta.
d) Correta. 4 x 8 x 15 = 480 mL. Como cada frasco possui 100 mL, ela deverá
comprar 5 frascos, e haverá uma sobra de xarope.
SAEB: Estimar/inferir medida de comprimento, capacidade ou massa de objetos, utilizando unidades de medida convencionais ou não ou medir comprimento, capacidade ou massa de objetos.
BNCC: EF04MA20 -- Medir e estimar comprimentos (incluindo perímetros), massas e capacidades, utilizando
unidades de medida padronizadas mais usuais, valorizando e respeitando a cultura local.

\item
a) Correta. Se a partida do voo foi às 10 horas e 42 minutos e a chegada ao destino foi às 14 horas e 8 minutos, o tempo de voo foi de 3 horas e 16 minutos, ou seja, 196 minutos, o que equivale a 11.760 segundos.
b)Incorreta. O aluno errou na subtração entre os dois horários.
c)Incorreta. O aluno errou na conversão para segundos.
d) Incorreta. O aluno errou na subtração entre os dois horários e na conversão para segundos.
SAEB: Resolver problemas que envolvam medidas de grandezas (comprimento, massa, tempo e capacidade) em que haja conversões entre as unidades mais usuais.
BNCC: EF04MA22 -- Ler e registrar medidas e intervalos de tempo em horas, minutos e segundos em
situações relacionadas ao seu cotidiano, como informar os horários de início e término de realização
de uma tarefa e sua duração.
\end{enumerate}

\colorsec{Matemática – Módulo 5 – Treino}

\begin{enumerate}
\item
a) Incorreta. O aluno contou errado a quantidade de quadradinhos.
b) Correta. Paulo deverá andar 5 lados de quadrado. Como cada lado de quadrado possui medida igual a 2 m, ele deverá andar 10 metros.
c) Incorreta. O aluno contou errado a quantidade de quadradinhos.
d) Incorreta. O aluno contou errado a quantidade de quadradinhos.
SAEB: Medir ou comparar perímetro ou área de figuras planas desenhadas em malha quadriculada.
BNCC: EF04MA20 -- Medir e estimar comprimentos (incluindo perímetros), massas e capacidades, utilizando
unidades de medida padronizadas mais usuais, valorizando e respeitando a cultura local.

\item
a) Incorreta. O aluno contou errado o número de quadradinhos de cada figura.
b) Incorreta. O aluno contou errado o número de quadradinhos de cada figura.
c) Correta. Todas as figuras possuem 6 quadradinhos de área.
d) Incorreta. O aluno contou errado o número de quadradinhos de cada figura.
SAEB: Resolver problemas que envolvam área de figuras planas
BNCC: EF04MA21 -- Medir, comparar e estimar área de figuras planas desenhadas em malha quadriculada,
pela contagem dos quadradinhos ou de metades de quadradinho, reconhecendo que duas figuras
com formatos diferentes podem ter a mesma medida de área.

\item
a) Incorreta. O aluno não soube interpretar o horário no relógio analógico.
b) Incorreta. O aluno trocou o ponteiro das horas com o dos minutos.
c) Incorreta. O aluno trocou o ponteiro das horas com o dos minutos.
d) Correta. O relógio estava marcando 11 horas e 35 minutos. Acrescentando a esse horário 55 minutos, tem-se no relógio 12 horas e 30 minutos.
SAEB: Identificar horas em relógios analógicos ou associar horas em relógios analógicos e digitais.
BNCC: EF04MA22 -- Ler e registrar medidas e intervalos de tempo em horas, minutos e segundos em
situações relacionadas ao seu cotidiano, como informar os horários de início e término de realização
de uma tarefa e sua duração.
\end{enumerate}

\colorsec{Matemática – Módulo 6 – Treino}

\begin{enumerate}
\item

\item

\item
\end{enumerate}

\colorsec{Matemática – Módulo 7 – Treino}

\begin{enumerate}
\item

\item

\item
\end{enumerate}

\colorsec{Matemática – Módulo 8 – Treino}

\begin{enumerate}
\item

\item

\item
\end{enumerate}

\colorsec{Simulado 1}

\begin{enumerate}
\item

\item

\item
\end{enumerate}


\colorsec{Simulado 2}

\begin{enumerate}
\item

\item

\item
\end{enumerate}

\colorsec{Simulado 3}

\begin{enumerate}
\item

\item

\item
\end{enumerate}

\colorsec{Simulado 4}

\begin{enumerate}
\item

\item

\item
\end{enumerate}