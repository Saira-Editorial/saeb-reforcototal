\chapter{Respostas}
\pagestyle{plain}
\footnotesize

\pagecolor{gray!40}

\colorsec{Matemática – Módulo 1 – Treino}

\begin{enumerate}
\item
a) Correta. Todas as ordens estão corretamente representadas e somadas entre si.
b) Incorreta. As quantidades indicadas pelos algarismos 5, 7 e 3 estão erradas.
c) Incorreta. As quantidades indicadas pelos algarismos 5 e 7 estão erradas.
d) Incorreta. As quantidades indicadas pelos algarismos 7 e 3 estão erradas.
SAEB: Compor ou decompor números naturais de até 6 ordens na forma aditiva, ou em suas ordens, ou em adições e multiplicações.
BNCC: EF04MA02 -- Mostrar, por decomposição e composição, que todo número natural pode ser escrito
por meio de adições e multiplicações por potências de dez, para compreender o sistema de
numeração decimal e desenvolver estratégias de cálculo.

\item
a)  Incorreta. Faltou considerar o valor do primeiro X.
b)  Correta. O número formado pelas duas últimas letras (IX), 9, deve ser somado ao número representado pela primeira letra (X), 10: 10 + 9 = 19.
c)  Incorreta. Dessa forma, não se leva em consideração a ordem da escrita.
d)  Incorreta. Faltou considerar o valor do segundo X.
SAEB: Escrever números racionais (naturais de até 6 ordens, representação fracionária ou decimal finita até a ordem dos milésimos) em sua representação por algarismos ou em língua materna ou associar o registro numérico ao registro em língua materna.
BNCC: EF04MA01 -- Ler, escrever e ordenar números naturais até a ordem de dezenas de milhar.

\item
a)  Incorreta. A identificação do valor na placa está correta, mas ele deveria estar como 500.
b)  Incorreta. O valor na placa deveria ser como 500, mas ele está como 5.
c)  Correta. Na placa, está representado o número 125 -- em que o algarismo 5 ocupa a ordem das unidades. O número representado deveria ser o 521 -- em que o algarismo 5 ocuparia a ordem das centenas.
d)  Incorreta. As duas identificações estão incorretas.
SAEB: Identificar a ordem ocupada por um algarismo ou seu valor posicional (ou valor relativo) em um número natural de até 6 ordens.
BNCC: EF04MA01 -- Ler, escrever e ordenar números naturais até a ordem de dezenas de milhar.
\end{enumerate}

\colorsec{Matemática – Módulo 2 – Treino}

\begin{enumerate}
\item

\item

\item
\end{enumerate}

\colorsec{Matemática – Módulo 3 – Treino}

\begin{enumerate}
\item

\item

\item
\end{enumerate}

\colorsec{Matemática – Módulo 4 – Treino}

\begin{enumerate}
\item

\item

\item
\end{enumerate}

\colorsec{Matemática – Módulo 5 – Treino}

\begin{enumerate}
\item

\item

\item
\end{enumerate}

\colorsec{Matemática – Módulo 6 – Treino}

\begin{enumerate}
\item

\item

\item
\end{enumerate}

\colorsec{Matemática – Módulo 7 – Treino}

\begin{enumerate}
\item

\item

\item
\end{enumerate}

\colorsec{Matemática – Módulo 8 – Treino}

\begin{enumerate}
\item

\item

\item
\end{enumerate}

\colorsec{Simulado 1}

\begin{enumerate}
\item

\item

\item
\end{enumerate}


\colorsec{Simulado 2}

\begin{enumerate}
\item

\item

\item
\end{enumerate}

\colorsec{Simulado 3}

\begin{enumerate}
\item

\item

\item
\end{enumerate}

\colorsec{Simulado 4}

\begin{enumerate}
\item

\item

\item
\end{enumerate}