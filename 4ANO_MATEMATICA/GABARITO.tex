\chapter{Respostas}
\pagestyle{plain}
\footnotesize

\pagecolor{gray!40}

\colorsec{Matemática – Módulo 1 – Treino}

\begin{enumerate}
\item
a) Correta. Todas as ordens estão corretamente representadas e somadas entre si.
b) Incorreta. As quantidades indicadas pelos algarismos 5, 7 e 3 estão erradas.
c) Incorreta. As quantidades indicadas pelos algarismos 5 e 7 estão erradas.
d) Incorreta. As quantidades indicadas pelos algarismos 7 e 3 estão erradas.
SAEB: Compor ou decompor números naturais de até 6 ordens na forma aditiva, ou em suas ordens, ou em adições e multiplicações.
BNCC: EF04MA02 -- Mostrar, por decomposição e composição, que todo número natural pode ser escrito
por meio de adições e multiplicações por potências de dez, para compreender o sistema de
numeração decimal e desenvolver estratégias de cálculo.

\item
a)  Incorreta. Faltou considerar o valor do primeiro X.
b)  Correta. O número formado pelas duas últimas letras (IX), 9, deve ser somado ao número representado pela primeira letra (X), 10: 10 + 9 = 19.
c)  Incorreta. Dessa forma, não se leva em consideração a ordem da escrita.
d)  Incorreta. Faltou considerar o valor do segundo X.
SAEB: Escrever números racionais (naturais de até 6 ordens, representação fracionária ou decimal finita até a ordem dos milésimos) em sua representação por algarismos ou em língua materna ou associar o registro numérico ao registro em língua materna.
BNCC: EF04MA01 -- Ler, escrever e ordenar números naturais até a ordem de dezenas de milhar.

\item
a)  Incorreta. A identificação do valor na placa está correta, mas ele deveria estar como 500.
b)  Incorreta. O valor na placa deveria ser como 500, mas ele está como 5.
c)  Correta. Na placa, está representado o número 125 -- em que o algarismo 5 ocupa a ordem das unidades. O número representado deveria ser o 521 -- em que o algarismo 5 ocuparia a ordem das centenas.
d)  Incorreta. As duas identificações estão incorretas.
SAEB: Identificar a ordem ocupada por um algarismo ou seu valor posicional (ou valor relativo) em um número natural de até 6 ordens.
BNCC: EF04MA01 -- Ler, escrever e ordenar números naturais até a ordem de dezenas de milhar.
\end{enumerate}

\colorsec{Matemática – Módulo 2 – Treino}

\begin{enumerate}
\item
a) Incorreta. Cometeu-se um erro na segunda etapa da conta.
b) Correta. 417 – 105 = 312.
c) Incorreta. Cometeu-se um erro na segunda etapa da conta.
d) Incorreta. Todo o procedimento foi errado.
SAEB: Calcular o resultado de adições ou subtrações envolvendo números naturais de até 6 ordens.
BNCC: EF04MA07 -- Resolver e elaborar problemas de divisão cujo divisor tenha no máximo dois algarismos,
envolvendo os significados de repartição equitativa e de medida, utilizando estratégias diversas,
como cálculo por estimativa, cálculo mental e algoritmos.

\item
a) Incorreta. Não foram operacionalizadas primeiro as multiplicações.
b) Incorreta. Cometeu-se um erro na ordem das operações.
c) Incorreta. O subtraendo da conta final foi calculado errado.
d) Correta. 200 -- 2 x (1 x 5 + 2 x 7) = 200 -- 2 x (5 + 14) = 200 -- 2 x 19 = 200 -- 38 = 162 peças.
SAEB: Calcular o resultado de multiplicações ou divisões envolvendo números naturais de até 6 ordens.
BNCC: EF04MA07 -- Resolver e elaborar problemas de divisão cujo divisor tenha no máximo dois algarismos,
envolvendo os significados de repartição equitativa e de medida, utilizando estratégias diversas,
como cálculo por estimativa, cálculo mental e algoritmos.

\item
a) Incorreta. A subtração foi realizada incorretamente.
b) Correta. 1.200 –- 540 = 660 vagas. 932 –- 660 = 272 pessoas não conseguirão assistir à sessão.
c) Incorreta. Só foi calculado o número de lugares vazios.
d) Incorreta. Errou-se na primeira subtração.
SAEB: Resolver problemas de adição ou de subtração, envolvendo números naturais de até 6 ordens, com os significados de juntar, acrescentar, separar, retirar, comparar ou completar.
BNCC: EF04MA07 -- Resolver e elaborar problemas de divisão cujo divisor tenha no máximo dois algarismos,
envolvendo os significados de repartição equitativa e de medida, utilizando estratégias diversas,
como cálculo por estimativa, cálculo mental e algoritmos.
\end{enumerate}

\colorsec{Matemática – Módulo 3 – Treino}

\begin{enumerate}
\item
a) Incorreta. O aluno não entendeu a lógica da sequência.
b) Incorreta. O aluno não identificou o padrão.
c) Incorreta. O aluno chegou a um número menor que o da resposta correta.
d) Correta. A sequência é esta: (2; 6; 12; 20; 30; 42).
SAEB: Inferir o padrão ou a regularidade de uma sequência de números naturais ordenados, objetos ou figuras.
BNCC: EF04MA11 -- Identificar regularidades em sequências numéricas compostas por múltiplos de um
número natural.

\item
a) Incorreta. O aluno se confundiu com o antecessor do número que falta.
b) Correta. A sequência aumenta de 100 em 100 unidades.
c) Incorreta. O aluno se confundiu com o sucessor do número que falta.
d) Incorreta. O aluno selecionou, incorretamente, o primeiro número da sequência.
SAEB: Inferir os elementos ausentes em uma sequência de números naturais ordenados, objetos ou figuras.
BNCC: EF04MA11 -- Identificar regularidades em sequências numéricas compostas por múltiplos de um
número natural.

\item
a) Incorreta. Esse é o antecessor do antecessor.
b) Incorreta. Esse é o antecessor.
c) Incorreta. Esse é o sucessor.
d) Correta. O sucessor do sucessor de 7081 é 7081 + 1 + 1 = 7083.
SAEB: Inferir ou descrever atributos ou propriedades comuns que os elementos que constituem uma sequência recursiva de números naturais apresentam.
BNCC: EF04MA11 -- Identificar regularidades em sequências numéricas compostas por múltiplos de um
número natural.
\end{enumerate}

\colorsec{Matemática – Módulo 4 – Treino}

\begin{enumerate}
\item
a) Incorreta. O aluno somou uma hora e meia a menos.
b) Incorreta. O aluno somou uma hora a menos.
c) Incorreta. O aluno somou meia hora a menos.
d) Correta. 8 + 4,5 = 12,5 (12:30).
SAEB: Determinar o horário de início, o horário de término ou a duração de um acontecimento.
BNCC: EF04MA22 -- Ler e registrar medidas e intervalos de tempo em horas, minutos e segundos em
situações relacionadas ao seu cotidiano, como informar os horários de início e término de realização
de uma tarefa e sua duração.

\item
a) Incorreta. O aluno chegou a um número menor que a metade do valor da resposta.
b) Incorreta. O aluno chegou a um valor que é maior que a metade do valor da resposta, mas ainda menor que a resposta.
c) Incorreta. O aluno se esqueceu de contar os 80 mL que ainda ficariam de fora da conta.
d) Correta. 4 x 8 x 15 = 480 mL. Como cada frasco possui 100 mL, ela deverá
comprar 5 frascos, e haverá uma sobra de xarope.
SAEB: Estimar/inferir medida de comprimento, capacidade ou massa de objetos, utilizando unidades de medida convencionais ou não ou medir comprimento, capacidade ou massa de objetos.
BNCC: EF04MA20 -- Medir e estimar comprimentos (incluindo perímetros), massas e capacidades, utilizando
unidades de medida padronizadas mais usuais, valorizando e respeitando a cultura local.

\item
a) Correta. Se a partida do voo foi às 10 horas e 42 minutos e a chegada ao destino foi às 14 horas e 8 minutos, o tempo de voo foi de 3 horas e 16 minutos, ou seja, 196 minutos, o que equivale a 11.760 segundos.
b)Incorreta. O aluno errou na subtração entre os dois horários.
c)Incorreta. O aluno errou na conversão para segundos.
d) Incorreta. O aluno errou na subtração entre os dois horários e na conversão para segundos.
SAEB: Resolver problemas que envolvam medidas de grandezas (comprimento, massa, tempo e capacidade) em que haja conversões entre as unidades mais usuais.
BNCC: EF04MA22 -- Ler e registrar medidas e intervalos de tempo em horas, minutos e segundos em
situações relacionadas ao seu cotidiano, como informar os horários de início e término de realização
de uma tarefa e sua duração.
\end{enumerate}

\colorsec{Matemática – Módulo 5 – Treino}

\begin{enumerate}
\item
a) Incorreta. O aluno contou errado a quantidade de quadradinhos.
b) Correta. Paulo deverá andar 5 lados de quadrado. Como cada lado de quadrado possui medida igual a 2 m, ele deverá andar 10 metros.
c) Incorreta. O aluno contou errado a quantidade de quadradinhos.
d) Incorreta. O aluno contou errado a quantidade de quadradinhos.
SAEB: Medir ou comparar perímetro ou área de figuras planas desenhadas em malha quadriculada.
BNCC: EF04MA20 -- Medir e estimar comprimentos (incluindo perímetros), massas e capacidades, utilizando
unidades de medida padronizadas mais usuais, valorizando e respeitando a cultura local.

\item
a) Incorreta. O aluno contou errado o número de quadradinhos de cada figura.
b) Incorreta. O aluno contou errado o número de quadradinhos de cada figura.
c) Correta. Todas as figuras possuem 6 quadradinhos de área.
d) Incorreta. O aluno contou errado o número de quadradinhos de cada figura.
SAEB: Resolver problemas que envolvam área de figuras planas
BNCC: EF04MA21 -- Medir, comparar e estimar área de figuras planas desenhadas em malha quadriculada,
pela contagem dos quadradinhos ou de metades de quadradinho, reconhecendo que duas figuras
com formatos diferentes podem ter a mesma medida de área.

\item
a) Incorreta. O aluno não soube interpretar o horário no relógio analógico.
b) Incorreta. O aluno trocou o ponteiro das horas com o dos minutos.
c) Incorreta. O aluno trocou o ponteiro das horas com o dos minutos.
d) Correta. O relógio estava marcando 11 horas e 35 minutos. Acrescentando a esse horário 55 minutos, tem-se no relógio 12 horas e 30 minutos.
SAEB: Identificar horas em relógios analógicos ou associar horas em relógios analógicos e digitais.
BNCC: EF04MA22 -- Ler e registrar medidas e intervalos de tempo em horas, minutos e segundos em
situações relacionadas ao seu cotidiano, como informar os horários de início e término de realização
de uma tarefa e sua duração.
\end{enumerate}

\colorsec{Matemática – Módulo 6 – Treino}

\begin{enumerate}
\item
a) Correta. (12 x R\$ 0,50) + (8 x R\$ 0,25) = R\$ 6,00 + R\$ 2,00 = R\$ 8,00 = 4 cédulas de R\$ 2,00.
b) Incorreta. O aluno fez a associação errada dos valores com as quantidades.
c) Incorreta. O aluno só calculou o valor que seria trocado.
d) Incorreta. O aluno só calculou a quantidade de moedas sem considerar o valor.
SAEB: Resolver problemas que envolvam moedas e/ou cédulas do sistema monetário brasileiro.
BNCC: EF04MA25 -- Resolver e elaborar problemas que envolvam situações de compra e venda e formas
de pagamento, utilizando termos como troco e desconto, enfatizando o consumo ético, consciente e
responsável.

\item
a) Incorreta. O aluno se esqueceu de contar as moedas.
b) Incorreta. O aluno não contou todas as moedas.
c) Incorreta. O aluno deixou de contar uma moeda de 5 centavos.
d) Correta. Letícia tem R\$ 9,00 em cédulas e R\$ 1,15 em moedas; portanto, no total, ela
encontrou em sua bolsa R\$ 10,15.
SAEB: Resolver problemas que envolvam moedas e/ou cédulas do sistema monetário brasileiro.
BNCC: EF04MA25 -- Resolver e elaborar problemas que envolvam situações de compra e venda e formas
de pagamento, utilizando termos como troco e desconto, enfatizando o consumo ético, consciente e
responsável.

\item
a) Incorreta. O aluno contou apenas os bombons.
b) Incorreta. O aluno errou nos cálculos.
c) Correta. (4 x 5,00) + (2 x 6,00) + (3 x 12,00) = 20,00 + 12,00 + 36,00 = R\$ 68,00.
d) Incorreta. O aluno contou um bombom a mais.
SAEB: Resolver problemas que envolvam moedas e/ou cédulas do sistema monetário brasileiro.
BNCC: EF04MA25 -- Resolver e elaborar problemas que envolvam situações de compra e venda e formas
de pagamento, utilizando termos como troco e desconto, enfatizando o consumo ético, consciente e
responsável.
\end{enumerate}

\colorsec{Matemática – Módulo 7 – Treino}

\begin{enumerate}
\item
a) Incorreta. Não podem ser contados o sábado e domingo.
b) Incorreta. A chance é sempre apenas uma, porque se trata somente da quinta-feira.
c) Correta. Entre os cinco dias possíveis, só a quinta-feira seria escolhida.
d) Incorreta. A chance é sempre apenas uma, porque se trata somente da quinta-feira.
SAEB: Identificar, entre eventos aleatórios, aqueles que têm menos, maiores ou iguais chances de ocorrência, sem utilizar frações.
BNCC: EF04MA26 -- Identificar, entre eventos aleatórios cotidianos, aqueles que têm maior chance de
ocorrência, reconhecendo características de resultados mais prováveis, sem utilizar frações.

\item
a) Incorreta. Há 28 clientes, e apenas 7 garçons.
b) Incorreta. É mais provável escolher um cliente, já que os clientes estão em maior número.
c) Correta. Como há mais clientes que garçons no restaurante, é mais provável que seja escolhido um cliente
d) Incorreta. As chances não são iguais, porque há números diferentes de clientes ou garçons.
SAEB: Identificar, entre eventos aleatórios, aqueles que têm menos, maiores ou iguais chances de ocorrência, sem utilizar frações.
BNCC: EF04MA26 -- Identificar, entre eventos aleatórios cotidianos, aqueles que têm maior chance de
ocorrência, reconhecendo características de resultados mais prováveis, sem utilizar frações.

\item
a) Incorreta. Carlos é terceiro com mais chances.
b) Correta. Márcia tem mais cupons que os outros; portanto é a que tem mais chance de ser sorteada.
c) Incorreta. Paulo é o segundo com mais chances.
d) Incorreta. Rogério é a quarta com mais chances.
SAEB: Identificar, entre eventos aleatórios, aqueles que têm menos, maiores ou iguais chances de ocorrência, sem utilizar frações.
BNCC: EF04MA26 -- Identificar, entre eventos aleatórios cotidianos, aqueles que têm maior chance de
ocorrência, reconhecendo características de resultados mais prováveis, sem utilizar frações.
\end{enumerate}

\colorsec{Matemática – Módulo 8 – Treino}

\begin{enumerate}
\item
a) Correta. O candidato A teve todas as notas acima de 30 e é o que teve mais notas iguais.
b) Incorreta. O candidato não teve o maior número de notas iguais, apesar de todas serem acima de 30.
c) Incorreta. O candidato tem uma nota abaixo de 30.
d) Incorreta. O candidato tem duas notas abaixo de 30.
SAEB: Ler/identificar ou comparar dados estatísticos expressos em tabelas (simples ou de dupla entrada).
BNCC: EF04MA27 -- Analisar dados apresentados em tabelas simples ou de dupla entrada e em gráficos de
colunas ou pictóricos, com base em informações das diferentes áreas do conhecimento, e produzir
texto com a síntese de sua análise.

\item
a)  Correta. O número total de alunos do 4º ano é dado por esta conta: 32 + 29 + 25 = 86.
b)  Incorreta. O aluno somou 4º e 5º anos da turma A.
c)  Incorreta. O aluno somou os alunos de 5º ano.
d)  Incorreta. O aluno somou a quantidade total de alunos da escola.
SAEB: Ler/identificar ou comparar dados estatísticos expressos em tabelas (simples ou de dupla entrada).
BNCC: EF04MA27 -- Analisar dados apresentados em tabelas simples ou de dupla entrada e em gráficos de
colunas ou pictóricos, com base em informações das diferentes áreas do conhecimento, e produzir
texto com a síntese de sua análise.

\item
a)  Incorreta. O aluno considerou as crianças de 4 a 6 anos.
b)  Incorreta. O aluno considerou as crianças de 7 a 9 anos.
c)  Incorreta. O aluno se confundiu com os cálculos.
d)  Correta. Segundo o gráfico apresentado, 12 + 9 = 21 crianças de 7 a 12 anos visitaram a loja.
SAEB: Ler/identificar ou comparar dados estatísticos expressos em tabelas (simples ou de dupla entrada).
BNCC: EF04MA27 -- Analisar dados apresentados em tabelas simples ou de dupla entrada e em gráficos de
colunas ou pictóricos, com base em informações das diferentes áreas do conhecimento, e produzir
texto com a síntese de sua análise.
\end{enumerate}

\colorsec{Matemática – Módulo 9 – Treino}

\begin{enumerate}
\item
a) Incorreta. Não deve ser representado o todo.
b) Incorreta. O aluno não numerou bem nem o total de bombons nem a quantidade de cada tipo.
c) Correta. Cada tipo de bombom tem \frac{3}{6} do total de bombons na caixa, ou seja, \frac{1}{2}.
d) Incorreta. São 3, e não 4 bombons de cada tipo.
SAEB: Representar frações menores ou maiores que a unidade (por meio de representações
pictóricas) ou associar frações a representações pictóricas.
BNCC: EF04MA09 -- Reconhecer as frações unitárias mais usuais (\frac{1}{2}, \frac{1}{3}, \frac{1}{4}, \frac{1}{5}, \frac{1}{10} e \frac{1}{100}) como
unidades de medida menores do que uma unidade, utilizando a reta numérica como recurso.

\item
a)  Incorreta. Não temos divisões iguais em 3 partes.
b)  Incorreta. Não temos divisões iguais em 4 partes.
c)  Correta. Há quatro divisões iguais de um todo.
d)  Incorreta. Não temos divisão em 3 partes iguais.
SAEB: Representar frações menores ou maiores que a unidade (por meio de representações
pictóricas) ou associar frações a representações pictóricas.
BNCC: EF04MA09 -- Reconhecer as frações unitárias mais usuais (\frac{1}{2}, \frac{1}{3}, \frac{1}{4}, \frac{1}{5}, \frac{1}{10} e \frac{1}{100}) como
unidades de medida menores do que uma unidade, utilizando a reta numérica como recurso.

\item
a)  Incorreta. \frac{1}{4} de 48 não é igual a 8.
b)  Correta. \frac{1}{4} de 48 é igual a 12.
c)  Incorreta. \frac{1}{4} de 48 não é igual a 20.
d)  Incorreta. \frac{1}{4} de 48 não é igual a 28.
SAEB: Representar frações menores ou maiores que a unidade (por meio de representações
pictóricas) ou associar frações a representações pictóricas.
BNCC: EF04MA09 -- Reconhecer as frações unitárias mais usuais (\frac{1}{2}, \frac{1}{3}, \frac{1}{4}, \frac{1}{5}, \frac{1}{10} e \frac{1}{100}) como
unidades de medida menores do que uma unidade, utilizando a reta numérica como recurso.
\end{enumerate}

\colorsec{Matemática – Módulo 10 – Treino}

\begin{enumerate}
\item

\item

\item
\end{enumerate}

\colorsec{Matemática – Módulo 11 – Treino}

\begin{enumerate}
\item

\item

\item
\end{enumerate}

\colorsec{Simulado 1}

\begin{enumerate}
\item

\item

\item
\end{enumerate}


\colorsec{Simulado 2}

\begin{enumerate}
\item

\item

\item
\end{enumerate}

\colorsec{Simulado 3}

\begin{enumerate}
\item

\item

\item
\end{enumerate}

\colorsec{Simulado 4}

\begin{enumerate}
\item

\item

\item
\end{enumerate}