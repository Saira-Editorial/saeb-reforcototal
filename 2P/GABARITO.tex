\chapter{Respostas}
\pagestyle{plain}
\footnotesize

\pagecolor{gray!40}

\section*{Língua Portuguesa – Módulo 1 – Treino}

\begin{enumerate}
\item
(A) Incorreta. O animal encontrado por Júlia é uma \textit{formiga}, 
que começa com a letra F, não com a letra V.
(B) Incorreta. O animal encontrado por Júlia é uma \textit{formiga}, 
que começa com a letra F, não com a letra G.
(C) Incorreta. O animal encontrado por Júlia é uma \textit{formiga}, 
que começa com a letra F, não com a letra B. 
(D) Correta. Tanto \textit{formiga} quanto \textit{foca} são palavras
que começam com a letra F.
SAEB: Relacionar elementos sonoros das palavras com sua
representação escrita.
BNCC: EF02LP03 -- Ler e escrever palavras com correspondências
regulares diretas entre letras e fonemas (f, v, t, d, p, b) e
correspondências~regulares contextuais (c e q; e e o, em posição átona
em final de palavra).

\item
(A) Incorreta. \textit{Caju} termina com a letra U, mas \textit{cabelo} termina com a letra O.
(B) Incorreta. \textit{Caju} termina com a letra U, mas \textit{sapato} termina com a letra O.
(C) Incorreta. \textit{Caju} termina com a letra U, mas \textit{relógio} termina com a letra O.
(D) Correta. As palavras \textit{caju} e \textit{canguru} terminam com a letra U.
SAEB: Relacionar elementos sonoros das palavras com sua
representação escrita.
BNCC: EF02LP03 -- Ler e escrever palavras com correspondências
regulares diretas entre letras e fonemas (f, v, t, d, p, b) e
correspondências~regulares contextuais (c e q; e e o, em posição átona
em final de palavra).

\item
(A) Incorreta. A letra C da palavra \textit{Cinto} tem som de S, mas,
na palavra \textit{cama}, essa letra tem som de K. 
(B) Incorreta. A letra C da palavra \textit{Cinto} tem som de S, mas,
na palavra \textit{cubo}, essa letra tem som de K.
(C) Correta. A letra C das palavras \textit{Cinto} e \textit{cebola}
têm som de S.
(D) Incorreta. A letra C da palavra \textit{Cinto} tem som de S, mas,
na palavra \textit{cocada}, essa letra tem som de K.
SAEB: Relacionar elementos sonoros das palavras com sua
representação escrita.
BNCC: EF02LP03 -- Ler e
escrever palavras com correspondências~regulares diretas entre letras e
fonemas (f, v, t, d, p, b) e correspondências~regulares contextuais (c e
q; e e o, em posição átona em final de palavra).
\end{enumerate}

\section*{Língua Portuguesa – Módulo 2 – Treino}

\begin{enumerate}
\item
(A) Incorreta. A primeira sílaba de \textit{boneca} é formada por CV 
(consoante + vogal), mas a primeira sílaba de \textit{abacate} é formada
por V (vogal): a-ba-ca-te.
(B) Correta. A primeira sílaba de \textit{boneca} é formada por CV 
(consoante + vogal), da mesma maneira que a primeira sílaba de 
\textit{janela}. 
(C) Incorreta. A primeira sílaba de \textit{boneca} é formada por CV 
(consoante + vogal), mas a primeira sílaba de \textit{traça} é formada
por CCV (consoante + consoante + vogal): tra-ça.
(D) Incorreta. A primeira sílaba de \textit{boneca} é formada por CV 
(consoante + vogal), mas a primeira sílaba de \textit{prego} é formada
por CCV (consoante + consoante + vogal): pre-go.
SAEB: Ler palavras.
BNCC: EF02LP04 --- Ler e escrever corretamente palavras com
sílabas CV, V, CVC, CCV, identificando que existem vogais em todas as
sílabas.

\item
(A) Incorreta. A imagem representa uma menina brincando com \textit{balões}
não com \textit{bolhas de sabão}.
(B) Correta. Nesta alternativa, a imagem representa uma menina brincando
com bolhas de sabão.
(C) Incorreta. Nesta alternativa, a imagem representa uma menina brincando
com um catavento.
(D) Incorreta. A imagem representa uma menina brincando com \textit{bola}
não com \textit{bolhas de sabão}.
SAEB: Ler frases.
BNCC: EF02LP05 -- Ler e escrever corretamente palavras com marcas de
nasalidade (til, m, n).

\item
(A) Correta. As primeiras sílabas das palavras \textit{formiga} e 
\textit{borboleta} têm a mesma formação: CVC (consoante + vogal + 
consoante): for-mi-ga e bor-bo-le-ta.   
(B) Incorreta. A primeira sílaba de \textit{formiga} é formada
por CVC (consoante + vogal + consoante): for-mi-ga; mas a primeira 
sílaba de \textit{dragão} é formada por CCV (consoante + consoante + 
vogal): dra-gão.
(C) Incorreta. A primeira sílaba de \textit{formiga} é formada
por CVC (consoante + vogal + consoante): for-mi-ga; mas a primeira 
sílaba de \textit{laranja} é formada por CV (consoante + vogal): 
la-ran-ja.
(D) Incorreta. A primeira sílaba de \textit{formiga} é formada
por CVC (consoante + vogal + consoante): for-mi-ga; mas a primeira 
sílaba de \textit{garota} é formada por CV (consoante + vogal): 
ga-ro-ta.
SAEB: Ler palavras.
BNCC: EF02LP04 -- Ler e escrever corretamente palavras com
sílabas CV, V, CVC, CCV, identificando que existem vogais em todas as
sílabas.
\end{enumerate}

\section*{Língua Portuguesa – Módulo 3 – Treino}

\begin{enumerate}
\item
(A) Correta. De acordo com a leitura do texto, quando quer comer
bichinho, o pintinho amarelinho cisca o chão.
(B) Incorreta. No texto, o bater das asas do pintinho não está 
relacionado com o desejo de comer bichinho.
(C) Incorreta. No texto, o ato de catar na mão não está 
relacionado com o desejo de comer bichinho.
(D) Incorreta. No texto, o ato de fazer piu-piu não está 
relacionado com o desejo de comer bichinho.
SAEB: Localizar informações explícitas em textos.
BNCC: EF15LP03 -- Localizar informações explícitas em textos.

\item
(A) Incorreta, pois esse detalhe não tem relação com  a pergunta. 
(B) Correta, pois isso indica que ela não lavou a roupa corretamente.
(C) Incorreta, pois isso não tem relação com a preguiça.
(D) Incorreta, pois apenas esse detalhe não explica a pergunta.  
SAEB: Localizar informações explícitas em textos.
BNCC: EF15LP03 -- Localizar informações explícitas em textos.

\item
(A) Está incorreta, por considerar que a bola ficou colada na cola.
(B) Está incorreta, pois se confundiu onde a cola caiu.
(C) Está incorreta, por considerar que a bola colou no sapato.
(D) Está correta, pois está claro no texto que a cola saiu da gaveta.
SAEB: Localizar informações explícitas em textos.
BNCC: EF15LP03 Localizar informações explícitas em textos.
\end{enumerate}

\section*{Língua Portuguesa – Módulo 4 – Treino}

\begin{enumerate}
\item
(A) Incorreta. A finalidade da receita é ensinar a preparar alimentos.
(B) Incorreta. A receita não serve apenas para \textit{informar}; sua
finalidade é ensinar.  
(C) Correta. O texto apresentado foi utilizado para convidar Ana para uma brincadeira.
(D) Incorreta. O texto que serve para anunciar é, de forma geral, a 
\textit{propaganda}.
SAEB: Reconhecer a finalidade de um texto.
BNCC: EF15LP01 -- Identificar a função social de textos que circulam em
campos da vida social dos quais participa cotidianamente (a casa, a rua,
a comunidade, a escola) e nas mídias impressa, de massa e digital,
reconhecendo para que foram produzidos, onde circulam, quem os produziu
e a quem se destinam.

\item
(A) Incorreta. Tanto o texto do cartaz quanto a imagem evidenciam que
o cartaz é destinando para donos de cães e gatos.
(B) Incorreta. Tanto o texto do cartaz quanto a imagem evidenciam que
o cartaz é destinando para donos de cães e gatos.
(C) Incorreta. Tanto o texto do cartaz quanto a imagem evidenciam que
o cartaz é destinando para donos de cães e gatos.
(D) Correta. Tanto o texto do cartaz (especialmente a expressão ``para
cães e gatos'') quanto a imagem evidenciam que o cartaz é destinando para
donos de cães e gatos.
SAEB: Reconhecer a finalidade de um texto.
BNCC: EF15LP01 -- Identificar a função social de textos que circulam em
campos da vida social dos quais participa cotidianamente (a casa, a rua,
a comunidade, a escola) e nas mídias impressa, de massa e digital,
reconhecendo para que foram produzidos, onde circulam, quem os produziu
e a quem se destinam.

\item
(A) Incorreta. As receitas costumam servir para explicar como se deve
preparar um alimento, não para apresentar características.
(B) Incorreta. Um bilhete costuma servir para comunicações pessoais,
não para apresentar características.
(C) Correta. Os verbetes servem para apresentar características. 
(D) Incorreta. As notícias são textos que servem para informar sobre 
acontecimentos do cotidiano, não para apresentar características. 
SAEB: Reconhecer a finalidade de um texto.
BNCC: EF15LP01 -- Identificar a função social de textos que circulam em
campos da vida social dos quais participa cotidianamente (a casa, a rua,
a comunidade, a escola) e nas mídias impressa, de massa e digital,
reconhecendo para que foram produzidos, onde circulam, quem os produziu
e a quem se destinam.
\end{enumerate}

\section*{Língua Portuguesa – Módulo 5 – Treino}

\begin{enumerate}
\item
(A) Está incorreta, pois considerou apenas o fato de o cartaz apresentar uma lista de convocação.
(B) Está incorreta, pois considerou apenas o fato de o cartaz trazer uma data.
(C) Está incorreta, por considerou apenas o fato de o cartaz informar um local.
(D) Está correta, pois o cartaz trata da vacinação infantil.
SAEB: Inferir o assunto de um texto.

\item
(A) Está correta, pois o texto fala sobre os animais que, assim como os humanos, pegam piolhos.
(B) Está incorreta, pois o texto afirma exatamente o contrário.
(C) Está incorreta, pois esse não é o tema principal do texto.
(D) Está incorreta, pois o texto cita diferentes tipos de animais.
SAEB: Inferir o assunto de um texto.

\item
(A) Está incorreta, pois considerou o fato de a raposa ter achado comida.
(B) Está incorreta, pois considerou o fato de a raposa não conseguir
sair do buraco.
(C) Está correta, pois a raposa ficou inchada porque comeu demais.
(D) Está incorreta, pois esse não é o assunto principal do texto.
SAEB: Inferir o assunto de um texto.
\end{enumerate}

\section*{Língua Portuguesa – Módulo 6 – Treino}

\begin{enumerate}
\item
(A) Está incorreta, pois essa é apenas uma constatação de algo que ocorre no texto.
(B) Está incorreta, pois esse fato não explica a culpa da árvore.
(C) Está correta, pois esse fato fez com que a árvore se sentisse culpada.
(D) Está incorreta, pois esse fato não explica a ação posterior do machado.
SAEB: Inferir informações em textos verbais.

\item
(A) Está correta, pois esse profissional trabalha com joias.
(B) Está incorreta, pois não é essa a característica que define um joalheiro.
(C) Está incorreta, pois o joalheiro trabalha na joalheria .
(D) Está incorreta, pois essa não é a ocupação de um joalheiro.
SAEB: Inferir informações em textos verbais.

\item
(A) Está incorreta, pois não os acontecimentos descritos não ocorrem nesse conto.
(B) Está incorreta, pois o fato de aparecer um Rei não justifica a associação.
(C) Está correta, pois o conto da Bela Adormecida fala sobre uma donzela que permaneceu adormecida durante anos.
(D) Está incorreta, pois a Chapeuzinho Vermelho não se passa em um reino.
SAEB: Inferir informações em textos verbais.
\end{enumerate}

\section*{Língua Portuguesa – Módulo 7 – Treino}

\begin{enumerate}
\item
(A) Está correta, pois esse detalhe demonstra seu sorriso.
(B) Está incorreta, pois a menina está segurando um livro.
(C) Está incorreta, pois a menina não está chorando.
(D) Está correta, pois esse detalhe não indica felicidade.
SAEB: Inferir informações em textos que articulam linguagem verbal e não verbal.
BNCC: EF15LP14 -- Construir o sentido de histórias em quadrinhos
e tirinhas, relacionando imagens e palavras e interpretando recursos
gráficos (tipos de balões, de letras, onomatopeias).

\item
(A) Está correta, pois esse elemento divulga a campanha de vacinação.
(B) Está incorreta, pois os animais não estão brincando.
(C) Está incorreta, pois há apenas um cachorro.
(D) Está incorreta, pois o cartaz não apresenta crianças.
SAEB: Inferir informações em textos que articulam linguagem verbal e não verbal.
BNCC: EF15LP14 -- Construir o sentido de histórias em quadrinhos
e tirinhas, relacionando imagens e palavras e interpretando recursos
gráficos (tipos de balões, de letras, onomatopeias).

\item
(A) Está incorreta, pois o rapaz está acordado.
(B) Está correta, pois o formato do balão e a expressão facial do rapaz
demonstram isso.
(C) Está incorreta, pois o balão que indica um pensamento possui outro formato.
(D) Está incorreta, pois a imagem demonstra o contrário.
SAEB: Inferir informações em textos que articulam linguagem
verbal e não verbal.
BNCC: EF15LP14 -- Construir o sentido de histórias em quadrinhos
e tirinhas, relacionando imagens e palavras e interpretando recursos
gráficos (tipos de balões, de letras, onomatopeias).
\end{enumerate}

\section*{Simulado 1}

\begin{enumerate}
\item
(A) Está incorreta, pois confundiu o som de f com v.
(B) Está correta, pois a palavra fada começa com a mesma letra inicial de foca.
(C) Está incorreta, pois a palavra em questão começa com a letra t.
(D) Está incorreta, pois a palavra em questão começa com a letra d.
SAEB: relacionar elementos sonoros das palavras com sua representação escrita.
BNCC: EF02LP03 -- Ler e escrever palavras com correspondências~regulares
diretas entre letras e fonemas (f, v, t, d, p, b) e correspondências~regulares contextuais (c e q; e e o, em posição átona em final de palavra).

\item
(A) Está correta, pois a palavra papagaio começa com o mesmo som da
sílaba inicial da palavra panela.
(B) Está incorreta, pois o som de ba foi confundido com o som de pa.
(C) Está incorreta, pois a palavra começa com a sílaba ja.
(D) Está incorreta, pois a palavra começa com a sílaba pe.
SAEB: relacionar elementos sonoros das palavras com sua representação escrita.
BNCC: EF02LP03 -- Ler e escrever palavras com correspondências~regulares
diretas entre letras e fonemas (f, v, t, d, p, b) e correspondências~regulares contextuais (c e q; e e o, em posição átona em final de palavra).

\item
(A) Está incorreta, pois a palavra em questão se inicia com a sílaba ce.
(B) Está incorreta, pois a palavra em questão se inicia com a sílaba ci.
(C) Está correta, pois ambas as palavras se iniciam com a sílaba ca.
(D) Está incorreta, pois a palavra em questão se inicia com a sílaba ci.
SAEB: relacionar elementos sonoros das palavras com sua representação escrita.
BNCC: EF02LP03 -- Ler e escrever palavras com correspondências~regulares diretas entre letras e fonemas (f, v, t, d, p, b) e correspondências~regulares contextuais (c e q; e e o, em posição átona em final de palavra).

\item
(A) Está incorreta, pois a palavra foi grafada sem a letra u.
(B) Está incorreta, pois a palavra foi grafada com a letra c.
(C) Está correta, pois essa é a escrita correta da palavra.
(D) Está incorreta, pois a palavra foi grafada sem a letra e e com a letra c.
SAEB: relacionar elementos sonoros das palavras com sua representação escrita.
BNCC: EF02LP03 -- Ler e escrever palavras com correspondências regulares diretas entre letras e fonemas (f, v, t, d, p, b) e correspondências~regulares contextuais (c e q; e e o, em posição átona em final de palavra).

\item
(A) Está correta, pois é assim que a palavra ovelha deve ser separada.
(B) Está incorreta, pois a letra e foi separada de sua sílaba.
(C) Está incorreta, pois a letra l foi separada de sua sílaba
(D) Está incorreta, pois a palavra foi grafada sem a letra inicial o e sua última sílaba foi separada erroneamente.
SAEB: Ler palavras.
BNCC: EF02LP04 -- Ler e escrever corretamente palavras com sílabas
CV, V, CVC, CCV, identificando que existem vogais em todas as sílabas.

\item
(A) Está incorreta, pois essa palavra é iniciada pela sílaba go.
(B) Está incorreta, pois essa palavra é iniciada pela sílaba ba.
(C) Está correta, pois pois essa palavra é iniciada pelas sílabas a e ba.
(D) Está incorreta, pois essa palavra é iniciada pela sílaba gra.
SAEB: Ler palavras.
BNCC: EF02LP04 -- Ler e escrever corretamente palavras com
sílabas CV, V, CVC, CCV, identificando que existem vogais em todas as
sílabas.

\item
(A) Está incorreta, pois a palavra começa com a sílaba pe.
(B) Está correta, pois a palavra começa com a sílaba ba.
(C) Está incorreta, pois a palavra começa com a sílaba mo.
(D) Está incorreta, pois a palavra começa com a sílaba com.
SAEB: Ler palavras.
BNCC: EF02LP04 -- Ler e escrever corretamente palavras com sílabas CV, V, CVC, CCV, identificando que existem vogais em todas as sílabas.

\item
(A) Está correta, pois a imagem mostra crianças brincado com o pião.
(B) Está incorreta, pois a imagem não apresenta uma pipa.
(C) Está incorreta, pois há mais de uma criança na imagem.
(D) Está incorreta, pois o pião aparece na imagem.
SAEB: Ler frases.
BNCC: EF02LP05 Ler e escrever corretamente palavras com marcas
de nasalidade (til, m,n).

\item
(A) Está correta, pois a barata tem uma saia de filó.
(B) Está incorreta, pois a música diz apenas que ela não tem esse tipo de saia.
(C) Está incorreta, pois a música diz apenas que o sapato é da mãe dela.
(D) Está incorreta, pois a música afirma que ela não tem dinheiro para o sabão.
SAEB: Localizar informações explícitas em textos.
BNCC: EF15LP03 -- Localizar informações explícitas em textos.

\item
(A) Está incorreta, pois uma agenda é usada para registrar acontecimentos.
(B) Está incorreta, pois um convite é usado para chamar alguém para um evento.
(C) Está incorreta, pois um verbete é usado para consultar o significado de uma palavra.
(D) Está correta, pois usamos uma receita para aprender a preparar um prato.
SAEB: Reconhecer a finalidade de um texto.
BNCC: EF15LP01 -- Identificar a função social de textos que circulam em
campos da vida social dos quais participa cotidianamente (a casa, a rua,
a comunidade, a escola) e nas mídias impressa, de massa e digital,
reconhecendo para que foram produzidos, onde circulam, quem os produziu
e a quem se destinam.

\item
(A) Está incorreta, pois o texto não transmite um ensinamento.
(B) Está incorreta, pois o texto não apresenta elementos lúdicos.
(C) Está incorreta, pois o texto não cumpre a função de avisar o interlocutor sobre algo perigoso.
(D) Está correta, pois o texto anuncia um evento sobre o meio ambiente.
SAEB: Reconhecer a finalidade de um texto.
BNCC: EF15LP01 -- Identificar a função social de textos que
circulam em campos da vida social dos quais participa cotidianamente (a
casa, a rua, a comunidade, a escola) e nas mídias impressa, de massa e
digital, reconhecendo para que foram produzidos, onde circulam, quem os
produziu e a quem se destinam.

\item
(A) Está incorreta, pois a música não menciona a cor do boné.
(B) Está incorreta, pois o texto não menciona um navio.
(C) Está correta, pois é mencionado que o marinheiro anda de branco.
(D) Está incorreta, pois não é mencionado o amigo do marinheiro na música.
SAEB:: Inferir informações em textos verbais.

\item
(A) Está incorreta, pois isso não é dito no texto.
(B) Está incorreta, pois a música apenas menciona a panela no fogo.
(C) Está incorreta, pois é dito apenas que o macaco assobia.
(D) Está correta, pois o almoço é servido ao meio-dia.
SAEB:~Inferir informações em textos verbais.

\item
(A) Está incorreta, pois o menino não está emitindo um som.
(B) Está correta, pois a onomatopeia indica que o menino dorme.
(C) Está incorreta, pois o menino não está triste.
(D) Está incorreta, pois o menino não está falando com outra pessoa.
SAEB: Inferir informações em textos que articulam linguagem verbal e não verbal.
BNCC: EF15LP14 -- Construir o sentido de histórias em quadrinhos
e tirinhas, relacionando imagens e palavras e interpretando recursos
gráficos (tipos de balões, de letras, onomatopeias).

\item
(A) Está incorreta, pois ela demonstra um sentimento oposto.
(B) Está incorreta, pois ela não aparenta estar falando.
(C) Está incorreta, pois ela não aparenta estar chorando.
(D) Está correta, pois sua expressão denota raiva.
SAEB: Inferir informações em textos que articulam linguagem verbal e não verbal.
BNCC: EF15LP14 -- Construir o sentido de histórias em quadrinhos e tirinhas, relacionando imagens e palavras e interpretando recursos gráficos (tipos de balões, de letras, onomatopeias).

\item
(A) Está incorreta, pois esse não é o foco do texto.
(B) Está incorreta, pois o texto fala de um soldadinho específico.
(C) Está correta, pois o texto fala do soldadinho diferente.
(D) Está incorreta, pois a caixa é apenas citada no texto.
SAEB: Inferir o assunto de um texto verbal.
\end{enumerate}

\section*{Simulado 2}

\begin{enumerate}
\item
(A) Está incorreta, pois a palavra em questão começa com a sílaba pa.
(B) Está correta, pois a palavra em questão começa com a sílaba bo.
(C) Está incorreta, pois a palavra em questão começa com a sílaba va.
(D) Está incorreta, pois a palavra em questão começa com a sílaba fi.
SAEB: relacionar elementos sonoros das palavras com sua representação escrita.
BNCC: EF02LP03 -- Ler e escrever palavras com correspondências~regulares
diretas entre letras e fonemas (f, v, t, d, p, b) e correspondências~regulares contextuais (c e q; e e o, em posição átona em final de palavra).

\item
(A) Está correta, pois a palavra vale começa com a mesma sílaba da palavra vaca.
(B) Está incorreta, pois a palavra em questão começa com a sílaba fa.
(C) Está incorreta, pois a palavra em questão começa com a sílaba to.
(D) Está incorreta, pois a palavra em questão começa com a sílaba pa.
SAEB: relacionar elementos sonoros das palavras com sua representação escrita.
BNCC EF02LP03 -- Ler e escrever palavras com correspondências~regulares
diretas entre letras e fonemas (f, v, t, d, p, b) e correspondências~regulares contextuais (c e q; e e o, em posição átona em final de palavra).

\item
(A) Está incorreta, pois a palavra em questão começa com a letra f.
(B) Está incorreta, pois a palavra em questão começa com a letra b.
(C) Está correta, pois a palavra em questão também começa com a letra c.
(D) Está incorreta, pois a palavra em questão começa com a letra k.
SAEB: relacionar elementos sonoros das palavras com sua representação escrita.
BNCC: EF02LP03 -- Ler e escrever palavras com correspondências
regulares diretas entre letras e fonemas (f, v, t, d, p, b) e
correspondências~regulares contextuais (c e q; e e o, em posição átona
em final de palavra).

\item
(A) Está incorreta, pois a palavra domba não existe.
(B) Está correta, pois ao trocar P por B formamos a palavra bomba.
(C) Está incorreta, pois a palavra fomba não existe.
(D) Está incorreta, pois a palavra vomba não existe.
SAEB: relacionar elementos sonoros das palavras com sua representação escrita.
BNCC: EF02LP03 -- Ler e escrever palavras com correspondências
regulares diretas entre letras e fonemas (f, v, t, d, p, b) e
correspondências~regulares contextuais (c e q; e e o, em posição átona
em final de palavra).

\item
(A) Está correta, pois a separação silábica da palavra amora está correta.
(B) Está incorreta, pois a letra o foi separada de sua sílaba.
(C) Está incorreta, pois a letra r foi separada de sua sílaba.
(D) Está incorreta, pois a letra r foi separada de sua sílaba e as primeiras sílabas foram indevidamente unidas.
SAEB: Ler palavras.
BNCC: EF02LP04 -- Ler e escrever corretamente palavras com
sílabas CV, V, CVC, CCV, identificando que existem vogais em todas as
sílabas.

\item
(A) Está incorreta, pois a palavra em questão começa com uma consoante.
(B) Está correta, pois a palavra apito também começa com uma vogal, assim como a palavra anel.
(C) Está incorreta, pois a palavra em questão começa com uma consoante.
(D) Está incorreta, pois a palavra em questão começa com uma consoante.
SAEB: Ler palavras.
BNCC: EF02LP04 -- Ler e escrever corretamente palavras com
sílabas CV, V, CVC, CCV, identificando que existem vogais em todas as
sílabas.

\item
(A) Está correta, pois a sílaba medial da palavra melancia também é formada por essa sequência.
(B) Está incorreta, pois a sílaba medial da palavra formiga é formada por uma consoante e uma vogal.
(C) Está incorreta, pois a sílaba medial da palavra moqueca é formada por consoante, vogal e vogal.
(D) Está incorreta, pois a palavra batom é formada apenas por duas sílabas.
SAEB: Ler palavras.
BNCC: EF02LP04 Ler e escrever corretamente palavras com
sílabas CV, V, CVC, CCV, identificando que existem vogais em todas as
sílabas.

\item
(A) Está incorreta, pois a palavra foi grafada sem a letra m.
(B) Está incorreta, pois a palavra foi grafada sem a letra n.
(C) Está incorreta, pois a palavra foi grafada com a letra m no lugar de n.
(D) Está correta, pois a palavra está escrita corretamente.
SAEB: Ler palavras.
BNCC: EF02LP05 -- Ler e escrever corretamente palavras com marcas
de nasalidade (til, m,n).

\item
(A) Está correta, pois as crianças estão no rio.
(B) Está incorreta, pois as crianças estão brincando no campo.
(C) Está incorreta, pois as crianças estão brincando na praia.
(D) Está incorreta, pois as crianças estão brincando no parquinho.
SAEB: Ler frases.
BNCC: F02LP04 -- Ler e escrever corretamente palavras com sílabas
CV, V, CVC, CCV, identificando que existem vogais em todas as sílabas.

\item
(A) Está incorreta, pois os animais parecem ser amigos.
(B) Está incorreta, pois o cachorro aparenta estar calmo.
(C) Está incorreta, pois os animais estão calmos.
(D) Está correta, pois o gato e o cachorro estão felizes.
SAEB: Ler frases.
BNCC: F02LP04 -- Ler e escrever corretamente palavras com sílabas
CV, V, CVC, CCV, identificando que existem vogais em todas as sílabas.

\item
(A) Está correta, pois o rouxinol mora no interior da mata.
(B) Está incorreta, pois o texto apenas menciona a presença de um jardim.
(C) Está incorreta, pois o texto apenas afirma a existência de flores.
(D) Está incorreta, pois o texto não associa o rouxinol ao jardim.
SAEB: Localizar informações explícitas em textos.
BNCC EF15LP03 -- Localizar informações explícitas em textos.

\item
(A) Está incorreta, pois não esse não é um ambiente utilizado para divulgar esse tipo de campanha.
(B) Está incorreta, pois fazendas são espaços privados.
(C) Está correta, pois esse tipo de cartaz é exposto em locais
públicos para alcançar um grande número de pessoas.
(D) Está incorreta, pois esse tipo de ambiente não apresenta cartazes desse tipo.
SAEB: Reconhecer a finalidade de um texto.
BNCC EF15LP01 -- Identificar a função social de textos que circulam em
campos da vida social dos quais participa cotidianamente (a casa, a rua,
a comunidade, a escola) e nas mídias impressa, de massa e digital,
reconhecendo para que foram produzidos, onde circulam, quem os produziu
e a quem se destinam.

\item
(A) Está incorreta, pois o lobo nao demonstrou raiva pelo cachorro.
(B) Está incorreta, pois o texto não mostra a despedida entre os animais.
(C) Está correta, pois o texto mostra o diálogo entre os animais.
(D) Está incorreta, pois o cão não fez nenum convite ao lobo.
SAEB:~Inferir o assunto de um texto.

\item
(A) Está correta, pois essa é a regra da brincadeira.
(B) Está incorreta, pois isso não é dito na canção.
(C) Está incorreta, pois a canção apresenta uma regra diferente.
(D) Está incorreta, pois não há um prêmio descrito na brincadeira.
SAEB: Inferir informações em textos verbais.

\item
(A) Está incorreta, pois o gato está dormindo.
(B) Está incorreta, pois a imagem indica um comportamento oposto.
(C) Está incorreta, pois a imagem mostra o gato parado.
(D) Está correta, pois a onomatopeia indica que o gato dorme.
SAEB: Inferir informações em textos que articulam linguagem verbal e não verbal.
BNCC: EF15LP14 -- Construir o sentido de histórias em quadrinhos
e tirinhas, relacionando imagens e palavras e interpretando recursos
gráficos (tipos de balões, de letras, onomatopeias).

\item
(A) Está incorreta, pois o garoto parece animado.
(B) Está incorreta, pois o garoto está parado.
(C) Está correta, pois a expressão do garoto indica que ele está gritando.
(D) Está incorreta, pois o garoto está acordado.
SAEB: Inferir informações em textos que articulam linguagem verbal e não verbal.
BNCC: EF15LP14 -- Construir o sentido de histórias em quadrinhos
e tirinhas, relacionando imagens e palavras e interpretando recursos
gráficos (tipos de balões, de letras, onomatopeias).
\end{enumerate}

\section*{Simulado 3}

\begin{enumerate}
\item
(A) Está incorreta, pois a palavra em questão começa com a letra f.
(B) Está correta, pois a palavra tomate começa com a letra t.
(C) Está incorreta, pois a palavra em questão começa com a letra d.
(D) Está incorreta, pois a palavra em questão começa com a letra v.
SAEB: relacionar elementos sonoros das palavras com sua representação escrita.
BNCC EF02LP03 -- Ler e escrever palavras com correspondências~regulares
diretas entre letras e fonemas (f, v, t, d, p, b) e correspondências~regulares contextuais (c e q; e e o, em posição átona em final de palavra).

\item
(A) Está incorreta, pois a palavra em questão se inicia com a letra f.
(B) Está incorreta, pois a palavra em questão se inicia com a letra b.
(C) Está correta, pois a palavra em questão também se inicia com a letra v.
(D) Está incorreta, pois a palavra em questão se inicia com a letra b.
SAEB: relacionar elementos sonoros das palavras com sua representação escrita.
BNCC: EF02LP03 -- Ler e escrever palavras com correspondências
regulares diretas entre letras e fonemas (f, v, t, d, p, b) e
correspondências~regulares contextuais (c e q; e e o, em posição átona
em final de palavra).

\item
(A) Está incorreta, pois a palavra foi grafada com a letra e na segunda sílaba e sem a letra i na terceira.
(B) Está incorreta, pois a palavra foi grafada com a letra c.
(C) Está correta, pois essa é a escrita correta da palavra.
(D) Está incorreta, pois a palavra foi grafada sem a letra u.
SAEB: relacionar elementos sonoros das palavras com sua representação escrita.
BNCC: EF02LP03 -- Ler e escrever palavras com correspondências
regulares diretas entre letras e fonemas (f, v, t, d, p, b) e
correspondências~regulares contextuais (c e q; e e o, em posição átona
em final de palavra).

\item
(A) Está correta, pois a palavra também começa com a sílaba ca.
(B) Está incorreta, pois a palavra começa com a sílaba ci.
(C) Está incorreta, pois a palavra começa com a sílaba ce.
(D) Está incorreta, pois a palavra começa com a sílaba co.
SAEB: relacionar elementos sonoros das palavras com sua representação escrita.
BNCC: EF02LP03 -- Ler e escrever palavras com correspondências
regulares diretas entre letras e fonemas (f, v, t, d, p, b) e
correspondências~regulares contextuais (c e q; e e o, em posição átona
em final de palavra).

\item
(A) Está incorreta, pois a palavra apresenta apenas duas sílabas.
(B) Está incorreta, pois a palavra apresenta uma sílaba medial formada por consoante e vogal.
(C) Está incorreta, pois a palavra apresenta uma sílaba medial formada por consoante e vogal.
(D) Está correta, pois a sílaba medial da palavra apresenta a mesma sequência.
SAEB: Ler palavras.
BNCC: EF02LP04 -- Ler e escrever corretamente palavras com
sílabas CV, V, CVC, CCV, identificando que existem vogais em todas as
sílabas.

\item
(A) Está incorreta, pois a palavra foi grafada sem o til.
(B) Está incorreta, pois a palavra foi grafada com a letra d.
(C) Está correta, pois a palavra balão está escrita corretamente.
(D) Está incorreta, pois a palavra foi grafada com a letra d e o til foi colocado na letra incorreta.
SAEB: Ler palavras.
BNCC: EF02LP05 -- Ler e escrever corretamente palavras com marcas
de nasalidade (til, m,n).

\item
(A) Está incorreta, pois a segunda sílaba foi separada indevidamente.
(B) Está correta, pois essa é a separação silábica da palavra apontador.
(C) Está incorreta, pois a segunda e a quarta sílabas foram separadas indevidamente.
(D) Está incorreta, pois a terceira sílaba foi separada indevidamente.
SAEB: Ler palavras.
BNCC: EF02LP04 -- Ler e escrever corretamente palavras com
sílabas CV, V, CVC, CCV, identificando que existem vogais em todas as
sílabas.

\item
(A) Está incorreta, pois a imagem mostra exatamente o contrário.
(B) Está incorreta, pois a imagem não mostra uma boneca.
(C) Está correta, pois a menina está brincando com equipamentos usados por médicos.
(D) Está incorreta, pois a imagem retrata uma brincadeira.
SAEB: Ler frases.
BNCC: EF02LP04 -- Ler e escrever corretamente palavras com sílabas
CV, V, CVC, CCV, identificando que existem vogais em todas as sílabas.

\item
(A) Está incorreta, pois essa quantidade é associada a outros ingredientes.
(B) Está incorreta, pois essa é a quantidade de xícaras de tomate.
(C) Está correta, pois serão usados seis damascos na receita.
(D) Está incorreta, pois essa quantidade é associada a outros ingredientes.
SAEB: Reconhecer a finalidade de um texto.
BNCC: EF15LP01 -- Identificar a função social de textos que
circulam em campos da vida social dos quais participa cotidianamente (a
casa, a rua, a comunidade, a escola) e nas mídias impressa, de massa e
digital, reconhecendo para que foram produzidos, onde circulam, quem os
produziu e a quem se destinam.

\item
(A) Está correta, pois a lista é usada para organizar os itens que serão comprados.
(B) Está incorreta, pois uma agenda é usada para organizar os acontecimentos em um período de tempo.
(C) Está incorreta, pois um bilhete é uma forma de comunicação rápida e informal.
(D) Está incorreta, pois um anúncio é usado para divulgar um produto ou evento.
SAEB: Reconhecer a finalidade de um texto.
BNCC: EF15LP01 --Identificar a função social de textos que
circulam em campos da vida social dos quais participa cotidianamente (a
casa, a rua, a comunidade, a escola) e nas mídias impressa, de massa e
digital, reconhecendo para que foram produzidos, onde circulam, quem os
produziu e a quem se destinam.

\item
(A) Está incorreta, pois a imagem retrata outro grupo.
(B) Está incorreta, pois não há adultos na imagem.
(C) Está correta, pois o anúncio retrata crianças.
(D) Está incorreta, pois esse não é o público-alvo do anúncio.
SAEB: Reconhecer a finalidade de um texto.
BNCC EF15LP01 -- Identificar a função social de textos que circulam em
campos da vida social dos quais participa cotidianamente (a casa, a rua,
a comunidade, a escola) e nas mídias impressa, de massa e digital,
reconhecendo para que foram produzidos, onde circulam, quem os produziu
e a quem se destinam.

\item
(A) Está incorreta, pois o foco do cartaz é a doação de alimentos.
(B) Está correta, pois o cartaz está divulgando a arrecadação de alimentos.
(C) Está incorreta, pois os alimentos serão doados.
(D) Está incorreta, pois o cartaz não menciona a seleção.
SAEB Inferir o assunto de um texto verbal.

\item
(A) Está correta, pois o corvo teve a ideia de jogar os seixos para a água subir.
(B) Está incorreta, pois o texto não menciona esse sentimento.
(C) Está incorreta, pois esse não é o foco do texto.
(D) Está incorreta, pois o texto trata da superação da sede.
SAEB:~Inferir informações em textos verbais.

\item
(A) Está correta, pois a pomba jogou a folha para a formiga não se afogar.
(B) Está incorreta, pois a pomba quis ajudar a formiga.
(C) Está incorreta, pois o texto apresenta algo oposto.
(D) Está incorreta, pois o texto não apresenta alimentos.
SAEB: Inferir informações em textos verbais.

\item
(A) Está incorreta, pois a imagem mostra um sentimento oposto.
(B) Está incorreta, pois a imagem não mostra sinais de cansaço.
(C) Está correta, pois pois sua expressão facial indica que ele está bravo.
(D) Está incorreta, pois a imagem não mostra elementos que levem a essa conclusão.
SAEB: Inferir informações em textos que articulam linguagem verbal e não verbal.
BNCC: EF15LP14 -- Construir o sentido de histórias em quadrinhos
e tirinhas, relacionando imagens e palavras e interpretando recursos
gráficos (tipos de balões, de letras, onomatopeias).

\item
(A) Está incorreta, pois o gesto não tem essa conotação.
(B) Está correta, pois o dedo na boca significa um pedido de silêncio.
(C) Está incorreta, pois isso seria indicado por um gesto diferente.
(D) Está incorreta, pois a imagem indica um pedido diferente.
SAEB: Inferir informações em textos que articulam linguagem
verbal e não verbal.
BNCC: EF15LP14 -- Construir o sentido de histórias em quadrinhos
e tirinhas, relacionando imagens e palavras e interpretando recursos
gráficos (tipos de balões, de letras, onomatopeias).
\end{enumerate}

\section*{Simulado 4}

\begin{enumerate}
\item
(A) Está incorreta, pois a palavra em questão começa com a letra v.
(B) Está incorreta, pois a palavra em questão começa com a letra f.
(C) Está incorreta, pois a palavra em questão começa com a letra b.
(D) Está correta, pois a palavra pipa começa com o mesmo som da palavra pato.
SAEB: relacionar elementos sonoros das palavras com sua representação
escrita.
BNCC: EF02LP03 -- Ler e escrever palavras com correspondências
regulares diretas entre letras e fonemas (f, v, t, d, p, b) e
correspondências~regulares contextuais (c e q; e e o, em posição átona
em final de palavra).

\item
(A) Está incorreta, pois a palavra em questão começa com a sílaba da.
(B) Está correta, pois a palavra em questão também começa com a sílaba ba.
(C) Está incorreta, pois a palavra em questão começa com a sílaba fa.
(D) Está incorreta, pois a palavra em questão começa com a sílaba pa.
SAEB: relacionar elementos sonoros das palavras com sua representação escrita.
BNCC: EF02LP03 -- Ler e escrever palavras com correspondências
regulares diretas entre letras e fonemas (f, v, t, d, p, b) e
correspondências~regulares contextuais (c e q; e e o, em posição átona
em final de palavra).

\item
(A) Está incorreta, pois a palavra em questão começa com a sílaba ci.
(B) Está incorreta, pois a palavra em questão começa com a sílaba cus.
(C) Está correta, pois a palavra em questão também começa com a sílaba ce.
(D) Está incorreta, pois a palavra em questão começa com a sílaba car.
SAEB: relacionar elementos sonoros das palavras com sua representação escrita.
BNCC: EF02LP03 -- Ler e escrever palavras com correspondências
regulares diretas entre letras e fonemas (f, v, t, d, p, b) e
correspondências~regulares contextuais (c e q; e e o, em posição átona
em final de palavra).

\item
(A) Está incorreta, pois a palavra danda não existe.
(B) Está correta, pois ao trocar b por p formamos a palavra panda.
(C) Está correta, pois a palavra fanda não existe.
(D) Está incorreta, pois a palavra canda não existe.
SAEB relacionar elementos sonoros das palavras com sua representação escrita.
BNCC: EF02LP03 Ler e escrever palavras com correspondências
regulares diretas entre letras e fonemas (f, v, t, d, p, b) e
correspondências~regulares contextuais (c e q; e e o, em posição átona
em final de palavra).

\item
(A) Está incorreta, pois a palavra em questão é iniciada por uma consoante.
(B) Está incorreta, pois a palavra em questão é iniciada por uma consoante.
(C) Está incorreta, pois a palavra em questão é iniciada por uma consoante.
(D) Está correta, pois a palavra abacaxi também começa com a vogal a.
SAEB: Ler palavras.
BNCC: EF02LP04 -- Ler e escrever corretamente palavras com
sílabas CV, V, CVC, CCV, identificando que existem vogais em todas as
sílabas.

\item
(A) Está incorreta, pois a última sílaba da palavra em questão apresenta a sequência consoante e vogal.
(B) Está incorreta, pois a última sílaba da palavra em questão apresenta a sequência consoante e vogal.
(C) Está correta, pois a palavra comadre termina com a sequência consoante consoante vogal.
(D) Está incorreta, pois a última sílaba da palavra em questão apresenta a sequência consoante e vogal.
SAEB: Ler palavras.
BNCC: EF02LP04 -- Ler e escrever corretamente palavras com
sílabas CV, V, CVC, CCV, identificando que existem vogais em todas as
sílabas.

\item
(A) Está incorreta, pois a palavra foi grafada sem a letra m.
(B) Está incorreta, pois a palavra foi grafada com a letra m em uma posição incorreta.
(C) Está correta, pois a palavra lâmpada é escrita dessa forma.
(D) Está incorreta, pois a palavra foi grafada com a letra n.
SAEB: relacionar elementos sonoros das palavras com sua representação
escrita.
BNCC: EF02LP05 -- Ler e escrever corretamente palavras com marcas de nasalidade
(til, m...).

\item
(A) Está incorreta, pois as meninas estão andando de bicicleta corretamente.
(B) Está correta, pois as meninas estão andando de bicicleta.
(C) Está incorreta, pois isso não é retratado na imagem.
(D) Está incorreta, pois a imagem mostra as meninas andando de bicicleta.
SAEB: Ler frases.
BNCC: F02LP04 -- Ler e escrever corretamente palavras com sílabas
CV, V, CVC, CCV, identificando que existem vogais em todas as sílabas.

\item
(A) Está incorreta, pois a sílaba medial foi separada incorretamente.
(B) Está incorreta, pois a sílaba inicial foi separada incorretamente.
(C) Está incorreta, pois as sílabas inicial e medial foram separadas incorretamente.
(D) Está correta, pois a palavra bicicleta deve ser separada dessa maneira.
SAEB: Ler palavras.
BNCC: EF02LP04 -- Ler e escrever corretamente palavras com
sílabas CV, V, CVC, CCV, identificando que existem vogais em todas as
sílabas.

\item
(A) Está incorreta, pois ele possui sete cartas.
(B) Está correta, pois Caio tem nove cartas.
(C) Está incorreta, pois ele tem quatro cartas.
(D) Está incorreta, pois ela tem oito cartas.
SAEB: Localizar informações explícitas em textos
BNCC: EF15LP03 -- Localizar informações explícitas em textos.

\item
(A) Está incorreta, pois um esportista não atua escrevendo textos.
(B) Está correta, pois a notícia é escrita por um jornalista.
(C) Está incorreta, pois um bombeiro desenvovle outras atividades.
(D) Está incorreta, pois um político normalmente não atua nessa área.
SAEB Reconhecer a finalidade de um texto.
BNCC EF15LP01 -- Identificar a função social de textos que circulam em
campos da vida social dos quais participa cotidianamente (a casa, a rua,
a comunidade, a escola) e nas mídias impressa, de massa e digital,
reconhecendo para que foram produzidos, onde circulam, quem os produziu
e a quem se destinam.

\item
(A) Está incorreta, pois o cartaz não apresenta descrições.
(B) Está correta, pois o texto é um convite.
(C) Está incorreta, pois o cartaz não descreve brincadeiras.
(D) Está incorreta, pois o texto não traz informações desse tipo.
SAEB: Reconhecer a finalidade de um texto.
BNCC: EF15LP01 -- Identificar a função social de textos que circulam em
campos da vida social dos quais participa cotidianamente (a casa, a rua,
a comunidade, a escola) e nas mídias impressa, de massa e digital,
reconhecendo para que foram produzidos, onde circulam, quem os produziu
e a quem se destinam.

\item
(A) Está correta, pois contar vitória antes do tempo não garante um resultado positivo.
(B) Está incorreta, pois não foi isso que ocorreu no conto.
(C) Está incorreta, pois o conto mostrou exatamente o contrário.
(D) Está incorreta, pois o conto apresentou uma quebra de justificativa.
SAEB: Inferir o assunto de um texto verbal.

\item
(A) Está correta, pois é possível chegarmos a essa conclusão.
(B) Está incorreta, pois essa resposta entra em contradição com a pergunta.
(C) Está incorreta, pois essa resposta não tem relação com a pergunta.
(D) Está incorreta, pois o texto mostra exatamente o contrário.
SAEB: Inferir o assunto de um texto verbal.

\item
(A) Está incorreta, pois a expressão da mulher não mostra isso.
(B) Está incorreta, pois a mulher não parece estar falando.
(C) Está incorreta, pois a mulher está feliz.
(D) Está correta, pois a onomatopeia mostra que ela está buzinando.
SAEB: Inferir informações em textos que articulam linguagem
verbal e não verbal.
BNCC: EF15LP14 -- Construir o sentido de histórias em quadrinhos
e tirinhas, relacionando imagens e palavras e interpretando recursos
gráficos (tipos de balões, de letras, onomatopeias).
\end{enumerate}