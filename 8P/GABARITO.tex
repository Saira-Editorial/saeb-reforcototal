%!TEX root=./LIVRO.tex
\chapter{Respostas}
\pagestyle{plain}
\footnotesize

\pagecolor{gray!40}

\section*{Língua Portuguesa – Módulo 1 – Treino}

\begin{enumerate}
\item

SAEB: Identificar teses, opiniões, posicionamentos explícitos e
argumentos em textos.

BNCC: EF89LP04 -- Identificar e avaliar
teses/opiniões/posicionamentos explícitos e implícitos, argumentos e
contra-argumentos em textos argumentativos do campo (carta de leitor,
comentário, artigo de opinião, resenha crítica etc.), posicionando-se
frente à questão controversa de forma sustentada.

(A) Incorreta. O texto não é do gênero crônica, pois a situação
explicitada não está situada no tempo, embora o texto de fato expresse
uma experiência literária da resenhista. 

(B) Incorreta. O texto não é do gênero diário, pois seu objetivo não é falar sobre o cotidiano literário da resenhista, nem sobre outro tipo de situação cotidiana. 

(C) Correta. O texto é do gênero resenha crítica, porque apresenta a avaliação e a opinião da resenhista sobre um livro que ela leu, para indicar a leitura ao leitor. 

(D) Incorreta. O texto não é do gênero anúncio publicitário,
pois a avaliação que a resenhista trará sobre o livro não tem o objetivo
de vendê-lo como mercadoria para o leitor, e sim dizer-lhe se deve
experimentar lê-lo ou não.

\item
SAEB: Identificar o uso de recursos persuasivos em textos verbais e não
verbais. 

BNCC: EF89LP04 -- Identificar e avaliar
teses/opiniões/posicionamentos explícitos e implícitos, argumentos e
contra-argumentos em textos argumentativos do campo (carta de leitor,
comentário, artigo de opinião, resenha crítica etc.), posicionando-se
frente à questão controversa de forma sustentada.

(A) Incorreta. O trecho expressa um convite para que o leitor continue
acompanhando a resenha para saber se a autora gostou ou não do livro.

(B) Incorreta. O trecho expressa um gosto da resenhista por um gênero
literário específico e não se refere ainda ao livro em questão, o qual, aliás, nesse ponto da resenha ainda nem tinha sido citado. 

(C) Incorreta. O trecho
expressa uma hipótese ou dúvida que a resenhista deixa em suspense, como
forma de incentivar o leitor a continuar lendo a resenha para saber a
opinião dela sobre o livro. 

(D) Correta. O trecho expressa o diferencial
do livro em questão, pois a resenhista diz que ele é diferente dos
outros livros do gênero distopia futurista por mostrar um apocalipse
financeiro nos Estados Unidos, em vez do mais comum, que é uma invasão
zumbi ou alienígena, e isso a deixa ainda mais instigada a lê-lo e pode
também incentivar o leitor.

\item
SAEB: Identificar o uso de recursos persuasivos em textos verbais e não
verbais. 

BNCC: EF89LP04 -- Identificar e avaliar
teses/opiniões/posicionamentos explícitos e implícitos, argumentos e
contra-argumentos em textos argumentativos do campo (carta de leitor,
comentário, artigo de opinião, resenha crítica etc.), posicionando-se
frente à questão controversa de forma sustentada.

(A) Incorreta. O personagem, tradicionalmente reconhecido como parte de campanhas de vacinação, já faz com que as crianças reconheçam um exemplo positivo em relaçãoà vacinação. 

(B) Correta. A motivação vem do fato de que o personagem é uma imagem lúdica e divertida, já tradiionalmente ligada às campanhas de vacinação.

(C) Incorreta. O fato de haver um escudo pode gerar confusão, mas esse escudo representa a proteção contra doenças, não um afastamento de qualquer manifestação ou sensação de felicidade.

(D) Incorreta. Todas s doenças mencionadas podem gerar consequências graves. Portanto, não se deve relativizar a vacinação ou ignorar a gravidade delas.

\end{enumerate}

\section*{Língua Portuguesa – Módulo 2 – Treino}

\begin{enumerate}
\item
SAEB: Identificar elementos constitutivos de textos pertencentes ao
domínio jornalístico/midiático. 

BNCC: EF08LP01 -- Identificar e comparar
as várias editorias de jornais impressos e digitais e de sites
noticiosos, de forma a refletir sobre os tipos de fato que são
noticiados e comentados, as escolhas sobre o que noticiar e o que não
noticiar e o destaque/enfoque dado e a fidedignidade da informação.

(A) Incorreta. O gênero textual é notícia e, assim, não se encontram no
texto marcas de valoração ou pontos de vista. Apesar de, naturalmente, qualquer texto apresentar, ainda que secundariamente, modalizações que indicam conteúdo opinativo.

(B) Incorreta. Embora o
texto possa gerar a sensação de alerta na população, por antecipar
grande volume de chuva, a finalidade comunicativa primária de uma
notícia não é alertar. 

(C) Incorreta. O gênero notícia não tem como
característica o didatismo, nem se encontram no texto marcas de
instrução ao leitor. 

(D) Correta. O gênero notícia tem a finalidade de
informar o leitor sobre acontecimentos passados e futuros relevantes
para a sociedade.

\item
SAEB: Analisar efeitos de sentido produzido pelo uso de formas de
apropriação textual (paráfrase, citação etc.). 

BNCC: EF08LP01 --
Identificar e comparar as várias editorias de jornais impressos e
digitais e de \emph{sites} noticiosos, de forma a refletir sobre os
tipos de fato que são noticiados e comentados, as escolhas sobre o que
noticiar e o que não noticiar e o destaque/enfoque dado e a
fidedignidade da informação.

(A) Incorreta. Outros tipos de destaque poderiam ser usados para chamar
a atenção do leitor, tais como negrito, itálico, sublinhado, os quais
não carregariam, entretanto, o mesmo valor do uso das aspas nesse trecho.

(B) Incorreta. O trecho entre aspas não traz dados no sentido estrito;
trata-se apenas de parte de um texto maior da Organização Pan-Americana
de Saúde, ao qual o leitor não tem acesso na notícia em questão. 

(C) Incorreta. Do ponto de vista semântico, o trecho entre aspas realmente
significa que o vírus se espalha de uma maneira nova, diferentemente do
modo até então conhecido. Porém, as aspas não foram usadas simplesmente
para informar isso. Seu valor é externo ao texto, pois está relacionado
aos princípios de isenção e objetividade do jornalismo. 

(D) Correta. O
texto jornalístico, em geral, preza pela objetividade e pela isenção.
Uma das formas de atender a esses requisitos é citar entre aspas as
palavras de outrem. No texto em questão, as aspas foram usadas para
evitar assumir a responsabilidade por uma afirmação categórica feita, na
realidade, por uma instituição que tem legitimidade para tal (a
Organização Pan-Americana de Saúde). Isso significa que tal instituição
tem maior credibilidade, de modo que as aspas servem, ao mesmo tempo,
para comprovar que a notícia veicula informações reais já confirmadas
por instituição competente.

\item

SAEB: Identificar elementos constitutivos de gêneros de divulgação
científica.

(A) Incorreta. O texto não é construído com linguagem técnica e, por
isso, seu público-alvo não são estudiosos nem pessoas com grande
conhecimento científico.

(B) Incorreta. O texto não é construído com linguagem técnica e, por
isso, seu público-alvo não são biólogos, que, para aprenderem conteúdos
de sua área, provavelmente procurarão textos científicos especializados,
normalmente divulgados em forma de artigos científicos, teses,
dissertações.

(C) Correta. O texto tem um tom bastante informal e utiliza vocabulário
e linguagem simples e cotidianos, sem perder seu caráter científico,
pois seu objetivo é fazer divulgação científica correta para o grande
público não especializado.

(D) Incorreta. Não é o objetivo do texto divulgar o ditado popular, que,
nesse caso, serve apenas para ativar um conhecimento prévio e corriqueiro
do leitor, de modo a estabelecer com ele uma interação e ganhar sua
atenção. Isso se confirma ainda pelo fato de apenas na introdução o
texto citar o ditado e não mais abordar o assunto.
\end{enumerate}

\section*{Língua Portuguesa – Módulo 3 – Treino}

\begin{enumerate} 
\item
SAEB: Analisar elementos constitutivos de textos pertencentes ao domínio
literário. 

BNCC: EF69LP44 -- Inferir a presença de valores sociais,
culturais e humanos e de diferentes visões de mundo, em textos
literários, reconhecendo nesses textos formas de estabelecer múltiplos
olhares sobre as identidades, sociedades e culturas e considerando a
autoria e o contexto social e histórico de sua produção.

(A) Incorreta. O apelo moral não está presente no poema.

(B) Correta. O apelo sensorial está presente no poema por meio de
referências à sensação térmica no Rio de Janeiro, à temperatura e à
degustação do café, ao paladar (``boca adocicada''), ao sabor doce do
caramelo e à própria boca.

(C) Incorreta. O apelo crítico não está presente no poema, o qual não traz visões de mundo ou traços opinativos.

(D) Incorreta. O apelo humorístico não está presente no poema, que tem como base descrições sensotiais diversas.

\item

SAEB: Analisar a intertextualidade entre textos literários ou entre
estes e outros textos verbais ou não verbais. 

BNCC: EF89LP32 -- Analisar
os efeitos de sentido decorrentes do uso de mecanismos de
intertextualidade (referências, alusões, retomadas) entre os textos
literários, entre esses textos literários e outras manifestações
artísticas (cinema, teatro, artes visuais e midiáticas, música), quanto
aos temas, personagens, estilos, autores etc., e entre o texto original
e paródias, paráfrases, pastiches, trailer honesto, vídeos-minuto,
vidding, dentre outros.

(A) Correta. O eu lírico exalta o momento em que toma café expresso com
caramelo em uma tarde do Rio de Janeiro, a qual ele considera rara, que,
por isso mesmo, se torna um momento de degustação e deleite. Isso
comprova que o eu lírico valoriza coisas simples como um singelo café
com caramelo.

(B) Incorreta. O poema descreve as sensações de um momento específico, e
não pensamentos. Por isso, não há questionamento, nos versos, acerca da importância da qualidade dos pensamentos de alguém.

(C) Incorreta. O poema descreve as sensações de um momento específico, e
não a saúde ou a sabedoria, que seriam parte de uma avaliação e de uma reflexão mais ampla sobre a vida.

(D) Incorreta. O poema descreve as sensações de um momento específico, e
não a busca da felicidade ou sua fugacidade, como seria no caso de compará-la a uma borboleta.

\item
SAEB: Inferir a presença de valores sociais, culturais e humanos em
textos literários. 

BNCC: EF69LP44 -- Inferir a presença de valores
sociais, culturais e humanos e de diferentes visões de mundo, em textos
literários, reconhecendo nesses textos formas de estabelecer múltiplos
olhares sobre as identidades, sociedades e culturas e considerando a
autoria e o contexto social e histórico de sua produção.

(A) Correta. A personagem é discriminada por sua origem social, pois ela
é uma \emph{dalit}, um grupo social excluído da sociedade indiana o qual
não participa sequer do sistema de castas, conforme diz a narrativa.

(B) Incorreta. A personagem é discriminada por ser uma \emph{dalit,} um
grupo social excluído da sociedade na Índia. O fato de ela ter uma filha
não é mencionado como causa da discriminação sofrida, mesmo porque não é
possível, no trecho, saber se é mãe solo.

(C) Incorreta. O gênero da personagem, inicialmente, parece ter relação
com a proibição de ir à escola, mas a narrativa logo esclarece que todo
o grupo social dos \emph{dalits}, do qual ela faz parte, vive à margem
da sociedade.

(D) Incorreta. O texto cita uma fala de Gandhi sobre os \emph{dalits}
serem filhos de Deus, mas isso não é mencionado com o objetivo de
situá-los em alguma religião. O texto não cita a religião desse grupo
social.
\end{enumerate}

\section*{Língua Portuguesa – Módulo 4 – Treino}

\begin{enumerate}
	\item

SAEB: Analisar efeitos de sentido produzido pelo uso de formas de
apropriação textual (paráfrase, citação etc.). 

BNCC: EF69LP43 --
Identificar e utilizar os modos de introdução de outras vozes no texto
-- citação literal e sua formatação e paráfrase --, as pistas
linguísticas responsáveis por introduzir no texto a posição do autor e
dos outros autores citados (``Segundo X; De acordo com Y; De minha/nossa
parte, penso/amos que''\ldots) e os elementos de normatização (tais como
as regras de inclusão e formatação de citações e paráfrases, de
organização de referências bibliográficas) em textos científicos,
desenvolvendo reflexão sobre o modo como a intertextualidade e a
retextualização ocorrem nesses textos.

(A) Incorreta. As vozes dos personagens estão presentes no texto; apenas
não estão marcadas graficamente, por se tratar de discurso indireto
livre. 
(B) Correta. A técnica narrativa empregada é o discurso indireto
livre, caracterizado por não marcar graficamente as vozes dos
personagens, inserindo-as dentro da voz do narrador. 
(C) Incorreta. As
vozes das personagens mantêm sua importância expressiva, embora estejam
inseridas na voz do narrador. 
(D) Incorreta. Os sinais gráficos de
marcação do discurso direto, como travessão e aspas, não são usados no
discurso indireto livre.

\item

SAEB: Analisar efeitos de sentido produzido pelo uso de formas de
apropriação textual (paráfrase, citação etc.). 

BNCC: EF89LP05 --
Analisar o efeito de sentido produzido pelo uso, em textos, de recurso a
formas de apropriação textual (paráfrases, citações, discurso direto,
indireto ou indireto livre).

(A) Incorreta. As angústias do menino e da galinha são claramente
perceptíveis nos trechos que marcam a presença de suas vozes no texto.
Essas angústias são escritas como perguntas retóricas feitas para si
mesmo.

(B) Incorreta. As vozes dos personagens aparecem tal como são ditas ou
pensadas por eles, embora integradas à voz do narrador, e o narrador não
tem o poder de exagerá-las, pois não são reportadas em discurso
indireto.

(C) Correta. O narrador conhece os pensamentos do menino e da galinha
porque suas vozes estão integradas à voz narrativa, pois é como se o
narrador as reproduzisse para o leitor. Entretanto, vale lembrar que as
vozes dos personagens são reproduzidas tal como são ditas ou pensadas
por eles.

(D) Incorreta. O narrador, no discurso indireto livre, reproduz as vozes
do menino e da galinha tal como são ditas ou pensadas, embora não
estejam marcadas graficamente como discurso direto.

\item
SAEB: Identificar o uso de recursos persuasivos em textos verbais e não
verbais. 

BNCC: EF89LP04 -- Identificar e avaliar
teses/opiniões/posicionamentos explícitos e implícitos, argumentos e
contra-argumentos em textos argumentativos do campo (carta de leitor,
comentário, artigo de opinião, resenha crítica etc.), posicionando-se
frente à questão controversa de forma sustentada.

(A) Incorreta. A análise da entrevista mostra que o nível de formalidade
entre os participantes foi baixo, o que se percebe pelo emprego de
linguagem informal.

(B) Incorreta. O trecho transcrito da entrevista ocorre entre dois
participantes apenas, o entrevistador e o entrevistado.

(C) Correta. Ao dar voz ao entrevistado, esse gênero textual permite um
contato direto do leitor com as respostas dele. Na entrevista
transcrita, o fã passa a conhecer os detalhes do encontro entre os
artistas Nando Reis e Pitty pelo ponto de vista do próprio cantor.
Informações de bastidores geralmente são dadas pelo próprio artista ou
por pessoas muito próximas a ele, pois estão no universo de sua vida
privada.

(D) Incorreta. A entrevista não traz perguntas que exijam emissão de
opinião; trata-se de assunto do cotidiano dos artistas.
\end{enumerate}

\section*{Língua Portuguesa – Módulo 5 – Treino}

\begin{enumerate}
	\item
SAEB: Distinguir fatos de opiniões em textos.

(A) Correta. O texto é do gênero artigo científico e, como tal,
privilegia a transmissão de informação com objetividade. Porém, há nele
um traço sutil de opinião. No trecho ``Isso era previsível'', avalia-se o
fato como uma decorrência óbvia, pois ambos estabelecem entre si uma
relação necessária de causa e consequência. 

(B) Incorreta. O trecho expressa um fato, que aponta para uma ação que, indiscutivelmente e por conhecimento empírico, leva as algas à morte. 

(C) Incorreta. O trecho expressa um fato, que, embora apresente algumas imprecisões, como "determinadas circunstâncias", que não são especificadas, é algo fruto de observação simples e isenta.

(D) Incorreta. O trecho expressa um fato, que, possivelmente, pode ser comprovado por meio do estudo de análises quantitativas e estatísticas.

\item

Saeb: Inferir informações implícitas em distintos textos.

(A) Incorreta. O narrador busca interagir com o leitor para estabelecer
com ele a correspondência entre suas vivências, de modo que não se sabe
se tal intenção provoca semelhanças ou descompassos entre as
subjetividades de narrador e leitor. 

(B) Incorreta. O narrador expõe
situações de sua vivência para buscar interação com as vivências do
leitor, de modo que as ideias são voltadas para ambas as subjetividades,
do narrador e do leitor. 

(C) Incorreta. O narrador expõe situações de
sua vivência para buscar interação com as vivências do leitor, de modo
que as ideias são voltadas para ambas as subjetividades, sem qualquer exclusão de uma das partes. 

(D) Correta. A conexão expressa no título é buscada quando o
narrador expõe sua subjetividade por meio do relato de vivências e
experiências pessoais, ao mesmo tempo fazendo perguntas retóricas ao
leitor para buscar uma interação entre ambas as vivências, do narrador e
do leitor.

\item
SAEB: Inferir informações implícitas em distintos textos.

(A) Correta. O leitor precisa, antes, compreender que o conceito de
cidadania é amplo e abrange diferentes direitos, dentre os quais o
direito ao voto. Assim, ao citar o exemplo do Brasil, fica claro que
esse direito de cidadania foi conquistado tardiamente no país. 

(B) Incorreta. Pelo contrário, o texto enfatiza que o conceito de cidadania
é amplo e abrange os diferentes direitos a que se pode ter acesso. Os
direitos específicos listados no texto são apenas exemplificações dos
tantos outros abrangidos. 

(C) Incorreta. O texto diz que a cidadania foi
uma conquista árdua da humanidade, porém não é possível deduzir daí que
ela seja garantida em todos os países do mundo, pois um conhecimento
prévio básico nos diz que há países autoritários. 

(D) Incorreta. O texto
nega essa afirmativa ao dizer que o conceito correto de cidadania ainda
é desconhecido, embora o termo seja muito usado no cotidiano.
\end{enumerate}

\section*{Língua Portuguesa – Módulo 6 – Treino}

\begin{enumerate}
\item

SAEB: Inferir, em textos multissemiótico, efeitos de humor, ironia e/ou
crítica. 

BNCC: EF69LP05 -- Inferir e justificar, em textos
multissemióticos --- tirinhas, charges, memes, gifs etc. ----, o efeito
de humor, ironia e/ou crítica pelo uso ambíguo de palavras, expressões
ou imagens ambíguas, de clichês, de recursos iconográficos, de pontuação
etc.

(A) Incorreta. O texto não apresenta contradição como forma de gerar
humor. 

(B) Correta. A situação humorística é criada pela brincadeira com
a expressão ``reunião de pais'', com a palavra ``pais'' ora significando
progenitor masculino, ora significando ambos os progenitores, pai e mãe.

(C) Incorreta. A situação narrada não é absurda, mas corriqueira, pois a
reunião de pais faz parte do cotidiano de famílias com filhos em idade
escolar. 

(D) Incorreta. Não há quebra de expectativa no texto.

\item
SAEB: Inferir, em textos multissemiótico, efeitos de humor, ironia e/ou
crítica.

BNCC: EF69LP05 -- Inferir e justificar, em textos
multissemióticos --- tirinhas, charges, memes, gifs etc. ----, o efeito
de humor, ironia e/ou crítica pelo uso ambíguo de palavras, expressões
ou imagens ambíguas, de clichês, de recursos iconográficos, de pontuação
etc.

(A) Correta. O texto ironiza a postura de procrastinação da classe
política, que ano após ano deixa de atuar na resolução dos problemas
causados pelo grande volume de chuvas que já é conhecido e esperado todo
ano. 

(B) Incorreta. A denúncia não se faz presente no texto, pois se
trata de um texto humorístico que se utiliza de outras estratégias
argumentativas mais aptas do que a denúncia para gerar o humor, tais
como a ironia. 

(C) Incorreta. Por ser um texto voltado para o humor, a
objetividade não está presente nele, já que ela pode prejudicar a
expressividade necessária aos efeitos de humor. 

(D) Incorreta. Por ser
um texto voltado para o humor, pressupõe-se o posicionamento crítico
frente ao fato, isto é, a parcialidade é que se faz presente.

\item

Saeb: Inferir, em textos multissemióticos, efeitos de humor, ironia e/ou
crítica. 

BNCC: EF69LP05 -- Inferir e justificar, em textos
multissemióticos --- tirinhas, charges, memes, gifs etc. ---, o efeito
de humor, ironia e/ou crítica pelo uso ambíguo de palavras, expressões
ou imagens ambíguas, de clichês, de recursos iconográficos, de pontuação
etc.


(A) Incorreta. O autor não apresenta passividade frente à clara
inquietude juvenil, pois seu texto apresenta questionamentos que indicam
a permanência da postura inquieta frente ao tema em questão. 

(B)
Incorreta. O autor chama o leitor para um diálogo voltado para a vida
prática, de modo a ter alguém em quem se espelhar, tal como o autor
fazia na juventude, sem prezar pelo perfil técnico dessa interação. 

(C) Correta. O autor relata que buscava exemplos
práticos para se espelhar e fugia de diálogos com viés acadêmico. 

(D)
Incorreta. Ao referir-se à abordagem acadêmica, o autor está, na
realidade, refutando-a, pois prefere exemplos práticos que sirvam de
espelho.
\end{enumerate}

\section*{Língua Portuguesa – Módulo 7 – Treino}

\begin{enumerate}
	\item
SAEB: Analisar marcas de parcialidade em textos jornalísticos.

(A) Incorreta. Não se trata de linguagem neutra, pois a escolha entre os
diferentes termos citados não se refere à abordagem de gênero. 

(B)
Incorreta. O texto não aborda o tema sob o ponto de vista da compaixão.

(C) Correta. O texto traz o tema sob um ponto de vista do problema
social que assola o país. A expressão traz um sentido implícito de
condição temporária que se opõe a ``morador de rua'' ou ``mendigo'',
formas linguísticas categóricas que expressam o sentido de condição
permanente. 

(D) Incorreta. O texto não aborda a solidariedade nem
campanhas de doação.

\item

SAEB: Avaliar diferentes graus de parcialidade em textos jornalísticos.

(A) Incorreta. O texto I foi publicado após a assembleia, não podendo
ser uma divulgação dela. O texto II critica a decisão tomada na
assembleia, e não o evento propriamente dito. 

(B) Correta. O texto I,
sendo uma notícia, informa os leitores de que a greve foi mantida. O
texto II, sendo um texto opinativo, avalia negativamente a decisão de
manter a greve. 

(C) Incorreta. O texto I apenas noticia um fato, e o
texto II de fato faz uma reclamação sobre a greve, por ser opinativo.

(D) Incorreta. O texto I apenas noticia um fato, e o texto II de fato se
solidariza com as pessoas que dependem do transporte público, avaliando
negativamente a greve.

\item

SAEB: Avaliar a fidedignidade de informações sobre um mesmo fato
divulgado em diferentes veículos e mídias.

(A) Incorreta. Os dois textos mencionam o ano de 2022 como o período
principal de referência das informações. 

(B) Incorreta. O fato certo
presente em ambos os textos, independentemente do viés, é a ocorrência
do desmatamento. 

(C) Correta. Os dois textos apresentam dados distintos
sobre o desmatamento, e o leitor não consegue, à primeira vista,
identificar qual é mais real, pois provavelmente os critérios por trás
do levantamento desses dados são completamente diferentes entre os
veículos de informação e voltados para o interesse do ponto de vista que
se quer transmitir no texto. 

(D) Incorreta. Ambos os textos expressam
posicionamentos políticos claros. O texto I opõe-se ao governo,
apresentando a notícia de modo desfavorável a ele, enquanto o texto II é
favorável ao governo, buscando um viés que favoreça a imagem do
presidente.
\end{enumerate}

\section*{Língua Portuguesa – Módulo 8 – Treino}

\begin{enumerate}
	\item
SAEB: Analisar os efeitos de sentido dos tempos, modos e/ou vozes
verbais com base no gênero textual e na intenção comunicativa. 

BNCC: EF08LP16 -- Explicar os efeitos de sentido do uso, em textos, de
estratégias de modalização e argumentatividade (sinais de pontuação,
adjetivos, substantivos, expressões de grau, verbos e perífrases
verbais, advérbios etc.).

(A) Correta. A notícia privilegia o relato de fatos relevantes do
cotidiano de uma sociedade e, para manter o interesse do leitor ou
ganhar sua atenção, precisa sempre parecer nova e recente, já que os
fatos novos sempre se sobrepõem aos antigos. Para isso, recorre ao
emprego do tempo verbal presente do indicativo. 

(B) Incorreta. A
veracidade dos fatos não depende exclusivamente do tempo verbal
presente, pois há trechos de notícia narrados com o tempo verbal
pretérito que não deixam de ser verdadeiros. 

(C) Incorreta. A
imparcialidade se constrói por meio de outras estratégias, como o uso de
terceira pessoa. 

(D) Incorreta. O gênero notícia busca, geralmente,
abordar os fatos com objetividade, embora possa apresentar traço de
opinião. Entretanto, o tempo verbal presente não é um dos elementos
linguísticos usados para marcar opinião.

\item
SAEB: Identificar os recursos de modalização em textos diversos. 

BNCC: EF08LP16 -- Explicar os efeitos de sentido do uso, em textos, de
estratégias de modalização e argumentatividade (sinais de pontuação,
adjetivos, substantivos, expressões de grau, verbos e perífrases
verbais, advérbios etc.).

(A) Incorreta. A veracidade está presente no texto, mas não se trata de
uma característica influenciada pelas expressões em questão. 

(B)
Correta. A expressão ``expansão urbana'' é atribuída à ONG MapBiomas e
revela uma postura menos avaliativa e mais neutra sobre o crescimento da
cidade de Manaus. Por outro lado, referindo-se ao mesmo fato, a
expressão ``crescimento desordenado'', presente em trechos que
correspondem à voz do veículo de notícias, revela uma apreciação sobre o
crescimento da cidade de Manaus. 

(C) Incorreta. Embora a objetividade
seja uma característica desejável no texto em questão, a expressão
``crescimento desordenado'' revela certo grau de subjetividade por
trazer em si uma postura avaliativa do veículo de notícias em relação ao
fato. 

(D) Incorreta. Embora a imparcialidade seja uma característica
desejável no texto em questão, apenas uma das expressões, o termo
``expansão urbana'', traz um efeito semântico mais imparcial.

\item
SAEB: Analisar os efeitos de sentido produzidos pelo uso de
modalizadores em textos diversos. BNCC: EF08LP16 -- Explicar os efeitos
de sentido do uso, em textos, de estratégias de modalização e
argumentatividade (sinais de pontuação, adjetivos, substantivos,
expressões de grau, verbos e perífrases verbais, advérbios etc.).

(A) Incorreta. A linguagem do texto é comprometida com o jornalismo e
com a postura do veículo de notícias frente ao fato. 

(B) Incorreta. A
linguagem empregada é sempre em terceira pessoa para marcar a
impessoalidade, característicadesejável no texto em questão. 

(C)
Incorreta. A citação do estudo feito pela ONG tem como objetivo aumentar
a credibilidade do texto, e não emitir apreciação. 

(D) Correta. O
veículo de notícia refere-se ao fato noticiado empregando a expressão
``crescimento desordenado'', citando ainda os respectivos prejuízos para
Manaus. O título demonstra esse posicionamento ao empregar o verbo
``sofre''. Por outro lado, a ONG citada emprega um termo mais neutro,
``expansão urbana'', para designar o mesmo fato.
\end{enumerate}

\section*{Língua Portuguesa – Módulo 9 – Treino}

\begin{enumerate}
\item

SAEB: Avaliar a eficácia das estratégias argumentativas em textos de
diferentes gêneros. 

BNCC: EF67LP38 -- Analisar os efeitos de sentido do
uso de figuras de linguagem, como comparação, metáfora, metonímia,
personificação, hipérbole, dentre outras.

(A) Incorreta. O menino acompanhou o velório da mãe, que foi realizado
na casa da família; portanto ele sabia da morte dela. 

(B) Correta. A
expressão se caracteriza como eufemismo, o que busca amenizar a dor do
menino, embora ele não tivesse a real dimensão do fato. Sendo ele uma
criança, a tia preocupou-se em confortá-lo pela perda da mãe. 

(C)
Incorreta. Por se tratar de uma criança, de modo algum a tia, que lhe
tinha afeto, teria a intenção de aumentar no menino a dor da perda da
mãe. 

(D) Incorreta. A expressão não explica o sentido real, mas dá ao
menino uma percepção que ele conseguiria apreender.

\item
SAEB: Analisar o uso de figuras de linguagem como estratégia
argumentativa.

BNCC: EF89LP37 -- Analisar os efeitos de sentido do uso
de figuras de linguagem como ironia, eufemismo, antítese, aliteração,
assonância, dentre outras.

(A) Correta. O emparelhamento de ideias opostas é feito pela figura de
linguagem chamada antítese, que no texto se identifica pelos versos em
que se acham as seguintes ideias: feliz/infeliz, sol/chuva,
felicidade/infelicidade, montanhas/planícies, rochedos/erva. 

(B)
Incorreta. A alternativa se refere ao eufemismo, que não é a principal
figura de linguagem no poema, pois não há recorrência dela. 

(C)
Incorreta. O exagero é obtido por meio da hipérbole, que não se faz
presente no poema. 

(D) Incorreta. Trata-se da ironia, figura de
linguagem que significa o contrário do que se diz, característica que
não se encontra no poema.

\item

SAEB: Avaliar a eficácia das estratégias argumentativas em textos de
diferentes gêneros.

BNCC: EF89LP14 -- Analisar, em textos argumentativos e propositivos, os
movimentos argumentativos de sustentação, refutação e negociação e os
tipos de argumentos, avaliando a força/tipo dos argumentos utilizados.

(A) Incorreta. O texto não menciona a participação de representante da
OMS no programa. 

(B) Incorreta. O argumento de autoridade empregado ao
citar a OMS pode relacionar-se com a necessidade do combate à obesidade
infantil, mas seu objetivo no texto é justificar a escolha do tema. 

(C)
Incorreta. O texto não visa influenciar a opinião ou formar a opinião do
leitor, mas influenciar sua decisão de acompanhar o programa divulgado.

(D) Correta. A menção à OMS, tendo em vista o objetivo do texto de
convidar o espectador para acompanhar o programa, visa convencer o
público de que o tema é relevante. O texto chega a dizer que a escolha
do tema não foi por acaso, justificando-a com dados estatísticos da OMS
sobre a doença.
\end{enumerate}

\section*{Língua Portuguesa – Módulo 10 – Treino}

\begin{enumerate}
\item
SAEB: Analisar os processos de referenciação lexical e pronominal. 

BNCC: EF08LP15 -- Estabelecer relações entre partes do texto, identificando o
antecedente de um pronome relativo ou o referente comum de uma cadeia de
substituições lexicais.

(A) Incorreta. O termo ``nenê'' se refere à boneca.

(B) Incorreta. O termo ``ligeira'' descreve a forma como a menina se
antecipou ao castigo da mãe naquela situação específica.

(C) Incorreta. O termo ``boneca'' se refere ao brinquedo propriamente
dito.

(D) Correta. O termo ``pequenita'' retoma o nome Georgeana e, ao mesmo
tempo, qualifica a menina, indicando que era uma criança.

\item

SAEB: Analisar os mecanismos que contribuem para a progressão textual.

BNCC: EF08LP15 -- Estabelecer relações entre partes do texto,
identificando o antecedente de um pronome relativo ou o referente comum
de uma cadeia de substituições lexicais.

(A) Incorreta. Os termos não apresentam sentidos opostos no texto, e sim
se complementam semanticamente.

(B) Incorreta. Os termos se relacionam semanticamente, mas um não é
conclusão do outro.

(C) Incorreta. Os termos se relacionam semanticamente, mas um não é a
explicação do outro.

(D) Correta. Os termos estabelecem entre si uma relação de coesão e
progressão textual, na medida em que o segundo e o terceiro retomam a
ideia do primeiro, porém especificando-o, numa relação semântica de todo
(recursos naturais) e parte (folhas e frutos).

\item

SAEB: Analisar os mecanismos que contribuem para a progressão textual.

BNCC: EF08LP14 -- Utilizar, ao produzir texto, recursos de coesão
sequencial (articuladores) e referencial (léxica e pronominal),
construções passivas e impessoais, discurso direto e indireto e outros
recursos expressivos adequados ao gênero textual.

(A) Incorreta. Brasil e mundo globalizado não são localidades opostas,
pois ambos são impactados pos
(B) Incorreta. Redes sociais e mídias
virtuais são citados como exemplos de tecnologias que promoveram
benefícios para a humanidade e não se opõem; pelo contrário,
complementam-se ao ampliarem o acesso ao conhecimento. 
(C) Correta. Os
avanços tecnológicos são trabalhados no texto sob a ótica da oposição
entre seus benefícios e seus prejuízos. A ampliação do acesso ao
conhecimento é um benefício e a manipulação comportamental é um
prejuízo, e estão em lados opostos. 
(D) Incorreta. A alternativa
apresenta exemplos não opostos, pois ambos são prejuízos da tecnologia,
segundo o texto.

\section*{Língua Portuguesa – Módulo 11 – Treino}

\item

Saeb: Analisar as variedades linguísticas em textos.

BNCC: EF69LP55

(A) Correta. As duas expressões promovem uma interação informal entre os
participantes da entrevista, pois estão presentes em ambas as falas, do
entrevistado e do entrevistador, de modo a aumentar o engajamento de
ambos no diálogo.

(B) Incorreta. Os termos não exemplificam nenhum tipo de variação
linguística.

(C) Incorreta. Os termos não exemplificam nenhum tipo de variação
linguística.

(D) Incorreta. Os termos não exemplificam nenhum tipo de variação
linguística.

\item

Saeb: Avaliar a adequação das variedades linguísticas em contextos de uso.

BNCC: EF69LP55

(A) Correta. O texto tem a finalidade de interagir com o leitor, o que
se percebe pelas perguntas pessoais que faz para buscar aproximar-se
dele. Para isso, emprega linguagem informal, que se percebe na colocação
pronominal diferente do que prescreve a norma-padrão -- ``me sinto'' --
e no emprego de expressões do dia a dia, como ``bater um papo''.

(B) Incorreta. A linguagem do texto pode ser considerada como um uso
culto da língua, pois o autor é alguém que reside em centro urbano e
teve acesso à escolaridade, o que se percebe por sua boa articulação em
língua escrita. Entretanto, o erro está em dizer que o autor visa exibir
seu domínio da língua para o leitor.

(C) Incorreta. O texto não visa a um público intelectual, e sim aos
diferentes tipos de leitores que podem ter acesso ao texto. Isso fica
claro na crítica feita ao viés acadêmico com que se aborda o tema. A
linguagem empregada no texto busca atingir o maior número de tipos de
leitores possível, daí seu caráter mais acessível.

(D) Incorreta. O texto visa a um público leitor diversificado. Além
disso, a linguagem coloquial não pressupõe que o leitor deva ser menos
escolarizado.

\item

Saeb: Analisar as variedades linguísticas em textos.

BNCC: EF69LP55

(A) Incorreta. O desvio ortográfico presente no texto não é proveniente
da idade da autora, pois não é marca de distinção de uma geração
específica.

(B) Correta. O desvio ortográfico presente no texto provém do nível de
escolaridade da autora, cuja escrita se mostra influenciada pela
oralidade diante de certas relações entre som e letra que são aprendidas
somente com ensino formal. O contexto também permite essa constatação,
pois a autora é moradora da periferia, onde normalmente está a maior
parte da população pouco escolarizada.

(C) Incorreta. O desvio de escrita em questão não é distintivo de
regionalidade.

(D) Incorreta. No texto, predomina a colocação pronominal enclítica
(atender-me/disse-lhe/maltrata-os), uma marca característica da
linguagem formal. Além disso, os desvios ortográficos em questão são
provenientes do baixo domínio das relações representativas entre som e
letra.
\end{enumerate}

\section*{Língua Portuguesa – Simulado 1}

\begin{enumerate}

\item

Saeb: Distinguir fatos de opiniões em textos.

(A) Correta. Trata-se de crítica às pessoas que usam com frequência a palavra ``cidadania'' sem realmente saber seu significado.

(B) Incorreta. Não há marcas linguísticas de conselho no trecho.

(C) Incorreta. O que a autor faz no trecho não é deduzir, pois isso
envolve chegar a uma conclusão com base em dados ou fatos. Ele pode até
ter se baseado em uma vivência pessoal, mas o texto não traz esse
conteúdo como condição para se compreender o trecho.

(D) Incorreta. O autor não supõe (sugere) e sim afirma (assevera) que o
uso da palavra ``cidadania'' é frequente, mas que seu significado é pouco
conhecido.

\item

Saeb: Identificar o uso de recursos persuasivos em textos verbais e não
verbais.

(A) Incorreta. O texto não verbal, isto é, as imagens de cachorros e
demais imagens, não tem relação com a extensão das frases no cartaz. A
presença de frases curtas obedece a outros critérios, como espaço
disponível.

(B) Incorreta. O texto não verbal, isto é, as imagens de cachorros e
demais imagens, não tem relação com a estrutura sintática das frases no
cartaz. A frase nominal é aquela que não tem verbo em sua estrutura
sintática. A presença desse tipo de frase obedece a outros critérios,
como estilo do autor.

(C) Incorreta. O texto não verbal tem uma relação meramente ilustrativa
com os nomes de animais (gato e cachorro) que aparecem no cartaz.

(D) Correta. O texto não verbal mostra cachorros e gatos vestidos com
determinado tipo de traje e adornos típicos de um grupo social que usa
as gírias de interação ``bicho'' e ``se liga'', presentes no texto
verbal.

\item

Saeb: Identificar os recursos de modalização em textos diversos.

BNCC: EF08LP16.

(A) Incorreta. narrador apenas diz que não é o autor da história que
vai contar - os autores, como se pode presumir, seriam pescadores e caçadores.

(B) Correta. O narrador, com essa ressalva, se exime de
garantir a
veracidade da história e deixa para o ouvinte/leitor a responsabilidade
de acreditar ou não, conforme queira.

(C) Incorreta. O narrador, nesse trecho, já está contando a história
para o ouvinte/leitor. Não se trata, portanto, de um comentário seu
sobre a veracidade dela.

(D) Incorreta. O narrador apenas situa a história no espaço para
contextualizar o ouvinte/leitor.

\item

Saeb: Analisar as variedades linguísticas em textos.

BNCC: EF69LP56.

(A) Incorreta. O sabor do biscoito não é discutido entre os amigos no
texto.

(B) Incorreta. A preferência dos amigos entre um ou outro alimento não é
discutida no texto.

(C) Correta. A região é fator determinante do registro linguístico de
cada um dos personagens, condicionando o emprego de ``bolacha'' ou
``biscoito'' para nomear o mesmo alimento.

(D) Incorreta. "Biscoito" e "bolacha" são variações do nome dado ao mesmo
alimento com base no critério de região, não de sabor.

\item

Saeb: Inferir, em textos multissemióticos, efeitos de humor, ironia e/ou
crítica.

BNCC: EF69LP05.

(A) Incorreta. A indiferença de um dos amigos em relação a qual lanche
vai comer (se bolacha ou biscoito) não é o trecho do texto que contém ou produz humor.

(B) Incorreta. O ambiente espacial do diálogo é pouco relevante para a
construção dos sentidos do texto.

(C) Incorreta. O diálogo é entre dois amigos, situação em que a
linguagem é usada de forma pouco monitorada e mais informal, geralmente
com mais desvios em relação à norma-padrão.

(D) Correta. Um dos amigos claramente zomba do outro por usar uma
palavra diferente para nomear o alimento. Isso fica claro quando ele
finge não saber o que é ``bolacha'', já que, na variedade linguística usada por ele, o termo mais aceito é ``biscoito''.

\item

Saeb: Avaliar a eficácia das estratégias argumentativas em textos de
diferentes gêneros.

BNCC: EF89LP14.

(A) Incorreta. O tema é o mesmo nos textos, pois ambos abordam a importância da vacinação.

(B) Correta. O público-alvo de cada texto condiciona a construção de sua
linguagem, tanto verbal quanto não verbal. No primeiro texto, o público é a população idosa e o cartaz é mais neutro, e, no segundo, a população é
infantil e o texto é mais lúdico.

(C) Incorreta. O meio de circulação de ambos os textos pode ser o mesmo - 
por exemplo, o quadro de avisos de um posto de saúde, de uma repartição
pública, de um estabelecimento, ou um site governamental, em razão de
ser um tema de interesse público.

(D) Incorreta. O objetivo comunicativo é o mesmo nos dois textos, pois
ambos visam a divulgar uma campanha de vacinação.

\item

Saeb: Avaliar a adequação das variedades linguísticas em contextos de
uso.

BNCC: EF69LP55.

(A) Incorreta. No ambiente acadêmico (seminário universitário) e para o
público-alvo (professores e estudantes) em questão, não é adequado ao
palestrante usar uma linguagem coloquial, pois não é o que se espera
nessa situação comunicativa.

(B) Incorreta. O texto não contém termos técnicos ou jargão
profissional. A vocabulário é compreensível para qualquer ouvinte.

(C) Correta. Sendo o meio de circulação do texto um seminário
universitário, espera-se que a linguagem utilizada pelo palestrante seja
mais próxima da norma-padrão, por se tratar de um ambiente acadêmico em
que o público-alvo são estudantes e professores, indicando alto nível de
formalidade.

(D) Incorreta. O texto não contém marcas de variação linguística
regional.

\item

Saeb: Avaliar a adequação das variedades linguísticas em contextos de
uso.

BNCC: EF69LP55.

(A) Correta. O termo remete ao seminário, isto é, o contexto
comunicativo imediato em que o falante está inserido no momento em que
profere sua fala.

(B) Incorreta. O termo não retoma elemento anterior a ele citado no
texto, não é anafórico.

(C) Incorreta. O termo não retoma elemento ainda a ser citado no texto.

(D) Incorreta. O termo não remete a algo que pertença ao sujeito enunciador da fala.

\item


Saeb: Inferir a presença de valores sociais, culturais e humanos em
textos literários.

BNCC: EF69LP44.

(A) Incorreta. O eu lírico demonstra um sentimento cosmopolita, voltado
para o mundo exterior e não para suas raízes ou para seu sentimento de
pertencimento, representados pela terra natal.

(B) Correta. O eu lírico, em tom afetivo, diz guardar no seu coração
vários lugares onde já esteve e paisagens que já viu mundo afora.
"Coração", no primeiro verso, e "cofre", no segundo, são metáforas de sua
memória. Portanto, o tema central do poema são as lembranças das viagens
do eu lírico pelo mundo.

(C) Incorreta. O eu lírico é alguém que já viajou pelo mundo e que criou
muitas memórias de suas viagens; isto é, ele não expressa uma vontade de
viajar pelo mundo, pois demonstra que já o conhece bastante.

(D) Incorreta. O eu lírico cita paisagens naturais que viu pelo mundo e,
de fato, as admira, porém ele se refere a elas por representarem
características particulares dos locais, não necessariamente por gostar
especifica e particularmente da natureza.

\item


Saeb: Identificar o uso de recursos persuasivos em textos verbais e não
verbais.

BNCC: EF89LP04.

(A) Incorreta. A forma de contágio da doença (viral) está colocada num
texto escrito em fonte pequena e com menor destaque.

(B) Incorreta. O texto é voltado para o público em geral (você, qualquer
pessoa que esteja lendo), não para um grupo específico.

(C) Correta. O texto sugere que os sintomas são silenciosos, pois o
portador pode não saber que tem a doença; daí o objetivo da campanha:
convencer as pessoas a procurar o diagnóstico antes que seja tarde.

(D) Incorreta. O texto não cita fatores de risco, diz apenas que o
contágio é viral. Um fator de risco, por exemplo, para uma doença viral,
é frequentar locais fechados com grande público.

\item

Saeb: Analisar a relação temática entre diferentes gêneros
jornalísticos.

BNCC: EF69LP02.

(A) Incorreta. Zé Gotinha é um personagem muito conhecido do público
infantil, de modo que dispensa apresentações. O texto, inclusive,
pressupõe essa popularidade do personagem.

(B) Incorreta. O texto não usa o verbo ``seguir'' no sentido de
acompanhar redes sociais de alguma personalidade, mas no sentido de
tomar o personagem como exemplo na proteção da saúde das crianças.

(C) Correta. O texto I aborda a baixa cobertura vacinal entre o público
infantil, e um dos motivos, não citados no texto, pode ser uma
divulgação deficitária e a falta de informação, o que pode ser combatido
com campanhas de divulgação e atração da população, como é feito no
texto II, um cartaz de divulgação.

(D) Incorreta. O texto II tem sua construção pautada no seu
público-alvo, as crianças, sobretudo por meio de ilustrações. Porém,
somente entretê-las não se relaciona com a notícia do texto I; a baixa
cobertura vacinal citada no texto I se relaciona com a necessidade de
atrair a população para vacinar suas crianças.

\item

Saeb: Analisar a relação temática entre diferentes gêneros
jornalísticos.

BNCC: EF69LP02.

(A) Incorreta. Zé Gotinha é um personagem muito conhecido do público
infantil, de modo que dispensa apresentações. O texto, inclusive,
pressupõe essa popularidade do personagem.

(B) Incorreta. O texto não usa o verbo ``seguir'' no sentido de
acompanhar redes sociais de alguma personalidade, e sim no sentido de
tomar o personagem como exemplo na proteção da saúde das crianças.

(C) Correta. O texto I aborda a baixa cobertura vacinal entre o público
infantil, e um dos motivos, não citados no texto, pode ser uma
divulgação deficitária e a falta de informação, o que pode ser combatido
com campanhas de divulgação e atração da população, como é feito no
texto II, um cartaz de divulgação.

(D) Incorreta. O texto II tem sua construção pautada no seu
público-alvo, as crianças, sobretudo por meio de ilustrações. Porém,
somente entretê-las não se relaciona com a notícia do texto I; a baixa
cobertura vacinal citada no texto I se relaciona com a necessidade de
atrair a população para vacinar suas crianças.

\item

Saeb: Avaliar diferentes graus de parcialidade em textos jornalísticos.

BNCC: EF07LP02.

(A) Correta. O texto I, uma notícia, visa apenas a divulgar a informação
nova que veicula, isto é, a decisão da CBF de punir atitudes racistas. O
texto II, um artigo de opinião, faz uma avaliação positiva dessa
decisão.

(B) Incorreta. O texto I não faz juízo de valor, apenas informa o fato.
O texto II não apenas informa o fato, ele o avalia e pressupõe que o
leitor já tem conhecimento da decisão da CBF.

(C) Incorreta. O texto I, uma notícia, visa apenas a divulgar a informação
nova que veicula. O texto II traz um ponto de vista, porém sua intenção
não é fazer denúncia.

(D) Incorreta. O texto I, uma notícia, visa apenas a divulgar a informação
nova que veicula. O texto II não visa a conscientizar, embora o tema seja
oportuno para tal empreitada.

\item

Saeb: Analisar os efeitos de sentido dos tempos, modos e/ou vozes
verbais com base no gênero textual e na intenção comunicativa.


(A) Incorreta. Embora uma informação quantitativa do volume de chuvas
esteja acompanhada de dois dos verbos ou locuções verbais citados ("alcançar" e "podem ter"),
ela poderia ser dada com uso de tantas outras formas da língua, sem
depender dos tempos e modos verbais.

(B) Incorreta. O texto é uma notícia e, como tal, visa apenas a transmitir
uma informação e não interage com o leitor - embora, em segundo plano, naturalmente, sirva como um alerta. Os modos e tempos verbais em
questão não proporcionam interação com o leitor.

(C) Incorreta. Os verbos não conferem nenhuma ênfase ou destaque nem têm
uma relação direta com a ideia de alertar a população.

(D) Correta. Os tempos e modos verbais dos verbos são o indicativo (deve
ser/podem ter -- verbos que indicam possibilidade, embora no indicativo)
e o subjuntivo (alcance), que expressa naturalmente a ideia de
hipótese/possibilidade. Por se tratar de um texto que fala de previsão
do tempo, sua construção linguística colabora para expressar hipóteses, 
não certezas.


\item

Saeb: Analisar os efeitos de sentido produzidos pelo uso de
modalizadores em textos diversos.

BNCC: EF89LP31.

(A) Correta. A personagem usa as expressões ``acho que não'' e
``talvez'' por não estar certa da resposta. Isso se comprova pela
reformulação da expressão ``acho que não'', que se torna ``talvez'', o
que demonstra que, mesmo em dúvida, ela quer cooperar com a continuidade
do diálogo.

(B) Incorreta. A personagem demonstra estar cooperando com a pessoa com quem interage no diálogo, de modo que não há o sentido de desinteresse
em sua última fala.

(C) Incorreta. A personagem coopera com o diálogo e responde às duas
perguntas espontaneamente, sem demonstrar que se sente constrangida.

(D) Incorreta. A personagem não tem certeza de sua resposta. O sentido
de desconhecimento não é coerente porque a personagem não responde nem
que sim, nem que não, ou seja, ela tem dúvida.


\item

Saeb: Analisar o uso de figuras de linguagem como estratégia
argumentativa.

BNCC: EF89LP37.

(A) Incorreta. A locutora não chega a emitir opinião sobre a melodia,
embora ela implicitamente associe o ato de ouvir a melodia ao ato de
ouvir a história de Olívia. Porém, a figura de linguagem não expressa
opinião sobre a melodia.

(B) Incorreta. O interlocutor de Olívia não recomenda que ela ouça a cação, mas sim pergunta se ela já conhece a melodia.

(C) Correta. Após se desculpar pelo esquecimento, a locutora pede que
Olívia repita sua história outras vezes. A figura de linguagem visa a
demonstrar para Olívia que esse esforço vai ser recompensado por uma
escuta atenta, tal como se compreende a frase em questão.

(D) Incorreta. A locutora já havia se desculpado antes, agora ela quer
convencer Olívia a recontar sua história, recebendo em troca uma escuta
atenta.
\end{enumerate}

\section*{Língua Portuguesa – Simulado 2}

\begin{enumerate}

\item

Saeb: Avaliar a adequação das variedades linguísticas em contextos de
uso.

BNCC: EF69LP55.

(A) Correta. Os desvios ortográficos presentes na escrita da autora em
nada afetam o sentido do texto, por isso sua escrita é um exemplo de
variação linguística possível dentro da língua e demonstra que, embora
desviante da norma-padrão, a autora se utilizou de outros conhecimentos
empíricos e outras experiências para enriquecer o conteúdo de seu texto.

(B) Incorreta. A alternativa reproduz um preconceito linguístico que
limita a língua à forma padrão prescrita pela gramática normativa,
desconsiderando que a construção de sentidos está para além dela.

(C) Incorreta. Os desvios ortográficos não configuram mero descuido, e
sim uma limitação da autora do ponto de vista da norma-padrão, o que, no
entanto, não diminui a qualidade literária de seu texto.

(D) Incorreta. A alternativa reproduz a noção de erro, geradora de
preconceito linguístico, segundo a qual a norma-padrão é suficiente para
dar conta de todos os aspectos envolvidos na produção textual (como a
coerência), o que é incorreto, pois trata-se de uma habilidade de
articulação que mobiliza outros conhecimentos para além das regras da
norma-padrão.

\item

Saeb: Identificar o uso de recursos persuasivos em textos verbais e não
verbais.

BNCC: EF89LP04.

(A) Incorreta. Não há menção ao preço do produto no texto.

(B) Incorreta. Embora a marca de fato apareça com destaque, esse não é o
ponto enfatizado no texto, nem mesmo se diz que a marca é de renome.

(C) Correta. A expressão ``pronto socorro'' enfatiza que o tratamento
proporcionado pelo produto é imediato (pronto = imediato), daí o
destaque à sua eficiência como recurso persuasivo.

(D) Incorreta. O texto não menciona os locais ou estabelecimentos onde
se pode comprar o produto.


\item

Saeb: Identificar teses, opiniões, posicionamentos explícitos e
argumentos em textos.

BNCC: EF89LP04.

(A) Incorreta. Pelo contrário, o cerne do texto é a influência da
tecnologia na sociedade.

(B) Incorreta. O personagem Neo é citado por ser o protagonista do
filme, mas seu perfil não é analisado. A citação a ele é curta e poderia
ser dispensada, sem prejuízo para a compreensão do enredo do filme.

(C) Incorreta. O texto não tem o objetivo de fazer avaliação do filme
para indicá-lo ao leitor, pois se trata de um artigo de opinião sobre
assunto diverso. A menção ao filme é apenas uma estratégia de introdução
do assunto, de modo a ativar um conhecimento prévio do leitor.

(D) Correta. O filme mencionado tem um enredo que, para o autor, se
assemelha à nossa realidade atual, no que diz respeito à influência da
tecnologia na sociedade. Essa estratégia de argumentação é de
comparação. No caso, compara-se a ficção do filme com a realidade da
sociedade tecnológica atual.

\item

Saeb: Analisar os efeitos de sentido produzidos pelo uso de
modalizadores em textos diversos.

BNCC: EF08LP16.

(A) Incorreta. O desejo de proibição da atitude em questão não torna o
narrador uma pessoa autoritária, pois, de acordo com o contexto, ele é
vítima de uma brincadeira de mau gosto.

(B) Correta. O narrador diz ter cometido um erro gramatical ao falar uma
língua que não domina. O problema é que seu erro não só foi
ridicularizado, como isso foi feito por sua professora da língua em
questão, de quem ele esperava um tratamento diferente, como apontar o
erro em vez de fazê-lo repetir e ficar ainda mais constrangido, daí sua
indignação.

(C) Incorreta. O narrador não demonstra ter se arrependido de ter se
arriscado na língua em que ainda era aprendiz. Pelo contrário, ele
sugere que seria uma oportunidade de aprender com o erro, caso tivesse
sido apontado pela professora.

(D) Incorreta. O narrador se mostra decepcionado não com o erro, mas com
a postura da professora ao caçoar dele duas vezes por ter errado.

\item

Saeb: Analisar os processos de referenciação lexical e pronominal.

BNCC: EF08LP15.

(A) Incorreta. O pronome ``dela'' não se refere a ``rua''.

(B) Incorreta. O pronome ``dela'' não se refere a ``estação''.

(C) Correta. O pronome ``dela'' tem no texto função catafórica, isto é,
ele aponta para um termo posterior na cadeia textual. Portanto, em
``estação azul igual à dela'' e ``nome semelhante à estação da casa
dela'', ``dela'' se refere a ``professora''.

(D) Incorreta. O pronome ``dela'' não se refere a ``língua
estrangeira''.

\item

Saeb: Analisar elementos constitutivos de textos pertencentes ao domínio
literário.

BNCC: EF69LP47.

(A) Incorreta. Um texto lúdico, diferentemente do lido, tem uma linguagem leve, divertida, figurada, privilegiando a imaginação. A narrativa em questão, por outro lado, traz um texto bastante literal e sóbrio, sem linguagem figurada.

(B) Incorreta. Apesar de a cena ser, possivelmente, narrada e descrita de forma realista, a construção textual não privilegia esse perfil.

(C) Incorreta. Apesar de narrar uma situação pouco comum de acontecer no
dia a dia, não se trata também de algo fantástico, isto é, sobrenatural (
pelo menos não nesse trecho do texto).

(D) Correta. A narrativa se refere ao personagem de forma genérica, sem
nomeá-lo, e sua descrição, no caso da vestimenta, indica que ele
pretende não ser reconhecido e está sozinho caminhando no escuro de uma
rodovia. Além disso, na descrição do cenário, percebe-se que é noite e
que há neblina, o que dificulta a visibilidade.

\item

Saeb: Analisar elementos constitutivos de textos pertencentes ao domínio
literário.

BNCC: EF69LP47.

(A) Incorreta. A frase inicial apresenta ao leitor uma cena posterior à
situação inicial da narrativa. Isso fica claro no trecho seguinte à
frase inicial: ``pouco antes desse evento''.

(B) Incorreta. O leitor ainda não sabe onde se passa a cena narrada na
frase inicial da narrativa, pois isso não é mencionado no início.

(C) Incorreta. O que auxilia na compreensão das ações narradas é o
trecho introduzido pela expressão ``pouco antes desse evento'', que não
faz parte da frase inicial da narrativa. Além disso, a frase inicial não
auxilia na compreensão das ações seguintes, pois ela é que precisa do
apoio do restante da narrativa para ser entendida.

(D) Correta. O leitor é ``jogado'' num ponto adiantado da história e se
depara com a narração de uma cena posterior à situação inicial, mas
ainda não sabe disso, pois até então aquela parece ser uma cena
qualquer. No decorrer da leitura, essa cena ganha importância
diferenciada na narrativa, porque o leitor descobre que a ação da
personagem, narrada no início, é algo proibido.

\item


Saeb: Identificar teses, opiniões, posicionamentos explícitos e
argumentos em textos.

BNCC: EF67LP05.

(A) Incorreta. O autor não acredita nisso; pelo contrário, aceita o fato
de que a tecnologia tem também seu lado negativo sobre a humanidade, não
trazendo apenas benefícios.

(B) Correta. O autor remete aos avanços científicos do século XXI e cita
benefícios e prejuízos da tecnologia para humanidade, comprovando ainda
com exemplos práticos e colocando, lado a lado, por oposição, o caráter
positivo e o caráter negativo dos avanços tecnológicos.

(C) Incorreta. O autor não faz juízo de valor dos benefícios comparados
aos prejuízos da tecnologia, embora os coloque em lados opostos.

(D) Incorreta. O autor cita o acesso ao conhecimento com um de exemplo
de benefício da tecnologia.

\item

Saeb: Analisar os mecanismos que contribuem para a progressão textual.

BNCC: EF08LP13.

(A) Incorreta. O trecho introduzido pelo articulador não traz novo
assunto, pois, embora sejam adicionadas novas informações, o assunto
principal não se altera.

(B) Incorreta. A reformulação é realizada por outros articuladores, tais
como ``ou seja'' e ``isto é''. Além disso, não há retificação de
informação no trecho introduzido pelo articulador em questão.

(C) Incorreta. Pelo contrário, o articulador introduz uma ideia oposta,
e não uma concordância. Além disso, ele não expressa sentido conclusivo.

(D) Correta. No trecho anterior, o autor apresenta vantagens da
tecnologia. O articulador introduz desvantagens, isto é, apresenta
ideias opostas às anteriores e que avaliam negativamente a tecnologia.

\item

Saeb: Analisar elementos constitutivos de textos pertencentes ao domínio
literário.

BNCC: EF69LP47.

(A) Incorreta. A pressão psicológica de ser um artista de sucesso
poderia ser a origem do problema que acomete o personagem, mas isso não
se comprova no texto.

(B) Incorreta. O personagem não consegue dormir, mas isso não parece ser
um problema para ele, pois está ciente de que está sob efeito da
adrenalina de uma apresentação musical recente.

(C) Incorreta. O personagem cita o calor do público e o grito das
meninas, o que sugere a boa recepção que teve.

(D) Correta. Tudo está perfeito para o personagem: o show foi um
sucesso, o público foi receptivo e a banda tocou bem. Entretanto, ele se
depara novamente com um problema que o incomoda pela frequência com que
acontece: a dificuldade de memorizar os nomes dos lugares onde se
apresenta.

\item

Saeb: Identificar o uso de recursos persuasivos em textos verbais e não
verbais.

(A) Incorreta. O cartaz não tem caráter religioso.

(B) Incorreta. O cartaz não tem caráter religioso.

(C) Correta. O sangue do possível doador é comparado ao de Cristo, que
para os cristãos é o salvador do mundo. Dessa forma, o sangue do doador
é valorizado como salvador da vida das pessoas que precisam de doação.

(D) Incorreta. O cartaz se destina ao público em geral, sem distinção de
crença, dada a importância dessa ação para quem precisa.

\item

Saeb: Identificar elementos constitutivos de gêneros de divulgação
científica.

(A) Incorreta. O gênero textual verbete tem como característica a
ausência de pontos de vista e, por caráter informativo/expositivo,
privilegia o uso da 3ª pessoa.

(B) Correta. O gênero textual verbete tem como característica a
linguagem denotativa, isto é, literal, pois visa a preservar ao máximo o
sentido dicionarizado das palavras e, consequentemente, evitar mais de
uma possibilidade de interpretação do conteúdo, dado o seu caráter
informativo/expositivo e a objetividade exigida num texto conceitual.

(C) Incorreta. O gênero em questão não apresenta opinião e, por
consequência, está ausente dele a subjetividade.

(D) Incorreta. A alternativa apresenta características de uma narrativa,
e não de um texto informativo/expositivo.


\item

Saeb: Identificar teses, opiniões, posicionamentos explícitos e
argumentos em textos.

BNCC: EF89LP04.

(A) Incorreta. A fala ``eu mereço'' apenas ilustra o pensamento que está
por trás do consumismo.

(B) Correta. O autor critica o consumismo com exemplos de gastos
desnecessários, como trocar de carro todo ano, e a justificativa dada
pelas pessoas de que ``merecem'' essas regalias, como se estivessem
compensando a si mesmas com produtos desnecessários.

(C) Incorreta. O materialismo está implícito no comportamento
consumista, mas não é o foco da crítica do autor. O apego, nesse caso, é
relativo, pois as pessoas compram novos produtos justamente por desapego
em relação aos que já têm ou tiveram.

(D) Incorreta. O conformismo é uma ideia que, de certa forma, está
presente no texto, mas não com destaque. O conformismo é mencionado
apenas no fim do texto como um tipo de sintoma do consumismo.

\item

Saeb: Identificar elementos constitutivos de textos pertencentes ao
domínio jornalístico/midiático.

BNCC: EF08LP01.

(A) Incorreta. O relato pessoal é um gênero narrativo, diferentemente do
texto em questão, que é argumentativo.

(B) Incorreta. A nota de repúdio aborda um fato específico que
repercutiu na sociedade e sobre o qual o autor pretende opinar refutando
e criticando o acontecimento.

(C) Correta. O artigo de opinião visa a avaliar uma situação relevante da
sociedade, apresentando argumentos que fundamentem a tese.

(D) Incorreta. A carta de reclamação tem destinatário certo, o que não
ocorre no texto em questão.

\item

Saeb: Analisar marcas de parcialidade em textos jornalísticos.

(A) Incorreta. A apresentação das informações fundamentais ``o quê'',
``quando'', ``quem'' e ``onde'' é uma característica do gênero notícia,
e essas informações estão presentes nos dois textos noticiosos
transcritos.

(B) Incorreta. O texto II especifica os crimes, enquanto o texto I
apenas se refere a eles de modo genérico, mas se trata de uma diferença
pouco relevante sob o ponto de vista das características do gênero
textual notícia.

(C) Correta. Os textos I e II são do gênero notícia, o que pressupõe
certas características comuns entre eles. Dentre essas características,
a que os diferencia é a maior parcialidade presente no Texto I, ao
considerar que a queda na criminalidade foi tão grande que o carnaval de
Minas Gerais foi um sucesso e o mais seguro do Brasil. Por outro lado, o
Texto I tem marcas de maior imparcialidade, ao caracterizar os mesmos
dados simplesmente como uma queda expressiva se comparados ao evento
anterior.

(D) Incorreta. No gênero textual notícia está pressuposto que as
informações são verdadeiras. Nos textos em questão, a veracidade é
comprovada pela fonte da notícia, que indica veículos com credibilidade
no meio jornalístico.

\end{enumerate}

\section*{Língua Portuguesa – Simulado 3}

\begin{enumerate}

	\item
Saeb: Analisar as variedades linguísticas em textos.

BNCC: EF69LP56.

(A) Incorreta. A variação social ocorre devido a diferenças entre
indivíduos, como idade, profissão, gênero, escolaridade.

(B) Incorreta. A variação regional ocorre devido a diferenças entre
lugares ou regiões, como, por exemplo, a diferença entre o dialeto
mineiro e o carioca.

(C) Correta. A variação histórica ocorre devido a diferenças entre
estágios da língua em diferentes momentos históricos dela. A diferença
entre a ortografia do Texto I e a do Texto II refere-se aos anos de 1911 e
2023.

(D) Incorreta. A variação situacional ocorre em razão da tentativa do
falante em adequar-se ao contexto sociocomunicativo, tal como num
tribunal, em que se exige um padrão de linguagem predominantemente
jurídica.

\item

Saeb: Analisar a intertextualidade entre textos literários ou entre
estes e outros textos verbais ou não verbais.

BNCC: EF69LP44.

(A) Incorreta. Embora a raposa tenha desejado comer as uvas, não é
possível deduzir que estivesse faminta ou fraca pela fala de
alimentação, uma vez que nada é mencionado a respeito.

(B) Incorreta. A raposa demonstra ser paciente e oportunista, ou seja,
não é precipitada, age quando as situações lhe são favoráveis, tanto que,
percebendo a dificuldade de alcançar as uvas, logo desistiu de comê-las.

(C) Correta. O narrador caracteriza as uvas como maduras e apetitosas,
e, estando a parreira carregada delas, provavelmente isso chamou a
atenção da raposa. Porém, a raposa colocou defeitos nas uvas após
perceber que não conseguiria alcançá-las, não admitindo assim sua
limitação.

(D) Incorreta. A raposa, ao contrário do narrador, aponta defeitos nas
uvas os quais provavelmente lhe trariam algum mal se fossem verdadeiros,
porém são inventados por ela para evitar admitir sua limitação em
alcançar as uvas.

\item

Saeb: Analisar efeitos de sentido produzido pelo uso de formas de
apropriação textual (paráfrase, citação etc.).

BNCC: EF89LP05.

(A) Incorreta. A relativização da dor serve ao propósito de levar o
leitor a concluir que a dor de doar sangue, comparada à dor de situações
cotidianas, é suportável e não deve ser o motivo da não doação.

(B) Correta. O verso musical citado exemplifica uma das situações
cotidianas que causam mais dor do que a de doar sangue. O humor é gerado
porque se trata de um verso muito conhecido que fala de amor não
correspondido de forma bastante peculiar.

(C) Incorreta. A referência ao cotidiano estabelece no texto uma escala
de comparação entre tipos de dor, de modo a levar o leitor a concluir
que a dor de doar sangue é suportável frente a outras do cotidiano.

(D) Incorreta. A oposição entre medo e solidariedade, na última frase da
campanha, apenas chama o leitor à ação, após o texto ``provar'' a ideia
de que a dor de doar sangue é menor que muitas outras na vida.

\item

Saeb: Analisar elementos constitutivos de textos pertencentes ao domínio
literário.

BNCC: EF69LP44.

(A) Correta. O texto é um poema e, assim, pertence ao gênero lírico, que
pressupõe subjetividade como característica principal; isto é, o texto
volta-se para o interior do eu.

(B) Incorreta. Sendo o texto um poema e pertencendo ao gênero lírico, a
clareza e objetividade são características dispensáveis, pois a poesia
trabalha justamente com a instabilidade dos sentidos, isto é, a
conotação.

(C) Incorreta. A interpretação dramática sugere a encenação feita por
atores, o que é uma característica do gênero dramático, que se manifesta
em textos teatrais, por exemplo.

(D) Incorreta. Por se tratar de um poema, o texto pertence ao gênero
lírico e não ao gênero narrativo.


\item

Saeb: Analisar os efeitos de sentido dos tempos, modos e/ou vozes
verbais com base no gênero textual e na intenção comunicativa.

(A) Correta. As informações que constam na bula de remédio são
importantes para o correto uso do medicamento. Além disso, sua
finalidade comunicativa é orientar o usuário nesse uso correto, daí a
necessidade de uma linguagem objetiva na exposição das informações do
medicamento e das orientações de uso, para que a saúde não seja posta em
risco pelo mau uso.

(B) Incorreta. A bula de remédio é um texto predominantemente
informativo e expositivo, pois a intenção é que o usuário faça um uso
seguro e correto. Assim, não há intenção de fazer juízo de valor sobre o
uso ou não uso do medicamento, nem sobre suas qualidades ou defeitos.

(C) Incorreta. Não há menção a tempo, espaço, personagens etc. no texto
da bula de remédio, nem relato de acontecimentos, e sim exposição de
informações.

(D) Incorreta. A linguagem figurada expressa sentidos pouco claros e
mais interpretativos, o que não é útil na compreensão dos dados de um
medicamento.


\item

Saeb: Analisar os mecanismos que contribuem para a progressão textual.

BNCC: EF08LP15.

(A) Correta. O termo ``dever'' sugere que o futebol ``renuncia à
alegria, atrofia a fantasia e proíbe a ousadia'', como se seguisse um
roteiro predeterminado e pouco dinâmico, enquanto, na ideia de prazer,
``algum atrevido sai do roteiro e comete o disparate de driblar o time
adversário inteirinho, além do juiz e do público das arquibancadas'',
dando a ideia de improviso e criatividade.

(B) Incorreta. O futebol como obrigação seria algo justamente no mesmo sentido de "dever", não no campo do improviso e da arte.

(C) Incorreta. Velocidade, embora seja uma característica positiva no
futebol, é tida como pejorativa no texto, pois, juntamente com a força,
opõe-se ao futebol alegre, ousado e criativo.

(D) Incorreta. Segundo o texto, a tecnocracia é um dos principais
responsáveis pelo futebol roteirizado, que se relaciona com o dever e,
consequentemente, opõe-se ao prazer.


\item

Saeb: Inferir informações implícitas em distintos textos.

(A) Incorreta. Arte e criatividade são características apenas do futebol
do prazer, conforme se depreende do texto.

(B) Incorreta. Velocidade e força são características apenas do futebol
do dever, conforme explicitado no texto.

(C) Correta. Monotonia relaciona-se aos trechos ``não é organizado para
ser jogado, mas para impedir que se jogue'' e ``renuncia à alegria,
atrofia a fantasia e proíbe a ousadia''. Divertimento está associado ao
trecho ``Por sorte ainda aparece nos campos {[}...{]} algum atrevido que
sai do roteiro e comete o disparate de driblar o time adversário
inteirinho, além do juiz e do público das arquibancadas''.

(D) Incorreta. O texto fala do futebol como prática profissional,
associando-o a um jogo mais pragmático e roteirizado. Entretanto, a
criatividade não é associada pelo autor ao futebol de várzea. Ele diz
que, por sorte, ainda se encontra um representante do futebol do prazer
nos campos profissionais.

\item

Saeb: Analisar o uso de figuras de linguagem como estratégia
argumentativa.

BNCC: EF89LP37.

(A) Incorreta. No trecho, o autor faz uma afirmação objetiva que pode
ser comprovada por números, embora tais números não sejam citados no texto.

(B) Incorreta. No trecho, o autor faz uma afirmação objetiva sobre as
mudanças que houve no futebol, causadas pela tecnocracia.

(C) Incorreta. No trecho, o autor faz uma afirmação objetiva que
descreve uma consequência da entrada do capital no futebol.

(D) Correta. No trecho, o autor descreve de forma exagerada a habilidade
do jogador atrevido, por meio de uma linguagem conotativa, como forma de
mostrar quanta criatividade está presente no futebol não roteirizado que
privilegia o prazer e não o dever.

\item

Saeb: Analisar os efeitos de sentido decorrentes dos mecanismos de
construção de textos jornalísticos/midiáticos.

BNCC: EF69LP43.

(A) Incorreta. A forma verbal expressa dúvida e hipótese, o que, portanto, opõe-se à ideia de que
estaria reforçando a denúncia, pois, para isso, seria mais adequado
empregar uma forma verbal de certeza. Além disso, a notícia procura
tratar a distorção dos conceitos como ação suspeita, evitando fazer
acusações categóricas.

(B) Incorreta. A forma verbal expressa dúvida e hipótese. Porém, a notícia não faz juízo de valor sobre o fato,
já que essa não é uma das características desse gênero textual.

(C) Correta. O jornal traz uma notícia que não é nova ou inédita. Ele
está apenas relatando uma denúncia já feita antes numa reportagem por
outro veículo de informação. Por isso, usa a forma verbal de modo a evitar assumir a responsabilidade pela
afirmação e, ao mesmo tempo, atribuí-la a um terceiro.

(D) Incorreta. A reportagem citada na notícia qualificou a distorção dos
conceitos como algo ilegal, por deturpar as orientações dadas a
policiais num curso de formação, e a notícia manteve o tom de gravidade
da prática.

\item

Saeb: Analisar elementos constitutivos de textos pertencentes ao domínio
literário.

BNCC: EF69LP47.

(A) Correta. As instruções de interpretação são características da
estrutura do texto teatral e servem para orientar os atores sobre o que
fazer na hora da encenação. Essas rubricas são colocadas entre
parênteses, isoladas das falas dos personagens.

(B) Incorreta. A peça teatral, por ser do gênero dramático, conta com
apelo às emoções, mas isso não é característica estrutural do texto.

(C) Incorreta. O trecho transcrito não faz referência a tempo e espaço.

(D) Incorreta. O texto teatral é reproduzido diretamente pelos
personagens, isto é, em discurso direto. O narrador pode aparecer
eventualmente, mas não reproduz falas de personagens.

\item

Saeb: Avaliar a adequação das variedades linguísticas em contextos de
uso.

BNCC: EF69LP55

(A) Incorreta. O autor não faz juízo de valor sobre o ensino de
gramática na escola. Ele apenas afirma que a escola ensina a modalidade
padrão, e que as pessoas é que erram ao pensar que essa modalidade é a
única possibilidade na língua.

(B) Incorreta. A suposta dificuldade dos estudantes em português não
advém da falta de dedicação aos estudos, mas sim de um conceito
equivocado que têm de língua, reduzindo-a às regras linguísticas
ensinadas na escola.

(C) Incorreta. A língua do cotidiano é governada por regras tanto quanto
a língua ensinada na escola, porém são regras diferentes baseadas em
critérios diferentes, às vezes não conscientes, como no caso da língua
cotidiana.

(D) Correta. O autor diferencia as regras linguísticas ensinadas na
escola (norma-padrão) do uso real do português. Segundo ele, as pessoas
não fazem essa distinção e acabam afirmando levianamente que o português
é difícil, porque reduzem a língua, em todos os seus usos, àquela
modalidade ensinada na escola.


\item

Saeb: Analisar os mecanismos que contribuem para a progressão textual.

BNCC: EF08LP15.

(A) Incorreta. As situações retratadas no texto não são de caráter
temporal.

(B) Incorreta. A autora não expressa preferência, apenas se conforma com
as situações que acontecem com quem mora em apartamento.

(C) Incorreta. O texto não narra ações ordenadas no tempo, pois não há
marcadores temporais.

(D) Correta. Cada ação citada causa a situação seguinte, o que é marcado
pela conjunção ``porque'', e todas as ações são consequência da ação
principal que é morar em apartamento.


\item

Saeb: Identificar formas de organização de textos normativos, legais
e/ou reinvindicatórios.

BNCC: EF69LP20.

(A) Incorreta. As leis costumam impor obrigações, mas isso não é fator
preponderante na organização textual do gênero.

(B) Incorreta. O vocabulário jurídico, embora possa estar presente, não
é fator preponderante da organização do texto de lei. O texto da
Constituição transcrito não apresenta jargão jurídico que limite a
compreensão.

(C) Correta. A estrutura do texto de lei tem como elemento básico o
artigo, que organiza a divisão dos assuntos a serem tratados. O texto de
lei é identificável facilmente por essa característica estrutural.

(D) Incorreta. O texto de lei não é narrativo, ou seja, não conta uma
história.

\item

Saeb: Analisar elementos constitutivos de textos pertencentes ao domínio
literário.

BNCC: EF69LP47.

(A) Incorreta. A descrição do personagem demonstra que ele não é tolo,
pois consegue distinguir sua linguagem simples da linguagem mais
complexa da gente da cidade, e até mesmo tem uma visão crítica sobre
ela, pois, embora admire as palavras compridas e difíceis dessa gente,
sabe que elas são inúteis e talvez perigosas.

(B) Correta. O personagem vive no campo e tem muito contato com a terra
e os animais. Não tem convívio social, excetuando-se a convivência com
um companheiro que ele menciona. Esse contexto de isolamento reflete no seu estado físico e jeito de ser, tal como descrito ao longo do texto.

(C) Incorreta. É possível que o personagem seja analfabeto, mas não é a
falta de conhecimento das letras o fio condutor do texto.

(D) Incorreta. Embora utilize nas relações com as pessoas a mesma língua
com que se dirige aos brutos, isso não é sinônimo de mau humor, e sim de
uma forma austera de tratamento.

\item

Saeb: Identificar os recursos de modalização em textos diversos.

BNCC: EF08LP16.

(A) Incorreta. Não há sentido de conclusão no trecho em questão.

(B) Correta. O narrador reformula a ideia dita antes, de modo a dar ao
leitor uma visão mais apurada do perfil do personagem descrito.

(C) Incorreta. Não há exemplificação no trecho.

(D) Incorreta. Não há particularização no trecho.

\end{enumerate}

\section*{Língua Portuguesa – Simulado 4}

\begin{enumerate}

	\item

	Saeb: Inferir informações implícitas em distintos textos.

(A) Correta. Os eletrônicos se tornam obsoletos devido aos lançamentos
cada vez mais frequentes de produtos mais atualizados com inovações
tecnológicas mais avançadas, levando as pessoas a comprarem novos
equipamentos sem necessidade, simplesmente para se manterem atualizadas
com essas inovações, inutilizando o equipamento ``antigo''.

(B) Incorreta. A alternativa descreve não uma causa, mas uma possível
consequência da grande quantidade de lixo eletrônico, que não conta com
muitos locais de destinação adequada e pode acabar sendo descartado no
meio ambiente.

(C) Incorreta. A inutilização dos equipamentos eletrônicos citada no
texto se relaciona não com a qualidade de fabricação dos produtos, mas
com o lançamento de inovações tecnológicas.

(D) Incorreta. A geração de lixo eletrônico independe da existência de
projetos sociais de destinação adequada e reutilização dos eletrônicos.

\item

Saeb: Analisar os processos de referenciação lexical e pronominal.

BNCC: EF08LP15.

(A) Correta. O termo ``doença'' recategoriza a obesidade infantil como
uma condição de saúde, trazendo-a para o horizonte de preocupação da
medicina.

(B) Incorreta. O termo ``doença'', pelo contrário, atribui à obesidade
infantil uma informação de que, sendo doença, apresenta gravidade relevante.

(C) Incorreta. A referenciação promovida na relação entre os termos em
questão não remete aos dados da OMS.

(D) Incorreta. A referenciação promovida na relação entre os termos em
questão não se relaciona diretamente com o convite feito ao espectador.

\item

Saeb: Avaliar a fidedignidade de informações sobre um mesmo fato
divulgado em diferentes veículos e mídias.

(A) Incorreta. O gênero notícia não conta com conhecimento prévio do
leitor sobre o fato relatado, pois sua finalidade é justamente trazer a
informação e o leitor busca a notícia para se informar dos fatos.

(B) Correta. As duas notícias abordam o mesmo fato sob enfoques
diferentes: uma apresenta uma pesquisa de opinião sobre a segurança no
carnaval e a outra, dados estatísticos sobre a segurança no evento.
Ambos os dados comprovam o fato principal de que o carnaval foi mais
seguro em São Paulo em relação a anos anteriores.

(C) Incorreta. Ambos os veículos de notícia têm credibilidade perante a
sociedade, porém isso não é suficiente para sanar possíveis dúvidas
sobre as informações, pois estas são de caráter quantitativo.

(D) Incorreta. A fonte bibliográfica das notícias indica que elas foram
publicadas na internet, mas não se trata de um fator que ajude a
esclarecer possíveis dúvidas sobre as informações.

\item

Saeb: Identificar elementos constitutivos de textos pertencentes ao
domínio jornalístico/midiático.

BNCC: EF69LP02.

(A) Incorreta. A data é uma informação secundária, pois está posicionada
num local de pouco destaque e com fonte pequena, menor que a do texto
principal, além de não estar relacionada com a linguagem não verbal.

(B) Incorreta. O motivo da vacinação é a proteção contra a paralisia
infantil, mas seu caráter é meramente informativo, não estabelecendo
relação com a linguagem não verbal.

(C) Correta. O texto se relaciona com a fotografia para enfatizar o
método de aplicação que ocorre por via oral, uma vez que o público é o
infantil, que teme a aplicação de injeção.

(D) Incorreta. A referência ao público no texto e na fotografia cumpre
apenas um caráter informativo.

\item

Saeb: Identificar elementos constitutivos de textos pertencentes ao
domínio jornalístico/midiático.

BNCC: EF69LP02.

(A) Incorreta. Pelo contrário, a intenção é influenciar o público a
tomar a vacina.

(B) Incorreta. Não há efeito humorístico no texto.

(C) Correta. A expressão ``mostre a língua'' apresenta o sentido literal
de mostrar a língua para aplicação das gotas da vacina, bem como o
sentido figurado de desdenhar da doença.

(D) Incorreta. O texto da campanha apresenta um jogo entre os sentidos
figurado e literal, como na expressão ``mostre a língua''.


\item

Saeb: Analisar elementos constitutivos de textos pertencentes ao domínio
literário.

BNCC: EF69LP47.

(A) Incorreta. Trata-se de uma caracterização física da via em que
transitava o rapaz.

(B) Correta. A alusão à montanha-russa, um brinquedo que se move por
trilhos irregulares de subida e descidas, no parque de diversões, remete
à péssima condição da estrada, possivelmente esburacada e muito
irregular.

(C) Incorreta. Trata-se de uma informação numérica objetiva que indica
distância.

(D) Incorreta. Trata-se da relação entre um fato (ligar o rádio) e o momento em que ele ocorre (ao passar pela última porteira).


\item

Saeb: Analisar elementos constitutivos de textos pertencentes ao domínio
literário.

BNCC: EF69LP47.

(A) Incorreta. Não se trata de desfecho, pois o leitor é apresentado à
situação inicial da história.

(B) Incorreta. Não se trata de conflito, pois o leitor é apresentado à
situação inicial da história.

(C) Correta. O leitor é apresentado às personagens -- como o motorista e
o narrador --, ao espaço -- uma cidade ``produtora de papel a uma hora
ao sul de Columbus que cheirava a ovo podre'' -- e ao tempo -- ``tarde
de quarta-feira do outono de 1945, não muito depois do fim da guerra''.

(D) Incorreta. Não se trata de clímax, pois o leitor é apresentado à
situação inicial da história.


\item

Saeb: Inferir, em textos multissemióticos, efeitos de humor, ironia e/ou
crítica.

BNCC: EF69LP05.

(A) Incorreta. Não se trata de situação geradora de humor na história.

(B) Incorreta. O rapaz interrompeu a leitura e guardou os versos como
forma de protestar contra o narrador, que não lhe deu a atenção
desejada.

(C) Incorreta. O narrador tentou manter-se atento, mas foi vencido pelo
cansaço.

(D) Correta. O narrador cochilou durante a conversa, o que deixou o
rapaz indignado. Entretanto, ele não percebeu que se tratou de cansaço e
não de indiferença.


\item

Saeb: Analisar os efeitos de sentido produzidos pelo uso de
modalizadores em textos diversos.

BNCC: EF08LP16.

(A) Correta. O narrador nega que dormiu durante a conversa e considera
que o rapaz exagerou em sua indignação. Para isso, ele se refere ao ato
de dormir como ``fechar os olhos três ou quatro vezes'', querendo dizer
que nem foi motivo suficiente para tamanha indignação do rapaz.

(B) Incorreta. O narrador nega que seu cochilo durante a conversa seja
motivo suficiente para a desistência do rapaz. Inclusive, o narrador
nega ter cochilado.

(C) Incorreta. A fala ``fechei os olhos três ou quatro vezes'' é carregada
de força de expressão, isto é, não expressa exatamente a quantidade de
vezes que o narrador dormiu.

(D) Incorreta. A justificativa do narrador para não ter prestado atenção
no rapaz foi o cansaço, pois ele nega que tenha sido indiferente ao
poema do rapaz.


\item

Saeb: Inferir, em textos multissemióticos, efeitos de humor, ironia e/ou
crítica.

BNCC: EF69LP05.

(A) Correta. O goleiro é caracterizado de forme pejorativa, por exemplo:
vítima, saco de pancadas, eterno penitente ou favorito das bofetadas.
Além disso, o exagero é expresso em ``onde ele pisa, nunca mais cresce a
grama'' e ``o condena à desgraça eterna. Até o fim de seus dias, será
perseguido pela maldição''.

(B) Incorreta. Os termos usados no texto não chegam a ser
discriminatórios no sentido de ofensa, embora pejorativos. O humor
presente no texto é leve.

(C) Incorreta. O goleiro é tido como um jogador confiável (como se
percebe nos termos porteiro, guarda-metas, arqueiro, guardião) até que
cometa um erro, pois, segundo o autor, ele pode perder a confiabilidade
num único lance. Não se trata do efeito de humor do texto.

(D) Incorreta. Os jogadores de linha não são exaltados, eles são citados
para fins de comparação do quanto um goleiro precisa fazer para ser
reconhecido ou para recuperar esse reconhecimento perante a torcida.


\item

Saeb: Analisar os efeitos de sentido dos tempos, modos e/ou vozes
verbais com base no gênero textual e na intenção comunicativa.

BNCC: EF08LP16.

(A) Incorreta. Sendo o texto do gênero manual de instruções, o efeito de
sentido do imperativo não é comando, pois é uma escolha do usuário
cuidar do equipamento como queira, seguindo ou não o manual.

(B) Correta. O manual, sendo produzido pelo fabricante, traz
recomendações de manutenção correta do equipamento de modo a estender
sua vida útil.

(C) Incorreta. O manual de instruções não tem caráter emotivo, como é o
caso de um apelo, pois privilegia a objetividade.

(D) Incorreta. Sendo o texto do gênero manual de instruções, não se
trata de convite.


\item


Saeb: Analisar a intertextualidade entre textos literários ou entre
estes e outros textos verbais ou não verbais.

BNCC: F89LP32.

(A) Incorreta. Fernando Pessoa é autor de ambos os poemas, mas esse não
é o ponto de intertextualidade - se fosse assim, todos os poemas do autor
seriam intertextuais entre si, o que não acontece, pois o autor escreve
sobre temas variados.

(B) Incorreta. Os poemas divergem em seus pontos de vista sobre a origem
da inspiração do fazer poético. Um deles se inspira na imaginação,
enquanto o outro se inspira no sentimento.

(C) Correta. Os dois poemas abordam os fatores que inspiram e
influenciam o poeta em seu fazer poético.

(D) Incorreta. A intertextualidade é um fenômeno identificável no
conteúdo e não na forma de textos.

\item

Saeb: Analisar elementos constitutivos de textos pertencentes ao domínio
literário.

(A) Incorreta. As nuvens podem ser lúdicas, por apresentarem formas
reconhecidas no mundo real, porém o caráter lúdico não é o que motivou
seu uso na comparação.

(B) Incorreta. As nuvens são, pelo contrário, multiformes e, quando
menos se espera, já assumiram outro formato ou sumiram. Portanto, não é
esse o motivo da comparação.

(C) Correta. O eu lírico compara as nuvens aos seus sonhos porque,
segundo ele, estes são passageiros, assim como as nuvens são
transitórias.

(D) Incorreta. As nuvens podem ser reconhecíveis, mas isso não é algo
referido no poema. O eu lírico, ao contrário, não reconhece seus sonhos
porque são passageiros.


\item

Saeb: Analisar os processos de referenciação lexical e pronominal.

BNCC: EF08LP15.

(A) Incorreta. Pelo contrário, os elementos citados promovem a
manutenção do tema enquanto são adicionadas novas informações ao texto.

(B) Incorreta. A referenciação promovida pelos elementos citados é
intratextual.

(C) Incorreta. Os elementos citados são anafóricos, isto e´, referem-se
a termos anteriores a eles.

(D) Correta. Os dois elementos citados evitam a repetição porque retomam
um único trecho mesmo sendo formas linguísticas diferentes entre si.
Eles retomam o seguinte trecho: ``quem ascende socialmente numa capital
acaba'puxando' os parentes do interior".

\item


Saeb: Inferir a presença de valores sociais, culturais e humanos em
textos literários.

BNCC: EF69LP44.

(A) Correta. O eu poético lida com as vicissitudes da vida com
naturalidade, defendendo a importância de momentos infelizes para a
valorização dos momentos felizes.

(B) Incorreta. O eu poético não deixa transparecer essa negação no texto.


(C) Incorreta. Pelo contrário, o eu poético é otimista em relação às
dificuldades da vida, pois lida com naturalidade com elas, reconhecendo
sua importância.

(D) Incorreta. O eu poético não se indigna com as dificuldades da vida,
e sim reconhece sua importância.

\end{enumerate}
