\chapter{Sufixos}
\markboth{Módulo 1}{}

\coment{Neste módulo, espera-se que os alunos
observem a formação do feminino de adjetivos pátrios terminados
em \textbf{-ês}; relacionem o substantivo ao adjetivo formado,
conforme os sufixos \textbf{-eza} e
\textbf{-esa}; escrevam corretamente as palavras utilizando os sufixos
\textbf{-eza} e \textbf{-esa}.\\
Habilidades da BNCC: EF04LP08 EF35LP03 EF15LP03 EF35LP04 EF35LP05}

\colorsec{Habilidades do SAEB}

\begin{itemize}
\item Identificar a ideia central o texto.

\item Localizar informação explícita.

\item Inferir informações implícitas em textos.

\item Inferir o sentido de palavras ou expressões em textos.

\item Reconhecer em textos o significado de palavras derivadas a partir de seus afixos.
\end{itemize}

\conteudo{
\textbf{Palavras terminadas em -ês, -esa, -ez, -eza}

%https://www.istockphoto.com/br/foto/letras-de-alfabeto-em-uma-ordem-aleat\%C3\%B3ria-em-um-fundo-branco-gm1393899924-449579131?utm\_source=pixabay\&utm\_medium=affiliate\&utm\_campaign=SRP\_image\_sponsored\&utm\_content=https\%3A\%2F\%2Fpixabay.com\%2Fpt\%2Fimages\%2Fsearch\%2Fletras\%2F\&utm\_term=letras\includegraphics[width=3.72379in,height=2.48194in]{media/image1.jpeg}

Na língua portuguesa, há palavras que terminam com \textbf{-eza} e
palavras que terminam com \textbf{-esa}. Esses sufixos (terminações) são
pronunciados do mesmo modo, mas há regras próprias que devem ser
seguidas em seu uso. Observe os quadros a seguir.

\begin{longtable}[]{@{}l@{}}
\toprule
\begin{minipage}[t]{0.97\columnwidth}\raggedright\strut
O sufixo \textbf{-eza} transforma adjetivos em substantivos abstratos.

Exemplos: fraco -- fraqueza; delicado -- delicadeza; firme --
firmeza).\strut
\end{minipage}\tabularnewline
\bottomrule
\end{longtable}

\begin{longtable}[]{@{}l@{}}
\toprule
O sufixo \textbf{-esa} é usado para formar adjetivos pátrios no
feminino: (holandês -- holandesa; francês -- francesa; inglês --
inglesa; chinês -- chinesa; príncipe -- princesa; duque -- duquesa;
camponês -- camponesa; burguês -- burguesa).\tabularnewline
\bottomrule
\end{longtable}

As palavras terminadas em \textbf{-esa} fazem parte da classe gramatical
dos \textbf{adjetivos} e as palavras terminadas em \textbf{-eza} fazem
parte da classe gramatical dos \textbf{substantivos}.
}

\colorsec{Atividades}

Leia, agora, o início do conto \textbf{A roupa nova do imperador}, do 
escritor Hans Christian Andersen. Nesse texto, você vai conhecer a 
história de um imperador muito vaidoso, que gostava de ter muitas
roupas, mas que foi enganado por dois falsos tecelões.

\coment{Oriente os alunos a realizar, primeiramente, leitura silenciosa do
texto, seguida de leitura em voz alta. Essa estratégia de leitura
possibilita que o aluno desenvolva fluência leitora, facilitando a
compreensão global do texto lido.
As atividades introdutórias são de compreensão do texto e têm a
finalidade de desenvolver habilidades de localizar
informações, realizar simples inferências, deduzir significados de
termos do texto e inferir o tempo em que ocorre a narrativa.}

\textbf{A roupa nova do imperador}

%https://www.istockphoto.com/pt/vetorial/roman-emperor-vector-wearing-laurel-wreath-cartoon-character-gm1420622902-466533012?phrase=imperador\%20roupa

%\includegraphics[width=4.23311in,height=3.13750in]{media/image2.jpeg}

\begin{quote}
Há muito, muito tempo, vivia em um reino distante um imperador
vaidosíssimo. Seu único interesse eram as roupas. Pensava apenas em
trocar de roupas, várias vezes ao dia; desfilava vestes belíssimas,
luxuosas e muito caras para a corte. Um belo dia, chegaram à capital do
reino dois pilantras, muito habilidosos em viver às custas do próximo.

Assim que os dois souberam da fraqueza do imperador por belas roupas,
espalharam a notícia de que eles eram especialistas em tecer um pano
único no mundo, de cores e padrões deslumbrantes. E --- o mais
impressionante, segundo eles: as roupas confeccionadas com aquele tecido
tinham o poder de serem invisíveis para as pessoas tolas, ou que
ocupassem um cargo sem merecê-lo. {[}...{]}
\end{quote}

\fonte{Hans Christian Andersen. Domínio Público. A roupa nova do imperador. Disponível
em: www.dominiopublico.gov.br/download/texto/me000589.pdf.
Acesso em: 24 abr. 2023.}

%Fábia e Felipe: O texto citado está dentro de um livro. Optei por nome do autor, sem citar o site nem o título do livro, para não complicar, mas quero o OK de vocês, por favor.    

\begin{tabular}{ll}
\textbf{Glossário} & \mbox{}\\
Pilantra & pessoa desonesta.\\
\end{tabular}

\num{1} Onde o imperador vivia?

\reduline{O imperador vivia em um reino distante.\hfill}
\linhas{2}

\num{2} Qual era o único interesse do imperador?

\reduline{O único interesse do imperador eram as roupas.\hfill}
\linhas{2}

\num{3} Qual foi a notícia que os dois pilantras espalharam 
ao saber da fraqueza do imperador?

\reduline{Eles espalharam a notícia de que eram especialistas
em tecer um pano único no mundo, de cores e padrões deslumbrantes. E que
as roupas confeccionadas com aquele tecido tinham o poder de serem
invisíveis para as pessoas tolas, ou que ocupassem um cargo sem
merecê-lo.\hfill}
\linhas{4}

\num{4} Releia um trecho do conto e observe os termos destacados.

\begin{quote}
Assim que os dois souberam da \textbf{fraqueza} do imperador por
\textbf{belas} roupas, espalharam a notícia de que eles eram
especialistas em tecer um pano único no mundo, de cores e padrões
deslumbrantes.
\end{quote}

\begin{escolha}
\item
A que classe de palavras pertence o termo \textbf{belas}?

\reduline{O termo \textbf{belas} pertence à classe dos adjetivos.\hfill}

\item
A que classe de palavras pertence a palavra \textbf{fraqueza}?

\reduline{A palavra \textbf{fraqueza} pertence à classe dos substantivos.\hfill}
\end{escolha}

\num{5} Complete as frases com os \textbf{adjetivos} do quadro.

\begin{mdframed}[linewidth=2pt,linecolor=salmao,roundcorner=20pt]
\hfill\textbf{macia}\hfill \textbf{fraco}\hfill \textbf{gentil}\hfill
\end{mdframed}

\begin{escolha}
\item O resfriado que peguei me deixou muito \reduline{fraco\hfill}.

\item Ela é uma pessoa muito \reduline{gentil\hfill}.

\item A toalha do hotel era muito \reduline{macia\hfill}.
\end{escolha}

\num{6} Transforme os adjetivos que você usou para completar as frases
da atividade anterior em \textbf{substantivos}.

\reduline{Fraqueza\hfill}

\reduline{gentileza\hfill}

\reduline{maciez.\hfill}

\num{7} O escritor do conto que você leu, Hans Christian Andersen, nasceu na Dinamarca.

\begin{escolha}
\item Um rapaz que nasce na Dinamarca é \reduline{dinamarquês\hfill}.

\item Uma moça que nasce na Dinamarca é \reduline{dinamarquesa\hfill}.
\end{escolha}

\num{8} Observe as bandeiras a seguir e leia o nome dos respectivos países a que elas correspondem.

%\begin{itemize}
%\item
  %https://pixabay.com/pt/vectors/portugal-bandeira-bandeira-nacional-162394/
%\item
  %https://pixabay.com/pt/illustrations/bandeira-holanda-europa-2292675/
%\item
 % https://www.istockphoto.com/br/foto/bandeira-do-jap\%C3\%A3o-em-o-vento-gm492042302-75969761?utm\_source=pixabay\&utm\_medium=affiliate\&utm\_campaign=SRP\_image\_sponsored\&utm\_content=https\%3A\%2F\%2Fpixabay.com\%2Fpt\%2Fimages\%2Fsearch\%2Fbandeira\%2520jap\%25C3\%25A3o\%2F\&utm\_term=bandeira+jap\%C3\%%A3o
%\end{itemize}
%
%\includegraphics[width=2.00012in,height=1.33333in]{media/image3.png}
%\includegraphics[width=1.98958in,height=1.32631in]{media/image4.png}
%\includegraphics[width=1.79857in,height=1.19792in]{media/image5.jpeg}

%Portugal Holanda Japão

Escreva na tabela o nome do país de origem e as nacionalidades correspondentes. Siga o exemplo.

\begin{longtable}[]{@{}lll@{}}
\toprule
\begin{minipage}[b]{0.32\columnwidth}\raggedright\strut
\textbf{País de origem}\strut
\end{minipage} & \begin{minipage}[b]{0.32\columnwidth}\raggedright\strut
\begin{quote}
\textbf{Nacionalidade masculina}
\end{quote}\strut
\end{minipage} & \begin{minipage}[b]{0.32\columnwidth}\raggedright\strut
\begin{quote}
\textbf{Nacionalidade feminina }
\end{quote}\strut
\end{minipage}\tabularnewline
\midrule
\endhead
\begin{minipage}[t]{0.32\columnwidth}\raggedright\strut
França\strut
\end{minipage} & \begin{minipage}[t]{0.32\columnwidth}\raggedright\strut
\begin{quote}
francês
\end{quote}\strut
\end{minipage} & \begin{minipage}[t]{0.32\columnwidth}\raggedright\strut
\begin{quote}
francesa
\end{quote}\strut
\end{minipage}\tabularnewline
Portugal & português & portuguesa\tabularnewline
Holanda & holandês & holandesa\tabularnewline
Japão & japonês & japonesa\tabularnewline
\bottomrule
\end{longtable}

\num{9} Em relação à terminação das palavras que você escreveu, o que é possível
concluir sobre a grafia delas?

\reduline{As palavras terminadas em \textbf{-ês}, quando flexionadas no feminino, terminam em \textbf{-esa}.\hfill}
\linhas{3}

\num{10} No caça-palavras, encontre palavras femininas de nacionalidades dos
países de origem que estão no quadro. Em seguida, escreva as palavras
encontradas na linhas.

\begin{mdframed}[linewidth=2pt,linecolor=salmao,roundcorner=20pt]
\textbf{Holanda}\hfill \textbf{Albânia}\hfill \textbf{Escócia}\hfill \textbf{Finlândia}\hfill \textbf{Inglaterra}
\end{mdframed}

\begin{longtable}[]{@{}llllllllllll@{}}
\toprule
T & N & L & H & E & A & C & L & V & N & E & S\tabularnewline
\midrule
\endhead
I & W & G & O & A & M & E & I & E & A & D & N\tabularnewline
H & O & L & A & N & D & E & S & A & L & O & V\tabularnewline
I & E & I & F & N & T & R & D & O & T & E & P\tabularnewline
T & F & I & N & L & A & N & D & E & S & A & E\tabularnewline
I & O & N & I & T & S & U & T & C & L & V & I\tabularnewline
E & R & G & E & I & T & L & O & B & E & T & T\tabularnewline
E & O & L & W & N & S & C & A & H & E & A & P\tabularnewline
K & D & E & L & E & E & N & H & E & W & A & N\tabularnewline
I & R & S & S & S & E & T & L & E & R & H & U\tabularnewline
I & E & A & A & S & P & I & I & H & N & H & A\tabularnewline
O & S & I & A & L & U & D & T & S & A & T & I\tabularnewline
\bottomrule
\end{longtable}

\linhas{5}

\colorsec{Treino}

\num{1} Observe a bandeira da Inglaterra. %(Fácil)

%https://br.freepik.com/vetores-gratis/ilustracao-de-bandeira-reino-unido\_2807791.htm\#query=bandeira\%20inglaterra\&position=0\&from\_view=search\&track=ais

%\includegraphics[width=5.90556in,height=3.11319in]{media/image6.jpeg}

Uma mulher e um homem que nascem na Inglaterra são, respectivamente

\begin{escolha}
\item ingleza e inglezo.

\item inglaterra e inglaterro.

\item inglaterrana e inglaterrano.

\item inglesa e inglês.
\end{escolha}

\coment{SAEB: Relacionar, na compreensão do texto, informações textuais
com conhecimentos de senso comum.

BNCC: EF04LP08: Reconhecer e grafar, corretamente, palavras derivadas
com os sufixos -agem, -oso, -eza, -izar/-isar (regulares morfológicas).

a) Incorreta. A formação do feminino e do masculino dos adjetivos
pátrios têm terminação -esa e ês.

b) Incorreta. Essas palavras não correspondem às nacionalidades.

c) Incorreta. Essas palavras não correspondem às nacionalidades.

d) Correta. Inglesa e inglês são os adjetivos pátrios corretos.}

\num{2} Analise um trecho de um dos contos populares tradicionais africanos, ``Oxóssi''.
%(Médio) 

\textbf{Oxóssi}

\begin{quote}
Olofin era um rei africano da terra de Ifé, lugar de origem de todos os
iorubás.

Cada ano, na época da colheita, Olofin comemorava, em seu reino, a Festa
dos Inhames.

Ninguém no país podia comer dos novos inhames antes da festa. Chegando o
dia, o rei se instalava no pátio do seu palácio. Suas mulheres sentavam 
à sua direita, seus ministros atrás dele, agitando leques e 
espanta-moscas, e os tambores soavam para saudá-lo.

As pessoas reunidas comiam inhame pilado e bebiam vinho de palma. Elas
comemoravam e brincavam.
\end{quote}

%Fábia e Felipe: padronizando textos que o autor extraiu dessa coleção de alfabetização: Nome da autora seguido de ``e outros'' (para evitar et al). Título do texto usado. Título do livro, com volume e subtítulo. Verifiquem se essa padronização agrada. 

\fonte{Ana Rosa Abreu e outros autores. Oxóssi. Alfabetização, Vol.2: contos, fábula, lendas e mitos. Disponível em:
www.dominiopublico.gov.br/download/texto/me000589.pdf.
Acesso em: 24 abr. 2023.}

No trecho de ``Oxóssi'', os tambores soavam com o objetivo de

\begin{escolha}
\item comemorar a colheita dos inhames.

\item permitir o início do jantar.

\item fazer homenagem ao rei durante a festa.

\item acompanhar os ministros na festa.
\end{escolha}

\coment{SAEB: Inferir o sentido de uma palavra ou expressão a partir do
contexto imediato.

BNCC: EF15LP03 -- Localizar informações explícitas em textos.

a) Incorreta. Os tambores são tocados com a intenção de homenagear e
saudar o rei Olofin durante a Festa dos Inhames. Não há alusão, no texto, ao uso desses instrumentos para comemorar a colheita dos inhames.

b) Incorreta. Os tambores são tocados com a intenção de homenagear e
saudar o rei Olofin durante a Festa dos Inhames. Não há alusão, no texto, ao uso desses instrumentos para permitir o início do jantar.

c) Correta. Os tambores são tocados com a intenção de homenagear e
saudar o rei Olofin durante a Festa dos Inhames.

d) Incorreta. A função dos tambores era saudar o rei, não os ministros.}

\num{3} Leia o texto a seguir. %(Difícil)

\textbf{Pequenas agricultoras}

\begin{quote}
As formigas cortadeiras cultivam seu próprio jardim dentro do
formigueiro. {[}...{]}

Formigas cortadeiras, como as saúvas, são especializadas em cultivar
fungos. Os pedacinhos de folhas que elas cortam são levados para câmaras
especiais dentro dos formigueiros onde servirão como base para crescimento
desses fungos. Essas câmaras são chamadas pelos cientistas de ``jardins de
fungos''. Mas cortar e carregar todas aquelas folhinhas é só o início de um
trabalho minucioso de cultivo.

\fonte{Vinícius São Pedro. Ciência Hoje das Crianças. Pequenas agricultoras. Disponível em:
http://chc.org.br/artigo/pequenas-agricultoras/. Acesso em: 24 abr.
2023.}
\end{quote}

O texto ``Pequenas agricultoras'' tem a finalidade de informar sobre

\begin{escolha}
\item a importância dos fungos no formigueiro.

\item o trabalho das formigas cortadeiras.

\item o modo de cultivar um jardim com formigas.

\item o trabalho de agricultores especializados.
\end{escolha}
%As alterações acima têm a finalidade de deixar o texto das alternativas com o mesmo tamanho. Nunca falamos sobre isso, Fábia e Felipe, mas, sempre que escrevi material didático, recebi essa orientação das editoras, para evitar induzir os alunos ao erro ou ao acerto. 

\coment{SAEB: Localizar informações num texto.

BNCC: EF35LP03 -- Identificar a ideia central do texto, demonstrando
compreensão global.

a) Incorreta. O objetivo do texto é discorrer sobre o trabalho das
formigas cortadeiras.

b) Correta. A finalidade do texto é informar os leitores sobre o
trabalho das formigas cortadeiras.

c) Incorreta. O texto não expõe modos de cultivar um jardim com formigas.

d) Incorreta. O texto não trata do trabalho de agricultores 
especializados, mas das formigas cortadeiras.}

\chapter{Cartas}
\markboth{Módulo 2}{}

\coment{Neste módulo, os alunos farão leitura de
cartas, de modo a aperfeiçoar a compreensão de textos desse
gênero do campo da vida cotidiana. As atividades terão foco na
identificação do objetivo da produção -- no que se refere à incitação de
seus autores e eventuais argumentos para justificá-la -- e de elementos
da organização textual, assim como a exploração da linguagem empregada.\\
Habilidades da BNCC: EF04LP10, EF04LP14, EF04LP27, EF35LP24, EF35LP26}

\colorsec{Habilidades do SAEB}

\begin{itemize}
\item Reconhecer diferentes gêneros textuais.

\item Identificar elementos constitutivos de textos narrativos.

\item Identificar as marcas de organização de textos dramáticos.

\item Analisar os efeitos de sentido de verbos de enunciação.
\end{itemize}

%https://pixabay.com/pt/illustrations/envelope-carta-papel-magia-7076001/\includegraphics[width=3.59956in,height=3.60972in]{media/image7.png}

\conteudo{
\textbf{Cartas}

Quando o celular e a internet ainda não eram tão populares, um dos meios
mais utilizados para comunicação entre pessoas que estavam distantes uma 
da outra era a carta.

Sempre houve quem gostasse de escrever cartas. E, ainda hoje,
mesmo com tantas outras possibilidades, existem aqueles que fazem
questão de escrever cartas aos amigos, parentes, etc.

A carta pessoal é uma mensagem que pode ser escrita à mão ou digitada. É
um texto escrito para alguém, ou seja, um destinatário, com o objetivo
de estabelecer contato com ele. A linguagem utilizada costuma ser informal.

No verso do envelope, devem ser escritas as informações de quem envia a 
carta, o \textbf{remetente}; na frente, ficam registrados os dados de
quem receberá a carta, o \textbf{destinatário}.
As informações de destinatário servem para que o serviço de Correios
entregue a carta no local correto. Para isso, ambas as informações devem
conter nome, endereço completo (rua, número, bairro, cidade e estado) e
CEP (código de endereçamento postal).

O conteúdo da carta deve vir em uma folha dentro do envelope, que deverá
ser lida apenas pelo destinatário a quem a carta foi endereçada.

Hoje em dia, entretanto, há outros meios de manter diálogo com pessoas 
distantes, como o celular, os aplicativos de mensagens instantâneas e 
os \emph{e-mails}.

Existem três tipos básicos de carta, independente da maneira como será
transmitida: a correspondência \textbf{oficial}, a correspondência 
\textbf{comercial} e a correspondência \textbf{pessoal}.
}

\colorsec{Atividades}

Leia a carta pessoal a seguir.

\coment{Realize uma leitura compartilhada do texto.}

%Suprimi o trecho a seguir porque não vi nenhuma rubrica. ``Chame a atenção dos alunos para as rubricas no texto e para sua função de orientar a dramatização (como indicações das formas de falar, caminhar, gesticular; indicação de características como altura da voz, ritmo. Solicite aos alunos que observem quem são as personagens e o papel de cada um.''}

%Inserir imagem texto na carta em branco: \url{https://www.pexels.com/pt-br/foto/em-branco-vazio-cartao-ficha-7958171/}

\begin{quote}
\begin{flushright}
Fortaleza, 03 de março de 2023.
\end{flushright}

Olá, amado papai!

Estou bem com os padrinhos aqui em Fortaleza, mas tenho muita saudade de
você e de todos da família. Quando eu terminar o curso que vim fazer
aqui, volto para casa. Gostei muito dos lugares que conheci em
Fortaleza, como as praias e os parques aquáticos, mas sinto falta de
vocês e dos meus amigos de infância. Escrevo só para dizer isso, que
estou bem e o quanto sinto falta de tudo e de todos daí.

\begin{flushright}
Beijos para todos,

Gui
\end{flushright}
\end{quote}

%\includegraphics[width=3.60417in,height=5.40625in]{media/image8.jpeg}

\num{1} O que são cartas pessoais?

\reduline{Cartas pessoais são correspondências trocadas entre pessoas.\hfill}
\linhas{2}

\num{2} Quem escreveu a carta?

\reduline{O autor da carta é Guilherme\hfill}
\linhas{1}

\num{3} Quem recebeu a carta?

\reduline{O destinatário da carta é o pai do Guilherme\hfill}
\linhas{1}

\num{4} De onde Guilherme escreveu a carta?

\reduline{Guilherme escreveu a carta de Fortaleza\hfill}
\linhas{1}

\num{5} Como é composta a saudação da carta lida e o que ela indica sobre a
relação entre remetente e destinatário?

\reduline{``Olá, amado papai!'' é a saudação que indica que Guilherme,
o remetente, é filho do destinatário e o trata com afetividade.\hfill}
\linhas{2}

\num{6} Qual é o assunto da carta?

\reduline{Na carta a seu pai, Guilherme conta que está
bem, comenta o que mais gostou em Fortaleza, diz quando pretende voltar
para casa e fala sobre a saudade que está sentindo dos familiares e
amigos de infância.\hfill}
\linhas{2}

\num{7} A carta pessoal apresentada tem uma linguagem formal ou informal?
Justifique sua resposta.

\reduline{A carta pessoal normalmente é escrita de modo informal, 
tendo em vista que é uma correspondência entre pessoas íntimas,
com demonstração de afetividade e carinho. Na carta de Guilherme a seu
pai, a informalidade fica evidente pela utilização de termos como ``amado 
papai'' e frases como ``sinto falta de vocês'', cujo conteúdo é indicador
de afetividade, além de formas verbais flexionadas na primeira pessoa, 
por meio das quais o remetente fala de suas impressões pessoais.\hfill}
\linhas{2}

%A questão abaixo foi suprimida, porque não faz sentido no contexto
%\num{8} O texto apresentado é teatral. Com qual objetivo esses textos são
%escritos?

%\reduline{O objetivo de se escrever textos desse tipo é para que sejam encenados.\hfill}
%\linhas{2}

\num{8} Leia a carta a seguir e responda às questões propostas.

Criar uma carta com bordas.

\begin{mdframed}[linewidth=10pt,linecolor=salmao!20,backgroundcolor=salmao!20,roundcorner=20pt]
Curitiba, 12 de fevereiro de 2023.

\emph{Querida filha Cecília},

Estou morrendo de saudades...

Por aqui, a vida está exatamente igual, seus irmãos e seu pai também
estão bem, e sentem muitas saudades de você.

Filhinha, você está fazendo o certo. Ir a São Paulo estudar vai te
ajudar muito a conseguir o emprego dos seus sonhos.

Aproveite muito filha, ajude a sua prima Sofia com as tarefas domésticas
e agradeça sempre por ela ter recebido você na casa dela.

\begin{flushright}
Beijo,

Mamãe.
\end{flushright}
\end{mdframed}

\begin{escolha}
\item \textbf{Vocativo} é o modo como é denominado quando chamamos uma
pessoa com quem estamos nos comunicando. Sublinhe o vocativo da carta
acima.

\item Quem escreveu a carta?

\begin{boxlist}
\boxitem[\rosa{X}] a mãe de Cecília 

\boxitem[] Sofia

\boxitem[] Cecília
\end{boxlist}

\item Quem é Cecília com relação a pessoa que escreve a carta??

\reduline{Filha.\hfill}
\linhas{1}

\item Quem é o destinatário dessa carta?

\reduline{Cecília.\hfill}
\linhas{1}

\item Escreva qual foi a despedida da carta. 

\reduline{Beijo, Mamãe.\hfill}
\linhas{1}
\end{escolha}

\colorsec{Treino}

\num{1} Leia a carta pessoal a seguir para responder à questão.

%(Fácil) 

\begin{mdframed}[linewidth=10pt,linecolor=salmao!20,backgroundcolor=salmao!20,roundcorner=20pt]
São Paulo, 01 de março de 2023.

Querida vó,

Como a senhora está? Estou com muitas saudades de você e mal posso
esperar pelo Natal, quando vamos nos reunir e relembrar os tempos de
infância. Estar longe de você tem sido muito triste, mas guardo os seus
conselhos que me ajudam a superar as dificuldades.

\begin{flushright}
Até breve!

Sandra
\end{flushright}
\end{mdframed}

O elemento ``Até breve,'' da carta lida, representa a:

\begin{escolha}
\item assinatura.

\item despedida.

\item saudação.

\item reclamação.
\end{escolha}

\coment{SAEB: Inferir o sentido de uma palavra ou expressão a partir do
contexto imediato.

BNCC: EF04LP10 -- Ler e compreender, com autonomia, cartas pessoais de
reclamação, dentre outros gêneros do campo da vida cotidiana, de acordo
com as convenções do gênero carta e considerando a situação comunicativa
e o tema/assunto/finalidade do texto.

a) Incorreta. A assinatura é o nome escrito pelo remetente. O trecho `Até 
breve!'' não é a assinatura de Sandra, mas a expressão de despedida que 
ela usa.

b) Correta. A expressão ``Até breve'' foi usada como forma de despedida 
na carta de Sandra.

c) Incorreta. Saudações servem para abrir a carta, mas `Até breve!'' é a 
expressão de despedida de Sandra.   

d) Incorreta. Não há reclamação na carta de Sandra.}

\num{2} Releia a carta pessoal a seguir para responder à questão.

%(Médio) 

\begin{mdframed}[linewidth=10pt,linecolor=salmao!20,backgroundcolor=salmao!20,roundcorner=20pt]
São Paulo, 01 de março de 2023.

Querida vovó,

Como a senhora está? Estou com muitas saudades de você e mal posso
esperar pelo Natal, quando vamos nos reunir e relembrar os tempos de
infância. Estar longe de você tem sido muito triste, mas guardo os seus
conselhos que me ajudam a superar as dificuldades.

\begin{flushright}
Até breve!

Sandra
\end{flushright}
\end{mdframed}

Que saudação seria mais adequada ao contexto dessa carta?

\begin{escolha}
\item Prezada avó.

\item Olá, vovó!

\item Avó.

\item Atenciosamente.
\end{escolha}

\coment{SAEB: Inferir o sentido de uma palavra ou expressão a partir do
contexto imediato.

BNCC: EF04LP10 -- Ler e compreender, com autonomia, cartas pessoais
de reclamação, dentre outros gêneros do campo da vida cotidiana, de
acordo com as convenções do gênero carta e considerando a situação
comunicativa e o tema/assunto/finalidade do texto.

a) Incorreta. A saudação ``Prezada avó'' é mais formal e não se adequaria
ao contexto de correspondência entre avó e neta.

b) Correta. As expressões ``Olá, vovó!'' e ``Querida vovó'' são 
equivalentes na informalidade e no tratamento afetivo. 

c) Incorreta. Somente a expressão ``avó'', no vocativo, não contém
a intimidade da saudação original

d) Incorreta. ``Atenciosamente'' é expressão de despedida, não de saudação.}

\num{3} Leia a carta de uma leitora para a revista de publicações
científicas denominada \textit{Planeta}.

%(Difícil) 

\textbf{Evolução com consciência}

\begin{quote}
Sou leitora e apreciadora assídua de PLANETA. Quero parabenizar Ricardo
Arnt por estar assumindo a direção de redação. Aproveito para desejar a
toda a equipe da revista um tempo de muita paz e luz. Tenho acompanhado
muitas mudanças, algumas muito positivas e outras, a meu ver, não tão
positivas sobre a filosofia da revista, mas mudanças não podem agradar a
todos. O importante é que continue a ser uma publicação que nos ajude em
nosso evoluir com consciência, mesmo que por trilhas diferenciadas das
antigas.

\begin{flushright}
Almira Lima, Rio Grande, RS, por \emph{e-mail}.
\end{flushright}
\end{quote}

\fonte{Almira Lima. Revista Planeta. Evolução com consciência.
Disponível em: www.revistaplaneta.com.br/cartas-15/. Acesso em: 24 Abr. 2023.}

\begin{tabular}{ll}
\textbf{Glossário} & \mbox{}\\
assídua & com bastante frequência.\\
\end{tabular}

Sobre a carta acima, é correto afirmar que 

\begin{escolha}
\item o remetente da carta é ``Planeta''.

\item ``muita paz e luz'' é a despedida.

\item o título é ``sou leitora e apreciadora''.

\item ``Rio Grande'' é a localização da leitora.
\end{escolha}

\coment{SAEB: Localizar informações num texto.

BNCC: EF04LP10 -- Ler e compreender, com autonomia, cartas pessoais de
reclamação, dentre outros gêneros do campo da vida cotidiana, de acordo
com as convenções do gênero carta e considerando a situação comunicativa
e o tema/assunto/finalidade do texto.

a) Incorreta. O remetente da carta acima é Almira Lima.

b)  Incorreta. A carta não apresenta uma despedida, e a expressão ``muita 
paz e luz'' está inserida no corpo do texto, sem caracterizar uma 
saudação final. 

c) Incorreta. O título do texto é ``Evolução com consciência''.

d) Correta. A identificação da autora é seguida do local do qual ela
enviou a carta.}

\chapter{Textos instrucionais}
\markboth{Módulo 3}{}

\coment{Neste módulo, o objetivo é habilitar os alunos a reconhecer 
textos instrucionais e suas características.\\
Habilidade da BNCC: EF04LP13.}

\colorsec{Habilidades do SAEB}

\begin{itemize}
\item Analisar elementos constitutivos de gêneros textuais diversos.

\item Reconhecer os usos da pontuação.

\item Analisar os efeitos de sentido decorrentes do uso da pontuação.
\end{itemize}

\conteudo{
\textbf{Textos instrucionais}

%\textbf{https://br.freepik.com/vetores-gratis/manual-de-instrucoes-guia-documento-com-elemento-de-design-isolado-de-roda-dentada-arquivo-de-analise-de-personagem-masculino-analise-de-negocios-processamento-de-dados-atualizacao\_12083052.htm\#page=4\&query=manual\&position=3\&from\_view=search\&track=sph}

%\includegraphics[width=2.93750in,height=2.93750in]{media/image9.jpeg}

No nosso dia a dia, em diversas situações, precisamos ler textos que nos
ajudem a executar uma tarefa. Quando queremos fazer um bolo, por
exemplo, procuramos uma \textbf{receita}. Lemos um \textbf{manual} quando
queremos saber quais são as regras de um jogo, ou a \textbf{bula} quando
queremos compreender como tomar determinada medicação. Todos esses textos
que nos ensinam o que fazer para atingir um objetivo são
denominados \textbf{instrucionais}, porque fornecem instruções.

O objetivo dos textos instrucionais é informar como devemos
proceder para realizar uma tarefa. Eles devem ser simples, objetivos,
diretos e esclarecedores e são repletos de verbos, como
``precisar'', ``necessitar'' e ``dever'' para informar ao leitor os
procedimentos a cumprir.
}

\colorsec{Atividades}

\num{1} Leia o texto a seguir. Em seguida, responda às atividades propostas.

%\url{https://www.istockphoto.com/br/foto/sumo-de-beterraba-gm1178095232-329133307?utm_source=pixabay\&utm_medium=affiliate\&utm_campaign=SRP_image_sponsored\&utm_content=https\%3A\%2F\%2Fpixabay.com\%2Fpt\%2Fimages\%2Fsearch\%2Fsuco\%2520beterraba\%2F\&utm_term=suco+beterraba}

%https://www.gov.br/saude/pt-br/assuntos/saude-brasil/eu-quero-me-alimentar-melhor/noticias/2018/suco-rico-em-fibras-5-receitas-para-misturar-frutas-e-vegetais/

%\includegraphics[width=4.65278in,height=3.48958in]{media/image10.jpeg}

\textbf{Suco de beterraba com limão}

\textbf{INGREDIENTES}

2 copos americanos duplos de água\\
1 beterraba média\\
1 limão sem casca e sementes\\
1 colher de sopa de açúcar

\textbf{MODO DE PREPARO}

Misture tudo no liquidificador. Se desejar, coloque gelo e sirva.

\textbf{RENDIMENTO}

3 porções.

DICA: O limão pode ser substituído por laranja ou maracujá.

\fonte{Ministério da Saúde. Suco rico em fibras: 5 receitas para misturar frutas e vegetais. Disponível em:
https://www.gov.br/saude/pt-br/assuntos/saude-brasil/eu-quero-me-alimentar-melhor/noticias/2018/suco-rico-em-fibras-5-receitas-para-misturar-frutas-e-vegetais/.
Acesso em: 24 abr. 2023.}

\begin{escolha}
\item Por que podemos dizer que o texto lido é do tipo instrucional?

\reduline{O texto lido pode ser considerado instrucional porque é uma receita de
suco, ou seja, propõe-se a ensinar como prepará-lo. Para isso, dá
instruções do que deve ser feito pelo leitor, quais são os ingredientes
necessários e o modo de preparo.\hfill}
\linhas{3}

\item Qual é a quantidade de beterraba e de açúcar que devem ser utilizadas
para fazer o suco?

\reduline{Para fazer o suco, é necessário usar 1 beterraba média e 1
colher de açúcar.\hfill}
\linhas{1}

\item Como você descobriu essa informação?

\reduline{A informação está presente na lista de ingredientes da receita.\hfill}
\linhas{1}

\item No modo de preparo, por que não é necessário expor a quantidade de
ingredientes que devem ser utilizadas?

\reduline{A quantidade já foi informada na lista de ingredientes\hfill}
\end{escolha}

\num{2} Releia este trecho da receita atentando aos verbos destacados:

\begin{quote}
\textbf{Misture} tudo no liquidificador. Se desejar, \textbf{coloque} gelo e sirva.
\end{quote}

\begin{escolha}
\item
  Os verbos destacados foram utilizados no texto para:

\begin{boxlist}
\boxitem[] mostrar quem faz a ação.

\boxitem[\rosa{X}] fazer pedidos.

\boxitem[] dar instruções.

\boxitem[] indicar o tempo em que a ação acontece.
\end{boxlist}

\item
  Os verbos destacados estão no modo:

\begin{boxlist}
\boxitem[\rosa{X}] imperativo

\boxitem[] afirmativo
\end{boxlist}

\end{escolha}

\num{3} Leia o passo a passo de como fazer um coração de origami.

%Arte: Pedir autorização do texto e imagem a seguir ou solicitar ilustração conforme modelo a seguir. \includegraphics[width=4.87700in,height=8.57778in]{media/image11.png}

%Disponível em: https://blog.brandili.com.br/diy-como-fazer-um-origami-de-coracao/. Acesso em: 6 mar. 2023.

\begin{escolha}
\item Por que o texto lido pode ser considerado um texto instrucional? Justifique sua resposta.

\reduline{O texto pode ser considerado instrucional porque sua finalidade
é auxiliar o leitor a realizar determinada tarefa, por meio de instruções
específicas.\hfill}

\item Qual é o objetivo do texto lido?

\reduline{O objetivo do texto lido é ensinar a fazer um origami em forma de coração.\hfill}

\item Que características do texto instrucional podem ser encontradas no texto lido?

\reduline{O título em destaque, apresentando claramente o que o leitor
vai aprender a fazer; a organização em quatro passos distintos, bem 
separados visualmente; as formas verbais no imperativo; as imagens 
explicativas --- todas essas características permitem afirmar que o 
texto em destaque é instrucional.\hfill}
\end{escolha}

\colorsec{Treino}

\num{1} Leia o texto a seguir.

%(Fácil)
%https://pixabay.com/pt/illustrations/fruta-cesta-tigela-fresco-cerejas-5004282/\includegraphics[width=3.05646in,height=2.55903in]{media/image12.jpeg}

\textbf{Salada de frutas}

\textbf{Ingredientes:}

\begin{itemize}
\item 1 maçã

\item 3 bananas

\item ½ mamão

\item 1 lata de leite condensado (opcional) ou açúcar.
\end{itemize}

\textbf{Modo de preparo:}

Lave bem todas as frutas. Retire a casca e as sementes. Corte as frutas
e quadradinhos. Coloque o leite condensado ou o açúcar. Misture e leve à
geladeira por 30 minutos.

O texto lido pode ser classificado como:

\begin{escolha}
\item
  receita culinária.
\item
  reportagem.
\item
  instrução de jogos.
\item
  fábula.
\end{escolha}

\coment{SAEB: Identificar o tema central do texto.

BNCC: EF04LP13 -- Identificar e reproduzir, em textos injuntivos
instrucionais (instruções de jogos digitais ou impressos), a formatação
própria desses textos (verbos imperativos, indicação de passos a ser
seguidos) e formato específico dos textos orais ou escritos desses
gêneros (lista/ apresentação de materiais e instruções/passos de jogo).

a) Correta. O texto tem as características fundamentais de receita
culinária, especialmente a lista de ingredientes e o modo de fazer o prato.

b) Incorreta. A reportagem é um texto informativo, publicado no jornal, 
que não pode ser caracterizado como instrucional.

c) Incorreta. O texto lido não é um manual de instrução de um jogo, mas
uma receita que ensina a fazer salada de frutas.  

d) Incorreta. O texto lido é uma receita culinária, que ensina a fazer 
salada de frutas. As fábulas são narrativas em animais agem como seres
humanos.}

\num{2} Leia o texto instrucional a seguir.

%(Médio)

\textbf{Quem toca mais, ganha}

\textbf{Material necessário}\\
1 bola

\textbf{Modo de jogar}\\
Em um campo aberto, duas equipes têm como
objetivo trocar o maior número possível de passes.

Cada toque representa um número na contagem, que deve ser feita
paralelamente aos passes, em voz alta.

Quando um passe sofrer interferência da equipe adversária, a contagem
recomeça do zero.

\fonte{Ana Rosa Abreu e outros autores. Quem toca mais, ganha. Alfabetização: livro
do aluno, Vol.3: textos informativos, textos instrucionais e biografias.
Disponível em: www.dominiopublico.gov.br/download/texto/me000590.pdf. 
Acesso em: 24 abr. 2023. com alterações}

Em relação à estrutura do texto instrucional, pode-se identificar a
presença de subtítulos que

\begin{escolha}
\item mostram como obter o material e escolher os participantes.

\item explicam o funcionamento do material ``bola''.

\item listam o objeto necessário e a maneira de jogar.

\item apresentam diferentes modos de jogar o jogo.
\end{escolha}

\coment{SAEB: Realizar inferências e antecipações em relação ao conteúdo e
à intencionalidade a partir de indicadores como tipo de texto e
características gráficas.

BNCC: EF04LP13 -- Identificar e reproduzir, em textos injuntivos
instrucionais (instruções de jogos digitais ou impressos), a formatação
própria desses textos (verbos imperativos, indicação de passos a ser
seguidos) e formato específico dos textos orais ou escritos desses
gêneros (lista/ apresentação de materiais e instruções/passos de jogo).

a) Incorreta. Os textos não explicam como obter uma bola, e a 
participação não tem restrições de escolha.  

b) Incorreta. Os textos não explicam o funcionamento da bola.

c) Correta. O texto ``Material necessário'' lista a bola como objeto 
necessário para jogar; em ``Modo de jogar'' explica-se como os 
participantes devem proceder.

d) Incorreta. Em ``Modo de jogar'', é explicado apenas um modo de jogar
a brincadeira.}

\num{3} Leia o trecho de um manual.
%(Difícil) 

\begin{quote}
UNO® é recomendado para crianças e adultos a partir de 7 anos de idade e número de jogadores pode variar entre 2 e 10 pessoas.

\textbf{Baralho}\\
Para jogar UNO® é necessário comprar um baralho próprio para o jogo. Esse baralho é composto por 108 cartas.

\textbf{Objetivo}\\
Ser o primeiro jogador a fazer 500 pontos. Para fazer pontos, você deve
livrar-se o quanto antes de todas as cartas da sua mão e usar as cartas 
de ação para evitar que os adversários façam o mesmo. A quantidade de 
pontos que você ganha é a soma dos números das cartas dos oponentes.
\end{quote}

\fonte{Alfabetizando e Letrando. Texto Instrucional: Jogo UNO®. Disponível em: http://nancinaliniprof.blogspot.com/2020/05/texto-instrucional-jogo-uno.html
Acesso em: 24 abr. 2023. com alterações.}
%Aqui um problema: o link apresentado pelo autor não levava a lugar nenhum. Encontrei o tal manual de jogo nesse blog de uma professora. Me digam se dá para usar.  

Conforme informações do texto, para ganhar o jogo a pessoa deve

\begin{escolha}
\item ser a primeira a fazer 500 pontos.

\item possuir pelo menos uma carta nas mãos.

\item evitar usar as cartas de ação durante o jogo.

\item ter menos pontos que as cartas dos oponentes.
\end{escolha}

\coment{SAEB: Localizar informações num texto.

BNCC: EF03LP16 -- Identificar e reproduzir, em textos injuntivos
instrucionais (receitas, instruções de montagem, digitais ou impressos),
a formatação própria desses textos (verbos imperativos, indicação de
passos a ser seguidos) e a diagramação específica dos textos desses
gêneros (lista de ingredientes ou materiais e instruções de execução --
``modo de fazer'').

a) Correta. A finalidade do jogo é ser o primeiro jogador a fazer 500
pontos, como se verifica em \textbf{Objetivo}.

b) Incorreta. O jogador deve se livrar de todas as cartas. Além disso, em 
em \textbf{Objetivo}, afirma-se explicitamente que a finalidade do jogo
é ser o primeiro jogador a fazer 500 pontos.  

c) Incorreta. O jogador precisa utilizar as cartas de ação para evitar
que os adversários se livrem de todas as suas cartas. Além disso, em 
em \textbf{Objetivo}, afirma-se explicitamente que a finalidade do jogo
é ser o primeiro jogador a fazer 500 pontos. 

d) Incorreta. O jogador precisa ter 500 pontos, o que será calculado
conforme a soma das cartas dos adversários. Além disso, em 
em \textbf{Objetivo}, afirma-se explicitamente que a finalidade do jogo
é ser o primeiro jogador a fazer 500 pontos.}

\chapter{Anúncio Publicitário}
\markboth{Módulo 4}{}

\coment{Neste módulo, espera-se que os alunos leiam
e compreendam autonomamente textos do campo da vida pública; relacionem a
imagem no texto à mensagem escrita (linguagens verbal e não verbal); relacionem
a finalidade do texto às estratégias de convencimento; identifiquem a
função social do texto, reconhecendo para que serve e a quem se destina;
identifiquem a ideia central do texto, compreendendo-o globalmente e
infiram informações implícitas no texto.\\
Habilidades da BNCC: EF03LP19}

\colorsec{Habilidades do SAEB}

\begin{itemize}
\item Analisar o uso de recursos de persuasão em textos verbais e/ou multimodais.

\item Analisar os efeitos de sentido de recursos multissemióticos em textos que circulam em diferentes suportes.

\item Julgar a eficácia de argumentos em textos.
\end{itemize}

\conteudo{
\textbf{Anúncio publicitário}

%\textbf{https://pixabay.com/pt/photos/cartazes-muro-propaganda-679177/}\includegraphics[width=5.90556in,height=4.42917in]{media/image13.jpeg}

O \textbf{anúncio publicitário} é um texto com a finalidade de
convencer o leitor a consumir um produto ou serviço. Para esse objetivo,
utiliza linguagem persuasiva, misturando recursos visuais, como cores,
formas, símbolos, figuras, imagens fictícias, entre outros, com uma
linguagem cheia de afeto, emoções com a intenção de envolver o
consumidor em um jogo linguístico, despertar o interesse e provocar
vontades que o levem a comprar o produto ou o serviço.

A \textbf{campanha publicitária} é o conjunto de peças publicitárias,
criado por uma agência, a fim de divulgar um produto ou
determinada ideia.
}

\colorsec{Atividades}

\num{1} O dia 20 de novembro foi oficialmente incluído no calendário 
nacional como data de celebração da memória de Zumbi dos Palmares e da
consciência negra. Leia o cartaz publicitário a seguir.

\coment{Explore com os alunos o cartaz, a imagem que o compõe e os elementos
verbais. Avalie com eles os motivos da escolha do produtor ao dar ênfase
a alguns elementos do texto verbal e se esse recurso foi efetivo na
divulgação do que pretendia.}

%https://www.guiricema.mg.gov.br/20-de-novembro-dia-da-consciencia-negra/\includegraphics[width=5.28125in,height=5.28125in]{media/image14.jpeg}

\begin{escolha}
\item O cartaz da campanha do \textbf{Dia da Consciência Negra} foi feito para:

\begin{boxlist}
\boxitem[\rosa{X}] convencer as pessoas em relação à importância de lutar pela igualdade racial.

\boxitem[] convencer as pessoas em relação à importância de todos terem acesso à vacinação.

\boxitem[] sensibilizar as pessoas, como ocorre nas poesias.
\end{boxlist}

\item Releia o \emph{slogan} do cartaz.

\begin{mdframed}[linewidth=10pt,linecolor=salmao!20,backgroundcolor=salmao!20,roundcorner=20pt]
\textbf{Não precisa ser negro para lutar pela igualdade racial}
\end{mdframed}

Que sentido pode ter essa frase? Converse com os colegas e escreva a
conclusão a que chegaram nas linhas a seguir.

\linhas{3}
\reduline{Resposta pessoal. Explique aos alunos que, embora no Brasil haja
uma intensa mistura de raças, a incidência de racismo pode não ser tão
evidente para determinadas pessoas, mas ele não deixa de existir. Em alguns
casos, ele ocorre de forma sutil e não é percebido pelas pessoas. É
necessário fortalecer a identidade étnico-racial e respeitar as
diversidades.\hfill}

\item Quem promoveu esse cartaz?

\linhas{1}
\reduline{A Prefeitura de Guiricema promoveu o cartaz analisado.\hfill}

\item Em sua opinião, o anúncio está cumprindo seu propósito? Justifique 
sua resposta.

\linhas{3}
\reduline{Resposta pessoal.\hfill}
\end{escolha}

\num{2} Leia, a seguir, o cartaz de promoção do Festival Nacional do Teatro
Infantil, de Feira de Santana, Bahia.

%https://jornalgrandebahia.com.br/2017/09/a-festa-do-teatro-infantil-brasileiro-vai-comecar-em-feira-de-santana/ \includegraphics[width=5.90556in,height=3.93681in]{media/image15.jpeg}

\begin{escolha}
\item Quais estratégias, na composição dessa capa, contribuem para
incentivar as pessoas a participar do festival?

\linhas{3}
\reduline{Os alunos podem citar o colorido do cartaz, a ilustração de uma
criança com nariz de palhaço e roupa de bilheteiro, a quantidade atrações 
no festival e as atividades paralelas que serão realizadas durante o evento.\hfill}

\item Quantas vezes ocorreu o Festival Nacional de Teatro Infantil de
Feira de Santana?

\linhas{1}
\reduline{O Festival Nacional de Teatro Infantil de
Feira de Santana já aconteceu 10 vezes\hfill}

\item Quais são as atividades paralelas que vão ocorrer durante o festival?

\linhas{2}
\reduline{Oficinas, palestras, \emph{Workshops}, exibição de filmes e documentários.\hfill}

\item As cores e letras usadas no cartaz foram utilizadas com o objetivo de:

\begin{boxlist}
\boxitem[] deixá-lo mais bonito.

\boxitem[\rosa{X}] destacar a mensagem escrita.

\boxitem[] destacar um produto
\end{boxlist}

\item Por que foi usado um desenho de um personagem infantil no cartaz?

\linhas{1}
\reduline{O desenho de um personagem infantil foi usado para atrair a atenção das crianças.\hfill}

\item Quando esse festival aconteceu? 

\linhas{1}
\reduline{O festival ocorreu de 01 a 12 de outubro de 2017.\hfill}
\end{escolha}

\num{3} Você sabe o que é um \textbf{festival}? Junte-se a um colega,
pesquisem no dicionário o significado dessa palavra e registrem a seguir 
o que vocês encontraram.

\linhas{5}
\reduline{Sugestão de resposta: festivais são eventos ou espetáculos 
culturais que ocorrem periodicamente, com diversas apresentações.\hfill}

\num{4} Leia, agora, outro anúncio de campanha publicitária.

%https://www.facebook.com/daejundiai/photos/a.609606715910778/1544172665787507/?type=3
%\includegraphics[width=5.90556in,height=5.90556in]{media/image16.jpeg}

\begin{escolha}
\item Qual é o objetivo dessa campanha?

\linhas{2}
\reduline{O objetivo da campanha é a conscientização a respeito da economia
de água durante o banho.\hfill}

\item Segundo o cartaz, como se pode economizar água durante o banho?

\linhas{2}
\reduline{Para economizar água, o banho não deve durar mais de cinco minutos
e o chuveiro deve ser fechado enquanto se ensaboa.\hfill}

\item O objetivo desse cartaz é convencer os leitores a repensar ou mudar
de atitude. Você foi convencido? Justifique sua resposta.

\linhas{5}
\reduline{Resposta pessoal. Aproveite o momento e explique aos alunos que a
economia de água não apenas preserva o ambiente, mas também implica
economia financeira, tendo em vista que a conta diminui. Vale lembrar,
ainda, que a água que chega à nossa casa tem custos de captação,
tratamento e distribuição.\hfill}
\end{escolha}

\colorsec{Treino}

\num{1} Leia o cartaz relacionado à água.
%(Fácil)
%https://www.saopedro.sp.gov.br/saaesp-faz-atividade-especial-para-celebrar-dia-mundial-da-agua
%\includegraphics[width=4.75972in,height=3.18681in]{media/image17.png}

%Paulo: colocar aqui a imagem do link a seguir. 

\fonte{Prefeitura de Ipê. O Dia Mundial da Água. Disponível em:
https://www.pmipe.rs.gov.br/noticias/o-dia-mundial-da-agua.
Acesso em: 25 abr. 2023.}

%Fábia e Felipe: o link sugerido pelo autor foi tirado do ar. Além disso, na minha opinião, a questão não era adequada. A imagem não permite afirmar que a finalidade do cartaz é ``conscientizar sobre o uso de forma econômica e sem desperdícios''. A alternativa correta força uma inferência que extrapola o cartaz. Tentei melhorar, mas ainda não dei por satisfeito.    

Relacionando a imagem do cartaz com as frases nele apresentadas, podemos
afirmar que 

\begin{escolha}
\item a qualidade de vida do homem não depende da água.

\item o uso adequado da água permite qualidade de vida.

\item a água é um recurso natural ilimitado da humanidade.

\item a escassez da água já foi solucionada pela ciência.
\end{escolha}

\coment{SAEB: Identificar o tema central do texto.

BNCC: EF03LP19 -- Identificar e discutir o propósito do uso de recursos de
persuasão (cores, imagens, escolha de palavras, jogo de palavras,
tamanho de letras) em textos publicitários e de propaganda, como
elementos de convencimento.

a) Incorreta. No cartaz, uma gota de água que sai da torneira abriga uma 
família que brinca em um parque de natureza exuberante. Isso quer dizer 
que a qualidade de vida (a felicidade de desfrutar da natureza em família)
depende diretamente do \textit{uso controlado} da água (representado pela
torneira). 

b) Correta. No cartaz, uma gota de água que sai da torneira abriga uma 
família que brinca em um parque de natureza exuberante. Isso quer dizer 
que a qualidade de vida (a felicidade de desfrutar da natureza em família)
depende diretamente do \textit{uso controlado} da água (representado pela
torneira). 

c) Incorreta. Não há elementos no cartaz que permitam afirmar que a água 
seja um recurso ilimitado. Pode-se inferir, aliás, o contrário: a torneira
sugere que o uso da água deve ser controlado, exatamente porque ela é 
recurso limitado. 

d) Incorreta. Não há elementos no cartaz que permitam afirmar que a 
escassez da água já tenha sido solucionada pela ciência.}

\num{2} Analise a campanha de doação de sangue.

%http://www.moreirasales.pr.gov.br/noticia/2203/doe-sangue-salve-vidas-campanha-de-doacao-de-sangue-sera-realizado-na-proxima-semana/\includegraphics[width=5.90556in,height=5.90556in]{media/image18.jpeg}
%Disponível em: \url{http://www.moreirasales.pr.gov.br/noticia/2203/doe-sangue-salve-vidas-campanha-de-doacao-de-sangue-sera-realizado-na-proxima-semana/}. Acesso em: 7 mar. 2023.

O \emph{slogan} da campanha é

\begin{escolha}
\item ``doe sangue, salve vidas''.

\item ``divida o amor que corre nas suas veias''.

\item ``participe da campanha de doação de sangue''.

\item ``Moreira Sales''.
\end{escolha}

\coment{SAEB: Utilizar informações oferecidas por um glossário, verbete de
dicionário ou texto informativo na compreensão ou interpretação do
texto.

BNCC: EF03LP19 -- Identificar e discutir o propósito do uso de recursos de
persuasão (cores, imagens, escolha de palavras, jogo de palavras,
tamanho de letras) em textos publicitários e de propaganda, como
elementos de convencimento.

a)  Correta. Slogan é uma pequena frase usada para resumir uma campanha.
No caso da campanha de doação de sangue, o slogan é ``doe sangue, salve
vidas''.

b)  Incorreta. ``Divida o amor que corre em suas veias'' é o título da
campanha.

c)  Incorreta. ``Participe da campanha de doação de sangue'' é um subtítulo
que precede as informações sobre onde e quando doar sangue.

d)  Incorreta. Moreira Sales é a instituição responsável pela veiculação
do anúncio.}

\num{3} Leia o cartaz a seguir, observando as imagens e o texto. 

%(Difícil)

%www.saude.pr.gov.br/modules/noticias/article.php?storyid=4950\includegraphics[width=5.90694in,height=8.26736in]{media/image19.png}
%SECRETARIA DA SAÚDE DO PARANÁ. Governo lança campanha para estimular prevenção de gripe. Disponível em: \textless{}www.saude.pr.gov.br/modules/noticias/article.php?storyid=4950\textgreater{}. Acesso em: 7 mar. 2023.

A finalidade dessa campanha é

%Fábia e Felipe: alterei as alternativas porque duas delas me pareceram ``pegadinhas''
%Paulo, detalhe importante: o link apresentado pelo autor não existe mais. Encontrei o cartaz no link a seguir: http://www.crfpr.org.br/noticia/visualizar/id/7066#:~:text=Preocupado%20com%20o%20aumento%20do,sintomas%20que%20caracterizam%20a%20doen%C3%A7a. 

\begin{escolha}
\item apresentar ao público sugestões de alimentação saudável.

\item conscientizar a respeito da importância de beber água.

\item motivar o isolamento social das pessoas que estão gripadas.

\item informar sobre sintomas e sugerir dicas de prevenção contra gripe.
\end{escolha}

\coment{SAEB: Realizar inferências e antecipações em relação ao conteúdo
e à intencionalidade a partir de indicadores como tipo de texto e
características gráficas.

BNCC: EF03LP19: Identificar e discutir o propósito do uso de recursos de
persuasão (cores, imagens, escolha de palavras, jogo de palavras,
tamanho de letras) em textos publicitários e de propaganda, como
elementos de convencimento.

a) Incorreta. O cartaz contém uma sugestão referente à alimentação 
saudável, mas essa não é sua finalidade. O objetivo é informar sobre 
sintomas e sugerir dicas de prevenção contra gripe.  

b) Incorreta. O cartaz contém uma sugestão referente ao consumo de água,
mas essa não é sua finalidade. O objetivo é informar sobre sintomas e 
sugerir dicas de prevenção contra gripe.

c) Incorreta. Não há informação no cartaz referente ao isolamento social 
das pessoas acometidas de gripe. 

d) Correta. A estrutura do cartaz (título no topo, em letras grandes; 
divisões em cores diferentes, distinguindo ``sintomas'' e ``prevenção'') 
permite afirmar que a campanha tem como finalidade informar ao público 
formas de se prevenir contra a gripe, além de informar a respeito dos
sintomas dessa doença.}

\chapter{Notícias}
\markboth{Módulo 5}{}

\coment{Neste módulo, espera-se que os alunos leiam
e compreendam textos do campo da vida pública; infiram significado de
expressões apresentadas no título do capítulo; identifiquem elementos
apresentados no primeiro parágrafo da notícia (antecipação sobre o
fato); estabeleçam expectativas em relação ao texto a ser lido com base
nos conhecimentos prévios; observem manchetes de jornais e reconheçam
suas características e função no texto.\\
Habilidade da BNCC: EF35LP16}

\colorsec{Habilidades do SAEB}

\begin{itemize}
  \item Reconhecer diferentes modos de organização composicional de textos em versos.

  \item Analisar a construção de sentidos de textos em versos com base em seus elementos constitutivos.
\end{itemize}

%https://br.freepik.com/vetores-gratis/pessoas-sem-rosto-com-jornais\_23822995.htm\#page=2\&query=manchete\&position=31\&from\_view=keyword\&track=sph\includegraphics[width=5.41667in,height=3.95422in]{media/image20.jpeg}

\conteudo{
\textbf{NOTÍCIAS}

A \textbf{notícia} é um gênero textual muito presente no
cotidiano das pessoas. Sua finalidade é informar sobre fatos e
acontecimentos de relevância da atualidade que aconteceram nos mais
variados locais, que normalmente interessam a grande parte da população.

As notícias são veiculadas nos meios de comunicação por meio da
\textbf{escrita} --- em jornais e revistas impressos --- ou da \textbf{fala},
como ocorre em noticiários da televisão, do rádio e da internet.

Seja qual for o formato, as notícias costumam ser acompanhadas por
\textbf{imagens} (fotos, mapas, infográficos ou vídeos). De acordo com o
público-alvo ou o assunto a ser abordado, a notícia pode apresentar tanto a
linguagem formal como a informal.

Ao escrever uma notícia, é importante lembrar-se de que seu principal
objetivo é transmitir informações de modo claro e objetivo. Assim sendo,
é fundamental deixar claro no texto qual fato é noticiado e com quem, quando,
onde, como e por que esse fato aconteceu.

A notícia apresenta uma estrutura bem definida, que deve conter: título,
lide e corpo da notícia. O \textbf{título}, que abre a notícia, normalmente,
deve ser chamativo e apresentar os elementos fundamentais que serão relatados.
A \textbf{lide} é um pequeno texto que aborda ou resume os principais temas de 
uma matéria jornalística. O \textbf{corpo da notícia} é o texto informativo, 
que explora em detalhes o que foi apresentado de maneira geral no título e na
lide. 
}

\colorsec{Atividades}

\num{1} Leia esta notícia.

%07/06/2022

%\href{https://observatorio3setor.org.br/author/maria-fernanda-garcia/}{MARIA FERNANDA GARCIA}~ \href{https://observatorio3setor.org.br/category/noticias/mundo/}{MUNDO}~\href{https://observatorio3setor.org.br/category/noticias/}{NOTÍCIAS}

\begin{quote}
\textbf{Milagre: menino de 4 anos sobrevive sozinho por 2 dias em mata densa e fria}

\emph{Ryker Webb, de 4 anos, foi encontrado após passar dois dias
sozinho em uma área de mata com baixas temperaturas, em Montana, nos
EUA. Segundo as autoridades, cerca de 53 pessoas se empenharam nas
buscas do menino}

Ryker Webb, de 4 anos, foi encontrado após passar dois dias sozinho em
uma área de mata com baixas temperaturas, no estado de Montana, nos
Estados Unidos.

O menino desapareceu na última sexta-feira (03/06) e foi encontrado no
domingo (05/06) após um grande esforço que envolveu equipes de
socorristas, drones, helicópteros e barcos. Quando foi localizado, Webb
estava ileso, mas ``faminto, com sede e frio'', segundo o texto que o
Gabinete do Xerife do Condado de Lincoln publicou em sua página no
Facebook.

As autoridades enviaram um ``código vermelho'' de alerta a todos os
vizinhos da família Webb, pedindo a eles que procurassem a criança em
suas propriedades.

O pequeno Ryker estava brincando com o cachorro de sua família, no
quintal da casa onde mora, quando de repente desapareceu. Ele foi
encontrado a cerca de 3,8 km do local, depois de enfrentar a temperatura
de 4°C e chuvas fortes, que atrapalharam o trabalho das equipes de
resgate.

O Gabinete do Xerife local ainda ressaltou que a vegetação densa da área
tornou a busca ``extremamente difícil''. Quando a criança foi
localizada, 53 pessoas estavam empenhadas no trabalho
de resgate na
região. Autoridades e voluntários consideraram um verdadeiro milagre o
pequeno conseguir sobreviver em um terreno tão hostil com baixas
temperaturas.
\end{quote}

\fonte{Maria Fernanda Garcia. Observatório do terceiro setor. Milagre: menino de 4 anos sobrevive sozinho por 2 dias em mata densa e fria. Disponível em:
https://observatorio3setor.org.br/noticias/milagre-menino-de-4-anos-sobrevive-sozinho-por-2-dias-em-mata-densa-e-fria/.
Acesso em: 07 mar. 2023.}

\begin{escolha}
\item Que fato é relatado na notícia?

\linhas{3}
\reduline{Na notícia, relata-se que um menino de 4 anos sobreviveu sozinho por 2 dias em mata densa e fria, no Estado de Montana, nos Estados Unidos.\hfill}

\item Onde o fato relatado aconteceu?

\linhas{1}
\reduline{O fato aconteceu em Montana, nos EUA.\hfill}

\item Quando o fato aconteceu?

\linhas{3}
\reduline{O menino desapareceu na sexta-feira, 03/06, e foi encontrado no 
domingo, 05/06.\hfill}

\item
  Sublinhe o trecho do texto que você usou para responder à questão anterior.

\reduline{Resposta pessoal.
  Espera-se que os alunos sublinhem o segundo parágrafo do texto.\hfill}

\item Quem são os envolvidos no fato?

\linhas{2}
\reduline{Os envolvidos no fato são Ryker Webb, equipes de socorristas,
autoridades e voluntários.\hfill}

\item Onde a notícia foi publicada?

\linhas{1}
\reduline{A notícia foi publicada no site Observatório do terceiro setor.\hfill}

\item Quem escreveu a notícia?

\linhas{1}
\reduline{A autora da notícia é Maria Fernanda Garcia\hfill}

\item Essa notícia foi escrita para quem?

\linhas{2}
\reduline{Espera-se que os alunos percebam que a notícia é destinada a
quaisquer leitores interessados no conteúdo divulgado.\hfill}

\item Por você acha que esse fato virou notícia?

\linhas{1}
\reduline{Resposta pessoal.
Espera-se que os alunos percebam que o fato virou notícia porque não é
comum um menino de 4 anos sobreviver sozinho por 2 dias em mata densa e
fria.\hfill}
\end{escolha}

\num{2} Releia o título da notícia e o texto que vem após o título.

\begin{quote}
\textbf{Milagre: menino de 4 anos sobrevive sozinho por 2 dias em mata densa e fria}

\emph{Ryker Webb, de 4 anos, foi encontrado após passar dois dias
sozinho em uma área de mata com baixas temperaturas, em Montana, nos
EUA. Segundo as autoridades, cerca de 53 pessoas se empenharam nas
buscas do menino}
\end{quote}

\begin{escolha}
\item Os títulos dos textos jornalísticos são denominados de \reduline{manchete\hfill}.

\item Qual é a função da manchete? Marque a(s) alternativa(s) correta(s).

\begin{boxlist}
\boxitem[\rosa{X}] Chamar a atenção do leitor para o assunto da notícia.

\boxitem[] Narrar a notícia.

\boxitem[\rosa{X}] Mostrar qual será o assunto tratado.
\end{boxlist}

\item Elabore um novo título para essa notícia.

\linhas{1}
\reduline{Sugere-se que os alunos compartilhem as respostas, averiguando
a coerência do título criado por eles com o tema da notícia.\hfill}
\end{escolha}

\colorsec{Treino}

\num{1} Leia o texto a seguir.
%(Fácil)

\begin{quote}
\textbf{Canudo sustentável de bambu ganha adeptos no AC e engenheiro
recebe até 500 encomendas de kits por mês}

O canudo plástico parece inofensivo, mas virou um vilão para o meio
ambiente, porque não é biodegradável e leva centenas de anos para se
decompor. {[}...{]}

Em 2019, o Acre aprovou uma lei --- ainda sem regulamentação ---
proibindo o canudo plástico.

No estado acreano, uma alternativa sustentável, leve, durável, prática e
reutilizável vem ganhando adeptos: o canudo de bambu. Em seis meses, o
agrônomo Emanuel Amaral, que confecciona o canudo, viu a procura pelos
kits aumentar em até 400\%.

{[}...{]}
\end{quote}

\fonte{Aline Nascimento. G1. Canudo sustentável de bambu ganha adeptos no AC e
engenheiro recebe até 500 encomendas de kits por mês. Disponível
em:
https://g1.globo.com/ac/acre/natureza/amazonia/noticia/2019/12/31/canudo-sustentavel-de-bambu-ganha-adeptos-no-ac-e-engenheiro-recebe-ate-500-encomendas-de-kits-por-mes.ghtml.
Acesso em: 25 abr. 2023.}

O texto pertence ao gênero textual

\begin{escolha}
\item fábula, porque narra as experiências vividas pelo autora do texto no Acre.

\item notícia, porque informa sobre a substituição do canudo plástico pelo de bambu.

\item poema, porque mostra, por meio de versos, uma ação sustentável ocorrida no Acre.

\item conto, porque contém uma narrativa curta sobre a produção de canudos de bambu.
\end{escolha}

\coment{SAEB: Relacionar, na compreensão do texto, informações textuais com
conhecimentos de senso comum.

BNCC: EF35LP16 -- Identificar e reproduzir, em notícias, manchetes, lides
e corpo de notícias simples para público infantil e cartas de reclamação
(revista infantil), digitais ou impressos, a formatação e diagramação
específica de cada um desses gêneros, inclusive em suas versões orais.

a)  Incorreta. As fábulas são narrativas em que as ações de animais se 
assemelham às dos humanos. O texto apresentado no exercício é uma notícia 
a respeito da invenção de um canudo feito de bambu, que substitui os 
tradicionais canudos de plástico.  

b)  Correta. O texto do exercício é uma notícia, porque sua finalidade é
informativa. Nele, apresenta-se ao leitor os detalhes da invenção de um 
canudo feito de bambu, que substitui os tradicionais canudos de plástico:
quem é o inventor, qual é a invenção, onde, como e por que ela ocorreu.

c)  Incorreta. O poema é um texto escrito em versos. O texto apresentado 
no exercício é escrito em prosa e é uma notícia a respeito da invenção de
um canudo feito de bambu, que substitui os tradicionais canudos de 
plástico.   

d)  Incorreta. O texto apresentado no exercício não é uma narrativa de 
ficção. É uma notícia a respeito da invenção de um canudo feito de bambu,
que substitui os tradicionais canudos de plástico.}

\num{2} Leia a notícia a seguir.
%(Médio)

\begin{quote}
\textbf{Defesas curiosas}

Para escapar dos seus inimigos, certos animais e vegetais possuem
maneiras curiosas para se defender. O gambá e o percevejo exalam mau
cheiro para afugentar seus atacantes. O ouriço-do-mar tem espinhos
protetores em volta do corpo. O polvo solta uma tinta que escurece a
água, facilitando assim a sua fuga. O cacto também tem espinhos
protetores. As flores do açafrão são parecidas com as de outra planta
chamada cólquico que, por ser venenosa, é evitada como alimento por
certos animais. Por causa dessa semelhança, o açafrão fica protegido
também.
\end{quote}

\fonte{Ana Rosa Abreu e outros autores. Defesas curiosas. Alfabetização: 
livro do aluno, Vol.3: textos informativos, textos instrucionais e 
biografias. Disponível em:
www.dominiopublico.gov.br/download/texto/me000590.pdf. Acesso em: 25 abr.
2023.}

A finalidade da notícia apresentada é apresentar

\begin{escolha}
\item animais que atacam seus inimigos de maneira curiosa.

\item plantas inteligentes que provocam seus predadores.

\item animais que são mais espertos do plantas para se defender.

\item animais e plantas que se defendem de maneira curiosa.
\end{escolha}

\fonte{SAEB: Inferir uma afirmação implícita num texto.

BNCC: EF35LP16 -- Identificar e reproduzir, em notícias, manchetes, lides
e corpo de notícias simples para público infantil e cartas de reclamação
(revista infantil), digitais ou impressos, a formatação e diagramação
específica de cada um desses gêneros, inclusive em suas versões orais.

a) Incorreta. Os animais descritos na notícia têm formas curiosas de 
defesa, não de ataque. 

b) Incorreta. As plantas descritas na notícia têm formas curiosas de 
defesa, não de provocação aos predadores. 

c) Incorreta. O texto não contém uma comparação entre animais e plantas.  

d) Correta. A finalidade do texto é apresentar animais e plantas que se 
protegem de seus predadores de maneira curiosa.} 

\num{3} O texto a seguir traz informações sobre uma competição esportiva. 
Leia-o atentamente.

\begin{quote}
\textbf{Seleção masculina de vôlei estreia hoje na Liga das Nações}

A seleção masculina de voleibol estreia nesta sexta-feira
(31), na Liga das Nações, jogando em Katowice, na Polônia, contra a 
seleção dos Estados Unidos.
Esta é a primeira etapa de cinco envolvendo 16 equipes. No grupo de 
Katowice estão Brasil, Estados Unidos, Austrália e Polônia. Todos as 16
seleções jogam entre si e em cada etapa mudam os adversários e a cidade 
sede.
\end{quote}

\fonte{Eurico Tavares. Seleção masculina de vôlei estreia hoje na Liga das
Nações. RadioagênciaNacional. Disponível em:
http://radioagencianacional.ebc.com.br/geral/audio/2019-05/selecao-masculina-de-volei-estreia-hoje-na-liga-das-nacoes.
Acesso em: 25 abr. 2023.}

Com base nas características apresentadas, pode-se afirmar que o texto
acima é 

\begin{escolha}
\item entrevista.

\item notícia.

\item diário.

\item conto.
\end{escolha}

\coment{SAEB: Identificar o tema central do texto.

BNCC: EF35LP16 -- Identificar e reproduzir, em notícias, manchetes, lides
e corpo de notícias simples para público infantil e cartas de reclamação
(revista infantil), digitais ou impressos, a formatação e diagramação
específica de cada um desses gêneros, inclusive em suas versões orais.

a) Incorreta. Na entrevista, um entrevistador faz perguntas a um 
entrevistado. Não há nenhum desses elementos no texto. 

b) Correta. O texto pode ser considerado notícia por conter informações a 
respeito de um fato.

c) Incorreta. O diário é um texto de caráter privado, pessoal, com entradas
normalmente precedidas de data, em que o autor registra impressões pessoais
sobre seu cotidiano, sentimentos e reflexões. Não há nenhum desses
elementos no texto. 

d) Incorreta. O conto é uma narrativa curta ficcional. O texto apresentado 
na questão não é ficcional.}  

\chapter{Discurso Direto}
\markboth{Módulo 6}{}

\coment{Neste módulo, os alunos vão reconhecer os efeitos dos verbos de
enunciação no discurso direto, percebendo a importância desses verbos
para indicar os turnos de fala dos diálogos e para especificar
entonações e sentidos das falas das personagens.}

\conteudo{
\textbf{Verbos de enunciação e variedades linguísticas no discurso direto}

%https://pixabay.com/pt/illustrations/gabarito-relat\%c3\%b3rio-de-volta-3387220/\includegraphics[width=5.01488in,height=3.34306in]{media/image21.png}

Na língua portuguesa, há palavras que anunciam quando um personagem vai falar e o modo como vai falar, por exemplo: ``disse'', ``perguntou'', ``respondeu''.

Observe este exemplo:

\begin{quote}
--- Eu não falei? Ele sempre volta! --- \textbf{disse} o dono do cachorro.
\end{quote}

Essas palavras podem ser encontradas antes, no meio ou depois das falas
da personagem e são chamados de \textbf{verbos de enunciação/elocução}. 
Estes são verbos utilizados no texto que servem para introduzir/iniciar 
a fala dos personagens, indicando suas atitudes e ações. Além disso, podemos
observar que as falas das personagens podem ser introduzidas
seguidas de dois-pontos e travessão, em forma de diálogo. Esses
elementos são importantes, pois ajudam a compreender formas e intenções
das personagens nos diálogos.

Agora, vamos treinar!
}

\colorsec{Atividades}

\num{1} Leia o texto a seguir.

%https://pixabay.com/pt/illustrations/lobo-fera-lobo-cinza-animal-6612744/
%\includegraphics[width=2.69792in,height=3.82382in]{media/image22.png}

\begin{quote}
\textbf{O LOBO E O CÃO}

Um lobo e um cão se encontraram num caminho. Disse o lobo:

--- Companheiro, você está com ótimo aspecto: gordo, o pelo
lustroso\ldots{} Estou até com inveja!

--- Ora, faça como eu --- respondeu o cão. --- Arranje um bom amo. Eu
tenho comida na hora certa, sou bem tratado\ldots{} Minha única
obrigação é latir à noite, quando aparecem ladrões. Venha comigo e você
terá o mesmo tratamento.

O lobo achou ótima a ideia e se puseram a caminho.

Mas, de repente, o lobo reparou numa coisa. --- O que é isso no seu
pescoço, amigo? Parece um pouco esfolado\ldots{} --- observou ele.

--- Bem --- disse o cão --- isso é da coleira. Sabe? Durante o dia,
meu amo me prende com uma coleira, que é para eu não assustar as pessoas
que vêm visitá-lo.

O lobo se despediu do amigo ali mesmo:

--- Vamos esquecer --- disse ele. --- Prefiro minha liberdade à sua
fartura.
\end{quote}

\fonte{Ana Rosa Abreu e outros autores. O lobo e o cão. Alfabetização, Vol.2:
contos, fábulas, lendas e mitos. Disponível em:
www.dominiopublico.gov.br/download/texto/me000589.pdf.
Acesso em: 24 abr. 2023.}

\begin{tabular}{ll}
\textbf{Glossário} & \mbox{}\\
Amo & dono do cão\\
Cativo & preso\\
Esfolado & machucado\\
Lustroso & brilhante\\
\end{tabular}

\begin{escolha}
\item Quem são as personagens da história?

\linhas{1}
\reduline{As personagens são o cão e o lobo.\hfill}

\item O texto que você acabou de ler é do gênero:

\begin{boxlist}
\boxitem[] diário, texto pessoal em que uma pessoa relata experiências e reflexões.

\boxitem[\rosa{X}] fábula, narrativa concluída com moral da história, em que animais agem como gente.

\boxitem[] notícia, texto de jornal que apresenta um acontecimento real para o grande público.
\end{boxlist}

\item Qual convite o lobo recebeu do cão?

\linhas{1}
\reduline{O cão convidou o lobo a viver com ele.\hfill}

\item Por que o lobo desistiu do convite?

\linhas{2}
\reduline{O lobo desistiu do convite depois de ver as marcas da coleira no pescoço do cão.\hfill}

\item Descreva o conflito principal da fábula.

\linhas{3}
\reduline{No texto, o cão é bem tratado pelo seu tutor, mas não tem a liberdade
do lobo, isto é: o desfrute da plena liberdade se opõe às regalias de abrir 
mão dela.\hfill}
\end{escolha}

\num{2} Quem conta a história \textbf{O lobo e o cão}? Assinale a alternativa correta.

\begin{boxlist}
\boxitem[] Um narrador que participa da história (narração em 1ª pessoa).

\boxitem[\rosa{X}] Um narrador que não participa das ações (narração em 3ª pessoa).
\end{boxlist}

\num{3} No trecho a seguir, quem está participando do diálogo?

\begin{quote}
--- Companheiro, você está com ótimo aspecto: gordo, o pelo
lustroso\ldots{} Estou até com inveja!

--- Ora, faça como eu --- respondeu o cão. --- Arranje um bom amo. Eu
tenho comida na hora certa, sou bem tratado\ldots{} Minha única
obrigação é latir à noite, quando aparecem ladrões. Venha comigo e você
terá o mesmo tratamento.
\end{quote}

\linhas{1}
\reduline{A primeira das duas falas é do lobo; a segunda é do cão.\hfill}

\num{4} Como ficaria a história se o diálogo fosse composto apenas pelas falas
das personagens, sem a intervenção do narrador?

\reduline{Resposta pessoal. Espera-se que os alunos percebam que o leitor não
saberia de que maneira as personagens falaram, pois não haveria os
comentários, nem os verbos de enunciação.\hfill}

\reduline{Explique aos alunos que, em algumas narrativas, pode acontecer de o
narrador não indicar quem fala; a identificação é feita pela sequência
do discurso e a apresentação pelos verbos de enunciação.\hfill}

\num{5} Releia a fábula e pinte no texto:

\begin{itemize}
\item de azul as falas do cão;

\item de verde as falas do lobo.
\end{itemize}

6. Qual o efeito obtido por meio do discurso direto na fábula ``\textbf{O
lobo e o cão''}?

\reduline{Espera-se que os alunos percebam que o discurso direto apresenta
ao leitor a conversa das personagens como se estivesse ocorrendo naquele
momento.\hfill}

\colorsec{Treino}

\num{1} Leia o trecho do conto ``Água da vida'', observando atentamente 
os diálogos.

%(Fácil)

\begin{quote}
\textbf{Água da vida}

Houve, uma vez, um rei muito poderoso, que vivia feliz e tranquilo em
seu reino. Um belo dia, adoeceu gravemente e ninguém tinha esperanças de
que escapasse. Ele tinha três filhos, {[}...{]}.

Encontravam-se eles no jardim do castelo a chorar e, de repente, viram
surgir à sua frente um velho de aspecto venerável, que indagou a causa
de tamanha tristeza. Disseram-lhe que estavam aflitos porque o pai
estava gravemente enfermo e os médicos já não tinham esperanças de o
salvar.

O velho, então, disse-lhe:

--- Eu conheço um remédio muito eficaz, que poderá curá-lo; é a famosa
Água da Vida. Mas é muito difícil obtê-la.
\end{quote}

\fonte{Irmãos Grimm. A água da vida. Contos de Grimm. Disponível em:
https://www.grimmstories.com/pt/grimm\_contos/a\_agua\_da\_vida. Acesso
em: 25 abr. 2023.}

O trecho ``--- Eu conheço um remédio muito eficaz, que poderá curá-lo; é a
famosa Água da Vida. Mas é muito difícil obtê-la'' está em discurso

\begin{escolha}
\item direto, porque é a fala do personagem separada por travessão.

\item indireto, porque é a fala do personagem por meio do narrador.

\item direto, porque é a transcrição da fala do próprio narrador do conto.

\item indireto, porque apresenta a introdução do narrador ``disse-lhe''.
\end{escolha}

\coment{SAEB: Relacionar, na compreensão do texto, informações textuais com
conhecimentos de senso comum.

BNCC: EF35LP22 -- Perceber diálogos em textos narrativos, observando o
efeito de sentido de verbos de enunciação e, se for o caso, o uso de
variedades linguísticas no discurso direto.

a) Correta. O trecho apresentado corresponde à fala literal do personagem
denominado como ``velho'', como se vê pela separação feita pelos dois
pontos e travessão.

b) Incorreta. O trecho destacado corresponde à fala literal do personagem; 
trata-se, portanto, de discurso direto. O discurso indireto
pode ser observado no trecho ``Disseram-lhe que estavam aflitos
porque o pai estava gravemente enfermo e os médicos já não tinham
esperanças de o salvar''.

c) Incorreta. O trecho apresentado corresponde à fala literal do personagem
denominado como ``velho'', não do narrador.  

d) Incorreta. O verbo de elocução é traço típico do discurso direto. Ele é
seguido de dois-pontos e travessão, marcas que servem para distinguir o 
discurso do narrador e a fala do personagem.} 

\num{2} No diálogo a seguir, do livro \textit{O mágico de Oz}, a personagem
Dorothy e a Bruxa do Norte estão conversando.

%(Médio) 

\begin{quote}
--- Eu sempre acreditei que todas as bruxas fossem más --- disse a menina,
que sentia medo por estar frente a frente com uma bruxa de verdade.

--- Oh, não, não, isso é um erro! Havia quatro bruxas na Terra de Oz: as
que vivem no Norte e no Sul são boas, as do Leste e do Oeste eram bruxas
malvadas, mas agora que você matou uma delas sobrou apenas uma bruxa má
em toda a Terra de Oz: a que vive no Reino do Oeste.

Depois de pensar um pouco, Dorothy disse:

--- Mas a tia Ema me contou que todas as bruxas morreram há muitos e
muitos anos.

--- Quem é a tia Ema? --- perguntou a bruxinha.

--- É a minha tia que mora no Kansas, eu vim de lá.
\end{quote}

%Suponho que o autor do material tenha feito a tradução, porque o livro citado abaixo está em inglês.
\fonte{Frank L. Baum. O mundo mágico de OZ. Tradução feita para este material. Disponível em:
www.dominiopublico.gov.br/download/texto/ph000102.pdf. Acesso em: 25 abr.
2023.}

No trecho ``--- Oh, não, não, isso é um erro!'', o uso de ``oh'' indica

\begin{escolha}
\item alegria.

\item surpresa.

\item alívio.

\item tristeza.
\end{escolha}

\coment{SAEB: Reconhecer o efeito
de sentido decorrente da escolha de uma determinada palavra ou
expressão.

BNCC: EF35LP30 -- Diferenciar discurso indireto e discurso direto,
determinando o efeito de sentido de verbos de enunciação e explicando o
uso de variedades linguísticas no discurso direto, quando for o caso.

a) Incorreta. Considerando o contexto em que a frase está inserida, 
pode-se afirmar que a bruxa usa a expressão para mostrar surpresa.

b) Correta. Considerando o contexto em que a frase está inserida, 
pode-se afirmar que a bruxa usa a expressão para mostrar surpresa em 
relação à afirmação de Dorothy: a menina achava que todas as bruxas eram
más, e a bruxa se espanta com essa hipótese.

c)  Incorreta. Normalmente, para indicar alívio, são utilizadas
expressões como ``ufa'' e ``ah''. Além disso, considerando o contexto em
que a frase está inserida, pode-se afirmar que a bruxa usa a expressão 
para mostrar surpresa.

d)  Incorreta. Considerando o contexto em que a frase está inserida, 
não há elementos que indiquem que a bruxa ficou \textit{triste} com a 
afirmação de Dorothy.}  

\num{3} O livro \textit{Viagem ao centro da Terra}, escrito por Júlio Verne e
publicado em 1864, é o relato do jovem Axel sobre a jornada que
percorreu com o seu tio. O diálogo a seguir refere-se a uma discussão
acerca da continuidade da expedição.
%(Difícil) 

\begin{quote}
{[}...{]}

--- Oh, não tenho medo de que o céu nos caia sobre a cabeça. Agora, meu
tio, quais são seus planos? O senhor não está pensando em voltar à
superfície do globo?

--- Voltar? Ora essa! Muito pelo contrário, pretendo continuar a viagem
já que tudo correu tão bem até aqui.

--- É que não vejo como atravessaremos essa planície líquida.

--- Não pretendo mergulhar de cabeça.
\end{quote}

\fonte{Júlio Verne. Viagem ao centro da Terra. Disponível em:
www.ebc.com.br/sites/\_portalebc2014/files/atoms/files/viagem\_ao\_centro\_da\_terra\_\_julio\_verne.pdf.
Acesso em: 25 abr. 2023.}

No trecho de diálogo entre tio e sobrinho, prevalece

\begin{escolha}
\item a linguagem jovem, com gírias.

\item a linguagem das crianças, com diminutivos.

\item a linguagem científica, com termos específicos.

\item a norma-padrão, com certa formalidade.
\end{escolha}

\coment{SAEB D27: Identificar características típicas da fala em um texto
escrito.

BNCC: EF35LP30 -- Diferenciar discurso indireto e discurso direto,
determinando o efeito de sentido de verbos de enunciação e explicando o
uso de variedades linguísticas no discurso direto, quando for o caso.

a)  Incorreta. O texto não contém gírias.

b)  Incorreta. Não há marcas, no texto, da linguagem infantil.

c)  Incorreta. Não há marcas, no texto, da linguagem científica.

d)  Correta. O trecho apresentado corresponde à norma-padrão da língua, 
com alguma formalidade, como se verifica, por exemplo, na regência em
``medo de que'' ou no uso do pronome oblíquo com valor possessivo em 
``o céu nos caia sobre a cabeça''.}

\chapter{Concordância nominal}
\markboth{Módulo 7}{}

\coment{Neste módulo, os alunos vão identificar, no texto, os adjetivos e os
substantivos a que se referem; perceber a concordância em número e
gênero entre artigo, substantivo e adjetivo; aplicar a concordância
nominal nos diversos contextos para corresponder à norma-padrão.}

\conteudo{
\textbf{Concordância nominal}

%https://pixabay.com/pt/illustrations/abc-alfabeto-letras-ler-aprender-916665/
%\includegraphics[width=4.16597in,height=2.95939in]{media/image23.png}

Imagine a confusão que seria a comunicação se as pessoas ou os objetos
não tivessem nomes para identificá-los.

\textbf{Substantivo} é o nome de cada coisa que existe no planeta. Há
variadas formas de atribuir qualidade a um substantivo. Por exemplo: 
uma camiseta pode ser branca, preta, azul, amarela (ou de muitas outras
cores); quanto ao tamanho, a mesma peça de roupa pode ser curta ou comprida.
E também podemos dizer que ela é barata ou cara, em relação ao preço.

As palavras que servem para atribuir noções de qualidade, 
natureza, estado, e outras, ao substantivo são 
chamadas de\textbf{adjetivos}.

Importante saber que o \textbf{substantivo} pode variar em
\textbf{gênero} (masculino ou feminino) e \textbf{número} (singular ou
plural). O \textbf{adjetivo} e o \textbf{artigo} devem concordar em
gênero e número com o substantivo a que se referem.
}

\colorsec{Atividades}

Conto é um gênero narrativo muito popular na literatura. Geralmente, 
os contos são narrativas curtas, em que as ações contadas por um 
narrador e são vividas por uma ou mais personagens, em um lugar e 
período específicos.  

\coment{Chame a atenção dos alunos para o título do texto: ``O soldadinho de
chumbo''. Peça a eles que levantem hipóteses sobre o enredo com base no
título. Provavelmente, alguns farão referência a um enredo de conto de
fadas. Incentive a participação dos alunos.}

\num{1} Leia o conto Joãozinho-sem-medo e, em seguida, responda às questões propostas.

\begin{quote}
\textbf{O Soldadinho de Chumbo}

%Fábia e Felipe: esse conto é muito mais extenso do que qualquer outro texto deste volume e de todos os outros que editei. Parece-me que esse tamanho escapa ao padrão do nosso material. 

%https://br.freepik.com/vetores-gratis/guarda-em-pe-com-uma-arma\_13571969.htm\#query=soldadinho\%20chumbo\&position=3\&from\_view=search\&track=ais
%\includegraphics[width=2.68750in,height=4.34722in]{media/image24.jpeg}

Numa loja de brinquedos havia uma caixa de papelão com vinte e cinco
soldadinhos de chumbo, todos iguaizinhos, pois haviam sido feitos com o
mesmo molde. Apenas um deles não estava inteiro: como fora o último a ser
fundido, faltou chumbo para completar-lhe uma das pernas. Mas esse soldadinho
logo aprendeu a ficar em pé sobre a perna que tinha e não fazia feio
ao lado dos irmãos.

Esses soldadinhos de chumbo eram muito bonitos e elegantes, cada qual
com seu fuzil ao ombro, a túnica escarlate, calça azul e uma bela pluma
no chapéu. Além disso, tinham feições de soldados corajosos e
cumpridores do dever.

Os valorosos soldadinhos de chumbo aguardavam o momento em que passariam
a pertencer a algum menino.

Chegou o dia em que a caixa foi dada de presente de aniversário a um
garoto. Foi o presente de que ele mais gostou:

--- Que lindos soldadinhos! --- exclamou maravilhado.

E os colocou enfileirados sobre a mesa, ao lado dos outros brinquedos. O
soldadinho de uma perna só era o último da fileira.

Ao lado do pelotão de chumbo se erguia um lindo castelo de papelão, um
bosque de árvores verdinhas e, em frente, havia um pequeno lago feito de
um pedaço de espelho.

A maior beleza, porém, era uma jovem que estava em pé na porta do
castelo. Ela também era de papel, mas vestia uma saia de tule bem
franzida e uma blusa bem justa. Seu lindo rostinho era emoldurado por
longos cabelos negros, presos por uma tiara enfeitada com uma pequenina
pedra azul.

A atraente jovem era uma bailarina, por isso mantinha os braços erguidos
em arco sobre a cabeça. Com uma das pernas dobrada para trás, tão
dobrada, mas tão dobrada, que acabava escondida pela saia de tule.

O soldadinho a olhou longamente e logo se apaixonou, e pensando que, tal
como ele, aquela jovem tão linda tivesse uma perna só.

``Mas é claro que ela não vai me querer para marido'', pensou
entristecido o soldadinho, suspirando. ``Tão elegante, tão
bonita\ldots{} Deve ser uma princesa. E eu? Nem cabo sou, vivo numa
caixa de papelão, junto com meus vinte e quatro
irmãos''.

À noite, antes de deitar, o menino guardou os soldadinhos na caixa, mas
não percebeu que aquele de uma perna só caíra atrás de uma grande
cigarreira.

Quando os ponteiros do relógio marcaram meia-noite, todos os brinquedos
se animaram e começaram a aprontar mil e uma. Uma enorme bagunça!

As bonecas organizaram um baile, enquanto o giz da lousa desenhava
bonequinhos nas paredes. Os soldadinhos de chumbo, fechados na caixa,
golpeavam a tampa para sair e participar da festa, mas continuavam
prisioneiros.

Mas o soldadinho de uma perna só e a bailarina não saíram do lugar em
que haviam sido colocados. Ele não conseguia parar de olhar aquela
maravilhosa criatura. Queria ao menos tentar conhecê-la, para ficarem
amigos.

De repente, se ergueu da cigarreira um homenzinho muito mal-encarado.
Era um gênio ruim, que só vivia pensando em maldades. Assim que ele
apareceu, todos os brinquedos pararam amedrontados, pois já sabiam de
quem se tratava.

O geniozinho olhou a sua volta e viu o soldadinho, deitado atrás da
cigarreira.

--- Ei, você aí, por que não está na caixa, com seus irmãos? --- gritou
o monstrinho.

Fingindo não escutar, o soldadinho continuou imóvel, sem desviar os
olhos da bailarina.

--- Amanhã vou dar um jeito em você, você vai ver! --- Gritou o geniozinho
enfezado. --- Pode esperar.

Depois disso, pulou de cabeça na cigarreira, levantando uma nuvem que
fez todos espirrarem.

Na manhã seguinte, o menino tirou os soldadinhos de chumbo da caixa,
recolheu aquele de uma perna só, que estava caído atrás da cigarreira, e
os arrumou perto da janela. O soldadinho de uma perna só, como de
costume, era o último da fila.

De repente, a janela se abriu, batendo fortemente as venezianas. Teria
sido o vento, ou o geniozinho maldoso? E o pobre soldadinho caiu de
cabeça na rua.

O menino viu quando o brinquedo caiu pela janela e foi correndo
procurá-lo na rua. Mas não o encontrou. Logo se consolou: afinal, tinha
ainda os outros soldadinhos, e todos com duas pernas.

Para piorar a situação, caiu um verdadeiro temporal. Quando a tempestade
foi cessando, e o céu limpou um pouco, chegaram dois moleques. Eles se
divertiam, pisando com os pés descalços nas poças de água. Um deles viu
o soldadinho de chumbo e exclamou:

--- Olhe! Um soldadinho! Será que alguém jogou fora porque ele está
quebrado?

--- É, está um pouco amassado. Deve ter vindo com a enxurrada.

--- Não, ele está só um pouco sujo.

--- O que nós vamos fazer com um soldadinho só? Precisaríamos pelo menos
meia dúzia, para organizar uma batalha.

--- Sabe de uma coisa? --- Disse o primeiro garoto. --- Vamos colocá-lo
num barco e mandá-lo dar a volta ao mundo.

E assim foi. Construíram um barquinho com uma folha de jornal, colocaram
o soldadinho dentro dele e soltaram o barco para navegar na água que
corria pela sarjeta.

Apoiado em sua única perna, com o fuzil ao ombro, o soldadinho de chumbo
procurava manter o equilíbrio. O barquinho dava saltos e esbarrões na
água lamacenta, acompanhado pelos olhares dos dois moleques que,
entusiasmados com a nova brincadeira, corriam pela calçada ao lado.

Lá pelas tantas, o barquinho foi jogado para dentro de um bueiro e
continuou seu caminho, agora subterrâneo, em uma imensa escuridão. Com o
coração batendo fortemente, o soldadinho voltava todos seus pensamentos
para a bailarina, que talvez nunca mais pudesse ver.

De repente, viu chegar em sua direção um enorme rato de esgoto, olhos
fosforescente e um horrível rabo fino e comprido, que foi logo
perguntando:

--- Você tem autorização para navegar? Então? Ande, mostre-a logo, sem
discutir.

O soldadinho não respondeu, e o barquinho continuou seu incerto caminho,
arrastado pela correnteza. Os gritos do rato do esgoto exigindo a
autorização foram ficando cada vez mais distantes.

Enfim, o soldadinho viu ao longe uma luz, e respirou aliviado; aquela
viagem no escuro não o agradava nem um pouco. Mal sabia ele que,
infelizmente, seus problemas não haviam acabado.

A água do esgoto chegara a um rio, com um grande salto; rapidamente, as
águas agitadas viraram o frágil barquinho de papel.

O barquinho virou, e o soldadinho de chumbo afundou. Mal tinha chegado
ao fundo, apareceu um enorme peixe que, abrindo a boca, engoliu-o.

O soldadinho se viu novamente numa imensa escuridão, espremido no
estômago do peixe. E não deixava de pensar em sua amada: ``O que estará
fazendo agora sua linda bailarina? Será que ainda se lembra de mim?''.

E, se não fosse tão destemido, teria chorado lágrimas de chumbo, pois
seu coração sofria de paixão.

Passou-se muito tempo --- quem poderia dizer quanto?

E, de repente, a escuridão desapareceu e ele ouviu quando falavam:

--- Olhe! O soldadinho de chumbo que caiu da janela!

Sabem o que aconteceu? O peixe havia sido fisgado por um pescador,
levado ao mercado e vendido a uma cozinheira. E, por cúmulo da
coincidência, não era qualquer cozinheira, mas sim a que trabalhava na
casa do menino que ganhara o soldadinho no aniversário. Ao limpar o
peixe, a cozinheira encontrara dentro dele o soldadinho, do qual se
lembrava muito bem, por causa daquela única perna.

Levou-o para o garotinho, que fez a maior festa ao revê-lo. Lavou-o com
água e sabão, para tirar o fedor de peixe, e endireitou a ponta do
fuzil, que amassara um pouco durante aquela aventura.

Limpinho e lustroso, o soldadinho foi colocado sobre a mesma mesa em que
estava antes de voar pela janela. Nada estava mudado. O castelo de
papel, o pequeno bosque de árvores muito verdes, o lago reluzente feito
de espelho. E, na porta do castelo, lá estava ela, a bailarina: sobre
uma perna só, com os braços erguidos acima da cabeça, mais bela do que
nunca.

O soldadinho olhou para a bailarina, ainda mais apaixonado, ela olhou
para ele, mas não trocaram palavra alguma. Ele desejava conversar, mas
não ousava. Sentia-se feliz apenas por estar novamente perto dela e
poder amá-la.

Se pudesse, ele contaria toda sua aventura; com certeza a linda
bailarina iria apreciar sua coragem. Quem sabe, até se casaria com
ele\ldots{}

Enquanto o soldadinho pensava em tudo isso, o garotinho brincava
tranquilo com o pião. De repente como foi, como não foi --- é caso de se
pensar se o geniozinho ruim da cigarreira não metera seu nariz ---, o
garotinho agarrou o soldadinho de chumbo e atirou-o na lareira, onde o
fogo ardia intensamente.

O pobre soldadinho viu a luz intensa e sentiu um forte calor. A única
perna estava amolecendo e a ponta do fuzil envergava para o lado. As
belas cores do uniforme, o vermelho escarlate da túnica e o azul da
calça perdiam suas tonalidades.

O soldadinho lançou um último olhar para a bailarina, que retribuiu com
silêncio e tristeza. Ele sentiu então que seu coração de chumbo começava
a derreter --- não só pelo calor, mas principalmente pelo amor que ardia
nele.

Naquele momento, a porta escancarou-se com violência, e uma rajada de
vento fez voar a bailarina de papel diretamente para a lareira, bem
junto ao soldadinho. Bastou uma labareda e ela desapareceu. O soldadinho
também se dissolveu completamente.

No dia seguinte, a arrumadeira, ao limpar a lareira, encontrou no meio
das cinzas um pequenino coração de chumbo: era tudo que restara do
soldadinho, fiel até o último instante ao seu grande amor.

Da pequena bailarina de papel só restou a minúscula pedra azul da tiara,
que antes brilhava em seus longos cabelos negros.
\end{quote}

\fonte{Hans Christian Andersen. O soldadinho de chumbo. Alfabetização, Vol.2: contos, fábula, lendas e mitos. Disponível em:
www.dominiopublico.gov.br/download/texto/me000589.pdf.
Acesso em: 24 abr. 2023. com alterações}

%Fábia e Felipe: aqui tomei a liberdade de fazer pequenas adaptações no primeiro parágrafo para evitar a palavra ``perneta'', que poderia ser considerada capacitismo. Me digam o que acham. Uma alternativa é deixar o termo como está e explicar em uma nota que ele não adequado.  

\begin{escolha}
\item Qual é o assunto do conto?

\reduline{O tema do conto são as aventuras de um soldadinho de chumbo
apaixonado por uma bailarina de papel.\hfill}

\item Onde se inicia a história?

\linhas{1}
\reduline{A história se inicia em uma loja de brinquedos.\hfill}

\item Como o Soldadinho foi parar na rua?

\linhas{5}
\reduline{O menino que era dono do Soldadinho decidiu arrumar a tropa na
janela. Mas, talvez por maldade do geniozinho da cigarreira, talvez por causa
de um vento forte, o Soldadinho acabou caindo na rua.\hfill}

\item Quem foi o primeiro a encontrar o Soldadinho de Chumbo?

\linhas{1}
\reduline{O primeiro a encontrar o Soldadinho de Chumbo foi um dos meninos que 
estava passando pela rua.\hfill}

\item O que os meninos fizeram com o Soldadinho de Chumbo?

\linhas{3}
\reduline{Os meninos fizeram um barco de papel e colocaram o soldado dentro, 
para ele dar a volta ao mundo.\hfill}

\item Quem engoliu o Soldadinho de Chumbo?

\linhas{1}
\reduline{O Soldadinho de Chumbo foi engolido por um peixe muito grande.\hfill}

\item Como o Soldadinho voltou para sua casa?

\linhas{5}
\reduline{O peixe que engoliu o Soldadinho de Chumbo foi comprado pela
cozinheira da família do menino. Ela encontrou o Soldadinho na barriga 
do peixe e o entregou ao seu primeiro dono.\hfill}

\item Qual é o fim da história?

\linhas{5}
\reduline{O Soldadinho de Chumbo é arremessado pelo menino na lareira, onde
derrete junto com a bailarina, que foi parar no fogo depois de um pé de vento.
\hfill}
\end{escolha}

\num{2} Releia um trecho do conto \textbf{O soldadinho de chumbo}.

\begin{quote}
Esses soldadinhos de chumbo eram muito bonitos e elegantes
\end{quote}

\begin{escolha}
\item
  Circule os adjetivos que aparecem neste trecho.

  \reduline{Os alunos devem circular os adjetivos \textit{bonitos} e 
  \textit{elegantes}.\hfill}

\item
  A quem esses adjetivos se referem?

  \linhas{1}
  \reduline{Os adjetivos se referem aos soldadinhos de chumbo.\hfill}

\item O verbo está no singular ou no plural? Justifique sua resposta.

\linhas{2}
\reduline{O verbo está no plural para concordar com o substantivo 
\textit{soldadinhos}.\hfill}
\end{escolha}

\num{3} Na frase ``Além disso, tinham feições de soldados corajosos'', qual é o
adjetivo usado para qualificar o substantivo?

\reduline{\textit{Corajosos} é o adjetivo usado para qualificar o substantivo.
\hfill}
\linhas{2}

\num{4} Por que o adjetivo está no plural e no masculino?

\reduline{O adjetivo está no plural e no masculino para concordar com o
substantivo \textit{soldados}, a que se refere.\hfill}
\linhas{2}

\num{5} Leia esta frase.

\begin{mdframed}[linewidth=10pt,linecolor=salmao!20,backgroundcolor=salmao!20,roundcorner=20pt]
O pobre soldadinho viu a luz intensa e sentiu um forte calor.
\end{mdframed}

\begin{escolha}
\item Qual é a classe gramatical da palavra destacada?

\reduline{A palavra destacada é um artigo.\hfill}
\linhas{1}

\item A concordância obedece às mesmas regras que você identificou na
  atividade anterior? Justifique sua resposta.

\reduline{Sim, a concordância obedece às mesmas regras da
  atividade anterior, porque, da mesma maneira que o adjetivo, o artigo deve concordar com o substantivo tanto em gênero quanto em número.\hfill}
\end{escolha}

\colorsec{Treino}

\num{1} Leia um trecho de um conto popular.
%(Fácil) 

\begin{quote}
\textbf{Acoitrapa E Chuquilhanto}

Na cordilheira que fica em cima do vale de Yyucay, em Cusco,
pode-se ouvir todos os sons. O vento sopra com sua bocarra;
a manhã, obrigada a se levantar sempre antes dos outros,
boceja morta de sono; os pássaros, seus eternos namorados,
acordam cantando ao ouvi-la se espreguiçar.

De repente, silêncio. Acaba de chegar Acoitrapa, o
pastor de lhamas. Ele é \textbf{jovem} e \textbf{belo}. Toca a quena tão
docemente, que até as flores mais \textbf{tímidas} se abrem e
despontam entre os galhos das árvores para escutá-lo. 
\end{quote}

\fonte{Ana Rosa Abreu e outros autores. Acoitrapa E Chuquilhanto.
Alfabetização, Vol.2: contos, fábulas, lendas e mitos. Disponível em:
www.dominiopublico.gov.br/download/texto/me000589.pdf.
Acesso em: 24 abr. 2023.}

As palavras destacadas no texto são:

\begin{escolha}
\item substantivos.

\item adjetivos.

\item artigos.

\item numeral.
\end{escolha}

\coment{SAEB: Inferir o sentido de uma palavra ou expressão a partir do
contexto imediato.

BNCC: EF04LP07 -- Identificar em textos e usar na produção textual a
concordância entre artigo, substantivo e adjetivo (concordância no grupo
nominal).

a)  Incorreta. As palavras destacadas são adjetivos.

b)  Correta. Os adjetivos \textit{jovem} e \textit{belo} caracterizam o
pastor; \textit{tímidas} qualifica as flores.

c)  Incorreta. As palavras destacadas são adjetivos.

d)  Incorreta. As palavras destacadas são adjetivos.}

\num{2} Leia um trecho da lenda indígena ``As lágrimas de Potira''.
%(Médio) 

\begin{quote}
\textbf{As lágrimas de Potira}

Muito antes de os brancos atingirem os sertões de Goiás, em busca de
pedras preciosas, existiam por aquelas partes do Brasil muitos povos
indígenas, vivendo em paz ou em guerra e segundo suas crenças e hábitos.
Numa das aldeias, que por muito tempo manteve a harmonia com seus
vizinhos, viviam Potira, menina bela contemplada por Tupã com a formosura das
flores, e Itagibá, jovem forte e valente. 
\end{quote}

\fonte{Ana Rosa Abreu e outros autores. As lágrimas de Potira.
Alfabetização, Vol.2: contos, fábulas, lendas e mitos. Disponível em:
www.dominiopublico.gov.br/download/texto/me000589.pdf.
Acesso em: 24 abr. 2023.}

Uma das palavra utilizadas no texto para caracterizar Potira é o adjetivo

\begin{escolha}
\item ``formosura''.

\item ``valente''.

\item ``forte''.

\item ``bela''.
\end{escolha}

\coment{SAEB: Utilizar informações oferecidas por um glossário, verbete de
dicionário ou texto informativo na compreensão ou interpretação do
texto.

BNCC: EF04LP07 -- Identificar em textos e usar na produção textual a
concordância entre artigo, substantivo e adjetivo (concordância no grupo
nominal).

a) Incorreta. ``Formosura'' é o substantivo utilizado para nomear a qualidade
de ser formosa.

b) Incorreta. O adjetivo ``valente'' refere-se a Itagibá.

c)  Incorreta. O adjetivo ``forte'' refere-se a Itagibá.

d)  Incorreta. O adjetivo ``bela'' refere-se a Potira.}

\num{3} Leia um trecho do texto ``Bruxas incompreendidas'',
da revista \textit{Ciência Hoje das Crianças}.
%(Difícil) 

\begin{quote}
\textbf{Bruxas incompreendidas}

Aposto que você já ouviu falar que, se pegar em uma mariposa e colocar a
mão no olho, você fica cego. Vou dizer uma coisa com franqueza: isso é
um mito. Acusadas de serem feias, sem graça e \textbf{venenosas}, as
mariposas são bichos muito incompreendidos. Quanta injustiça!
\end{quote}

\fonte{Bruxas incompreendidas. Ciência Hoje das Crianças. Disponível em:
https://chc.org.br/bruxas-incompreendidas/. Acesso em: 25 abr. 2023.}

A palavra ``venenosas'', destacada no texto, está concordando com

\begin{escolha}
\item bichos.

\item feias.

\item mariposas.

\item injustiça.
\end{escolha}

\coment{SAEB: Localizar informações num texto.

BNCC: EF04LP07 -- Identificar em textos e usar na produção textual a
concordância entre artigo, substantivo e adjetivo (concordância no grupo
nominal).

a) Incorreta. No texto, o adjetivo ``venenosas'' concorda com o substantivo ``mariposas''.

b) Incorreta. No texto, o adjetivo ``venenosas'' concorda com o substantivo ``mariposas''.

c) Correta. No texto, o adjetivo ``venenosas'' concorda com o substantivo ``mariposas''.

d) Incorreta. No texto, o adjetivo ``venenosas'' concorda com o substantivo ``mariposas''.}

\chapter{Fatos e Opiniões}
\markboth{Módulo 8}{}

\coment{Neste módulo, espera-se que os alunos localizem informações explícitas
no texto; façam distinção entre fato e opiniões em texto jornalístico; e
identifiquem a função social do texto.}

%https://br.freepik.com/vetores-gratis/ilustracao-do-conceito-de-opiniao\_20547312.htm\#query=opini\%C3\%A3o\&position=9\&from\_view=search\&track=sph\includegraphics[width=3.70764in,height=3.70764in]{media/image25.jpeg}

\conteudo{
\textbf{Fato ou opinião?}

\textbf{Fato} é algo que ocorreu e pode ser comprovado de algum modo,
por meio de documentos, estatísticas, vídeos, estudos ou registros.
Observe o exemplo:

%\includegraphics[width=3.07626in,height=2.19722in]{media/image26.jpeg}
%https://www.pexels.com/pt-br/foto/pessoa-segurando-injecao-3825529/
%Vacinas salvam vidas.

A informação é um fato, tendo em vista que há registros dos casos e, por
meio deles, é possível fazer uma afirmação.

\textbf{Opinião} consiste em uma interpretação do fato, isto é, uma
forma pessoal de olhar o fato. Essa opinião pode ser diferente de pessoa
para pessoa a depender de muitos fatores. Veja o exemplo:

%\includegraphics[width=1.98958in,height=2.97842in]{media/image27.jpeg}
%https://www.pexels.com/pt-br/foto/prato-de-sopa-em-tigela-de-ceramica-branca-6072108/
%Sopa é a melhor comida do mundo!
}

\colorsec{Atividades}

\num{1} Leia o texto a seguir.

\coment{Realize a primeira leitura, em voz alta, até o fim do texto. Depois da
sua leitura, solicite aos alunos que leiam alternadamente o texto.
Incentive-os a consultar o dicionário quando não entenderem o
significado de alguma palavra.}

\begin{quote}
\textbf{Dia Internacional dos Direitos Humanos}

%\includegraphics[width=5.90556in,height=3.33125in]{media/image28.jpeg}

``Todos os seres humanos nascem livres e iguais em dignidade e
direitos. São dotados de razão e consciência e devem agir em relação
uns aos outros com espírito de fraternidade''. Esse é o Artigo I de um
documento muito importante chamado Declaração Universal dos Direitos
Humanos.

O texto é um dos documentos básicos da Organização das Nações Unidas que
faz aniversário dia 10 de dezembro, dia em que a Declaração foi
assinada. Na declaração, são enumerados os direitos que todos os seres
humanos têm, sem distinção alguma, seja de raça, de cor, de sexo, de
língua, de religião, de opinião política ou outra, de origem nacional ou
social, de fortuna, de nascimento ou de qualquer outra situação. E, é
claro, devem ser respeitados por todos.

São direitos básicos para promover a liberdade, a justiça e a paz no
mundo (saúde, educação, cultura e arte, liberdade de informação e
expressão, habitação e alimentação adequadas, entre outros). A
declaração serve como um tipo de guia para que governos, entidades e
cidadãos em cada país respeitem e sejam respeitados, além de inspirar
tratados internacionais sobre direitos humanos.

Talvez você não saiba, mas o francês René Cassin, conhecido como ``o
homem dos direitos humanos'', foi um dos ``pais espirituais'' e redator
principal do primeiro projeto de Declaração Universal dos Direitos
Humanos. Ele perseguiu sem descanso a sua missão internacional de
humanista. Assim, por sua ação em favor do ``respeito aos direitos
humanos no contexto mundial'', recebeu, em 1968, o Prêmio Nobel da Paz.

O Dia Internacional dos Direitos Humanos é mais que uma data
comemorativa. É um momento para que todos lembrem e reflitam sobre a
garantia dos direitos humanos na sua comunidade, cidade, país e em todo
o mundo. O respeito a esses direitos e a garantia de uma vida digna
depende da vigilância e participação de todos os povos e nações.

\textbf{Reconhecimento aos defensores dos Direitos Humanos no Brasil}\\
O respeito aos direitos humanos é condição para o desenvolvimento de
qualquer país. Por isso, aqui no Brasil, institui-se o Prêmio Direitos
Humanos. Ele é a mais alta condecoração do governo brasileiro a pessoas
e entidades que se destacam na defesa e na promoção e também no
enfrentamento e no combate a violações dos direitos humanos no país.

\fonte{DIA Internacional dos Direitos Humanos. Plenarinho. Disponível em:
https://plenarinho.leg.br/index.php/2017/01/dia-internacional-dos-direitos-humanos/. com alterações. 
Acesso em: 25 abr. 2023.}
\end{quote}

\begin{escolha}
\item Qual é o assunto do texto?

\reduline{O assunto do texto é o Dia Internacional dos Direitos Humanos.\hfill}
\linhas{1}

\item Qual é o Artigo I da Declaração Universal dos Direitos Humanos?

\reduline{``Todos os seres humanos nascem livres e iguais em dignidade e
direitos. São dotados de razão e consciência e devem agir em relação
uns aos outros com espírito de fraternidade''.\hfill}
\linhas{5}

\item Por que René Cassin recebeu, em 1968, o Prêmio Nobel da Paz?

\reduline{René Cassin recebeu o Prêmio Nobel da Paz por sua
ação em favor do ``respeito aos direitos humanos no contexto mundial''.\hfill}
\linhas{3}

\item Segundo o texto, o que é o Prêmio Direitos Humanos?

\reduline{O Prêmio Direitos Humanos é a mais alta
condecoração do governo brasileiro a pessoas e entidades que se destacam
na defesa e na promoção e também no enfrentamento e no combate a
violações dos direitos humanos no país.\hfill}
\linhas{5}
\end{escolha}

\num{2} Nos trechos reproduzidos a seguir, identifique os fatos com \textbf{F}
e as opiniões com \textbf{O}.

\coment{Antes da realização desta
atividade, importante salientar aos alunos que um modo de perceber se
alguma informação é um fato ou opinião é observar se ela pode ser
comprovada com evidências (no caso do fato) ou se é possível concordar
ou não com o que é dito (no caso da opinião).}

\begin{boxlist}
\boxitem[\rosa{F}] ``Todos os seres humanos nascem livres e iguais em
dignidade e direitos. São dotados de razão e consciência e devem agir 
em relação uns aos outros com espírito de fraternidade''

\boxitem[\rosa{O}] ... ``E, é claro, devem ser respeitados por todos.''

\boxitem[\rosa{O}] ... ``A declaração serve como um tipo de guia para que
governos, entidades e cidadãos em cada país respeitem e sejam respeitados,
além de inspirar tratados internacionais sobre direitos humanos''.

\boxitem[\rosa{O}] ... ``Talvez você não saiba, mas o francês René Cassin,
conhecido como ``o homem dos direitos humanos''.

\boxitem[\rosa{F}] ... ``o francês René Cassin, conhecido como ``o homem 
dos direitos humanos'', foi um dos ``pais espirituais'' e redator principal do
primeiro projeto de Declaração Universal dos Direitos Humanos''.
\end{boxlist}

\num{3} Leia o trecho do texto a respeito do Dia Mundial da Liberdade de
Pensamento:

\begin{quote}
\textbf{Dia Mundial da Liberdade de Pensamento}

Pode ser a falta de proximidade, de contato olho no olho; pode ser a
certeza de que não haverá um revide imediato; ou, ainda, a ilusão do
anonimato que a internet proporciona -- o fato é que as pessoas insultam
outras pela web de um jeito que jamais fariam se estivessem frente a
frente. E as redes sociais são o grande palco dessas agressões.

Em todo o mundo, grandes empresas como Facebook, YouTube e Twitter
abriram canais em que os usuários podem denunciar conteúdos e contas que
disseminam ou incentivam a violência. Embora qualquer tipo de agressão
seja passível de punição, há uma que se distingue pela gravidade: é o
discurso de ódio.

O discurso de ódio tem como principal característica o fato de querer
atingir uma minoria {[}...{]} Ele é especialmente nocivo porque promove
a intolerância e impede a pluralidade de vozes, ferindo assim
a democracia. {[}...{]}

Precisamos formar cidadãos que, cada vez mais, saibam que discurso de
ódio não é piada, opinião, polêmica ou controvérsia. Liberdade de
expressão é um direito valiosíssimo, que deve ser usado para fortalecer
a democracia, não para silenciar minorias.

\fonte{DIA Mundial da Liberdade de Pensamento. Plenarinho. Disponível em:
https://plenarinho.leg.br/index.php/2021/07/dia-mundial-da-liberdade-de-pensamento/.
Acesso em: 25 abr. 2023.}
\end{quote}

\begin{escolha}
\item Escreva um trecho do texto que expressa:

\coment{Ajude os estudantes a diferenciar fato de opinião. Esse tema causa  
dificulades e dúvidas para eles, mas é fundamental que, de modo gradual, 
eles se habituem a realizar esse tipo de análise.}

\begin{itemize}
\item o \textbf{fato} que foi noticiado: \reduline{Foi noticiado o Dia Mundial
  da Liberdade de Pensamento e o combate ao discurso de ódio.\hfill}

\item A \textbf{opinião} dos autores do texto: \reduline{Sugestão de resposta:
  Precisamos formar cidadãos que, cada vez mais, saibam que discurso de
  ódio não é piada, opinião, polêmica ou controvérsia.\hfill}
  
\item O \textbf{motivo} da opinião dos autores do texto: \reduline{Liberdade de
  expressão é um direito valiosíssimo, que deve ser usado para
  fortalecer a democracia, não para silenciar minorias.\hfill}
\end{itemize}

\end{escolha}

\colorsec{Treino}

\num{1} Leia o trecho de uma carta do leitor.
%(Fácil) 

\begin{quote}
\textbf{Fogo na Amazônia}

É um absurdo a ineficácia das autoridades responsáveis por esse assunto
{[}\ldots{}{]}. Parecem ignorar que na região o Exército tem Batalhões
de Selva, a Marinha tem embarcações preparadas e batalhões de fuzileiros
navais, e a Força Aérea, aeronaves de patrulhamento para agir de
imediato assim que se detecta o início de um desmatamento ou de um foco
de incêndio.

Com tudo isso, poderiam impedir essas situações e prender os envolvidos.
Mas o resultado aí está, com o fogo atingindo o Pantanal e a fumaça já
chegando até o Rio Grande do Sul.

\fonte{Painel do Leitor. Folha de São Paulo. Disponível em:
https://www1.folha.uol.com.br/paineldoleitor/2020/09/fome-assim-no-brasil-no-seculo-21-e-barbarie-diz-leitor.shtml.
Acesso em: 15 mar. 2023.}
\end{quote}

Com base na leitura da carta do leitor, pode-se identificar que o leitor
quis apresentar

\begin{escolha}
\item uma sugestão para o jornal realizar uma reportagem a respeito do tema.

\item uma opinião sobre uma questão publicada no jornal.

\item uma crítica ao veículo jornalístico sobre a questão da Amazônia.

\item uma pergunta às autoridades sobre a questão da Amazônia.
\end{escolha}

\coment{SAEB: Estabelecer, no interior de um texto, relação entre um fato e
uma opinião relativa a este fato.

BNCC: EF04LP15 -- Distinguir fatos de opiniões/sugestões em textos
(informativos, jornalísticos, publicitários etc.).

a)  Incorreta. O texto não contém sugestão para o jornal, mas a
opinião do autor da carta.

b)  Correta. No trecho ``É um absurdo a ineficácia das autoridades
responsáveis por esse assunto'', pode-se identificar que o objetivo da
carta é expressar a opinião do leitor sobre o assunto publicado no jornal.

c)  Incorreta. A crítica é direcionada ao governo.

d)  Incorreta. No texto, não há nenhuma pergunta endereçada às autoridades.}

\num{2} Leia um trecho da reportagem a seguir.
%(Médio)

\begin{quote}
\textbf{População brasileira é a 5ª mais feliz do mundo, diz pesquisa}

Os brasileiros nunca foram tão felizes, mas apenas quatro em cada dez
estão satisfeitos com a economia, segundo uma pesquisa do instituto
Ipsos que avaliou a felicidade da população em 32 países.

No Brasil, 83\% dos entrevistados consideram-se muito felizes ou felizes
--- uma alta de 20 pontos percentuais em relação ao último levantamento,
feito em dezembro de 2021, quando o índice foi de 63\%. No mundo, a
percepção de felicidade também subiu, de 67\% para 73\%.

``As pessoas estão vendo este ano como o encerramento de um capítulo
extremamente desafiador em nossa história: a covid-19, ainda que a
pandemia não tenha sido totalmente erradicada, seu impacto é
infinitamente menor do que nos últimos anos. Esse sentimento reforça a
percepção de felicidade'', diz Marcos Calliari, CEO da Ipsos no Brasil.

\fonte{População brasileira é a 5ª mais feliz do mundo, diz pesquisa. BBC
Brasil. Disponível em:
https://www.bbc.com/portuguese/articles/cye4ll78l3wo. Acesso em: 16 mar.
2023.}
\end{quote}

No terceiro parágrafo, o trecho que está entre aspas mostra a

\begin{escolha}
\item opinião do autor da reportagem.

\item opinião do CEO da Ipsos no Brasil.

\item explicação sobre a pesquisa.

\item opinião dos entrevistados.
\end{escolha}

\coment{SAEB: Estabelecer, no interior de um texto, relação entre um fato e
uma opinião relativa a este fato.

BNCC: EF04LP15 -- Distinguir fatos de opiniões/sugestões em textos
(informativos, jornalísticos, publicitários etc.).

a) Incorreta. As aspas servem para destacar a opinião do CEO da Ipsos. 
Não é possível identificar, no conjunto, a opinião do autor do texto.  

b) Correta. As aspas servem para destacar a opinião do CEO da Ipsos.

c) Incorreta. As aspas servem para destacar a opinião do CEO da Ipsos.
Nesse trecho, ele não explica nem analisa os resultados da pesquisa.

d) Incorreta. As aspas servem para destacar a opinião do CEO da Ipsos, não
a dos entrevistados.}

\num{3} Leia o texto a seguir.
%(Difícil)

\begin{quote}
\textbf{'Corvo-Correio', da escritora Isabel Cintra, também foi lançado 
em Angola}

José é um corvo que sonhava voar para entregar cartinhas ao lado dos
pombos brancos. No entanto, o irredutível chefe do serviço postal do
bosque, Coruja Mafalda, não permite. ``Mas você é um corvo! Certamente já
deve ter ouvido dizer que os corvos não servem para carteiros. Todos
sabem disso!'', diz. Após ser praticamente ignorado e rejeitado, José
decide ir embora da região. Mas o jogo vira quando o inverno chega. O
livro infantil \textit{Corvo-Correio} fala sobre valores como tolerância,
igualdade e representatividade, conceitos que precisam ser cada vez mais
trabalhados com os pequeninos.

``O Corvo José é orgulhosamente meu pássaro negro mensageiro. Na
verdade, este protagonista cheio de força sempre esteve presente em
algum lugar em mim, durante toda a minha infância e se escondia
assustado a cada situação de preconceito vivida. Foi preciso uma espera
de crescimento, abandonar o silêncio da criança para deixar a coragem
emergir em forma de palavras que propagam afeto'', explica a autora
Isabel Cintra.

\fonte{\textit{Corvo-Correio}, da escritora Isabel Cintra, também foi
lançado em Angola. Estadão. Disponível em:
https://www.estadao.com.br/emais/comportamento/livro-infantil-traz-licoes-sobre-preconceito-exclusao-e-resiliencia/.
Acesso em: 25 abr. 2023.}
\end{quote}

O trecho em que o autor da reportagem apresenta um fato objetivo sobre
o conteúdo do livro é

\begin{escolha}
\item ``O Corvo José é orgulhosamente meu pássaro negro mensageiro''.

\item ``fala sobre
valores como tolerância, igualdade e representatividade''.

\item ``Na verdade, este protagonista cheio de força sempre esteve
presente em algum lugar em mim''.

\item ``Foi preciso uma espera de crescimento, abandonar o silêncio da
criança para deixar a coragem emergir em forma de palavras que propagam
afeto''.
\end{escolha}

\coment{SAEB: Estabelecer, no interior de um texto, relação entre um fato e
uma opinião relativa a este fato.

BNCC: EF04LP15 -- Distinguir fatos de opiniões/sugestões em textos
(informativos, jornalísticos, publicitários etc.).

a) Incorreta. O trecho contém uma opinião da autora sobre o livro.

b) Correta. O trecho destacado nessa alternativa contém a descrição
objetiva de temas do livro feita pelo autor. Trata-se, portanto, da apresentação de 
um fato. 

c) Incorreta. O trecho contém uma opinião da autora sobre o livro.

d) Incorreta. O trecho contém uma opinião da autora sobre o livro.}

\chapter{Infográficos}
\markboth{Módulo 9}{}

\coment{Neste módulo, os alunos vão interpretar infográficos,
relacionando-os ao contexto em que estão inseridos.}

\conteudo{
\textbf{Infográficos}

%https://www.pexels.com/pt-br/foto/realizacao-conquista-facanha-sucesso-7948039/
%\includegraphics[width=2.81075in,height=4.21050in]{media/image29.jpeg}

Infográficos servem para informar ou explicar algo de modo breve,
objetivo, dinâmico e em linguagem acessível por meio de textos curtos
escritos e elementos não verbais. Os temas abordados nos infográficos
costumam tratar das mais diversificadas áreas.

Normalmente, os infográficos podem ser encontrados em revistas, jornais
e \emph{sites} da internet, acompanhando, complementando e/ou resumindo
as informações veiculadas em notícias, reportagens, entrevistas e textos
informativos. Há também infográficos que apresentam as informações sem
fazer parte de outro texto.

Nos infográficos, as informações podem ser apresentadas por meio de
desenhos, fotos, gráficos, tabelas, mapas, cores, formas, entre outros
recursos visuais, e de textos curtos e dados numéricos. Título, imagens 
e legendas são comuns nesses textos.

Os infográficos são exemplos de \textbf{textos multimodais}, isto é,
que apresentam várias linguagens (escrita, visual, verbal).
}

\colorsec{Atividades}

\coment{Explicar aos alunos que o gráfico deste exercício é conhecido 
como gráfico de barras. Caso julgue pertinente, mostre aos alunos como
se elabora um gráfico usando um programa de computador.}

\num{1} Leia um trecho da notícia a seguir e analise os infográficos,
observando suas funções e as informações apresentadas.

\begin{quote}
\textbf{Recicla Santos quase dobra coleta de recicláveis no último
semestre -- confira infográfico}

%PUBLICADO:~31~DE~JANEIRO~DE~2018~ 18h~20

Coordenador de Políticas Ambientais da Secretaria de Meio Ambiente,
Marcus Fernandes atribui os resultados diretamente ao Recicla Santos.
``Entre junho e julho de 2017, a coleta seletiva passou de 270 para 420
toneladas. Nos 26 anos de história, nunca houve um aumento nessa
dimensão''.

Fernandes lembra que a aplicação do Recicla Santos foi precedida de
palestras, encontros e reuniões realizadas pela Prefeitura para divulgar
a lei entre a população.

Além de aliviar o aterro sanitário do Sítio das Neves, o aumento da
reciclagem traz efeitos positivos para a sociedade e todo o planeta.
``Temos de lembrar, por exemplo, que para se fabricar plástico se gasta
petróleo. A reciclagem reduz a retirada de matéria-prima para a produção
de diversos tipos de produtos. Estamos, assim, utilizando menos água e
energia elétrica''.

O coordenador lembra, ainda, que a nova lei criou postos de trabalho na
área da reciclagem. Atualmente, há duas cooperativas de catadores
atuando na Cidade, a Comares e a Sem Fronteira.

%\includegraphics[width=5.90556in,height=3.90764in]{media/image30.jpeg}
%\includegraphics[width=5.90556in,height=2.92847in]{media/image31.jpeg}

\fonte{Recicla Santos quase dobra coleta de recicláveis
no último semestre -- confira infográfico. Prefeitura de Santos.
Disponível em:
https://www.santos.sp.gov.br/?q=noticia/recicla-santos-quase-dobra-coleta-de-reciclaveis-no-ultimo-semestre-confira-infografico.
Acesso em: 26 abr. 2023.}
\end{quote}

\begin{escolha}
\item Qual é a finalidade desses infográficos?

\reduline{A finalidade dos
infográficos é acrescentar novas informações ao texto verbal, além 
de representar, em linguagem não verbal, o conteúdo dele.\hfill}
\linhas{3}

\item Por que recursos como esse são importantes em textos informativos?

\reduline{Os infográficos e outros recursos similares, como gráficos, 
são importantes porque facilitam a compreensão do conteúdo do texto.\hfill}
\linhas{3}

\item De acordo com o segundo infográfico, qual foi o aumento obtido
pelo Coleta Seletiva entre julho e dezembro?

\reduline{O aumento foi de 92\%.\hfill}
\linhas{1}

\item Qual foi o aumento no acumulado anual?
\reduline{O aumento acumulado anual foi de 21\%.\hfill}
\linhas{1}

\item Marque \textbf{CD} para o que deve descartado na coleta diária,
\textbf{CS} para o que deve ser descartado na coleta seletiva e \textbf{D}
para o que deve ser devolvido aos postos de venda.

\begin{boxlist}
\boxitem[\rosa{CS}] Resíduos limpos.

\boxitem[\rosa{CD}] Resíduos orgânicos.

\boxitem[\rosa{D}] Resíduos especiais.
\end{boxlist}

\item Quantas cooperativas atuam na cidade de Santos? Quais são elas?

\linhas{1}
\reduline{Há duas cooperativas atuantes em Santos: a Comares e a Sem Fronteira.\hfill}

\item Um recurso bastante utilizado em infográficos é a variedade de
tamanhos e cores das letras. Em sua opinião, fazer essa diferença visual
no texto verbal do infográfico é importante? Por quê?

\linhas{3}
\reduline{Resposta pessoal.
Espera-se que os alunos respondam que esses recursos são utilizados para
destacar elementos que o produtor do infográfico deseja que o leitor
visualize e também para organizar as informações.\hfill}
\end{escolha}

\colorsec{Treino}

\num{1} Leia o infográfico a seguir que traz informações relacionadas a
como fazer um clube de leitura.
%(Fácil) 

%\includegraphics[width=1.67569in,height=9.72500in]{media/image32.jpeg}

\fonte{Infográfico: saiba como montar um clube de leitura. Secretaria da 
Educação do Estado de São Paulo. Disponível em:
\textless{}www.educacao.sp.gov.br/noticia/infografico-saiba-como-montar-um-clube-de-leitura/\textgreater{}.
Acesso em: 26 abr. 2023.}

Para organizar as informações nesse infográfico, o autor

\begin{boxlist}
\item usou a mesma cor para dividir os itens do passo a passo.

\item escolheu um tipo de letra para cada um dos itens.

\item optou por usar mais imagens e pouco texto.

\item selecionou cores específicas para separar os itens.
\end{boxlist}

\coment{SAEB: Identificar o tema central do texto.
BNCC: EF04LP20 -- Reconhecer a função de gráficos, diagramas e tabelas em
textos, como forma de apresentação de dados e informações.

a) Incorreta. Cada um dos itens do infográfico é caracterizado com um cor
específica. 

b) Incorreta. Cada um dos itens do infográfico contém mais de um tipo de
letra.

c) Incorreta. O infográfico analisado combina imagens e textos, sem dar
preferência a um ou outro.

d) Correta. Cada um dos itens do infográfico é caracterizado com um cor
específica.}

\num{2} O infográfico mostra a presença de alunos estrangeiros na rede
estadual de ensino no Estado de São Paulo.
%(Médio) 

%https://www.educacao.sp.gov.br/infografico-alunos-estrangeiros-na-rede-estadual-de-ensino/
%\includegraphics[width=2.04375in,height=9.72500in]{media/image33.png}

\fonte{\#Infográfico: alunos estrangeiros na rede estadual de ensino.
Secretaria da Educação do Estado de São Paulo. Disponível em:
www.educacao.sp.gov.br/noticias/infografico-alunos-estrangeiros-na-rede-estadual-de-ensino/.
Acesso em: 18 mar. 2023.}

Podemos observar no infográfico que o número de matrículas em 2019 foi

\begin{escolha}
\item 3,5 vezes maior do que o registrado no ano anterior.

\item 12 mil vezes maior que o registrado no ano anterior.

\item 18\% maior do que o registrado no ano anterior.

\item 124\% maior do que o registrado no ano anterior.

\end{escolha}

\coment{SAEB: Estabelecer relação entre informações num texto ou entre
diferentes textos.

BNCC: EF04LP20 -- Reconhecer a função de gráficos, diagramas e tabelas em
textos, como forma de apresentação de dados e informações.

Todos os itens da questão utilizaram dados numéricos contidos no 
infográfico. Para escolher a alternativa correta, contudo, o aluno 
deve selecionar aquela que corresponde especificamente ao que foi
solicitado no enunciado. 

a) Incorreta. 3,5 milhões é o número de alunos matriculados e não se 
refere ao aumento do número de matrículas.

b) Incorreta. 12 mil é o número de alunos estrangeiros matriculados
e não se refere ao aumento do número de matrículas.

c) Correta. Conforme a segunda informação em destaque no infográfico
depois do título, o número de matrículas em 2019 foi 18\% maior que o
registrado no ano anterior. 

d)  Incorreta. 124 é o número de alunos bolivianos matriculados na 
Escola Padre Anchieta e não se refere ao aumento do número de matrículas.}

\num{3} Leia o infográfico a seguir.
%(Difícil) 

%www.inca.gov.br/en/node/3201
%\includegraphics[width=3.58333in,height=6.36458in]{media/image34.jpeg}

\fonte{Atividade física para o controle do câncer. National Cancer 
Institute. Disponível em: www.inca.gov.br/en/node/3201. Acesso em: 26
abr. 2023.}

Qual é o tema central do infográfico?

\begin{escolha}
\item a população brasileira que é insuficientemente ativa.

\item a atividade física para prevenção do câncer.

\item os malefícios de ser uma pessoa insuficientemente ativa.

\item atividades para prevenção ao câncer de intestino.
\end{escolha}

\coment{SAEB: Identificar o tema central do texto.

BNCC: EF04LP20 -- Reconhecer a função de gráficos, diagramas e tabelas em
textos, como forma de apresentação de dados e informações.

a)  Incorreta. O infográfico contém informação sobre a população 
brasileira que é insuficientemente ativa, mas o título ``Ser fisicamente
ativo é uma das formas de se proteger do câncer'' permite afirmar que o 
tema central é a atividade física como forma de prevenção ao câncer. 

b)  Correta. O título ``Ser fisicamente ativo é uma das formas de se
proteger do câncer'' permite afirmar que o tema central é a atividade 
física como forma de prevenção ao câncer. O conjunto das informações 
confirma essa afirmação.  

c)  Incorreta. É possível inferir, a partir das informações apresentadas,
que quem é fisicamente inativo tende a ser menos saudável, mas esse não é
o tema do infográfico. 

d)  Incorreta. Há informações sobre a prevenção ao câncer de intestino,
mas o tema do infográfico é mais amplo: a atividade física como forma de
prevenção ao câncer.}

\chapter{Coesão textual}
\markboth{Módulo 10}{}

\coment{
Nesta seção, os alunos vão analisar palavras utilizadas em trechos de
texto para fazer referência aos substantivos; verificar o sentido
expresso por essas palavras e observar que estabelecem ligação entre os
trechos; identificar os pronomes anafóricos e estabelecer a coesão ao
completar trechos com eles.}

%https://www.istockphoto.com/br/foto/bal\%C3\%A3o-de-di\%C3\%A1logo-letras-em-cortar-revista-gm518358614-89988479?phrase=palavras
%\includegraphics[width=5.17173in,height=3.44883in]{media/image35.jpeg}

\conteudo{
\textbf{Coesão}

Na língua portuguesa, há palavras e expressões que podem ser usadas para
ligar as ideias e relacionar os parágrafos, dando progressão ao texto.

\textbf{Coesão} consiste na ligação entre os elementos gramaticais.
Refere-se ao modo como as frases e palavras se relacionam e são
combinadas. A \textbf{coerência}, por sua vez, é a formação lógica 
do texto, o que dá sentido por meio de uma linguagem e sequência 
adequadas. Sem coesão, o texto fica sem qualquer conexão entre as partes;
sem coerência, ele confunde o leitor. Um texto coeso e coerente assegura 
que a sua produção textual cumpra a sua finalidade: comunicar algo,
convencer o leitor ou impactar de algum modo.

Por meio da coesão referencial, cria-se um sistema de relações entre as
palavras e expressões dentro de um texto, fazendo com que o leitor
reconheça os termos aos quais se referem. O termo que indica a entidade
ou situação a que o falante se refere é chamado de referente.
}

\colorsec{Atividades}

\num{1} Leia a narrativa a seguir.

\coment{Sugere-se propor inicialmente uma leitura
silenciosa. Após a leitura, retomar os aspectos que chamaram a atenção
de cada aluno. Pode-se fazer, em seguida, uma leitura compartilhada,
parágrafo a parágrafo, conversando sobre as palavras desconhecidas e as
impressões relacionadas ao texto.}

%https://www.istockphoto.com/br/vetor/ugly-duckling-bullying-concept-vector-cartoon-ilustra\%C3\%A7\%C3\%A3o-gm1372690734-441739165?utm\_source=pixabay\&utm\_medium=affiliate\&utm\_campaign=SRP\_illustration\_sponsored\&utm\_content=https\%3A\%2F\%2Fpixabay.com\%2Fpt\%2Fillustrations\%2Fsearch\%2Fpatinho\%2520feio\%2F\&utm\_term=patinho+feio
%\includegraphics[width=3.35140in,height=2.03748in]{media/image36.jpeg}

\begin{quote}
\textbf{O patinho feio}

Em uma bonita manhã de outono, a pata Sofia construiu seu ninho de
gravetos perto do lago. Então passou a chocar. E, depois de trinta e
três dias, cinco de seis ovos se quebraram, e os filhotinhos nasceram
--- todos belos e saudáveis.

Sorrindo, a bicharada foi visitar a mamãe e os bebês:

--- Que lindos patinhos, tão amarelinhos, já aprendendo a nadar. Sejam
bem-vindos!

Mas ainda havia um ovo, que não se abria.

--- Será que não vingará? --- os animais se perguntavam.

Preocupada e esperançosa, Sofia continuou a chocar. Enfim a casca
trincou, e nasceu uma avezinha bem diferente, que não tinha a mesma cor
e graciosidade de seus irmãos. A família achava isso estranho:

--- Quá-quá-quá!

--- Aquele patinho é cinzento!

--- Patinho desajeitado!

--- Patinho feio!

O pobrezinho era sempre excluído, sentindo-se triste e solitário. De
tanto sofrer, resolveu fugir.

Nadou durante todo o dia, em busca de um lar que o acolhesse. Já
anoitecendo, o patinho chegou a uma lagoa cheia de marrecos. Ele se
aproximou e tentou se agrupar. Novamente zombaram dele:

--- Você não pertence à nossa família, pato feio, que não sabe
mergulhar!

Rejeitado, o patinho partiu. Não só nadou, como andou muito. Quando
quase se abeirava de um rio, viu um bando de gansos flutuando sobre as
águas.

--- Eles são cinzas e se parecem comigo. Achei a minha família!

Mas os gansos o expulsaram com ruídos estridentes:

--- Não aceitamos estranhos em nosso lar!

No entanto, o patinho desprezado nunca desistia\ldots{} Enquanto procurava,
ia crescendo e se emplumando. Certo dia, encontrou uma grande lagoa,
onde viviam aves de pescoços longos e sinuosos, de plumas alvas, com
elegância inigualável.

Essas aves foram dóceis com o recém-chegado. Então, ele resolveu ficar
todo o inverno, sendo bem cuidado e amado.

No início da primavera, em uma manhã perfumada pelas cerejeiras, o pato
acordou com um grande alvoroço:

--- Que linda plumagem! Quanta beleza!

Sem acreditar nos elogios, ele olhou para o reflexo na água e se deu
conta de que pertencia àquela família. Na verdade, o patinho feio era um
cisne --- o mais bonito de todos!

\fonte{Hans Christian Andersen. O patinho feio. Disponível em:
https://alfabetizacao.mec.gov.br/images/conta-pra-mim/livros/versao\_digital/o\_patinho\_feio\_versao\_digital.pdf.
Acesso em: 16 abr. 2023.}
\end{quote}

\begin{escolha}
\item De que trata o texto?

\reduline{O texto trata de um patinho que acreditava ser feio até
descobrir que era um belo cisne.\hfill}
\linhas{2}

\item Qual era a cor do último patinho a nascer?

\reduline{Ao nascer, o patinho era cinzento.\hfill}
\linhas{2}

\item O fim da história relata uma descoberta do patinho. Que descoberta
foi essa?

\reduline{O patinho feio descobriu que era um lindo cisne.\hfill}
\linhas{2}
\end{escolha}

\num{2} Releia o trecho a seguir:

\begin{quote}
Nadou durante todo o dia, em busca de um lar que o acolhesse. Já
anoitecendo, o patinho chegou a uma lagoa cheia de marrecos. Ele
se aproximou e tentou se agrupar.
\end{quote}

\begin{escolha}
\item Sublinhe a palavra usada para fazer referência ao patinho feio.
\coment{O aluno deve sublinhar o pronome \textit{ele}.}

\item Por que essa palavra foi usada?

\reduline{O pronome foi usado para substituir o nome patinho, de modo 
a evitar a repetição.\hfill}
\end{escolha}
\linhas{2}

\num{3}

\begin{mdframed}[linewidth=10pt,linecolor=salmao!20,backgroundcolor=salmao!20,roundcorner=20pt]
Nadou durante todo o dia, em busca de um lar que o acolhesse
\end{mdframed}

A palavra \textbf{o} aparece duas vezes nesta frase, com funções diferentes.

\begin{escolha}
\item Qual dessas palavras tem a função de determinar o substantivo \textbf{dia}?

\reduline{O primeiro \textbf{o}.\hfill}
\linhas{1}

\item Qual delas foi usada para substituir a palavra patinho?

\reduline{O segundo \textbf{o}.\hfill}
\linhas{1}
\end{escolha}

\num{4} Quando existe muita repetição em um texto, ele fica com pouca
qualidade. Leia o texto a seguir e observe.

\begin{quote}
O \textbf{Patinho Feio}, na verdade, transformou-se em um cisne
belíssimo. O \textbf{Patinho Feio} ficou muito feliz com essa nova
condição. Os amigos do \textbf{Patinho Feio} fizeram uma grande festa
para homenagear o \textbf{Patinho Feio}. \textbf{O Patinho Feio e seus
amigos} se divertiram muito!
\end{quote}

\begin{escolha}
\item Reescreva o trecho e elimine as repetições que não são necessárias,
mantendo a clareza do texto.

\reduline{Sugestão de resposta: O Patinho Feio, na
verdade, transformou-se em um cisne belíssimo. Ele ficou muito feliz com
essa nova condição. Os amigos dele fizeram uma grande festa para
homenageá-lo. Eles se divertiram muito!\hfill}
\linhas{5}

\item Que alterações no texto você fez evitar as repetições?

\reduline{Foram usados pronomes para evitar repetições.\hfill}
\linhas{1}
\end{escolha}

\num{5} Leia o trecho a seguir e responda: Que pronome pode ser usado 
no trecho para evitar a repetição?

\begin{quote}
Eu, minha mãe e meu irmão assistimos ao filme \textit{O Patinho Feio} na
casa da minha madrinha. \textbf{Eu, minha mãe e meu irmão} adoramos passar
esse tempo juntos.
\end{quote}

\reduline{Nós\hfill}
\linhas{1}

\colorsec{Treino}

\num{1} Leia a charadinha a seguir.
%(Fácil) 

\begin{quote}
Você sabe por que a água foi presa?

Porque ela matou a sede.
\end{quote}

A palavra ``ela'' se refere a

\begin{escolha}
\item ``presa''.

\item ``sede''.

\item ``água''.

\item ``matou''.
\end{escolha}

\coment{SAEB: Inferir uma afirmação implícita num texto.

a) Incorreta. Pela coesão do texto, o pronome ``ela'' refere-se a
``água''.

b) Incorreta. Pela coesão do texto, o pronome ``ela'' refere-se a
``água''.

c) Correta. Pela coesão do texto, o pronome ``ela'' refere-se a
``água''.

d) Incorreta. Pela coesão do texto, o pronome ``ela'' refere-se a
``água''.}

\num{2} Leia o trecho de uma carta do leitor sobre uma reportagem
publicada na Revista Planeta.
%(Médio) 

\begin{quote}
A reportagem ``Voluntários sem fronteira'' (edição nº 470) me emocionou
do início ao fim devido à riqueza de detalhes e imagens de Moçambique.
Tenho 16 anos e pretendo ser médica. Agora, mais do que nunca, sei que é
essa a profissão que quero para minha vida, para seguir exemplos de
pessoas como os médicos citados na reportagem. Esses profissionais
sacrificam sua vida para ajudar ao próximo e não buscam reconhecimento
ou algo parecido, e sim a valorização da vida de cidadãos que não
possuem ninguém por eles, mas, mesmo assim, ``transmitem a alegria de
viver''.

\fonte{Revista Planeta. Cartas. Disponível em:
www.revistaplaneta.com.br/cartas-29/.  com adaptações. 
Acesso em: 26 abr. 2023.}
\end{quote}

No trecho ``não possuem ninguém por eles'', a palavra ``eles'' se refere
aos

\begin{escolha}
\item ``médicos''.

\item ``cidadãos''.

\item ``profissionais''.

\item ``detalhes''.
\end{escolha}

\coment{SAEB: Realizar inferências e antecipações em relação ao conteúdo
e à intencionalidade a partir de indicadores como tipo de texto e
características gráficas.

a) Incorreta. Pela coesão do texto, o pronome ``eles'' refere-se a
``cidadãos''.

b) Correta. Pela coesão do texto, o pronome ``eles'' refere-se a
``cidadãos''

c) Incorreta. Pela coesão do texto, o pronome ``eles'' refere-se a
``cidadãos''

d) Incorreta. Pela coesão do texto, o pronome ``eles'' refere-se a
``cidadãos''.}

\num{3} Leia a piadinha a seguir.
%(Difícil) 

\begin{quote}
--- Uma criança vai pela rua com seu avô e encontra um caramelo no chão.
Vai pegá-lo, e seu avô lhe diz: ``não se pega nada do chão''.

Mais adiante a criança encontra uma moeda de um real e seu avô lhe diz:
``não se pega nada do chão''.

Seguem caminhando e seu avô tropeça e cai no chão, e pede ajuda à
criança. E ela lhe diz: ``vovô, não se pega nada do chão''.
\end{quote}

\fonte{Piadas de criança para criança. Guia Infantil. Disponível em:
https://br.guiainfantil.com/piadas-infantis/144-piadas-de-crianca-para-crianca.html.
Acesso em: 26 abr. 2023.}

No trecho ``Vai pegá-lo, e seu avô lhe diz: `não se pega nada do
chão'.'', ``-lo'' está sendo usado no lugar da palavra

\begin{escolha}
\item ``criança''.

\item ``chão''.

\item ``caramelo''.

\item ``avô''.
\end{escolha}

\coment{SAEB: Inferir uma afirmação implícita num texto.

a) Incorreta. Pela coesão do texto, a forma pronominal ``-lo'' refere-se a
``caramelo''.

b) Incorreta. Pela coesão do texto, a forma pronominal ``-lo'' refere-se a
``caramelo''.

c) Correta. Pela coesão do texto, a forma pronominal ``-lo'' refere-se a
``caramelo''.

d) Incorreta. Pela coesão do texto, a forma pronominal ``-lo'' refere-se a
``caramelo''.}

\chapter{Simulado 1}
\markboth{Simulado 1}{}

\num{1} Leia o trecho da notícia.

\begin{quote}
\textbf{Lei proíbe casamento a menores de 16 anos}

Você sabia que o Brasil, em números absolutos, é o quarto país onde mais
ocorrem casamentos infantis? E que 36\% das mulheres daqui se casam
antes de completarem os 18 anos? Isso é o que aponta uma pesquisa do
Banco Mundial, divulgada em 2015. Mas a perspectiva é que essa realidade
mude. Isso porque, em março de 2019, foi aprovada a Lei 13.811/19, que
proíbe o casamento para menores de 16 anos. Isso significa que agora, de
acordo com o texto, ``não será permitido, em qualquer caso, o casamento
de quem não atingiu a idade núbil'', no caso, 16 anos.

\fonte{Plenarinho. Lei proíbe casamento a menores de 16 anos. Disponível
em: https://plenarinho.leg.br/index.php/2019/03/lei-proibe-casamento-menores-de-16-anos.
Acesso em: 26 abr. 2023.}
\end{quote}

As características que dão credibilidade ao texto acima são

\begin{escolha}
\item números e dados apresentados com base em pesquisas.

\item opiniões pessoais sobre pesquisas de opinião.

\item comentários de pesquisadores e especialistas.

\item linguagem subjetiva baseada em opiniões pessoais.
\end{escolha}

\coment{SAEB: Localizar informações num texto.

BNCC: EF35LP03 -- Identificar a ideia central do texto, demonstrando
compreensão global.}

a) Correta. Frases como ``36\% das mulheres daqui se casam
antes de completarem os 18 anos {[}...{]} aponta uma pesquisa do Banco
Mundial, divulgada em 2015'' apresentam números e dados com base em
pesquisas de confiança conferem credibilidade ao texto.

b) Incorreta. O texto analisado não contém trechos com opiniões pessoais.

c) Incorreta. O texto analisado não contém comentários de pesquisadores e especialistas.

d)  Incorreta. A linguagem do texto, embora informal no trecho inicial,
é predominantemente objetiva, baseada em dados.

\num{2} A reportagem ``Seres humanos cresceram 8,6 centímetros nos últimos cem
anos'' apresenta os resultados de um mapeamento sobre altura.

\begin{quote}
\textbf{Seres humanos cresceram 8,6 centímetros nos últimos cem anos}

Segundo uma pesquisa realizada pela revista \textit{E-Life}, os homens
mais altos do mundo são os holandeses (com uma média de 1,83 m) e as
mulheres da Letônia (1,70 m). Os mais baixinhos são do Timor Leste (1,60 m)
e as mulheres da Guatemala (1,50 m). O estudo mapeou a tendência de
crescimento em quase 200 países desde 1914.

Os pesquisadores afirmam que as mudanças nos padrões de crescimento
aconteceram, principalmente, por influências climáticas. Porém, a 
genética e as condições de saúde, saneamento e nutrição também têm sua
contribuição nestas alterações.

\coment{Agência Brasil. Brasileiros cresceram 8,6 centímetros nos últimos cem anos.
Disponível em: www.ebc.com.br/infantil/voce-sabia/2016/07/brasileiros-cresceram-86-centimetros-nos-ultimos-cem-anos. 
Acesso em: 26 abr. 2023.}
\end{quote}

Segundo a reportagem, a média de altura dos homens do Timor Leste é

\begin{escolha}
\item menor que a das mulheres da Letônia.

\item menor que a das mulheres da Guatemala.

\item maior que a dos homens da Holanda.

\item igual à das mulheres da Letônia.
\end{escolha}

\coment{SAEB: Inferir uma afirmação implícita num texto.

BNCC: EF35LP04 -- Inferir informações implícitas nos textos lidos.

a)  Correta. A média de altura dos homens do Timor Leste (1,60 m) é
menor que a média de altura das mulheres da Letônia (1,70 m).

b)  Incorreta. A média de altura das mulheres da Guatemala é 1,50 m, e a
dos homens do Timor Leste é 1,60 m.

c)  Incorreta. A média de altura dos homens da Holanda é 1,83 m, e a dos
homens do Timor Leste é 1,60 m.

d)  Incorreta. A média de altura das mulheres da Letônia é 1,70 m, e a
dos homens do Timor Leste é 1,60 m.}

\num{3} Leia um trecho do poema ``O ramo verde'', de Adelina.

\begin{verse}
\textbf{O ramo verde}

{[}\ldots{}{]}

Foram à tarde a passeio\\
no jardim os dois; Sofia\\
colhia rosas; em meio\\
disse ao irmão: --- que alegria!

Vou dar à mamãe um \textbf{ramo}\\
das minhas amadas flores!
\end{verse}

\fonte{Adelina Lopes Vieira. Domínio Público. O ramo verde. Disponível em:
www.dominiopublico.gov.br/download/texto/wk000077.pdf.
Acesso em: 19 mar. 2023.}

A palavra destacada no texto pode ser substituída, sem perda de sentido,
por:

\begin{escolha}
\item ramalhete.

\item arbusto.

\item grama.

\item espinho.
\end{escolha}

\coment{SAEB: Inferir o sentido de uma palavra ou expressão a partir do
contexto imediato.

BNCC: EF35LP05 -- Inferir o sentido de palavras ou expressões
desconhecidas em textos, com base no contexto da frase ou do texto.

a)  Correta. A palavra ``ramo'' pode ser substituída por ``ramalhete'',
que é um conjunto de flores.

b)  Incorreta. O ``ramo'' é um ramalhete de flores; ``arbusto'' é 
um vegetal baixo, que fica próximo do solo. 

c)  Incorreta. O ``ramo'' é um ramalhete de flores; ``grama'' é 
um vegetal baixo, dos gramados de jardins e parques. 

d)  Incorreta. O ``ramo'' é um ramalhete de flores; o ``espinho'' é
o órgão duro e pontiagudo encontrado em certas plantas.}

\num{4} O poema ``Dom Quixote'' mostra os irmãos Paulo e Mário unindo
forças para atingir um objetivo em comum.

\begin{verse}
\textbf{Dom Quixote}

Paulo tinha seis anos incompletos;\\
tinha só quatro o louro e gentil Mário.\\
Foram à biblioteca, sorrateiros,\\
e ficaram instantes, mudos, quietos,\\
a espreitar se alguém vinha; então, ligeiros\\
como o vento, correram para o armário,\\
que encerrava os volumes cobiçados:\\
eram dois grandes livros encarnados,\\
cheios de formosíssimas gravuras,\\
mas pesados, meu Deus!\\
Os pequeninos\\
porfiavam, cansados, vermelhitos,\\
por tirá-los da estante. Que torturas!\\
{[}...{]}\\
vamos ver à vontade o D. Quixote,\\
sem os ralhos ouvir, impertinentes,\\
da avó, que adormeceu. Oh! que ventura!
\end{verse}

\fonte{Adelina Lopes Vieira. Domínio Público. Dom Quixote. Disponível em:
www.dominiopublico.gov.br/download/texto/wk000074.pdf. 
Acesso em: 26 abr. 2023.}

\begin{tabular}{ll}
\textbf{Glossário} & \mbox{}\\
cobiçados & desejados\\
porfiavam & competiam\\
ralhos & censuras\\
\end{tabular}

A intenção dos irmãos era

\begin{escolha}
\item pegar dois livros da estante sem serem vistos.

\item conseguir entrar na biblioteca sem que a avó os visse.

\item brincar na biblioteca sem que fossem vistos.

\item chamar a atenção da avó entrando na biblioteca.
\end{escolha}

\coment{SAEB: Localizar informações num texto.

BNCC EF15LP03 - Localizar informações explícitas em textos.}

a) Correta. Os irmãos querem pegar secretamente dois livros da estante da
biblioteca, o que fica evidente nos versos ``que encerrava os volumes
cobiçados: / eram dois grandes livros encarnados''. 

b) Incorreta. Os dois irmãos não tiveram dificuldade de entrar na 
biblioteca. Sua intenção era conseguir pegar os dois livros.

c) Incorreta. A intenção dos dois irmãos não era brincar. Eles queriam
pegar, sem serem vistos, dois livros da estante da biblioteca. 

d)  Incorreta. Os irmãos não queriam chamar a atenção da avó, como fica 
evidente nos versos ``sem os ralhos ouvir, impertinentes, / da avó, que
adormeceu''.

\num{5} Leia o texto e responda à pergunta.

\begin{quote}
\textbf{Estado de SP realiza 'Dia V' de mobilização para aplicação de
segunda dose da vacina contra COVID-19}

\textit{Estratégia acontece no sábado (2) e também servirá para os
municípios atualizarem o sistema VaciVida e inserirem imunizados que
eventualmente ainda não foram registrados}

Governo de São Paulo anunciou nesta quarta-feira (29) o ``Dia V'' de
vacinação contra COVID-19. A iniciativa, em parceria com os 645
municípios, acontecerá neste sábado (2) e tem como objetivo a aplicação
da segunda dose da vacina e a atualização dos registros de vacinação na
plataforma VaciVida. {[}\ldots{}{]}

Mais de cinco mil pontos de vacinação no estado estarão abertos das 7h
às 19h para a aplicação exclusivamente destas doses neste sábado
(consulte a programação e horários de funcionamento dos postos de seu
município).
\end{quote}

\fonte{Secretaria da Saúde do Estado de São Paulo. Estado de SP realiza 
Dia V de mobilização para aplicação de segunda dose da vacina contra 
COVID-19.Disponível em:
http://www.saude.sp.gov.br/ses/perfil/cidadao/homepage/destaques/estado-de-sp-realiza-dia-v-de-mobilizacao-para-aplicacao-de-segunda-dose-da-vacina-contra-covid-19.
Acesso em: 4 mai. 2023.}

O texto reproduzido acima trata

\begin{escolha}
  \item do número de vacinados contra COVID-19 no estado de São Paulo.

  \item da aprovação da campanha de vacinação por parte da população.

  \item de uma mobilização para vacinação contra COVID-19.

  \item da resistência da população em tomar o imunizante.
\end{escolha}

\coment{SAEB: Identificar a ideia central o texto.
BNCC: EF35LP03 -- Identificar a ideia central do texto, demonstrando
compreensão global.

a) Incorreta. O texto não faz menção direta ao número de
vacinados.

b) Incorreta. O texto não menciona nenhum juízo de valor da população 
em relação à vacina.

c) Correta. O texto trata da grande mobilização do Governo do Estado
de São Paulo para a segunda dose da vacina.

d) Incorreta. A relação da população com a vacina não é citada no
texto.}

\num{6} Leia o texto e responda à pergunta.

\begin{quote}
\textbf{Ministério da Saúde lança campanha contra malária}

\textit{Ação tem foco na Região Amazônica, com 99\% dos casos no país}

O Ministério da Saúde (MS) lançou hoje (25) uma campanha voltada para a
prevenção e combate à malária. Com o slogan \textit{O combate à malária acontece
com a participação de todos: cidadãos, comunidade e governo}, a campanha
tem como foco a Região Amazônica, que concentra 99\% dos casos no país. A
doença, cuja incidência ocorre nas populações de maior vulnerabilidade
social, representa um grande problema de saúde pública no país. A data
marca o Dia Mundial de Luta Contra a Malária e os 20 anos de atuação do
Programa Nacional de Prevenção e Controle da Malária.
\end{quote}

\fonte{Agência Brasil. Ministério da Saúde lança campanha contra malária.
Disponível em:
https://agenciabrasil.ebc.com.br/saude/noticia/2023-04/ministerio-da-saude-lanca-campanha-contra-malaria.
Acesso em: 4 mai. 2023.}

De acordo com o texto reproduzido acima, a campanha contra a Malária
terá como foco

\begin{escolha}
  \item a região Sul do Brasil

  \item ações realizadas na internet.

  \item o ensino de formas de prevenção à doença.

  \item a região amazônica.

\coment{SAEB: Localizar informação explícita.

BNCC: EF15LP03 -- Localizar informações explícitas em textos.

a) Incorreta. Segundo o texto, ``a campanha tem como foco a Região
Amazônica''.

b) Incorreta. Segundo o texto, ``a campanha tem como foco a Região
Amazônica''.

c) Incorreta. Segundo o texto, ``a campanha tem como foco a Região
Amazônica''.

d) Correta. Segundo o texto, ``a campanha tem como foco a Região
Amazônica''.}

\num{7} Leia o texto e responda à pergunta.

\begin{quote}
D. CECÍLIA (à parte) --- E titia não vem\ldots Que demora!\ldots Não sei
que lhe diga\ldots estou tão vexada\ldots (O Barão tira um livro da algibeira
e folheia-o). Se eu pudesse deixá-lo\ldots É o que vou fazer. (Sobe).

BARÃO (fechando o livro e erguendo-se) --- V. Excia. há de desculpar-me.
Recebi hoje mesmo este livro da Europa; é obra que vai fazer revolução na
ciência; nada menos que uma monografia das gramíneas, premiadas pela 
Academia de Estocolmo.
\end{quote}

\fonte{Machado de Assis. Lição de Botânica. Disponível em:
<https://machado.mec.gov.br/obra-completa-lista/item/download/65_2b8fb9ad43aa37653a58bc2bd56e33aa>.
Acesso em: 23 abr. 2023.}

Tendo em vista as características formais do gênero dramático, quais são
as personagens envolvidas no trecho transcrito acima?

\begin{escolha}

  \item D. Cecília e Barão.

  \item Machado de Assis e D. Cecília.

  \item Barão.

  \item D. Cecília.

\end{escolha} 

\coment{SAEB: Identificar as marcas de organização de textos dramáticos.

BNCC: EF04LP27 -- Identificar, em textos dramáticos, marcadores das
falas das personagens e de cena.

a) Correta. É possível identificar as duas personagens do trecho por
meio dos nomes que lhes precedem as falas.

b) Incorreta. Machado de Assis é o autor da obra.

c) Incorreta. Faltou acrescentar a personagem D. Cecília.

d) Incorreta. Faltou acrescentar a personagem Barão.}

\num{8} Leia o texto e responda à pergunta.

\textbf{O mar}

\begin{verse}
Que nostalgia vem de tuas vagas\\
Ó velho mar, ó lutador Oceano!\\
Tu de saudades íntimas alagas\\
O mais profundo coração humano.

Sim! Do teu choro enorme e soberano,\\
Do teu gemer nas desoladas plagas\\
Sai o que quer que é, rude sultão ufano,\\
Que abre nos peitos verdadeiras chagas.

Ó mar! Ó mar! Embora esse eletrismo,\\
Tu tens em ti o germe do lirismo.\\
És um poeta lírico demais.

E eu para rir com humor das tuas\\
Nevroses colossais, bastam-me as luas\\
Quando fazem luzir os seus metais.
\end{verse}

\fonte{Cruz e Sousa. O Mar. Disponível em:
<https://pt.wikisource.org/wiki/O_Mar_(Cruz_e_Sousa)>. Acesso em 4 mai. 2023.}

O texto acima está organizado em versos e estrofes; logo, trata-se de
um

\begin{escolha}
  \item poema.

  \item romance.

  \item anúncio.

  \item drama.
\begin{escolha}

\coment{SAEB: Reconhecer diferentes modos de organização composicional de 
textos em versos.

BNCC: EF35LP27 -- Ler e compreender, com certa autonomia, textos em
versos, explorando rimas, sons e jogos de palavras, imagens poéticas
(sentidos figurados) e recursos visuais e sonoros.

a) Correta. A poesia é organizada em versos e estrofes.

b) Incorreta. O romance é um gênero narrativo.

c) Incorreta. No anúncio, um produto é divulgado para venda.

d) Incorreta. Um texto dramático é organizado em diálogos.}

\num{9} Leia o texto e responda à pergunta.

\begin{quote}
--- Vou ser atacado? exclamou D. Antônio pensativo.

--- Sim: podes contar.

--- \textbf{E por quem?}

--- Pelo Aimoré.

--- \textbf{E como sabes isto?} --- perguntou D. Antônio fitando nele um olhar
desconfiado.
\end{quote}

\fonte{José de Alencar. O Guarani. Disponível em:
<https://pt.wikisource.org/wiki/O_Guarani/II/X>. Acesso em: 4 mai.
2023.}

O sinal de pontuação utilizado nos trechos destacados indica

\begin{escolha}
  \item o final de uma frase.

  \item uma frase destacada.

  \item uma dúvida.

  \item a divisão de uma frase.
\end{escolha} 

\coment{SAEB: Reconhecer os usos da pontuação.

BNCC: EF04LP05 -- Diferenciar, na leitura de textos, vírgula, 
ponto e vírgula, dois-pontos e reconhecer, na leitura de textos, 
o efeito de sentido que decorre do uso de reticências, aspas, parênteses.

a) Incorreta. Para encerrar uma frase, utiliza-se o ponto-final.

b) Incorreta. Para destacar uma frase, utiliza-se o ponto de exclamação. 

c) Correta. O ponto de interrogação indica uma dúvida.

d) Incorreta. A vírgula não é usada para indicar dúvidas.}

\num{10} Observe a imagem e responda à pergunta.

%Rogério diz: ``não sei colocar imagens aqui''
![a Prefeitura de São José dos Campos realiza durante todo o mês de
junho, uma série de ações para alertar a população sobre o crescimento
do trabalho
infantil](./imgQ4PORT/media/image1.jpeg){width=``5.905555555555556in``
height=``3.9402777777777778in``}

\fonte{Prefeitura de São José dos Campos. São José realiza campanha contra
o trabalho infantil. Disponível em: <https://www.sjc.sp.gov.br/noticias/2021/maio/31/sao-jose-realiza-campanha-contra-o-trabalho-infantil/>.
Acesso em: 4 mai. 2023.}

Na imagem acima, um recurso utilizado para convencer o público é/são

\begin{escolha}
  \item as figuras coloridas.

  \item o uso das palavras ``sim'' e ``não'' em letras maiúsculas.

  \item os semblantes tristes das crianças.

  \item o uso da linguagem formal.
\end{escolha}

\coment{SAEB: Analisar o uso de recursos de persuasão em textos verbais
e/ou multimodais.

a) Incorreta. As cores não são um recurso persuasivo.

b) Correta. O uso de letras maiúsculas destaca a mensagem
transmitida.

c) Incorreta. As crianças estão alegres na imagem.

d) Incorreta. Na imagem, prevalece a linguagem informal.}

\num{11} Leia o texto e responda à pergunta.

\begin{quote}
--- Como vai, bacharel?

--- Menos mal, ignoto viajor.

--- Tomando a fresca, não?

--- \textit{C'est vrai}, como dizem os franceses.

--- Bem, té-logo bacharel, estou meio afobado\...
\end{quote}

\fonte{Mário de Andrade. Macunaíma. Disponível em:
<https://pt.wikisource.org/wiki/Macunaíma/1928/IV>. Acesso em: 4
mai. 2023.}

No texto reproduzido acima, encontramos o uso de uma variante informal
da língua portuguesa na expressão

\begin{escolha}
  \item ignoto viajor 

  \item \textit{C'est vrai}

  \item té logo

  \item Como vai, bacharel?
\end{escolha}

\coment{SAEB: Identificar as variedades linguísticas em textos.

BNCC: EF35LP30 -- Diferenciar discurso indireto e discurso direto,
determinando o efeito de sentido de verbos de enunciação e explicando o
uso de variedades linguísticas no discurso direto, quando for o caso.

a) Incorreta. A expressão ``ignoto viajor'' não pode ser considerada 
informal, pois contém duas palavras pouco comuns no cotidiano.

b) Incorreta. A expressão ``\textit{C'est vrai}'' pertence à língua
francesa.

c) Correta. A expressão ``té logo'' é recorrente no uso cotidiano e 
informal da língua portuguesa.

d) Incorreta. A expressão ``Como vai, bacharel?'' não contém traços
de informalidade.}

\num{12} Leia o texto e responda à pergunta.

\textbf{Bullying: combate deve ser feito de forma coletiva e intersetorial}

Um em cada três alunos em todo o mundo já foi vítima de
\textit{bullying},segundo a Organização das Nações Unidas para Educação,
Ciência e Cultura (Unesco). Esse tipo de violência, ainda comum nas
escolas, \textbf{gera consequências arrasadoras no desempenho dos alunos},
além de sequelas negativas para a saúde física e mental das crianças.
Hoje (7) é comemorado no país o Dia Nacional de Combate ao Bullying e à
Violência nas Escolas.

\fonte{Agência Brasil. Bullying: combate deve ser feito de forma coletiva
e intersetorial. 
Disponível em: https://agenciabrasil.ebc.com.br/educacao/noticia/2023-04/bullying-combate-deve-ser-feito-de-forma-coletiva-e-intersetorial. 
Acesso em: 4 mai. 2023.}

O adjetivo encontrado no trecho destacado indica uma conotação

\begin{escolha}
  \item negativa.

  \item positiva.

  \item indiferente.

  \item duvidosa.
\end{escolha}

\coment{SAEB: Analisar os efeitos de sentido decorrentes do uso dos
adjetivos.
BNCC: EF04LP07 -- Identificar em textos e usar na produção textual a
concordância entre artigo, substantivo e adjetivo (concordância no grupo
nominal).

a) Correta. Levando em consideração o contexto, o adjetivo ``arrasadoras''
tem conotação negativa: o \textit{bullying} é considerado ``violência'',
suas consequências são danosas e suas sequelas são ``negativas''.

b) Incorreta. No texto, as consequências do \textit{bullying} são 
``arrasadoras'', isto é, têm conotação negativa.

c) Incorreta. No texto, as consequências do \textit{bullying} são 
``arrasadoras'', isto é, têm conotação negativa.

d) Incorreta. O trecho demonstra uma visão clara a respeito do
problema. : o \textit{bullying} é considerado ``violência'',
suas consequências são danosas e suas sequelas são ``negativas''.}

\num{13} Leia o texto e responda à pergunta.

\begin{quote}
\textbf{No verão, banhistas devem redobrar os cuidados para evitar
afogamentos}

Tempo quente e férias. Essa é a combinação ideal para que praias e
piscinas fiquem repletas de banhistas. Mas alguns cuidados são
fundamentais para que diversão não coloque a vida das pessoas em risco.
Segundo dados do Corpo de Bombeiros, no ano passado foram registrados
3,1 mil afogamentos no Estado, um aumento de 11,7\% em relação a 2021,
quando ocorreram 2,81 mil casos em SP.

``Neste verão, não abuse da sorte e, na praia, sempre respeite as
orientações do salva-vidas. Lembre-se de que quase metade dos afogados
acreditavam que sabiam nadar'', alerta a coordenadora do Grupo de
Atendimento e Resgate às Urgências (Grau), Cecilia Damasceno.
\end{quote}

\fonte{Secretaria da Saúde do Estado de São Paulo. No verão, banhistas 
devem redobrar os cuidados para evitar afogamentos. Disponível em:
http://www.saude.sp.gov.br/ses/perfil/cidadao/homepage/destaques/no-verao-banhistas-devem-redobrar-os-cuidados-para-evitar-afogamentos>.
Acesso em 4 mai. 2023.}

A citação de uma especialista no assunto tratado no texto torna o
argumento mais

\begin{escolha}
  \item confuso.

  \item restrito.

  \item subjetivo.

  \item confiável.
\end{escolha}

\coment{SAEB: Avaliar a fidedignidade de informações sobre um mesmo fato
veiculadas em diferentes mídias.

BNCC: EF04LP15 -- Distinguir fatos de opiniões/sugestões em textos
(informativos, jornalísticos, publicitários etc.).

a) Incorreta. A citação tem como objetivo conferir precisão e clareza
ao assunto.

b) Incorreta. A citação acrescenta informações de uma especialista 
sobre o tema do texto.

c) Incorreta. A citação torna o tema mais objetivo para o leitor.

d) Correta. O uso de citações provenientes de especialistas
atribui maior confiabilidade ao argumento.}

\num{14} Observe a imagem e responda à pergunta.

%Rogério diz: ``não sei colocar imagens aqui''

![](./imgQ4PORT/media/image2.png){width=``3.966666666666667in``
height=``3.748163823272091in``}

\fonte{Governo do Estado de São Paulo. Entenda por que uso da máscara
ajuda a reduzir risco de contaminação por COVID-19.Disponível em:
<https://www.saopaulo.sp.gov.br/spnoticias/entenda-por-que-uso-da-mascara-ajuda-a-reduzir-risco-de-contaminacao-por-covid-19/>.
Acesso em: 4 mai. 2023.}

De acordo com o infográfico, a máscara de proteção

\begin{escolha}
  \item ajuda a evitar a transmissão do vírus pelo ar.

  \item não deve ser usada por crianças.

  \item protege apenas seu usuário.

  \item deverá ser utilizada somente se o indivíduo estiver doente.
\end{escolha}

\coment{SAEB: Analisar informações apresentadas em gráficos, 
infográficos ou tabelas.
BNCC: EF04LP20 -- Reconhecer a função de gráficos, diagramas e tabelas
em textos, como forma de apresentação de dados e informações.

a) Correta. O infográfico traz textualmente essa informação.

b) Incorreta. O infográfico não menciona a restrição do uso de máscaras 
por crianças.

c) Incorreta. O infográfico menciona que as máscaras auxiliam o
usuário e as pessoas ao seu redor.

d) Incorreta. O infográfico afirma exatamente o contrário.}

\num{15} Leia o texto e responda à pergunta.

\textbf{``Não sou o substituto do Google'', afirma ChatGPT ao ser entrevistado}

Desde que foi apresentado ao mundo, em novembro de 2022, ainda em uma
versão de teste, o ChatGPT vem causando assombro. A reação inicial de
quem vê o sistema de inteligência artificial (criado pela empresa
norte-americana OpenAI em funcionamento) respondendo a uma pergunta
qualquer, pode variar entre o fascínio e o entusiasmo. Em um segundo
momento, contudo, é inevitável pensar nas consequências de tamanho salto
tecnológico. {[}\ldots{}{]}

Por ora, o ChatGPT produz apenas textos sobre praticamente qualquer
assunto, no formato requisitado pelo usuário. Há, no entanto, outros
programas de inteligência artificial capazes de gerar imagens
\textbf{hiper-realistas} a partir de descrições textuais fornecidas pelos
usuários.
\end{quote}

\fonte{Agência Brasil. Não sou o substituto do Google, afirma ChatGPT ao
ser entrevistado. Disponível em: https://agenciabrasil.ebc.com.br/geral/noticia/2023-01/nao-sou-o-substituto-do-google-afirma-chatgpt-ao-ser-entrevistado. com alterações. 
Acesso em: 4 mai. 2023}

O prefixo ``hiper'' destacado no trecho acima indica

\begin{quote}
  \item negação.

  \item intensidade.

  \item diminuição.

  \item oposição.
\end{quote}

\coment{SAEB: Reconhecer em textos o significado de palavras derivadas
a partir de seus afixos.
BNCC: EF35LP05 -- Inferir o sentido de palavras ou expressões
desconhecidas em textos, com base no contexto da frase ou do texto.

a) Incorreta. A ideia de negação está associada ao prefixo ``anti''.

b) Correta. O prefixo ``hiper'' indica uma relação de intensidade.

c) Incorreta. O prefixo ``hiper'' indica uma relação de intensidade, não
de diminuição.

d) Incorreta. A ideia de oposição está associada ao prefixo
``contra''.}

\chapter{Simulado 2}
\markboth{Simulado 2}{}

\num{1} Leia a fábula para responder à questão.

\begin{quote}
\textbf{Os viajantes e o urso}

Dois homens viajavam juntos quando, de repente, surgiu um urso de dentro
da floresta e parou diante deles, urrando. um dos homens tratou de subir
na árvore mais próxima e agarrar-se aos ramos. O outro, vendo que não
tinha tempo para esconder-se, deitou-se no chão, esticado, fingindo de
morto {[}\ldots{}{]}.

\textbf{Na hora do perigo é que se conhece os amigos.}
\end{quote}

\fonte{Ana Rosa Abreu e outros autores. Os viajantes e o urso.
Alfabetização, Vol.2: contos, fábula, lendas e mitos. Disponível em:
www.dominiopublico.gov.br/download/texto/me000589.pdf.
Acesso em: 24 abr. 2023.}

Na leitura da fábula, pode-se compreender que o homem que subiu na árvore

\begin{escolha}
\item abandonou o amigo no perigo, em vez de ajudá-lo.

\item pensou rápido em como ele e seu amigo podiam se proteger.

\item ajudou seu amigo, pois mostrou uma forma de se esconder.

\item aconselhou o amigo a se fingir de morto para se esconder.
\end{escolha}

\coment{SAEB: Identificar o tema central do texto.
BNCC: EF35LP29 -- Identificar, em narrativas, cenário, personagem central,
conflito gerador, resolução e o ponto de vista com base no qual
histórias são narradas, diferenciando narrativas em primeira e terceira
pessoas.

a) Correta. O texto caracteriza o homem que subiu na árvore como um
não confiável, já que abandonou seu amigo durante uma situação de perigo.
É a conclusão a que se pode chegar por meio da leitura da curta narrativa 
e da moral da história, em destaque, ao final.

b) Incorreta. A leitura da curta narrativa e da moral da história, em
destaque, ao final, permite afirmar que o homem que subiu na árvore pensou 
apenas em si mesmo, já que escapou do urso sem ajudar o seu amigo.

c) Incorreta. O homem que escapou do urso não ajudou o amigo; ele apenas
se escondeu, sem tentar pensar naquele que o acompanhava na caminhada.

d) Incorreta. Não há, no texto, menção a aconselhamento de um homem a 
outro.}

\num{2} Leia o trecho de uma carta do leitor coletiva.

\begin{quote}
Olá, CHC! Somo alunos do 5º ano. Lemos durante toda a semana na nossa
roda de leitura curiosidades e histórias da revista CHC. Lemos o texto
``Por que o cérebro nunca deixa de aprender?'' e achamos muito legal.
Gostamos da parte que fala que o cérebro de uma criança armazena
informações muito mais rapidamente que o de um adulto. Por isso, vamos
continuar lendo as matérias publicadas na revista. Assim vamos aprender
muito mais, não acha?

\textbf{Alunos do 5º ano D. Escola Estadual José Ariano Rodrigues. Lins/SP.}
%Paulo: coloquei o trecho acima em itálico para destacar a assinatura dos autores da carta; também seria bom alinhar essa assinatura à esquerda, por favor. 
\end{quote}

\fonte{Ciência Hoje das Crianças. Neurônios em movimento. Disponível em:
http://chc.org.br/artigo/fala-aqui/. Acesso em: 26 abr. 2023.}

Pela leitura da carta do leitor, entende-se que o seu tema central é

\begin{escolha}
\item os elogios dos alunos à revista e ao texto ``Por que o cérebro nunca 
deixa de aprender?''.

\item a pergunta para a revista se eles concordam que ler mais permite que
se aprenda mais também.

\item a roda de leitura que as crianças fazem na escola para discutir
os textos da revista.

\item a frequência com que as crianças leem a revista, sendo que, por
isso, elas aprenderão muito mais.
\end{escolha}

\coment{SAEB: Identificar o tema central do texto.
BNCC: EF04LP10 -- Ler e compreender, com autonomia, cartas pessoais de
reclamação, dentre outros gêneros do campo da vida cotidiana, de acordo
com as convenções do gênero carta e considerando a situação comunicativa
e o tema/assunto/finalidade do texto.

a) Correta. O tema central são os elogios dos alunos à revista e ao
texto``Por que o cérebro nunca deixa de aprender?''. Eles leem a revista 
periodicamente e julgaram que esse artigo específico é ``muito legal''.

b) Incorreta. A pergunta ``Assim vamos aprender muito mais, não acha?''
é quase retórica, isto é: os autores da carta a fizeram sabendo que a
a resposta é positiva. Além disso, essa pergunta contribui para reafirmar
o tema central do texto: o elogio dos alunos à revista e ao artigo nela 
publicado.  

c) Incorreta. A roda de leitura é mencionada apenas para reafirmar
o tema central do texto: o elogio dos alunos à revista e ao artigo nela 
publicado.  

d) Incorreta. A frequência de leitura da revista é mencionada apenas 
para reafirmar o tema central do texto: o elogio dos alunos à revista 
e ao artigo nela publicado.}

\num{3} Leia um trecho do conto tradicional ``João e Maria'', duas crianças muito pobres.

\begin{quote}
\textbf{João e Maria}

Às margens de uma extensa mata existia, há muito tempo, uma cabana
pobre, feita de troncos de árvore, na qual morava um lenhador com sua
segunda esposa e seus dois filhinhos, nascidos do primeiro casamento. 
O garoto chamava-se João e a menina, Maria.
\end{quote}

\fonte{Ana Rosa Abreu e outros autores. João e Maria.
Alfabetização, Vol.2: contos, fábula, lendas e mitos. Disponível em:
www.dominiopublico.gov.br/download/texto/me000589.pdf.
Acesso em: 26 abr. 2023.}

No trecho, os artigos

\begin{escolha}
\item ``um'' e ``uma'' são utilizados para se referir a seres que não estão vivos.

\item ``o'' e ``a'' são utilizados para se referir às personagens principais.

\item femininos são utilizados para se referir a seres que não estão vivos.

\item masculinos são utilizados para todas as personagens da história.
\end{escolha}

\coment{SAEB: Estabelecer relação entre partes de um texto a partir de
mecanismos de concordância verbal e nominal.
BNCC: EF04LP07 -- Identificar em textos e usar na produção textual a
concordância entre artigo, substantivo e adjetivo (concordância no grupo
nominal).

a) Incorreta. O uso de ``um'' para se referir ao lenhador torna essa
alternativa inválida.

b) Correta. O texto refere-se às personagens principais João e Maria
por artigos definidos ``o'' e ``a'', diferente de ``uma extensa mata'',
``uma cabana pobre'' e ``um lenhador'', referidos por artigos indefinidos.

c) Incorreta. A utilização de ``a'' para caracterizar a menina, Maria,
torna essa alternativa inválida.

d) Incorreta. Os artigos masculinos são usados para caracterizar os
personagens masculinos.}

\num{4} Quando não se sabe o significado de uma palavra, muitas vezes, é
possível inferi-lo a partir da análise do contexto em que ela está
inserida.

\begin{quote}
\textbf{Dos 4 aos 6 anos, crianças têm um salto de desenvolvimento em autonomia}

Quando completam 4 anos de idade, as crianças já percorreram um amplo
caminho de desenvolvimento. Estão já bastante independentes e conseguem
fazer algumas atividades simples da rotina praticamente sem ajuda, como
escovar os dentes, escolher as roupas e se vestir. Nessa etapa, é muito
importante incentivar essas conquistas e apoiar sua crescente autonomia.
{[}...{]}

Ao longo dessa etapa, também haverá muitos ganhos \textbf{cognitivos}. A
capacidade de raciocínio aumenta e possibilita que a criança faça
relações mais complexas. Com 6 anos, muitas já conseguem levar em
consideração regras de diferentes situações sociais e diferenciar com
clareza acontecimentos reais daqueles que são faz-de-conta.
\end{quote}

\fonte{Agência Brasil. Dos 4 aos 6 anos, crianças têm um salto de desenvolvimento em autonomia. 
Disponível em:
www.ebc.com.br/infantil/para-pais/2016/10/dos-4-aos-6-anos-criancas-tem-um-salto-de-desenvolvimento-em-autonomia.
Acesso em: 19 mar. 2023.}

Analisando a palavra destacada no contexto em que se insere, 
pode-se afirmar que diz respeito

\begin{escolha}
\item ao ganho de conhecimento.

\item ao desenvolvimento da fala.

\item aos movimentos corporais.

\item às regras de convívio.
\end{escolha}

\coment{SAEB: Inferir o sentido de uma palavra ou expressão a partir do
contexto imediato.
BNCC: EF35LP05 -- Inferir o sentido de palavras ou expressões
desconhecidas em textos, com base no contexto da frase ou do texto.

a) Correta. Pela análise do contexto, pode-se inferir que a palavra 
destacada refere-se ao ganho de conhecimento.

b) Incorreta. Pela análise do contexto, pode-se inferir que a palavra 
destacada refere-se ao ganho de conhecimento, não ao desenvolvimento 
da fala.

c) Incorreta. Não há menção, no texto, a movimentos corporais.

d) Incorreta. Não há menção, no texto, às regras de convívio.}

\num{5} Leia o texto e responda à pergunta.

\begin{quote}
Já se não ria; tinha só um resto de sorriso forçado e resignado. Olhou
bem para ela, e perguntou-lhe o que era.

--- Você promete o que lhe disse?

--- Vá lá. Que foi?

--- Pois saiba que ouvi nada menos que uma declaração de amor.
\end{quote}

\fonte{Machado de Assis. Quincas Borba. Disponível em:
https://machado.mec.gov.br/obra-completa-lista/item/download/14_7bbc6c42393beeac1fd963c16d935f40.
Acesso em: 4 mai. 2023.}

Os travessões empregados no trecho acima cumprem a função de

\begin{escolha}
  \item descrever a aparência das personagens.

  \item introduzir a fala de uma personagem.

  \item concluir um parágrafo da narrativa.

  \item introduzir a descrição do espaço.
\end{escolha}

\coment{SAEB: Identificar elementos constitutivos de textos narrativos.
BNCC: EF35LP26 -- Ler e compreender, com certa autonomia, narrativas
ficcionais que apresentem cenários e personagens, observando os
elementos da estrutura narrativa: enredo, tempo, espaço, personagens,
narrador e a construção do discurso indireto e discurso direto.

a) Incorreta. As descrições de aparência das personagens normalmente 
ocorrem ao longo do texto.

b) Correta. O travessão para introduzir falas de personagens, 
exatamente como ocorre no texto.

c) Incorreta. Parágrafos são concluídos com pontos-finais.

d) Incorreta. As descrições de espaço acontecem ao longo do texto.}

\num{6} Leia o texto e responda à pergunta.

\begin{quote}
Que tal agora desenhar a sua própria amarelinha? Para facilitar, siga o
passo a passo:

1. Você vai precisar de giz e uma pedrinha;

2. Desenhe a sequência de números de 1 a 10 (quadrados) no chão;

3. Vamos às regras: Ninguém pode pisar fora do quadrado; a pedrinha
precisa ficar dentro da marcação e do número correto; só pode colocar um
pé em cada quadrado; quando for um quadrado, pule de um pé só e, quando
tiver dois, é um pé em cada; não pode pisar no quadrado que estiver com a
pedra; deve recolher a pedra na volta.

4. Se possível convide sua família ou amigos para brincar com você.
\end{quote}

\fonte{Secretaria Municipal de Educação de Goiânia. Brincar de 
Amarelinha. Disponível em:
https://sme.goiania.go.gov.br/conexaoescola/ensino_fundamental/brincar-de-amarelinha/.
Acesso em: 4 mai. 2023.}

As instruções para a brincadeira descrita no texto acima são
introduzidas por

\begin{escolha}
  \item verbos no imperativo.

  \item verbos no indicativo.

  \item adjetivos.

  \item substantivos.
\end{escolha}

\coment{ SAEB: Analisar elementos constitutivos de gêneros textuais
diversos.
BNCC: EF04LP05 -- Identificar e reproduzir, em textos injuntivos
instrucionais (instruções de jogos digitais ou impressos), a formatação
própria desses textos (verbos imperativos, indicação de passos a ser
seguidos) e formato específico dos textos orais ou escritos desses
gêneros (lista/ apresentação de materiais e instruções/passos de jogo).

a) Correta. O modo imperativo é usado para transmitir as
instruções.

b) Incorreta. As formas verbais do texto estão flexionadas no modo
imperativo. O modo indicativo não é usado para transmitir
instruções.

c) Incorreta. Adjetivos são usados para qualificar substantivos.

d) Incorreta. Substantivos são usados para se referir a objetos
animados ou inanimados.}

\num{7} Leia o texto e responda a pergunta.

\begin{quote}
\textbf{O velho abriu as pálpebras e cerrou-as logo:}

--- Filha de Araken, escolhe para teu hóspede o presente da volta e
prepara o moquém da viagem. Se o estrangeiro precisa de guia, o
guerreiro Cauby, senhor do caminho, o acompanhará.
\end{quote}

\fonte{José de Alencar. Iracema. Disponível em:
https://pt.wikisource.org/wiki/Iracema. Acesso em: 4 mai. 2023. com
adaptações.}

O sinal de pontuação utilizado no final do trecho destacado cumpre a
função de

\begin{escolha}
  \item concluir uma frase.

  \item enfatizar um dos termos da frase.

  \item marcar uma dúvida da personagem.

  \item indicar o início de um diálogo.
\end{escolha}

\coment{SAEB: Analisar os efeitos de sentido decorrentes do uso da
pontuação.

BNCC: EF04LP05 -- Identificar a função na leitura e usar,
adequadamente, na escrita ponto final, de interrogação, de exclamação,
dois-pontos e travessão em diálogos (discurso direto), vírgula em
enumerações e em separação de vocativo e de aposto.

a) Incorreta. O ponto-final é utilizado para concluir uma frase. Os 
dois-pontos é que servem para introduzir o diálogo.

b) Incorreta. Os dois-pontos da frase destacada servem para introduzir
diálogo.

c) Incorreta. O ponto de interrogação é usado para indicar uma
dúvida. Os dois-pontos é que servem para introduzir o diálogo.

d) Correta. Os dois-pontos servem para introduzir o diálogo.}

\num{8} Leia o texto e responda à pergunta.

\begin{quote}
\textbf{Formação de professores é entrave ao uso de tecnologia em sala de
aula}

Estudo do British Council --- organização internacional do Reino Unido
para relações culturais e oportunidades educacionais --- mostra que a
formação docente é um dos mais graves empecilhos ao uso de tecnologia em
laboratórios ou em sala de aula. Paralelamente a essa questão, as
escolas brasileiras enfrentam problemas de infraestrutura.

Os dados constam do estudo \textit{O ensino de ciências da natureza e suas
tecnologias na educação básica brasileira -- um panorama entre os anos
de 2010 e 2020}, feito em parceria com a Fundação Carlos Chagas e lançado
nesta quarta-feira (12).
\end{quote}

\fonte{Agência Brasil. Formação de professores é entrave ao uso de
tecnologia em sala de aula. Disponível em:
https://agenciabrasil.ebc.com.br/educacao/noticia/2023-04/formacao-de-professores-e-entrave-ao-uso-de-tecnologia-em-sala-de-aula.
Acesso em: 4 mai. 2023.}

Após a leitura do texto, pode-se inferir que

\begin{escolha}
  \item o uso da tecnologia é muito importante no contexto da sala de aula.

  \item o uso da tecnologia deve ser evitado no contexto da sala de aula.

  \item os professores brasileiros sabem utilizar tecnologia em sala de aula.

  \item as escolas brasileiras utilizarão tecnologias britânicas em sala de
aula.
\end{escolha}

\coment{SAEB: Inferir informações implícitas em textos.
BNCC: EF35LP04 -- Inferir informações implícitas nos textos lidos.

a) Correta. O estudo citado no texto avalia as deficiências da formação dos
professores no que se refere à tecnologia. Pode-se inferir, portanto, 
que o uso da tecnologia é muito importante no contexto da sala de aula.   

b) Incorreta. O conjunto do texto permite inferir que o uso da tecnologia 
é muito importante no contexto da sala de aula.

c) Incorreta. De acordo com as afirmações do texto, os professores 
brasileiros \textit{não sabem} utilizar tecnologia em sala de aula.

d) Incorreta. Segundo as afirmações do texto, a pesquisa foi realizada
por uma instituição britânica, sem menção a uso de tecnologia britânica
nas escolas brasileiras.}

\num{9} Leia o texto e responda à pergunta.

\begin{quote}
Si fosse ser água os outros bebiam, si fosse ser formiga esmagavam, si
fosse mosquito flitavam, si fosse trem de ferro descarrilava, si fosse
rio punham no mapa \ldots Resolveu:

``Vou ser Lua''. Gritou:

--- Abram a porta, gente, que quero umas coisas!
\end{quote}

\fonte{Mário de Andrade. Macunaíma. Disponível em:
https://pt.wikisource.org/wiki/Macunaíma/1928. Acesso em: 4 mai.
2023.}

No trecho reproduzido acima, o verbo utilizado para introduzir uma fala
é

\begin{escolha}
  \item ``abram''.

  \item ``gritou''.

  \item ``punham''.

  \item ``fosse''.
\end{escolha}

\coment{SAEB: Analisar os efeitos de sentido de verbos de enunciação.

BNCC: EF35LP30 -- Diferenciar discurso indireto e discurso direto,
determinando o efeito de sentido de verbos de enunciação e explicando o
uso de variedades linguísticas no discurso direto, quando for o caso.

a) Incorreta. O verbo que introduz a fala da personagem é ``gritou''.

b) Correta. O verbo que introduz a fala da personagem é ``gritou''.

c) Incorreta. O verbo que introduz a fala da personagem é ``gritou''.

d) Incorreta. O verbo que introduz a fala da personagem é ``gritou''.}

\num{10} Leio texto e responda à pergunta.

\begin{quote}
Se, em conversa com o ex-presidente de província, disse todo o bem que
pensava do Governo Provisório, não lhe ouviu palavras de acordo nem de
contestação. Não entrou mais fundo na confissão do homem, porque a moça
o atraía, e ele gostava mais \textbf{dela} que do pai.
\end{quote}

\fonte{Machado de Assis. Esaú e Jacó. Disponível em:
https://machado.mec.gov.br/obra-completa-lista/item/download/12_ab2c739d2e8293712078e7b6b0c12abb.
Acesso em: 4 mai. 2023.}

O termo destacado no trecho acima é utilizado para retomar a palavra

\begin{escolha}
  \item ``confissão''.

  \item ``palavras''.

  \item ``moça''.

  \item ``ele''.
\end{escolha}

\coment{SAEB: Identificar os mecanismos de progressão textual.

a) Incorreta. No contexto em que se insere, a expressão ``dela'' retoma 
a palavra ``moça''.

b) Incorreta. No contexto em que se insere, a expressão ``dela'' retoma 
a palavra ``moça''.

c) Correta. No contexto em que se insere, a expressão ``dela'' retoma 
a palavra ``moça''.

d) Incorreta. No contexto em que se insere, a expressão ``dela'' retoma 
a palavra ``moça''.}

\num{11} Leia o texto e responda à pergunta.

\begin{quote}
\textbf{Brasileiros preferem cursos online para qualificação profissional}

Cursos online têm uma vantagem em termos de se adequar mais ao tempo das
pessoas, embora um curso presencial, a interação com os colegas e
professores ao vivo, tenha as suas vantagens. ``Eu diria que um curso
híbrido \textbf{talvez} seja uma solução a esses cursos puramente digitais,
embora reconheça que sejam mais complexos e caros esses cursos que têm
um lado presencial'', afirmou o diretor da FGV Social.
\begin{quote}

\fonte{Agência Brasil. Brasileiros preferem cursos online para
qualificação profissional. Disponível em:
https://agenciabrasil.ebc.com.br/geral/noticia/2023-04/brasileiro-prefere-cursos-online-para-qualificacao-ao-mercado.
Acesso em: 25 abr. 2023.}

O termo destacado no trecho acima indica

\begin{escolha}
  \item lugar.

  \item tempo.

  \item negação.

  \item dúvida.
\end{escolha}

\coment{SAEB: Analisar os efeitos de sentido decorrentes do uso dos
advérbios.

a) Incorreta. O advérbio ``talvez'' expressa a circunstância de dúvida,
não a de lugar.

b) Incorreta. O advérbio ``talvez'' expressa a circunstância de dúvida,
não a de tempo.

c) Incorreta. O advérbio ``talvez'' expressa a circunstância de dúvida,
não a de negação.

d) Correta. O advérbio ``talvez'' expressa a circunstância de dúvida.}

\num{12} Leia o texto e responda à pergunta.

\begin{quote}
\textbf{Centro de Referência em Saúde Indígena é inaugurado em terra
Yanomami}

O território Yanomami em Surucucu recebeu um Centro de Referência em
Saúde Indígena para combater crise humanitária de saúde no local. A
unidade foi inaugurada nessa sexta-feira (21) e é preparada para
atendimentos de urgência, consultas, exames e o tratamento de malária e
desnutrição. Desde o começo do ano, o governo federal mobiliza uma
operação interministerial para o atendimento aos povos dessa região.
\end{quote}

\fonte{Agência Brasil. Centro de Referência em Saúde Indígena é inaugurado
em terra Yanomami. Disponível em: https://agenciabrasil.ebc.com.br/saude/noticia/2023-04/centro-de-referencia-em-saude-indigena-e-inaugurado-em-terra-yanomami. Acesso em: 4 mai. 2023}

O texto acima cumpre a função de informar o leitor a respeito de um
acontecimento; logo, trata-se de um(a)

\begin{escolha}
  \item anúncio.

  \item poema.

  \item notícia.

  \item conto.
\end{escolha}

\coment{SAEB: Reconhecer diferentes gêneros textuais.

BNCC: EF04LP14 -- Identificar, em notícias, fatos, participantes, local
e momento/tempo da ocorrência do fato noticiado.

a) Incorreta. Um anúncio cumpre a função de divulgar um produto ou
evento.

b) Incorreta. Um poema é estruturado em versos e estrofes.

c) Correta. O gênero notícia tem como objetivo principal informar
o leitor.

d) Incorreta. O conto é um texto narrativo ficcional.}

\num{13} Leia o texto e responda à pergunta.

\begin{quote}
\textbf{Governo de SP amplia Campanha de Multivacinação até 30 de novembro}

Devido à baixa procura, o Governo de SP amplia a Campanha de
Multivacinação e até o final de novembro, quando crianças e adolescentes
com menos de 15 anos podem atualizar a caderneta de vacinação. Apenas
34\% do público-alvo, entre crianças e adolescentes de até 15 anos de
idade, procuraram os postos para se vacinar.

``É imprescindível que pais e responsáveis levem seus filhos aos postos
para tomarem os imunizantes e, deste modo, garantirem a saúde e
segurança de suas famílias e da sociedade. As baixas procuras estão
ameaçando o retorno de doenças já erradicadas, é preciso aproveitar a
oportunidade para prevenir e salvar'', destaca Tatiana Lang, do Centro de
Vigilância Epidemiológica da Secretaria de Estado da Saúde.
\end{quote}

\fonte{Governo de São Paulo. Governo de SP amplia Campanha de
Multivacinação até 30 de novembro. Disponível em:
https://www.saopaulo.sp.gov.br/spnoticias/governo-de-sp-amplia-campanha-de-multivacinacao-ate-30-de-novembro/.
Acesso em: 24 abr. 2023.}

O texto reproduzido acima pode ser considerado mais eficiente por
apresentar

\begin{escolha}
  \item perspectivas conflitantes.

  \item a opinião de uma especialista.

  \item uma linguagem complexa.

  \item informações confusas.
\end{escolha}

\coment{SAEB: Julgar a eficácia de argumentos em textos.

a) Incorreta. O texto não apresenta diferentes pontos de vista.

b) Correta. O texto traz a opinião de uma especialista no tema, no segundo 
parágrafo.

c) Incorreta. O texto apresenta uma linguagem objetiva, de fácil
compreensão.

d) Incorreta. O texto traz informações claras.}

\num{14} Leia o texto e responda à pergunta.

\textbf{O muro}

\begin{verse}
Movendo os pés doirados, lentamente,\\
Horas brancas lá vão, de amor e rosas\\
As impalpáveis formas, no ar, cheirosas\\
Sombras, sombras que são da alma doente!
\end{verse}

\fonte{Pedro Kilkerry. O Muro. Disponível em:
https://pt.wikisource.org/wiki/O_Muro.
Acesso em 4 mai. 2023.}

A rima entre as palavras ``rosas'' e ``cheirosas'' encontrada na primeira
estrofe indica uma relação de

\begin{escolha}
  \item oposição.

  \item aproximação.

  \item contraste.

  \item negação.
\end{escolha}

\coment{SAEB: Analisar a construção de sentidos de textos em versos 
com base em seus elementos constitutivos.

BNCC: EF35LP27 -- Ler e compreender, com certa autonomia, textos em
versos, explorando rimas, sons e jogos de palavras, imagens poéticas
(sentidos figurados) e recursos visuais e sonoros.

a) Incorreta. A rima presente no poema faz com o que o substantivo
seja associado ao adjetivo.

b) Correta. A presença da rima faz com as duas palavras sejam
lidas em conjunto.

c) Incorreta. O poema pretende complementar o sentido da primeira
palavra.

d) Incorreta. As duas palavras fazem parte do mesmo campo
semântico.}

\num{15} Observe a imagem e responda à pergunta.

%Rogério diz: ``não sei colocar imagens aqui''
![Foto Noticia Principal
Grande](./imgQ4PORT/media/image3.jpeg){width=``4.175in`` height=``4.175in``}

\begin{quote}
O calor e as chuvas constantes são alguns dos fatores que mais
favorecem a proliferação do \textit{Aedes aegypti}, mosquito transmissor da
dengue, zika e chikungunya. Por isso, o número de casos dessas doenças
vem aumentando a cada dia. Grande parte dos focos estão dentro das
residências e para evitar que o mosquito se prolifere, simples atitudes
como virar para baixo garrafas, colocar telas nas janelas, deixar sempre
com areia os pratos de plantas, manter calhas e vasilhas de animais de
estimação sempre limpas, fazem a diferença no combate do mosquito 
\textit{Aedes aegypti}. Vale ressaltar que o descarte incorreto de 
lixo é uma das principais causas para o acúmulo de larvas dos mosquitos,
por isso, descarte seu lixo no local adequado. Todos juntos contra a
Dengue! Já combateu o mosquito hoje? Faça a sua parte e ajude a mantê-lo
longe em 2022!
\end{quote}

\fonte{Prefeitura Municipal de Pontes Gestal. Campanha conta dengue.
Disponível em: https://www.pontesgestal.sp.gov.br/portal/noticias/0/3/435/campanha-contra-a-dengue.
Acesso em: 4 mai. 2023}

A ilustração reproduzida acima estabelece com o texto uma relação de

\begin{escolha}

  \item contraste.

  \item repetição.

  \item complementação.

  \item negação.

\end{escolha}

\coment{SAEB: Analisar os efeitos de sentido de recursos multissemiótico em
textos que circulam em diferentes suportes.

a) Incorreta. A imagem e o texto seguem a mesma linha argumentativa.

b) Incorreta. A imagem acrescenta informações à argumentação.

c) Correta. A ilustração reforça o argumento, dando destaque à imagem do
mosquito e às frases mais importantes do texto.

d) Incorreta. Texto e imagem complementam um ao outro.}

\chapter{Simulado 3}
\markboth{Simulado 3}{}

\num{1} As instruções de jogos são textos que apresentam o passo a passo
de como realizar o jogo. Se alterarmos algo, as informações e os objetivos
mudam, assim como a forma de jogar. É preciso ler as regras e descobrir
como se joga e, depois, jogar.

Qual é o elemento fundamental nas instruções de jogos?

\begin{escolha}
\item Modo de preparar.

\item Penalizar o perdedor.

\item Definir um vencedor.

\item Imagens das regras definidas.
\end{escolha}

\coment{SAEB: Estabelecer relação entre informações num texto ou entre
diferentes textos.

BNCC: EF04LP09 -- Ler e compreender, com autonomia, boletos, faturas e
carnês, dentre outros gêneros do campo da vida cotidiana, de acordo com
as convenções do gênero (campos, itens elencados, medidas de consumo,
código de barras) e considerando a situação comunicativa e a finalidade
do texto.

a) Incorreta. Não existe ``modo de preparo'' em instruções de jogos,
mas em receitas.

b) Incorreta. Os jogos não preveem punições aos perdedores.

c) Correta. É importante para os jogos saber como estabelecer um
vencedor.

d) Incorreta. Alguns textos instrucionais trazem imagens, mas não há
obrigatoriedade nesse gênero.}

\num{2} Leia o infográfico a seguir.

%\includegraphics[width=5.90556in,height=3.72118in]{media/image37.jpeg}

\fonte{Prefeitura de Santos. Controlar mosquito ajuda a evitar três 
doenças -- confira infográfico. Disponível em:
www.santos.sp.gov.br/?q=content/controlar-mosquito-ajuda-a-evitar-tres-doencas-confira-infografico.
Acesso em: 26 abr. 2023.}

O infográfico descreve

\begin{escolha}
\item as fases da vida do mosquito \emph{Aedes aegypti}, desde o ovo até a vida adulta.

\item a fase adulta da vida do mosquito \emph{Aedes aegypti}, que costuma durar cinco dias.

\item o momento em que o mosquito \emph{Aedes aegypti} começa a transmitir doenças, ainda como larva.

\item os locais necessários para que o mosquito \emph{Aedes aegypti} se reporduza e transmita doenças.
\end{escolha}

\coment{SAEB: Identificar o tema central do texto.

BNCC: EF04LP20 -- Reconhecer a função de gráficos, diagramas e tabelas em
textos, como forma de apresentação de dados e informações.

a) Correta. O infográfico apresenta o ciclo de vida do mosquito, desde o
momento em que os ovos são depositados em criadouros próximos à água até
a fase adulta, quando já pode transmitir as doenças.

b) Incorreta. O infográfico não se restringe à vida adulta do mosquito.  

c)  Incorreta. O infográfico não se restringe ao momento em que o mosquito 
começa a transmitir doenças.

d)  Incorreta. Como se observa no título, o infográfico se refere ao 
ciclo de vida do mosquito, aludindo, de maneira geral, a alguns espaços
em que ele se reproduz e cresce.}

\num{3} Leia o infográfico a seguir.

%\includegraphics[width=4.87500in,height=7.31250in]{media/image38.jpeg}

\fonte{Saúde Brasil. Conheça os fatores que influenciam a alimentação de pessoas com mais de
60 anos. Disponível em:
http://saudebrasil.saude.gov.br/eu-quero-me-alimentar-melhor/conheca-os-fatores-que-influenciam-a-alimentacao-de-pessoas-com-mais-de-60-anos.
Acesso em: 19 mar. 2023.}

%Não encontrei essa página de jeito nenhum

Qual é o tema central do infográfico?

\begin{escolha}
\item As proteínas presentes em alimentos que contêm muito sódio.

\item Os tipos de alimentos frescos adequados para o consumo da população.

\item As doenças causadas pelo consumo de alimentos ultraprocessados.

\item Os alimentos saudáveis e os não saudáveis para os idosos.
\end{escolha}

\coment{SAEB: Identificar o tema central do texto.

BNCC: EF35LP03 -- Identificar a ideia central do texto, demonstrando
compreensão global.

a) Incorreta. O infográfico trata sobre a alimentação dos idosos, não
de proteínas.

b) Incorreta. O infográfico trata sobre a alimentação dos idosos, não
da população em geral.

c) Incorreta. O infográfico não expõe as doenças causadas pelo consumo
de alimentos ultraprocessados.

d) Correta. O infográfico apresenta quais são os alimentos saudáveis e os
não saudáveis para o consumo de idosos.}

\num{4} Leia o poema abaixo para responder à pergunta.

\begin{quote}
\textbf{Meus oito anos}

\begin{verse}
Oh! que saudades que eu tenho\\
Da \textbf{aurora} da minha vida,\\
Da minha infância querida\\
Que os anos não trazem mais! 
\end{verse}

\end{quote}

\fonte{Casimiro de Abreu. Meus oito anos. Disponível em: http://www.pb.utfpr.edu.br/literaturalusofona/Infancia/Casimiro\%20de\%20Abreu/Meus_oito_anos_Casimiro_de_Abreu.pdf. Acesso em: 5 mai.2023.}

No contexto em que está inserida, pode-se afirmar que a palavra destacada
tem o sentido de

\begin{escolha}
  \item início.
  \item fim.
  \item beleza.
  \item tristeza.
\end{escolha} 

\coment{SAEB: Inferir o sentido de palavras ou expressões em textos.

BNCC: EF35LP05 -- Inferir o sentido de palavras ou expressões
desconhecidas em textos, com base no contexto da frase ou do texto.

a) Correta. ``Aurora'' é a luminosidade das primeiras horas do dia; 
no contexto corresponde, portanto, ao início da vida, porque é vocábulo
associado à infância. O poeta sente saudades dos primeiros dias de sua 
vida.  

b) Incorreta. ``Aurora'', no contexto, signfica ``início'', que é 
o oposto do fim. 

c) Incorreta.  ``Aurora'', no contexto, signfica ``início'', que é
diferente da beleza.

d) Incorreta. ``Aurora'', no contexto, signfica ``início'', que não
guarda relação de sentido com a tristeza.}

\num{5} Leia o texto e responda à pergunta.

\begin{quote}
\textbf{Jovens de até 24 anos veem 7 vezes menos TV aberta do que idosos}

Gabriela Borges, coordenadora do Observatório da Qualidade no
Audiovisual e professora da Universidade do Algarve, confirma essa
tendência.

Ela cita outra pesquisa recente, feita em 2022, pela Ofcom, a agência
reguladora britânica. O levantamento mostra que jovens entre 16 e 24
anos assistem quase sete vezes menos televisão do que pessoas com 65
anos ou mais, passando menos de uma hora em frente à TV.

``Eu acho que esse dado também é importante, porque os jovens já não
assistem a televisão, mas eles migram e eles assistem conteúdo
audiovisual nas plataformas de streaming ou conteúdos \textit{on demand}
e em vídeos de redes sociais''.
\end{quote}

\fonte{Agência Brasil. Jovens de até 24 anos veem 7 vezes menos TV 
aberta do que idosos. Disponível em:
https://agenciabrasil.ebc.com.br/geral/noticia/2023-02/jovens-de-ate-24-anos-veem-7-vezes-menos-tv-aberta-do-que-idosos.
Acesso em: 4 mai. 2023}

O trecho entre aspas encontrado ao final do texto reproduzido acima
indica um(a)

\begin{escolha}
  \item opinião.

  \item fato.

  \item narrativa.

  \item verso.
\end{escolha}

\coment{SAEB: Distinguir fatos de opiniões em textos.
BNCC: EF04LP15 -- Distinguir fatos de opiniões/sugestões em textos
(informativos, jornalísticos, publicitários etc.).

a) Correta. O trecho entre aspas contém transcrição da opinião da
especialista.

b) Incorreta. O trecho entre aspas contém transcrição da opinião da
especialista, não a exposição de um fato.

c) Incorreta. O trecho indicado entre aspas não apresenta características 
de narrativa.

d) Incorreta. O trecho entre aspas contém transcrição da opinião da
especialista, não versos de um poema.}

\num{6} Leia o texto e responda à pergunta.

\begin{quote}
\textbf{Telemedicina chegou com a pandemia e veio para ficar, indica estudo}

Uma pesquisa feita com 1.183 médicos dos Estados de São Paulo e do
Maranhão mostrou que os diversos usos da telemedicina -- que despontaram
como alternativa durante a crise sanitária causada pela COVID-19 --
devem permanecer no sistema de saúde brasileiro.

``Os sistemas de saúde, ao se adaptarem às crises -- econômica, política
ou sanitária --, acabam encontrando soluções e alternativas que podem
ser \textbf{transitórias} ou permanentes. Como nosso projeto de pesquisa
estava em andamento quando veio a pandemia, decidimos, a partir do
estudo do trabalho dos médicos, entender mudanças na saúde que possam
ter sido aceleradas pela COVID-19'', explica o pesquisador à Agência
FAPESP.
\end{quote}

\fonte{Governo do Estado de São Paulo. Telemedicina chegou com a
pandemia e veio para ficar, indica estudo. Disponível em:
https://www.saopaulo.sp.gov.br/spnoticias/telemedicina-chegou-com-a-pandemia-e-veio-para-ficar-indica-estudo/.
Acesso em: 24 abr. 2023. com adaptações.}

A partir do contexto, é possível inferir que o termo destacado se
refere a algo

\begin{escolha}
  \item eterno.

  \item fútil.

  \item de boa qualidade.

  \item passageiro.
\end{escolha}

\coment{SAEB: Inferir o sentido de palavras ou expressões em textos.

BNCC: EF35LP05 -- Inferir o sentido de palavras ou expressões
desconhecidas em textos, com base no contexto da frase ou do texto.

a) Incorreta. ``Transitórias'' signfica ``breves''; ``eterno'' é 
exatamente o oposto disso.

b) Incorreta. ``Transitórias'' signfica ``breves''; ``fútil'' é
aquilo que não tem importância, valor ou relevância.

c) Incorreta.  ``Transitórias'' signfica ``breves'', sem relação de
de sentido com ``qualidade''.

d) Correta. ``Transitórias'' signfica ``breves'', ``passageiras''.}

\num{7} Leia o texto e responda à pergunta.

\begin{quote}
\textbf{Lugares e monumentos contam a história do 7 de Setembro em São
Paulo}

Foi às margens do riacho Ipiranga, há 199 anos, em um 7 de setembro como
hoje, que Dom Pedro I (1789-1834) declarou a independência do Brasil em
relação a Portugal. O Brasil então se torna uma monarquia e Dom Pedro I
passa a ser imperador.

Em São Paulo, onde a independência foi declarada, diversos museus e
monumentos ajudam a contar essa história e a entender que esse
\textbf{acontecimento} foi um processo e não se encerrou no momento do
grito.
\end{quote}

\fonte{Agência Brasil. Lugares e monumentos contam a história do 
7 de Setembro em São Paulo. Disponível em:
https://agenciabrasil.ebc.com.br/geral/noticia/2021-09/lugares-historicos-contam-historia-do-7-de-setembro-em-sp.
Acesso em: 4 mai. 2023.}

O termo destacado acima refere-se ao/à

\begin{escolha}
  \item grito do Ipiranga.

  \item cidade de São Paulo.

  \item monumentos.

  \item independência do Brasil.
\end{escolha}

\coment{SAEB: Identificar os mecanismos de referenciação lexical 
e pronominal.

BNCC: EF05LP27 -- Utilizar, ao produzir o texto, recursos de coesão
pronominal (pronomes anafóricos) e articuladores de relações de sentido
(tempo, causa, oposição, conclusão, comparação), com nível adequado de
informatividade.

a) Incorreta. A palavra ``grito'', referente ao grito do Ipiranga, 
aparece no texto após o termo destacado.

b) Incorreta. O contexto em que se insere permite afirmar que o termo
``acontecimento'' se refere à Independência do Brasil, não à cidade
de São Paulo.

c) Incorreta. O contexto em que se insere permite afirmar que o termo
``acontecimento'' se refere à Independência do Brasil, não a 
``monumentos''.

d) Correta. O contexto em que se insere permite afirmar que o termo
``acontecimento'' se refere à Independência do Brasil.}

\num{8} Leia o texto e responda à pergunta.

\begin{quote}
\textbf{Poluição em São Paulo cai 50\% com a quarentena}

Apesar de todos os problemas acarretados com a pandemia do novo
coronavírus em todo o mundo, a imposição do isolamento social para
controlar o avanço da doença ajudou a reduzir os níveis de poluição nas
grandes cidades, como mostra a reportagem de Daniel Antonio à Agência
FAPESP.

\textbf{Isso acontece porque} a principal fonte de emissão de poluentes é a
frota veicular, que durante a quarentena está menos ativa do que o
normal. A avaliação é da professora do Instituto de Astronomia,
Geofísica e Ciências Atmosféricas da Universidade de São Paulo
(IAG-USP), Maria de Fátima Andrade.
\end{quote}

\fonte{Governo do Estado de São Paulo. Poluição em São Paulo cai 50\% com a quarentena. Disponível em:
https://www.saopaulo.sp.gov.br/noticias-coronavirus/poluicao-em-sao-paulo-cai-50-com-a-quarentena/.
Acesso em: 4 mai. 2023. com adaptações.}

O trecho destacado acima apresenta uma causa para

\begin{escolha}
  \item o aumento da poluição na cidade de São Paulo.

  \item a diminuição das taxas de poluição na cidade de São Paulo.

  \item a transmissão do coronavírus na cidade de São Paulo.

  \item o aumento na frota de veículos na cidade de São Paulo.
\end{escolha}

\coment{SAEB: Analisar relações de causa e consequência.

BNCC: EF05LP27 -- Utilizar, ao produzir o texto, recursos de coesão
pronominal (pronomes anafóricos) e articuladores de relações de sentido
(tempo, causa, oposição, conclusão, comparação), com nível adequado de
informatividade.

a) Incorreta. A expressão ``Isso acontece porque'', que inicia o segundo 
parágrafo, se refere às informações do primeiro, isto é: a redução (não o
aumento) da poluição em São Paulo durante a pandemia. 

b) Correta. A expressão ``Isso acontece porque'', que inicia o segundo 
parágrafo, se refere às informações do primeiro, isto é: a redução da 
poluição em São Paulo durante a pandemia.

c) Incorreta. A expressão ``Isso acontece porque'', que inicia o segundo 
parágrafo, se refere às informações do primeiro, isto é: a redução da 
poluição em São Paulo durante a pandemia (não o aumento na frota de
veículos na cidade de São Paulo).

d) Incorreta. A expressão ``Isso acontece porque'', que inicia o segundo 
parágrafo, se refere às informações do primeiro, isto é: a redução da poluição em São Paulo durante a pandemia.}

\num{9} Leia o texto e responda à pergunta.

\begin{quote}
\textbf{Tributo a Guilherme de Almeida revela as várias facetas do 
escritor}

O museu Casa Guilherme de Almeida realiza no mês de julho mais uma
edição do \textit{Guilherme de Almeida em Cena} -- evento que homenageia a obra
do poeta em seu mês de nascimento e falecimento. A atividade, que
acontece de 21 a 24 de julho, tem como objetivo apresentar a pluralidade
do trabalho do escritor e contempla desde sua atuação como cronista e
crítico de cinema até seu interesse pela cultura japonesa,
principalmente pelo haicai -- modelo de poema de origem japonesa que tem
a brevidade como uma de suas características. Neste ano a programação
abre as inscrições para as atividades presenciais e virtuais.

O tributo inicia no dia 21 de julho, quinta-feira, a partir das 19h, com
a palestra ``Crônicas de Guilherme de Almeida'', de Cesar Veneziani,
poeta e mestre em Estudos da Tradução pela FFLCH-USP, e Marlene Laky,
jornalista pela PUC-CAMP, conservadora-restauradora formada pelo SENAI e
coordenadora de oficinas sobre conservação de livros na Casa Guilherme
de Almeida.
\end{quote}

\fonte{Disponível em: Governo do Estado de São Paulo. Tributo a
Guilherme de Almeida revela as várias facetas do escritor
Disponível em: https://www.saopaulo.sp.gov.br/ultimas-noticias/tributo-a-guilherme-de-almeida-revela-as-varias-facetas-do-escritor/.
Acesso em: 24 abr. 2023.}

O texto reproduzido acima fala sobre

\begin{quote}
  \item uma homenagem ao poeta Guilherme de Almeida.

  \item a necessidade de mais eventos literários serem realizados.

  \item os poemas mais importantes do escritor Guilherme de Almeida.

  \item a tradução de poemas no Brasil.
\end{quote}

\coment{SAEB: Identificar a ideia central o texto.
BNCC: EF35LP03 -- Identificar a ideia central do texto, demonstrando
compreensão global.

a) Correta. O texto contém informações sobre o evento \textit{Guilherme de
Almeida em Cena}, que homenageia a obra do poeta em seu mês de nascimento 
e falecimento.

b) Incorreta. No texto, não existe menção à necessidade de mais eventos 
literários serem realizados.

c) Incorreta. No texto, não existe menção a poemas específicos do escritor.

d) Incorreta. No texto, há referência à ocupação de tradutor de Guilherme de Almeida, mas não à tradução de poemas no Brasil.}

\num{10} Leia o texto e responda à pergunta.

\begin{quote}
\textbf{Governo da Bahia reforça combate às fake news via WhatsApp}

O Governo do Estado amplia as ações de combate à desinformação com mais
um canal para denúncias de notícias falsas e conteúdos enganosos que
circulam pela internet. Através do WhatsApp, os usuários podem mandar
mensagens de texto, áudios, imagens ou vídeos para o número (71)
9646-4095. Desde 2020, os baianos e baianas contam também com o site
www.bahiacontraofake.ba.gov.br no enfrentamento às fake news.

As informações enviadas são analisadas pela equipe da Secretaria de
Comunicação Social do Estado (Secom) e encaminhadas para os órgãos
competentes. Após apuração, os esclarecimentos são feitos por meio da
seção ``Fato ou Fake'' do site. A plataforma pioneira do governo baiano
disponibiliza ainda artigos e vídeos que abordam a temática da
desinformação e os efeitos da disseminação de notícias falsas.
\end{quote}

\fonte{Governo do Estado da Bahia. Governo da Bahia reforça combate 
às fake news via WhatsApp. Disponível em:
https://www.bahia.ba.gov.br/2023/01/noticias/comunicacao/governo-da-bahia-reforca-combate-as-fake-news-via-whatsapp/.
Acesso em: 4 mai. 2023.}

De acordo com o texto, uma das ações do governo da Bahia para combater
as \textit{fake news} consiste em

\begin{escolha}

  \item criar secretarias competentes para analisar as notícias.

  \item apreender os aparelhos envolvidos na disseminação de notícias
falsas.

  \item aumentar a fiscalização das redes sociais utilizadas no estado.

  \item analisar as informações e encaminhá-las aos órgãos competentes.
\end{escolha}

\coment{SAEB: Localizar informação explícita.
BNCC: EF35LP03 -- Identificar a ideia central do texto, demonstrando
compreensão global.

a) Incorreta. No texto, há menção à Secretaria de Comunicação Social 
do Estado (Secom), sem alusão à criação de novas secretarias.

b) Incorreta. No texto, não há referência a punições.

c) Incorreta. No texto, não há referência a redes sociais.

d) Correta. A análise de informações e seu encaminhamento aos órgãos
competentes é citada explicitamente no texto.}

\num{11} Leia o texto e responda à pergunta.

\begin{quote}
\textbf{Prêmio literário para mulheres é lançado no Planalto}

O governo federal lançou nesta quarta-feira (5) o Prêmio Carolina Maria
de Jesus de Literatura Produzida por Mulheres 2023. O edital prevê a
seleção de 40 obras escritas exclusivamente por mulheres no valor de R\$
50 mil por agraciada, totalizando R\$ 2 milhões.

Pelo menos 20\% das obras selecionadas deverão ser escritas por mulheres
negras. Cotas para autoras indígenas, quilombolas, ciganas e com
deficiência também estão previstas. Os gêneros literários aceitos serão
\textbf{conto, crônica, romance, quadrinho e roteiro de teatro}. As
inscrições deverão ser realizadas no site da Ministério da Cultura de 12
de abril a 10 de junho.
\end{quote}

\fonte{Agência Brasil. Prêmio literário para mulheres é lançado no 
Planalto. Disponível em:
https://agenciabrasil.ebc.com.br/geral/noticia/2023-04/premio-literario-para-mulheres-e-lancado-no-planalto.
Acesso em: 4 mai. 2023.}

No trecho destacado, a vírgula é utilizada para

\begin{escolha}
  \item destacar uma informação.

  \item pontuar uma dúvida.

  \item enumerar diferentes informações.

  \item concluir uma frase.
\end{escolha}

\coment{SAEB: Reconhecer os usos da pontuação.
BNCC: EF04LP05 -- Identificar a função na leitura e usar,
adequadamente, na escrita ponto final, de interrogação, de exclamação,
dois-pontos e travessão em diálogos (discurso direto), vírgula em
enumerações e em separação de vocativo e de aposto.

a) Incorreta. No trecho em destaque, a vírgula foi usada para separar os 
itens de uma enumeração, não para destacar informações. 

b) Incorreta. O ponto de interrogação é que deve ser usado para marcar
dúvidas. No trecho em destaque, a vírgula foi usada para separar os 
itens de uma enumeração. 

c) Correta. No trecho em destaque, a vírgula foi usada para separar os 
itens de uma enumeração. 

d) Incorreta. O ponto-final é que deve ser usado para concluir uma frase.
No trecho em destaque, a vírgula foi usada para separar os 
itens de uma enumeração. 

\num{12} Leia o texto e responda à pergunta.

\textbf{Flores luxemburguesas}

\begin{verse}
Não é, não é alegria,\\
Nem é tristeza sombria\\
Que sinto me atravessar.\\
Grato, grato sentimento\\
De um passado encantamento ---\\
Por toda parte a lembrar!

Eram as roxas florestas,\\
As sagradas sombras mestas\\
Nossos berços da soidão:\\
Se deles tendes as flores, ---\\
A saudade dos amores\\
Em vós reconheço estão.
\end{verse}

\fonte{Sousândrade. Flores Luxemburguesas. Disponível em:
https://www.literaturabrasileira.ufsc.br/documentos/?action=download&id=43476#02. Acesso em: 4 mai. 2023.}

O poema reproduzido acima apresenta

\begin{escolha}
  \item 12 estrofes.

  \item 6 estrofes.

  \item 1 estrofe.

  \item 2 estrofes.
\end{escolha}

\coment{SAEB: Reconhecer diferentes modos de organização composicional 
de textos em versos.

BNCC: EF35LP27 -- Ler e compreender, com certa autonomia, textos em
versos, explorando rimas, sons e jogos de palavras, imagens poéticas
(sentidos figurados) e recursos visuais e sonoros.

a) Incorreta. O poema apresenta 12 versos divididos em duas
estrofes.

b) Incorreta. O poema apresenta 12 versos divididos em duas
estrofes.

c) Incorreta. O poema apresenta 12 versos divididos em duas
estrofes.

d) Correta. O poema apresenta 12 versos divididos em duas
estrofes.}

\num{13} Observe a imagem e responda à pergunta.

%Rogério diz: ``não sei colocar imagens aqui''
![](./imgQ4PORT/media/image4.png){width=``5.425in``
height=``2.7661373578302713in``}

\fonte{Governo de São Paulo. Conheça alguns dos alimentos típicos 
da primavera Disponível em:
https://www.saopaulo.sp.gov.br/spnoticias/conheca-alguns-dos-alimentos-tipicos-da-primavera/.
Acesso em: 4 mai. 2023.}

De acordo com o infográfico, um dos benefícios da beterraba é

\begin{escolha}
  \item sua ação antioxidante.

  \item sua ação anti-inflamatória.

  \item a presença de vitamina A.

  \item a presença de muitas fibras.
\end{escolha}

\coment{SAEB: Analisar informações apresentadas em gráficos, 
infográficos ou tabelas.

BNCC: EF04LP20 -- Reconhecer a função de gráficos, diagramas e tabelas
em textos, como forma de apresentação de dados e informações.

a) Incorreta. A ação antioxidante é uma característica do abacaxi.

b) Correta. A ação anti-inflamatória é uma característica da beterraba.

c) Incorreta. A presença de vitamina A é uma característica do almeirão.

d) Incorreta. A presença de muitas fibras é uma característica da abóbora paulista.}

\num{14} Leia o texto e responda à pergunta.

\begin{quote}
Camilo pegou-lhe nas mãos, e olhou para ela sério e fixo. Jurou que lhe
queria muito, que os seus sustos pareciam de criança; em todo o caso,
quando tivesse algum receio, a melhor cartomante era ele mesmo. Depois,
repreendeu-a; disse-lhe que era imprudente andar por essas casas. Vilela
podia sabê-lo, e depois \ldots
\end{quote}

\fonte{Machado de Assis. A Cartomante. Disponível em:
https://machado.mec.gov.br/obra-completa-lista/item/download/26_29eaa69154e158508ef8374fcb50937a.
Acesso em: 4 mai. 2023.}

O trecho reproduzido acima apresenta a fala de/do

\begin{escolha}
  \item Machado de Assis.

  \item narrador.

  \item Camilo.

  \item Vilela.
\end{escolha}

\coment{SAEB: Identificar elementos constitutivos de textos narrativos.
BNCC: EF35LP26 -- Ler e compreender, com certa autonomia, narrativas
ficcionais que apresentem cenários e personagens, observando os
elementos da estrutura narrativa: enredo, tempo, espaço, personagens,
narrador e a construção do discurso indireto e discurso direto.

a) Incorreta. Machado de Assis é o autor do conto.

b) Correta. O texto contém um discurso indireto, em que o narrador
descreve o que Camilo está dizendo a uma personagem feminina. 

c) Incorreta. Camilo é uma das personagens do conto. A fala dele é 
descrita pelo narrador, em discurso indireto.

d) Incorreta. Vilela é uma das personagens do conto.}

\num{15} Observe a imagem e responda à pergunta.

%Rogério diz: ``não sei colocar imagens aqui''

![Uma imagem contendo Diagrama Descrição gerada
automaticamente](./imgQ4PORT/media/image5.jpeg){width=``3.658333333333333in``
height=``3.658333333333333in``}

\fonte{Disponível em: Secretaria da Saúde do Governo do Estado da Bahia.
Peças de Campanha – Covid-19. 
https://www.saude.ba.gov.br/temasdesaude/coronavirus/campanhacovid19/>.
Acesso em: 4 mai. 2023.}

De acordo com a ilustração reproduzida acima, podemos inferir que

\begin{escolha}
  \item o uso de máscaras é necessário.

  \item a higiene não combate o vírus.

  \item não é preciso preocupar-se com a infecção.

  \item o uso de máscaras é opcional.
\end{escolha}

\comente{SAEB: Analisar os efeitos de sentido de recursos 
multissemióticos em textos que circulam em diferentes suportes.

a) Correta. A imagem mostra uma mulher utilizando uma máscara.

b) Incorreta. O cartaz afirma exatamente o contrário: medidas de higiene
devem ser adotadas.

c) Incorreta. O cartaz menciona cuidados necessários contra o
vírus.

d) Incorreta. A ilustração mostra exatamente o contrário: é preciso
usar máscaras.}

\chapter{Simulado 4}
\markboth{Simulado 4}{}

\num{1} Leia um trecho do conto a seguir.

\begin{quote}
\textbf{O asno, o boi e o lavrador}

Um lavrador muito rico tinha várias casas no campo onde criava muitos
animais. Vivia em uma delas com a mulher e seus filhos. Ele possuía,
como Salomão, o dom de entender a língua em que falavam os animais,
embora não lhe fosse permitido traduzir o que ouvia às outras pessoas,
sob pena de perder a vida.

Tinha no mesmo curral um boi e um asno e, certo dia, enquanto observava
as brincadeiras de seus filhos, ouviu o boi dizer ao asno:

--- Não posso deixar de invejar tua sorte, ao ver o muito que descansas e
o pouco que trabalhas. Há um empregado que cuida de ti, te dá boa cevada
para comeres e água cristalina para beberes; e se não fossem as poucas
vezes que levas nosso dono nas curtas viagens que faz, passarias a vida
de pernas para o ar. Já a mim, tratam de modo diferente, sendo minha
situação tão desgraçada quanto é agradável a tua. Mal começa o dia, me
prendem a uma carreta, trabalho até não ter mais forças, e o lavrador,
além disso, me espanca sem parar, dando-me depois para comer algumas
favas secas. Vês que tenho razão de invejar tua sorte.
\end{quote}

\fonte{O asno, o boi e o lavrador. \textit{As mil e uma noites: contos
árabes}. Rio de Janeiro: Revan, set. 2010. 5. ed. p. 15-16.}

Na opinião do boi, o asno

\begin{escolha}
\item entende facilmente a língua falada pelo dono.

\item trabalha demais, pois sempre acompanha o dono nas viagens

\item tem uma vida tranquila, por não precisar trabalhar tanto.

\item é muito mais forte porque se alimenta de boa cevada.
\end{escolha}

\coment{SAEB: Inferir uma afirmação implícita num texto.

BNCC: EF35LP26 -- Ler e compreender, com certa autonomia, narrativas
ficcionais que apresentem cenários e personagens, observando os
elementos da estrutura narrativa: enredo, tempo, espaço, personagens,
narrador e a construção do discurso indireto e discurso direto.

a) Incorreta. De acordo com o texto, o dono tem o dom de entender a
língua dos animais. Não há alusão, no texto, quanto à capacidade do asno
de compreender a língua do dono. 

b) Incorreta. Para o boi, o asno trabalha pouco: sua labuta se reduz às 
poucas vezes em que é levado pelo dono em curtas viagens.

c) Correta. O boi inveja o asno, pois acredita que este não
trabalha muito e, por isso, tem uma vida melhor.

d) Incorreta. Não há, no texto, alusão à força do asno.}

\num{2} Leia o trecho do conto ``As mil e uma noites''.

\begin{quote}
Finalmente chegaram a uma região montanhosa, que era
exatamente o lugar onde o falso tio desejava pôr em prática o plano que
o fizera vir de um ponto extremo da África.

--- Já chegamos --- disse ele a Aladin. --- E aqui poderás ver coisas
maravilhosas nunca vistas por nenhum mortal. Agora, junta algumas ervas
secas para acendermos uma fogueira.

Assim que as ervas começaram a queimar, o mágico derramou sobre elas
gotas de um perfume que trazia consigo, fazendo subir da fogueira uma
fumaça espessa, enquanto pronunciava palavras estranhas que Aladin não
compreendia. Nesse momento, a terra estremeceu e se abriu diante dos
dois, deixando ver uma pedra quadrada de mais ou menos 60 centímetros de
lado e 30 centímetros de espessura, tendo ao centro uma argola de metal
que servia para levantá-la. Aladin, assustado com o que acontecia,
tentou fugir, mas foi detido pelo mágico, que o atingiu com um soco no
rosto, tão forte que o derrubou. Aladin, chorando, falou:

--- Tio, que fiz eu para me baterdes deste modo?

--- Tenho minhas razões. Sou teu tio e faço as vezes de teu pai, de modo
que tens de me obedecer. Mas, meu caro sobrinho, se fizerdes exatamente
o que eu mando serás muito bem recompensado.

Essas palavras aliviaram um pouco o temor e o ressentimento de Aladin.
\end{quote}

\fonte{A história de Aladin e a lâmpada
maravilhosa. \emph{As mil e uma noites: contos árabes}. Rio
de Janeiro: Revan, set. 2010. 5. ed. p. 144.}

De acordo com o trecho, o mágico está

\begin{escolha}
\item mostrando-se assustado com as palavras inteligíveis pronunciadas pelo sobrinho.

\item apresentando coisas maravilhosas a Aladin, com a finalidade de aprofundar laços afetivos.

\item ajudando o sobrinho Aladin, ensinando-o a fazer magias na região extrema da África.

\item fingindo ser o tio de Aladin, apenas para o atrair e poder dar início aos seus planos.
\end{escolha}

\coment{SAEB: Estabelecer relação entre informações num texto ou entre
diferentes textos.

BNCC: EF35LP26 -- Ler e compreender, com certa autonomia, narrativas
ficcionais que apresentem cenários e personagens, observando os
elementos da estrutura narrativa: enredo, tempo, espaço, personagens,
narrador e a construção do discurso indireto e discurso direto.

a) Incorreta. Quem se mostra assustado é Aladin, e não o mágico.

b) Incorreta. Não há alusão, no texto, ao proósito de aprofundamento de laços afetivos. 

c) Incorreta. O mágico não ensina magia a Aladin.

d) Correta. O mágico é, na realidade, um falso tio, que usa Aladin para
alcançar seus objetivos. Essa alternativa corresponde rigorosamente às 
afirmações do primeiro parágrafo do texto.}

\num{3} Leia, a seguir, o trecho de uma notícia que aborda o Dia do
Cooperativismo.

\begin{quote}
\textbf{Dia do Cooperativismo é celebrado em várias cidades brasileiras}

{[}...{]} ``O cooperativismo trabalha na linha de frente do
desenvolvimento socioeconômico do país. E o Dia de Cooperar é uma forma
de expressar a força do nosso movimento, que por meio de ações
voluntárias, ajuda pessoas a transformarem suas vidas'', afirma o
presidente do Sindicato e Organização das Cooperativas do Estado do Rio
(OCB/RJ), Vinícius Mesquita.
\end{quote}

\fonte{Vitor Abdala. Agência Brasil. Dia do Cooperativismo é celebrado em várias cidades
brasileiras. Disponível em:
http://agenciabrasil.ebc.com.br/economia/noticia/2019-07/dia-do-cooperativismo-e-celebrado-em-varias-cidades-brasileiras. Acesso em: 26 abr. 2023.}

No trecho, o autor insere a declaração de uma pessoa com a intenção de

\begin{escolha}
\item dar credibilidade à notícia.

\item dar destaque a aspectos importantes.

\item fazer um complemento ao título.

\item ilustrar o que foi relatado.
\end{escolha}

\coment{SAEB: Estabelecer relação entre informações num texto ou entre
diferentes textos.

BNCC: EF35LP16 -- Identificar e reproduzir, em notícias, manchetes, lides
e corpo de notícias simples para público infantil e cartas de reclamação
(revista infantil), digitais ou impressos, a formatação e diagramação
específica de cada um desses gêneros, inclusive em suas versões orais.}

a) Correta. O autor insere a declaração de uma pessoa com a intenção de
dar credibilidade à notícia.

b) Incorreta. O título tem a função de destacar os aspectos mais
importantes do fato relatado.

c) Incorreta. A linha fina complementa as informações apresentadas no
título.

d) Incorreta. Não há imagens ilustrando os fatos.

\num{4} Leia o trecho retirado de uma entrevista com um aluno de 7 anos
de idade sobre futebol do jornal Folha de S. Paulo.

\begin{quote}
Estudante da escola pública estadual Professor Laerte Panighel, o garoto
de 7 anos se revelou esperto desde a primeira visita da reportagem ao
colégio.

\textbf{Por que tem muito corintiano nessa escola?}

Porque o Corinthians tá ganhando de todos os times. O Palmeiras perdeu
de 1 x 0. Então aqui o Corinthians é mais participado, se o
Palmeiras ganhar vai ser todo mundo ser palmeirense.

\textbf{Ah, o time depende de se está ganhando ou não...}

É. É todo mundo assim. Menos eu. Eu sou só corintiano. Nem que perca,
nem que ganhe.
\end{quote}

\fonte{Fábio Victor. Folha de S. Paulo. Não importa para que time você
torce, o que importa é a amizade'', diz Ryan, 7.  Disponível em:
\&lt;http://temas.folha.uol.com.br/crianca-do-dia/esporte/nao-importa-pra-que-time-voce-torce-o-que-importa-e-a-amizade-diz-ryan-7.shtml\&gt;.
Acesso em: 19 mar. 2023.}

Conforme informações do texto, o aluno

\begin{escolha}
\item muda de time a cada partida, torcendo sempre para o vencedor.

\item torce para o mesmo time que a maioria da escola, o Palmeiras.

\item não entende por que seus colegas mudam de time o tempo todo.

\item gosta muito do mesmo time e não muda sua torcida por nada.
\end{escolha}

\coment{SAEB: Localizar informações num texto.

BNCC: EF15LP03 -- Localizar informações explícitas em textos.

a) Incorreta. O aluno é o único da escola que não muda de time

b) Incorreta. No momento da entrevista, os alunos da escola estão
torcendo pelo Corinthians, não pelo Palmeiras.

c) Incorreta. O entrevistado entende o motivo pelo qual os seus amigos 
trocam de time: eles torcem por quem está ganhando.

d) Correta. O entrevistado torce por um único time, o Corinthians, do qual
gosta muito.}

\num{5} Leia o texto e responda à pergunta.

\begin{quote}
\textbf{Mais livros: governo quer retomar políticas públicas para leitura}

Fazer uma nação leitora, este é o desafio do atual governo. Em
entrevista exclusiva para a Agência Brasil, o secretário de Formação,
Livro e Leitura do Ministério da Cultura, Fabiano Piúba, destaca as
ações de retomada das políticas para a área, assim como aponta propostas
da pasta para o novo Programa de Aceleração do Crescimento (PAC). De
acordo com ele, a formação leitora dos brasileiros é uma das prioridades
da gestão.

**Além disso**, existe a expectativa de destacar recursos orçamentários
para o programa de tradução de obras de autores brasileiros, coordenado
pela Fundação Biblioteca Nacional. Dessa forma, a pasta espera
repercutir nossa criação literária em línguas diversas.
\end{quote}

\fonte{Agência Brasil. Mais livros: governo quer retomar políticas 
públicas para leitura. Disponível em:
https://agenciabrasil.ebc.com.br/geral/noticia/2023-04/mais-livros-governo-quer-retomar-politicas-publicas-para-leitura.
Acesso em: 5 mai. 2023. com adaptações.}

A expressão destacada no trecho acima cumpre a função de

\begin{escolha}
  \item concluir as informações apresentadas no texto.

  \item contrastar dados de duas partes diferentes do texto.

  \item indicar que haverá acréscimo às informações anteriores.

  \item negar as informações discutidas no início do texto.
\end{escolha}

\coment{SAEB: Identificar os mecanismos de progressão textual.

a) Incorreta. A expressão indica que haverá acréscimo às informações 
apresentadas anteriormente.

b) Incorreta. A expressão não tem valor de oposição, mas de adição.

c) Correta. A expressão indica que haverá adição de informações às que 
já foram apresentadas no parágrafo anterior.

d) Incorreta. A expressão não indica uma relação de negação com
as informações apresentadas anteriormente.}

\num{6} -- Observe a imagem e responda à pergunta.

%Rogério diz: ``não sei colocar imagens aqui''

![Diagrama Descrição gerada automaticamente com confiança
média](./imgQ4PORT/media/image6.jpeg){width=``4.813232720909887in``
height=``3.433333333333333in``}

\fonte{Prefeitura de Bauru. Prefeitura realiza campanha do Dia Mundial
Contra o Trabalho Infantil. Disponível em: https://www2.bauru.sp.gov.br/materia.aspx?n=34083. Acesso em: 5 mai. 2023.}

Um dos recursos utilizados no cartaz para convencer o leitor da
importância do combate ao trabalho infantil é

\begin{escolha}
  \item a presença de citações de especialistas.

  \item o excesso de textos informativos.

  \item a apresentação de dados estatísticos.

  \item a ilustração de uma garotinha feliz.
\end{escolha}

\coment{SAEB: Analisar o uso de recursos de persuasão em textos 
verbais e/ou multimodais.

a) Incorreta. Não há citações de especialistas no cartaz.

b) Incorreta. Não há textos textos informativos no cartaz.

c) Incorreta. Não são apresentados dados estatísticos no cartaz.

d) Correta. Por meio da ilustração, a felicidade da criança fica associada
ao combate ao trabalho infantil.}

\num{7} Leia o texto e responda à pergunta.

\begin{quote}
Araci tomou o arco e entrou na floresta. A imagem do guerreiro
\textbf{amado} fugia naquele instante de seus olhos; eles buscaram entre as
folhas o sinal de seus passos e não o descobriram.
\end{quote}

\fonte{José de Alencar. Ubirajara. Disponível em:
<http://objdigital.bn.br/Acervo_Digital/Livros_eletronicos/ubirajara.pdf>.
Acesso em: 25 abr. 2023.}

O adjetivo destacado no trecho expressa

\begin{escolha}
  \item carinho.

  \item tristeza.

  \item raiva.

  \item oposição.
\end{escolha}

\coment{SAEB: Analisar os efeitos de sentido decorrentes do uso dos 
adjetivos.

BNCC: EF04LP07 -- Identificar em textos e usar na produção textual a
concordância entre artigo, substantivo e adjetivo (concordância no grupo
nominal).

a) Correta. O adjetivo ``amado'' denota carinho de Araci pelo guerriero.

b) Incorreta. O adjetivo ``amado'' tem conotação positiva.

c) Incorreta. O adjetivo ``amado'' não expressa raiva.

d) Incorreta. O adjetivo ``amado'' não expressa oposição.}

\num{8} Leia o texto e responda à pergunta.

\begin{quote}
---Ingênuo! respondeu Procópio Dias batendo-lhe alegremente no ombro.
Se eu confesso que ela não está muito estragada é porque não a quero
para mim. É grande demais; e depois, fica muito \textbf{longe} da cidade.
Se fosse mais para baixo\ldots
\end{quote}

\fonte{Machado de Assis. Iaiá Garcia. Disponível em:
https://machado.mec.gov.br/obra-completa-lista/item/download/17_bff5cb6c81c213a22c492d69505ac411. Acesso em: 5 mai. 2023.}

O termo destacado no texto é classificado como advérbio de

\begin{escolha}
  \item intensidade.

  \item tempo.

  \item lugar.

  \item modo.
\end{escolha}

\coment{SAEB: Analisar os efeitos de sentido decorrentes do uso dos advérbios.

BNCC: EF04LP07 -- Identificar em textos e usar na produção textual a
concordância entre artigo, substantivo e adjetivo (concordância no grupo
nominal).

a) Incorreta. A circustância expressa por ``longe'' é de lugar, 
não de intensidade.

b) Incorreta. A circustância expressa por ``longe'' é de lugar, 
não de tempo.

c) Correta.  A circustância expressa por ``longe'' é de lugar. 

d) Incorreta. A circustância expressa por ``longe'' é de lugar, 
não de modo.}

\num{9} Leia o texto e responda à pergunta.

\begin{quote}
--- Então, uma hora depois do baile? disse eu alçando a voz.

--- Sim; mas segredo! respondeu Nina levando o dedo à boca.
\end{quote}

\fonte{José de Alencar. Lucíola. Disponível em:
https://pt.wikisource.org/wiki/Lucíola. Acesso em: 5 mai. 2023.}

No trecho reproduzido acima, os dois verbos usados para se referir às
falas das personagens são

\begin{escolha}
  \item ``respondeu'' e ``levando''.

  \item ``disse'' e ``respondeu''.

  \item ``levando'' e ``alçando''.

  \item ``alçando'' e ``disse''.
\end{escolha}

\coment{SAEB: Analisar os efeitos de sentido de verbos de enunciação.

BNCC: EF35LP30 -- Diferenciar discurso indireto e discurso direto,
determinando o efeito de sentido de verbos de enunciação e explicando o
uso de variedades linguísticas no discurso direto, quando for o caso.

a) Incorreta. A forma verbal ``levando'' não se refere a uma das falas.

b) Correta. As formas verbais em questão se referem às falas das
personagens.

c) Incorreta. As formas verbais em questão não se referem ao diálogo do
trecho.

d) Incorreta. A forma verbal  ``alçando'' não se refere a uma das falas.}

\num{10} Observe a imagem e responda à pergunta.

%Rogério diz: ``não sei colocar imagens aqui''
 
![Interface gráfica do usuário Descrição gerada
automaticamente](./imgQ4PORT/media/image7.jpeg){width=``3.425in``
height=``3.420569772528434in``}

\fonte{Prefeitura Municipal de Delta. Vacine-se contra a gripe 
–- Influenza. Disponível em: https://www.delta.mg.gov.br/vacine-se-contra-a-gripe-influenza/.
Acesso em: 5 mai. 2023.}

O fato de a imagem apresentar diversas pessoas no mesmo espaço sugere
que

\begin{escolha}
  \item a vacinação contra gripe é facultativa.

  \item a vacinação não é uma preocupação da saúde pública.

  \item a vacinação é para diversos setores da população.

  \item a vacinação contra a gripe tem um público-alvo restrito.
\end{escolha}

\coment{SAEB: - Analisar os efeitos de sentido de recursos multissemiótico em
textos que circulam em diferentes suportes.

a) Incorreta. A imagem sugere que todos devem se vacinar.

b) Incorreta.  A imagem demonstra a preocupação dos órgãos de saúde
com a vacinação.

c) Correta. A imagem traz diferentes grupos que devem ser vacinados.

d) Incorreta. A imagem sugere exatamente o contrário: diversos grupos
devem ser vacinados.}

\num{11} Leia o texto e responda à pergunta.

\begin{quote}
D. LEOCÁDIA (\textit{pegando-lhe nas mãos}) --- Olhe bem para mim. 
(\textit{Pausa}). Suspire. (\textit{Cavalcante suspira}). O senhor está
doente: não negue que está doente --- moralmente, entenda-se; não negue!
(\textit{Solta-lhe as mãos}).
\end{quote}

\fonte{Machado de Assis. Não Consultes Médico. Disponível em:
https://machado.mec.gov.br/obra-completa-lista/item/download/66_390921fb4791464b4885563dc04a042c.
Acesso em: 5 mai. 2023.}

No trecho acima, as informações entre parênteses apresentam

\begin{escolha}
  \item as falas das personagens.

  \item descrições dos figurinos.

  \item as ações das personagens.

  \item descrições do espaço.
\end{escolha}

\coment{SAEB: Identificar as marcas de organização de textos dramáticos.

BNCC: EF04LP27 -- Identificar, em textos dramáticos, marcadores das
falas das personagens e de cena.

a) Incorreta. Os diálogos surgem ao longo do texto.

b) Incorreta. Não há descrição de figurino nos trechos entre parênteses.

c) Correta. Os trechos entre parênteses contêm as chamadas 
\textit{rubricas}, isto é, indicações do autor da peça referentes 
ao modo de execução de movimentos cênicos, falas ou gestos dos atores
em cena. 

d) Incorreta. Não há descrições do espaço nos trechos entre parênteses.}

\num{12} Leia o texto e responda à pergunta.

\begin{quote}
\textbf{Informação e planejamento são chaves para profissionalizar negócios}

Informação e planejamento são as palavras-chaves para quem decide
empreender, seja por necessidade ou por oportunidade de negócio. Colocar
no papel informações básicas sobre o mercado consumidor e os custos são
essenciais para alinhar as expectativas e garantir a sustentabilidade de
uma empresa no longo prazo, em meio à concorrência.

``Se eu colocar as informações da empresa na internet ou nas redes
sociais, o cliente está nesse canal? Ele vai me enxergar ou eu preciso
fazer algum impresso físico e distribuir para que as pessoas saibam que
estou oferecendo aquele serviço? Ou por meio de algum aplicativo eu
consigo que as pessoas me enxerguem?'', explicou o gerente do Sebrae,
sobre as questões a serem respondidas pelo empreendedor.
\end{quote}

\fonte{Agência Brasil. Informação e planejamento são chaves para profissionalizar negócios. Disponível em:
https://agenciabrasil.ebc.com.br/economia/noticia/2021-03/informacao-e-planejamento-sao-chaves-para-profissionalizar-negocios.
Acesso em: 5 mai. 2023. com adaptações.}

As informações veiculadas no trecho acima podem ser consideradas
confiáveis pelo fato de

\begin{escolha}
  \item utilizarem uma linguagem informal.

  \item serem publicadas na internet.

  \item utilizarem uma linguagem complexa.

  \item contarem com a opinião de um especialista.
\end{escolha}

\coment{SAEB: Julgar a eficácia de argumentos em textos.

a) Incorreta. O registro informal não está associado à credibilidade
do texto.

b) Incorreta. O fato de informações serem publicadas na internet
não as torna mais (ou menos) confiáveis.

c) Incorreta. A complexidade da linguagem não está associada à
qualidade dos argumentos.

d) Correta. A opinião do especialista conferiu credibilidade ao
texto.}

\num{13} Leia o texto e responda à pergunta.

\begin{quote}
--- Os cálculos não são precisos, disse ele, porque o Dr. Bacamarte não
arranja nada. Quem é que viu agora meter todos os doidos dentro da mesma
casa?
\end{quote}

\fonte{Machado de Assis. O Alienista. Disponível em:
https://machado.mec.gov.br/obra-completa-lista/item/download/29_008edfdf58623bb13d27157722a7281e.
Acesso em: 5 mai. 2023.}

O sinal de pontuação utilizado no final do trecho indica

\end{escolha}
  \item uma dúvida.

  \item a fala de uma personagem.

  \item a enumeração de diferentes informações.

  \item um trecho destacado.
\end{escolha}

\coment{SAEB: Analisar os efeitos de sentido decorrentes do uso da pontuação.

BNCC: EF04LP05 -- Identificar a função na leitura e usar,
adequadamente, na escrita ponto final, de interrogação, de exclamação,
dois-pontos e travessão em diálogos (discurso direto), vírgula em
enumerações e em separação de vocativo e de aposto.

a) Correta. O ponto de interrogação é usado para indicar uma
pergunta.

b) Incorreta. Em uma narrativa, utilizamos os dois-pontos para
introduzir uma fala, não para indicar uma pergunta.

c) Incorreta. A vírgula é mais comumente utilizada em uma enumeração, 
não para indicar uma pergunta.

d) Incorreta. O ponto de exclamação é mais comumente usado para 
expressar surpresa, não para indicar uma pergunta.}

\num{14} Leia o texto e responda à pergunta.

\begin{quote}
\textbf{Exposição virtual apresenta registros das transformações 
da cidade de São Paulo}

O Acervo Artístico-Cultural dos Palácios do Governo do Estado de São
Paulo apresenta a exposição virtual ``Na Paisagem de São Paulo: Rebolo e
o Grupo Santa Helena'', com mais de 30 pinturas de Francisco Rebolo e
integrantes do chamado Grupo Santa Helena que apresentam importantes
registros das transformações da cidade de São Paulo e seus arredores nas
décadas de 1930 e 1940.

Com curadoria de Ana Cristina Carvalho e Lisbeth Rebolo Gonçalves, a
mostra destaca a relevância desses artistas artesões na consolidação do
modernismo brasileiro e nos desdobramentos do impacto cultural causado
pelos primeiros modernistas pós-Semana de Arte Moderna de 1922.
\end{quote}

\fonte{Governo de São Paulo. Exposição virtual apresenta registros das 
transformações da cidade de São Paulo. Disponível em:
https://www.saopaulo.sp.gov.br/spnoticias/exposicao-virtual-apresenta-registros-das-transformacoes-da-cidade-de-sao-paulo/.
Acesso em: 5 mai. 2023.}

O tema do texto reproduzido acima é

\begin{escolha}
  \item o processo de urbanização ocorrido em São Paulo na década de 1940.

  \item os novos movimentos artísticos encontrados na cidade de São Paulo.

  \item uma exposição virtual sobre a cidade de São Paulo.

  \item a Semana de Arte Moderna de 1922.
\end{escolha}

\coment{SAEB: Identificar a ideia central o texto.

BNCC: EF35LP03 -- Identificar a ideia central do texto, demonstrando
compreensão global.

a) Incorreta. O texto apenas menciona a década de 1940, o que não é 
suficiente para afirmar que ela é o tema.

b) Incorreta. O texto não faz menção a quaisquer movimentos artísticos.

c) Correta. O assunto do texto é a exposição virtual ``Na Paisagem de 
São Paulo: Rebolo e o Grupo Santa Helena''. 

d) Incorreta. O texto apenas menciona a Semana de Arte Moderna de 1922,
o que não é suficiente para afirmar que ela é o tema.}

\num{15} Leia o texto e responda à pergunta.

\begin{quote}
\textbf{Estudantes surdos poderão ter acesso a vídeo com prova do Enem
traduzida}

Pela primeira vez, estudantes surdos poderão ter acesso a vídeo com as
questões do Enem traduzidas na Língua Brasileira de Sinais (Libras). O
Instituto Nacional de Estudos e Pesquisas Educacionais Anísio Teixeira
(Inep) vai disponibilizar salas adaptadas, e o participante poderá
escolher, na inscrição, se deseja participar da aplicação.

Os estudantes que optarem pela tradução no vídeo terão também acesso a
um tradutor por dupla de candidatos, que poderá apenas esclarecer
dúvidas pontuais de vocabulário. Eles preencherão o cartão de respostas
normalmente. A disponibilização do vídeo será feita este ano em caráter
experimental.
\end{quote}

\fonte{Agência Brasil. Estudantes surdos poderão ter acesso a vídeo com prova do Enem
traduzida. Disponível em:
<https://agenciabrasil.ebc.com.br/educacao/noticia/2017-04/estudantes-surdos-poderao-ter-acesso-video-com-prova-do-enem-traduzida>.
Acesso em: 26 abr. 2023. com adaptações.}

De acordo com o texto

\begin{escolha}
  \item a iniciativa já foi realizada anteriormente.

  \item estudantes surdos não podem realizar a prova do Enem.

  \item será disponibilizado um vídeo para que os estudantes surdos.

  \item não será disponibilizado um tradutor para os candidatos.
\end{escolha}

\coment{SAEB: - Localizar informação explícita.

BNCC: EF15LP03 -- Localizar informações explícitas em textos.

a) Incorreta. O texto menciona o caráter experimental da iniciativa, o que
significa que ela não foi realizada anteriormente.

b) Incorreta. O texto afirma exatamente o contrário: pela primeira vez,
os estudantes surdos terão acesso a um vídeo com a prova do Enem
traduzida.

c) Correta. O texto menciona que os estudantes contarão com um
vídeo para realizar a prova.

d) Incorreta. O texto afirma exatamente o contrário: pela primeira vez,
os estudantes surdos terão acesso a um vídeo com a prova do Enem
traduzida.}

\chapter{Referências}
\markboth{Referências}{}

\begin{bibliohedra}
\tit{brasil}. Ministério da Educação. \textbf{Base nacional comum curricular}:
educação é a base.

\tit{brasil}. Ministério da Educação. \textbf{Pacto nacional pela
alfabetização na idade certa}. Disponível em:
\textless{}http://www.serdigital.com.br/gerenciador/clientes/ceel/material/149.pdf\textgreater{}.
Acesso em: fev. 2023.

\tit{kaufman}, Ana María; RODRÍGUEZ, María Helena. \textbf{Escola, leitura e
produção de textos}. Porto Alegre: Artmed, 1995.

\tit{koch}, Ingedore G. Villaça. \textbf{Ler e escrever}: estratégias de
produção textual. São Paulo: Contexto, 2010.

\tit{lerner}, Delia. \textbf{Ler e escrever na escola}: o real, o possível e o
necessário. Porto Alegre: Artmed, 2002.

\tit{nóbrega}, Maria José. \textbf{Ortografia}. São Paulo: Melhoramentos,
2013.
\end{bibliohedra}

\chapter{Sites}
\markboth{Sites}{}

\begin{itemize}
\item\textbf{Ciência Hoje das Crianças}. Disponível em:
http://chc.org.br/. Acesso em: fev. 2023.

\item\textbf{DOMÍNIO PÙBLICO}. Disponível em:
http://www.dominiopublico.gov.br/pesquisa/PesquisaObraForm.jsp.
Acesso em: 28 fev. 2023.

\item\textbf{RÁDIOS EBC}. Disponível em: https://radios.ebc.com.br/. Acesso
em: 23 fev. 2023.

\item\textbf{PORTAL Teatro na escola}. Disponível em:
https://www.teatronaescola.com/. Acesso em: 28 fev. 2023.
\end{itemize}
