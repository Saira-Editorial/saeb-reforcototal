\chapter{Respostas}
\pagestyle{plain}
\footnotesize

\pagecolor{gray!40}

\colorsec{Módulo 1 – Treino}

\begin{enumerate}
\item
SAEB: Relacionar, na compreensão do texto, informações textuais
com conhecimentos de senso comum.
BNCC: EF04LP08: Reconhecer e grafar, corretamente, palavras derivadas
com os sufixos -agem, -oso, -eza, -izar/-isar (regulares morfológicas).
a) Incorreta. A formação do feminino e do masculino dos adjetivos
pátrios têm terminação -esa e ês.
b) Incorreta. Essas palavras não correspondem às nacionalidades.
c) Incorreta. Essas palavras não correspondem às nacionalidades.
d) Correta. Inglesa e inglês são os adjetivos pátrios corretos.

\item
SAEB: Inferir o sentido de uma palavra ou expressão a partir do
contexto imediato.
BNCC: EF15LP03 -- Localizar informações explícitas em textos.
a) Incorreta. Os tambores são tocados com a intenção de homenagear e
saudar o rei Olofin durante a Festa dos Inhames. Não há alusão, no texto, ao uso desses instrumentos para comemorar a colheita dos inhames.
b) Incorreta. Os tambores são tocados com a intenção de homenagear e
saudar o rei Olofin durante a Festa dos Inhames. Não há alusão, no texto, ao uso desses instrumentos para permitir o início do jantar.
c) Correta. Os tambores são tocados com a intenção de homenagear e
saudar o rei Olofin durante a Festa dos Inhames.
d) Incorreta. A função dos tambores era saudar o rei, não os ministros.

\item
SAEB: Localizar informações num texto.
BNCC: EF35LP03 -- Identificar a ideia central do texto, demonstrando
compreensão global.
a) Incorreta. O objetivo do texto é discorrer sobre o trabalho das
formigas cortadeiras.
b) Correta. A finalidade do texto é informar os leitores sobre o
trabalho das formigas cortadeiras.
c) Incorreta. O texto não expõe modos de cultivar um jardim com formigas.
d) Incorreta. O texto não trata do trabalho de agricultores 
especializados, mas das formigas cortadeiras.
\end{enumerate}

\colorsec{Módulo 2 – Treino}

\begin{enumerate}
\item
SAEB: Inferir o sentido de uma palavra ou expressão a partir do
contexto imediato.
BNCC: EF04LP10 -- Ler e compreender, com autonomia, cartas pessoais de
reclamação, dentre outros gêneros do campo da vida cotidiana, de acordo
com as convenções do gênero carta e considerando a situação comunicativa
e o tema/assunto/finalidade do texto.
a) Incorreta. A assinatura é o nome escrito pelo remetente. O trecho `Até 
breve!'' não é a assinatura de Sandra, mas a expressão de despedida que 
ela usa.
b) Correta. A expressão ``Até breve'' foi usada como forma de despedida 
na carta de Sandra.
c) Incorreta. Saudações servem para abrir a carta, mas `Até breve!'' é a 
expressão de despedida de Sandra.   
d) Incorreta. Não há reclamação na carta de Sandra.

\item
SAEB: Inferir o sentido de uma palavra ou expressão a partir do
contexto imediato.
BNCC: EF04LP10 -- Ler e compreender, com autonomia, cartas pessoais
de reclamação, dentre outros gêneros do campo da vida cotidiana, de
acordo com as convenções do gênero carta e considerando a situação
comunicativa e o tema/assunto/finalidade do texto.
a) Incorreta. A saudação ``Prezada avó'' é mais formal e não se adequaria
ao contexto de correspondência entre avó e neta.
b) Correta. As expressões ``Olá, vovó!'' e ``Querida vovó'' são 
equivalentes na informalidade e no tratamento afetivo. 
c) Incorreta. Somente a expressão ``avó'', no vocativo, não contém
a intimidade da saudação original
d) Incorreta. ``Atenciosamente'' é expressão de despedida, não de saudação.

\item
SAEB: Localizar informações num texto.
BNCC: EF04LP10 -- Ler e compreender, com autonomia, cartas pessoais de
reclamação, dentre outros gêneros do campo da vida cotidiana, de acordo
com as convenções do gênero carta e considerando a situação comunicativa
e o tema/assunto/finalidade do texto.
a) Incorreta. O remetente da carta acima é Almira Lima.
b)  Incorreta. A carta não apresenta uma despedida, e a expressão ``muita 
paz e luz'' está inserida no corpo do texto, sem caracterizar uma 
saudação final. 
c) Incorreta. O título do texto é ``Evolução com consciência''.
d) Correta. A identificação da autora é seguida do local do qual ela
enviou a carta.
\end{enumerate}

\colorsec{Módulo 3 – Treino}

\begin{enumerate}
\item
SAEB: Identificar o tema central do texto.
BNCC: EF04LP13 -- Identificar e reproduzir, em textos injuntivos
instrucionais (instruções de jogos digitais ou impressos), a formatação
própria desses textos (verbos imperativos, indicação de passos a ser
seguidos) e formato específico dos textos orais ou escritos desses
gêneros (lista/ apresentação de materiais e instruções/passos de jogo).
a) Correta. O texto tem as características fundamentais de receita
culinária, especialmente a lista de ingredientes e o modo de fazer o prato.
b) Incorreta. A reportagem é um texto informativo, publicado no jornal, 
que não pode ser caracterizado como instrucional.
c) Incorreta. O texto lido não é um manual de instrução de um jogo, mas
uma receita que ensina a fazer salada de frutas.  
d) Incorreta. O texto lido é uma receita culinária, que ensina a fazer 
salada de frutas. As fábulas são narrativas em animais agem como seres
humanos.

\item
SAEB: Realizar inferências e antecipações em relação ao conteúdo e
à intencionalidade a partir de indicadores como tipo de texto e
características gráficas.
BNCC: EF04LP13 -- Identificar e reproduzir, em textos injuntivos
instrucionais (instruções de jogos digitais ou impressos), a formatação
própria desses textos (verbos imperativos, indicação de passos a ser
seguidos) e formato específico dos textos orais ou escritos desses
gêneros (lista/ apresentação de materiais e instruções/passos de jogo).
a) Incorreta. Os textos não explicam como obter uma bola, e a 
participação não tem restrições de escolha.  
b) Incorreta. Os textos não explicam o funcionamento da bola.
c) Correta. O texto ``Material necessário'' lista a bola como objeto 
necessário para jogar; em ``Modo de jogar'' explica-se como os 
participantes devem proceder.
d) Incorreta. Em ``Modo de jogar'', é explicado apenas um modo de jogar
a brincadeira.

\item
SAEB: Localizar informações num texto.
BNCC: EF03LP16 -- Identificar e reproduzir, em textos injuntivos
instrucionais (receitas, instruções de montagem, digitais ou impressos),
a formatação própria desses textos (verbos imperativos, indicação de
passos a ser seguidos) e a diagramação específica dos textos desses
gêneros (lista de ingredientes ou materiais e instruções de execução --
``modo de fazer'').
a) Correta. A finalidade do jogo é ser o primeiro jogador a fazer 500
pontos, como se verifica em \textbf{Objetivo}.
b) Incorreta. O jogador deve se livrar de todas as cartas. Além disso, em 
em \textbf{Objetivo}, afirma-se explicitamente que a finalidade do jogo
é ser o primeiro jogador a fazer 500 pontos.  
c) Incorreta. O jogador precisa utilizar as cartas de ação para evitar
que os adversários se livrem de todas as suas cartas. Além disso, em 
em \textbf{Objetivo}, afirma-se explicitamente que a finalidade do jogo
é ser o primeiro jogador a fazer 500 pontos. 
d) Incorreta. O jogador precisa ter 500 pontos, o que será calculado
conforme a soma das cartas dos adversários. Além disso, em 
em \textbf{Objetivo}, afirma-se explicitamente que a finalidade do jogo
é ser o primeiro jogador a fazer 500 pontos.
\end{enumerate}

\colorsec{Módulo 4 – Treino}

\begin{enumerate}
\item
SAEB: Identificar o tema central do texto.
BNCC: EF03LP19 -- Identificar e discutir o propósito do uso de recursos de
persuasão (cores, imagens, escolha de palavras, jogo de palavras,
tamanho de letras) em textos publicitários e de propaganda, como
elementos de convencimento.
a) Incorreta. No cartaz, uma gota de água que sai da torneira abriga uma 
família que brinca em um parque de natureza exuberante. Isso quer dizer 
que a qualidade de vida (a felicidade de desfrutar da natureza em família)
depende diretamente do \textit{uso controlado} da água (representado pela
torneira). 
b) Correta. No cartaz, uma gota de água que sai da torneira abriga uma 
família que brinca em um parque de natureza exuberante. Isso quer dizer 
que a qualidade de vida (a felicidade de desfrutar da natureza em família)
depende diretamente do \textit{uso controlado} da água (representado pela
torneira). 
c) Incorreta. Não há elementos no cartaz que permitam afirmar que a água 
seja um recurso ilimitado. Pode-se inferir, aliás, o contrário: a torneira
sugere que o uso da água deve ser controlado, exatamente porque ela é 
recurso limitado. 
d) Incorreta. Não há elementos no cartaz que permitam afirmar que a 
escassez da água já tenha sido solucionada pela ciência.

\item
SAEB: Utilizar informações oferecidas por um glossário, verbete de
dicionário ou texto informativo na compreensão ou interpretação do
texto.
BNCC: EF03LP19 -- Identificar e discutir o propósito do uso de recursos de
persuasão (cores, imagens, escolha de palavras, jogo de palavras,
tamanho de letras) em textos publicitários e de propaganda, como
elementos de convencimento.
a)  Correta. Slogan é uma pequena frase usada para resumir uma campanha.
No caso da campanha de doação de sangue, o slogan é ``doe sangue, salve
vidas''.
b)  Incorreta. ``Divida o amor que corre em suas veias'' é o título da
campanha.
c)  Incorreta. ``Participe da campanha de doação de sangue'' é um subtítulo
que precede as informações sobre onde e quando doar sangue.
d)  Incorreta. Moreira Sales é a instituição responsável pela veiculação
do anúncio.

\item
SAEB: Realizar inferências e antecipações em relação ao conteúdo
e à intencionalidade a partir de indicadores como tipo de texto e
características gráficas.
BNCC: EF03LP19: Identificar e discutir o propósito do uso de recursos de
persuasão (cores, imagens, escolha de palavras, jogo de palavras,
tamanho de letras) em textos publicitários e de propaganda, como
elementos de convencimento.
a) Incorreta. O cartaz contém uma sugestão referente à alimentação 
saudável, mas essa não é sua finalidade. O objetivo é informar sobre 
sintomas e sugerir dicas de prevenção contra gripe.  
b) Incorreta. O cartaz contém uma sugestão referente ao consumo de água,
mas essa não é sua finalidade. O objetivo é informar sobre sintomas e 
sugerir dicas de prevenção contra gripe.
c) Incorreta. Não há informação no cartaz referente ao isolamento social 
das pessoas acometidas de gripe. 
d) Correta. A estrutura do cartaz (título no topo, em letras grandes; 
divisões em cores diferentes, distinguindo ``sintomas'' e ``prevenção'') 
permite afirmar que a campanha tem como finalidade informar ao público 
formas de se prevenir contra a gripe, além de informar a respeito dos
sintomas dessa doença.
\end{enumerate}

\colorsec{Módulo 5 – Treino}

\begin{enumerate}
\item
SAEB: Relacionar, na compreensão do texto, informações textuais com
conhecimentos de senso comum.
BNCC: EF35LP16 -- Identificar e reproduzir, em notícias, manchetes, lides
e corpo de notícias simples para público infantil e cartas de reclamação
(revista infantil), digitais ou impressos, a formatação e diagramação
específica de cada um desses gêneros, inclusive em suas versões orais.
a)  Incorreta. As fábulas são narrativas em que as ações de animais se 
assemelham às dos humanos. O texto apresentado no exercício é uma notícia 
a respeito da invenção de um canudo feito de bambu, que substitui os 
tradicionais canudos de plástico.  
b)  Correta. O texto do exercício é uma notícia, porque sua finalidade é
informativa. Nele, apresenta-se ao leitor os detalhes da invenção de um 
canudo feito de bambu, que substitui os tradicionais canudos de plástico:
quem é o inventor, qual é a invenção, onde, como e por que ela ocorreu.
c)  Incorreta. O poema é um texto escrito em versos. O texto apresentado 
no exercício é escrito em prosa e é uma notícia a respeito da invenção de
um canudo feito de bambu, que substitui os tradicionais canudos de 
plástico.   
d)  Incorreta. O texto apresentado no exercício não é uma narrativa de 
ficção. É uma notícia a respeito da invenção de um canudo feito de bambu,
que substitui os tradicionais canudos de plástico.

\item
SAEB: Inferir uma afirmação implícita num texto.
BNCC: EF35LP16 -- Identificar e reproduzir, em notícias, manchetes, lides
e corpo de notícias simples para público infantil e cartas de reclamação
(revista infantil), digitais ou impressos, a formatação e diagramação
específica de cada um desses gêneros, inclusive em suas versões orais.
a) Incorreta. Os animais descritos na notícia têm formas curiosas de 
defesa, não de ataque. 
b) Incorreta. As plantas descritas na notícia têm formas curiosas de 
defesa, não de provocação aos predadores. 
c) Incorreta. O texto não contém uma comparação entre animais e plantas.  
d) Correta. A finalidade do texto é apresentar animais e plantas que se 
protegem de seus predadores de maneira curiosa.

\item
SAEB: Identificar o tema central do texto.
BNCC: EF35LP16 -- Identificar e reproduzir, em notícias, manchetes, lides
e corpo de notícias simples para público infantil e cartas de reclamação
(revista infantil), digitais ou impressos, a formatação e diagramação
específica de cada um desses gêneros, inclusive em suas versões orais.
a) Incorreta. Na entrevista, um entrevistador faz perguntas a um 
entrevistado. Não há nenhum desses elementos no texto. 
b) Correta. O texto pode ser considerado notícia por conter informações a 
respeito de um fato.
c) Incorreta. O diário é um texto de caráter privado, pessoal, com entradas
normalmente precedidas de data, em que o autor registra impressões pessoais
sobre seu cotidiano, sentimentos e reflexões. Não há nenhum desses
elementos no texto. 
d) Incorreta. O conto é uma narrativa curta ficcional. O texto apresentado 
na questão não é ficcional.
\end{enumerate}

\colorsec{Módulo 6 – Treino}

\begin{enumerate}
\item
SAEB: Relacionar, na compreensão do texto, informações textuais com
conhecimentos de senso comum.
BNCC: EF35LP22 -- Perceber diálogos em textos narrativos, observando o
efeito de sentido de verbos de enunciação e, se for o caso, o uso de
variedades linguísticas no discurso direto.
a) Correta. O trecho apresentado corresponde à fala literal do personagem
denominado como ``velho'', como se vê pela separação feita pelos dois
pontos e travessão.
b) Incorreta. O trecho destacado corresponde à fala literal do personagem; 
trata-se, portanto, de discurso direto. O discurso indireto
pode ser observado no trecho ``Disseram-lhe que estavam aflitos
porque o pai estava gravemente enfermo e os médicos já não tinham
esperanças de o salvar''.
c) Incorreta. O trecho apresentado corresponde à fala literal do personagem
denominado como ``velho'', não do narrador.  
d) Incorreta. O verbo de elocução é traço típico do discurso direto. Ele é
seguido de dois-pontos e travessão, marcas que servem para distinguir o 
discurso do narrador e a fala do personagem.

\item
SAEB: Reconhecer o efeito
de sentido decorrente da escolha de uma determinada palavra ou
expressão.
BNCC: EF35LP30 -- Diferenciar discurso indireto e discurso direto,
determinando o efeito de sentido de verbos de enunciação e explicando o
uso de variedades linguísticas no discurso direto, quando for o caso.
a) Incorreta. Considerando o contexto em que a frase está inserida, 
pode-se afirmar que a bruxa usa a expressão para mostrar surpresa.
b) Correta. Considerando o contexto em que a frase está inserida, 
pode-se afirmar que a bruxa usa a expressão para mostrar surpresa em 
relação à afirmação de Dorothy: a menina achava que todas as bruxas eram
más, e a bruxa se espanta com essa hipótese.
c)  Incorreta. Normalmente, para indicar alívio, são utilizadas
expressões como ``ufa'' e ``ah''. Além disso, considerando o contexto em
que a frase está inserida, pode-se afirmar que a bruxa usa a expressão 
para mostrar surpresa.
d)  Incorreta. Considerando o contexto em que a frase está inserida, 
não há elementos que indiquem que a bruxa ficou \textit{triste} com a 
afirmação de Dorothy.

\item
SAEB D27: Identificar características típicas da fala em um texto
escrito.
BNCC: EF35LP30 -- Diferenciar discurso indireto e discurso direto,
determinando o efeito de sentido de verbos de enunciação e explicando o
uso de variedades linguísticas no discurso direto, quando for o caso.
a)  Incorreta. O texto não contém gírias.
b)  Incorreta. Não há marcas, no texto, da linguagem infantil.
c)  Incorreta. Não há marcas, no texto, da linguagem científica.
d)  Correta. O trecho apresentado corresponde à norma-padrão da língua, 
com alguma formalidade, como se verifica, por exemplo, na regência em
``medo de que'' ou no uso do pronome oblíquo com valor possessivo em 
``o céu nos caia sobre a cabeça''.
\end{enumerate}

\colorsec{Módulo 7 – Treino}

\begin{enumerate}
\item
SAEB: Inferir o sentido de uma palavra ou expressão a partir do
contexto imediato.
BNCC: EF04LP07 -- Identificar em textos e usar na produção textual a
concordância entre artigo, substantivo e adjetivo (concordância no grupo
nominal).
a)  Incorreta. As palavras destacadas são adjetivos.
b)  Correta. Os adjetivos \textit{jovem} e \textit{belo} caracterizam o
pastor; \textit{tímidas} qualifica as flores.
c)  Incorreta. As palavras destacadas são adjetivos.
d)  Incorreta. As palavras destacadas são adjetivos.

\item
SAEB: Utilizar informações oferecidas por um glossário, verbete de
dicionário ou texto informativo na compreensão ou interpretação do
texto.
BNCC: EF04LP07 -- Identificar em textos e usar na produção textual a
concordância entre artigo, substantivo e adjetivo (concordância no grupo
nominal).
a) Incorreta. ``Formosura'' é o substantivo utilizado para nomear a qualidade
de ser formosa.
b) Incorreta. O adjetivo ``valente'' refere-se a Itagibá.
c)  Incorreta. O adjetivo ``forte'' refere-se a Itagibá.
d)  Incorreta. O adjetivo ``bela'' refere-se a Potira.

\item
SAEB: Localizar informações num texto.
BNCC: EF04LP07 -- Identificar em textos e usar na produção textual a
concordância entre artigo, substantivo e adjetivo (concordância no grupo
nominal).
a) Incorreta. No texto, o adjetivo ``venenosas'' concorda com o substantivo ``mariposas''.
b) Incorreta. No texto, o adjetivo ``venenosas'' concorda com o substantivo ``mariposas''.
c) Correta. No texto, o adjetivo ``venenosas'' concorda com o substantivo ``mariposas''.
d) Incorreta. No texto, o adjetivo ``venenosas'' concorda com o substantivo ``mariposas''.
\end{enumerate}

\colorsec{Módulo 8 – Treino}

\begin{enumerate}
\item
SAEB: Estabelecer, no interior de um texto, relação entre um fato e
uma opinião relativa a este fato.
BNCC: EF04LP15 -- Distinguir fatos de opiniões/sugestões em textos
(informativos, jornalísticos, publicitários etc.).
a)  Incorreta. O texto não contém sugestão para o jornal, mas a
opinião do autor da carta.
b)  Correta. No trecho ``É um absurdo a ineficácia das autoridades
responsáveis por esse assunto'', pode-se identificar que o objetivo da
carta é expressar a opinião do leitor sobre o assunto publicado no jornal.
c)  Incorreta. A crítica é direcionada ao governo.
d)  Incorreta. No texto, não há nenhuma pergunta endereçada às autoridades.

\item
SAEB: Estabelecer, no interior de um texto, relação entre um fato e
uma opinião relativa a este fato.
BNCC: EF04LP15 -- Distinguir fatos de opiniões/sugestões em textos
(informativos, jornalísticos, publicitários etc.).
a) Incorreta. As aspas servem para destacar a opinião do CEO da Ipsos. 
Não é possível identificar, no conjunto, a opinião do autor do texto.  
b) Correta. As aspas servem para destacar a opinião do CEO da Ipsos.
c) Incorreta. As aspas servem para destacar a opinião do CEO da Ipsos.
Nesse trecho, ele não explica nem analisa os resultados da pesquisa.
d) Incorreta. As aspas servem para destacar a opinião do CEO da Ipsos, não
a dos entrevistados.

\item
SAEB: Estabelecer, no interior de um texto, relação entre um fato e
uma opinião relativa a este fato.
BNCC: EF04LP15 -- Distinguir fatos de opiniões/sugestões em textos
(informativos, jornalísticos, publicitários etc.).
a) Incorreta. O trecho contém uma opinião da autora sobre o livro.
b) Correta. O trecho destacado nessa alternativa contém a descrição
objetiva de temas do livro feita pelo autor. Trata-se, portanto, da apresentação de 
um fato. 
c) Incorreta. O trecho contém uma opinião da autora sobre o livro.
d) Incorreta. O trecho contém uma opinião da autora sobre o livro.
\end{enumerate}

\colorsec{Módulo 9 – Treino}

\begin{enumerate}
\item
SAEB: Identificar o tema central do texto.
BNCC: EF04LP20 -- Reconhecer a função de gráficos, diagramas e tabelas em
textos, como forma de apresentação de dados e informações.
a) Incorreta. Cada um dos itens do infográfico é caracterizado com um cor
específica. 
b) Incorreta. Cada um dos itens do infográfico contém mais de um tipo de
letra.
c) Incorreta. O infográfico analisado combina imagens e textos, sem dar
preferência a um ou outro.
d) Correta. Cada um dos itens do infográfico é caracterizado com um cor
específica.

\item
SAEB: Estabelecer relação entre informações num texto ou entre
diferentes textos.
BNCC: EF04LP20 -- Reconhecer a função de gráficos, diagramas e tabelas em
textos, como forma de apresentação de dados e informações.
Todos os itens da questão utilizaram dados numéricos contidos no 
infográfico. Para escolher a alternativa correta, contudo, o aluno 
deve selecionar aquela que corresponde especificamente ao que foi
solicitado no enunciado. 
a) Incorreta. 3,5 milhões é o número de alunos matriculados e não se 
refere ao aumento do número de matrículas.
b) Incorreta. 12 mil é o número de alunos estrangeiros matriculados
e não se refere ao aumento do número de matrículas.
c) Correta. Conforme a segunda informação em destaque no infográfico
depois do título, o número de matrículas em 2019 foi 18\% maior que o
registrado no ano anterior. 
d)  Incorreta. 124 é o número de alunos bolivianos matriculados na 
Escola Padre Anchieta e não se refere ao aumento do número de matrículas.

\item
SAEB: Identificar o tema central do texto.
BNCC: EF04LP20 -- Reconhecer a função de gráficos, diagramas e tabelas em
textos, como forma de apresentação de dados e informações.
a)  Incorreta. O infográfico contém informação sobre a população 
brasileira que é insuficientemente ativa, mas o título ``Ser fisicamente
ativo é uma das formas de se proteger do câncer'' permite afirmar que o 
tema central é a atividade física como forma de prevenção ao câncer. 
b)  Correta. O título ``Ser fisicamente ativo é uma das formas de se
proteger do câncer'' permite afirmar que o tema central é a atividade 
física como forma de prevenção ao câncer. O conjunto das informações 
confirma essa afirmação.  
c)  Incorreta. É possível inferir, a partir das informações apresentadas,
que quem é fisicamente inativo tende a ser menos saudável, mas esse não é
o tema do infográfico. 
d)  Incorreta. Há informações sobre a prevenção ao câncer de intestino,
mas o tema do infográfico é mais amplo: a atividade física como forma de
prevenção ao câncer.
\end{enumerate}

\colorsec{Módulo 10 – Treino}

\begin{enumerate}
\item
SAEB: Inferir uma afirmação implícita num texto.
a) Incorreta. Pela coesão do texto, o pronome ``ela'' refere-se a
``água''.
b) Incorreta. Pela coesão do texto, o pronome ``ela'' refere-se a
``água''.
c) Correta. Pela coesão do texto, o pronome ``ela'' refere-se a
``água''.
d) Incorreta. Pela coesão do texto, o pronome ``ela'' refere-se a
``água''.

\item
SAEB: Realizar inferências e antecipações em relação ao conteúdo
e à intencionalidade a partir de indicadores como tipo de texto e
características gráficas.
a) Incorreta. Pela coesão do texto, o pronome ``eles'' refere-se a
``cidadãos''.
b) Correta. Pela coesão do texto, o pronome ``eles'' refere-se a
``cidadãos''
c) Incorreta. Pela coesão do texto, o pronome ``eles'' refere-se a
``cidadãos''
d) Incorreta. Pela coesão do texto, o pronome ``eles'' refere-se a
``cidadãos''.

\item
SAEB: Inferir uma afirmação implícita num texto.
a) Incorreta. Pela coesão do texto, a forma pronominal ``-lo'' refere-se a
``caramelo''.
b) Incorreta. Pela coesão do texto, a forma pronominal ``-lo'' refere-se a
``caramelo''.
c) Correta. Pela coesão do texto, a forma pronominal ``-lo'' refere-se a
``caramelo''.
d) Incorreta. Pela coesão do texto, a forma pronominal ``-lo'' refere-se a
``caramelo''.
\end{enumerate}

\colorsec{Simulado 1}

\begin{enumerate}
\item
SAEB: Localizar informações num texto.
BNCC: EF35LP03 -- Identificar a ideia central do texto, demonstrando
compreensão global.
a) Correta. Frases como ``36\% das mulheres daqui se casam
antes de completarem os 18 anos {[}...{]} aponta uma pesquisa do Banco
Mundial, divulgada em 2015'' apresentam números e dados com base em
pesquisas de confiança conferem credibilidade ao texto.
b) Incorreta. O texto analisado não contém trechos com opiniões pessoais.
c) Incorreta. O texto analisado não contém comentários de pesquisadores e especialistas.
d)  Incorreta. A linguagem do texto, embora informal no trecho inicial,
é predominantemente objetiva, baseada em dados.

\item
SAEB: Inferir uma afirmação implícita num texto. BNCC: EF35LP04 -- Inferir informações implícitas nos textos lidos. a)  Correta. A média de altura dos homens do Timor Leste (1,60 m) é menor que a média de altura das mulheres da Letônia (1,70 m). b)  Incorreta. A média de altura das mulheres da Guatemala é 1,50 m, e a dos homens do Timor Leste é 1,60 m. c)  Incorreta. A média de altura dos homens da Holanda é 1,83 m, e a dos homens do Timor Leste é 1,60 m. d)  Incorreta. A média de altura das mulheres da Letônia é 1,70 m, e a dos homens do Timor Leste é 1,60 m.

\item
SAEB: Inferir o sentido de uma palavra ou expressão a partir do contexto imediato. BNCC: EF35LP05 -- Inferir o sentido de palavras ou expressões desconhecidas em textos, com base no contexto da frase ou do texto. a)  Correta. A palavra ``ramo'' pode ser substituída por ``ramalhete'', que é um conjunto de flores. b)  Incorreta. O ``ramo'' é um ramalhete de flores; ``arbusto'' é um vegetal baixo, que fica próximo do solo. c)  Incorreta. O ``ramo'' é um ramalhete de flores; ``grama'' é um vegetal baixo, dos gramados de jardins e parques. d)  Incorreta. O ``ramo'' é um ramalhete de flores; o ``espinho'' é o órgão duro e pontiagudo encontrado em certas plantas.

\item
SAEB: Localizar informações num texto. BNCC EF15LP03 - Localizar informações explícitas em textos. a) Correta. Os irmãos querem pegar secretamente dois livros da estante da biblioteca, o que fica evidente nos versos ``que encerrava os volumes cobiçados: / eram dois grandes livros encarnados''. b) Incorreta. Os dois irmãos não tiveram dificuldade de entrar na biblioteca. Sua intenção era conseguir pegar os dois livros. c) Incorreta. A intenção dos dois irmãos não era brincar. Eles queriam pegar, sem serem vistos, dois livros da estante da biblioteca. d)  Incorreta. Os irmãos não queriam chamar a atenção da avó, como fica evidente nos versos ``sem os ralhos ouvir, impertinentes, / da avó, que adormeceu''.

\item
SAEB: Identificar a ideia central o texto. BNCC: EF35LP03 -- Identificar a ideia central do texto, demonstrando compreensão global. a) Incorreta. O texto não faz menção direta ao número de vacinados. b) Incorreta. O texto não menciona nenhum juízo de valor da população em relação à vacina. c) Correta. O texto trata da grande mobilização do Governo do Estado de São Paulo para a segunda dose da vacina. d) Incorreta. A relação da população com a vacina não é citada no texto.

\item
SAEB: Localizar informação explícita. BNCC: EF15LP03 -- Localizar informações explícitas em textos. a) Incorreta. Segundo o texto, ``a campanha tem como foco a Região Amazônica''. b) Incorreta. Segundo o texto, ``a campanha tem como foco a Região Amazônica''. c) Incorreta. Segundo o texto, ``a campanha tem como foco a Região Amazônica''. d) Correta. Segundo o texto, ``a campanha tem como foco a Região Amazônica''.

\item
SAEB: Identificar as marcas de organização de textos dramáticos. BNCC: EF04LP27 -- Identificar, em textos dramáticos, marcadores das falas das personagens e de cena. a) Correta. É possível identificar as duas personagens do trecho por meio dos nomes que lhes precedem as falas. b) Incorreta. Machado de Assis é o autor da obra. c) Incorreta. Faltou acrescentar a personagem D. Cecília. d) Incorreta. Faltou acrescentar a personagem Barão.

\item
SAEB: Reconhecer diferentes modos de organização composicional de textos em versos. BNCC: EF35LP27 -- Ler e compreender, com certa autonomia, textos em versos, explorando rimas, sons e jogos de palavras, imagens poéticas (sentidos figurados) e recursos visuais e sonoros. a) Correta. A poesia é organizada em versos e estrofes. b) Incorreta. O romance é um gênero narrativo. c) Incorreta. No anúncio, um produto é divulgado para venda. d) Incorreta. Um texto dramático é organizado em diálogos.

\item
SAEB: Reconhecer os usos da pontuação. BNCC: EF04LP05 -- Diferenciar, na leitura de textos, vírgula, ponto e vírgula, dois-pontos e reconhecer, na leitura de textos, o efeito de sentido que decorre do uso de reticências, aspas, parênteses. a) Incorreta. Para encerrar uma frase, utiliza-se o ponto-final. b) Incorreta. Para destacar uma frase, utiliza-se o ponto de exclamação. c) Correta. O ponto de interrogação indica uma dúvida. d) Incorreta. A vírgula não é usada para indicar dúvidas.

\item
SAEB: Analisar o uso de recursos de persuasão em textos verbais e/ou multimodais. a) Incorreta. As cores não são um recurso persuasivo. b) Correta. O uso de letras maiúsculas destaca a mensagem transmitida. c) Incorreta. As crianças estão alegres na imagem. d) Incorreta. Na imagem, prevalece a linguagem informal.

\item
SAEB: Identificar as variedades linguísticas em textos. BNCC: EF35LP30 -- Diferenciar discurso indireto e discurso direto, determinando o efeito de sentido de verbos de enunciação e explicando o uso de variedades linguísticas no discurso direto, quando for o caso. a) Incorreta. A expressão ``ignoto viajor'' não pode ser considerada informal, pois contém duas palavras pouco comuns no cotidiano. b) Incorreta. A expressão ``\textit{C'est vrai}'' pertence à língua francesa. c) Correta. A expressão ``té logo'' é recorrente no uso cotidiano e informal da língua portuguesa. d) Incorreta. A expressão ``Como vai, bacharel?'' não contém traços de informalidade.

\item
SAEB: Analisar os efeitos de sentido decorrentes do uso dos adjetivos. BNCC: EF04LP07 -- Identificar em textos e usar na produção textual a concordância entre artigo, substantivo e adjetivo (concordância no grupo nominal). a) Correta. Levando em consideração o contexto, o adjetivo ``arrasadoras''tem conotação negativa: o \textit{bullying} é considerado ``violência'', suas consequências são danosas e suas sequelas são ``negativas''. b) Incorreta. No texto, as consequências do \textit{bullying} são ``arrasadoras'', isto é, têm conotação negativa. c) Incorreta. No texto, as consequências do \textit{bullying} são ``arrasadoras'', isto é, têm conotação negativa. d) Incorreta. O trecho demonstra uma visão clara a respeito do problema. : o \textit{bullying} é considerado ``violência'', suas consequências são danosas e suas sequelas são ``negativas''.

\item
SAEB: Avaliar a fidedignidade de informações sobre um mesmo fato veiculadas em diferentes mídias. BNCC: EF04LP15 -- Distinguir fatos de opiniões/sugestões em textos (informativos, jornalísticos, publicitários etc.). a) Incorreta. A citação tem como objetivo conferir precisão e clareza ao assunto. b) Incorreta. A citação acrescenta informações de uma especialista sobre o tema do texto. c) Incorreta. A citação torna o tema mais objetivo para o leitor. d) Correta. O uso de citações provenientes de especialistas atribui maior confiabilidade ao argumento.

\item
SAEB: Analisar informações apresentadas em gráficos, infográficos ou tabelas. BNCC: EF04LP20 -- Reconhecer a função de gráficos, diagramas e tabelas em textos, como forma de apresentação de dados e informações. a) Correta. O infográfico traz textualmente essa informação. b) Incorreta. O infográfico não menciona a restrição do uso de máscaras por crianças. c) Incorreta. O infográfico menciona que as máscaras auxiliam o usuário e as pessoas ao seu redor. d) Incorreta. O infográfico afirma exatamente o contrário.

\item
SAEB: Reconhecer em textos o significado de palavras derivadas a partir de seus afixos. BNCC: EF35LP05 -- Inferir o sentido de palavras ou expressões desconhecidas em textos, com base no contexto da frase ou do texto. a) Incorreta. A ideia de negação está associada ao prefixo ``anti''. b) Correta. O prefixo ``hiper'' indica uma relação de intensidade. c) Incorreta. O prefixo ``hiper'' indica uma relação de intensidade, não de diminuição. d) Incorreta. A ideia de oposição está associada ao prefixo ``contra''.
\end{enumerate}

\colorsec{Simulado 2}

\begin{enumerate}
\item
SAEB: Identificar o tema central do texto. BNCC: EF35LP29 -- Identificar, em narrativas, cenário, personagem central, conflito gerador, resolução e o ponto de vista com base no qual histórias são narradas, diferenciando narrativas em primeira e terceira pessoas. a) Correta. O texto caracteriza o homem que subiu na árvore como um não confiável, já que abandonou seu amigo durante uma situação de perigo. É a conclusão a que se pode chegar por meio da leitura da curta narrativa e da moral da história, em destaque, ao final. b) Incorreta. A leitura da curta narrativa e da moral da história, em destaque, ao final, permite afirmar que o homem que subiu na árvore pensou apenas em si mesmo, já que escapou do urso sem ajudar o seu amigo. c) Incorreta. O homem que escapou do urso não ajudou o amigo; ele apenas se escondeu, sem tentar pensar naquele que o acompanhava na caminhada. d) Incorreta. Não há, no texto, menção a aconselhamento de um homem a outro.

\item
SAEB: Identificar o tema central do texto. BNCC: EF04LP10 -- Ler e compreender, com autonomia, cartas pessoais de reclamação, dentre outros gêneros do campo da vida cotidiana, de acordo com as convenções do gênero carta e considerando a situação comunicativa e o tema/assunto/finalidade do texto. a) Correta. O tema central são os elogios dos alunos à revista e ao texto``Por que o cérebro nunca deixa de aprender?''. Eles leem a revista periodicamente e julgaram que esse artigo específico é ``muito legal''. b) Incorreta. A pergunta ``Assim vamos aprender muito mais, não acha?''é quase retórica, isto é: os autores da carta a fizeram sabendo que a a resposta é positiva. Além disso, essa pergunta contribui para reafirmar o tema central do texto: o elogio dos alunos à revista e ao artigo nela publicado. c) Incorreta. A roda de leitura é mencionada apenas para reafirmar o tema central do texto: o elogio dos alunos à revista e ao artigo nela publicado. d) Incorreta. A frequência de leitura da revista é mencionada apenas para reafirmar o tema central do texto: o elogio dos alunos à revista e ao artigo nela publicado.

\item
SAEB: Estabelecer relação entre partes de um texto a partir de mecanismos de concordância verbal e nominal. BNCC: EF04LP07 -- Identificar em textos e usar na produção textual a concordância entre artigo, substantivo e adjetivo (concordância no grupo nominal). a) Incorreta. O uso de ``um'' para se referir ao lenhador torna essa alternativa inválida. b) Correta. O texto refere-se às personagens principais João e Maria por artigos definidos ``o'' e ``a'', diferente de ``uma extensa mata'', ``uma cabana pobre'' e ``um lenhador'', referidos por artigos indefinidos. c) Incorreta. A utilização de ``a'' para caracterizar a menina, Maria, torna essa alternativa inválida. d) Incorreta. Os artigos masculinos são usados para caracterizar os personagens masculinos.

\item
SAEB: Inferir o sentido de uma palavra ou expressão a partir do contexto imediato. BNCC: EF35LP05 -- Inferir o sentido de palavras ou expressões desconhecidas em textos, com base no contexto da frase ou do texto. a) Correta. Pela análise do contexto, pode-se inferir que a palavra destacada refere-se ao ganho de conhecimento. b) Incorreta. Pela análise do contexto, pode-se inferir que a palavra destacada refere-se ao ganho de conhecimento, não ao desenvolvimento da fala. c) Incorreta. Não há menção, no texto, a movimentos corporais. d) Incorreta. Não há menção, no texto, às regras de convívio.

\item
SAEB: Identificar elementos constitutivos de textos narrativos. BNCC: EF35LP26 -- Ler e compreender, com certa autonomia, narrativas ficcionais que apresentem cenários e personagens, observando os elementos da estrutura narrativa: enredo, tempo, espaço, personagens, narrador e a construção do discurso indireto e discurso direto. a) Incorreta. As descrições de aparência das personagens normalmente ocorrem ao longo do texto. b) Correta. O travessão para introduzir falas de personagens, exatamente como ocorre no texto. c) Incorreta. Parágrafos são concluídos com pontos-finais. d) Incorreta. As descrições de espaço acontecem ao longo do texto.

\item
SAEB: Analisar elementos constitutivos de gêneros textuais diversos. BNCC: EF04LP05 -- Identificar e reproduzir, em textos injuntivos instrucionais (instruções de jogos digitais ou impressos), a formatação própria desses textos (verbos imperativos, indicação de passos a ser seguidos) e formato específico dos textos orais ou escritos desses gêneros (lista/ apresentação de materiais e instruções/passos de jogo). a) Correta. O modo imperativo é usado para transmitir as instruções. b) Incorreta. As formas verbais do texto estão flexionadas no modo imperativo. O modo indicativo não é usado para transmitir instruções. c) Incorreta. Adjetivos são usados para qualificar substantivos. d) Incorreta. Substantivos são usados para se referir a objetos animados ou inanimados.

\item
SAEB: Analisar os efeitos de sentido decorrentes do uso da pontuação. BNCC: EF04LP05 -- Identificar a função na leitura e usar, adequadamente, na escrita ponto final, de interrogação, de exclamação, dois-pontos e travessão em diálogos (discurso direto), vírgula em enumerações e em separação de vocativo e de aposto. a) Incorreta. O ponto-final é utilizado para concluir uma frase. Os dois-pontos é que servem para introduzir o diálogo. b) Incorreta. Os dois-pontos da frase destacada servem para introduzir diálogo. c) Incorreta. O ponto de interrogação é usado para indicar uma dúvida. Os dois-pontos é que servem para introduzir o diálogo. d) Correta. Os dois-pontos servem para introduzir o diálogo.

\item
SAEB: Inferir informações implícitas em textos. BNCC: EF35LP04 -- Inferir informações implícitas nos textos lidos. a) Correta. O estudo citado no texto avalia as deficiências da formação dos professores no que se refere à tecnologia. Pode-se inferir, portanto, que o uso da tecnologia é muito importante no contexto da sala de aula. b) Incorreta. O conjunto do texto permite inferir que o uso da tecnologia é muito importante no contexto da sala de aula. c) Incorreta. De acordo com as afirmações do texto, os professores brasileiros \textit{não sabem} utilizar tecnologia em sala de aula. d) Incorreta. Segundo as afirmações do texto, a pesquisa foi realizada por uma instituição britânica, sem menção a uso de tecnologia britânica nas escolas brasileiras.

\item
SAEB: Analisar os efeitos de sentido de verbos de enunciação. BNCC: EF35LP30 -- Diferenciar discurso indireto e discurso direto, determinando o efeito de sentido de verbos de enunciação e explicando o uso de variedades linguísticas no discurso direto, quando for o caso. a) Incorreta. O verbo que introduz a fala da personagem é ``gritou''. b) Correta. O verbo que introduz a fala da personagem é ``gritou''. c) Incorreta. O verbo que introduz a fala da personagem é ``gritou''. d) Incorreta. O verbo que introduz a fala da personagem é ``gritou''.

\item
SAEB: Identificar os mecanismos de progressão textual. a) Incorreta. No contexto em que se insere, a expressão ``dela'' retoma a palavra ``moça''. b) Incorreta. No contexto em que se insere, a expressão ``dela'' retoma a palavra ``moça''. c) Correta. No contexto em que se insere, a expressão ``dela'' retoma a palavra ``moça''. d) Incorreta. No contexto em que se insere, a expressão ``dela'' retoma a palavra ``moça''.

\item
SAEB: Analisar os efeitos de sentido decorrentes do uso dos advérbios. a) Incorreta. O advérbio ``talvez'' expressa a circunstância de dúvida, não a de lugar. b) Incorreta. O advérbio ``talvez'' expressa a circunstância de dúvida, não a de tempo. c) Incorreta. O advérbio ``talvez'' expressa a circunstância de dúvida, não a de negação. d) Correta. O advérbio ``talvez'' expressa a circunstância de dúvida.

\item
SAEB: Reconhecer diferentes gêneros textuais. BNCC: EF04LP14 -- Identificar, em notícias, fatos, participantes, local e momento/tempo da ocorrência do fato noticiado. a) Incorreta. Um anúncio cumpre a função de divulgar um produto ou evento. b) Incorreta. Um poema é estruturado em versos e estrofes. c) Correta. O gênero notícia tem como objetivo principal informar o leitor. d) Incorreta. O conto é um texto narrativo ficcional.

\item
SAEB: Julgar a eficácia de argumentos em textos. a) Incorreta. O texto não apresenta diferentes pontos de vista. b) Correta. O texto traz a opinião de uma especialista no tema, no segundo parágrafo. c) Incorreta. O texto apresenta uma linguagem objetiva, de fácil compreensão. d) Incorreta. O texto traz informações claras.

\item
SAEB: Analisar a construção de sentidos de textos em versos com base em seus elementos constitutivos. BNCC: EF35LP27 -- Ler e compreender, com certa autonomia, textos em versos, explorando rimas, sons e jogos de palavras, imagens poéticas (sentidos figurados) e recursos visuais e sonoros. a) Incorreta. A rima presente no poema faz com o que o substantivo seja associado ao adjetivo. b) Correta. A presença da rima faz com as duas palavras sejam lidas em conjunto. c) Incorreta. O poema pretende complementar o sentido da primeira palavra. d) Incorreta. As duas palavras fazem parte do mesmo campo semântico.

\item
SAEB: Analisar os efeitos de sentido de recursos multissemiótico em textos que circulam em diferentes suportes. a) Incorreta. A imagem e o texto seguem a mesma linha argumentativa. b) Incorreta. A imagem acrescenta informações à argumentação. c) Correta. A ilustração reforça o argumento, dando destaque à imagem do mosquito e às frases mais importantes do texto. d) Incorreta. Texto e imagem complementam um ao outro.
\end{enumerate}

\colorsec{Simulado 3}

\begin{enumerate}
\item
SAEB: Estabelecer relação entre informações num texto ou entre diferentes textos. BNCC: EF04LP09 -- Ler e compreender, com autonomia, boletos, faturas e carnês, dentre outros gêneros do campo da vida cotidiana, de acordo com as convenções do gênero (campos, itens elencados, medidas de consumo, código de barras) e considerando a situação comunicativa e a finalidade do texto. a) Incorreta. Não existe ``modo de preparo'' em instruções de jogos, mas em receitas. b) Incorreta. Os jogos não preveem punições aos perdedores. c) Correta. É importante para os jogos saber como estabelecer um vencedor. d) Incorreta. Alguns textos instrucionais trazem imagens, mas não há obrigatoriedade nesse gênero.

\item
SAEB: Identificar o tema central do texto. BNCC: EF04LP20 -- Reconhecer a função de gráficos, diagramas e tabelas em textos, como forma de apresentação de dados e informações. a) Correta. O infográfico apresenta o ciclo de vida do mosquito, desde o momento em que os ovos são depositados em criadouros próximos à água até a fase adulta, quando já pode transmitir as doenças. b) Incorreta. O infográfico não se restringe à vida adulta do mosquito. c)  Incorreta. O infográfico não se restringe ao momento em que o mosquito começa a transmitir doenças. d)  Incorreta. Como se observa no título, o infográfico se refere ao ciclo de vida do mosquito, aludindo, de maneira geral, a alguns espaços em que ele se reproduz e cresce.

\item
SAEB: Identificar o tema central do texto. BNCC: EF35LP03 -- Identificar a ideia central do texto, demonstrando compreensão global. a) Incorreta. O infográfico trata sobre a alimentação dos idosos, não de proteínas. b) Incorreta. O infográfico trata sobre a alimentação dos idosos, não da população em geral. c) Incorreta. O infográfico não expõe as doenças causadas pelo consumo de alimentos ultraprocessados. d) Correta. O infográfico apresenta quais são os alimentos saudáveis e os não saudáveis para o consumo de idosos.

\item
SAEB: Inferir o sentido de palavras ou expressões em textos. BNCC: EF35LP05 -- Inferir o sentido de palavras ou expressões desconhecidas em textos, com base no contexto da frase ou do texto. a) Correta. ``Aurora'' é a luminosidade das primeiras horas do dia; no contexto corresponde, portanto, ao início da vida, porque é vocábulo associado à infância. O poeta sente saudades dos primeiros dias de sua vida. b) Incorreta. ``Aurora'', no contexto, signfica ``início'', que é o oposto do fim. c) Incorreta.  ``Aurora'', no contexto, signfica ``início'', que é diferente da beleza. d) Incorreta. ``Aurora'', no contexto, signfica ``início'', que não guarda relação de sentido com a tristeza.

\item
SAEB: Distinguir fatos de opiniões em textos. BNCC: EF04LP15 -- Distinguir fatos de opiniões/sugestões em textos (informativos, jornalísticos, publicitários etc.). a) Correta. O trecho entre aspas contém transcrição da opinião da especialista. b) Incorreta. O trecho entre aspas contém transcrição da opinião da especialista, não a exposição de um fato. c) Incorreta. O trecho indicado entre aspas não apresenta características de narrativa. d) Incorreta. O trecho entre aspas contém transcrição da opinião da especialista, não versos de um poema.

\item
SAEB: Inferir o sentido de palavras ou expressões em textos. BNCC: EF35LP05 -- Inferir o sentido de palavras ou expressões desconhecidas em textos, com base no contexto da frase ou do texto. a) Incorreta. ``Transitórias'' signfica ``breves''; ``eterno'' é exatamente o oposto disso. b) Incorreta. ``Transitórias'' signfica ``breves''; ``fútil'' é aquilo que não tem importância, valor ou relevância. c) Incorreta.  ``Transitórias'' signfica ``breves'', sem relação de de sentido com ``qualidade''. d) Correta. ``Transitórias'' signfica ``breves'', ``passageiras''.

\item
SAEB: Identificar os mecanismos de referenciação lexical e pronominal. BNCC: EF05LP27 -- Utilizar, ao produzir o texto, recursos de coesão pronominal (pronomes anafóricos) e articuladores de relações de sentido (tempo, causa, oposição, conclusão, comparação), com nível adequado de informatividade. a) Incorreta. A palavra ``grito'', referente ao grito do Ipiranga, aparece no texto após o termo destacado. b) Incorreta. O contexto em que se insere permite afirmar que o termo ``acontecimento'' se refere à Independência do Brasil, não à cidade de São Paulo. c) Incorreta. O contexto em que se insere permite afirmar que o termo ``acontecimento'' se refere à Independência do Brasil, não a ``monumentos''. d) Correta. O contexto em que se insere permite afirmar que o termo ``acontecimento'' se refere à Independência do Brasil.

\item
SAEB: Analisar relações de causa e consequência. BNCC: EF05LP27 -- Utilizar, ao produzir o texto, recursos de coesão pronominal (pronomes anafóricos) e articuladores de relações de sentido (tempo, causa, oposição, conclusão, comparação), com nível adequado de informatividade. a) Incorreta. A expressão ``Isso acontece porque'', que inicia o segundo parágrafo, se refere às informações do primeiro, isto é: a redução (não o aumento) da poluição em São Paulo durante a pandemia. b) Correta. A expressão ``Isso acontece porque'', que inicia o segundo parágrafo, se refere às informações do primeiro, isto é: a redução da poluição em São Paulo durante a pandemia. c) Incorreta. A expressão ``Isso acontece porque'', que inicia o segundo parágrafo, se refere às informações do primeiro, isto é: a redução da poluição em São Paulo durante a pandemia (não o aumento na frota de veículos na cidade de São Paulo). d) Incorreta. A expressão ``Isso acontece porque'', que inicia o segundo parágrafo, se refere às informações do primeiro, isto é: a redução da poluição em São Paulo durante a pandemia.

\item
SAEB: Identificar a ideia central o texto. BNCC: EF35LP03 -- Identificar a ideia central do texto, demonstrando compreensão global. a) Correta. O texto contém informações sobre o evento \textit{Guilherme de Almeida em Cena}, que homenageia a obra do poeta em seu mês de nascimento e falecimento. b) Incorreta. No texto, não existe menção à necessidade de mais eventos literários serem realizados. c) Incorreta. No texto, não existe menção a poemas específicos do escritor. d) Incorreta. No texto, há referência à ocupação de tradutor de Guilherme de Almeida, mas não à tradução de poemas no Brasil.

\item
SAEB: Localizar informação explícita. BNCC: EF35LP03 -- Identificar a ideia central do texto, demonstrando compreensão global. a) Incorreta. No texto, há menção à Secretaria de Comunicação Social do Estado (Secom), sem alusão à criação de novas secretarias. b) Incorreta. No texto, não há referência a punições. c) Incorreta. No texto, não há referência a redes sociais. d) Correta. A análise de informações e seu encaminhamento aos órgãos competentes é citada explicitamente no texto.

\item
SAEB: Reconhecer os usos da pontuação. BNCC: EF04LP05 -- Identificar a função na leitura e usar, adequadamente, na escrita ponto final, de interrogação, de exclamação, dois-pontos e travessão em diálogos (discurso direto), vírgula em enumerações e em separação de vocativo e de aposto. a) Incorreta. No trecho em destaque, a vírgula foi usada para separar os itens de uma enumeração, não para destacar informações. b) Incorreta. O ponto de interrogação é que deve ser usado para marcar dúvidas. No trecho em destaque, a vírgula foi usada para separar os itens de uma enumeração. c) Correta. No trecho em destaque, a vírgula foi usada para separar os itens de uma enumeração. d) Incorreta. O ponto-final é que deve ser usado para concluir uma frase. No trecho em destaque, a vírgula foi usada para separar os itens de uma enumeração.

\item
SAEB: Reconhecer diferentes modos de organização composicional de textos em versos. BNCC: EF35LP27 -- Ler e compreender, com certa autonomia, textos em versos, explorando rimas, sons e jogos de palavras, imagens poéticas (sentidos figurados) e recursos visuais e sonoros. a) Incorreta. O poema apresenta 12 versos divididos em duas estrofes. b) Incorreta. O poema apresenta 12 versos divididos em duas estrofes. c) Incorreta. O poema apresenta 12 versos divididos em duas estrofes. d) Correta. O poema apresenta 12 versos divididos em duas estrofes.

\item
SAEB: Analisar informações apresentadas em gráficos, infográficos ou tabelas. BNCC: EF04LP20 -- Reconhecer a função de gráficos, diagramas e tabelas em textos, como forma de apresentação de dados e informações. a) Incorreta. A ação antioxidante é uma característica do abacaxi. b) Correta. A ação anti-inflamatória é uma característica da beterraba. c) Incorreta. A presença de vitamina A é uma característica do almeirão. d) Incorreta. A presença de muitas fibras é uma característica da abóbora paulista.

\item
SAEB: Identificar elementos constitutivos de textos narrativos. BNCC: EF35LP26 -- Ler e compreender, com certa autonomia, narrativas ficcionais que apresentem cenários e personagens, observando os elementos da estrutura narrativa: enredo, tempo, espaço, personagens, narrador e a construção do discurso indireto e discurso direto. a) Incorreta. Machado de Assis é o autor do conto. b) Correta. O texto contém um discurso indireto, em que o narrador descreve o que Camilo está dizendo a uma personagem feminina. c) Incorreta. Camilo é uma das personagens do conto. A fala dele é descrita pelo narrador, em discurso indireto. d) Incorreta. Vilela é uma das personagens do conto.

\item
SAEB: Analisar os efeitos de sentido de recursos multissemióticos em textos que circulam em diferentes suportes. a) Correta. A imagem mostra uma mulher utilizando uma máscara. b) Incorreta. O cartaz afirma exatamente o contrário: medidas de higiene devem ser adotadas. c) Incorreta. O cartaz menciona cuidados necessários contra o vírus. d) Incorreta. A ilustração mostra exatamente o contrário: é preciso
usar máscaras.
\end{enumerate}

\colorsec{Simulado 4}

\begin{enumerate}
\item
SAEB: Inferir uma afirmação implícita num texto. BNCC: EF35LP26 -- Ler e compreender, com certa autonomia, narrativas ficcionais que apresentem cenários e personagens, observando os elementos da estrutura narrativa: enredo, tempo, espaço, personagens, narrador e a construção do discurso indireto e discurso direto. a) Incorreta. De acordo com o texto, o dono tem o dom de entender a língua dos animais. Não há alusão, no texto, quanto à capacidade do asno de compreender a língua do dono. b) Incorreta. Para o boi, o asno trabalha pouco: sua labuta se reduz às poucas vezes em que é levado pelo dono em curtas viagens. c) Correta. O boi inveja o asno, pois acredita que este não trabalha muito e, por isso, tem uma vida melhor.
d) Incorreta. Não há, no texto, alusão à força do asno.

\item
SAEB: Estabelecer relação entre informações num texto ou entre diferentes textos. BNCC: EF35LP26 -- Ler e compreender, com certa autonomia, narrativas ficcionais que apresentem cenários e personagens, observando os elementos da estrutura narrativa: enredo, tempo, espaço, personagens, narrador e a construção do discurso indireto e discurso direto. a) Incorreta. Quem se mostra assustado é Aladin, e não o mágico. b) Incorreta. Não há alusão, no texto, ao proósito de aprofundamento de laços afetivos. c) Incorreta. O mágico não ensina magia a Aladin. d) Correta. O mágico é, na realidade, um falso tio, que usa Aladin para alcançar seus objetivos. Essa alternativa corresponde rigorosamente às afirmações do primeiro parágrafo do texto.

\item
SAEB: Estabelecer relação entre informações num texto ou entre diferentes textos. BNCC: EF35LP16 -- Identificar e reproduzir, em notícias, manchetes, lides e corpo de notícias simples para público infantil e cartas de reclamação (revista infantil), digitais ou impressos, a formatação e diagramação específica de cada um desses gêneros, inclusive em suas versões orais. a) Correta. O autor insere a declaração de uma pessoa com a intenção de dar credibilidade à notícia. b) Incorreta. O título tem a função de destacar os aspectos mais importantes do fato relatado. c) Incorreta. A linha fina complementa as informações apresentadas no título. d) Incorreta. Não há imagens ilustrando os fatos.

\item
SAEB: Localizar informações num texto. BNCC: EF15LP03 -- Localizar informações explícitas em textos. a) Incorreta. O aluno é o único da escola que não muda de time b) Incorreta. No momento da entrevista, os alunos da escola estão torcendo pelo Corinthians, não pelo Palmeiras. c) Incorreta. O entrevistado entende o motivo pelo qual os seus amigos trocam de time: eles torcem por quem está ganhando. d) Correta. O entrevistado torce por um único time, o Corinthians, do qual gosta muito.

\item
SAEB: Identificar os mecanismos de progressão textual. a) Incorreta. A expressão indica que haverá acréscimo às informações apresentadas anteriormente. b) Incorreta. A expressão não tem valor de oposição, mas de adição. c) Correta. A expressão indica que haverá adição de informações às que já foram apresentadas no parágrafo anterior. d) Incorreta. A expressão não indica uma relação de negação com as informações apresentadas anteriormente.

\item
SAEB: Analisar o uso de recursos de persuasão em textos verbais e/ou multimodais. a) Incorreta. Não há citações de especialistas no cartaz. b) Incorreta. Não há textos textos informativos no cartaz. c) Incorreta. Não são apresentados dados estatísticos no cartaz. d) Correta. Por meio da ilustração, a felicidade da criança fica associada ao combate ao trabalho infantil.

\item
SAEB: Analisar os efeitos de sentido decorrentes do uso dos adjetivos. BNCC: EF04LP07 -- Identificar em textos e usar na produção textual a concordância entre artigo, substantivo e adjetivo (concordância no grupo nominal). a) Correta. O adjetivo ``amado'' denota carinho de Araci pelo guerriero. b) Incorreta. O adjetivo ``amado'' tem conotação positiva. c) Incorreta. O adjetivo ``amado'' não expressa raiva. d) Incorreta. O adjetivo ``amado'' não expressa oposição.

\item
SAEB: Analisar os efeitos de sentido decorrentes do uso dos advérbios. BNCC: EF04LP07 -- Identificar em textos e usar na produção textual a concordância entre artigo, substantivo e adjetivo (concordância no grupo nominal). a) Incorreta. A circustância expressa por ``longe'' é de lugar, não de intensidade. b) Incorreta. A circustância expressa por ``longe'' é de lugar, não de tempo. c) Correta.  A circustância expressa por ``longe'' é de lugar. d) Incorreta. A circustância expressa por ``longe'' é de lugar, não de modo.

\item
SAEB: Analisar os efeitos de sentido de verbos de enunciação. BNCC: EF35LP30 -- Diferenciar discurso indireto e discurso direto, determinando o efeito de sentido de verbos de enunciação e explicando o uso de variedades linguísticas no discurso direto, quando for o caso. a) Incorreta. A forma verbal ``levando'' não se refere a uma das falas. b) Correta. As formas verbais em questão se referem às falas das personagens. c) Incorreta. As formas verbais em questão não se referem ao diálogo do trecho. d) Incorreta. A forma verbal  ``alçando'' não se refere a uma das falas.

\item
SAEB: - Analisar os efeitos de sentido de recursos multissemiótico em textos que circulam em diferentes suportes. a) Incorreta. A imagem sugere que todos devem se vacinar. b) Incorreta.  A imagem demonstra a preocupação dos órgãos de saúde com a vacinação. c) Correta. A imagem traz diferentes grupos que devem ser vacinados. d) Incorreta. A imagem sugere exatamente o contrário: diversos grupos devem ser vacinados.

\item
SAEB: Identificar as marcas de organização de textos dramáticos. BNCC: EF04LP27 -- Identificar, em textos dramáticos, marcadores das falas das personagens e de cena. a) Incorreta. Os diálogos surgem ao longo do texto. b) Incorreta. Não há descrição de figurino nos trechos entre parênteses. c) Correta. Os trechos entre parênteses contêm as chamadas \textit{rubricas}, isto é, indicações do autor da peça referentes ao modo de execução de movimentos cênicos, falas ou gestos dos atores em cena. d) Incorreta. Não há descrições do espaço nos trechos entre parênteses.

\item
SAEB: Julgar a eficácia de argumentos em textos. a) Incorreta. O registro informal não está associado à credibilidade do texto. b) Incorreta. O fato de informações serem publicadas na internet não as torna mais (ou menos) confiáveis. c) Incorreta. A complexidade da linguagem não está associada à qualidade dos argumentos. d) Correta. A opinião do especialista conferiu credibilidade ao texto.

\item
SAEB: Analisar os efeitos de sentido decorrentes do uso da pontuação. BNCC: EF04LP05 -- Identificar a função na leitura e usar, adequadamente, na escrita ponto final, de interrogação, de exclamação, dois-pontos e travessão em diálogos (discurso direto), vírgula em enumerações e em separação de vocativo e de aposto. a) Correta. O ponto de interrogação é usado para indicar uma pergunta. b) Incorreta. Em uma narrativa, utilizamos os dois-pontos para introduzir uma fala, não para indicar uma pergunta. c) Incorreta. A vírgula é mais comumente utilizada em uma enumeração, não para indicar uma pergunta. d) Incorreta. O ponto de exclamação é mais comumente usado para expressar surpresa, não para indicar uma pergunta.

\item
SAEB: Identificar a ideia central o texto. BNCC: EF35LP03 -- Identificar a ideia central do texto, demonstrando compreensão global. a) Incorreta. O texto apenas menciona a década de 1940, o que não é suficiente para afirmar que ela é o tema. b) Incorreta. O texto não faz menção a quaisquer movimentos artísticos. c) Correta. O assunto do texto é a exposição virtual ``Na Paisagem de São Paulo: Rebolo e o Grupo Santa Helena''. d) Incorreta. O texto apenas menciona a Semana de Arte Moderna de 1922, o que não é suficiente para afirmar que ela é o tema.

\item
SAEB: - Localizar informação explícita. BNCC: EF15LP03 -- Localizar informações explícitas em textos. a) Incorreta. O texto menciona o caráter experimental da iniciativa, o que significa que ela não foi realizada anteriormente. b) Incorreta. O texto afirma exatamente o contrário: pela primeira vez, os estudantes surdos terão acesso a um vídeo com a prova do Enem traduzida. c) Correta. O texto menciona que os estudantes contarão com um vídeo para realizar a prova. d) Incorreta. O texto afirma exatamente o contrário: pela primeira vez, os estudantes surdos terão acesso a um vídeo com a prova do Enem traduzida.
\end{enumerate}