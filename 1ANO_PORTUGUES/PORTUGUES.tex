\section{1. BRINCANDO COM LETRAS E SONS
}\label{muxf3dulo-1-brincando-com-letras-e-sons}

Habilidade do SAEB

Relacionar elementos sonoros das palavras com sua representação.

Habilidades da BNCC
EF01LP04, EF01LP05, EF01LP07, EF01LP08, EF01LP09.

\subsection{CONTEÚDO}\label{conteuxfado}

Faça uma breve revisão das letras do alfabeto e aponte as relações entre
letras e fonemas. Pergunte aos alunos: qual é o som dessa letra? Essa
mesma letra representa apenas um som? Que outros sons essa mesma letra
pode representar? Depois, de forma lúdica, apresente as vogais,
mostrando que elas podem ser todas ditas em sequência; proceda da mesma
forma com as consoantes, explicitando que, para
construir o som delas, é preciso interromper a passagem do ar em algum
momento. Compare sons parecidos. Finalmente, explique as sílabas.

VOCÊ CONHECE AS LETRAS DO ALFABETO?

AS LETRAS SÃO SINAIS GRÁFICOS USADOS PARA ESCREVER AS PALAVRAS. JUNTAS,
AS 26 LETRAS FORMAM O NOSSO ALFABETO. VEJA:

\includegraphics[width=5.19167in,height=2.71250in]{media/image1.png}

CADA LETRA DO ALFABETO REPRESENTA UM SOM CHAMADO DE FONEMA. SÃO OS FONEMAS QUE DÃO ORIGEM ÀS PALAVRAS. A LETRA H É A ÚNICA LETRA DO ALFABETO QUE NÃO
APRESENTA SOM.

O ALFABETO ESTÁ DIVIDIDO EM DUAS PARTE: VOGAIS E CONSOANTES.

AS VOGAIS SÃO A - E - I - O - U, E AS CONSOANTES SÃO B, C, D, F, G, H, J,
L, M, N, P, Q, R, S, T, V, X, Z. AINDA EXISTEM TRÊS CONSOANTES ESPECIAIS - K,
Y, W - QUE SÃO USADAS PARA ESCREVER PALAVRAS DE ORIGEM ESTRANGEIRA.

UMA PALAVRA PODE SER DIVIDIDA EM UM OU MAIS PEDACINHOS CHAMADOS DE
SÍLABAS. AS SÍLABAS DAS PALAVRAS PODEM SER CLASSIFICADAS COMO INICIAIS, MEDIAIS E FINAIS. OBSERVE O EXEMPLO:

\begin{longtable}[]{@{}lll@{}}
\toprule
\textbf{SA} & \textbf{PA} & \textbf{TO}\tabularnewline
\bottomrule
\end{longtable}

A SÍLABA OU O SOM INICIAL É "SA", A MEDIAL É "PA" E A FINAL É "TO".

\subsection{ATIVIDADES}\label{atividades}

\subsubsection{1. COMPLETE AS LETRAS DO ALFABETO QUE ESTÃO
FALTANDO}\label{complete-as-letras-do-alfabeto-que-estuxe3o-faltando}

Para essa atividade, você pode levar o alfabeto móvel e colocá-lo em uma
caixa junto de números, placas de trânsito e outros símbolos, como
@,\#,\&,\%. Solicitar aos alunos que separem as letras dos outros símbolos para,
em seguida, colocá-las na ordem correta. Por fim, peça para separarem as vogais
da consoantes.

\begin{longtable}[]{@{}lllllllll@{}}
\toprule
A & B & C & D & E & F & G & H & I\tabularnewline
J & K & L & M & N & O & P & Q & R\tabularnewline
S & T & U & V & W & X & Y & Z\tabularnewline
\bottomrule
\end{longtable}

A) AGORA, PINTE AS VOGAIS DE AZUL E AS CONSOANTES DE VERMELHO.

O aluno deverá pintar as letra A -E -- I- O -- U de azul e as
outras letras de vermelho.

2. MARQUE COM UM X NA IMAGEM EM QUE SÓ APARECEM
LETRAS.\includegraphics[width=2.18819in,height=1.11111in]{media/image2.png}\includegraphics[width=2.03681in,height=1.04861in]{media/image3.png}

\includegraphics[width=1.20556in,height=1.11111in]{media/image4.png}

x
\includegraphics[width=0.40972in,height=0.30069in]{media/image5.png}\includegraphics[width=0.40972in,height=0.30069in]{media/image5.png}\includegraphics[width=0.40972in,height=0.30069in]{media/image5.png}

\href{https://www.freepik.com/free-vector/flat-design-license-plate-collection_28280136.htm\#query=PLACA\%20DE\%20CARRO\&position=42\&from_view=search\&track=ais}{\emph{https://www.freepik.com/free-vector/flat-design-license-plate-collection\_28280136.htm\#query=PLACA\%20DE\%20CARRO\&position=42\&from\_view=search\&track=ais}}

\subsubsection{3. PINTE SOMENTE O QUE É USADO PARA ESCREVER AS
PALAVRAS.}\label{pinte-somente-o-que-uxe9-usado-para-escrever-as-palavras.}

\includegraphics[width=2.23393in,height=1.58569in]{media/image6.png}\includegraphics[width=1.96211in,height=1.66818in]{media/image7.png}\includegraphics[width=1.56875in,height=1.56875in]{media/image8.png}

O aluno deve pintar a imagem B.

Colocar as imagens dentro de um quadro a b e c.

\href{https://www.freepik.com/free-vector/different-numerical-figures_23820257.htm\#query=numeros\%20preto\%20e\%20branco\&position=23\&from_view=search\&track=ais}{\emph{https://www.freepik.com/free-vector/different-numerical-figures\_23820257.htm\#query=numeros\%20preto\%20e\%20branco\&position=23\&from\_view=search\&track=ais}}

\href{https://www.freepik.com/premium-vector/user-interface-line-icon-pack_6717413.htm?query=simbolos\%20gr\%C3\%A1focospreto\%20e\%20branco\#from_view=detail_alsolike}{\emph{https://www.freepik.com/premium-vector/user-interface-line-icon-pack\_6717413.htm?query=simbolos\%20gr\%C3\%A1focospreto\%20e\%20branco\#from\_view=detail\_alsolike}}

\href{https://www.freepik.com/premium-vector/simple-handdrawn-alphabet-english-alphabet-marker-style-doodle-poster-card-prints-design_26544475.htm?query=letra\%20preto\%20e\%20branco\#fro}{\emph{https://www.freepik.com/premium-vector/simple-handdrawn-alphabet-english-alphabet-marker-style-doodle-poster-card-prints-design\_26544475.htm?query=letra\%20preto\%20e\%20branco\#fro}}

m\_view=detail\_alsolike

\subsubsection{4. ESCREVA A PRIMEIRA LETRA DO NOME DAS
FIGURAS.}\label{escreva-a-primeira-letra-do-nome-das-figuras.}

Para essa atividade, é importante apresentar as letras do alfabeto para
ensinar o som de cada uma. Leve algumas imagens e peça às
crianças para formar seus nomes explorando o som das sílabas inicial, medial e
final.

\includegraphics[width=0.62222in,height=1.51597in]{media/image9.png}\includegraphics[width=0.92986in,height=1.19236in]{media/image10.png}\includegraphics[width=0.74583in,height=1.09722in]{media/image11.png}\includegraphics[width=1.52708in,height=0.72014in]{media/image12.png}

\includegraphics[width=1.23958in,height=1.03019in]{media/image13.png}\includegraphics[width=0.93333in,height=0.87447in]{media/image14.png}\includegraphics[width=0.88403in,height=0.83253in]{media/image15.png}\includegraphics[width=1.11736in,height=1.18440in]{media/image16.png}

\href{https://www.freepik.com/premium-vector/cute-frog-cartoon_7056860.htm\#query=SAPO\&position=12\&from_view=search\&track=sph}{\emph{https://www.freepik.com/premium-vector/cute-frog-cartoon\_7056860.htm\#query=SAPO\&position=12\&from\_view=search\&track=sph}}

\href{https://www.freepik.com/free-vector/guitar-realistic-isolated_1538267.htm\#query=VIOLAO\&position=1\&from_view=search\&track=sph}{\emph{https://www.freepik.com/free-vector/guitar-realistic-isolated\_1538267.htm\#query=VIOLAO\&position=1\&from\_view=search\&track=sph}}

\href{https://www.freepik.com/free-vector/vector-set-different-red-black-blue-green-dice-isolated-white-background_11062556.htm\#query=DADO\&position=11\&from_view=search\&track=sph}{\emph{https://www.freepik.com/free-vector/vector-set-different-red-black-blue-green-dice-isolated-white-background\_11062556.htm\#query=DADO\&position=11\&from\_view=search\&track=sph}}

\href{https://www.freepik.com/free-photo/modern-lifestyle-furniture-chair-white-background_1006994.htm\#query=CADEIRA\&position=17\&from_view=search\&track=sph}{\emph{https://www.freepik.com/free-photo/modern-lifestyle-furniture-chair-white-background\_1006994.htm\#query=CADEIRA\&position=17\&from\_view=search\&track=sph}}

\href{https://www.freepik.com/free-vector/realistic-color-men-s-shoes-set_14683166.htm\#query=SAPATO\&position=44\&from_view=search\&track=sph}{\emph{https://www.freepik.com/free-vector/realistic-color-men-s-shoes-set\_14683166.htm\#query=SAPATO\&position=44\&from\_view=search\&track=sph}}

\href{https://www.freepik.com/premium-vector/cute-baby-monkey-waving-hand_6925466.htm\#query=MACACO\&position=14\&from_view=search\&track=sph}{\emph{https://www.freepik.com/premium-vector/cute-baby-monkey-waving-hand\_6925466.htm\#query=MACACO\&position=14\&from\_view=search\&track=sph}}

\href{https://www.freepik.com/premium-vector/multicolored-kite-toy-with-bowties-icon_2567728.htm\#query=PIPA\&position=19\&from_view=search\&track=sph}{\emph{https://www.freepik.com/premium-vector/multicolored-kite-toy-with-bowties-icon\_2567728.htm\#query=PIPA\&position=19\&from\_view=search\&track=sph}}

https://www.freepik.com/premium-vector/set-cute-bee-mascot-logo-vector-cartoon-insect-cute-character-bee-fly-template-icon\_24593361.htm\#query=ABELHA\&position=18\&from\_view=search\&track=sph

\subsubsection{5. ESCREVA AS SÍLABAS QUE ESTÃO FALTANDO PARA COMPLETAR O
NOME DO
DESENHO.}\label{escreva-as-suxedlabas-que-estuxe1-faltando-para-completar-o-nome-do-desenho.}

Retome o alfabeto móvel para formar o nome das palavras, contar as
letras que formam cada uma com seus respectivos sons iniciais, mediais e finais e comparar as palavras em que aparecem sons iguais.

\includegraphics[width=0.94792in,height=1.44104in]{media/image17.png}

\includegraphics[width=0.97153in,height=1.43750in]{media/image18.png}\includegraphics[width=1.48140in,height=1.32986in]{media/image19.png}\includegraphics[width=1.32769in,height=1.21097in]{media/image20.png}

\begin{longtable}[]{@{}ll@{}}
\toprule
PA & TO\tabularnewline
BO & LA\tabularnewline
\bottomrule
\end{longtable}

\begin{longtable}[]{@{}lll@{}}
\toprule
\textbf{ME} & \textbf{NI} & \textbf{NA}\tabularnewline
GI & RA & FA\tabularnewline
\bottomrule
\end{longtable}

\includegraphics[width=0.55139in,height=0.68958in]{media/image22.png}\includegraphics[width=0.75833in,height=0.98819in]{media/image23.png}\includegraphics[width=1.09306in,height=1.05139in]{media/image24.png}\includegraphics[width=1.89097in,height=0.84583in]{media/image25.png}

\begin{longtable}[]{@{}ll@{}}
\toprule
CO & PO\tabularnewline
TA & PE\tabularnewline
LA & RAN\tabularnewline
RA & TO\tabularnewline
\bottomrule
\end{longtable}

https://www.freepik.com/premium-photo/rubber-duckling\_6684923.htm?query=

PATO\#from\_view=detail\_alsolike

https://www.freepik.com/free-vector/girl-shy-character\_11782630.htm\#query=BONECA\&position=22\&from\_view=search\&track=sph

https://www.freepik.com/free-vector/inflatable-beach-ball-striped-air-balloon\_12207989.htm\#query=BOLA\&position=20\&from\_view=search\&track=sph

https://www.freepik.com/free-vector/cute-giraffe-cartoon-vector-illustration\_7038528.htm\#query=GIRAFA\&position=34\&from\_view=search\&track=sph

\href{https://www.freepik.com/premium-vector/cartoon-wool-carpets-bath-rug-woven-mat-carpet-roll-home-floor-textile-decor-vector-illustration-set_28995572.htm\#page=2\&query=TAPETE\&position=16\&from_view=search\&track=sph}{\emph{https://www.freepik.com/premium-vector/cartoon-wool-carpets-bath-rug-woven-mat-carpet-roll-home-floor-textile-decor-vector-illustration-set\_28995572.htm\#page=2\&query=TAPETE\&position=16\&from\_view=search\&track=sph}}

\href{https://www.freepik.com/premium-photo/whole-ripe-orange-fruit-isolated-white-background-with-clipping-path_12657425.htm?query=LARANJA\#from_view=detail_alsolike}{\emph{https://www.freepik.com/premium-photo/whole-ripe-orange-fruit-isolated-white-background-with-clipping-path\_12657425.htm?query=LARANJA\#from\_view=detail\_alsolike}}

https://www.freepik.com/free-vector/paper-cups-vector-realistic-mock-up-detailed-illustrations-isolated-set\_16515845.htm\#query=COPO\&position=39\&from\_view=search\&track=sph

\href{https://www.freepik.com/free-vector/sticker-design-with-cute-mouse-isolated_16455570.htm\#query=RATO\&position=17\&from_view=search\&track=sph\#page=1\&query=R\&from_query=undefined\&position=0\&from_view=search\&track=sph}{\emph{https://www.freepik.com/free-vector/sticker-design-with-cute-mouse-isolated\_16455570.htm\#query=RATO\&position=17\&from\_view=search\&track=sph\#page=1\&query=R\&from\_query=undefined\&position=0\&from\_view=search\&track=sph}}

\subsubsection{6. CONTE QUANTAS LETRAS CADA PALAVRA APRESENTA E ESCREVA NO
QUADRO.}\label{conte-quantas-letras-tem-as-palavras-e-escreva-no-quadro.}

DADO

AVIÃO

CARRINHO

FOCA

TESOURA

NAVIO

\subsubsection{7. PINTE DE AMARELO AS PALAVRAS QUE COMEÇAM COM
O MESMO
SOM.}\label{pinte-de-amarelo-as-palavras-no-quadro-que-comeuxe7a-com-o-mesmo-som.}

Para essa atividade, é interessante brincar com o jogo das rimas.

\begin{longtable}[]{@{}llll@{}}
\toprule
\textbf{JANELA} & \textbf{CANECA} & \textbf{PANELA} &
\textbf{GATO}\tabularnewline
\textbf{BANANA} & \textbf{CANELA} & \textbf{JUCA} &
\textbf{JACARÉ}\tabularnewline
\textbf{JABUTI} & \textbf{JACA} & \textbf{FADA} &
\textbf{MALA}\tabularnewline
\bottomrule
\end{longtable}

\subsubsection{8. SEPARE AS SÍLABAS DOS NOMES DOS ANIMAIS
DA FAZENDO DE SEU ANTÔNIO. ESCREVA A QUANTIDADE DE PEDACINHOS NO
QUADRO.}\label{separe-as-suxedlabas-das-palavras-dos-nomes-dos-animais-da-fazendo-de-seu-antuxf4nio.-e-escreva-a-quantidade-de-pedacinho-no-quadro.}

Leia as palavras com os alunos e, em seguida, convide-os a baterem
palmas para observar quantas vezes abrem a boca para falar as sílabas.


\includegraphics[width=1.29126in,height=1.13542in]{media/image26.jpg}

\textbf{SÍLABAS QUANTIDADE}

\begin{longtable}[]{@{}llll@{}}
\toprule
CA & VA & LO & 3\tabularnewline
\bottomrule
\end{longtable}

CAVALO

\begin{longtable}[]{@{}lll@{}}
\toprule
VA & CA & 2\tabularnewline
\bottomrule
\end{longtable}

\includegraphics[width=1.19577in,height=0.98873in]{media/image27.jpg}

VACA

\includegraphics[width=1.06250in,height=1.09969in]{media/image28.jpg}

\begin{longtable}[]{@{}llll@{}}
\toprule
O & VE & LHA & 3\tabularnewline
\bottomrule
\end{longtable}

OVELHA

\includegraphics[width=1.03125in,height=1.09419in]{media/image27.jpg}

\begin{longtable}[]{@{}llll@{}}
\toprule
GA & LI & NHA & 3\tabularnewline
\bottomrule
\end{longtable}

GALINHA

\begin{longtable}[]{@{}lll@{}}
\toprule
POR & CO & 2\tabularnewline
\bottomrule
\end{longtable}

\includegraphics[width=1.66606in,height=0.94389in]{media/image27.jpg}

PORCO

\href{https://www.freepik.com/free-vector/flat-design-farm-animal-collection_4751955.htm\#query=VACA\&position=0\&from_view=search\&track=sph}{\emph{https://www.freepik.com/free-vector/flat-design-farm-animal-collection\_4751955.htm\#query=VACA\&position=0\&from\_view=search\&track=sph}}

\href{https://www.freepik.com/free-vector/horse-racing-cartoon-icon-illustration_11167807.htm\#query=CAVALO\&position=22\&from_view=search\&track=sph}{\emph{https://www.freepik.com/free-vector/horse-racing-cartoon-icon-illustration\_11167807.htm\#query=CAVALO\&position=22\&from\_view=search\&track=sph}}

\href{https://www.freepik.com/free-vector/flat-design-farm-animal-collection_4751955.htm\#query=VACA\&position=0\&from_view=search\&track=sp}{\emph{https://www.freepik.com/free-vector/flat-design-farm-animal-collection\_4751955.htm\#query=VACA\&position=0\&from\_view=search\&track=sp}}

\subsubsection{9. ENCONTRE E PINTE OS NOMES DOS DESENHOS NO CAÇA-PALAVRAS.}\label{encontre-e-pinte-no-cauxe7a-palavras-os-nomes-dos-desenhos.}

\protect\hypertarget{_heading=h.1fob9te}{}{}Retome o alfabeto móvel para
ajudar as crianças a formarem as palavras observando os sons
finais iguais ou palavras que rimam. Em seguida, pintá-las no caça-palavras.

\includegraphics[width=3.83611in,height=2.98958in]{media/image29.jpg}\includegraphics[width=1.00625in,height=0.99857in]{media/image30.png}

\includegraphics[width=0.61042in,height=0.95923in]{media/image31.png}

\includegraphics[width=0.78750in,height=0.70972in]{media/image32.png}\includegraphics[width=1.02361in,height=0.88333in]{media/image33.png}

\includegraphics[width=1.07569in,height=1.14514in]{media/image34.png}\includegraphics[width=0.85069in,height=0.80069in]{media/image35.png}\includegraphics[width=0.63333in,height=0.94653in]{media/image36.png}

Fazer um caça-palavras igual a esse.

Resposta caça- palavras

\includegraphics[width=2.13750in,height=1.44792in]{media/image37.jpg}

AGORA ESCREVA AS PALAVRAS QUE RIMAM.

BOLA E COLA

\_\_\_\_\_\_\_\_\_\_\_\_\_\_\_\_\_\_\_\_\_\_\_\_\_\_\_\_\_\_\_\_\_\_\_\_\_\_\_\_\_\_\_\_\_\_\_\_\_\_\_\_\_\_\_\_\_\_\_\_\_\_\_\_\_\_\_\_\_\_\_\_\_\_\_\_\_\_\_\_\_\_\_\_\_\_\_\_\_\_\_\_\_\_\_\_\_\_\_\_\_\_\_\_\_\_\_\_\_\_\_\_\_\_\_\_\_\_\_\_\_\_\_\_\_

\href{https://www.freepik.com/free-vector/realistic-metal-springs_3817801.htm\#query=MOLA\&position=3\&from_view=search\&track=sph}{\emph{https://www.freepik.com/free-vector/realistic-metal-springs\_3817801.htm\#query=MOLA\&position=3\&from\_view=search\&track=sph}}

\href{https://www.freepik.com/free-vector/cute-duck-white_7042481.htm\#query=PATO\&position=1\&from_view=search\&track=sph}{\emph{https://www.freepik.com/free-vector/cute-duck-white\_7042481.htm\#query=PATO\&position=1\&from\_view=search\&track=sph}}

\href{https://www.freepik.com/premium-vector/funny-elephant-cartoon-illustration_14166793.htm\#query=ELEFANTE\&position=10\&from_view=search\&track=sph}{\emph{https://www.freepik.com/premium-vector/funny-elephant-cartoon-illustration\_14166793.htm\#query=ELEFANTE\&position=10\&from\_view=search\&track=sph}}

\href{https://www.freepik.com/free-vector/coffee-cup-set-eight_1544838.htm\#query=X\%C3\%8DCARA\&position=38\&from_view=search\&track=sph}{\emph{https://www.freepik.com/free-vector/coffee-cup-set-eight\_1544838.htm\#query=X\%C3\%8DCARA\&position=38\&from\_view=search\&track=sph}}

\subsubsection{9. PINTE AS PALAVRAS QUE COMEÇA E TERMINAM COM O SOM DE
VOGAL.}\label{pinte-as-palavras-que-comeuxe7a-e-terminam-com-o-som-de-vogal.}

\includegraphics[width=0.83333in,height=0.84004in]{media/image38.jpg}

\begin{longtable}[]{@{}lll@{}}
\toprule
U & V & A\tabularnewline
\bottomrule
\end{longtable}

\includegraphics[width=1.06463in,height=0.91667in]{media/image39.jpg}

\begin{longtable}[]{@{}llll@{}}
\toprule
S & A & P & O\tabularnewline
\bottomrule
\end{longtable}

\includegraphics[width=0.86458in,height=0.86667in]{media/image40.jpg}

\begin{longtable}[]{@{}lll@{}}
\toprule
Í & M & A\tabularnewline
\bottomrule
\end{longtable}

\includegraphics[width=1.07292in,height=0.97005in]{media/image41.jpg}

\begin{longtable}[]{@{}llll@{}}
\toprule
P & A & T & O\tabularnewline
\bottomrule
\end{longtable}

\includegraphics[width=0.85417in,height=1.14167in]{media/image42.jpg}

\begin{longtable}[]{@{}llll@{}}
\toprule
A & N & J & O\tabularnewline
\bottomrule
\end{longtable}

\href{https://www.freepik.com/free-vector/c-white-background_2413097.htm\#query=UVA\&position=33\&from_view=search\&track=sph}{\emph{https://www.freepik.com/free-vector/c-white-background\_2413097.htm\#query=UVA\&position=33\&from\_view=search\&track=sph}}

\href{https://www.freepik.com/free-vector/happy-frong-sitting-lotus-leaf_6952752.htm\#query=SAPO\&position=10\&from_view=search\&track=sph}{\emph{https://www.freepik.com/free-vector/happy-frong-sitting-lotus-leaf\_6952752.htm\#query=SAPO\&position=10\&from\_view=search\&track=sph}}

\href{https://www.freepik.com/free-vector/magnet-attracting-paperclips-white-background_20721479.htm\#query=\%C3\%8DMA\&position=10\&from_view=search\&track=sph}{\emph{https://www.freepik.com/free-vector/magnet-attracting-paperclips-white-background\_20721479.htm\#query=\%C3\%8DMA\&position=10\&from\_view=search\&track=sph}}

\href{https://www.freepik.com/free-vector/illustration-cute-yellow-rubber-duck-water_5422561.htm\#query=PATO\&position=3\&from_view=search\&track=sph}{\emph{https://www.freepik.com/free-vector/illustration-cute-yellow-rubber-duck-water\_5422561.htm\#query=PATO\&position=3\&from\_view=search\&track=sph}}

\subsubsection{10. LIGUE SO DESENHO PARA O SEU
NOMES.}\label{ligue-so-desenho-para-o-seu-nomes.}

PANELA\includegraphics[width=1.02083in,height=0.81667in]{media/image43.png}

GELATINA\includegraphics[width=0.75000in,height=0.73958in]{media/image44.png}

PIÃO

\includegraphics[width=0.93750in,height=0.62014in]{media/image45.jpg}

BICICLETA

\includegraphics[width=0.79167in,height=0.55208in]{media/image46.png}

SOFÁ\includegraphics[width=1.16667in,height=0.59375in]{media/image48.png}

\href{https://www.freepik.com/free-vector/red-leather-sofa-realistic-illustration_3907761.htm\#query=SOFA\&position=16\&from_view=search\&track=sph}{\emph{https://www.freepik.com/free-vector/red-leather-sofa-realistic-illustration\_3907761.htm\#query=SOFA\&position=16\&from\_view=search\&track=sph}}

\href{https://www.freepik.com/premium-vector/pink-jelly-platter-isolated-white_12545065.htm\#query=GELATINA\&position=6\&from_view=search\&track=sph}{\emph{https://www.freepik.com/premium-vector/pink-jelly-platter-isolated-white\_12545065.htm\#query=GELATINA\&position=6\&from\_view=search\&track=sph}}

\href{https://www.freepik.com/free-icon/spinning-top_14485201.htm\#query=spinning\%20top\&from_query=PI\%C3\%83O\&position=39\&from_view=search\&track=sph}{\emph{https://www.freepik.com/free-icon/spinning-top\_14485201.htm\#query=spinning\%20top\&from\_query=PI\%C3\%83O\&position=39\&from\_view=search\&track=sph}}

\href{https://www.freepik.com/free-vector/frying-pans-saucepans-cartoon-illustration-set-metal-cooking-pots-with-lid-different-sizes-stainless-utensils-making-soup-boiling-water-household-kitchen-concept_26921753.htm\#query=PANELA\&position=2\&from_view=search\&track=sph}{\emph{https://www.freepik.com/free-vector/frying-pans-saucepans-cartoon-illustration-set-metal-cooking-pots-with-lid-different-sizes-stainless-utensils-making-soup-boiling-water-household-kitchen-concept\_26921753.htm\#query=PANELA\&position=2\&from\_view=search\&track=sph}}

\href{https://www.freepik.com/free-vector/realistic-bicycle-set-with-different-models-illustration_13805753.htm\#query=BICICLETA\&position=21\&from_view=search\&track=sph}{\emph{https://www.freepik.com/free-vector/realistic-bicycle-set-with-different-models-illustration\_13805753.htm\#query=BICICLETA\&position=21\&from\_view=search\&track=sph}}

\href{https://www.freepik.com/premium-vector/cartoon-chair-sofa-couch-house-comfort-soft-furniture-cozy-armchairs-scandinavian-style-interior-relaxing-elements-vector-set-illustration-couch-sofa-chairs-furniture_21608691.htm\#page=2\&query=SOFA\&position=24\&from_view=search\&track=sph}{\emph{https://www.freepik.com/premium-vector/cartoon-chair-sofa-couch-house-comfort-soft-furniture-cozy-armchairs-scandinavian-style-interior-relaxing-elements-vector-set-illustration-couch-sofa-chairs-furniture\_21608691.htm\#page=2\&query=SOFA\&position=24\&from\_view=search\&track=sph}}

\subsubsection{11. ENCONTRE E CIRCULE O INTRUSO QUE ESTÁ NAS
PALAVRAS.}\label{encontre-e-circule-o-intruso-que-estuxe1-nas-palavras.}

\begin{longtable}[]{@{}llll@{}}
\toprule
\textbf{M4ÇÃ} & \textbf{R\%TO} & \textbf{TAT\#} &
\textbf{TAPET3}\tabularnewline
\textbf{CAS7} & \textbf{MOL\%} & \textbf{BUL@} &
\textbf{DOC2}\tabularnewline
\bottomrule
\end{longtable}

AGORA ESCREVA A PALAVRA CORRETAMENTE.
\_\_\_\_\_\_\_\_\_\_\_\_\_\_\_\_\_\_\_\_\_\_\_\_\_\_\_\_\_\_\_\_\_\_\_\_\_\_\_\_\_\_\_\_\_\_\_\_\_\_\_\_\_\_\_\_\_\_\_\_\_\_\_\_\_\_\_\_\_\_\_\_\_\_\_\_\_\_\_\_\_\_\_\_\_\_\_\_\_\_\_\_\_\_\_\_\_\_\_\_\_\_\_\_\_\_\_\_\_\_\_\_\_\_\_\_\_\_\_\_\_\_\_\_\_\_\_\_\_\_\_\_\_\_\_\_\_\_\_\_\_\_\_\_\_\_\_\_\_\_\_\_\_\_\_\_\_\_\_\_\_\_\_\_\_\_\_\_\_\_\_\_\_\_\_\_\_\_\_\_\_\_\_\_\_\_\_

Maçâ- rato -- tatu - tapete -- casa -- mola -- bule -- doce.

\subsubsection{12. CIRCULE O DESENHO CUJO NOME COMEÇA COM O MESMO SOM DA
FIGURA EM
DESTAQUE.}\label{circule-o-desenho-cujo-nome-comeuxe7a-com-o-mesmo-som-da-figura-em-destaque.}

\begin{longtable}[]{@{}ll@{}}
\toprule
\textbf{BOLA}\includegraphics[width=0.81250in,height=0.75903in]{media/image49.jpg}
&
\includegraphics[width=0.67361in,height=1.02083in]{media/image50.jpg}\includegraphics[width=0.68472in,height=0.56389in]{media/image51.jpg}\includegraphics[width=0.67014in,height=0.61944in]{media/image52.jpg}\tabularnewline
\begin{minipage}[t]{0.48\columnwidth}\raggedright\strut
\includegraphics[width=1.12431in,height=0.78125in]{media/image53.jpg}

\textbf{PANELA}\strut
\end{minipage} & \begin{minipage}[t]{0.48\columnwidth}\raggedright\strut
\includegraphics[width=0.64097in,height=0.53194in]{media/image54.jpg}\includegraphics[width=0.77778in,height=0.58681in]{media/image55.jpg}\includegraphics[width=0.50000in,height=0.49583in]{media/image56.jpg}\strut
\end{minipage}\tabularnewline
\textbf{TAPETE}\includegraphics[width=1.22986in,height=0.61458in]{media/image57.jpg}
&
\includegraphics[width=0.59375in,height=0.78819in]{media/image58.jpg}\includegraphics[width=0.56458in,height=0.56458in]{media/image59.jpg}\includegraphics[width=0.61042in,height=0.56250in]{media/image60.jpg}\tabularnewline
\begin{minipage}[t]{0.48\columnwidth}\raggedright\strut
\textbf{MAÇÂ}

\includegraphics[width=0.77569in,height=0.78125in]{media/image61.jpg}\strut
\end{minipage} & \begin{minipage}[t]{0.48\columnwidth}\raggedright\strut
\includegraphics[width=0.93472in,height=0.70486in]{media/image55.jpg}\includegraphics[width=0.85347in,height=0.80208in]{media/image62.jpg}\includegraphics[width=0.70833in,height=1.05957in]{media/image63.jpg}\strut
\end{minipage}\tabularnewline
\bottomrule
\end{longtable}

\href{https://www.freepik.com/premium-photo/3d-soccer-ball-isolated-white-with-clipping-path_4985141.htm\#query=BOLA\&position=25\&from_view=search\&track=sph}{\emph{https://www.freepik.com/premium-photo/3d-soccer-ball-isolated-white-with-clipping-path\_4985141.htm\#query=BOLA\&position=25\&from\_view=search\&track=sph}}

\href{https://www.freepik.com/free-vector/delicious-cakes-set_13187639.htm\#query=BOLO\&position=3\&from_view=search\&track=sph}{\emph{https://www.freepik.com/free-vector/delicious-cakes-set\_13187639.htm\#query=BOLO\&position=3\&from\_view=search\&track=sph}}

\href{https://www.freepik.com/free-vector/set-makar-sankranti-kites_5712499.htm\#query=PIPA\&position=8\&from_view=search\&track=sph}{\emph{https://www.freepik.com/free-vector/set-makar-sankranti-kites\_5712499.htm\#query=PIPA\&position=8\&from\_view=search\&track=sph}}

\href{https://www.freepik.com/free-vector/different-kinds-suitcases-illustrations-set-collection-travel-bags-with-wheels-luggage-baggage-briefcase-isolated-white_20827653.htm\#page=2\&query=MALA\&position=2\&from_view=search\&track=sph}{\emph{https://www.freepik.com/free-vector/different-kinds-suitcases-illustrations-set-collection-travel-bags-with-wheels-luggage-baggage-briefcase-isolated-white\_20827653.htm\#page=2\&query=MALA\&position=2\&from\_view=search\&track=sph}}

\href{https://www.freepik.com/free-vector/set-carpets-rugs-different-shapes-designs-colors_14977758.htm\#query=TAPETE\&position=1\&from_view=search\&track=sph}{\emph{https://www.freepik.com/free-vector/set-carpets-rugs-different-shapes-designs-colors\_14977758.htm\#query=TAPETE\&position=1\&from\_view=search\&track=sph}}

\href{https://www.freepik.com/free-photo/home-appliance-seat-interior-ergonomic-sign_1047824.htm\#query=CADEIRA\&position=27\&from_view=search\&track=sph}{\emph{https://www.freepik.com/free-photo/home-appliance-seat-interior-ergonomic-sign\_1047824.htm\#query=CADEIRA\&position=27\&from\_view=search\&track=sph}}

\href{https://www.freepik.com/free-vector/set-fruits-berries-with-banana-grapes-apple-others-drawing-isolated-white-background-flat-vector-illustration_29174377.htm\#query=UVA\&position=35\&from_view=search\&track=sph}{\emph{https://www.freepik.com/free-vector/set-fruits-berries-with-banana-grapes-apple-others-drawing-isolated-white-background-flat-vector-illustration\_29174377.htm\#query=UVA\&position=35\&from\_view=search\&track=sph}}

\href{https://www.freepik.com/premium-vector/cute-little-armadillo-cartoon-character_27006344.htm\#query=TATU\&position=5\&from_view=search\&track=sph}{\emph{https://www.freepik.com/premium-vector/cute-little-armadillo-cartoon-character\_27006344.htm\#query=TATU\&position=5\&from\_view=search\&track=sph}}

\href{https://www.freepik.com/free-vector/magic-fairy_8171958.htm\#query=FADA\&position=19\&from_view=search\&track=sph}{\emph{https://www.freepik.com/free-vector/magic-fairy\_8171958.htm\#query=FADA\&position=19\&from\_view=search\&track=sph}}

\subsubsection{13. ESCREVA A SÍLABA DO MEIO DOS NOMES DAS
FIGURAS}\label{escreva-a-suxedlaba-do-meio-dos-nomes-das-figuras}

Retome o alfabeto móvel para formar os nomes dos desenhos explorando
os sons inicial. Medial e final e ainda quantas vezes abre a boca para
falar a palavra.

\includegraphics[width=1.50139in,height=1.07153in]{media/image64.png}\includegraphics[width=1.12986in,height=1.39097in]{media/image65.png}\includegraphics[width=1.74752in,height=1.49479in]{media/image66.png}

NE VO TI

\includegraphics[width=1.56319in,height=1.45278in]{media/image67.png}\includegraphics[width=1.42083in,height=1.99583in]{media/image68.png}\includegraphics[width=1.67778in,height=1.67778in]{media/image69.png}

MA MI RA

\href{https://www.freepik.com/premium-vector/wooden-classic-vintage-open-window_5748457.htm\#query=JANELA\&position=36\&from_view=search\&track=sph}{\emph{https://www.freepik.com/premium-vector/wooden-classic-vintage-open-window\_5748457.htm\#query=JANELA\&position=36\&from\_view=search\&track=sph}}

\href{https://www.freepik.com/free-vector/tree_6132448.htm\#query=ARVORE\&position=11\&from_view=search\&track=sph}{\emph{https://www.freepik.com/free-vector/tree\_6132448.htm\#query=ARVORE\&position=11\&from\_view=search\&track=sph}}

\href{https://www.freepik.com/premium-vector/isolated-beautiful-female-outfit_9820058.htm?query=VESTIDO\#from_view=detail_alsolike}{\emph{https://www.freepik.com/premium-vector/isolated-beautiful-female-outfit\_9820058.htm?query=VESTIDO\#from\_view=detail\_alsolike\#position=0\&query=VESTIDO}}

\href{https://www.freepik.com/free-vector/fresh-tomato_957857.htm\#query=TOMATE\&position=16\&from_view=search\&track=sph}{\emph{https://www.freepik.com/free-vector/fresh-tomato\_957857.htm\#query=TOMATE\&position=16\&from\_view=search\&track=sph}}

\href{https://www.freepik.com/free-vector/heart_2900812.htm\#query=CORA\%C3\%87\%C3\%83O\&position=13\&from_view=search\&track=sph}{\emph{https://www.freepik.com/free-vector/heart\_2900812.htm\#query=CORA\%C3\%87\%C3\%83O\&position=13\&from\_view=search\&track=sph}}

\href{https://www.freepik.com/free-vector/set-insect-cartoon-character-its-silhouette-white-background_10162952.htm\#query=FORMIGA\&position=34\&from_view=search\&track=sph}{\emph{https://www.freepik.com/free-vector/set-insect-cartoon-character-its-silhouette-white-background\_10162952.htm\#query=FORMIGA\&position=34\&from\_view=search\&track=sph}}

\subsubsection{14. PINTE AS PALAVRAS QUE TERMINAM COM O MESMO
SOM.}\label{pinte-as-palavras-que-terminam-com-o-mesmo-som.}

\begin{longtable}[]{@{}lll@{}}
\toprule
\textbf{TAPETE} & \textbf{POMADA} & \textbf{GILETE}\tabularnewline
\textbf{PANELA} & \textbf{TOMATE} & \textbf{JIBOIA}\tabularnewline
\textbf{JACARÉ} & \textbf{GATO} & \textbf{SALADA}\tabularnewline
\bottomrule
\end{longtable}

\subsubsection{15. OBSERVE A FIGURA E PINTE O SEU
NOME.}\label{observe-a-figura-e-pinte-o-seu-nome.}

\begin{longtable}[]{@{}lll@{}}
\toprule
\includegraphics[width=1.66771in,height=1.63535in]{media/image70.png} &
\includegraphics[width=1.03234in,height=1.99479in]{media/image72.png} &
\includegraphics[width=1.86146in,height=1.63165in]{media/image73.png}\tabularnewline
COELHO & MELÃO & TATU\tabularnewline
COBRA & MACACO & TELEFONE\tabularnewline
ESPELHO & MENINA & TELEVISÃO\tabularnewline
\includegraphics[width=1.44836in,height=1.64634in]{media/image74.png} &
\includegraphics[width=1.86458in,height=1.77083in]{media/image75.png} &
\includegraphics[width=1.26319in,height=2.15625in]{media/image76.png}\tabularnewline
RÉGUA & ESCOLA & MOLA\tabularnewline
RELÓGIO & ESCOVA & MAMÃO\tabularnewline
ÉGUA & ESMALTE & MALA\tabularnewline
\bottomrule
\end{longtable}

\href{https://www.freepik.com/free-vector/white-rabbit-cartoon-white-background_18481938.htm\#query=COELHO\&position=24\&from_view=search\&track=sph}{\emph{https://www.freepik.com/free-vector/white-rabbit-cartoon-white-background\_18481938.htm\#query=COELHO\&position=24\&from\_view=search\&track=sph}}

\href{https://www.freepik.com/free-vector/young-woman-pink-dress-smiling_10577612.htm\#query=MENINA\&position=14\&from_view=search\&track=sph}{\emph{https://www.freepik.com/free-vector/young-woman-pink-dress-smiling\_10577612.htm\#query=MENINA\&position=14\&from\_view=search\&track=sph}}

\href{https://www.freepik.com/free-photo/3d-render-concept-old-telephone-3d-art-design-illustration_24009640.htm\#query=TELEFONE\&position=10\&from_view=search\&track=sph}{\emph{https://www.freepik.com/free-photo/3d-render-concept-old-telephone-3d-art-design-illustration\_24009640.htm\#query=TELEFONE\&position=10\&from\_view=search\&track=sph}}

\href{https://www.freepik.com/free-vector/round-wall-quartz-clock-red-color-isolated-white-background_13031909.htm\#query=RELOGIO\&position=32\&from_view=search\&track=sph}{\emph{https://www.freepik.com/free-vector/round-wall-quartz-clock-red-color-isolated-white-background\_13031909.htm\#query=RELOGIO\&position=32\&from\_view=search\&track=sph}}

\href{https://www.freepik.com/premium-psd/oval-hair-brush-mockup-perspective_15970932.htm\#page=2\&query=ESCOVA\&position=27\&from_view=search\&track=sph}{\emph{https://www.freepik.com/premium-psd/oval-hair-brush-mockup-perspective\_15970932.htm\#page=2\&query=ESCOVA\&position=27\&from\_view=search\&track=sph}}

\subsubsection{16. COLOQUE AS SÍLABAS EM ORDEM PARA FORMAR O NOME DAS
FIGURAS.}\label{coloque-as-suxedlabas-em-ordem-para-formar-o-nome-das-figuras.}

Para essa atividade pode ser retomada a utilização do alfabeto móvel.

\begin{longtable}[]{@{}lll@{}}
\toprule
CA & COL & CHE\tabularnewline
CACHECOL\tabularnewline
\bottomrule
\end{longtable}

\includegraphics[width=1.59418in,height=1.11488in]{media/image77.png}

\includegraphics[width=1.39583in,height=1.25000in]{media/image78.png}

\begin{longtable}[]{@{}ll@{}}
\toprule
PIS & LÁ\tabularnewline
LÁPIS\tabularnewline
\bottomrule
\end{longtable}

\begin{longtable}[]{@{}lll@{}}
\toprule
GE & TI & LA\tabularnewline
TIGELA\tabularnewline
\bottomrule
\end{longtable}

\includegraphics[width=1.41771in,height=0.94514in]{media/image79.jpg}

\begin{longtable}[]{@{}lll@{}}
\toprule
CU & LOS & Ó\tabularnewline
ÓCULOS\tabularnewline
\bottomrule
\end{longtable}

\includegraphics[width=1.78518in,height=0.95594in]{media/image80.png}

\begin{longtable}[]{@{}lll@{}}
\toprule
CA & CO & MA\tabularnewline
MACACO\tabularnewline
\bottomrule
\end{longtable}

\includegraphics[width=2.14688in,height=2.14688in]{media/image81.png}

\href{https://www.freepik.com/free-photo/red-winter-scarf-isolated-white-background_3837376.htm\#query=cachecol\&from_query=CACHICOL\&position=7\&from_view=search\&track=sph}{\emph{https://www.freepik.com/free-photo/red-winter-scarf-isolated-white-background\_3837376.htm\#query=cachecol\&from\_query=CACHICOL\&position=7\&from\_view=search\&track=sph}}

\href{https://www.freepik.com/free-vector/writting-pencil-design_850418.htm\#query=L\%C3\%81PIS\&position=2\&from_view=search\&track=sph}{\emph{https://www.freepik.com/free-vector/writting-pencil-design\_850418.htm\#query=L\%C3\%81PIS\&position=2\&from\_view=search\&track=sph}}

https://www.freepik.com/free-vector/modern-sunglasses-collection-flat-style\_2251083.htm\#query=OCULOS\&position=19\&from\_view=search\&track=sph

\href{https://www.freepik.com/premium-vector/cute-baby-monkey-cartoon-sitting_16729620.htm\#query=MACACO\&position=19\&from_view=search\&track=sph}{\emph{https://www.freepik.com/premium-vector/cute-baby-monkey-cartoon-sitting\_16729620.htm\#query=MACACO\&position=19\&from\_view=search\&track=sph}}

\subsection{TREINO}\label{treino}

\textbf{Os três itens a seguir estão ordenados do mais fácil para o mais
difícil. }

\subsubsection{01}\label{section}

FELIPE ENCONTROU ALGUMAS PLACAS GUARDADAS NO ARMÁRIO DA BIBLIOTECA. A
PLACA EM QUE SÓ APARECEM LETRAS É

(A)
\includegraphics[width=2.16042in,height=2.19514in]{media/image82.png}

(B)

(C)

(D)

IMAGEM ELABORADA PELO AUTOR

Saeb

Relacionar elementos sonoros das palavras com sua representação escrita.

BNCC EF01LP04 Distinguir as letras do alfabeto de outros sinais
gráficos.

(A ) Correta. Na placa aparece apenas letras do alfabeto.

(B) Incorreta. Na placa aparece outros sinais gráficos como números e o
@.

(C) Incorreta. Na placa aparece outros sinais gráficos como números \$ e
\%.

(D) Incorreta. Na placa aparece outros sinais gráficos como números e o
\&.

\protect\hypertarget{_heading=h.n783suxaxbxa}{}{}

\protect\hypertarget{_heading=h.pxfg7nlkhght}{}{}

\subsubsection{02}\label{section-1}

\textbf{OBSERVE O BRINQUEDO QUE ANA GANHOU EM UM SORTEIO.}

\includegraphics[width=1.65484in,height=1.36089in]{media/image83.png}

Disponível
em:\textbf{\href{https://www.freepik.com/premium-vector/electric-scooter-kick-scooter-vector-illustration_28324432.htm?query=PATINETE\#from_view=detail_alsolike}{\emph{https://www.freepik.com/premium-vector/electric-scooter-kick-scooter-vector-illustration\_28324432.htm?query=PATINETE\#from\_view=detail\_alsolike}}.Acesso
em 10 fev 2023.}

\textbf{O SOM INICIAL DO NOME DO BRINQUEDO DE ANA É}

\textbf{(A) PA.}

\textbf{(B) NE.}

\textbf{(C)GA.}

\textbf{( D)TE.}

Saeb

Relacionar elementos sonoros das palavras com sua representação escrita.

BNCC EF01LP05 Reconhecer o sistema de escrita alfabética como
representação dos sons da fala.

(A) Está correta, pois a palavra começa com o som pa.

(B) Está incorreta, pois o ne não é o som inicial ele está no meio da
palavra.

(C) Está incorreta, pois pode ter confundido os sons do pa com ga. .

(D) Está incorreta, pois essa é o som final.

\subsubsection{03}\label{section-2}

PEDRO ESTAVA PASSANDO NO JARDIM DA SUA CASA E ENCONTROU UM PAPEL COM UMA
PALAVRA NA GRAMA.VEJA.

\includegraphics[width=2.73976in,height=1.64660in]{media/image84.png}

Disponível
em:\href{https://www.freepik.com/free-vector/flat-design-spring-landscape-illustrated_12239756.htm\#query=jardim\&position=2\&from_view=search\&track=sph}{\emph{https://www.freepik.com/free-vector/flat-design-spring-landscape-illustrated\_12239756.htm\#query=jardim\&position=2\&from\_view=search\&track=sph}}.
Acesso em 11 de fev 2023.

QUAL PALAVRA COMEÇA COM O SOM DA SÍLABA FINAL DA PALAVRA QUE PEDRO
ENCONTROU?

(A) TELEFONE

(B) BONECA.

(C) POMADA.

(D) TAPETE.

\protect\hypertarget{_heading=h.2et92p0}{}{}SAEB Relacionar elementos
sonoros das palavras com sua representação escrita.

BNCC EF01LP09:Comparar palavras, identificando semelhanças e diferenças
entre sons de sílabas iniciais, mediais e finais.

\protect\hypertarget{_heading=h.tyjcwt}{}{}(A) Está incorreta, por
considerar o som inicial te que também tem na palavra peteca.

(B) Está correta, pois a palavra termina com o mesmo som.

(C) Está incorreta. pois confundiu o som de da com ca.

(D) Está incorreta, pois confundiu o som medial com a final.

\section{\texorpdfstring{\\
}{ }}\label{section-3}

\section{2. LENDO PALAVRAS E FRASES
}\label{muxf3dulo-2-lendo-palavras-e-frases}

Habilidades do SAEB

\protect\hypertarget{_heading=h.3dy6vkm}{}{}Ler palavras.

Escrever palavras.

Ler frases.

Habilidades da BNCC 
EF01LP01, EF12LP01, EF01LP02, EF01LP13.

\subsection{CONTEÚDO}\label{conteuxfado-1}

\protect\hypertarget{_heading=h.4d34og8}{}{}Para iniciar as atividades
do módulo leve a música "Canoa Virou" em um cartaz. Fazer a leitura com as
crianças explorando como ler as palavras e os textos. 

VOCÊ CONSEGUE LER ESSA PALAVRA?

CANOA

E ESSA FRASE?

A CANOA VIROU.

PARA ESCREVER ESSA PALAVRA FOI USADA AS LETRAS E SEUS SONS.

JÁ PARA ESCREVER ESSA FRASE FOI USADA AS PALAVRAS.

E PARA ESCREVER OS TEXTO É PRECISO USAR UM CONJUNTO DE FRASES.VEJA:

\textbf{A CANOA VIROU}

A CANOA VIROU

POIS DEIXARAM VIRAR

FOI POR CAUSA DE MARIA

QUE NÃO SOUBE REMAR.

\includegraphics[width=1.02083in,height=1.63125in]{media/image85.jpg}

\protect\hypertarget{_heading=h.2s8eyo1}{}{}\textbf{VOCÊ SABIA?}

QUANDO VAMOS LER E ESCREVER UMA PALAVRA OU UM TEXTO INICIAMOS DA
ESQUERDA PARA A DIREITA.

OBSERVE:

\textbf{CANOA}

TAMBÉM LEMOS E ESCREVEMOS DE CIMA PARA BAIXO.

OBSERVE:

A CANOA VIROU

POIS DEIXARAM VIRAR

http://www.dominiopublico.gov.br/download/texto/me000588.pdf

\subsection{ATIVIDADES}\label{atividades-1}

\textbf{Para atividade leve a parlenda o macaco foi a feira em um
cartaz. Fazer a leitura com as crianças explorando leitura de palavras,
frase e do texto inteiro mostrado sempre como começa a leitura.
Habilidades BNCC} EF01LP01 .

\subsubsection{1. LEIA A PALAVRA.}\label{leia-a-palavra.}

\textbf{MACACO}

A) ENCONTRE E ESCREVA A LETRA QUE INICIA A PALAVRA.

\_\_\_\_\_\_\_\_\_\_\_\_\_\_\_\_\_\_\_\_\_\_\_\_\_\_\_\_\_\_\_\_\_\_\_\_\_\_\_\_\_\_\_\_\_\_\_\_\_\_\_\_\_\_\_\_\_\_\_\_\_\_\_

Resposta: m

B) CIRCULE A ÚLTIMA LETRA DA PALAVRA.

\subsubsection{2. AGORA LEIA A FRASE.}\label{agora-leia-a-frase.}

\textbf{O MACACO FOI A FEIRA.}

A) CIRCULE A PRIMEIRA PALAVRA DA FRASE.

B) ENCONTE E COPIE A ÚLTIMA PALAVRA.

\_\_\_\_\_\_\_\_\_\_\_\_\_\_\_\_\_\_\_\_\_\_\_\_\_\_\_\_\_\_\_\_\_\_\_\_\_\_\_\_\_\_\_\_\_\_\_\_\_\_\_\_\_\_\_\_\_\_\_\_\_\_\_

Resposta: Feira

\subsubsection{3. LEIA:}\label{leia}

O MACACO FOI À FEIRA

NÃO SABIA O QUE COMPRAR

COMPROU UMA CADEIRA

PARA A COMADRE SE SENTAR

\protect\hypertarget{_heading=h.17dp8vu}{}{}http://www.dominiopublico.gov.br/download/texto/me000588.pdf

A) PINTE DE AZUL A PRIMEIRA PALAVRA DO TEXTO.

B) AGORA ESCREVA A PALAVRA QUE VOCÊ PINTOU.

\_\_\_\_\_\_\_\_\_\_\_\_\_\_\_\_\_\_\_\_\_\_\_\_\_\_\_\_\_\_\_\_\_\_\_\_\_\_\_\_\_\_\_\_\_\_\_\_\_\_

C) PINTE DE VERDE A ÚTIMA PALAVRA DO TEXTO E ESCREVA NO ESPAÇO ABAIXO.

\_\_\_\_\_\_\_\_\_\_\_\_\_\_\_\_\_\_\_\_\_\_\_\_\_\_\_\_\_\_\_\_\_\_\_\_\_\_\_\_\_\_\_\_\_\_\_\_\_\_\_\_\_\_\_\_\_\_\_\_\_

\subsubsection{4. DESENHE O ANIMAL QUE APARECE NA PARLENDA E ESCREVA O
SEU
NOME.}\label{desenhe-o-animal-que-aparece-na-parlenda-e-escreva-o-seu-nome.}

O aluno deverá desenhar um macaco.

\subsubsection{5. VAMOS CANTAR}\label{vamos-cantar}

Cantar e dançar a música com as crianças na sala, explorar a palavra que
inicia a música sílaba inicial, medial e final bater palmas para
descobrir quantas sílabas tem a palavra.

PIRULITO QUE BATE, BATE

PIRULITO QUE JÁ BATEU

QUEM GOSTA DE MIM É ELA

QUEM GOSTA DELA SOU EU.

\textbf{http://www.dominiopublico.gov.br/download/texto/me000588.pdf}

A) ENCONTRE E CIRCULE A PALAVRA QUE INICIA A MÚSICA.

B) COPIE A PALAVRA QUE VOCÊ PINTOU.

\begin{longtable}[]{@{}llll@{}}
\toprule
\textbf{PI} & \textbf{RU} & \textbf{LI} & \textbf{TO}\tabularnewline
\bottomrule
\end{longtable}

C) QUANTAS VEZES VOCÊ ABRE A BOCA PRA FALAR ESSA PALAVRA?

\_\_\_\_\_\_\_\_\_\_\_\_\_\_\_\_\_\_\_\_\_\_\_\_\_\_\_\_\_\_\_\_\_\_\_\_\_\_\_\_\_\_\_\_\_\_\_\_\_\_\_\_\_\_\_\_\_\_\_\_\_\_\_\_\_\_\_\_

D) AGORA PINTE A ÚLTIMA PALAVRA DA MÚSICA.

\subsubsection{6. CIRCULE E COPIE A ÚLTIMA SÍLABA DOS NOMES DAS
FIGURAS.}\label{circule-e-copie-a-uxfaltima-suxedlaba-dos-nomes-das-figuras.}

\includegraphics[width=1.22307in,height=1.24422in]{media/image86.jpg}\includegraphics[width=1.53669in,height=1.29710in]{media/image87.jpg}\includegraphics[width=1.09375in,height=1.09375in]{media/image88.png}\includegraphics[width=1.03854in,height=1.20356in]{media/image89.jpg}

ALFACE COROA GAIOLA FOGUEIRA

\href{https://www.freepik.com/free-vector/king-queen-golden-crowns_4393665.htm\#query=coroa\&position=0\&from_view=search\&track=sph}{\emph{https://www.freepik.com/free-vector/king-queen-golden-crowns\_4393665.htm\#query=coroa\&position=0\&from\_view=search\&track=sph}}

https://www.freepik.com/free-icon/cage\_14903361.htm\#query=GAIOLA\&position=48\&from\_view=search\&track=sph

\href{https://www.freepik.com/premium-vector/wood-campfire-vector-drawing-bonfire-fire-burning-wooden-logs-camp-fireplace-cartoon-drawing_26800211.htm\#query=FOGUEIRA\&position=35\&from_view=search\&track=sp}{\emph{https://www.freepik.com/premium-vector/wood-campfire-vector-drawing-bonfire-fire-burning-wooden-logs-camp-fireplace-cartoon-drawing\_26800211.htm\#query=FOGUEIRA\&position=35\&from\_view=search\&track=sp}}

\subsubsection{7. OBSERVE A NUMERAÇÃO DE CADA ANIMAL E PREECHA A
CRUZADINHA COM OS SEUS
NOMES.}\label{observe-a-numerauxe7uxe3o-de-cada-animal-e-preecha-a-cruzadinha-com-os-seus-nomes.}

Utilizar alfabeto móvel para realizar essa atividade. Formar com as
crianças os nomes dos desenhos para escrever na cruzadinha.

\includegraphics[width=4.90625in,height=3.34375in]{media/image90.jpg}

\includegraphics[width=1.06250in,height=0.94306in]{media/image91.jpg}\includegraphics[width=1.24583in,height=0.90625in]{media/image92.jpg}\includegraphics[width=0.97639in,height=0.92708in]{media/image93.jpg}

\includegraphics[width=1.25000in,height=0.99097in]{media/image94.jpg}

\includegraphics[width=0.99722in,height=1.03125in]{media/image95.jpg}

\includegraphics[width=5.90556in,height=4.47101in]{media/image96.jpg}

\href{https://www.freepik.com/free-vector/beagle-dog-cartoon-character-white-background_20438304.htm?query=COELHO\#from_view=detail_alsolike}{\emph{https://www.freepik.com/free-vector/beagle-dog-cartoon-character-white-background\_20438304.htm?query=COELHO\#from\_view=detail\_alsolike}}

\href{https://www.freepik.com/free-vector/white-rabbit-cartoon-white-background_18481938.htm\#query=COELHO\&position=25\&from_view=search\&track=sph}{\emph{https://www.freepik.com/free-vector/white-rabbit-cartoon-white-background\_18481938.htm\#query=COELHO\&position=25\&from\_view=search\&track=sph}}

\href{https://www.freepik.com/free-vector/smiling-cats-collection_1121437.htm\#from_view=detail_alsolike}{\emph{https://www.freepik.com/free-vector/smiling-cats-collection\_1121437.htm\#from\_view=detail\_alsolike}}

\subsubsection{8. ENCONTRE NO CAÇA-PALAVRAS OS NOMES DAS FRUTAS. DEPOIS
COPIE NOS LUGARES
INDICADOS.}\label{encontre-no-cauxe7a-palavras-os-nomes-das-frutas.-depois-copie-nos-lugares-indicados.}

Utilizar alfabeto móvel para realizar essa atividade. Formar com as
crianças os nomes dos desenhos para escrever na cruzadinha.

\includegraphics[width=3.58333in,height=3.51667in]{media/image97.jpg}

\includegraphics[width=0.92708in,height=0.84375in]{media/image98.jpg}

\includegraphics[width=1.02083in,height=0.67986in]{media/image99.jpg}

\includegraphics[width=1.38958in,height=0.65208in]{media/image100.jpg}\includegraphics[width=0.71875in,height=0.71875in]{media/image101.jpg}

\includegraphics[width=0.89097in,height=0.85417in]{media/image102.jpg}\includegraphics[width=1.07847in,height=0.97917in]{media/image103.jpg}

\_\_\_\_\_\_\_\_\_\_\_\_\_\_\_\_\_\_\_\_\_\_\_
\_\_\_\_\_\_\_\_\_\_\_\_\_\_\_\_\_\_\_\_\_\_\_\_\_\_\_\_

Respostas: banana, morango, acerola, uva, maçã.

\href{https://www.freepik.com/premium-vector/strawberry-vector-illustratiom_23633177.htm?query=MORANGO\#from_view=detail_alsolike}{\emph{https://www.freepik.com/premium-vector/strawberry-vector-illustratiom\_23633177.htm?query=MORANGO\#from\_view=detail\_alsolike}}

RESPOSTA CAÇA-PALAVRAS

\includegraphics[width=3.23199in,height=2.51443in]{media/image104.jpg}

\subsubsection{9. TRADUZA OS CÓDIGOS E ESCREVA
PALAVRAS.}\label{traduza-os-cuxf3digos-e-escreva-palavras.}

\begin{longtable}[]{@{}llll@{}}
\toprule
& & &\tabularnewline
\textbf{BA} & \textbf{SA} & \textbf{LÃO} & \textbf{LA}\tabularnewline
& & &\tabularnewline
\textbf{PA} & \textbf{TINS} & \textbf{PAI} & \textbf{TO}\tabularnewline
\bottomrule
\end{longtable}

A) + = \_\_\_\_\_\_\_\_\_\_\_\_\_\_\_balão

B) + + + = \_\_\_\_\_\_\_\_\_\_\_\_\_\_\_\_\_sapato

C) + + = \_\_\_\_\_\_\_\_\_\_\_\_\_\_\_\_\_\_\_\_papai

D) + = \_\_\_\_\_\_\_\_\_\_\_\_\_\_\_\_\_\_\_\_\_\_bala

E) + = \_\_\_\_\_\_\_\_\_\_\_\_\_\_\_\_\_\_\_\_\_patins

F) + =\_\_\_\_\_\_\_\_\_\_\_\_\_\_\_\_\_\_\_sala

\subsubsection{10. DESEMBARALHE AS PALAVRAS E FORME UMA
FRASE.}\label{desembaralhe-as-palavras-e-forme-uma-frase.}

\begin{longtable}[]{@{}lll@{}}
\toprule
AZUL. & MENINO & JOGOU\tabularnewline
O & BOLA & A\tabularnewline
\bottomrule
\end{longtable}

O menino jogou a bola azul.

\begin{longtable}[]{@{}llll@{}}
\toprule
A & BONECA & ÁGUA & CAIU\tabularnewline
NA & ALICE & DE\tabularnewline
\bottomrule
\end{longtable}

A boneca de Alice caiu na água.

\subsubsection{11. MARQUE UM X NAS FIGURAS CUJO O NOME COMEÇA IGUAL A
PALAVRA
CEBOLA.}\label{marque-um-x-nas-figuras-cujo-o-nome-comeuxe7a-igual-a-palavra-cebola.}

Levar a palavra cebola em fichas recortada em sílabas. Formar os
nomes dos desenhos com as letras móveis para fazer comparação das
sílabas das palavras.

\begin{longtable}[]{@{}lllll@{}}
\toprule
\includegraphics[width=1.36272in,height=0.72875in]{media/image105.jpg} &
\includegraphics[width=0.90977in,height=1.00720in]{media/image106.jpg} &
\includegraphics[width=0.98256in,height=1.40209in]{media/image107.jpg} &
\includegraphics[width=1.12500in,height=1.12500in]{media/image108.jpg} &
\includegraphics[width=1.13713in,height=1.09392in]{media/image109.jpg}\tabularnewline
\textbf{( ) } & \textbf{( )} & \textbf{( x ) } & \textbf{( ) } &
\textbf{( ) }\tabularnewline
\bottomrule
\end{longtable}

\href{https://www.freepik.com/free-vector/carrot-with-green-leaves-vector-isolated_29314762.htm\#page=2\&query=cenoura\&position=43\&from_view=search\&track=sph}{\textbf{\emph{https://www.freepik.com/free-vector/carrot-with-green-leaves-vector-isolated\_29314762.htm\#page=2\&query=cenoura\&position=43\&from\_view=search\&track=sph}}}

\href{https://www.freepik.com/free-vector/certified-approved-rubber-stamps-seal-set_8289944.htm\#query=selo\&position=14\&from_view=search\&track=sph}{\textbf{\emph{https://www.freepik.com/free-vector/certified-approved-rubber-stamps-seal-set\_8289944.htm\#query=selo\&position=14\&from\_view=search\&track=sph}}}

\href{https://www.freepik.com/free-psd/premium-mobile-phone-screen-mockup-template_3891016.htm\#query=celular\&position=20\&from_view=search\&track=sph}{\textbf{\emph{https://www.freepik.com/free-psd/premium-mobile-phone-screen-mockup-template\_3891016.htm\#query=celular\&position=20\&from\_view=search\&track=sph}}}

\href{https://www.freepik.com/free-vector/black-mug-isolated_10602866.htm\#query=caneca\&position=34\&from_view=search\&track=sph}{\textbf{\emph{https://www.freepik.com/free-vector/black-mug-isolated\_10602866.htm\#query=caneca\&position=34\&from\_view=search\&track=sph}}}

\href{https://www.freepik.com/premium-vector/cute-baby-shower-cartoon_5320893.htm\#query=cegonha\&position=12\&from_view=search\&track=sph}{\textbf{\emph{https://www.freepik.com/premium-vector/cute-baby-shower-cartoon\_5320893.htm\#query=cegonha\&position=12\&from\_view=search\&track=sph}}}

\subsubsection{12. COPIE A PRIMEIRA E A ÚLTIMA SÍLABA DO NOME DO
DESENHO.}\label{copie-a-primeira-e-a-uxfaltima-suxedlaba-do-nome-do-desenho.}

\begin{longtable}[]{@{}ll@{}}
\toprule
\begin{minipage}[t]{0.48\columnwidth}\raggedright\strut
\includegraphics[width=1.08333in,height=1.19792in]{media/image110.jpg}

\textbf{ÁRVORE}\strut
\end{minipage} & \begin{minipage}[t]{0.48\columnwidth}\raggedright\strut
SÍLABA INICIAL\strut
\end{minipage}\tabularnewline
& SÍLABA FINAL\tabularnewline
\begin{minipage}[t]{0.48\columnwidth}\raggedright\strut
\includegraphics[width=1.47778in,height=0.91667in]{media/image111.jpg}

\textbf{BORBOLETA}\strut
\end{minipage} & \begin{minipage}[t]{0.48\columnwidth}\raggedright\strut
SÍLABA INICIAL\strut
\end{minipage}\tabularnewline
& SÍLABA FINAL\tabularnewline
\begin{minipage}[t]{0.48\columnwidth}\raggedright\strut
\includegraphics[width=1.11458in,height=1.08403in]{media/image112.jpg}

\textbf{ARARA}\strut
\end{minipage} & \begin{minipage}[t]{0.48\columnwidth}\raggedright\strut
SÍLABA INICIAL\strut
\end{minipage}\tabularnewline
& SÍLABA FINAL\tabularnewline
\textbf{JARDIM}\includegraphics[width=2.53125in,height=1.44792in]{media/image113.jpg}
& SÍLABA INICIAL\tabularnewline
& SÍLABA FINAL\tabularnewline
\bottomrule
\end{longtable}

\href{https://www.freepik.com/free-vector/set-simple-tree_3875024.htm\#page=2\&query=ARVORE\&position=40\&from_view=search\&track=sph}{\textbf{\emph{https://www.freepik.com/free-vector/set-simple-tree\_3875024.htm\#page=2\&query=ARVORE\&position=40\&from\_view=search\&track=sph}}}

\href{https://www.freepik.com/premium-photo/blue-butterfly-morpho-anaxibia-isolated_5001266.htm\#query=BORBOLETA\&position=24\&from_view=search\&track=sph}{\textbf{\emph{https://www.freepik.com/premium-photo/blue-butterfly-morpho-anaxibia-isolated\_5001266.htm\#query=BORBOLETA\&position=24\&from\_view=search\&track=sph}}}

\href{https://www.freepik.com/free-vector/cute-parrot-bird-branch-cartoon-animal-wildlife-icon-concept-isolated-flat-cartoon-}{\textbf{\emph{https://www.freepik.com/free-vector/cute-parrot-bird-branch-cartoon-animal-wildlife-icon-concept-isolated-flat-cartoon-}}}\protect\hypertarget{_heading=h.hanzzj3wwcdx}{}{\protect\hypertarget{_heading=h.bw5gy3j8k2c7}{}{}}

\subsubsection{13. NUMERE OS DESENHOS DE ACORDO COM O NÚMERO DAS
PALAVRAS.}\label{numere-os-desenhos-de-acordo-com-o-nuxfamero-das-palavras.}

\begin{longtable}[]{@{}llll@{}}
\toprule
\begin{minipage}[t]{0.24\columnwidth}\raggedright\strut
\includegraphics[width=1.31458in,height=1.32014in]{media/image114.jpg}\strut
\end{minipage} & \begin{minipage}[t]{0.24\columnwidth}\raggedright\strut
\includegraphics[width=0.93056in,height=0.76042in]{media/image115.jpg}

8\strut
\end{minipage} & \begin{minipage}[t]{0.24\columnwidth}\raggedright\strut
\includegraphics[width=0.86956in,height=0.97145in]{media/image116.jpg}

2\strut
\end{minipage} & \begin{minipage}[t]{0.24\columnwidth}\raggedright\strut
\includegraphics[width=0.84722in,height=0.84722in]{media/image117.jpg}

1\strut
\end{minipage}\tabularnewline
\begin{minipage}[t]{0.24\columnwidth}\raggedright\strut
\includegraphics[width=0.95347in,height=0.54444in]{media/image118.jpg}

3\strut
\end{minipage} & \begin{minipage}[t]{0.24\columnwidth}\raggedright\strut
\includegraphics[width=1.00972in,height=0.84375in]{media/image119.jpg}

7\strut
\end{minipage} & \begin{minipage}[t]{0.24\columnwidth}\raggedright\strut
\includegraphics[width=0.96875in,height=0.99167in]{media/image120.jpg}

4\strut
\end{minipage} & \begin{minipage}[t]{0.24\columnwidth}\raggedright\strut
\includegraphics[width=1.30694in,height=0.88542in]{media/image121.jpg}

5\strut
\end{minipage}\tabularnewline
& & &\tabularnewline
\begin{minipage}[t]{0.24\columnwidth}\raggedright\strut
BAMBOLÊ

1\strut
\end{minipage} & \begin{minipage}[t]{0.24\columnwidth}\raggedright\strut
MACACO

2\strut
\end{minipage} & \begin{minipage}[t]{0.24\columnwidth}\raggedright\strut
ÁRMARIO

3\strut
\end{minipage} & \begin{minipage}[t]{0.24\columnwidth}\raggedright\strut
ABELHA

4\strut
\end{minipage}\tabularnewline
\begin{minipage}[t]{0.24\columnwidth}\raggedright\strut
SAPATO

5\strut
\end{minipage} & \begin{minipage}[t]{0.24\columnwidth}\raggedright\strut
PAPAGAIO

6\strut
\end{minipage} & \begin{minipage}[t]{0.24\columnwidth}\raggedright\strut
ANEL

7\strut
\end{minipage} & \begin{minipage}[t]{0.24\columnwidth}\raggedright\strut
MANGUEIRA

8\strut
\end{minipage}\tabularnewline
\bottomrule
\end{longtable}

\textbf{https://www.freepik.com/premium-vector/parrot-cartoon-vector-illustration\_4852095.htm\#query=PAPAGAIO\&position=38\&from\_view=search\&track=sph}

\href{https://www.freepik.com/premium-vector/watermelon-fruit-vector_18313713.htm\#query=MELANCIA\&position=47\&from_view=search\&track=sph}{\emph{https://www.freepik.com/premium-vector/watermelon-fruit-vector\_18313713.htm\#query=MELANCIA\&position=47\&from\_view=search\&track=sph}}

\href{https://img.freepik.com/premium-photo/green-garden-hose-isolated_51524-4720.jpg}{\emph{https://img.freepik.com/premium-photo/green-garden-hose-isolated\_51524-4720.jpg}}

\href{https://www.freepik.com/free-vector/monkey-climbing-up-vine_20424275.htm\#query=MACACO\&position=17\&from_view=search\&track=sph}{\emph{https://www.freepik.com/free-vector/monkey-climbing-up-vine\_20424275.htm\#query=MACACO\&position=17\&from\_view=search\&track=sph}}

\href{https://www.freepik.com/free-vector/blue-locker-closed-cabinet-with-locks-doors-storage-clothes-public-sport-gym-school-office-security-closet-wardrobe-cupboard-isolated-white_28945982.htm\#query=ARMARIO\&position=2\&from_view=search\&track=sph}{\emph{https://www.freepik.com/free-vector/blue-locker-closed-cabinet-with-locks-doors-storage-clothes-public-sport-gym-school-office-security-closet-wardrobe-cupboard-isolated-white\_28945982.htm\#query=ARMARIO\&position=2\&from\_view=search\&track=sph}}

\href{https://www.freepik.com/premium-photo/golden-ring-isolated-white-background_4632271.htm\#query=ANEL\&position=32\&from_view=search\&track=sph}{\emph{https://www.freepik.com/premium-photo/golden-ring-isolated-white-background\_4632271.htm\#query=ANEL\&position=32\&from\_view=search\&track=sph}}

https://www.freepik.com/free-vector/cute-cartoon-bee-characters-listening-music-headphones-saying-hi-holding-pencil-set\_28189509.htm\#query=ABELHA\&position=38\&from\_view=search\&track=sph

\subsubsection{14. OBSERVE A IMAGEM E ESCREVA UMA
FRASE.}\label{observe-a-imagem-e-escreva-uma-frase.}

Explorar os elementos da imagem com as crianças cores, formas e
situação.

\includegraphics[width=3.88542in,height=2.03581in]{media/image122.jpg}

Disponível
em:https://www.freepik.com/premium-vector/kids-playing-street-side\_2172119.htm\#query=MENINO\%20DE\%20BICICLETA\&position=45\&from\_view=search\&track=ais.acesso
em 11 de fev 2023.

\_\_\_\_\_\_\_\_\_\_\_\_\_\_\_\_\_\_\_\_\_\_\_\_\_\_\_\_\_\_\_\_\_\_\_\_\_\_\_\_\_\_\_\_\_\_\_\_\_\_\_\_\_\_\_\_\_\_

As crianças brincam no parque.

\subsection{TREINO}\label{treino-1}

\textbf{Os três itens a seguir estão ordenados do mais fácil para o mais
difícil. }

\subsubsection{01 }\label{section-4}

VEJA A
IMAGEM:\includegraphics[width=1.82292in,height=1.84306in]{media/image123.jpg}

Disponivel
em:https://www.freepik.com/free-vector/vintage-tv\_763025.htm\#query=TELEVIS\%C3\%83O\&position=12\&from\_view=search\&track=sph.
Acesso dia 18 fev 2023.

A PRIMEIRA SÍLABA QUE FORMA O NOME DA FIGURA É

A) TE.

B) ET.

C) PE

D) LE

Saeb . Ler palavras.

BNCC EF12LP01 Ler palavras novas com precisão na decodificação, no caso
de palavras de uso frequente, ler globalmente, por memorização.

(A) Está correta, pois a palavra começa com o som te

(B) Está incorreta, pois o ne confundiu a ordem do som das letras.

\protect\hypertarget{_heading=h.3rdcrjn}{}{}(C) Está incorreta, pois
pode ter confundido os sons do te com pe

(D) Está incorreta, pois pode ter confundido os sons do te com le.

\subsubsection{02 }\label{section-5}

LEIA FRASE.

PAULO GOSTA DE PASSEAR NO PARQUE.

QUAL É A PRIMEIRA PALAVRA DA FRASE?

A) GOSTA.

B) PAULO

C) PARQUE.

D) PASSEAR.

Saeb Ler frases

BNCC EF01LP01 Reconhecer que textos são lidos e escritos da esquerda
para a direita e de cima para baixo da página.

(A) Está incorreta, pois essa palavra aparece no meio da frase.

(B) Está correta, pois a frase começa com essa palavra.

.(C) Está incorreta, por confundido a frase final com a inicial.

(D) Está incorreta, pois não observou que essa palavra está no meio da
frase.

\subsubsection{03 }\label{section-6}

LEIA

FUI AO TORORÓ

BEBER ÁGUA E NÃO ACHEI

ENCONTREI BELA MORENA

QUE NO TORORÓ DEIXEI.

A PALAVRA QUE TERMINA O TEXO É

A) FUI.

B) QUE.

C) ACHEI.

D) DEIXE.

Saeb Ler frases

BNCC. EF01LP01 Reconhecer que textos são lidos e escritos da esquerda
para a direita e de cima para baixo da página.

(A) Está incorreta, pois não se atentou a direção da leitura.

(B) Está incorreta, pois confundiu com primeira palavra da última frase.

(c) Está incorreta, pois confundiu com a última palavra de uma das
frases.

(D) Está correta, pois o texto termina com essa palavra.

\section{\texorpdfstring{\\
}{ }}\label{section-7}

\section{3. ENCONTRANDO
INFORMAÇÕES}\label{modulo-3-encontrando-informauxe7uxf5es}

\protect\hypertarget{_heading=h.lnxbz9}{}{}Habilidade do SAEB 
Localizar informações explícitas em textos.

Habilidade da BNCC 
EF15LP03.

\subsection{CONTEÚDO}\label{conteuxfado-2}

VAMOS OBSERVAR A IMAGEM DO CARTAZ.

\protect\hypertarget{_heading=h.35nkun2}{}{}Para iniciar o módulo você
poderá falar o tipo do texto será lido explorar bastante a oralidade
fazendo diversos questionamentos de informações que estão nos textos.

\includegraphics[width=5.14986in,height=2.17762in]{media/image124.jpg}

https://www.riopardo.rs.gov.br/portal/noticias/0/3/1477/21-campanha-nacional-de-vacinacao-contra-a-influenza

VOCÊ SABE O QUE ESSE CARTAZ ANUNCIA?

PARA RESPONDER À PERGUNTA VOCÊ PRECISA BUSCAR AS INFORMAÇÕES NO TEXTO
POIS ELA ESTÁ BEM CLARA BASTA OBSERVAR COM ATENÇÃO.

ELE ESTA ANUNSIANDO A CAMPANHA DE VACINAÇÃO CONTRA A GRIPE INFLUENZA QUE
ACONTECERÁ ENTRE OS DIA 10 DE ABRIL A 31 DE MAIO.

VEJA ESSE OUTRO
TEXTO\includegraphics[width=1.50833in,height=1.35903in]{media/image125.jpg}

\textbf{A CASINHA DA VOVÓ}

A CASINHA DA VOVÓ

\emph{CERCADINHA DE CIPÓ}

O CAFÉ ESTÁ DEMORANDO

COM CERTEZA NÃO TEM PÓ.

COMO É A CASA DA VOVÒ? A CASA DA VOVÓ É CERCADINHA DE SIPÓ.

VEJA QUE ESSA INFORMAÇÃO APARECE NO TEXTO.

FIQUE LIGADOPARA LOCALIZAR INFORMAÇÕES EXPLICITAS EM TEXTO BASTA LER COM
ATENÇÃO POIS ELA VAI ESTÁ LÁ BEM CLARA NO TEXTO COLOCAR EM UM BOX AO
LADO DA PÁGINA.

\subsection{ATIVIDADES}\label{atividades-2}

Para realizar as atividades a seguir você poderá fazer a leitura com os
alunos e explorar bastante a oralidade dos diferentes tipos de textos
fazendo diversos questionamentos de informações que estão nos textos.

\includegraphics[width=2.02153in,height=2.05208in]{media/image126.jpg}

\subsubsection{LEIA O TEXTO.}\label{leia-o-texto.}

\textbf{POMBINHA BRANCA}

POMBINHA BRANCA

O QUE ESTÁ FAZENDO

LAVANDO A ROUPA

DO CASAMENTO.

\protect\hypertarget{_heading=h.1ksv4uv}{}{}http://www.dominiopublico.gov.br/download/texto/me000588.pdf

https://www.freepik.com/free-vector/hand-drawn-dove-outline-illustration\_22340867.htm\#query=POMBA\&position=6\&from\_view=search\&track=sph

\subsubsection{1. ENCONTRE E PINTE O TÍTULO DO TEXTO COM A COR
VERDE.}\label{encontre-e-pinte-o-tuxedtulo-do-texto-com-a-cor-verde.}

\subsubsection{2. QUAL É A COR DA
POMBINHA?}\label{qual-uxe9-a-cor-da-pombinha}

\_\_\_\_\_\_\_\_\_\_\_\_\_\_\_\_\_\_\_\_\_\_\_\_\_\_\_\_\_\_\_\_\_\_\_\_\_\_\_\_\_\_\_\_\_\_\_\_\_\_\_\_\_\_\_\_\_\_\_\_\_\_\_\_\_\_\_\_\_\_

R: Branca

\subsubsection{3. O QUE A POMBINHA ESTÁ
FAZENDO?}\label{o-que-a-pombinha-estuxe1-fazendo}

\_\_\_\_\_\_\_\_\_\_\_\_\_\_\_\_\_\_\_\_\_\_\_\_\_\_\_\_\_\_\_\_\_\_\_\_\_\_\_\_\_\_\_\_\_\_\_\_\_\_\_\_\_\_\_\_\_\_\_\_\_\_\_\_\_\_\_\_\_\_

R: Lavando roupa

\subsubsection{LEIA AS INFORMAÇÕES DO
CARTAZ.}\label{leia-as-informauxe7uxf5es-do-cartaz.}

\includegraphics[width=4.08681in,height=4.04722in]{media/image127.jpg}

Disponível em
:https://jundiai.sp.gov.br/noticias/2014/02/04/jundiai-retoma-levantamento-larvario-contra-dengue/Acesso
dia 18 fev 2023.

AGORA RESPONDA:

\subsubsection{4. QUAL DOENÇA O CARTAZ SE REFERE?
}\label{qual-doenuxe7a-o-cartaz-se-refere}

\_\_\_\_\_\_\_\_\_\_\_\_\_\_\_\_\_\_\_\_\_\_\_\_\_\_\_\_\_\_\_\_\_\_\_\_\_\_\_\_\_\_\_\_\_\_\_\_\_\_\_\_\_\_\_\_\_\_\_\_\_\_\_\_\_\_\_\_\_\_

R:A dengue

\subsubsection{5. QUAL PEDIDO VEM ESCRITO NO CARTAZ PARA ACABAR COM A
DENGUE?}\label{qual-pedido-vem-escrito-no-cartaz-para-acabar-com-a-dengue}

\_\_\_\_\_\_\_\_\_\_\_\_\_\_\_\_\_\_\_\_\_\_\_\_\_\_\_\_\_\_\_\_\_\_\_\_\_\_\_\_\_\_\_\_\_\_\_\_\_\_\_\_\_\_\_\_\_\_\_\_\_\_\_\_\_\_\_\_\_\_

R:Elimine os criadores do mosquito da dengue.

\subsubsection{6. CIRCULE NO CARTAZ ALGUNS CUIDADOS PARA COMBATER A
DENGUE.}\label{circule-no-cartaz-alguns-cuidados-para-combater-a-dengue.}

Deve circular as imagens de cuidado no cartaz.

\subsubsection{LEIA COM ATENÇÃO}\label{leia-com-atenuxe7uxe3o}

\textbf{MEIGUICE}\includegraphics[width=2.83472in,height=2.19792in]{media/image128.jpg}

(ADELINA LOPES VIEIRA)

DERAM À LINDA CLARISSE

\protect\hypertarget{_heading=h.44sinio}{}{}UMA GATINHA MIMOSA,

TÃO BRANCA, TÃO CARINHOSA,

TÃO ENGRAÇADA, TÃO MANSA

QUE A ENCANTADORA CRIANÇA

POR NOME LHE PÔS --- MEIGUICE.

Disponível em:
\emph{\href{about:blank}{file:///C:/Users/User/Downloads/integrado\_-\_5o\_ano\_-\_daii\_-\_08022021\_0\%20(9).pdf}
Acesso em 18 Fev 2023.}

https://www.freepik.com/premium-vector/happy-cute-little-kid-boy-girl-play-with-pet-dog\_9558974.htm\#query=MENINA\%20COM\%20CACHORRO\&position=13\&from\_view=search\&track=ais

\subsubsection{7. O QUE CLARISSE GANHOU?}\label{o-que-clarisse-ganhou}

\_\_\_\_\_\_\_\_\_\_\_\_\_\_\_\_\_\_\_\_\_\_\_\_\_\_\_\_\_\_\_\_\_\_\_\_\_\_\_\_\_\_\_\_\_\_\_\_\_\_\_\_\_\_\_\_\_\_\_\_\_\_\_\_\_\_\_\_\_

R: Uma gatinha.

\subsubsection{8. COMO ERA A GATINHA DE
CLARISSE?}\label{como-era-a-gatinha-de-clarisse}

\_\_\_\_\_\_\_\_\_\_\_\_\_\_\_\_\_\_\_\_\_\_\_\_\_\_\_\_\_\_\_\_\_\_\_\_\_\_\_\_\_\_\_\_\_\_\_\_\_\_\_\_\_\_\_\_\_\_\_\_\_\_\_\_\_\_\_\_\_\_

R: uma gatinha mimosa, branca, carinhosa, engraçada, mansa

\subsubsection{9. ENCONTRE NO TEXTO O NOME QUE CLARISSE DEU PARA SUA
GATINHA.PINTE E DEPOIS
ESCREVA.}\label{encontre-no-texto-o-nome-que-clarisse-deu-para-sua-gatinha.pinte-e-depois-escreva.}

\_\_\_\_\_\_\_\_\_\_\_\_\_\_\_\_\_\_\_\_\_\_\_\_\_\_\_\_\_\_\_\_\_\_\_\_\_\_\_\_\_\_\_\_\_\_\_\_\_\_\_\_\_\_\_\_\_\_\_\_\_\_\_\_\_\_\_\_\_\_

\subsubsection{R: Meiguice.}\label{r-meiguice.}

\subsubsection{10. COMO VOCÊ IMAGINA QUE ERA A GATINHA DE CLARRISE?
DESENHE
AQUI.}\label{como-vocuxea-imagina-que-era-a-gatinha-de-clarrise-desenhe-aqui.}

R: Resposta pessoal.

\subsubsection{LEIA O TEXTO}\label{leia-o-texto}

\textbf{PINTINHO}

MEU PINTINHO AMARELINHO

CATA AQUI NA MINHA
MÃO,\includegraphics[width=2.51042in,height=2.51042in]{media/image129.png}

NA MINHA MÃO.

QUANDO QUER COMER BICHINHO

\protect\hypertarget{_heading=h.2jxsxqh}{}{}COM SEU PEZINHO

ELE CISCA O CHÃO.

ELE BATE AS ASAS

ELE FAZ PIU-PIU

MAS TEM MUITO MEDO DO GAVIÃO.

Disponível
em:http://www.dominiopublico.gov.br/download/texto/me000588.pdf .Acesso
em 18 Fev 2023.

https://www.freepik.com/free-icon/chicken\_14319139.htm\#query=PIMTINHO\%20AMARELINHO\&position=41\&from\_view=search\&track=ais

\subsubsection{11. ENCONTRE E CIRCULE O TÍTULO DO
TEXTO.}\label{encontre-e-circule-o-tuxedtulo-do-texto.}

\subsubsection{12. PINTE A COR DO PINTINHO NO
TEXTO.}\label{pinte-a-cor-do-pintinho-no-texto.}

\subsubsection{13. O QUE O PINTINHO FAZ QUANDO QUER COMER
BICHINHOS?}\label{o-que-o-pintinho-faz-quando-quer-comer-bichinhos}

\_\_\_\_\_\_\_\_\_\_\_\_\_\_\_\_\_\_\_\_\_\_\_\_\_\_\_\_\_\_\_\_\_\_\_\_\_\_\_\_\_\_\_\_\_\_\_\_\_\_\_\_\_\_\_\_\_\_\_\_

R: Com seu pezinho, ele cisca o chão.

\subsubsection{14. DE QUEM O PINTINHO TEM
MEDO?}\label{de-quem-o-pintinho-tem-medo}

\_\_\_\_\_\_\_\_\_\_\_\_\_\_\_\_\_\_\_\_\_\_\_\_\_\_\_\_\_\_\_\_\_\_\_\_\_\_\_\_\_\_\_\_\_\_\_\_\_\_\_\_\_\_\_\_\_\_\_\_\_

R:Do gavião.

\subsubsection[LEIA]{\texorpdfstring{LEIA\protect\includegraphics[width=1.61944in,height=1.07917in]{media/image130.jpg}}{LEIA}}\label{leia-1}

\textbf{BOLO DE IOGURTE}

\textbf{INGREDIENTES:}\\
3 OVOS\\
1 COPO DE IOGURTE NATURAL\\
1 COPO DE ÓLEO (USE O COPO DE IOGURTE PARA OBTER A QUANTIDADE CERTA)\\
1 ½ XÍCARA DE CHÁ DE AÇÚCAR REFINADO\\
2 XÍCARAS CHÁ DE FARINHA DE TRIGO\\
1 COLHER DE SOPA DE FERMENTO

\textbf{MODO DE PREPARO:}

COLOQUE OS OVOS NO LIQUIDIFICADOR E BATA. ACRESCENTE O IOGURTE E O ÓLEO
E DÊ UM LEVE PULSAR. INCLUA O AÇÚCAR E BATA MAIS UM POUCO. PONHA A
FARINHA EM UMA TIGELA...

Disponivel
em;\href{https://g1.globo.com/sao-paulo/sorocaba-jundiai/nosso-campo/noticia/2016/04/bolo-de-iogurte-facil-de-fazer-bonito-e-muito-gostoso.html.acesso}{\emph{https://g1.globo.com/sao-paulo/sorocaba-jundiai/nosso-campo/noticia/2016/04/bolo-de-iogurte-facil-de-fazer-bonito-e-muito-gostoso.html.acesso}}
12 de fev 2023.

https://www.freepik.com/free-photo/sweet-garnished-easter-cake-wooden-table\_1758020.htm\#query=bolo\%20de\%20iorgute\&position=22\&from\_view=search\&track=ais

\subsubsection{15. QUAL É O NOME DA
RECEITA?}\label{qual-uxe9-o-nome-da-receita}

\_\_\_\_\_\_\_\_\_\_\_\_\_\_\_\_\_\_\_\_\_\_\_\_\_\_\_\_\_\_\_\_\_\_\_\_\_\_\_\_\_\_\_\_\_\_\_\_\_\_\_\_\_\_\_\_\_\_\_\_\_\_\_\_\_\_\_

R:Bolo de iogurte.

\subsubsection{16. AGORA ESCREVA O NOME DE DOIS INGREDIENTES USADOS PARA
FAZER O
BOLO.}\label{agora-escreva-o-nome-de-dois-ingredientes-usados-para-fazer-o-bolo.}

\_\_\_\_\_\_\_\_\_\_\_\_\_\_\_\_\_\_\_\_\_\_\_\_\_\_\_\_\_\_\_\_\_\_\_\_\_\_\_\_\_\_\_\_\_\_\_\_\_\_\_\_\_\_\_\_\_\_\_\_\_\_\_\_\_\_\_\_\_\_

R: ovo e iogurte.

\subsubsection{17. ESCREVA O NOME DE UM UTENSÍLIO QUE FOI USADO PARA
FAZER A RECEITA.
}\label{escreva-o-nome-de-um-utensuxedlio-que-foi-usado-para-fazer-a-receita.}

\_\_\_\_\_\_\_\_\_\_\_\_\_\_\_\_\_\_\_\_\_\_\_\_\_\_\_\_\_\_\_\_\_\_\_\_\_\_\_\_\_\_\_\_\_\_\_\_\_\_\_\_\_\_\_\_\_\_\_\_\_\_\_\_\_\_\_\_

R:Xícara e copo.

LEIA

\includegraphics[width=3.23444in,height=2.93410in]{media/image131.png}

\subsubsection{18. QUAL É A CAMPANHA MOSTRADA NO
CARTAZ?}\label{qual-uxe9-a-campanha-mostrada-no-cartaz}

\_\_\_\_\_\_\_\_\_\_\_\_\_\_\_\_\_\_\_\_\_\_\_\_\_\_\_\_\_\_\_\_\_\_\_\_\_\_\_\_\_\_\_\_\_\_\_\_\_\_\_\_\_\_\_\_\_\_\_\_\_\_\_\_\_\_\_\_\_\_

R: Vacinação contra a raiva.

\subsubsection{19. PARA QUEM É INDICADA ESSA
VACINAÇÃO?}\label{para-quem-uxe9-indicada-essa-vacinauxe7uxe3o}

\_\_\_\_\_\_\_\_\_\_\_\_\_\_\_\_\_\_\_\_\_\_\_\_\_\_\_\_\_\_\_\_\_\_\_\_\_\_\_\_\_\_\_\_\_\_\_\_\_\_\_\_\_\_\_\_\_\_\_\_\_\_\_\_\_\_\_\_\_\_

R: Para cães e gatos.\protect\hypertarget{_heading=h.81uy4mz4w5m6}{}{}

\subsubsection{20. QUAL O PERÍODO QUE VAI ACONTECER A
CAMPANHA?}\label{qual-o-peruxedodo-que-vai-acontecer-a-campanha}

\_\_\_\_\_\_\_\_\_\_\_\_\_\_\_\_\_\_\_\_\_\_\_\_\_\_\_\_\_\_\_\_\_\_\_\_\_\_\_\_\_\_\_\_\_\_\_\_\_\_\_\_\_\_\_\_\_\_\_\_\_\_\_

R;17/08 a 17/09.

\subsection{TREINO}\label{treino-2}

\textbf{Os três itens a seguir estão ordenados do mais fácil para o mais
difícil. }

\subsubsection{01}\label{section-8}

LEIA:

\textbf{AS BORBOLETAS}

BRANCAS

AZUIS

AMARELAS

E PRETAS

BRINCAM

NA PAZ

AS BELAS

BORBOLETAS

BORBOLETAS BRANCAS

SÃO ALEGRES E FRANCAS

BORBOLETAS AZUIS

GOSTAM MUITO DE LUZ.

AS AMARELINHAS

SÃO TÃO BONITINHAS!

E AS PRETAS, ENTÃO\ldots{}

OH QUE ESCURIDÃO!\\
Disponível em
:\emph{http://www.dominiopublico.gov.br/download/texto/me000588.pdf.
Acesso} em 12 de fev 2023.

QUAL É A COR DA BOBORLETA QUE GOSTA DE LUZ?

\begin{enumerate}
\def\labelenumi{\Alph{enumi})}
\item
  AZUL.
\item
  PRETA.
\item
  BRANCA.
\item
  AMARELA.
\end{enumerate}

\protect\hypertarget{_heading=h.z337ya}{}{}Habilidade do SAEB Localizar
informações explícitas em textos.

Habilidade BNCC EF15LP03 Localizar informações explícitas em textos.

(A) Está correta, pois Aa borboleta que gosta de luz é azul.

(B) Está incorreta, por acreditar que como são escuras gostam de luz.

(C) Está incorreta, por acreditar que as brancas gostam de luz por serem
claras.

(D) Está incorreta, por acreditar que como a luz também é amarela essas
gostam de luz.

\subsubsection{02}\label{section-9}

LEIA OS INGREDIENTES DE UMA RECEITA.

\textbf{INGREDIENTES:}\\
6 FATIAS DE PÃO INTEGRAL OU 3 MINI BAGUETES DE FARINHA INTEGRAL\\
6 COLHERES (DE SOPA) DE QUEIJO RICOTA\\
200 G LEGUMES (CENOURA COZIDA, ERVILHA, MILHO VERDE, BATATA)\\
3 FOLHAS GRANDES RASGADAS DE ALFACE AMERICANA\\
15 G PEPINO CORTADO EM FATIAS\\
100 ML IOGURTE NATURAL

Disponivel
em:\href{https://g1.globo.com/minas-gerais/noticia/2015/01/aprenda-preparar-tres-tipos-de-sanduiche-para-o-verao.html.Acesso}{\emph{https://g1.globo.com/minas-gerais/noticia/2015/01/aprenda-preparar-tres-tipos-de-sanduiche-para-o-verao.html.Acesso}}
dia 12 de fev 2023.

O INGREDIENTE QUE VAI SER USADO 6 FATIAS É

A) PÃO.

B) QUEIJO.

C) PEPINO.

D) CENOURA.

Habilidade SAEB localizar informações explícitas em textos.

Habilidade BNCC EF15LP03 Localizar informações explícitas em textos.

(A) Está correta, pois será usada 6 fatias de pães.

(B) Está incorreta, por acreditar que o queijo é usado em fatias.

(C) Está incorreta, pois confundiu a quantidade uma vez que o pepino
também será usado em fatias.

(D) Está incorreta, por acreditar que a cenoura pode ser cortada em
fatias.

\subsubsection{03}\label{section-10}

\textbf{A FOCA}

RIO DE JANEIRO , 1970

QUER VER A FOCA

FICAR FELIZ?

É PÔR UMA BOLA

NO SEU NARIZ.

QUER VER A FOCA

BATER PALMINHA?

É DAR A ELA

UMA SARDINHA.

QUER VER A FOCA

FAZER UMA BRIGA?

É ESPETAR ELA

BEM NA BARRIGA!

\href{https://www2.bauru.sp.gov.br/arquivos/arquivos_site/sec_educacao/atividades_pedagogica_distancia/1;Infantil/26;EMEII\%20Irene\%20Ferreira\%20Chermont/6;PROF.\%C2\%AA\%20FL\%C3\%81VIA/INF\%20III\%20E\%20IV\%20-\%20PROF\%C2\%AA\%20FL\%C3\%81VIA\%20(12\%20A\%2016\%20DE\%20JULHO).pdf}{\emph{https://www2.bauru.sp.gov.br/arquivos/arquivos\_site/sec\_educacao/atividades\_pedagogica\_distancia/1;Infantil/26;EMEII\%20Irene\%20Ferreira\%20Chermont/6;PROF.\%C2\%AA\%20FL\%C3\%81VIA/INF\%20III\%20E\%20IV\%20-\%20PROF\%C2\%AA\%20FL\%C3\%81VIA\%20(12\%20A\%2016\%20DE\%20JULHO).pdf}}

O QUE FAZ A FOCA BATER PALMINHA?

A) POR UMA BOLA NO SEU NARIZ.

B) ESPETAR BEM NA BARRIGA.

C) DAR UMA SARDINHA.

D) FAZER UMA BRIGA.

\protect\hypertarget{_heading=h.3j2qqm3}{}{}Habilidade SAEB localizar
informações explícitas em textos.

Habilidade BNCC EF15LP03 Localizar informações explícitas em textos.

(A) Está incorreta, por acreditar que a foca bate palma se por uma bola
no nariz pois gosta de brincar.

(B) Está incorreta, por acreditar que se espetar a barriga ela sente
sossegas e bate palma.

(C) Está correta, pois a foca bate palmas se dar a ela uma sardinha.

(D) Está incorreta, por confundir briga com sardinha que pode se também
considerada uma brincadeira de bater.

\section{\texorpdfstring{\\
}{ }}\label{section-11}

\section{4. POR QUE ESSE TEXTO?
}\label{muxf3dulo-4-por-que-esse-texto}

\protect\hypertarget{_heading=h.4i7ojhp}{}{}Habilidade do SAEB
Reconhecer a finalidade de um texto.

Habilidade da BNCC
EF15LP01.

\subsection{CONTEÚDO}\label{conteuxfado-3}

\textbf{Para iniciar o módulo apresente os textos pergunte se eles
conhecem algum deles em seguida explique a função de cada texto e para
quem ele se destina.}

VOCÊ SABE O QUE É UM TEXTO?

UM TEXTO É UM CONJUNTO DE PALAVRAS COM LINGUAGEM E ESTRUTURA PRÓPRIA E
COM FUNÇÕES DIFERENTES.

EXISTEM VÁRIOS TIPOS DE TEXTOS. VEJA ALGUNS DELES E SUAS FUNÇÕES.

CONVITE
RECEITA\includegraphics[width=1.39444in,height=1.86944in]{media/image132.jpg}\includegraphics[width=2.02014in,height=1.72778in]{media/image134.png}

SERVE PARA FAZER UM CONVITE SERVE PARA ENSINAR

\textbf{NOTÍCIA}

Menino de 5 anos é brasileiro mais novo

a entrar para clube mundial de pessoas com alto QI.

SERVE PARA INFORMAR \textbf{TIRINHA}

\textbf{LISTA}
\includegraphics[width=1.65625in,height=1.20625in]{media/image135.png}\includegraphics[width=3.04583in,height=1.19514in]{media/image136.png}

\href{https://www.flickr.com/photos/guinanet/2429808365}{\textbf{\emph{https://www.flickr.com/photos/guinanet/2429808365}}}
SERVE PARA DIVERTIR

SERVE PARA ORGANIZAR

\subsection{ATIVIDADES}\label{atividades-3}

\subsubsection{1. TIA NINA VAI APRENDER FAZER UMA COMIDA PARA O ALMOÇO.
QUAL TEXTO ELA DEVE USAR? MARQUE COM UM
X.}\label{tia-nina-vai-aprender-fazer-uma-comida-para-o-almouxe7o.-qual-texto-ela-deve-usar-marque-com-um-x.}

\includegraphics[width=2.15625in,height=2.60417in]{media/image137.png}\includegraphics[width=2.08333in,height=2.49514in]{media/image138.png}

\includegraphics[width=2.09375in,height=2.10278in]{media/image139.png}\includegraphics[width=1.94236in,height=2.23611in]{media/image140.png}

\subsubsection{2. ENCONTRE E CIRCULE O TEXTO QUE SERVE PARA INFORMAR OS
LEITORES.
}\label{encontre-e-circule-o-texto-que-serve-para-informar-os-leitores.}

\includegraphics[width=3.23889in,height=1.08958in]{media/image141.png}

\includegraphics[width=2.03472in,height=1.44444in]{media/image142.png}

\includegraphics[width=0.90835in,height=0.87172in]{media/image143.png}
\includegraphics[width=0.90912in,height=0.90912in]{media/image145.png}\includegraphics[width=1.74653in,height=1.74653in]{media/image147.png}\includegraphics[width=1.18333in,height=1.18333in]{media/image152.png}

Notícia
\href{https://www.gov.br/ibama/pt-br}{\textbf{\emph{https://www.gov.br/ibama/pt-br}}}

\href{https://lizeseusamigos.org.br/tirinhas/tirinhas-tres}{\textbf{\emph{https://lizeseusamigos.org.br/tirinhas/tirinhas-tres}}}

\subsubsection{OBSERVE O TEXTO. }\label{observe-o-texto.}

\textbf{BOLO DE CENOURA}

INGREDIENTES\includegraphics[width=2.79286in,height=1.86042in]{media/image153.jpg}

4 OVOS

1 XÍCARA DE ÓLEO

3 CENOURAS MÉDIAS DESCASCADAS E PICADAS

2 XÍCARAS DE AÇÚCAR

2 XÍCARAS DE FARINHA DE TRIGO

1 COLHER DE SOPA DE FERMENTO

MODO DE PREPARAR

\begin{itemize}
\item
  MISTURE A FARINHA DE TRIGO, O AÇÚCAR E O FERMENTO NUMA
\end{itemize}

TIGELA.

\begin{itemize}
\item
  BATA OS OVOS, O ÓLEO E AS CENOURAS NO LIQUIDIFICADOR.
\item
  MISTURE O QUE FOI BATIDO NO LIQUIDIFICADOR AOS INGREDIENTES QUE ESTÃO
  NA TIGELA.
\end{itemize}

https://www.freepik.com/premium-photo/sweet-carrot-cake-slice-wooden-table\_7672480.htm\#page=2\&query=bolo\%20de\%20cenoura\&position=24\&from\_view=search\&track=ais

Disponível
em:https://portal.educacao.go.gov.br/wp-content/uploads/2021/08/Atividade-12-1-o-ano-Lingua-Portuguesa-Tema-Receitas-FIinalidade-do-texto-Composicao-do-texto-Professor.pdf.
Acesso em 18 Fev 2023.

\subsubsection{3. QUAL É O NOME DESSE
TEXTO?}\label{qual-uxe9-o-nome-desse-texto}

\_\_\_\_\_\_\_\_\_\_\_\_\_\_\_\_\_\_\_\_\_\_\_\_\_\_\_\_\_\_\_\_\_\_\_\_\_\_\_\_\_\_\_\_\_\_\_\_\_\_\_\_\_\_\_\_\_\_\_\_\_\_\_\_\_\_\_\_\_\_

R: Receita.\protect\hypertarget{_heading=h.sb19dr2f5edy}{}{}

\subsubsection{4. PARA QUE SERVE ESSE
TEXTO?}\label{para-que-serve-esse-texto}

\_\_\_\_\_\_\_\_\_\_\_\_\_\_\_\_\_\_\_\_\_\_\_\_\_\_\_\_\_\_\_\_\_\_\_\_\_\_\_\_\_\_\_\_\_\_\_\_\_\_\_\_\_\_\_\_\_\_\_\_\_\_\_\_\_\_\_\_\_\_

R Para ensinar fazer uma comida.

\subsubsection{5. ESCREVA DOIS NOMES DOS INGREDIENTES DA
RECEITA.}\label{escreva-dois-nomes-dos-ingredientes-da-receita.}

\_\_\_\_\_\_\_\_\_\_\_\_\_\_\_\_\_\_\_\_\_\_\_\_\_\_\_\_\_\_\_\_\_\_\_\_\_\_\_\_\_\_\_\_\_\_\_\_\_\_\_\_\_\_\_\_\_\_\_\_\_\_\_\_\_\_\_\_\_\_\_\_\_\_\_\_\_\_\_\_\_\_\_\_\_\_\_\_\_\_\_\_\_\_\_\_\_\_\_\_\_\_\_\_\_\_\_\_\_\_\_\_\_\_\_\_\_\_\_\_\_\_\_\_\_\_\_\_\_\_\_\_\_\_\_\_\_\_\_\_

R:cenoura, açúcar.

\subsubsection{6. CONSTRUA UM TEXTO QUE SERVE PARA ORGANIZAR AS COMPRAS
DE FRUTAS DO MÊS.
}\label{construa-um-texto-que-serve-para-organizar-as-compras-de-frutas-do-muxeas.}

\begin{longtable}[]{@{}l@{}}
\toprule
\tabularnewline
\tabularnewline
\tabularnewline
\tabularnewline
\tabularnewline
\tabularnewline
\tabularnewline
\tabularnewline
\bottomrule
\end{longtable}

Resposta pessoal

\subsubsection{7. PREECHA A CRUZADINHA COM DE ACORDO COM A
CARACTERISTICA DE CADA
TEXTO.}\label{preecha-a-cruzadinha-com-de-acordo-com-a-caracteristica-de-cada-texto.}

\includegraphics[width=5.80417in,height=6.50556in]{media/image155.png}

\protect\hypertarget{_heading=h.vv31p5wq2469}{}{}

\textbf{RESPOSTA:}

\textbf{1- Receita.}

\textbf{2- Bilhete.}

\textbf{3- Convite.}

\textbf{4- Tirinha.}

\textbf{5- Notícia.}

\textbf{6- Cartaz.}

\subsubsection{LEIA ESSE TEXTO.}\label{leia-esse-texto.}

\includegraphics[width=2.82569in,height=2.82569in]{media/image156.png}

\textbf{Disponível em:
\href{https://riovermelho.mg.gov.br/rio-vermelho-iniciara-vacinacao-de-criancas/}{\emph{https://riovermelho.mg.gov.br/rio-vermelho-iniciara-vacinacao-de-criancas/}}
Acesso em 18 Fev 2023.}

\subsubsection{8. COMO É CHAMADO ESSE
TEXTO?}\label{como-uxe9-chamado-esse-texto}

\textbf{\_\_\_\_\_\_\_\_\_\_\_\_\_\_\_\_\_\_\_\_\_\_\_\_\_\_\_\_\_\_\_\_\_\_\_\_\_\_\_\_\_\_\_\_\_\_\_\_\_\_\_\_\_\_\_\_\_\_\_\_\_\_\_\_\_\_\_\_\_\_}

\textbf{R:cartaz.}\protect\hypertarget{_heading=h.jmn0osvpcycz}{}{\protect\hypertarget{_heading=h.7rmterfq7uhk}{}{}}

\subsubsection{9. PARA QUE ELE SERVE?}\label{para-que-ele-serve}

\textbf{\_\_\_\_\_\_\_\_\_\_\_\_\_\_\_\_\_\_\_\_\_\_\_\_\_\_\_\_\_\_\_\_\_\_\_\_\_\_\_\_\_\_\_\_\_\_\_\_\_\_\_\_\_\_\_\_\_\_\_\_\_\_\_\_\_\_\_}

\textbf{R:Iinformar sobre a
vacinas}\protect\hypertarget{_heading=h.v322bx8tzcgo}{}{}

\subsubsection{10. PARA QUAL GRUPO ELE É
DESTINANDO?}\label{para-qual-grupo-ele-uxe9-destinando}

\textbf{\_\_\_\_\_\_\_\_\_\_\_\_\_\_\_\_\_\_\_\_\_\_\_\_\_\_\_\_\_\_\_\_\_\_\_\_\_\_\_\_\_\_\_\_\_\_\_\_\_\_\_\_\_\_\_\_\_\_\_\_\_\_\_\_\_\_\_\_\_}

\textbf{R: Para as crianças.}

\subsection{TREINO}\label{treino-3}

\textbf{Os três itens a seguir estão ordenados do mais fácil para o mais
difícil. }

\subsubsection{01 }\label{section-12}

OBSERVE O TEXTO.

\includegraphics[width=3.10106in,height=3.10106in]{media/image157.png}
ELEBORADO PELA AUTORA

ESTE TEXTO É

A) UMA CARTA.

B) UMA RECEITA.

C) UM CONVITE.

D) UMA AGENDA.

Habilidade SAEB Reconhecer a finalidade de um texto. .

Habilidade BNCC EF15LP01 Identificar a função social de textos que
circulam em campos da vida social dos quais participa cotidianamente (a
casa, a rua, a comunidade, a escola) e nas mídias impressa, de massa e
digital, reconhecendo para que foram produzidos, onde circulam, quem os
produziu e a quem se destinam.

(A) Está incorreta, por acreditar que como aparece o nome de uma pessoa
seria uma carta.

B) Está incorreta, pois acreditou que na fazenda a vovó faz muitas
receitas

(C) Está correta, pois o texto é um convite para um aniversário.

(D) Está incorreta, por achar que como o texto tem data
estáply766j6jo777777i agendando alguma
coisa.\protect\hypertarget{_heading=h.gz9yp347w1lz}{}{}

\subsubsection{02 }\label{section-13}

\textbf{LEIA O TEXTO:}

\includegraphics[width=4.03663in,height=3.60910in]{media/image158.jpg}

Disponível
em:https://www.boituva.sp.gov.br/imprensa/noticias/dia-das-criancas-no-centro-de-eventos.
Acesso em 18 fev 2023.

A FINALIDADE O TEXTO É

A) ORGANIZAR TAREFAS.

B) ENSINAR UMA RECEITA

C) INFORMAR UM EVENTO.

D) CONTAR UMA HISTÓRIA.

Habilidade SAEB Reconhecer a finalidade de um texto. .

Habilidade BNCC EF15LP01 Identificar a função social de textos que
circulam em campos da vida social dos quais participa cotidianamente (a
casa, a rua, a comunidade, a escola) e nas mídias impressa, de massa e
digital, reconhecendo para que foram produzidos, onde circulam, quem os
produziu e a quem se destinam.

(A) Está incorreta, pelo fato de acreditar que os horários do cartaz
estavam organizados tarefa.

(B) Está incorreta, por acreditar que a sequência de horários seria a
quantidade de ingrediente de uma receita.

(C) Está correta, pois esse cartaz informa um evento para as crianças.

(D) Está incorreta, por acreditar que o fato de ter crianças no cartaz
elas estariam contando uma história.

\subsubsection{03 }\label{section-14}

\textbf{LEIA:}

\textbf{SALADA DE FRUTAS}

INGREDIENTES

3
BANANAS\includegraphics[width=2.90350in,height=1.63264in]{media/image159.jpg}

1 MAMÃO PEQUENO

3 MAÇÃS

4 LARANJAS

METADE DE 1 ABACAXI

MODO DE PREPARAR

\begin{itemize}
\item
  LAVE AS FRUTA.
\item
  DESCASQUE E PIQUE TODAS BEM PEQUENINHAS, MENOS A LARANJA.
\item
  DESCASQUE E PIQUE SÓ 2 LARANJAS.
\item
  COM AS OUTRAS 2 LARANJAS FAÇA UM SUCO E DESPEJE POR CIMA DA SALADA.
\end{itemize}

Disponível em:
\href{https://portal.educacao.go.gov.br/wp-content/uploads/2021/08/Atividade-12-1-o-ano-Lingua-Portuguesa-Tema-Receitas-FIinalidade-do-texto-Composicao-do-texto-Professor.pdf}{\emph{https://portal.educacao.go.gov.br/wp-content/uploads/2021/08/Atividade-12-1-o-ano-Lingua-Portuguesa-Tema-Receitas-FIinalidade-do-texto-Composicao-do-texto-Professor.pdf}}.
Acesso em 18 Fev 2023.

https://www.freepik.com/free-photo/mixed-assorted-fruits\_4124941.htm\#query=salada\%20de\%20frutas\&position=6\&from\_view=search\&track=ais

ESTE TEXTO SERVE PARA

A) ENSINAR A PREPARAR UMA COMIDA.

B) INFORMAR AS VITAMINAS DAS FRUTAS.

C) FAZER A PROPAGANDO DE UMA COMIDA.

D) FALAR SOBRE OS NUTRIENTES DE UM SUCO.

Habilidade SAEB Reconhecer a finalidade de um texto..

Habilidade BNCC EF15LP01 Identificar a função social de textos que
circulam em campos da vida social dos quais participa cotidianamente (a
casa, a rua, a comunidade, a escola) e nas mídias impressa, de massa e
digital, reconhecendo para que foram produzidos, onde circulam, quem os
produziu e a quem se destinam.

\protect\hypertarget{_heading=h.2xcytpi}{}{}(A) Está correta, pois esse
texto ensina fazer uma comida.

(B) Está incorreta, por acreditar como é uma salada de fruta o texto
fala sobre os nutrientes.

(C) Está incorreta, por acreditar que estaria divulgado a comida para
vender.

(D) Está incorreta, por acreditar esse texto está falando sobre os
nutrientes de cada fruta.

\section{\texorpdfstring{\\
}{ }}\label{section-15}

\section{5. COMPREENDER O
TEXTO}\label{muxf3dulo-5-compreender-o-texto}

\protect\hypertarget{_heading=h.3whwml4}{}{}Habilidade do SAEB
Inferir o assunto de um texto.

\subsection{CONTEÚDO}\label{conteuxfado-4}

Para realizar as atividades desse módulo você precisa ler junto com os
alunos todos os textos explorando o tipo de texto, título , personagens
as informações contida nele levantado hipótese sobre o assunto de cada
texto.

COMO VOCÊ ACHA QUE É FORMADO UM TEXTO?

UM TEXTO E FORMADO DE PALAVRAS E FRASES AO LER UM TEXTO VOCÊ PODE
LEVANTAR HIPÓTESES, ANALISAR, COMPARAR, RELACIONAR A PARTIR DAS
INFORMAÇÕES CONTIDAS NELE CHEGANDO A UMA COCLUSÃO DA MENSAGEM QUE ELE
DESEJA TRANSMITIR.

VAMOS LER ESSE TEXTO.

\textbf{A RÃ E O TOURO}

\includegraphics[width=2.01214in,height=1.68750in]{media/image160.png}\includegraphics[width=2.04167in,height=1.99274in]{media/image161.jpg}

\includegraphics[width=2.66528in,height=0.47292in]{media/image162.jpg}

UM GRANDE TOURO PASSEAVA PELA MARGEM DE UM RIACHO. A RÃ FICOU COM MUITA
INVEJA DE SEU TAMANHO E DE SUA FORÇA. ENTÃO, COMEÇOU A INCHAR, FAZENDO
ENORME ESFORÇO, PARA TENTAR FICAR TÃO GRANDE QUANTO O TOURO. FEZ ISSO
COM TANTA FORÇA QUE ACABOU EXPLODINDO, POR CULPA DE TANTA INVEJA.

Disponível
em:\href{http://www.dominiopublico.gov.br/download/texto/me001614.pdf}{\emph{http://www.dominiopublico.gov.br/download/texto/me001614.pdf}}.
Acesso dia 13 fev 2023.

\href{https://www.freepik.com/free-vector/cartoon-frog-holding-lotus-leaf_27286948.htm\#page=2\&query=frog\&position=23\&from_view=search\&track=sph}{\emph{https://www.freepik.com/free-vector/cartoon-frog-holding-lotus-leaf\_27286948.htm\#page=2\&query=frog\&position=23\&from\_view=search\&track=sph}}

https://www.freepik.com/free-vector/brown-cow-cartoon-character\_31561944.htm\#page=2\&query=touro\&position=1\&from\_view=search\&track=sph

AO LER ESSE TEXTO PODEMOS PERCEBER QUE O ASSUNTO DELE É A INVEJA DA RÃ
PELO TAMNAHO DO TOURO.

PODEMOS TAMBÉM IDETIFICAR OS PESONAGENS DO TEXTO QUE É A RÃ E O
TOURO.\protect\hypertarget{_heading=h.nmjmay4k137u}{}{}

\subsection{ATIVIDADES}\label{atividades-4}

\subsubsection{LEIA O TEXTO}\label{leia-o-texto-1}

\includegraphics[width=1.66667in,height=1.30208in]{media/image163.jpg}

ERA UMA VEZ DUAS IRMÃS CHAMADAS CLOÉ E MARLI.ELAS MORAVAM NUM VILAREJO
BEM DISTANTE; SEU PAI ERA O ÚNICO PROFESSOR DO LUGAR. A FAMÍLIA VIVIA
NOS FUNDOS DA ESCOLA.

CERTO DIA, A MÃE DE UM COLEGA E AMIGA DA FAMÍLIA RESOLVEU LEVAR UM
PRESENTE PARA CLOÉ. O PROFESSOR FOI À JANELA E... - CLOÉ! VISITA PARA
VOCÊ! AO CHEGAR, DONA ELIZETE ENTREGOU-LHE UMA PEQUENA CAIXA. - ISTO É
PARA VOCÊ. CUIDE BEM DELA. CLOÉ AGRADECEU O PRESENTE, E SAIU PARA
MOSTRÁ-LO À IRMÃ.

Disponível em:
\href{http://www.dominiopublico.gov.br/download/texto/eu000006.pdf}{\emph{http://www.dominiopublico.gov.br/download/texto/eu000006.pdf}}
\textbf{Acesso em 13 de fev 2023.}

\subsubsection{1. ESCREVA O TÍTULO DE
TEXTO?}\label{escreva-o-tuxedtulo-de-texto}

\_\_\_\_\_\_\_\_\_\_\_\_\_\_\_\_\_\_\_\_\_\_\_\_\_\_\_\_\_\_\_\_\_\_\_\_\_\_\_\_\_\_\_\_\_\_\_\_\_\_\_\_\_\_\_\_\_\_\_\_\_\_\_\_\_\_

R: O mistério do anel de pérola

\subsubsection{2. ONDE VIVIAM CLOÉ, MARLI E SUA
FAMÍLIA?}\label{onde-viviam-clouxe9-marli-e-sua-famuxedlia}

\_\_\_\_\_\_\_\_\_\_\_\_\_\_\_\_\_\_\_\_\_\_\_\_\_\_\_\_\_\_\_\_\_\_\_\_\_\_\_\_\_\_\_\_\_\_\_\_\_\_\_\_\_\_\_\_\_\_\_\_\_\_\_\_\_\_\_\_\_

R: Nos fundos da escola.

\subsubsection{3. ENCONTRE E PINTE NO TEXO O NOME DA MENINA QUE GANHOU O
PRESENTE.}\label{encontre-e-pinte-no-texo-o-nome-da-menina-que-ganhou-o-presente.}

\subsubsection{4. QUAL É O ASSUNTO DESSE
TEXTO?}\label{qual-uxe9-o-assunto-desse-texto}

\_\_\_\_\_\_\_\_\_\_\_\_\_\_\_\_\_\_\_\_\_\_\_\_\_\_\_\_\_\_\_\_\_\_\_\_\_\_\_\_\_\_\_\_\_\_\_\_\_\_\_\_\_\_\_\_\_\_\_\_\_\_\_\_\_\_\_

R: O presente que Cloé ganhou.

\subsubsection{LEIA O TEXTO.}\label{leia-o-texto.-1}

\includegraphics[width=2.95254in,height=2.09138in]{media/image164.jpg}Refazer
o texto colorido

\href{https://commons.wikimedia.org/wiki/File:Obra-de-aquila-davi-arte-fora-de-contexto.jpg}{\emph{https://commons.wikimedia.org/wiki/File:Obra-de-aquila-davi-arte-fora-de-contexto.jpg}}
\textbf{Acesso em 13 de fev 2023.}

\subsubsection{5. MARQUE VERDADEIRO OU
FALSO.}\label{marque-verdadeiro-ou-falso.}

( ) NO TEXTO APRECE UM MENINO E UMA MENINA.

( ) SEGUNDO O TEXTO VOCÊ PODE NÃO PODE ESCOLHER SER O QUE QUISER.

( ) O ASSUNTO DO TEXTO É QUE VOCÊ PODE SER O QUE QUISER.

( ) O MENINO DO TEXTO QUER SER UM DESENHO.

\subsubsection{LEIA A NOTÍCIA.}\label{leia-a-notuxedcia.}

\textbf{IBAMA DEVOLVE À NATUREZA 140 PAPAGAIOS EM GO}

RASÍLIA (19/12/2017) -- O CENTRO DE TRIAGEM DE ANIMAIS SILVESTRES
(CETAS) DO IBAMA EM GOIÁS PROGRAMOU PARA ESTE MÊS \emph{A SOLTURA DE 140
PAPAGAIOS} EM TRÊS ÁREAS CADASTRADAS NO PROJETO ASAS (ÁREAS DE SOLTURA
DE ANIMAIS SILVESTRES), INICIADO NO ESTADO EM 2008.~

\textbf{Disponível
em:http://www.ibama.gov.br/noticias/422-2017/1293-ibama-devolve-a-natureza-140-papagaios-ego\#:\textasciitilde{}:text=A\%20devolu\%C3\%A7\%C3\%A3o\%20\%C3\%A0\%20natureza\%20ocorre,\%2C\%20em\%20Hidrol\%C3\%A2ndia\%20(GO).Acesso
em 14 de fev 2023.}

\subsubsection{6. PASSE UM TRAÇO COM LÁPIS VERMELHO O ASSUNTO PRINCIPAL
DESSA
NOTÍCIA.}\label{passe-um-trauxe7o-com-luxe1pis-vermelho-o-assunto-principal-dessa-notuxedcia.}

\subsubsection{7. QUAL ANIMAL APARECE NA
NOTÍCIA?}\label{qual-animal-aparece-na-notuxedcia}

\_\_\_\_\_\_\_\_\_\_\_\_\_\_\_\_\_\_\_\_\_\_\_\_\_\_\_\_\_\_\_\_\_\_\_\_\_\_\_\_\_\_\_\_\_\_\_\_\_\_\_\_\_\_\_\_\_\_\_\_\_\_\_\_\_\_\_\_\_\_\_\_\_

R: Papagaio

\subsubsection{VAMOS LER MAIS UM TEXTO.}\label{vamos-ler-mais-um-texto.}

RAPUNZEL

ERA UMA VEZ UM CASAL QUE HÁ MUITO TEMPO DESEJAVA INUTILMENTE TER UM
FILHO. OS ANOS SE PASSAVAM, E SEU SONHO NÃO SE REALIZAVA. AFINAL, UM
BELO DIA, A MULHER PERCEBEU QUE DEUS OUVIRA SUAS PRECES. ELA IA TER UMA
CRIANÇA! POR UMA JANELINHA QUE HAVIA NA PARTE DOS FUNDOS DA CASA DELES,
ERA POSSÍVEL VER, NO QUINTAL VIZINHO, UM MAGNÍFICO JARDIM CHEIO DAS MAIS
LINDAS FLORES E DAS MAIS VIÇOSAS HORTALIÇAS.

MAS EM TORNO DE TUDO SE ERGUIA UM MURO ALTÍSSIMO, QUE NINGUÉM SE ATREVIA
A ESCALAR. AFINAL, ERA A PROPRIEDADE DE UMA FEITICEIRA MUITO TEMIDA E
PODEROSA.

\href{http://www.dominiopublico.gov.br/download/texto/me001614.pdf}{\emph{http://www.dominiopublico.gov.br/download/texto/me001614.pdf}}
\textbf{Acesso em 14 de fev 2023.}

AGORA RESPONDA\protect\hypertarget{_heading=h.4pnvitxnujbb}{}{}

\subsubsection{8. QUAL É O TÍTULO DO
TEXTO?}\label{qual-uxe9-o-tuxedtulo-do-texto}

\_\_\_\_\_\_\_\_\_\_\_\_\_\_\_\_\_\_\_\_\_\_\_\_\_\_\_\_\_\_\_\_\_\_\_\_\_\_\_\_\_\_\_\_\_\_\_\_\_\_\_\_\_\_\_\_\_\_\_\_\_\_\_\_\_\_\_

R: Rapunzel

\subsubsection{9. DE QUE FALA O TEXTO?}\label{de-que-fala-o-texto}

\textbf{\_\_\_\_\_\_\_\_\_\_\_\_\_\_\_\_\_\_\_\_\_\_\_\_\_\_\_\_\_\_\_\_\_\_\_\_\_\_\_\_\_\_\_\_\_\_\_\_\_\_\_\_\_\_\_\_\_\_\_\_\_\_\_\_\_\_}

R: O desejo do casal em ter m filho.

\subsubsection{10. PINTE OS BALÕES COM A INFORMAÇÃO
CORRETA.}\label{pinte-os-baluxf5es-com-a-informauxe7uxe3o-correta.}

\textbf{O ASSUNTO DO TEXTO É O DESEJO DE UM CASAL TER UM FILHO.}

\textbf{A MULHER IA TER UMA CRIANÇA.}

\textbf{DA JANELA ELA VIA UMA FAZENDA COM ANIMAIS.}

\textbf{NO TEXTO APARECE UM FEITICEIRA.}

\textbf{NO TEXTO APARECE UMA PLANTAÇÃO DE MILHO.}

\textbf{COLOCAR AS FRASES DENTRO DE BALÕES.}

\subsubsection{LEIA O TEXTO.}\label{leia-o-texto.-2}

\includegraphics[width=4.50249in,height=3.85413in]{media/image165.jpg}

\textbf{\href{https://www.josebonifacio.sp.gov.br/portal/noticias/0/3/1776/campanha-de-vacinacao-contra-raiva---zona-rural}{\emph{https://www.josebonifacio.sp.gov.br/portal/noticias/0/3/1776/campanha-de-vacinacao-contra-raiva-\/-\/-zona-rural}}
Acesso em 13 de fev 2023.}

\subsubsection{RESPONDA}\label{responda}

\subsubsection{11. QUAL É O ASSUNTO DESSE
TEXTO?}\label{qual-uxe9-o-assunto-desse-texto-1}

\textbf{\_\_\_\_\_\_\_\_\_\_\_\_\_\_\_\_\_\_\_\_\_\_\_\_\_\_\_\_\_\_\_\_\_\_\_\_\_\_\_\_\_\_\_\_\_\_\_\_\_\_\_\_\_\_\_\_\_\_\_\_\_\_\_\_\_\_\_\_\_\_}

R: Vacinação contra a raiva.

\subsubsection{12. ESCREVA O NOME DOS ANIMAIS QUE APARECE NO
TEXTO.}\label{escreva-o-nome-dos-animais-que-aparece-no-texto.}

\_\_\_\_\_\_\_\_\_\_\_\_\_\_\_\_\_\_\_\_\_\_\_\_\_\_\_\_\_\_\_\_\_\_\_\_\_\_\_\_\_\_\_\_\_\_\_\_\_\_\_\_\_\_\_\_\_\_\_\_\_\_\_\_\_\_

R: Gato e cachorro

\subsection{TREINO}\label{treino-4}

\textbf{Os três itens a seguir estão ordenados do mais fácil para o mais
difícil. }

\subsubsection{01 }\label{section-16}

VAMOS LER O TEXTO

\textbf{O PATO TIRA RETRATO}

\textbf{O PATO GANHOU SAPATO.}

\textbf{FOI LOGO TIRAR RETRATO.}

\textbf{O MACACO RETRATISTA}

\textbf{ERA MESMO UM GRANDE ARTISTA.}

\textbf{DISSE A O PATO: ``NÃO SE MEXA}

\textbf{PARA DEPOIS NÃO TER QUEIXA''.}

\textbf{E O PATO, DURO E SEM GRAÇA}

\textbf{COMO SE FOSSE DE MASSA!}

\textbf{``OLHE PRA CÁ DIREITINHO:}

\textbf{VAI SAIR UM PASSARINHO''.}

\textbf{O PASSARINHO SAIU.}

Disponivel
em:\textbf{\href{http://www.dominiopublico.gov.br/download/texto/me000588.pdf}{\emph{http://www.dominiopublico.gov.br/download/texto/me000588.pdf}}.
acesso em 18 Fev 2023.}

\textbf{QUAL É O ASSUNTO DO TEXTO?}

\textbf{A) O RETRATO DO MACACO.}

\textbf{B) O PRESENTE DO MACACO.}

\textbf{C) O RETRATO DO PATO.}

\textbf{D) A QUEIXA DO PATO.}

Habilidade SAEB~Inferir o assunto de um texto.

( A) Está incorreta, pois não analisou que o macaco era quem tirava o
retrato.

( B) Está incorreta, por confundiu que quem ganhou o sapato foi o
macaco.

(C) Está correta, pois descobriu que o assunto do texto era o retrato
que o pato que foi tirar assim que ganhou sapato.

(D) Está incorreta, por acreditar que o pato foi dar queixa ao macaco.

\subsubsection{02 }\label{section-17}

\textbf{LEIA O TEXTO.}

\includegraphics[width=2.37500in,height=2.25000in]{media/image166.jpg}

\textbf{\emph{https://www.arroiodopadre.rs.gov.br/portal/noticias/0/3/833/semana-municipal-do-meio-ambiente-de-arroio-do-padre}.
Acesso em 13 de fev 2023.}

O ASSUNTO DESSE TEXTO É

A) GARANTIR DE UM FUTURO MELHOR.

B) PRESERVAR DO MEIO AMBIENTE.

C) DOAR DE MUDAS DE ÁRVORES.

D) PLANTAR MUDAS DE ÁRVORES.

\protect\hypertarget{_heading=h.qsh70q}{}{}Habilidade SAEB~Inferir o
assunto de um texto.

(A) Está incorreta, pois acreditou que o fato da frase está no texto ela
seria o assunto principal.

(B) Está correta, pois o texto traz uma mensagem de conscientização para
a preservação do meio ambiente.

C) Está incorreta, por acreditar que como aparece no texto essa
informação seria ideia principal.

(D) Está incorreta, por acreditar que como tem as mãos abaixo da imagem
da planta essa seria o assunto do texto.

\subsubsection{03 }\label{section-18}

LEIA O TEXTO.

\textbf{VAMOS CANTAR.}

HÁ TRÊS NOITES QUE EU NÃO DURMO

Ó LÁ LÁ

POIS PERDI O MEU GALINHO

O LÁ LÁ

COITADINHO O LÁ LÁ,

POBREZINHO O LÁ LÁ

SE PERDEU LÁ NO JARDIM.

ELE É BRANCO E AMARELO

O LÁ LÁ

TEM A CRISTA VERMELHINHA

O LÁ LÁ

BATE AS ASAS, O LÁ LÁ,

ABRE O BICO O LÁ LÁ

E FAZ QUI QUI RI QUI QUI

JÁ RODEI O MATO GROSSO, Ó LÁ LÁ

AMAZONAS E PARÁ, O LÁ LÁ

ENCONTREI O LÁ LÁ,

MEU GALINHO, O LÁ LÁ

NO SERTÃO DO CEARÁ.

\href{http://www.dominiopublico.gov.br/download/texto/me000588.pdf}{\emph{http://www.dominiopublico.gov.br/download/texto/me000588.pdf}}.
Acesso dia 15 de fev 2023.

ESSE TEXTO FALA DO (A)

A) CARACTERÍSTICA DO GALINHO

B) ENCONTRO COMO GALINHO.

C) TEIMOSIA DO GALINHO.

D) O SUMIÇO DO GALINHO.

Habilidade SAEB~Inferir o assunto de um texto.

(A) Está incorreta, por acreditar que como o texto diz como era o
galinho esse seria o assunto do texto.

(B) Está incorreta, pois como no final do texto o galinho foi encontrado
esse seria o assunto do texto.

(D) Está incorreta, por acreditar que o galinho desapareceu porque era
teimoso.

(D) Está correta, pois o galinho tinha se perdido no texto e
desaparecido.

\section{\texorpdfstring{\\
}{ }}\label{section-19}

\section{6. AS INFORMAÇÕES DO
TEXTO}\label{muxf3dulo-6-as-informauxe7uxf5es-do-texto}

\protect\hypertarget{_heading=h.1pxezwc}{}{}Habilidade do SAEB
Inferir informações em textos verbais.

\subsection{CONTEÚDO}\label{conteuxfado-5}

Para realizar as atividades desse módulo você precisa ler junto com os
alunos todos os textos explorando o tipo de texto, título, personagens
as informações contidas nele levantado hipótese sobre as informações que
não está de forma clara no texto . Habilidade SAEB Inferir informações
em textos verbais.

PARA DESCOBRIR UMA INFORMAÇÃO EM UM TEXTO VOCÊ PRECISA PENSAR UM POUCO
SOBRE ELAS POIS NÃO APARECE DE FORMA CLARA.

VAMOS LER ESSE TEXTO.

\textbf{CACHORRINHO}

\textbf{CACHORRINHO ESTÁ LATINDO}

\textbf{LÁ NO FUNDO DO QUINTAL}

\textbf{CALA A BOCA CACHORRINHO}

\textbf{DEIXA O MEU BENZINHO ENTRAR.}

VOCÊ CONSEGUE DESCOBRIR PORQUE O CACHORRINHO ESTÁ LATINDO?

VEJA QUE ELE ESTÁ LATINDO POR QUE CHEGOU UMA PESSOA NA CASA ONDE MORA
COM SEU DONO.

O CACHORRO ESTÁ SOZINHO EM CASA?

OBSERVE QUE NÃO ESTÁ SOZINHO, POIS ALGUÉM PEDIU PARA ELA CALAR A BOCA
POIS ALGUÉM PRECISAVA ENTRAR.

\subsection{ATIVIDADES}\label{atividades-5}

Cantar a música com as crianças, explorando a oralidade. Habilidade do SAEB
Inferir informações em textos verbais.

\subsubsection{VAMOS CANTAR A
CANÇÃO.}\label{vamos-cantar-a-canuxe7uxe3o.}

FUI AO TORORÓ

FUI AO TORORÓ

BEBER ÁGUA E NÃO ACHEI

ENCONTREI BELA MORENA

QUE NO TORORÓ DEIXEI.

APROVEITA MINHA GENTE

QUE UMA NOITE NÃO É NADA

QUEM NÃO DORMIR A GORA

DORMIRÁ DE MADRUGADA.

Disponível
em;\href{http://www.dominiopublico.gov.br/download/texto/me000588.pdf}{\emph{http://www.dominiopublico.gov.br/download/texto/me000588.pdf}}
Acesso em 18 de Fev 2023.

\subsubsection{1. PORQUE NÃO TINHA ÁGUA NO
ITOTORÓ?}\label{porque-nuxe3o-tinha-no-itotoruxf3}

\_\_\_\_\_\_\_\_\_\_\_\_\_\_\_\_\_\_\_\_\_\_\_\_\_\_\_\_\_\_\_\_\_\_\_\_\_\_\_\_\_\_\_\_\_\_\_\_\_\_\_\_\_\_\_\_\_\_\_\_\_\_\_\_\_\_

R:Porque ele tinha secado.

\subsubsection{2. A PESSOA ESTAVA SOZINHO (A) NO
ITORORÓ?}\label{a-pessoa-estava-sozinho-a-no-itororuxf3}

\_\_\_\_\_\_\_\_\_\_\_\_\_\_\_\_\_\_\_\_\_\_\_\_\_\_\_\_\_\_\_\_\_\_\_\_\_\_\_\_\_\_\_\_\_\_\_\_\_\_\_\_\_\_\_\_\_\_

R: Não, pois lá encontrei bela Morena.

\subsubsection{3. QUAL FOI O PERÍODO QUE ESSA PESSOA FOI NO
ITORORÓ?}\label{qual-foi-o-peruxedodo-que-essa-pessoa-foi-no-itororuxf3}

\_\_\_\_\_\_\_\_\_\_\_\_\_\_\_\_\_\_\_\_\_\_\_\_\_\_\_\_\_\_\_\_\_\_\_\_\_\_\_\_\_\_\_\_\_\_\_\_\_\_\_\_\_\_\_\_\_\_\_\_

R: À noite.

\subsubsection{VAMOS LER O TEXTO}\label{vamos-ler-o-texto}

\textbf{A PORTA}

EU SOU FEITA DE MADEIRA.

MADEIRA, MATÉRIA MORTA.

MAS NÃO HÁ COISA NO MUNDO

MAIS VIVA DO QUE UMA PORTA.

EU ABRO DEVAGARINHO

PRA PASSAR O MENININHO

EU ABRO BEM PRAZENTEIRA

PRA PASSAR A COZINHEIRA

Disponivel em:
\textbf{\emph{\url{http://www.dominiopublico.gov.br/download/texto/me000588.pdf}.
Acesso dia 14 de fev 2023.}}

\subsubsection{4. POR QUE A MATÉRIA QUE FEZ A PORTA ESTÁ
MORTA?}\label{por-que-a-maneira-que-fez-a-morta-estuxe1-morta}

\_\_\_\_\_\_\_\_\_\_\_\_\_\_\_\_\_\_\_\_\_\_\_\_\_\_\_\_\_\_\_\_\_\_\_\_\_\_\_\_\_\_\_\_\_\_\_\_\_\_\_\_\_\_\_\_\_\_\_\_\_\_\_\_\_

R:Porque ela foi cortada para fazer a porta.

\subsubsection{5. POR QUE A PORTA ESTÁ
VIVA?}\label{por-que-a-porta-estuxe1-viva}

\_\_\_\_\_\_\_\_\_\_\_\_\_\_\_\_\_\_\_\_\_\_\_\_\_\_\_\_\_\_\_\_\_\_\_\_\_\_\_\_\_\_\_\_\_\_\_\_\_\_\_\_\_\_\_\_\_\_\_\_\_\_\_

R:Porque ela abre o tempo todo.

\subsubsection{6. MARQUE UM X NO QUADRO COM A INFORMAÇÃO
CORRETA.}\label{marque-um-x-no-quadro-com-a-informauxe7uxe3o-correta.}

\begin{longtable}[]{@{}ll@{}}
\toprule
& ELA ANDA DE UL LADO PARA OUTRO.\tabularnewline
x & ELA ABRE PARA AS PESSOAS PASSAREM.\tabularnewline
& ELA FALA COM AS PESSOAS.\tabularnewline
& ELA BRINCA DE CORRER.\tabularnewline
\bottomrule
\end{longtable}

\subsubsection{VAMOS CANTAR}\label{vamos-cantar-1}

A CANOA VIROU

A CANOA VIROU

POIS DEIXARAM VIRAR

FOI POR CAUSA DE MARINA QUE NÃO SOUBE REMAR.

SE EU FOSSE UM PEIXINHO

E SOUBESSE NADAR

EU TIRAVA A MARINA

DO FUNDO DO MAR.

Disponível
em:http://www.dominiopublico.gov.br/download/texto/me000588.pdf. acesso
18 Fev 2023.

\subsubsection{7. O QUE ACONTECEU COM A
MARINA?}\label{o-que-aconteceu-com-a-marina}

\_\_\_\_\_\_\_\_\_\_\_\_\_\_\_\_\_\_\_\_\_\_\_\_\_\_\_\_\_\_\_\_\_\_\_\_\_\_\_\_\_\_\_\_\_\_\_\_\_\_\_\_\_\_\_\_\_\_

\textbf{R: E}la se afogou.

\subsubsection{8. ELA ESTAVA SOZINHA NO
MAR?}\label{ela-estava-sozinha-no-mar}

\textbf{\_\_\_\_\_\_\_\_\_\_\_\_\_\_\_\_\_\_\_\_\_\_\_\_\_\_\_\_\_\_\_\_\_\_\_\_\_\_\_\_\_\_\_\_\_\_\_\_\_\_\_\_\_\_\_\_\_\_\_\_\_\_\_}

\textbf{R: Não}

\subsubsection{9. POR QUE A CANOA VIROU? CIRCULE O BALÃO COM A RESPOSTA
CERTA.}\label{por-que-a-canoa-virou-circule-o-baluxe3o-com-a-resposta-certa.}

\textbf{PORQUE MARINA ESTAVA SOZINHA.}

\textbf{PORQUE MARINA NÃO SOUBE REMAR A CANOA.}

\textbf{PORQUE O MAR ESTAVA AGITADO.}

\textbf{PORQUE VEIO UMA ONDA FORTE.}

\textbf{COLOCAR AS INFORMAÇÕES EM BALÕES}

\subsubsection{VAMOS LER.}\label{vamos-ler.}

SAMBA LÊ LÊ

SAMBA LÊ LÊ ESTÁ DOENTE

ESTÁ COM A CABEÇA QUEBRADA

SAMBA LÊ LÊ PRECISAVA

DE UMAS BOAS LAMBADAS.

Disponível
em:\href{http://www.dominiopublico.gov.br/download/texto/me000588.pdf}{\emph{http://www.dominiopublico.gov.br/download/texto/me000588.pdf}}
Acesso 15 de fev 2023.

\subsubsection{10. COMO ESTÁ O SAMBA LÊ
LÊ?}\label{como-estuxe1-o-samba-luxea-luxea}

\_\_\_\_\_\_\_\_\_\_\_\_\_\_\_\_\_\_\_\_\_\_\_\_\_\_\_\_\_\_\_\_\_\_\_\_\_\_\_\_\_\_\_\_\_\_\_\_\_\_\_\_\_\_\_\_\_\_\_\_\_\_

R: Doente

\subsubsection{11. O QUE ELE PRECISA PARA SE CURAR DA DOENÇA? CIRCULE NO
TEXTO.}\label{o-que-ele-precisa-para-se-curar-da-doenuxe7a-circule-no-texto.}

\subsection{TREINO}\label{treino-5}

\textbf{Os três itens a seguir estão ordenados do mais fácil para o mais
difícil. }

\subsubsection{01}\label{section-20}

TANTA TINTA

AH! MENINA TONTA,

TODA SUJA DE TINTA

MAL O SOL DESPONTA!

(SENTOU-SE NA PONTE,

MUITO DESATENTA\ldots{}

E AGORA SE ESPANTA:

QUEM É QUE A PONTE PINTA

COM TANTA TINTA?\ldots{})

Disponível
em:\textbf{\href{http://www.dominiopublico.gov.br/download/texto/me000588.pdf}{\emph{http://www.dominiopublico.gov.br/download/texto/me000588.pdf}}
Acesso 15 de fev 2023.}

ONDE A MENINA SE SUJOU?

A) NO CHÃO.

B) NO BANCO.

C) NA PONTE.

D) NA ESTRADA.

\protect\hypertarget{_heading=h.49x2ik5}{}{}Habilidade SAEB Inferir
informações em textos verbais.

(A) Está incorreta, pois achou que a menina sentou no chão e se sujou.

C) Está incorreta, por acreditar que a menina sentou no banco.

(C) Está correta, pois a menina sentou na ponte.

(D) Está incorreta, por acreditar que ela caiu no poço na estrada e
caiu.

\subsubsection{02}\label{section-21}

\textbf{LEIA}

A LÍNGUA DO NHEM

HAVIA UMA VELHINHA

QUE ANDAVA ABORRECIDA

POIS DAVA A SUA VIDA

PARA FALAR COM ALGUÉM.

E ESTAVA SEMPRE EM CASA

A BOA DA VELHINHA,

RESMUNGANDO SOZINHA:\\
NHEM-NHEM-NHEM-NHEM-NHEM-NHEM\ldots{}

Disponível
em:\href{http://www.dominiopublico.gov.br/download/texto/me000588.pdf}{\emph{http://www.dominiopublico.gov.br/download/texto/me000588.pdf}}
Acesso 15 de fev 2023.

\textbf{A VELHINHA RESMUNGAVA SOZINHA PORQUE}

A) NÃO TINHA NINGUÉM PARA CONVERSAR.

B) NÃO GOSTAVA DE FALAR COM AS PESSOAS.

C) FICAVA EM CASA SOZINHA.

D) GOSTAVA DE FALAR SOZINHA.

Habilidade SAEB Inferir informações em textos verbais.

(A) Está correta, pois a velhinha dava sua vida para ter alguém para
conversar com ela.

(B) Está incorreta, por acreditar que ela gostava de ficar sozinha.

C) Está incorreta, por acreditar que não tinha ninguém para falar.

(D) Está incorreta, por acreditar a velhinha tinha um hábito da falar
sozinha.

\subsubsection{03}\label{section-22}

A CASA

ERA UMA CASA

MUITO ENGRAÇADA

NÃO TINHA TETO

NÃO TINHA NADA

NINGUÉM PODIA

ENTRAR NELA NÃO

PORQUE NA CASA

NÃO TINHA CHÃO

NINGUÉM PODIA

DORMIR NA REDE

PORQUE NA CASA

NÃO TINHA PAREDE

NINGUÉM PODIA

FAZER PIPI

PORQUE PENICO NÃO TINHA ALI.

MAS ERA FEITA COM MUITO ESMERO NA

RUA DOS BOBOS

NÚMERO ZERO.

Disponível
em:\href{http://www.dominiopublico.gov.br/download/texto/me000588.pdf}{\emph{http://www.dominiopublico.gov.br/download/texto/me000588.pdf}}
Acesso 15 de fev 2023.

NINGUÉM PODIA ENTRAR NA CASA PORQUE

A) NÃO EXISTIA.

B) NÃO TINHA CHÃO.

C) ERA MUITO PEQUENA.

D) ERA MUITO ENGRAÇADA.

\protect\hypertarget{_heading=h.2p2csry}{}{}Habilidade SAEB Inferir
informações em textos verbais.

(A) Está correta, pois não pode existir uma casa sem chão e sem teto,

(B) Está incorreta, por acreditar que se não tinha chão não tinha onde
as pessoas pisarem.

C) Está incorreta, por acreditar que a casa era pequena não tinha como
as pessoas entrarem.

(D) Está incorreta, por acreditar que como a casa era engraçada só podia
entrar palhaço.

\section{7. OS TEXTOS E AS
IMAGENS}\label{muxf3dulo-7-os-textos-e-as-imagens}

\protect\hypertarget{_heading=h.3c1jaqhv8nr}{}{}

\protect\hypertarget{_heading=h.kyrxefm3laa1}{}{}Habilidade do SAEB

\protect\hypertarget{_heading=h.er3p7jk5yewm}{}{}Inferir informações em textos que articulam linguagem verbal e não verbal.

\protect\hypertarget{_heading=h.69rmtrb8c7jb}{}{}

\protect\hypertarget{_heading=h.23ckvvd}{}{}Habilidade da BNCC
EF15LP14.

\subsection{CONTEÚDO}\label{conteuxfado-6}

Explique com clareza o texto da introdução do módulo, leia a tirinha de
forma clara instigando as crianças falarem sobre o que está acontecendo.
\textbf{Habilidade BNCC} EF15LP14

PARA DESCOBRIR UMA INFORMAÇÃO EM UM TEXTO VERBAIS E NÃO VERBAIS VOCÊ
PRECISA PENSAR UM POUCO SOBRE ELAS POIS NÃO APARECE DE FORMA CLARA
MUITAS VEZES ELAS APARECEM NA EXPRESSÃO FACIAL MOSTRANDO AS EMOCOES QUE
ESTÃO SENTIDO, NO FORMATO DOS BALÕES, NAS ONOMATOPEIAS ENTRES OUTROS.

OBSERVE E LEIA ESSE TEXTO:
\includegraphics[width=6.51667in,height=3.14097in]{media/image170.png}

\href{https://olamundo0.files.wordpress.com/2010/05/tn7.png}{\textbf{\emph{https://olamundo0.files.wordpress.com/2010/05/tn7.png}}}

VOCÊ PODE PERCEBER QUE O PERSONAGEM NÃO CONCLUIU O TRABALHO NO FINAL DO
DIA.

CONSEGUE DESCOBRIR PORQUE ISSO ACONTECEU?

NOTE QUE ELE SE DESTRAIU COM OUTRAS COISAS E DEIXOU O TRABALHO DE LADO.

\subsection{ATIVIDADES}\label{atividades-6}

\subsubsection{LEIA O TEXTO}\label{leia-o-texto-2}

\includegraphics[width=5.50365in,height=2.75183in]{media/image171.png}

\href{https://br-linux.org/wparchive/2011/quadrinhos-do-nerdson-em-novo-endereco.php}{\emph{https://br-linux.org/wparchive/2011/quadrinhos-do-nerdson-em-novo-endereco.php}}

\subsubsection{1. O QUE O PERSONAGEM DO PRIMEIRO QUADRINHO ESTÁ
FAZENDO?}\label{o-que-o-personagem-do-primeiro-quadrinho-estuxe1-fazendo}

\_\_\_\_\_\_\_\_\_\_\_\_\_\_\_\_\_\_\_\_\_\_\_\_\_\_\_\_\_\_\_\_\_\_\_\_\_\_\_\_\_\_\_\_\_\_\_\_\_\_\_\_\_\_\_

R: Trabalhando

\subsubsection{2. PARA VOCÊ QUEM É QUE O JACARÉ QUE APARECE NO SEGUNDO
QUADRINHO?}\label{para-vocuxea-quem-uxe9-que-o-jacaruxe9-que-aparece-no-segundo-quadrinho}

\_\_\_\_\_\_\_\_\_\_\_\_\_\_\_\_\_\_\_\_\_\_\_\_\_\_\_\_\_\_\_\_\_\_\_\_\_\_\_\_\_\_\_\_\_\_\_\_\_\_\_\_\_\_\_\_\_\_\_\_

R: O patrão

\subsubsection{LEIA}\label{leia-2}

\includegraphics[width=5.11873in,height=2.15447in]{media/image172.png}

Disponível
em:\textbf{\href{https://aberta.org.br/tag/quadrinhos/}{\emph{https://aberta.org.br/tag/quadrinhos/}}
acesso em 18 Fev 2023.}

\subsubsection{MARQUE A RESPOSTA CERTA COM UM X
}\label{marque-a-resposta-certa-com-um-x}

\subsubsection{3. O MENINO DO PRIMEIRO QUADRINHO LIGOU PARA DANIEL
RESOLVER UM PROBLEMA. COM QUÊ DANIEL
TRABALHA?}\label{o-menino-do-primeiro-quadrinho-ligou-para-daniel-resolver-um-problema.-com-quuxea-daniel-trabalha}

\textbf{x COM COMPUTADOR.}

\textbf{COM DESENHOS.}

\subsubsection{4. O QUE SIGNIFICA A ESPRESSÃO ``OLHA, SÃO INFINITOS! NO
ÚLTIMO BALÃO? MARQUE COM UM
X.}\label{o-que-significa-a-espressuxe3o-olha-suxe3o-infinitos-no-uxfaltimo-baluxe3o-marque-com-um-x.}

\textbf{QUE SÃO POUCO JOGOS PARA INSTALAR.}

\textbf{x QUE SÃO MUITOS JOGOS PARA INSTALAR.}

\subsubsection{LEIA A TIRINHA.}\label{leia-a-tirinha.}

\textbf{Para essa atividade leia com clareza a tirinha para os alunos
explorado falas, gestos e expressão facial.}

\includegraphics[width=2.02083in,height=1.75347in]{media/image173.png}\includegraphics[width=1.93472in,height=1.75347in]{media/image174.png}\includegraphics[width=1.70069in,height=1.64028in]{media/image175.png}

Disponível
em:\textbf{\href{https://www.storyboardthat.com/portal/storyboards/b5bbeb4e/class-storyboard/unknown-story3}{\emph{https://www.storyboardthat.com/portal/storyboards/b5bbeb4e/class-storyboard/unknown-story3}}
acesso 17 Fev 2023.}

\subsubsection{5. QUAIS SÃO OS NOME DOS PERSONAGENS DA
TIRINHA?}\label{quais-suxe3o-os-nome-dos-personagens-da-tirinha}

\textbf{\_\_\_\_\_\_\_\_\_\_\_\_\_\_\_\_\_\_\_\_\_\_\_\_\_\_\_\_\_\_\_\_\_\_\_\_\_\_\_\_\_\_\_\_\_\_\_\_\_\_\_\_\_\_\_\_\_\_\_\_\_\_\_\_\_\_}

\textbf{R: Talita e Felipe.}

\subsubsection{6. COMO TALITA FICOU QUANDO FICOU SABENDO QUE SEU AMIGO
FELIPE IA
EMBORA?}\label{como-talita-ficou-quando-ficou-sabendo-que-seu-amigo-felipe-ia-embora}

\textbf{\_\_\_\_\_\_\_\_\_\_\_\_\_\_\_\_\_\_\_\_\_\_\_\_\_\_\_\_\_\_\_\_\_\_\_\_\_\_\_\_\_\_\_\_\_\_\_\_\_\_\_\_\_\_\_\_\_\_\_\_\_\_\_\_\_\_\_\_}

\textbf{R: chorando.}\protect\hypertarget{_heading=h.nqfz6n7t1lqi}{}{}

\subsubsection{7. A EXPRESSÃO FACIAL É UM RECURSO USADO NAS HISTÓRIAS EM
QUADRINHOS. QUAL RECUSRO FOI USADO PARA MOSTRAR QUE A MENINA ESTÁ
CHORANDO?}\label{a-expressuxe3o-facial-uxe9-um-recurso-usado-nas-histuxf3rias-em-quadrinhos.-qual-recusro-foi-usado-para-mostrar-que-a-menina-estuxe1-chorando}

\includegraphics[width=3.23889in,height=2.81042in]{media/image173.png}

\protect\hypertarget{_heading=h.ihv636}{}{}

\textbf{x AS LÁGRIAMS.}

\textbf{MÃO NO ROSTO.}

\textbf{BOCA ABERTA.}

\subsubsection{LEIA ESSE QUADRINHO.}\label{leia-esse-quadrinho.}

\includegraphics[width=2.31319in,height=2.23333in]{media/image175.png}

\subsubsection{8. O QUE SIGNIFICA A EXPRESSÃO USADA PELA MENINA QUE ESTÁ
AJOELHADA?}\label{o-que-significa-a-expressuxe3o-usada-pela-menina-que-estuxe1-ajoelhada}

\textbf{x ALÍVIO.}

\textbf{TRISTEZA.}

\textbf{MEDO. }

\subsubsection{LEIA O QUADRINHO}\label{leia-o-quadrinho}

\includegraphics[width=2.81319in,height=2.49306in]{media/image176.png}

\href{https://www.storyboardthat.com/portal/storyboards/b5bbeb4e/class-storyboard/unknown-story5}{\textbf{\emph{https://www.storyboardthat.com/portal/storyboards/b5bbeb4e/class-storyboard/unknown-story5}}}

\subsubsection{9. OS BALÕES TEM UM FORMATO DIFERENTE PARA MOSTRAR QUE AS
CRIANÇAS
ESTÃO:}\label{os-baluxf5es-tem-um-formato-diferente-para-mostrar-que-as-crianuxe7as-estuxe3o}

\textbf{x GRITANDO. }

\textbf{CANTADO.}

\textbf{SORRINDO.}

\subsubsection[10. O QUE VOCÊ ACHA QUE ESTÁ ACONTECENDO NESSAS IMAGENS?
VAMOS CRIAR AS FALAS DOS PESSONAGENS E ESCREVER NOS
BALÕES.]{\texorpdfstring{10. O QUE VOCÊ ACHA QUE ESTÁ ACONTECENDO NESSAS
IMAGENS? VAMOS CRIAR AS FALAS DOS PESSONAGENS E ESCREVER NOS
BALÕES.\protect\includegraphics[width=1.55139in,height=1.32153in]{media/image177.png}\protect\includegraphics[width=1.70417in,height=1.53958in]{media/image178.png}\protect\includegraphics[width=1.56667in,height=1.56736in]{media/image179.png}}{10. O QUE VOCÊ ACHA QUE ESTÁ ACONTECENDO NESSAS IMAGENS? VAMOS CRIAR AS FALAS DOS PESSONAGENS E ESCREVER NOS BALÕES.}}\label{o-que-vocuxea-acha-que-estuxe1-acontecendo-nessas-imagens-vamos-criar-as-falas-dos-pessonagens-e-escrever-nos-baluxf5es.}

\href{https://www.storyboardthat.com/portal/storyboards/b5bbeb4e/class-storyboard/unknown-story4}{\textbf{\emph{https://www.storyboardthat.com/portal/storyboards/b5bbeb4e/class-storyboard/unknown-story4}}}

\textbf{Resposta pessoal.}

\subsection{TREINO}\label{treino-6}

\textbf{Os três itens a seguir estão ordenados do mais fácil para o mais
difícil. }

\subsubsection{01 }\label{section-23}

\includegraphics[width=5.90556in,height=1.95278in]{media/image180.png}

Disponível em
https://www.storyboardthat.com/portal/storyboards/b5bbeb4e/class-storyboard/unknown-story6

O QUE A MENINA ESTÁ PEDIDO PARA A AMIGA DEVOLVER?

A) O SAPATO.

B) O VESTIDO.

C) A CORTINA.

D) A CAMA.

Habilidade SAEB Inferir informações em textos verbais e não verbais.

\textbf{Habilidade BNCC} EF15LP14 Construir o sentido de histórias em
quadrinhos e tirinhas, relacionando imagens e palavras e interpretando
recursos gráficos (tipos de balões, de letras, onomatopeias).

(A) Está incorreta, pois tem um sapato no quarto.

(B) Está correta, por a menina de blusa amarela está o tempo todo com o
vestido na mão.

(C) Está incorreta, por acreditar que a menina pegou a cortina da janela.

(D) Está incorreta, por acreditar que a menina ia dormir na cama.

\subsubsection{02 }\label{section-24}

\includegraphics[width=6.12986in,height=2.72778in]{media/image181.png}

\textbf{https://www.storyboardthat.com/portal/storyboards/b5bbeb4e/class-storyboard/unknown-story7}

A MENINA DE CAMISETA VERDE ESTÁ BRAVA COM A AMIGA PORQUE

A) FOI EMBORA.

B FEZ CHACOTAS.

C) PULOU NA PISCINA.

D) JOGOU SUA BOLSA NA PSICINA

Habilidade SAEB Inferir informações em textos verbais e não verbais.

\textbf{Habilidade BNCC} EF15LP14 Construir o sentido de histórias em
quadrinhos e tirinhas, relacionando imagens e palavras e interpretando
recursos gráficos (tipos de balões, de letras, onomatopeias).

(A) Está incorreta, pois no segundo quadrinho a menina está indo embora.

(B) Está incorreta, pois no primeiro quadrinho a menina pediu que ela
adivinhasse.

(C) Está incorreta, por acreditar que o recurso utilizado para fazer
barulho na piscina seria a menina pulando na piscina.

(D) Está correta, pois a bolsa está dentro da
piscina.\protect\hypertarget{_heading=h.tt4s01xrdwwn}{}{}

\subsubsection{03 }\label{section-25}

LEIA O
TEXO.\includegraphics[width=6.07986in,height=2.24583in]{media/image182.png}

\textbf{https://www.storyboardthat.com/portal/storyboards/b5bbeb4e/class-storyboard/unknown-story6}

DE ACORDO COM O TEXTO A MENINA DE BLUSA ROSA NO PRIMEIRO QUADRINHO ESTÁ

A) CONFUSA.

B) NERVOSA.

C) CHATEADA.

D) RECLAMANDO.

Habilidade SAEB Inferir informações em textos verbais e não verbais.

\textbf{Habilidade BNCC} EF15LP14 Construir o sentido de histórias em
quadrinhos e tirinhas, relacionando imagens e palavras e interpretando
recursos gráficos (tipos de balões, de letras, onomatopeias).

(A) Está incorreta, por acreditar que a menina não fazia ideia do que a
amiga estava fazendo no quarto.

(B) Está incorreta, por acreditar a menina ficou nervosa ao ver a outra
nos seu quarto.

(C) Está incorreta, por acreditar que ela não gostou em ver a amiga no
seu quarto.

(D) Está correta, pois a posição dos braços mostra que ele está
reclamando alguém.

\section{SIMULADO 1}\label{simulado-1}

\subsubsection{01}\label{section-26}

\textbf{VEJA OS CARTAZES QUE UMA GRÁFICA FEZ PARA UMA EMCOMNDA.}

\begin{longtable}[]{@{}llll@{}}
\toprule
\includegraphics[width=1.51042in,height=2.00000in]{media/image183.jpg} &
\includegraphics[width=1.19792in,height=1.68472in]{media/image184.jpg} &
\includegraphics[width=1.08681in,height=1.72778in]{media/image185.jpg} &
\includegraphics[width=1.30208in,height=1.73819in]{media/image186.jpg}\tabularnewline
\textbf{( 1 )} & \textbf{(2)} & \textbf{( 3)} & \textbf{( 4
)}\tabularnewline
\bottomrule
\end{longtable}

\href{https://www.freepik.com/free-vector/traffic-signs-set_724943.htm\#query=SINAIS\%20DE\%20TRASITON\&position=0\&from_view=search\&track=ais}{\textbf{\emph{https://www.freepik.com/free-vector/traffic-signs-set\_724943.htm\#query=SINAIS\%20DE\%20TRASITON\&position=0\&from\_view=search\&track=ais}}}

\textbf{https://img.freepik.com/free-vector/hand-drawn-colorful-alphabet\_52683-270.jpg?t=st=1676138441\textasciitilde{}exp=1676139041\textasciitilde{}hmac=fd4a6f3745122f90a5c50b91c6585606060da896eafa738ce6744da24203a83bhttps://www.freepik.com/free-vector/hand-drawn-colorful-alphabet\_2920858.htm\#query=ALFABETO\&position=1\&from\_view=search\&track=sph}

\textbf{O CARTAZ QUE SÓ APARECE LETRAS É}

\textbf{(A) 4}

\textbf{(B) 3}

\textbf{(C) 2 }

\textbf{(D) 1}

Habilidade Saeb

Relacionar elementos sonoros das palavras com sua representação escrita.

Habilidade BNCC EF01LP04 Distinguir as letras do alfabeto de outros
sinais gráficos.

(A) Correta, pois na placa aparece apenas letras do alfabeto.

(B) Incorreta, pois na placa aparece outros sinais gráficos como números
e o @.

(C) Incorreta, pois na placa aparece outros sinais gráficos como números
\$ e \%.

(D) Incorreta, pois na placa aparece outros sinais gráficos como números
e o \&.

\subsubsection{02}\label{section-27}

VEJA O ANIMAL DE ESTIMAÇÃO QUE DANIEL GANHOU.

\includegraphics[width=1.91250in,height=1.79167in]{media/image187.png}

https://www.freepik.com/free-vector/cute-cat-cartoon-characters-illustrations-set-cats-with-heart-shaped-noses-happy-fluffy-kittens-smiling-orange-grey-kitties-sitting-white\_20827610.htm\#query=GATO\&position=26\&from\_view=search\&track=sph

O SOM DA LETRA INICIAL DO NOME DO ANIMAL QUE DANIEL GANHOU É

(A) R.

(B) P.

(C) G .

(D) K.

Habilidade Saeb Relacionar elementos sonoros das palavras com sua
representação escrita.

Habilidade BNCC EF01LP05 Reconhecer o sistema de escrita alfabética como
representação dos sons da fala.

(A) Incorreta, pois confundiu o som das letras g com r.

(B) Incorreta, pois confundiu o som das letras g com o som de p.

(C) Está correta, pois a palavra começa com esse som.

(D) Está incorreta. pois confundiu o som de g com com k.

\subsubsection{03}\label{section-28}

OBSERVE A PALAVRA QUE JOÃO ESCREVEU NO
PAPEL.\includegraphics[width=1.29792in,height=1.27986in]{media/image188.png}

Elaborado pela autora.

QUAL PALAVRA TEM A MESMA QUANTIDADE DE SOM DA PALAVRA QUE ELE ESCREVEU?

(A) MACACO.

(B) BONECA.

(C) PATINETE.

(D) PIPA.

Habilidade Saeb Relacionar elementos sonoros das palavras com sua
representação escrita.

Habilidade BNCC EF01LP07 Identificar fonemas e sua representação por
letras.

(A) Incorreta. A palavra está representada por seis sons.

(B) Incorreta.A palavra possui mais de seis sons.

(C) Incorreta. A palavra possui oito sons.

(D) Correta:A palavra está representada por quatro sons, assim como na
palavra bola.

\subsubsection{04}\label{section-29}

\protect\hypertarget{_heading=h.1hmsyys}{}{}VEJA O TÊNIS QUE MATEUS
GANHOU DA VOVÓ.

\includegraphics[width=1.87986in,height=2.41667in]{media/image189.jpg}

Disponível em:
\href{https://www.freepik.com/premium-photo/bright-yellow-leather-sneakers-casual-women-s-style-white-lacing-white-rubber-soles-isolated-closeup-white-background-top-view-fashion-shoes_28012413.htm\#page=7\&query=TENIS\%20AMARELO\&position=15\&from_view=search\&track=ais}{\emph{https://www.freepik.com/premium-photo/bright-yellow-leather-sneakers-casual-women-s-style-white-lacing-white-rubber-soles-isolated-closeup-white-background-top-view-fashion-shoes\_28012413.htm\#page=7\&query=TENIS\%20AMARELO\&position=15\&from\_view=search\&track=ais}}.
Acesso em 19 de Fev 2023.

QUAL É A SÍLABA FINAL DO NOME DA COR DO TÊNIS QUE MATEUS GANHOU?

A) MA.

B) RE.

C) LO.

D) LU.

\protect\hypertarget{_heading=h.41mghml}{}{}Habilidade Saeb Relacionar
elementos sonoros das palavras com sua representação escrita.

Habilidade BNCC EF01LP08 Relacionar elementos sonoros (sílabas, fonemas,
partes de palavras) com sua representação escrita.

\protect\hypertarget{_heading=h.2grqrue}{}{}(A) Incorreta, pois não se
atentou a posição da sílaba.

(B) Incorreta, pois confundiu a posição da sílaba..

(C) Correta, pois a sílaba final da palavra amarela é LO.

(D) Incorreta, pois confundiu o som das sílabas.

\subsubsection{05}\label{section-30}

JULIA MOSTROU PARA SUA PRIMA QUE SABE LER PALVRAS.

VEJA A PALVRA QUE ELA LEU.

\includegraphics[width=2.64104in,height=1.80504in]{media/image190.png}\textbf{Elaborado
pela autora}

\textbf{Disponível
em:https://www.storyboardthat.com/class-storyboardboards/b5bbeb4e/unknown-story9/editAcesso
18 fev 2023.}

QUAL PALAVRA TEM O MESMO O SOM SÍLABA MEDIAL DA PALAVRA QUE JULIA LEU?

A) SALADA.

B) SACOLA.

C) LÂMPADA.

D) PANELA.

Habilidade Saeb Relacionar elementos sonoros das palavras com sua
representação escrita.

Habilidade BNCC EF01LP09 Comparar palavras, identificando semelhanças e
diferenças entre sons de sílabas iniciais, mediais e finais.

(A) correta, pois as palavras terminam com o mesmo som.

(B) incorreta, pois confundiu o som medial como final.

(C) Incorreta, pois confundiu o som medial com o inicial.

(D) Incorreta, pois confundiu o som medial com o final.

\subsubsection{06}\label{section-31}

VEJA O PRESENTE QUE TOMAS GANHOU DO SEU AMIGO.

\includegraphics[width=2.60694in,height=1.48889in]{media/image191.jpg}

Disponivel
em:https://www.freepik.com/free-vector/realistic-color-men-s-shoes-set\_14683166.htm\#query=sapato\&position=44\&from\_view=search\&track=sph.acesso
19 de Fev 2023.

O NOME DO PRESENTE QUE TOMAS GANHOU É

A) SALADA.

B) SAPATO.

C) TOMATE.

D) PANELA.

Habilidade SAEB Ler palavras.

Habilidade BNCC EF12LP01 Ler palavras novas com precisão na
decodificação, no caso de palavras de uso frequente, ler globalmente,
por memorização.

(A) Incorreta, pois confundiu em razão da sílaba inicial ser a mesma.

(B) Correta, pois o nome da figura é sapato.

\protect\hypertarget{_heading=h.vx1227}{}{}
(C) Incorreta, pois acreditou que pelo fato de ter a silaba to igual seria o nome da palavra.

(D) Incorreta, pois acreditou que pelo fato de ter a silaba pa igual
seria o nome da
palavra.\protect\hypertarget{_heading=h.bdexdhbfzr3s}{}{}

\subsubsection{07}\label{section-32}

OBSERVE A FRUTA PREFERIDA DE SOFIA.

\includegraphics[width=1.15208in,height=1.72708in]{media/image192.jpg}

Disponível
em: https://www.freepik.com/free-photo/pineapple-fruit\_1123681.htm\#query=ABACAXI\&position=3\&from\_view=search\&track=sphAcesso
19 fev 2023.

ESCREVA O NOME DESSA FRUTA.

\_\_\_\_\_\_\_\_\_\_\_\_\_\_\_\_\_\_\_\_\_\_\_\_\_\_\_\_\_\_\_\_\_\_\_\_\_\_\_\_\_\_\_\_\_\_\_\_\_\_\_\_\_\_\_\_\_\_\_\_\_\_\_\_\_\_\_\_\_\_

\protect\hypertarget{_heading=h.3fwokq0}{}{}POSSIVEIS RESPOSTAS:

Abacaxi

Abxica

Acaxi

abacai

Habilidade Saeb Escrever palavras.

\protect\hypertarget{_heading=h.1v1yuxt}{}{}Habilidade BNCC EF01LP02
Escrever, espontaneamente ou por ditado, palavras e frases de forma
alfabética -- usando letras/grafemas que representem fonemas

Justificativa:

O estudante que escrever corretamente a palavra ABACAXI domina a
habilidade de escrever palavras.

\subsubsection{08}\label{section-33}

VEJA A PALAVRA QUE GUTO ESCREVEU.

\includegraphics[width=1.76042in,height=3.19514in]{media/image197.png}

Elaborada pela autora

Disponível
em;https://www.storyboardthat.com/class-storyboardboards/b5bbeb4e/unknown-story9/edita
Acesso em 19 fev 2023.

A PALAVRA QUE TERMINA COM O MESMO SOM DA SÍLABA PALAVRA QUE GUTO ESCREVEU
É

A) GILETE.

B) PETECA.

C) TESOURA.

D) TAMANCO.

Habilidade SAEB Ler palavras.

Habilidade BNCC EF01LP13 Comparar palavras, identificando semelhanças e
diferenças entre sons de sílabas iniciais, mediais e finais.

( A ) correta, pois a duas palavras termina com o mesmo som.

(B) incorreta. por acreditar eu pelo fato de ter parte da palavra
escrito.

(C) Incorreta por confundir a sílaba final com a inicial.

(D) Incorreta por acreditar que pelo fato das duas palavras ter a sílaba
inicial igual.

\subsubsection{09}\label{section-34}

LEIA ESSA FRASE.

\includegraphics[width=2.80293in,height=1.64007in]{media/image198.png}elaborada
pela autora

Disponível em
https://www.storyboardthat.com/class-storyboardboards/b5bbeb4e/unknown-story9/edit

A IMAGEM QUE REPRESENTA O QUE ESTÁ ESCRITO NA FRASE É

A)
B)\includegraphics[width=1.53194in,height=2.16250in]{media/image199.jpg}

\includegraphics[width=1.73760in,height=1.66822in]{media/image200.jpg}

C) D)
\includegraphics[width=1.65208in,height=2.13056in]{media/image201.jpg}

\includegraphics[width=1.21736in,height=2.20972in]{media/image202.jpg}

Habilidade Saeb LER FRASES.

Habilidade BNCC EF01LP01 Reconhecer que textos são lidos e escritos da
esquerda para a direita e de cima para baixo da página.

(A) correta, pois o menino está chutando a bola.

(B) incorreta, pois o menino seguro a bola.

(C) Incorreta, pois o menino está com a bola na cabeça.

(D) Incorreta, pois o menino está batendo a bola com o braço.

\href{https://www.freepik.com/free-vector/boy-cartoon-character-playing-football_14801681.htm\#query=MENINO\%20JOGA\%20BOLA\&position=40\&from_view=search\&track=ais}{\emph{https://www.freepik.com/free-vector/boy-cartoon-character-playing-football\_14801681.htm\#query=MENINO\%20JOGA\%20BOLA\&position=40\&from\_view=search\&track=ais}}

\href{https://img.freepik.com/free-vector/american-boy-holding-color-ball_1308-46958.jpg?t=st=1676805030~exp=1676805630~hmac=b62a5a2379b794be31f502b669f8575bc0960322087c2677a62dbfda14da314b}{\emph{https://img.freepik.com/free-vector/american-boy-holding-color-ball\_1308-46958.jpg?t=st=1676805030\textasciitilde{}exp=1676805630\textasciitilde{}hmac=b62a5a2379b794be31f502b669f8575bc0960322087c2677a62dbfda14da314b}}

\href{https://www.freepik.com/free-vector/young-boy-playing-volleyball_5284614.htm\#page=3\&query=MENINO\%20CHUTA\%20BOLA\&position=47\&from_view=search\&track=ais}{\emph{https://www.freepik.com/free-vector/young-boy-playing-volleyball\_5284614.htm\#page=3\&query=MENINO\%20CHUTA\%20BOLA\&position=47\&from\_view=search\&track=ais}}

\subsubsection{10}\label{section-35}

\textbf{O CRAVO E A ROSA}

O CRAVO BRIGOU COM A ROSA

DEBAIXO DE UMA SACADA

O CRAVO SAIU FERIDO

E A ROSA DESPEDAÇADA.

O CRAVO FICOU DOENTE

A ROSA FOI VISITAR

O CRAVO TEVE UM DESMAIO

E A ROSA PÔS-SE A CHORAR.

\emph{\url{http://www.dominiopublico.gov.br/download/texto/me000588.pdf}.
Acesso dia 15 de fev 2023.}

O ASSUNTO DESSE TEXTO É

A) O CRAVO QUE FICOU DOENTE.

B) A ROSA QUE FICOU DESPEDAÇADA.

C) A BRIGA DO CRAVO COM A ROSA.

D) A VISITA DO CRAVO PARA A ROSA.

Habilidade SAEB~Inferir o assunto de um texto.

\protect\hypertarget{_heading=h.4f1mdlm}{}{}(A) Incorreta. Por acreditar
que como o cravo ficou doente no texto esse seria o assunto principal.

(B) Incorreta, por considerar a rosa ficou assim depois da briga.

( C ) Correta, pois houve a briga do cravo e da rosa debaixo de uma
sacada.

(D) Incorreta, pois confundiu a briga com uma visita.

\subsubsection{11}\label{section-36}

LEIA O TEXTO:

A SEMANA INTEIRA

A SEGUNDA FOI À FEIRA

PRECISAVA DE FEIJÃO;

A TERÇA FOI À FEIRA

PRA COMPRAR UM PIMENTÃO;

A QUARTA FOI À FEIRA

PRA BUSCAR QUIABO E PÃO;

A QUINTA FOI À FEIRA,

POIS GOSTAVA DE AGRIÃO;

A SEXTA FOI A FEIRA,

TEM BANANA? TEM MAMÃO?

SÁBADO NÃO TEM FEIRA

E DOMINGO TAMBÉM NÃO.

Sérgio Caparelli.

Disponível
em:\textbf{\emph{\href{https://www2.bauru.sp.gov.br/arquivos/arquivos_site/sec_educacao/atividades_pedagogica_distancia/1;Infantil/61;EMEI\%20Rosangela\%20Vieira\%20M.\%20de\%20Carvalho/05;PROF.\%C2\%AA\%20MARISA/Semana\%2019\%20(26\%20a\%2030\%20de\%20Julho)\%20Infantil\%20V.\%20Prof\%C2\%AA\%20Marisapdf.pdf}{https://www2.bauru.sp.gov.br/arquivos/arquivos\_site/sec\_educacao/atividades\_pedagogica\_distancia/1;infantil/61;emei\%20rosangela\%20vieira\%20m.\%20de\%20carvalho/05;prof.\%c2\%aa\%20marisa/semana\%2019\%20(26\%20a\%2030\%20de\%20julho)\%20infantil\%20v.\%20prof\%c2\%aa\%20marisapdf.pdf}.
acesso em 18 de Fev 2023.}}

QUAIS DIAS DA SEMANA PODE COMPRAR PIMENTÃO E MAMÃO?

A) SÁBADO E DOMINGO.

B) DOMINGO E TERÇA.

C) QUARTA E SÁBADO.

D) TERÇA E SEXTA.

SAEB Localizar informações explícitas em textos.

(A) Incorreta, pois sábado e domingo não tem feira.

( B )Incorreta, pois na terça vende pimentão mas aos domingo não tem
feira.

(C) Incorreta. pois quarta vende quiabo e pão e sábado não tem feira.

(C) Correta, pois na terça vende pimentão e sexta vende mamão.

\subsubsection{12}\label{section-37}

LARA VAI CONVIDAR ALICE PARA SEU ANIVERSÁRIO.

QUAL TEXTO ELA VAI MANDAR PARA ELA?

A) BILHETE.

B). CONVITE.

C) NOTÍCIA.

D) RECEITA

Habilidade SAEB Reconhecer a finalidade de um texto.

Habilidade BNCC EF15LP01 Identificar a função social de textos que
circulam em campos da vida social dos quais participa cotidianamente (a
casa, a rua, a comunidade, a escola) e nas mídias impressa, de massa e
digital, reconhecendo para que foram produzidos, onde circulam, quem os
produziu e a quem se destinam.

(A)Incorreta, pois se confundiu pelo fato do bilhete deixar um recado.

(B) Correta, pois

(C) Incorreta. pois acreditou que ia sair uma notícia do evento.

(D) Incorreta, pois acreditou que ia mandar uma receita do bolo do
aniversário.\protect\hypertarget{_heading=h.s4vmdpjsaa88}{}{}

\subsubsection{13}\label{section-38}

VEJA O CARTAZ.

\includegraphics[width=3.10650in,height=1.94326in]{media/image204.png}

Disponível
em:\href{https://www.itarana.es.gov.br/portal/artigo/prefeitura-municipal-de-itarana-apresenta-programacao-do-carnaval-2020-com-blocos-de-rua-e-shows-noturnos}{\emph{https://www.itarana.es.gov.br/portal/artigo/prefeitura-municipal-de-itarana-apresenta-programacao-do-carnaval-2020-com-blocos-de-rua-e-shows-noturnos}}
Acesso 19 Fev 2023.

\protect\hypertarget{_heading=h.2u6wntf}{}{}ESSE TEXTO SERVE PARA

A) ORGANIZAR TAREFAS.

B) ENSINAR UMA RECEITA

C) ANUNCIAR UM EVENTO.

D) CONTAR UMA HISTÓRIA.

Habilidade SAEB Reconhecer a finalidade de um texto.

Habilidade BNCC EF15LP01 Identificar a função social de textos que
circulam em campos da vida social dos quais participa cotidianamente (a
casa, a rua, a comunidade, a escola) e nas mídias impressa, de massa e
digital, reconhecendo para que foram produzidos, onde circulam, quem os
produziu e a quem se destinam.

(A) Incorreta, por acreditar que todos deveria ser organizar para
participar da festa.

(B) Incorreta, por acreditar que ia ensinar uma receita para ficar
forte no carnaval.

(C) Correta. pois o cartaz anuncia o evento do carnaval.

(D) Incorreta, por acreditar que o cartaz seria a capa de um livro de
conto.\protect\hypertarget{_heading=h.gluii2ldfhnb}{}{}

\subsubsection{14}\label{section-39}

LEIA O TRECHO DA HISTÓRIA JOÃO E MARIA.

NA MADRUGADA DO DIA SEGUINTE, A MADRASTA ACORDOU AS CRIANÇAS E FORAM
NOVAMENTE PARA A MATA. ENQUANTO CAMINHAVAM, JOÃOZINHO ESFARELOU TODO O
SEU PÃO E O DA IRMÃ, FAZENDO UMA TRILHA. DESSA VEZ SE AFASTARAM AINDA
MAIS DE CASA E, CHEGANDO A UMA CLAREIRA, OS PAIS DEIXARAM AS CRIANÇAS
COM A DESCULPA DE CORTAR LENHA, ABANDONANDO-AS. JOÃO E MARIA
ADORMECERAM, POR FOME E CANSAÇO E, QUANDO ACORDARAM, ESTAVA MUITO
ESCURO. MARIA DESATOU A CHORAR. MAS, DESTA VEZ, NÃO CONSEGUIRAM
ENCONTRAR O CAMINHO: OS PÁSSAROS DA MATA TINHAM COMIDO TODAS AS
MIGALHAS.

Disponível
em:\href{http://www.dominiopublico.gov.br/download/texto/me001614.pdf}{\emph{http://www.dominiopublico.gov.br/download/texto/me001614.pdf}}
Acesso 19 de Fev 2023.

JOÃO FEZ A TRILHA DE PÃO PARA

A) ALIMENTAR OS PASSARINHOS.

B) CONSEGUIR VOLTAR PARA CASA.

C) COMER QUANDO SENTISSE FOME.

D) FUGIR DO PAI E DA MADASTRA.

Habilidade SAEB~Inferir informações em textos verbais.

(A) incorreta, por acreditar que os passarinhos da floresta estavam com
fome por João estava alimentado.

(B) correta, pois João teve esse plano pois sabia que ele e sua irmão
iriam ficar sozinhos na floresta.

(C) incorreta. pois acreditou que João e Maria iam precisar encomonnizar
o pão para comer por muitos dias na floresta.

(D) incorreta, pois acreditou que João e Maria planejava fugir dos pais
já que queria abandona- ló na floresta.

\subsubsection{15}\label{section-40}

OBSERVE A TIRINHA:

\includegraphics[width=5.42742in,height=2.68116in]{media/image206.png}

Disponivel em:
https://www.storyboardthat.com/portal/storyboards/b5bbeb4e/class-storyboard/unknown-story9

COMO A MENINA FICOU QUANDO FALTOU LUZ?

A) TRISTE.

B) GRITANDO.

C) ASSUSTADA.

D) RECLAMANDO.

Habilidade Saeb~Inferir informações em textos que articulam linguagem
verbal e não verbal

\textbf{Habilidade BNCC} EF15LP14 Construir o sentido de histórias em
quadrinhos e tirinhas, relacionando imagens e palavras e interpretando
recursos gráficos (tipos de balões, de letras, onomatopeias).

(A) incorreta, por acreditar que a menina não pode assistir seu desenho
favorito.

(B) Incorreta, pois acreditou que a menina estava gritando de medo.do
escuro.

(C) Correta, pois a expressão facial da menina mostra que ela ficou
muito assustada.

(D) incorreta, por acreditar que a menina estava reclamando por que não
terminou de assistir o desenho.

\section{SIMULADO 2}\label{simulado-2}

\subsubsection{01}\label{section-41}

MARIANA QUER MOSTRAR PARA SUA AMIGA BIA QUE JÁ SABE ESCREVER O NOME DA
SUA GATINHA MILU. PARA ISSO ELA VAI ESCOLHER UMA MALETA DE SÍMBOLOS.

\includegraphics[width=1.40208in,height=1.03194in]{media/image209.png}

\begin{enumerate}
\def\labelenumi{\arabic{enumi}.}
\item
\end{enumerate}

\includegraphics[width=1.62986in,height=0.98819in]{media/image209.png}

2.

\includegraphics[width=1.55139in,height=0.85000in]{media/image209.png}

3.

\includegraphics[width=1.31218in,height=0.92930in]{media/image209.png}

4.

IMAGENS ELABORADAS PELA AUTORA

QUAL MALETA MARIANA DEVE ESCOLHER?

(A) 1

(B) 2

(C) 3

D) 4

Habilidade Saeb

Relacionar elementos sonoros das palavras com sua representação escrita.

BNCC EF01LP04 Distinguir as letras do alfabeto de outros sinais
gráficos.

( A ) Incorreta, pois na maleta aparece outros símbolos que não serve
para escrever palavras.

(B) Incorreta, pois na maleta aparece apenas números.

(C) Correta, pois na maleta aparece as letras do alfabeto usadas para
formas palavras.

(D) Incorreta, pois na placa aparece outros símbolos diversos que não
serve para escrever palavras.

\subsubsection{02}\label{section-42}

\textbf{OBSERVE O NOME DA FRUTA PREFERIDA DE ALICE.}

\includegraphics[width=3.32292in,height=1.46928in]{media/image212.png}

\textbf{Disponível
em:https://www.freepik.com/free-vector/square-labels-with-fresh-fruits-illustration\_1164321.htm\#query=FRUTA\%20PLACA\&position=7\&from\_view=search\&track=ais}

\textbf{QUAL PALAVRA COMEÇA E TERMINA COM SOM DE VOGAL ASSIM COMO O NOME
DA FRUTA PREFERIDA DE ALICE?}

\textbf{(A) MACACO.}

\textbf{(B) ABELHA.}

\textbf{(C) COELHO.}

\textbf{(D) GATO.}

Habilidade Saeb Relacionar elementos sonoros das palavras com sua
representação escrita.

Habilidade BNCC EF01LP07 Identificar fonemas e sua representação por
letras.

(A) Incorreta, pois a palavra começa com a consoante m e termina com a
vogal o.

(B) Correta, pois a palavra começa e termina com vogal assim como a
palavra abacate.

(C) Incorreta, pois a palavra começa com a consoante c e termina com a
vogal o.

(D) Incorreta, pois a palavra começa com a consoante g e termina com a
vogal o.

\subsubsection{03}\label{section-43}

OBSERVE A PALAVRA QUE ANA APRENDEU A LER.

\textbf{FADA}

QUAL PALAVRA TEM A MESMA QUANTIDADE DE SOM DA PALAVRA QUE ELA LEU É

(A) SALA.

(B) TOMATE.

(C) BONECA.

(D) BICICLETA.

Habilidade Saeb Relacionar elementos sonoros das palavras com sua
representação escrita.

Habilidade BNCC EF01LP07 Identificar fonemas e sua representação por
letras.

(A) Correta, pois palavra está representada por quatro sons, assim como
na palavra bola.

(B) Incorreta, pois a palavra está representada por seis sons.

(C) Incorreta, pois a palavra possui mais de quatro sons.

(D) Incorreta, pois a palavra possui oito sons.

\subsubsection{04}\label{section-44}

MARINA JÁ CONSEGUE LER ALGUMAS PALAVRAS.

VEJA A PALAVRA QUE ELA LEU.

\includegraphics[width=2.07917in,height=1.19306in]{media/image213.png}

\textbf{Elaborada pela autora}

AS LETRAS QUE FORMAM O SOM DO MEIO DA PALAVRA QUE ANA LEU SÃO

(A) B+O.

(B) N+E.

(C) C+A.

(D) M+E.

Habilidade Saeb Relacionar elementos sonoros das palavras com sua
representação escrita.

Habilidade BNCC EF01LP05 Reconhecer o sistema de escrita alfabética como
representação dos sons da fala.

(A) Incorreta, pois confundiu com o som inicial.

(B) Correta, pois o som do meio da palavra boneca é ne.

(C) Incorreta, pois confundiu com o som final.

(D) Incorreta, pois confundiu o som de me com
ne.\protect\hypertarget{_heading=h.1meprgjlpdif}{}{}

\subsubsection{05}\label{section-45}

VEJA O ANIMAL DE ESTIMAÇÃO DE MIGUEL.

\includegraphics[width=1.65208in,height=1.75972in]{media/image214.jpg}

Disponível
em:\href{https://www.freepik.com/free-vector/different-kind-puppy-dogs-illustration_1169375.htm\#page=5\&query=CACHORRO\&position=37\&from_view=search\&track=sph}{\emph{https://www.freepik.com/free-vector/different-kind-puppy-dogs-illustration\_1169375.htm\#page=5\&query=CACHORRO\&position=37\&from\_view=search\&track=sph}}.
Acesso 19 Fev 2023.

A PALAVRA QUE COMEÇA COM A MESMA SÍLABA DO NOME DO ANIMAL DE MIGUEL É

A) CANECA.

B) BONECA.

C) CAVALO.

D) TUCANO.

Habilidade Saeb Relacionar elementos sonoros das palavras com sua
representação escrita.

Habilidade BNCC EF01LP09 Comparar palavras, identificando semelhanças e
diferenças entre sons de sílabas iniciais, mediais e finais.

(A) Correta, pois a palavra cachorro começa com ca assim como a palavra
caneca.

(B) Incorreta, pois confundiu a última sílaba com a primeira.

(C) Incorreta, pois confundiu a primeira sílaba com a última.

(D) Incorreta, pois confundiu a sílaba inicial com a do meio.

\subsubsection{06}\label{section-46}

VEJA O ANIMAL QUE TIAGO COMPROU PARA ALEGRAR SUA
FAZENDA.\includegraphics[width=1.43478in,height=1.76408in]{media/image215.jpg}

Disponível em;
https://www.freepik.com/premium-vector/cute-cartoon-duck\_3219176.htm\#query=pato\&position=48\&from\_view=search\&track=sph.
Acesso em 19 Fev 2023.

A SÍLABA INICIAL DO NOME DESSE ANIMAL É

A) GA.

B) TO.

C) BA.

D) PA.

Habilidade Saeb Relacionar elementos sonoros das palavras com sua
representação escrita.

Habilidade BNCC EF01LP08 Relacionar elementos sonoros (sílabas, fonemas,
partes de palavras) com sua representação escrita.

\protect\hypertarget{_heading=h.19c6y18}{}{}(A) Incorreta, pois
confundiu a som silabas.

(C) Incorreta, pois confundiu o som final com o inicial.

(D) Incorreta, pois confundiu o som de pa com ba.

(A) Correta, pois a sílaba inicial da palavra pato é pa.

\subsubsection{07}\label{section-47}

LEIA A PALAVRA ESCRITA NA PLACA.

\includegraphics[width=1.83704in,height=1.58560in]{media/image216.png}

https://www.storyboardthat.com/class-storyboardboards/b5bbeb4e/unknown-story9/edit

A ÚTIMA LETRA DESSA PALAVRA É

A) P

B) N

C) E

D) A

Habilidade Saeb Ler palavras.

Habilidade BNCC EF01LP08 Relacionar elementos sonoros (sílabas, fonemas,
partes de palavras) com sua representação escrita.

(A) Incorreta, pois confundiu a última letra com a primeira.

(B) Incorreta, pois a palavra confundiu com a letra inicial da segunda
sílaba.

(C) Incorreta, pois confundiu com a letra da segunda sílaba da palavra.

(D). Correta, pois a palavra termina com a.

\subsubsection{08}\label{section-48}

OBSERVE O BRIQUEDO QUE MONIQUE COMPROU PARA SUA COLEÇÃO.

\includegraphics[width=2.45625in,height=1.54583in]{media/image217.jpg}

\textbf{Disponível
em:https://www.freepik.com/premium-vector/cute-children-bicycle-toy-white-background\_26635648.htm\#query=BICICLETA\%20BRINQUEDO\&position=29\&from\_view=search\&track=ais.Acesso
19 Fev 2023.}

O NOME DO BRINQUEDO QUE ELA COMPROU É

A) BISCOITO.

B) BICICLETA.

C) TRICICOLO.

D) CANETA.

\textbf{Habilidade Saeb} Ler palavras.

Habilidade BNCC EF12LP01 Ler palavras novas com precisão na
decodificação, no caso de palavras de uso frequente, ler globalmente,
por memorização.

\protect\hypertarget{_heading=h.3tbugp1}{}{}( A ) Incorreta, pois se
atentou apenas a primeira sílaba da palavra.

(B) Correta, pois essa palavra é o nome do brinquedo.

(C) Incorreta, pois observou apenas a segunda sílaba da palavra.

(D) Incorreta, pois só observou a última sílaba.

\subsubsection{09}\label{section-49}

OBSERVE A CENA.

\includegraphics[width=1.57569in,height=1.57569in]{media/image218.jpg}

QUAL FRASE REPRESENTA ESSA IMAGEM?

A) O MACACO ESTÁ COMENDO A BANANA.

B) O MACACO JOGOU A BANANA NO CHÃO.

C) A BANANA DO MACACO CAIU.

D) A BANANA ESTÁ VERDE.

Habilidade Saeb Ler frases

HABILIDADES BNCC

EF01LP01 Reconhecer que textos são lidos e escritos da esquerda para a
direita e de cima para baixo da página.

(A) Correta, pois o macaco está comendo uma banana como mostra a imagem.

(C) Incorreta, por acreditar que o macaco não queria mais comer a
banana.

(D) Incorreta, por acreditar que o macaco deixou a banana cair sem
querer.

(A) Incorreta, por acreditar que o macaco não comeu a banana porque
estava verde.

\subsubsection{10}\label{section-50}

\textbf{OBSERVE O ANIMAL QUE PEDRO GOSTA DE VISITAR NO
ZOOLÓGICO.}\includegraphics[width=1.36806in,height=1.47778in]{media/image219.jpg}

\textbf{ESCREVA O NOME DESSE ANIMAL.}

\textbf{\_\_\_\_\_\_\_\_\_\_\_\_\_\_\_\_\_\_\_\_\_\_\_\_\_\_\_\_\_\_\_\_\_\_\_\_\_\_\_\_\_\_\_\_\_\_\_\_\_\_\_\_\_\_\_\_\_\_\_\_\_\_\_\_}

POSSIVEIS RESPOSTAS:

girafa

irafa

grafa

girfa

\textbf{Habilidade Saeb} Escrever palavras.

Habilidade BNCC EF01LP02 Escrever, espontaneamente ou por ditado,
palavras e frases de forma alfabética -- usando letras/grafemas que
representem fonemas

\subsubsection{11}\label{section-51}

\textbf{A CASA FEIA}

O GATO FEZ UMA CASA

VEIO O RATO E FALOU:

- HUM! ... QUE CASA FEIA!

CASA BONITA TEM TELHADO,

LOGO, LOGO O GATO FEZ O TELHADO.

VEIO O PATO E FALOU:

- HUM QUE CASA FEIA!

CASA BONITA TEM VARANDA.

O GATO FEZ UMA VARANDA.

VEIO O BODE E FALOU:

-HUM... QUE CASA FEIA!

CASA BONITA É PINTADA.

E O GATO PINTOU A CASA.

MAS ELE FALOU:

-HUM... CASA BONITA TEM JARDIM.

VEIO O RATO, VEIO O PATO, VEIO O BODE DE NOVO.

-NOSSA! QUE CASA LINDA! - ELES DISSERAM.

E O GATO CONVIDOU TODOS PARA ENTRAR

Disponível
em:\href{https://www2.bauru.sp.gov.br/arquivos/arquivos_site/sec_educacao/atividades_pedagogica_distancia/1;Infantil/61;EMEI\%20Rosangela\%20Vieira\%20M.\%20de\%20Carvalho/05;PROF.\%C2\%AA\%20MARISA/Semana\%2019\%20(26\%20a\%2030\%20de\%20Julho)\%20Infantil\%20V.\%20Prof\%C2\%AA\%20Marisapdf.pdf}{\textbf{\emph{https://www2.bauru.sp.gov.br/arquivos/arquivos\_site/sec\_educacao/atividades\_pedagogica\_distancia/1;Infantil/61;EMEI\%20Rosangela\%20Vieira\%20M.\%20de\%20Carvalho/05;PROF.\%C2\%AA\%20MARISA/Semana\%2019\%20(26\%20a\%2030\%20de\%20Julho)\%20Infantil\%20V.\%20Prof\%C2\%AA\%20Marisapdf.pdf}}}

\textbf{\emph{Acesso em 20 Fev 2023.}}

QUEM DISSE AO GATO QUE CASA BONITA TEM VARADA?

A) RATO.

B) GATO.

C) PATO.

D) BODE.

\protect\hypertarget{_heading=h.28h4qwu}{}{}SAEB Localizar informações
explicitas em textos

( A) Incorreta, porá acreditar que o rato gosta de procurar migalhas nas
varada das casas

( B ) Incorreta, por acreditar que como o gato estava fazendo a casa ele
queria com varanda.

(C) Correta, pois o pato falou que casa bonita tem que ter varada.

(C) Incorreta, pois confundiu a opinião do pato com a do bode.

\subsubsection{12}\label{section-52}

\textbf{A BARATA}

A BARATA DIZ QUE TEM

SETE SAIAS DE FILÓ.

É MENTIRA DA BARATA

ELA TEM É UMA SÓ.

AH! AH! AH!

OH! OH! OH!

ELA TEM É UMA SÓ.

A BARATA DIZ QUE TEM

\protect\hypertarget{_heading=h.nmf14n}{}{}SETE SAIAS DE BALÃO.

É MENTIRA ELA NÃO TEM

NEM DINHEIRO PRO SABÃO.

AH! AH! AH!

OH! OH! OH!

Disponível
em:http://www.dominiopublico.gov.br/download/texto/me000588.pdf.Acesso
19 Fev 2023.

QUAL É O ASSUNTO DESSE TEXTO?

A) O DINHEIRO DA BARATA.

B) A MENTIRA DA BARATA.

C) O DESEJO DA BARATA.

D) A INVEJA DA BARATA.

Habilidade SAEB Inferir o assunto de um texto.

(A) Incorreta, por considerar o fato de aparecer dinheiro no texto.

( B ) Correta, pois a barata contou várias mentiras dizendo que tinha o
que na verdade não tinha.

(C) Incorreta, pois considerou que a barato tinha o desejo de ter as
coisas.

(C) Incorreta, pois considerou que a barata tinha inveja das pessoas que
tinha as coisas e ela não tinha.

\subsubsection{13 }\label{section-53}

OBSERVE O TEXTO.

\includegraphics[width=3.04633in,height=1.71374in]{media/image220.png}
ELABORADO PELA AUTORA

ESTE TEXTO SERVE PARA INDICAR

A) DEIXAR UM RECADO.

B) INDORMAR UM EVENTO.

C) ORGANIZAR AS TAREFAS.

D) ENSINAR FAZER UMA COMIDA.

Habilidade SAEB Reconhecer a finalidade de um texto.

Habilidade BNCC EF15LP01 Identificar a função social de textos que
circulam em campos da vida social dos quais participa cotidianamente (a
casa, a rua, a comunidade, a escola) e nas mídias impressa, de massa e
digital, reconhecendo para que foram produzidos, onde circulam, quem os
produziu e a quem se destinam.

(A) Correta, pois um bilhete serve para deixar um recado.

( B ) Incorreta, por considerar o fato de ter doces no texto ele estava
informando um evento.

(C) Incorreta, pois considerou que a amiga levou os doces ela estava
cumprindo uma organização de tarefas feitas por elas.

(C) Incorreta, pois considerou que o texto fala de doces e salgados ele
poderia estar ensinando uma
comida.\protect\hypertarget{_heading=h.xdbqd2s8fd2q}{}{}

\subsubsection{14}\label{section-54}

LEIA O TEXTO:

\textbf{O LEÃO E O RATINHO}

UM LEÃO, CANSADO DE TANTO CAÇAR, DORMIA ESPICHADO À SOMBRA

DE UMA BOA ÁRVORE. VIERAM UNS RATINHOS PASSEAR EM CIMA DELE

E ELE ACORDOU.TODOS CONSEGUIRAM FUGIR, MENOS UM, QUE O LEÃO

PRENDEU EMBAIXO DA PATA. TANTO O RATINHO PEDIU E IMPLOROU

QUE O LEÃO DESISTIU DE ESMAGÁ-LO E DEIXOU QUE FOSSE EMBORA.

ALGUM TEMPO DEPOIS, O LEÃO FICOU PRESO NA REDE DE

UNS CAÇADORES. NÃO CONSEGUIA SE SOLTAR, E FAZIA A FLORESTA

INTEIRA TREMER COM SEUS URROS DE RAIVA.

NISSO, APARECEU O RATINHO. COM SEUS DENTES AFIADOS,

ROEU AS CORDAS E SOLTOU O LEÃO.

Disponível
em:http://www.dominiopublico.gov.br/download/texto/me001614.pdf Acesso
20 Fev 2023.

O RATINHO AJUDOU O LEÃO PORQUE ELE

A) ERA MUITO BONZINHO E GENTIL.

B) ERA MUITO VALENTE E RAIVOSO.

C) TINHA LHE ESMAGADO UM DIA.

D) TINHA LHE AJUDADO NO PASSASO

Habilidade SAEB Inferir informações em textos verbais.

\protect\hypertarget{_heading=h.37m2jsg}{}{}( A ) Incorreta, por
considerar que quem ajuda é bom e gentil.

(B) Incorreta, pois acreditou que como o Leão é o rei da selva ele é
muito valente.

(C) Incorreta, pois considerou o medo que teve de ser esmagado pelo
Leão.

(D) Correta, pois no passado o rato tinha também lhe ajudado e o leão
estava retribuindo com ele.

\subsubsection{15}\label{section-55}

\includegraphics[width=6.13750in,height=2.02917in]{media/image221.png}

Disponível em:
\href{https://www.storyboardthat.com/portal/storyboards/b5bbeb4e/class-storyboard/unknown-story10}{\emph{https://www.storyboardthat.com/portal/storyboards/b5bbeb4e/class-storyboard/unknown-story10}}.
Acesso 20 Fev 2023.

BETO ESTÁ FELIZ NO SEGUNDO QUADRINHO PORQUE

A) TEM SEU CACHORRO.

B) GANHOU UM CACHORRO.

C) GANHOU UM CARRINHO.

D) TEM UM AMIGO PRA BRINCAR.

Habilidade Saeb~Inferir informações em textos que articulam linguagem
verbal e não verbal

\textbf{Habilidade BNCC} EF15LP14 Construir o sentido de histórias em
quadrinhos e tirinhas, relacionando imagens e palavras e interpretando
recursos gráficos (tipos de balões, de letras, onomatopeias).

(A) Incorreta, por considerar que Beto tem um cachorro.

(B) Incorreta, pois acreditou que o cachorro foi o amigo que tinha dado
para Beto.

(C) Incorreta, pois considerou o presente que Lucas o tinha dado no
terceiro quadrinho.

(D) Correta, pois no segundo quadrinho na fala de Beto ele diz sorrindo
que agora não ia brincar sozinho.

\protect\hypertarget{_heading=h.1mrcu09}{}{}

\section{\texorpdfstring{\\
}{ }}\label{section-56}

\section{SIMULADO 3}\label{simulado-3}

\subsubsection{01}\label{section-57}

\textbf{IZABEL VAI ESCOLHER UMA PLACA PARA ESCREVER O SEU NOME EM UM
JOGO.}

\textbf{A PLACA QUE ELA PRECISA ESCOLHER É }

\textbf{A)}
\includegraphics[width=2.32569in,height=1.18472in]{media/image222.jpg}

\textbf{B)}
\includegraphics[width=2.03125in,height=1.17361in]{media/image223.jpg}

\textbf{C)}
\includegraphics[width=2.20764in,height=1.07569in]{media/image224.jpg}

\includegraphics[width=1.75000in,height=0.54792in]{media/image225.jpg}

\textbf{D)}

\href{https://www.freepik.com/free-vector/flat-design-license-plate-collection_28280136.htm\#page=2\&query=PLACA\%20DE\%20N\%C3\%9AMEROS\&position=39\&from_view=search\&track=ais}{\emph{https://www.freepik.com/free-vector/flat-design-license-plate-collection\_28280136.htm\#page=2\&query=PLACA\%20DE\%20N\%C3\%9AMEROS\&position=39\&from\_view=search\&track=ais}}

\href{https://www.freepik.com/free-vector/retro-style-font_739275.htm\#query=PLACA\%20DE\%20ALFABETO\&position=12\&from_view=search\&track=ais}{\emph{https://www.freepik.com/free-vector/retro-style-font\_739275.htm\#query=PLACA\%20DE\%20ALFABETO\&position=12\&from\_view=search\&track=ais}}

\href{https://www.freepik.com/free-vector/traffic-signs-set_724943.htm\#query=PLACA\%20DE\%20TRANSITO\&position=4\&from_view=search\&track=ais}{\emph{https://www.freepik.com/free-vector/traffic-signs-set\_724943.htm\#query=PLACA\%20DE\%20TRANSITO\&position=4\&from\_view=search\&track=ais}}

\href{https://www.freepik.com/premium-vector/set-numbers-paper-style-with-realistic-shadow-green-background_8717932.htm\#query=NUMEROS\&position=47\&from_view=search\&track=sph}{\emph{https://www.freepik.com/premium-vector/set-numbers-paper-style-with-realistic-shadow-green-background\_8717932.htm\#query=NUMEROS\&position=47\&from\_view=search\&track=sph}}

Habilidade Saeb Relacionar elementos sonoros das palavras com sua
representação escrita.

Habilidade EF01LP04 Distinguir as letras do alfabeto de outros sinais
gráficos.

(A) Incorreta, pois essa placa apresenta números a poucas letras.

(B ) Correta, pois para escrever palavras usamos as letras do alfabeto.

(C) Incorreta, pois a placa só aparece placa de trânsito.

(D) Incorreta, pois a placa só aparece
números.\protect\hypertarget{_heading=h.vsxh2fkjtnnn}{}{}

\subsubsection{02}\label{section-58}

VEJA O NOME QUE MIGUEL ESCOLHEU PARA SEU CACHORRO.

\includegraphics[width=2.53194in,height=1.44514in]{media/image226.png}

\href{https://www.freepik.com/free-vector/signage-beside-doghouse_3136850.htm\#query=CACHORRO\%20SEGURANDO\%20PLACA\&position=16\&from_view=search\&tra}{\emph{https://www.freepik.com/free-vector/signage-beside-doghouse\_3136850.htm\#query=CACHORRO\%20SEGURANDO\%20PLACA\&position=16\&from\_view=search\&tra}}

QUAL PALAVRA TEM A MESMA QUANTIDADE DE VOGAIS DO NOME DO CACHORRO DE
MIGUEL?

(A) FADA.

(B) CAVALO.

(C) TELEVISÃO.

(D) GELATINA.

Habilidade Saeb Relacionar elementos sonoros das palavras com sua
representação escrita.

Habilidade BNCC EF01LP07 Identificar fonemas e sua representação por
letras.

(A) Incorreta, pois a palavra tem apenas duas vogais.

(B) Correta, pois a palavra possui três vogais, assim como o nome do
cachorro.

(C) Incorreta, pois a palavra possui cinco vogais.

(D) Incorreta, pois a palavra tem quatro vogais.

\subsubsection{03}\label{section-59}

\textbf{NÍVEL MÉDIO}

OBSERVE O DESENHO QUE BIANCA COLORIU.

\includegraphics[width=2.45833in,height=1.69432in]{media/image227.png}

Disponivel
em:https://www.freepik.com/free-vector/butterfly-collection\_3902532.htm\#query=BORBOLETA\&position=21\&from\_view=search\&track=sph.
Acesso em 11 de Fev 2023.

A PALAVRA QUE TERMINA COM A MESMA SÍLABA DO NOME DO DESENHO DE BIANCA É

(A) BATATA.

(B) TAPETE.

(C) BONECA.

( D) GILETE.

Habilidade Saeb Relacionar elementos sonoros das palavras com sua
representação escrita.

Habilidade BNCC EF01LP09 Comparar palavras, identificando semelhanças e
diferenças entre sons de sílabas iniciais, mediais e finais.

(A) Correta, pois a palavra borboleta termina com ta o mesmo som final
da palavra batata.

(B) Incorreta, pois confundiu a som final com o inicial borboleta e
tapete.

(C) Incorreta, pois confundiu o som final com inicial da questão.

(D) Incorreta, pois acreditaram que poderiam ser qualquer som igual
presentes nas duas palavras.

\subsubsection{04}\label{section-60}

VEJA A PALAVRA NOVA QUE JOANA ESCREVEU NO SEU CADERNO.

\includegraphics[width=2.24931in,height=2.80435in]{media/image230.png}

AS LETRAS QUE FORMAM O SOM FINAL DA PALAVRA QUE ELA ESCREVEU SÃO

(A) E + E.

(B) L+ E

(C) F + A

(D) T+ E

HABILIDADE MATRIZ SAEB Relacionar elementos sonoros das palavras com sua
representação escrita

HABILIDADES BNCC EF01LP05 Reconhecer o sistema de escrita alfabética
como representação dos sons da fala.

(A) Incorreta, pois confundiu com a sílaba inicial achado que ela não
podia ser formada por uma única letra e acrescentou mais uma letra E.

(B) Incorreta, pois confundiu os sons das letras.

(C) Incorreta, pois confundiu a posição da sílaba.

(D) Correta, pois a palavra termina com TE.

\subsubsection{05}\label{section-61}

VEJA A PALAVRA QUE UMA MENINA ESCREVEU NA LOUSA.

\includegraphics[width=2.13044in,height=2.33234in]{media/image231.jpg}

\textbf{Disponível em
\href{https://www.freepik.com/free-vector/realistic-green-black-chalkboard-withwoodenframe_12569028.htm\#query=LOUSA\&position=29\&from_view=search\&track=sph}{\emph{https://www.freepik.com/free-vector/realistic-green-black-chalkboard-withwoodenframe\_12569028.htm\#query=LOUSA\&position=29\&from\_view=search\&track=sph}}
acesso 20 Fev 2023.}

A SÍLABA DO MEIO DESSA PALAVRA É

\textbf{A) CA.}

\textbf{B) V A.}

\textbf{C) L O}

\textbf{D) O L.}

HABILIDADE MATRIZ SAEB

Relacionar elementos sonoros das palavras com sua representação escrita

EF01LP08 Relacionar elementos sonoros (sílabas, fonemas, partes de
palavras) com sua representação escrita.

A) Incorreta, pois confundiu a sílaba inicial com a do meio.

(B) Correta, pois a palavra termina com V + A.

(C) Incorreta, pois confundiu a posição da sílaba.

(D) Incorreta, pois confundiu a posição os sons das letras

\subsubsection{06}\label{section-62}

LEIA.

CARANGUEJO PEIXE É.

CARANGUEJO NÃO É PEIXE,

CARANGUEJO PEIXE É,

CARANGUEJO SÓ É PEIXE

NA VAZANTE DA MARÉ.

Disponível em
\href{http://www.dominiopublico.gov.br/download/texto/me000588.pdf}{\emph{http://www.dominiopublico.gov.br/download/texto/me000588.pdf}}.
Acesso 21 Fev 2023.

A ÚLTIMA PALAVRA DO TEXTO É

A) CARANGUEJO.

B) VAZANTE.

C) PEIXE.

D) MARÉ.

HABILIDADE MATRIZ SAEB Ler palavras.

EF01LP01 Reconhecer que textos são lidos e escritos da esquerda para a
direita e de cima para baixo da página.

A) Incorreta, pois confundiu a primeira palavra com a última.

(B) Incorreta, pois acreditou que essa palavra fosse junto com da maré.

(C) Incorreta, pois confundiu com a alguns versos que termina com essa
palavra.

(D) Correta, pois essa é a última palavra do texto.

\subsubsection{07}\label{section-63}

\textbf{LEIA A PALAVRA.}

\textbf{BOLA}

\textbf{A PALAVRA QUE NÃO APARECE NENHUMA SÍLABA IGUAL A PALAVRA QUE
VOCÊ LEU É}

\textbf{A) BOLO.}

\textbf{B) LATA.}

\textbf{C) LOBO.}

\textbf{D) FADA.}

Habilidade SAEB Ler palavras

Habilidade BNCC EF01LP13 Comparar palavras, identificando semelhanças e
diferenças entre sons de sílabas iniciais, mediais e finais.

(A) Incorreta, pois confundiu a som silabas.

(C) Incorreta, pois confundiu o som final com o inicial.

(D) Incorreta, pois confundiu o som de pa com ba.

(A) Correta, pois a sílaba inicial da palavra pato é
pa.\protect\hypertarget{_heading=h.7lb9brf4wytg}{}{}

\subsubsection{08 }\label{section-64}

VEJA O PRESENTE QUE TALIA DEU PARA SUA AMIGA NO SEU
ANIVERSÁRIO.\includegraphics[width=1.45972in,height=1.93403in]{media/image232.jpg}

https://www.freepik.com/free-vector/summer-clothes-set\_4559041.htm\#query=vestido\&position=0\&from\_view=search\&track=sph

ESCREVA O NOME DO PRESENTE QUE ELA GANHOU.

\_\_\_\_\_\_\_\_\_\_\_\_\_\_\_\_\_\_\_\_\_\_\_\_\_\_\_\_\_\_\_\_\_\_\_\_\_\_\_\_\_\_\_\_\_\_\_\_\_\_\_\_\_\_\_\_\_\_\_\_\_\_

POSSIVEIS RESPOSTAS:

Vestido.

Vetido.

Etido.

Vetdo.

HABILIDADE MATRIZ SAEB Escrever palavras.

EF01LP02 Escrever, espontaneamente ou por ditado, palavras e frases de
forma alfabética -- usando letras/grafemas que representem fonemas.

\subsubsection{09 }\label{section-65}

OBSERVE ESSA IMAGEM.

\includegraphics[width=3.84043in,height=3.16853in]{media/image233.jpg}

Disponível em:
\href{https://www.freepik.com/premium-vector/cute-boy-park-with-kite_6796136.htm?query=MENINO\%20PIPA\#from_view=detail_alsolike}{\emph{https://www.freepik.com/premium-vector/cute-boy-park-with-kite\_6796136.htm?query=MENINO\%20PIPA\#from\_view=detail\_alsolike}}.
Acesso 20 Fev 2023.

A FRASE QUE REPRESENTA A IMAGEM É

A) O MEMINO SOBE NA ÁRVORE.

B) O MENINO BRINCA COM A PIPA.

C) O MENINO TOMA BANHO NO RIO.

D) O MENINO SE ESCODE ATRÁS DO ARBUSTO.

(A) Incorreta, pois confundiu a som silabas.

(C) Incorreta, pois confundiu o som final com o inicial.

(D) Incorreta, pois confundiu o som de pa com ba.

(A) Correta, pois a sílaba inicial da palavra pato é
pa.\protect\hypertarget{_heading=h.qmyrjsh4y5fg}{}{}

\subsubsection{10}\label{section-66}

VEJA ANIMAL PREFERIDO DE BRUNO.

\includegraphics[width=2.82210in,height=1.90975in]{media/image234.jpg}

Disponível
em:https://www.freepik.com/premium-vector/cartoon-brown-horse-running-white-background\_18654839.htm?query=CAVALO\#from\_view=detail\_alsolike.
Acesso em 20 Fev 2023.

O NOME CORRETO DESSE ANIMAL É

A) VALO

B) KAVA

C) CAVALO.

D) VALOCA.

Ler palavras

HABILIDADES BNCC EF12LP01 Ler palavras novas com precisão na
decodificação, no caso de palavras de uso frequente, ler globalmente,
por memorização.

(A) Incorreta, pois leu só as últimas sílabas.

(C) Incorreta, pois confundiu o som da letra c.

(D). Correta, pois essa é a palavra certa.

(A) Incorreta, pois trocou a ordem das
sílabas.\protect\hypertarget{_heading=h.oxa1c0ljwovb}{}{}

\subsubsection{11}\label{section-67}

\textbf{OBSERVE O MÓVEL QUE CARLA COMPROU PARA SEU QUARTO.}

\includegraphics[width=1.84783in,height=1.84783in]{media/image235.jpg}

\textbf{Disponível em:
\href{https://www.freepik.com/premium-vector/illustrator-bed-isolated_2871016.htm\#page=2\&query=CAMA\&position=13\&from_view=search\&track=sph}{\emph{https://www.freepik.com/premium-vector/illustrator-bed-isolated\_2871016.htm\#page=2\&query=CAMA\&position=13\&from\_view=search\&track=sph}}}

\textbf{A PALAVRA QUE TERMINA COM O MESMO SOM DO NOME DO MÓLVEL DE CARLA
É}

\textbf{A) CANELA.}

\textbf{B) BONECA.}

\textbf{C) SACADA.}

\textbf{D) MACACO.}

Ler palavras.

EF01LP13 Comparar palavras, identificando semelhanças e diferenças entre
sons de sílabas iniciais, mediais e finais.

(A) Incorreta, pois confundiu a última sílaba com a primeira.

(C) Correta, pois essa é a palavra certa

(D)Incorreta, pois considerou a sílaba do meio.

(A) Incorreta, pois considerou as sílabas escondidas dentro da palavra.

\subsubsection{12}\label{section-68}

\textbf{LEIA}

SAMBA CRIOLA QUE VEIO DA BAHIA

PEGA ESTA CRIANÇA E JOGA NA

BACIA.

A BACIA É DE OURO, AREADA COM

SABÃO,

DEPOIS DE TUDO PRONTO, ENXUGA

NO ROUPÃO.

O ROUPÃO É DE SEDA,

CAMINHA DE FILÓ

QUEM NÃO PEGAR SEU PAR

FICARÁ PARA A VOVÓ.

A BÊNÇÃO VOVÓ, A BÊNÇÃO VOVÓ!

Disponível em
\href{http://www.dominiopublico.gov.br/download/texto/me000588.pdf}{\emph{http://www.dominiopublico.gov.br/download/texto/me000588.pdf}}.
Acesso 21 Fev 2023.

O QUE JOGARAM NA BACIA?

A) O OURO.

B) O ROUPÃO.

C) A CRIANÇA.

D) A CAMINHA.

HABILIDADE SAEB Localizar informações explícitas em textos.

HABILIDADE BNCC EF15LP03 Localizar informações explícitas em textos.

(A) Incorreta, por acreditar que o ouro seria guardado na bacia.

(C) Correta, pois no texto fala pega essa criança e joga na bacia.

(D)Incorreta, pois acreditou que o roupão ia ser lavado na bacia.

(A) Incorreta, pois acreditou que a caminha estava suja e ia ser lavada
na bacia.

\subsubsection{13 }\label{section-69}

LEIA A LISTA.

\textbf{1- ARROZ}

\textbf{2- FEIJÃO}

\textbf{3- MACARRÃO}

\textbf{4- ÓLEO}

\textbf{5- BISCOITO.}

\textbf{6- EXTRATO}

\textbf{7- FARINHA}

\textbf{8- LEITE}

ESSE TEXTO SERVE PARA

A) ORGANIZAR TAREFAS.

B) ENSINAR FAZER COMIDA.

C) INDICAR OS PRODUTOS.

D) INFORMAR UM EVENTO.

Habilidade SAEB Reconhecer a finalidade de um texto.

Habilidade BNCC EF15LP01 Identificar a função social de textos que
circulam em campos da vida social dos quais participa cotidianamente (a
casa, a rua, a comunidade, a escola) e nas mídias impressa, de massa e
digital, reconhecendo para que foram produzidos, onde circulam, quem os
produziu e a quem se destinam

(A) Incorreta, por considerar que o texto está bem organizado.

(C) Incorreta, por considerar que o texto traz muitos nomes de comida

(D) Correta, pois a lista indica os itens de uma compra.

(A) Incorreta, pois acreditou que o texto informava um evento de comida.

\subsubsection{14}\label{section-70}

LEIA O TEXTO.

\textbf{TEREZINHA DE JESUS}

TEREZINHA DE JESUS

DE UMA QUEDA FOI AO CHÃO

ACUDIRAM TRÊS CAVALHEIROS

TODOS TRÊS, CHAPÉU NA MÃO.

O PRIMEIRO, FOI SEU PAI

O SEGUNDO, SEU IRMÃO

O TERCEIRO FOI AQUELE

A QUE TERESA DEU A MÃO.

Disponível em
\href{http://www.dominiopublico.gov.br/download/texto/me000588.pdf}{\emph{http://www.dominiopublico.gov.br/download/texto/me000588.pdf}}.
Acesso 21 Fev 2023.

\textbf{DE QUE O TEXTO FALA?}

A) DO PAI DA TEREZINHA.

B) DA MÃO DA TEREZINHA.

C) DO IRMÃO DA TEREZINHA.

D) DA QUEDA DE TEREZINHA.

Habilidade SAEB~Inferir o assunto de um texto.

(A) Incorreta, por considerar que ele foi o primeiro a lhe acudi.

( B) Incorreta, por acreditar que a Terezinha estava estendendo a mão
para alguém acudi.

(C) Incorreta, por considerar que seu irmão tinha a obrigação de lhe
ajudar.

(D)correta, pois no texto fala da que ela foi ao chão de uma queda.

\subsubsection{15 }\label{section-71}

\textbf{A RAPOSA E O CORVO}

O CORVO CONSEGUIU ARRANJAR UM PEDAÇO DE QUEIJO, EM ALGUM

LUGAR. SAIU VOANDO, COM O QUEIJO NO BICO, ATÉ POUSAR NUMA

ÁRVORE.

QUANDO VIU O QUEIJO, A RAPOSA RESOLVEU SE APODERAR

DELE. CHEGOU AO PÉ DA ÁRVORE E COMEÇOU A BAJULAR O CORVO:

--- Ó SENHOR CORVO! O SENHOR É CERTAMENTE O MAIS BELO

DOS ANIMAIS! SE SOUBER CANTAR TÃO BEM QUANTO A SUA PLUMAGEM

É LINDA, NÃO HAVERÁ AVE QUE POSSA SE COMPARAR AO SENHOR.

Disponível em:
\href{http://www.dominiopublico.gov.br/download/texto/me001614.pdf}{\emph{http://www.dominiopublico.gov.br/download/texto/me001614.pdf}}.
Acesso 20d Fev 2023.

A RAPOSA QUERIA O CORVO CANTASSE PARA

A) ALEGRAR A FLORESTA.

B) MOSTRAR SEU TALENTO.

C) GANHAR ELOGIOS.

D) CAIR O QUEIJO.

Habilidade SAEB~Inferir informações em textos verbais.

(A) Incorreta, por acreditar que a floresta estava triste.

( B) Incorreta, por acreditar que a raposa ia lhe elogiar juntos com os
animais.

(C) Incorreta, por acreditar que a raposa estava feliz em ver o corvo

(D) correta, pois a raposa queria que o queijo caísse para ela
pegar.\protect\hypertarget{_heading=h.jen2jvep4ux3}{}{}

\subsubsection{16}\label{section-72}

LEIA A TIRINHA.

\includegraphics[width=1.88542in,height=1.88542in]{media/image236.png}
\includegraphics[width=1.92708in,height=1.92708in]{media/image237.png}
\includegraphics[width=1.90625in,height=1.90625in]{media/image238.png}

Disponível em https://lizeseusamigos.org.br/tirinhas/tirinhas-tres.
Acesso 20 Fev 2023.

A BOLA ACERTOU JUNINHO NO SEGUNDO QUEDRINHO POR QUE ELE

A) É MUITO DESTRAIDO.

B) TENTOU PEGAR A BOLA.

C) ESTAVA SENDO O GOLEIRO.

D) TEM DIFICULDADE DE ENXERGAR.

Habilidade Saeb~Inferir informações em textos que articulam linguagem
verbal e não verbal

\textbf{Habilidade BNCC} EF15LP14 Construir o sentido de histórias em
quadrinhos e tirinhas, relacionando imagens e palavras e interpretando
recursos gráficos (tipos de balões, de letras, onomatopeias).

(A) Incorreta, por acreditar que o menino não estava atento.

(B) Incorreta, por acreditar que ele ia segurar a bola.

(C) Incorreta, por acreditar que ele estava brincado de futebole ele
estava sendo o goleiro.

(D) Correta, pois no terceiro quadrinho afirma que ele tem pouca visão.

\section{SIMULADO 4}\label{simulado-4}

\subsubsection{01}\label{section-73}

BIA QUER MOSTRAR O NOME QUE ESCOLHEU PARA SUA BONECA.

A PLACA QUE ELA VAI ESCOLHER FORMAR O NOME É

A)
\includegraphics[width=0.48889in,height=0.48889in]{media/image239.jpg}

\includegraphics[width=0.60625in,height=0.60625in]{media/image240.jpg}

B)

C)
\includegraphics[width=0.54236in,height=0.54236in]{media/image241.jpg}

\includegraphics[width=0.57431in,height=0.57431in]{media/image242.jpg}

D)

\href{https://www.freepik.com/free-vector/retro-red-christmas-alphabetical-letters_10806055.htm\#page=4\&query=ALFABETO\%20M\%C3\%93LVEL\&position=30\&from_view=search\&track=ais}{\emph{https://www.freepik.com/free-vector/retro-red-christmas-alphabetical-letters\_10806055.htm\#page=4\&query=ALFABETO\%20M\%C3\%93LVEL\&position=30\&from\_view=search\&track=ais}}

\href{https://www.freepik.com/free-vector/colorful-number-collection-with-flat-design_2303697.htm\#page=2\&query=N\%C3\%9AMEROS\%20M\%C3\%93LVEL\&position=21\&from_view=search\&track=ais}{\emph{https://www.freepik.com/free-vector/colorful-number-collection-with-flat-design\_2303697.htm\#page=2\&query=N\%C3\%9AMEROS\%20M\%C3\%93LVEL\&position=21\&from\_view=search\&track=ais}}

\href{https://www.freepik.com/free-vector/flat-emoticon-reaction-collectio_4362971.htm?query=SINAIS\%20DE\%20RANSITO\#from_view=detail_alsolike}{\emph{https://www.freepik.com/free-vector/flat-emoticon-reaction-collectio\_4362971.htm?query=SINAIS\%20DE\%20RANSITO\#from\_view=detail\_alsolike}}

\href{https://www.freepik.com/free-vector/flat-design-arrow-collection_12689920.htm\#query=SETA\&position=27\&from_view=search\&track=sph}{\emph{https://www.freepik.com/free-vector/flat-design-arrow-collection\_12689920.htm\#query=SETA\&position=27\&from\_view=search\&track=sph}}

HABILIDADE MATRIZ SAEB

Relacionar elementos sonoros das palavras com sua representação escrita

HABILIDADES BNCC

EF01LP04 Distinguir as letras do alfabeto de outros sinais gráficos.

(A) Incorreta, pois essa placa aparece números.

(B ) Incorreta, pois a placa só aparece setas.

(C) Correta, pois para formar palavras usamos as letras do alfabeto.

(D) Incorreta, pois a placa só aparece
emojis.\protect\hypertarget{_heading=h.kl8qv6rrx63u}{}{}

\subsubsection{02}\label{section-74}

VEJA A PALAVRA QUE MARIA ESCREVEU.

SALADA

Elaborada pela autora.

AS LETRAS QUE FORMAM O SOM FINAL DA PALAVRA SÃO

A) D + A.

B) L + A.

C) S + A.

D) A + D.

HABILIDADE MATRIZ SAEB

Relacionar elementos sonoros das palavras com sua representação escrita

HABILIDADES BNCC

EF01LP05 Reconhecer o sistema de escrita alfabética como representação
dos sons da fala.

( A ) Correta, pois o som final da palavra salada é DA.

(B) Incorreta, pois confundiu com o som do meio da palavra.

(C) Incorreta, pois confundiu com o som inicial.

(D) Incorreta, pois confundiu a posição do som das
letras.\protect\hypertarget{_heading=h.ebpkik53igmj}{}{}

\subsubsection{03}\label{section-75}

VEJA A PALAVRA QUE DINO CONSEGUIU
ESCREVER.\includegraphics[width=2.21250in,height=1.04375in]{media/image243.png}

Elaborada pela autora.

A PALAVRA QUE TEM A MESMA QUANTIDADE DE SOM DA PALAVRA QUE DINO ESCREVEU
É

A) BICICLETA.

B) PATINETE.

C) PATINS.

D) BONÉ.

HABILIDADE MATRIZ SAEB

Relacionar elementos sonoros das palavras com sua representação escrita

EF01LP07 Identificar fonemas e sua representação por letras.

(A) Correta, pois palavra está representada por nove sons, assim como na
palavra

borboleta.

( B ) Incorreta, pois a palavra possui oito sons.

(C) Incorreta, pois a palavra está representada por seis sons.

(D) Incorreta, pois a palavra possui quatro sons.

\subsubsection{04}\label{section-76}

VEJA A PALAVRA QUE ISA ENCONTROU DENTRO DO SEU LIVRO.

Elaborado pela
autora.\includegraphics[width=1.29722in,height=0.57917in]{media/image244.png}

A PALAVRA QUE TEM A MESMA SÍLABA INICIAL DA PALAVRA QUE ISA ENCONTROU É

A) NEVE.

B) JACARÉ.

C) PANELA.

D) LARANJA.

HABILIDADE MATRIZ SAEB

Relacionar elementos sonoros das palavras com sua representação escrita

EF01LP08 Relacionar elementos sonoros (sílabas, fonemas, partes de
palavras) com sua representação escrita.

( A ) Incorreta, pois considerou a segunda sílaba da palavra.

(B) Correta, pois a palavra começa com já, assim como o nome da janela.

(C) Incorreta, pois considerou a última sílaba.

(D) Incorreta, pois confundiu a posição da
sílaba.\protect\hypertarget{_heading=h.33l0sngjt3hj}{}{}

\subsubsection{05}\label{section-77}

VEJA O BRINQUEDO QUE ENZO
GANHOU.\includegraphics[width=1.73681in,height=1.03125in]{media/image245.jpg}

https://www.freepik.com/vectors/motorcycle\#referrer=detail\&resource=11672041

A PALAVRA QUE COM TODOS OS SONS DIFERENTES DO NOME DO BRINQUEDO DE ENZO
É

A) FADA.

B) GATO.

C) MOLA.

D) RATO.

HABILIDADE MATRIZ SAEB

Relacionar elementos sonoros das palavras com sua representação escrita

EF01LP09 Comparar palavras, identificando semelhanças e diferenças entre
sons de sílabas iniciais, mediais e finais.

A) Correta, pois a palavra não apresenta nenhum som igual a palavra
moto.

(B) Incorreta, pois a palavra apresenta os sons t+o igual de moto.

(C) Incorreta, pois apresenta os sons m+o igual de moto.

(D) Incorreta, pois apresenta os sons t+o igual a sílaba final da
palavra.

\subsubsection{06}\label{section-78}

LEIA A ADIVINHA.

COM DEZ PATAS VAI DE LADO,

CONSTELAÇÃO TEM SEU NOME,

NÃO TEM PESCOÇO E É CAÇADO

PORQUE É GOSTOSO E SE COME.

Disponível em
\href{http://www.dominiopublico.gov.br/download/texto/me000588.pdf}{\emph{http://www.dominiopublico.gov.br/download/texto/me000588.pdf}}.
Acesso 21 Fev 2023.

A PRIMEIRA PALAVRA DESSE TEXTO É

A) COM.

B) LADO.

C) COME.

D) POR QUE.

HABILIDADE MATRIZ SAEB

Ler palavras.

EF01LP01 ~Reconhecer que textos são lidos e escritos da esquerda para a
direita e de cima para baixo da página.

A) Correta, pois essa é a primeira palavra do texto.

(B) Incorreta, pois considerou a última palavra do primeiro verso.

(C) Incorreta, pois confundiu a primeira palavra com a última.

(D) Incorreta, pois considerou a primeira palavra do último
verso.\protect\hypertarget{_heading=h.bww9mmn8u5sl}{}{}

\subsubsection{07}\label{section-79}

OBSERVE ANIMAL DE ESTIMAÇÃO QUE LETÍCIA GANHOU.

\includegraphics[width=1.75903in,height=1.18611in]{media/image246.jpg}

O NOME CORRETO DESSE ANIMAL É

A) LHOECO

B) COOLHO.

C) EOCLHO.

D) COELHO.

HABILIDADE MATRIZ SAEB

Ler palavras.

HABILIDADES BNCC

EF12LP01 Ler palavras novas com precisão na decodificação, no caso de
palavras de uso frequente, ler globalmente, por memorização.

(A) Incorreta, pois não considerou a ordem das sílabas.

(B) Incorreta, pois repetiu a letra o no lugar da e.

(C) Incorreta, pois não considerou a ordem das letras.

(D) Correta, pois essa é a palavra
certa.\protect\hypertarget{_heading=h.9bc4fu3nu5it}{}{}

\subsubsection{08}\label{section-80}

LEIA A PALVRA QUE CATARINA ESCREVEU.

\includegraphics[width=2.25361in,height=0.86599in]{media/image247.png}
Elaborado pela autora.

A PALVRA QUE COMEÇA COM OS SOM IGUAL A PALAVRA QUE ELA ESCREVEU É

A) NABO.

B) CANECA.

C) COCADA.

D) MACARRÃO.

HABILIDADE MATRIZ SAEB

Ler palavras.

EF01LP13 Comparar palavras, identificando semelhanças e diferenças entre
sons de sílabas iniciais, mediais e finais.

(A) Incorreta, pois confundiu som de n com o do m.

(B) Incorreta, pois confundiu o som do meio.

(C) Incorreta, pois confundiu o som final.

(D) Correta, pois o som inicial da palavra macaco é m igual a de
macarrão.\protect\hypertarget{_heading=h.z3c8l2ojus6u}{}{}

\subsubsection{09}\label{section-81}

OBSERVE O PRESENTE QUE SARA GANHOU DA SUA TIA.

ESCREVA O NOME DO PRESENTE QUE ELA GANHOU.

\_\_\_\_\_\_\_\_\_\_\_\_\_\_\_\_\_\_\_\_\_\_\_\_\_\_\_\_\_\_\_\_\_\_\_\_\_\_\_\_\_\_\_\_\_\_\_\_\_\_\_\_\_\_\_\_\_

POSSIVEIS RESPOSTAS:

telefone.

fonete.

lefone.

tefone.

HABILIDADE MATRIZ SAEB Escrever palavras.

HABILIDADES BNCC EF01LP02 Escrever, espontaneamente ou por ditado,
palavras e frases de forma alfabética -- usando letras/grafemas que
representem fonemas.

\subsubsection{10}\label{section-82}

LEIA.

\includegraphics[width=2.46358in,height=0.98459in]{media/image249.png}

QUAL DESENHO REPRESENTA ESSA FRASE?

A) B)
\includegraphics[width=1.85903in,height=1.21806in]{media/image250.jpg}\includegraphics[width=1.71806in,height=1.21875in]{media/image251.jpg}

C) D)
\includegraphics[width=1.96319in,height=1.09028in]{media/image252.jpg}\includegraphics[width=1.78681in,height=0.96806in]{media/image253.jpg}

\href{https://www.freepik.com/premium-vector/river-forest-scene-with-wild-animals_23724751.htm\#query=PATO\%20NA\%20LAGOA\&position=47\&from_view=search\&track=ais}{\emph{https://www.freepik.com/premium-vector/river-forest-scene-with-wild-animals\_23724751.htm\#query=PATO\%20NA\%20LAGOA\&position=47\&from\_view=search\&track=ais}}

\href{https://www.freepik.com/free-vector/scene-with-ducks-bird-by-pond_19747783.htm\#query=PATO\%20NA\%20LAGOA\&position=22\&from_view=search\&track=ais}{\emph{https://www.freepik.com/free-vector/scene-with-ducks-bird-by-pond\_19747783.htm\#query=PATO\%20NA\%20LAGOA\&position=22\&from\_view=search\&track=ais}}

Ler frases

HABILIDADES BNCC

EF01LP02 Escrever, espontaneamente ou por ditado, palavras e frases de
forma alfabética -- usando letras/grafemas que representem fonemas.

( A ) incorreta, pois quem está no logo é a tartaruga.

(B) incorreta, pois quem está no logo é o sapo.

(C) correta, pois os patos estão nadando na lagoa.

(D) Incorreta, pois os patos estão o lago mas não estão nadando.

\subsubsection{11 }\label{section-83}

\textbf{POMBINHA}

POMBINHA QUANDO TU FORES

ME ESCREVA PELO CAMINHO

SE NÃO ACHARES PAPEL

NAS ASAS DE UM PASSARINHO.

DO BICO FAZ UM TINTEIRO

DA LÍNGUA PENA DOURADA

DOS DENTES LETRA MIÚDA

DOS OLHOS CARTA FECHADA.

A POMBINHA VOOU, VOOU {[}BIS{]}

FOI-SE EMBORA E ME DEIXOU

O QUE A POMBINHA VAI FAZER DOS OLHOS?

A) TINTEIRO.

B) LETRA MIÚDA.

C) CARTA FECHADA.

D) PENA DOURADA.

HABILIDADE SAEB Localizar informações explícitas em textos.

HABILIDADE BNCC EF15LP03 Localizar informações explícitas em textos.

(A) Incorreta, pois considerou que precisa dos olhos o tinteiro para
pintar.

( B )Incorreta, pois considerou que como as letra eram pequenas os olhos
podiam ver.

(C) Correta, pois no texto está claro que dos olhos era para fazer carta
fechada.

(C) Incorreta, pois por considerar que existem várias cores de
olhos.\protect\hypertarget{_heading=h.xyn2rjyn0l4h}{}{}

\subsubsection{12}\label{section-84}

LEIA O TRECHO DA HISTÓRIA.

CHAPEUZINHO VERMELHO

ERA UMA VEZ, NUMA PEQUENA CIDADE ÀS MARGENS DA FLORESTA,

UMA MENINA DE OLHOS NEGROS E LOUROS CABELOS CACHEADOS, TÃO.

GRACIOSA QUANTO VALIOSA.

UM DIA, COM UM RETALHO DE TECIDO VERMELHO, SUA MÃE

COSTUROU PARA ELA UMA CURTA CAPA COM CAPUZ; FICOU UMA

BELEZINHA, COMBINANDO MUITO BEM COM OS CABELOS LOUROS E

OS OLHOS NEGROS DA MENINA.\\
Disponível
em:http://www.dominiopublico.gov.br/download/texto/me001614.pdf. acesso
em 21 Fev 2023.

ESSE TEXTO E DESTINADO PARA

A) OS JOVENS.

B) AS CRIANÇAS.

C) OS ADULTOS.

D) AS FAMÍLIAS.

Habilidade SAEB Reconhecer a finalidade de um texto.

Habilidade BNCC EF15LP01 Identificar a função social de textos que
circulam em campos da vida social dos quais participa cotidianamente (a
casa, a rua, a comunidade, a escola) e nas mídias impressa, de massa e
digital, reconhecendo para que foram produzidos, onde circulam, quem os
produziu e a quem se destinam.

(A) Incorreta, pois considerou que os jovens gostam de vaidade como a
menina.

(B) Correta, pois geralmente as crianças que gosta de ouvir contos.

(C) Incorreta, pois considerou que os adultos contam histórias para as
crianças.

(D) Incorreta, pois considerou que alguma família se reúne para contar
história.

\subsubsection{13}\label{section-85}

LEIA O TEXTO.

NO REINO DAS LETRAS FELIZES

NUM LUGAR MUITO DISTANTE, EXISTIA UM REINO SILENCIOSO, HABITADO APENAS
POR LETRAS, ELAS ERAM MUITO DESUNIDAS. VIVIA CADA UMA PARA SI, E NUNCA
SE REUNIAM PARA FORMAR UMA PALAVRA SEQUER.\textbf{~}A RAINHA ENTÃO
RESOLVEU ACABAR COM AQUELE SILÊNCIO, AQUELE SILÊNCIO TODO, CHAMOU SEUS
CONSELHEIROS: BETA, GAMA E ÔMEGA E ORDENOU:

~-- QUERO QUE ORGANIZEM UM GRANDE BAILE E QUE CONVIDEM TODAS AS LETRAS
DO REINO E TAMBÉM OS DEMAIS REINOS.

A RAINHA TAMBÉM DISSE:

-- QUERO QUE AS LETRAS DO REINO USEM A CRIATIVIDADE E FAÇAM ALGUMAS
APRESENTAÇÕES A FIM DE MARCAR PARA SEMPRE A HISTÓRIA DO REINO.

Disponível em:
\href{http://www.dominiopublico.gov.br/pesquisa/DetalheObraDownload.do?select_action=\&co_obra=53489\&co_midia=}{\emph{http://www.dominiopublico.gov.br/pesquisa/DetalheObraDownload.do?select\_action=\&co\_obra=53489\&co\_midia=}}
Acesso 21 Fev 2023.

DE QUE FALA O TEXTO?

A) O SILÊNCIO DAS LETRAS.

B) A TRISTEZA DAS LETRAS.

C) A APRESENTAÇÃO DAS LETRAS.

D). AS LETRAS QUE ERAM FELIZES.

Habilidade SAEB~Inferir o assunto de um texto.

A) Correta, pois a rainha promoveu um baile para fazer as letras se
apresentarem

(B) Incorreta, por considerar que as letras estavam em silêncio porque
estavam tristes.

( C ) Incorreta, por considerar que as letras que a rainha iria promover
esse evento.

(D) Incorreta, pois confundiu com o título do texto.

\subsubsection{14}\label{section-86}

LEIA.

O GALO E A RAPOSA

O GALO E AS GALINHAS VIRAM QUE LÁ LONGE VINHA UMA RAPOSA.

EMPOLEIRARAM-SE NA ÁRVORE MAIS PRÓXIMA, PARA ESCAPAR DA INIMIGA.

COM SUA ESPERTEZA, A RAPOSA CHEGOU PERTO DA ÁRVORE E

SE DIRIGIU A ELES:

--- ORA, MEUS AMIGOS, PODEM DESCER DAÍ. NÃO SABEM QUE

FOI DECRETADA A PAZ ENTRE OS ANIMAIS? DESÇAM E VAMOS FESTEJAR ESSE

DIA TÃO FELIZ!

\href{http://www.dominiopublico.gov.br/download/texto/me001614.pdf}{\emph{http://www.dominiopublico.gov.br/download/texto/me001614.pdf}}

A RAPOSA QUERIA QUE O GALO E A GALINHA DECESEM DA ÁRVORE PARA

A) FAZER UMA FESTA.

B) QUERER COMER-LÓ.

C) BRINCAR COM ELES.

D) CONTAR UM SEGREDO.

Habilidade SAEB~Inferir informações em textos verbais.

(A) Incorreta, pois considerou que a raposa disse que ia festejar.

(B ) Correta, pois a intenção da raposa era comê-lo.

(C) Incorreta, pois considerou que ela pediu para descer.

(C) Incorreta. pois por considerar que como ia contar um segredo tinha
que ser de perto.\protect\hypertarget{_heading=h.jyvxl0weyrwn}{}{}

\subsubsection{15}\label{section-87}

OBSEVE A
CENA.\includegraphics[width=3.56528in,height=2.66389in]{media/image254.png}

https://www.storyboardthat.com/portal/storyboards/b5bbeb4e/class-storyboard/unknown-story16

TITO PAROU O CARRO POR QUE

A) ESQUECEU O DOCUMENTO.

B) FALTOU COMBUSTÍVEL.

C) HOUVE UMA EXPLOSÃO.

D) DESISTIU DE VIAJAR.

Habilidade Saeb~Inferir informações em textos que articulam linguagem
verbal e não verbal

\textbf{Habilidade BNCC} EF15LP14 Construir o sentido de histórias em
quadrinhos e tirinhas, relacionando imagens e palavras e interpretando
recursos gráficos (tipos de balões, de letras, onomatopeias).

(A) Incorreta, por acreditar sem documento o carro pode ser apreendido.

(B) Incorreta, por acreditar que não passou no posto para abastecer.

(C) Correta, pois o balão de boom no texto significa que teve uma
explosão.

(D) Incorreta, por acreditar que ele não estava mais interessado em
viajar.

.

.
