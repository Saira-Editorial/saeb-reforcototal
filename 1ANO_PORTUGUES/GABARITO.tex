\chapter{Respostas}
\pagestyle{plain}
\footnotesize

\pagecolor{gray!40}

\colorsec{Língua Portuguesa – Módulo 1 – Treino}

\begin{enumerate}
\item
(A) Correta. Na placa aparece apenas letras do alfabeto.
(B) Incorreta. Na placa aparece outros sinais gráficos como números e o @.
(C) Incorreta. Na placa aparece outros sinais gráficos como números \$ e \%.
(D) Incorreta. Na placa aparece outros sinais gráficos como números e o \&.

\item
(A) Está correta, pois a palavra começa com o som pa.
(B) Está incorreta, pois o ne não é o som inicial ele está no meio da palavra.
(C) Está incorreta, pois pode ter confundido os sons do pa com ga. .
(D) Está incorreta, pois essa é o som final.

\item
(A) Está incorreta, por considerar o som inicial te que também tem na palavra peteca.
(B) Está correta, pois a palavra termina com o mesmo som.
(C) Está incorreta. pois confundiu o som de da com ca.
(D) Está incorreta, pois confundiu o som medial com a final.
\end{enumerate}

\colorsec{Língua Portuguesa – Módulo 2 – Treino}

\begin{enumerate}
\item
(A) Está correta, pois a palavra começa com o som te
(B) Está incorreta, pois o ne confundiu a ordem do som das letras.
(C) Está incorreta, pois pode ter confundido os sons do te com pe
(D) Está incorreta, pois pode ter confundido os sons do te com le.

\item
(A) Está incorreta, pois essa palavra aparece no meio da frase.
(B) Está correta, pois a frase começa com essa palavra.
(C) Está incorreta, por confundido a frase final com a inicial.
(D) Está incorreta, pois não observou que essa palavra está no meio da frase.

\item
(A) Está incorreta, pois não se atentou a direção da leitura.
(B) Está incorreta, pois confundiu com primeira palavra da última frase.
(c) Está incorreta, pois confundiu com a última palavra de uma das frases.
(D) Está correta, pois o texto termina com essa palavra.
\end{enumerate}

\colorsec{Língua Portuguesa – Módulo 3 – Treino}

\begin{enumerate}
\item
(A) Está correta, pois Aa borboleta que gosta de luz é azul.
(B) Está incorreta, por acreditar que como são escuras gostam de luz.
(C) Está incorreta, por acreditar que as brancas gostam de luz por serem claras.
(D) Está incorreta, por acreditar que como a luz também é amarela essas gostam de luz.

\item
(A) Está correta, pois será usada 6 fatias de pães.
(B) Está incorreta, por acreditar que o queijo é usado em fatias.
(C) Está incorreta, pois confundiu a quantidade uma vez que o pepino também será usado em fatias.
(D) Está incorreta, por acreditar que a cenoura pode ser cortada em fatias.

\item
(A) Está incorreta, por acreditar que a foca bate palma se por uma bola
no nariz pois gosta de brincar.
(B) Está incorreta, por acreditar que se espetar a barriga ela sente
sossegas e bate palma.
(C) Está correta, pois a foca bate palmas se dar a ela uma sardinha.
(D) Está incorreta, por confundir briga com sardinha que pode se também
considerada uma brincadeira de bater.
\end{enumerate}

\colorsec{Língua Portuguesa – Módulo 4 – Treino}

\begin{enumerate}
\item
(A) Está incorreta, por acreditar que como aparece o nome de uma pessoa seria uma carta.
(B) Está incorreta, pois acreditou que na fazenda a vovó faz muitas receitas
(C) Está correta, pois o texto é um convite para um aniversário.
(D) Está incorreta, por achar que como o texto tem data está agendando alguma coisa.

\item
(A) Está incorreta, pelo fato de acreditar que os horários do cartaz
estavam organizados tarefa.
(B) Está incorreta, por acreditar que a sequência de horários seria a
quantidade de ingrediente de uma receita.
(C) Está correta, pois esse cartaz informa um evento para as crianças.
(D) Está incorreta, por acreditar que o fato de ter crianças no cartaz
elas estariam contando uma história.

\item
(A) Está correta, pois esse texto ensina fazer uma comida.
(B) Está incorreta, por acreditar como é uma salada de fruta o texto fala sobre os nutrientes.
(C) Está incorreta, por acreditar que estaria divulgado a comida para vender.
(D) Está incorreta, por acreditar esse texto está falando sobre os
nutrientes de cada fruta.
\end{enumerate}

\colorsec{Língua Portuguesa – Módulo 5 – Treino}

\begin{enumerate}
\item
(A) Está incorreta, pois não analisou que o macaco era quem tirava o retrato.
(B) Está incorreta, por confundiu que quem ganhou o sapato foi o macaco.
(C) Está correta, pois descobriu que o assunto do texto era o retrato
que o pato que foi tirar assim que ganhou sapato.
(D) Está incorreta, por acreditar que o pato foi dar queixa ao macaco.

\item
(A) Está incorreta, pois acreditou que o fato da frase está no texto ela
seria o assunto principal.
(B) Está correta, pois o texto traz uma mensagem de conscientização para
a preservação do meio ambiente.
C) Está incorreta, por acreditar que como aparece no texto essa
informação seria ideia principal.
(D) Está incorreta, por acreditar que como tem as mãos abaixo da imagem
da planta essa seria o assunto do texto.

\item
(A) Está incorreta, por acreditar que como o texto diz como era o
galinho esse seria o assunto do texto.
(B) Está incorreta, pois como no final do texto o galinho foi encontrado
esse seria o assunto do texto.
(D) Está incorreta, por acreditar que o galinho desapareceu porque era
teimoso.
(D) Está correta, pois o galinho tinha se perdido no texto e
desaparecido.
\end{enumerate}

\colorsec{Língua Portuguesa – Módulo 6 – Treino}

\begin{enumerate}
\item
(A) Está incorreta, pois achou que a menina sentou no chão e se sujou.
(B) Está incorreta, por acreditar que a menina sentou no banco.
(C) Está correta, pois a menina sentou na ponte.
(D) Está incorreta, por acreditar que ela caiu no poço na estrada e caiu.

\item
(A) Está correta, pois a velhinha dava sua vida para ter alguém para conversar com ela.
(B) Está incorreta, por acreditar que ela gostava de ficar sozinha.
(C) Está incorreta, por acreditar que não tinha ninguém para falar.
(D) Está incorreta, por acreditar a velhinha tinha um hábito da falar sozinha.

\item
(A) Está correta, pois não pode existir uma casa sem chão e sem teto,
(B) Está incorreta, por acreditar que se não tinha chão não tinha onde
as pessoas pisarem.
(C) Está incorreta, por acreditar que a casa era pequena não tinha como
as pessoas entrarem.
(D) Está incorreta, por acreditar que como a casa era engraçada só podia
entrar palhaço.
\end{enumerate}

\colorsec{Língua Portuguesa – Módulo 7 – Treino}

\begin{enumerate}
\item
(A) Está incorreta, pois tem um sapato no quarto.
(B) Está correta, por a menina de blusa amarela está o tempo todo com o
vestido na mão.
(C) Está incorreta, por acreditar que a menina pegou a cortina da janela.
(D) Está incorreta, por acreditar que a menina ia dormir na cama.

\item
(A) Está incorreta, pois no segundo quadrinho a menina está indo embora.
(B) Está incorreta, pois no primeiro quadrinho a menina pediu que ela
adivinhasse.
(C) Está incorreta, por acreditar que o recurso utilizado para fazer
barulho na piscina seria a menina pulando na piscina.
(D) Está correta, pois a bolsa está dentro da
piscina.

\item
(A) Está incorreta, por acreditar que a menina não fazia ideia do que a
amiga estava fazendo no quarto.
(B) Está incorreta, por acreditar a menina ficou nervosa ao ver a outra
nos seu quarto.
(C) Está incorreta, por acreditar que ela não gostou em ver a amiga no
seu quarto.
(D) Está correta, pois a posição dos braços mostra que ele está
reclamando alguém.
\end{enumerate}