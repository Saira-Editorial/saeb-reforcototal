\chapter{Respostas}
\pagestyle{plain}
\footnotesize

\pagecolor{gray!40}

\section*{Módulo 1 – Treino}

\begin{enumerate}
\item
(A) Correta. Na placa aparece apenas letras do alfabeto.
(B) Incorreta. Na placa aparece outros sinais gráficos como números e o @.
(C) Incorreta. Na placa aparece outros sinais gráficos como números \$ e \%.
(D) Incorreta. Na placa aparece outros sinais gráficos como números e o \&.
SAEB: Relacionar elementos sonoros das palavras com sua representação escrita. 
BNCC: EF01LP04 -- Distinguir as letras do alfabeto de outros sinais gráficos. 

\item
(A) Está correta, pois a palavra começa com o som pa.
(B) Está incorreta, pois o ne não é o som inicial ele está no meio da palavra.
(C) Está incorreta, pois pode ter confundido os sons do pa com ga.
(D) Está incorreta, pois essa é o som final.
SAEB: Relacionar elementos sonoros das palavras com sua representação escrita.
BNCC: EF01LP0 -- Reconhecer o sistema de escrita alfabética como
representação dos sons da fala.

\item
(A) Está incorreta, por considerar o som inicial te que também tem na palavra peteca.
(B) Está correta, pois a palavra termina com o mesmo som.
(C) Está incorreta. pois confundiu o som de da com ca.
(D) Está incorreta, pois confundiu o som medial com a final.
SAEB: Relacionar elementos
sonoros das palavras com sua representação escrita.
BNCC: EF01LP09 -- Comparar palavras, identificando semelhanças e diferenças
entre sons de sílabas iniciais, mediais e finais.
\end{enumerate}

\section*{Módulo 2 – Treino}

\begin{enumerate}
\item
(A) Está correta, pois a palavra começa com o som te
(B) Está incorreta, pois o ne confundiu a ordem do som das letras.
(C) Está incorreta, pois pode ter confundido os sons do te com pe
(D) Está incorreta, pois pode ter confundido os sons do te com le.
SAEB: Ler palavras.
BNCC: EF12LP01 -- Ler palavras novas com precisão na decodificação, no caso
de palavras de uso frequente, ler globalmente, por memorização.

\item
(A) Está incorreta, pois essa palavra aparece no meio da frase.
(B) Está correta, pois a frase começa com essa palavra.
(C) Está incorreta, por confundido a frase final com a inicial.
(D) Está incorreta, pois não observou que essa palavra está no meio da frase.
SAEB: Ler frases.
BNCC: EF01LP01 -- Reconhecer que textos são lidos e escritos da esquerda
para a direita e de cima para baixo da página.

\item
(A) Está incorreta, pois não se atentou a direção da leitura.
(B) Está incorreta, pois confundiu com primeira palavra da última frase.
(c) Está incorreta, pois confundiu com a última palavra de uma das frases.
(D) Está correta, pois o texto termina com essa palavra.
SAEB: Ler frases.
BNCC: EF01LP01 -- Reconhecer que textos são lidos e escritos da esquerda
para a direita e de cima para baixo da página.
\end{enumerate}

\section*{Módulo 3 – Treino}

\begin{enumerate}
\item
(A) Está correta, pois Aa borboleta que gosta de luz é azul.
(B) Está incorreta, por acreditar que como são escuras gostam de luz.
(C) Está incorreta, por acreditar que as brancas gostam de luz por serem claras.
(D) Está incorreta, por acreditar que como a luz também é amarela essas gostam de luz.
SAEB: Localizar informações explícitas em textos.
BNCC: EF15LP03 -- Localizar informações explícitas em textos.

\item
(A) Está correta, pois será usada 6 fatias de pães.
(B) Está incorreta, por acreditar que o queijo é usado em fatias.
(C) Está incorreta, pois confundiu a quantidade uma vez que o pepino também será usado em fatias.
(D) Está incorreta, por acreditar que a cenoura pode ser cortada em fatias.
SAEB: localizar informações explícitas em textos.
BNCC: EF15LP03 -- Localizar informações explícitas em textos.

\item
(A) Está incorreta, por acreditar que a foca bate palma se por uma bola
no nariz pois gosta de brincar.
(B) Está incorreta, por acreditar que se espetar a barriga ela sente
sossegas e bate palma.
(C) Está correta, pois a foca bate palmas se dar a ela uma sardinha.
(D) Está incorreta, por confundir briga com sardinha que pode se também
considerada uma brincadeira de bater.
SAEB: localizar informações explícitas em textos.
BNCC: EF15LP03 -- EF15LP03 Localizar informações explícitas em textos.
\end{enumerate}

\section*{Módulo 4 – Treino}

\begin{enumerate}
\item
(A) Está incorreta, por acreditar que como aparece o nome de uma pessoa seria uma carta.
(B) Está incorreta, pois acreditou que na fazenda a vovó faz muitas receitas
(C) Está correta, pois o texto é um convite para um aniversário.
(D) Está incorreta, por achar que como o texto tem data está agendando alguma coisa.
SAEB: Reconhecer a finalidade de um texto.
BNCC: EF15LP01 -- Identificar a função social de textos que
circulam em campos da vida social dos quais participa cotidianamente (a
casa, a rua, a comunidade, a escola) e nas mídias impressa, de massa e
digital, reconhecendo para que foram produzidos, onde circulam, quem os
produziu e a quem se destinam.

\item
(A) Está incorreta, pelo fato de acreditar que os horários do cartaz
estavam organizados tarefa.
(B) Está incorreta, por acreditar que a sequência de horários seria a
quantidade de ingrediente de uma receita.
(C) Está correta, pois esse cartaz informa um evento para as crianças.
(D) Está incorreta, por acreditar que o fato de ter crianças no cartaz
elas estariam contando uma história.
SAEB: Reconhecer a finalidade de um texto.
BNCC: EF15LP01 -- Identificar a função social de textos que
circulam em campos da vida social dos quais participa cotidianamente (a
casa, a rua, a comunidade, a escola) e nas mídias impressa, de massa e
digital, reconhecendo para que foram produzidos, onde circulam, quem os
produziu e a quem se destinam.

\item
(A) Está correta, pois esse texto ensina fazer uma comida.
(B) Está incorreta, por acreditar como é uma salada de fruta o texto fala sobre os nutrientes.
(C) Está incorreta, por acreditar que estaria divulgado a comida para vender.
(D) Está incorreta, por acreditar esse texto está falando sobre os
nutrientes de cada fruta.
SAEB: Reconhecer a finalidade de um texto.
BNCC: EF15LP01 -- Identificar a função social de textos que
circulam em campos da vida social dos quais participa cotidianamente (a
casa, a rua, a comunidade, a escola) e nas mídias impressa, de massa e
digital, reconhecendo para que foram produzidos, onde circulam, quem os
produziu e a quem se destinam.
\end{enumerate}

\section*{Módulo 5 – Treino}

\begin{enumerate}
\item
(A) Está incorreta, pois não analisou que o macaco era quem tirava o retrato.
(B) Está incorreta, por confundiu que quem ganhou o sapato foi o macaco.
(C) Está correta, pois descobriu que o assunto do texto era o retrato
que o pato que foi tirar assim que ganhou sapato.
(D) Está incorreta, por acreditar que o pato foi dar queixa ao macaco.
SAEB: Inferir o assunto de um texto.

\item
(A) Está incorreta, pois acreditou que o fato da frase está no texto ela
seria o assunto principal.
(B) Está correta, pois o texto traz uma mensagem de conscientização para
a preservação do meio ambiente.
C) Está incorreta, por acreditar que como aparece no texto essa
informação seria ideia principal.
(D) Está incorreta, por acreditar que como tem as mãos abaixo da imagem
da planta essa seria o assunto do texto.
SAEB: Inferir o assunto de um texto.

\item
(A) Está incorreta, por acreditar que como o texto diz como era o
galinho esse seria o assunto do texto.
(B) Está incorreta, pois como no final do texto o galinho foi encontrado
esse seria o assunto do texto.
(D) Está incorreta, por acreditar que o galinho desapareceu porque era
teimoso.
(D) Está correta, pois o galinho tinha se perdido no texto e
desaparecido.
SAEB: Inferir o assunto de um texto.
\end{enumerate}

\section*{Módulo 6 – Treino}

\begin{enumerate}
\item
(A) Está incorreta, pois achou que a menina sentou no chão e se sujou.
(B) Está incorreta, por acreditar que a menina sentou no banco.
(C) Está correta, pois a menina sentou na ponte.
(D) Está incorreta, por acreditar que ela caiu no poço na estrada e caiu.
SAEB: Inferir informações em textos verbais.

\item
(A) Está correta, pois a velhinha dava sua vida para ter alguém para conversar com ela.
(B) Está incorreta, por acreditar que ela gostava de ficar sozinha.
(C) Está incorreta, por acreditar que não tinha ninguém para falar.
(D) Está incorreta, por acreditar a velhinha tinha um hábito da falar sozinha.
SAEB: Inferir informações em textos verbais.

\item
(A) Está correta, pois não pode existir uma casa sem chão e sem teto,
(B) Está incorreta, por acreditar que se não tinha chão não tinha onde
as pessoas pisarem.
(C) Está incorreta, por acreditar que a casa era pequena não tinha como
as pessoas entrarem.
(D) Está incorreta, por acreditar que como a casa era engraçada só podia
entrar palhaço.
SAEB: Inferir informações em textos verbais.
\end{enumerate}

\section*{Módulo 7 – Treino}

\begin{enumerate}
\item
(A) Está incorreta, pois tem um sapato no quarto.
(B) Está correta, por a menina de blusa amarela está o tempo todo com o
vestido na mão.
(C) Está incorreta, por acreditar que a menina pegou a cortina da janela.
(D) Está incorreta, por acreditar que a menina ia dormir na cama.
SAEB: Inferir informações em textos verbais e não verbais.
BNCC: EF15LP14 -- Construir o sentido de histórias em
quadrinhos e tirinhas, relacionando imagens e palavras e interpretando
recursos gráficos (tipos de balões, de letras, onomatopeias).

\item
(A) Está incorreta, pois no segundo quadrinho a menina está indo embora.
(B) Está incorreta, pois no primeiro quadrinho a menina pediu que ela
adivinhasse.
(C) Está incorreta, por acreditar que o recurso utilizado para fazer
barulho na piscina seria a menina pulando na piscina.
(D) Está correta, pois a bolsa está dentro da
piscina.
SAEB: Inferir informações em textos verbais e não verbais.
BNCC: EF15LP14 -- EF15LP14 Construir o sentido de histórias em
quadrinhos e tirinhas, relacionando imagens e palavras e interpretando
recursos gráficos (tipos de balões, de letras, onomatopeias).

\item
(A) Está incorreta, por acreditar que a menina não fazia ideia do que a
amiga estava fazendo no quarto.
(B) Está incorreta, por acreditar a menina ficou nervosa ao ver a outra
nos seu quarto.
(C) Está incorreta, por acreditar que ela não gostou em ver a amiga no
seu quarto.
(D) Está correta, pois a posição dos braços mostra que ele está
reclamando alguém.
SAEB: Inferir informações em textos verbais e não verbais.
Habilidade BNCC: EF15LP14 -- Construir o sentido de histórias em
quadrinhos e tirinhas, relacionando imagens e palavras e interpretando
recursos gráficos (tipos de balões, de letras, onomatopeias).
\end{enumerate}

\section*{Simulado 1}

\begin{enumerate}
\item
(A) Correta, pois na placa aparece apenas letras do alfabeto.
(B) Incorreta, pois na placa aparece outros sinais gráficos como números
e o @.
(C) Incorreta, pois na placa aparece outros sinais gráficos como números
\$ e \%.
(D) Incorreta, pois na placa aparece outros sinais gráficos como números
e o \&.
SAEB: Relacionar elementos sonoros das palavras com sua representação escrita.
BNCC: EF01LP04 -- Distinguir as letras do alfabeto de outros sinais gráficos.

\item
(A) Incorreta, pois confundiu o som das letras g com r.
(B) Incorreta, pois confundiu o som das letras g com o som de p.
(C) Está correta, pois a palavra começa com esse som.
(D) Está incorreta. pois confundiu o som de g com com k.
SAEB: Relacionar elementos sonoros das palavras com sua
representação escrita.
BNCC: EF01LP05 -- Reconhecer o sistema de escrita alfabética como
representação dos sons da fala.

\item
(A) Incorreta. A palavra está representada por seis sons.
(B) Incorreta.A palavra possui mais de seis sons.
(C) Incorreta. A palavra possui oito sons.
(D) Correta: A palavra está representada por quatro sons, assim como na
palavra bola.
SAEB: Relacionar elementos sonoros das palavras com sua
representação escrita.
BNCC: EF01LP07 -- Identificar fonemas e sua representação por
letras.

\item
(A) Incorreta, pois não se atentou a posição da sílaba.
(B) Incorreta, pois confundiu a posição da sílaba..
(C) Correta, pois a sílaba final da palavra amarela é LO.
(D) Incorreta, pois confundiu o som das sílabas.
SAEB: Relacionar
elementos sonoros das palavras com sua representação escrita.
BNCC: EF01LP08 -- Relacionar elementos sonoros (sílabas, fonemas,
partes de palavras) com sua representação escrita.

\item
(A) correta, pois as palavras terminam com o mesmo som.
(B) incorreta, pois confundiu o som medial como final.
(C) Incorreta, pois confundiu o som medial com o inicial.
(D) Incorreta, pois confundiu o som medial com o final.
SAEB: Relacionar elementos sonoros das palavras com sua
representação escrita.
BNCC: EF01LP09 -- Comparar palavras, identificando semelhanças e
diferenças entre sons de sílabas iniciais, mediais e finais.

\item
(A) Incorreta, pois confundiu em razão da sílaba inicial ser a mesma.
(B) Correta, pois o nome da figura é sapato.
(C) Incorreta, pois acreditou que pelo fato de ter a silaba to igual seria o nome da palavra.
(D) Incorreta, pois acreditou que pelo fato de ter a silaba pa igual seria o nome da palavra.
SAEB: Ler palavras.
BNCC: EF12LP01 -- Ler palavras novas com precisão na
decodificação, no caso de palavras de uso frequente, ler globalmente,
por memorização.

\item
POSSIVEIS RESPOSTAS:
Abacaxi, Abxica, Acaxi, abacai.
SAEB: Escrever palavras.
BNCC: EF01LP02 -- Escrever, espontaneamente ou por ditado, palavras e frases de forma alfabética usando letras/grafemas que representem fonemas.

\item
(A) correta, pois a duas palavras termina com o mesmo som.
(B) incorreta. por acreditar eu pelo fato de ter parte da palavra escrito.
(C) Incorreta por confundir a sílaba final com a inicial.
(D) Incorreta por acreditar que pelo fato das duas palavras ter a sílaba inicial igual.
SAEB: Ler palavras.
BNCC: EF01LP13 -- Comparar palavras, identificando semelhanças e
diferenças entre sons de sílabas iniciais, mediais e finais.

\item
(A) correta, pois o menino está chutando a bola.
(B) incorreta, pois o menino seguro a bola.
(C) Incorreta, pois o menino está com a bola na cabeça.
(D) Incorreta, pois o menino está batendo a bola com o braço.
SAEB: Ler frases.
BNCC: EF01LP01 -- Reconhecer que textos são lidos e escritos da
esquerda para a direita e de cima para baixo da página.

\item
(A) Incorreta. Por acreditar que como o cravo ficou doente no texto esse seria o assunto principal.
(B) Incorreta, por considerar a rosa ficou assim depois da briga.
(C) Correta, pois houve a briga do cravo e da rosa debaixo de uma sacada.
(D) Incorreta, pois confundiu a briga com uma visita.
SAEB: Inferir o assunto de um texto.

\item
(A) Incorreta, pois sábado e domingo não tem feira.
(B) Incorreta, pois na terça vende pimentão mas aos domingo não tem feira.
(C) Incorreta. pois quarta vende quiabo e pão e sábado não tem feira.
(D) Correta, pois na terça vende pimentão e sexta vende mamão.
SAEB: Localizar informações explícitas em textos.

\item
(A) Incorreta, pois se confundiu pelo fato do bilhete deixar um recado.
(B) Correta, pois
(C) Incorreta. pois acreditou que ia sair uma notícia do evento.
(D) Incorreta, pois acreditou que ia mandar uma receita do bolo do aniversário.
SAEB: Reconhecer a finalidade de um texto.
BNCC: EF15LP01 -- Identificar a função social de textos que
circulam em campos da vida social dos quais participa cotidianamente (a
casa, a rua, a comunidade, a escola) e nas mídias impressa, de massa e
digital, reconhecendo para que foram produzidos, onde circulam, quem os
produziu e a quem se destinam.

\item
(A) Incorreta, por acreditar que todos deveria ser organizar para
participar da festa.
(B) Incorreta, por acreditar que ia ensinar uma receita para ficar
forte no carnaval.
(C) Correta. pois o cartaz anuncia o evento do carnaval.
(D) Incorreta, por acreditar que o cartaz seria a capa de um livro de
conto.
SAEB: Reconhecer a finalidade de um texto.
BNCC: EF15LP01 -- Identificar a função social de textos que
circulam em campos da vida social dos quais participa cotidianamente (a
casa, a rua, a comunidade, a escola) e nas mídias impressa, de massa e
digital, reconhecendo para que foram produzidos, onde circulam, quem os
produziu e a quem se destinam.

\item
(A) incorreta, por acreditar que os passarinhos da floresta estavam com
fome por João estava alimentado.
(B) correta, pois João teve esse plano pois sabia que ele e sua irmão
iriam ficar sozinhos na floresta.
(C) incorreta. pois acreditou que João e Maria iam precisar encomonnizar
o pão para comer por muitos dias na floresta.
(D) incorreta, pois acreditou que João e Maria planejava fugir dos pais
já que queria abandona- ló na floresta.
SAEB: Inferir informações em textos verbais.

\item
(A) incorreta, por acreditar que a menina não pode assistir seu desenho
favorito.
(B) Incorreta, pois acreditou que a menina estava gritando de medo.do
escuro.
(C) Correta, pois a expressão facial da menina mostra que ela ficou
muito assustada.
(D) incorreta, por acreditar que a menina estava reclamando por que não
terminou de assistir o desenho.
SAEB: Inferir informações em textos que articulam linguagem
verbal e não verbal.
BNCC: EF15LP14 -- Construir o sentido de histórias em
quadrinhos e tirinhas, relacionando imagens e palavras e interpretando
recursos gráficos (tipos de balões, de letras, onomatopeias).
\end{enumerate}

\section*{Simulado 2}

\begin{enumerate}
\item
(A) Incorreta, pois na maleta aparece outros símbolos que não serve
para escrever palavras.
(B) Incorreta, pois na maleta aparece apenas números.
(C) Correta, pois na maleta aparece as letras do alfabeto usadas para
formas palavras.
(D) Incorreta, pois na placa aparece outros símbolos diversos que não
serve para escrever palavras.
SAEB: Relacionar elementos sonoros das palavras com sua representação escrita.	
BNCC: EF01LP04 -- Distinguir as letras do alfabeto de outros sinais
gráficos.

\item
(A) Incorreta, pois a palavra começa com a consoante m e termina com a vogal o.
(B) Correta, pois a palavra começa e termina com vogal assim como a
palavra abacate.
(C) Incorreta, pois a palavra começa com a consoante c e termina com a
vogal o.
(D) Incorreta, pois a palavra começa com a consoante g e termina com a
vogal o.
SAEB: Relacionar elementos sonoros das palavras com sua
representação escrita.
BNCC: EF01LP07 -- Identificar fonemas e sua representação por
letras.

\item
(A) Correta, pois palavra está representada por quatro sons, assim como
na palavra bola.
(B) Incorreta, pois a palavra está representada por seis sons.
(C) Incorreta, pois a palavra possui mais de quatro sons.
(D) Incorreta, pois a palavra possui oito sons.
SAEB: Relacionar elementos sonoros das palavras com sua
representação escrita.
BNCC: EF01LP07 -- Identificar fonemas e sua representação por
letras.

\item
(A) Incorreta, pois confundiu com o som inicial.
(B) Correta, pois o som do meio da palavra boneca é ne.
(C) Incorreta, pois confundiu com o som final.
(D) Incorreta, pois confundiu o som de me com ne.
SAEB: Relacionar elementos sonoros das palavras com sua
representação escrita.
BNCC: EF01LP05 -- Reconhecer o sistema de escrita alfabética como
representação dos sons da fala.

\item
(A) Correta, pois a palavra cachorro começa com ca assim como a palavra
caneca.
(B) Incorreta, pois confundiu a última sílaba com a primeira.
(C) Incorreta, pois confundiu a primeira sílaba com a última.
(D) Incorreta, pois confundiu a sílaba inicial com a do meio.
SAEB: Relacionar elementos sonoros das palavras com sua
representação escrita.
BNCC: EF01LP09 -- Comparar palavras, identificando semelhanças e
diferenças entre sons de sílabas iniciais, mediais e finais.

\item
(A) Incorreta, pois confundiu a som silabas.
(B) Incorreta, pois confundiu o som final com o inicial.
(C) Incorreta, pois confundiu o som de pa com ba.
(D) Correta, pois a sílaba inicial da palavra pato é pa.
SAEB: Relacionar elementos sonoros das palavras com sua
representação escrita.
BNCC: EF01LP08 -- Relacionar elementos sonoros (sílabas, fonemas,
partes de palavras) com sua representação escrita.

\item
(A) Incorreta, pois confundiu a última letra com a primeira.
(B) Incorreta, pois a palavra confundiu com a letra inicial da segunda
sílaba.
(C) Incorreta, pois confundiu com a letra da segunda sílaba da palavra.
(D) Correta, pois a palavra termina com a.
SAEB: Ler palavras.
BNCC: EF01LP08 -- Relacionar elementos sonoros (sílabas, fonemas,
partes de palavras) com sua representação escrita.

\item
(A) Incorreta, pois se atentou apenas a primeira sílaba da palavra.
(B) Correta, pois essa palavra é o nome do brinquedo.
(C) Incorreta, pois observou apenas a segunda sílaba da palavra.
(D) Incorreta, pois só observou a última sílaba.
SAEB: Ler palavras.
BNCC: EF12LP01 -- Ler palavras novas com precisão na
decodificação, no caso de palavras de uso frequente, ler globalmente,
por memorização.

\item
(A) Correta, pois o macaco está comendo uma banana como mostra a imagem.
(B) Incorreta, por acreditar que o macaco não queria mais comer a banana.
(C) Incorreta, por acreditar que o macaco deixou a banana cair sem querer.
(D) Incorreta, por acreditar que o macaco não comeu a banana porque estava verde.
SAEB: Ler frases.
BNCC: EF01LP01 -- Reconhecer que textos são lidos e escritos da esquerda para a
direita e de cima para baixo da página.

\item
POSSIVEIS RESPOSTAS:
girafa, irafa, grafa e girfa.
SAEB: Escrever palavras.
BNCC: EF01LP02 -- Escrever, espontaneamente ou por ditado,
palavras e frases de forma alfabética -- usando letras/grafemas que
representem fonemas.

\item
(A) Incorreta, porá acreditar que o rato gosta de procurar migalhas nas
varada das casas
(B) Incorreta, por acreditar que como o gato estava fazendo a casa ele
queria com varanda.
(C) Correta, pois o pato falou que casa bonita tem que ter varada.
(D) Incorreta, pois confundiu a opinião do pato com a do bode.
SAEB: Localizar informações explicitas em textos.

\item
(A) Incorreta, por considerar o fato de aparecer dinheiro no texto.
(B) Correta, pois a barata contou várias mentiras dizendo que tinha o
que na verdade não tinha.
(C) Incorreta, pois considerou que a barato tinha o desejo de ter as
coisas.
(C) Incorreta, pois considerou que a barata tinha inveja das pessoas que
tinha as coisas e ela não tinha.
SAEB: Inferir o assunto de um texto.

\item
(A) Correta, pois um bilhete serve para deixar um recado.
(B) Incorreta, por considerar o fato de ter doces no texto ele estava
informando um evento.
(C) Incorreta, pois considerou que a amiga levou os doces ela estava
cumprindo uma organização de tarefas feitas por elas.
(D) Incorreta, pois considerou que o texto fala de doces e salgados ele
poderia estar ensinando uma
comida.
SAEB: Reconhecer a finalidade de um texto.
BNCC: EF15LP01 -- Identificar a função social de textos que
circulam em campos da vida social dos quais participa cotidianamente (a
casa, a rua, a comunidade, a escola) e nas mídias impressa, de massa e
digital, reconhecendo para que foram produzidos, onde circulam, quem os
produziu e a quem se destinam.

\item
(A) Incorreta, por considerar que quem ajuda é bom e gentil.
(B) Incorreta, pois acreditou que como o Leão é o rei da selva ele é
muito valente.
(C) Incorreta, pois considerou o medo que teve de ser esmagado pelo
Leão.
(D) Correta, pois no passado o rato tinha também lhe ajudado e o leão
estava retribuindo com ele.
SAEB: Inferir informações em textos verbais.

\item
(A) Incorreta, por considerar que Beto tem um cachorro.
(B) Incorreta, pois acreditou que o cachorro foi o amigo que tinha dado
para Beto.
(C) Incorreta, pois considerou o presente que Lucas o tinha dado no
terceiro quadrinho.
(D) Correta, pois no segundo quadrinho na fala de Beto ele diz sorrindo
que agora não ia brincar sozinho.
SAEB: Inferir informações em textos que articulam linguagem
verbal e não verbal.
BNCC: EF15LP14 -- Construir o sentido de histórias em
quadrinhos e tirinhas, relacionando imagens e palavras e interpretando
recursos gráficos (tipos de balões, de letras, onomatopeias).
\end{enumerate}

\section*{Simulado 3}

\begin{enumerate}
\item
(A) Incorreta, pois essa placa apresenta números a poucas letras.
(B) Correta, pois para escrever palavras usamos as letras do alfabeto.
(C) Incorreta, pois a placa só aparece placa de trânsito.
(D) Incorreta, pois a placa só aparece números.
SAEB: Relacionar elementos sonoros das palavras com sua
representação escrita.
BNCC: EF01LP04 -- Distinguir as letras do alfabeto de outros sinais
gráficos.

\item
(A) Incorreta, pois a palavra tem apenas duas vogais.
(B) Correta, pois a palavra possui três vogais, assim como o nome do
cachorro.
(C) Incorreta, pois a palavra possui cinco vogais.
(D) Incorreta, pois a palavra tem quatro vogais.
SAEB: Relacionar elementos sonoros das palavras com sua
representação escrita.
BNCC: EF01LP07 -- Identificar fonemas e sua representação por
letras.

\item
(A) Correta, pois a palavra borboleta termina com ta o mesmo som final
da palavra batata.
(B) Incorreta, pois confundiu a som final com o inicial borboleta e
tapete.
(C) Incorreta, pois confundiu o som final com inicial da questão.
(D) Incorreta, pois acreditaram que poderiam ser qualquer som igual
presentes nas duas palavras.
SAEB: Relacionar elementos sonoros das palavras com sua
representação escrita.
BNCC: EF01LP09 -- Comparar palavras, identificando semelhanças e
diferenças entre sons de sílabas iniciais, mediais e finais.

\item
(A) Incorreta, pois confundiu com a sílaba inicial achado que ela não
podia ser formada por uma única letra e acrescentou mais uma letra E.
(B) Incorreta, pois confundiu os sons das letras.
(C) Incorreta, pois confundiu a posição da sílaba.
(D) Correta, pois a palavra termina com TE.
SAEB: Relacionar elementos sonoros das palavras com sua
representação escrita.
BNCC: EF01LP05 -- Reconhecer o sistema de escrita alfabética
como representação dos sons da fala.

\item
(A) Incorreta, pois confundiu a sílaba inicial com a do meio.
(B) Correta, pois a palavra termina com V + A.
(C) Incorreta, pois confundiu a posição da sílaba.
(D) Incorreta, pois confundiu a posição os sons das letras.
SAEB: Relacionar elementos sonoros das palavras com sua representação escrita.
BNCC: EF01LP08 -- Relacionar elementos sonoros (sílabas, fonemas, partes de
palavras) com sua representação escrita.

\item
(A) Incorreta, pois confundiu a primeira palavra com a última.
(B) Incorreta, pois acreditou que essa palavra fosse junto com da maré.
(C) Incorreta, pois confundiu com a alguns versos que termina com essa palavra.
(D) Correta, pois essa é a última palavra do texto.
SAEB: Ler palavras.
BNCC: EF01LP01 -- Reconhecer que textos são lidos e escritos da esquerda para a
direita e de cima para baixo da página.

\item
(A) Incorreta, pois confundiu a som silabas.
(B) Incorreta, pois confundiu o som final com o inicial.
(C) Incorreta, pois confundiu o som de pa com ba.
(D) Correta, pois a sílaba inicial da palavra pato é pa.
SAEB: Ler palavras.
BNCC: EF01LP13 -- Comparar palavras, identificando semelhanças e
diferenças entre sons de sílabas iniciais, mediais e finais.

\item
Possiveis respostas:
Vestido, Vetido, Etido e Vetdo. SAEB: Escrever palavras.
BNCC: EF01LP02 -- Escrever, espontaneamente ou por ditado, palavras e frases de
forma alfabética usando letras/grafemas que representem fonemas.

\item
(A) Incorreta, pois confundiu a som silabas.
(B) Incorreta, pois confundiu o som final com o inicial.
(C) Incorreta, pois confundiu o som de pa com ba.
(D) Correta, pois a sílaba inicial da palavra pato é pa.

\item
(A) Incorreta, pois leu só as últimas sílabas.
(B) Incorreta, pois confundiu o som da letra c.
(C) Correta, pois essa é a palavra certa.
(D) Incorreta, pois trocou a ordem das sílabas.
SAEB: Ler palavras.
BNCC: EF12LP01 -- Ler palavras novas com precisão na
decodificação, no caso de palavras de uso frequente, ler globalmente,
por memorização.

\item
(A) Incorreta, pois confundiu a última sílaba com a primeira.
(B) Correta, pois essa é a palavra certa
(C)Incorreta, pois considerou a sílaba do meio.
(D) Incorreta, pois considerou as sílabas escondidas dentro da palavra.
SAEB: Ler palavras.
BNCC: EF01LP13 -- Comparar palavras, identificando semelhanças e diferenças entre sons de sílabas iniciais, mediais e finais.

\item
(A) Incorreta, por acreditar que o ouro seria guardado na bacia.
(B) Correta, pois no texto fala pega essa criança e joga na bacia.
(C)Incorreta, pois acreditou que o roupão ia ser lavado na bacia.
(D) Incorreta, pois acreditou que a caminha estava suja e ia ser lavada
na bacia.
SAEB: Localizar informações explícitas em textos.
BNCC: EF15LP03 -- Localizar informações explícitas em textos.

\item
(A) Incorreta, por considerar que o texto está bem organizado.
(B) Incorreta, por considerar que o texto traz muitos nomes de comida
(C) Correta, pois a lista indica os itens de uma compra.
(D) Incorreta, pois acreditou que o texto informava um evento de comida.
SAEB: Reconhecer a finalidade de um texto.
BNCC: EF15LP01 -- Identificar a função social de textos que
circulam em campos da vida social dos quais participa cotidianamente (a
casa, a rua, a comunidade, a escola) e nas mídias impressa, de massa e
digital, reconhecendo para que foram produzidos, onde circulam, quem os
produziu e a quem se destinam.

\item
(A) Incorreta, por considerar que ele foi o primeiro a lhe acudi.
(B) Incorreta, por acreditar que a Terezinha estava estendendo a mão
para alguém acudi.
(C) Incorreta, por considerar que seu irmão tinha a obrigação de lhe
ajudar.
(D)correta, pois no texto fala da que ela foi ao chão de uma queda.
SAEB: Inferir o assunto de um texto.

\item
(A) Incorreta, por acreditar que a floresta estava triste.
(B) Incorreta, por acreditar que a raposa ia lhe elogiar juntos com os
animais.
(C) Incorreta, por acreditar que a raposa estava feliz em ver o corvo
(D) correta, pois a raposa queria que o queijo caísse para ela
pegar.
SAEB: Inferir informações em textos verbais.

\item
(A) Incorreta, por acreditar que o menino não estava atento.
(B) Incorreta, por acreditar que ele ia segurar a bola.
(C) Incorreta, por acreditar que ele estava brincado de futebole ele
estava sendo o goleiro.
(D) Correta, pois no terceiro quadrinho afirma que ele tem pouca visão.
SAEB: Inferir informações em textos que articulam linguagem
verbal e não verbal.
BNCC: EF15LP14 -- Construir o sentido de histórias em
quadrinhos e tirinhas, relacionando imagens e palavras e interpretando
recursos gráficos (tipos de balões, de letras, onomatopeias).
\end{enumerate}

\section*{Simulado 4}

\begin{enumerate}
\item
(A) Incorreta, pois essa placa aparece números.
(B) Incorreta, pois a placa só aparece setas.
(C) Correta, pois para formar palavras usamos as letras do alfabeto.
(D) Incorreta, pois a placa só aparece emojis.
SAEB: Relacionar elementos sonoros das palavras com sua representação escrita.
BNCC: EF01LP04 -- Distinguir as letras do alfabeto de outros sinais gráficos.

\item
(A) Correta, pois o som final da palavra salada é DA.
(B) Incorreta, pois confundiu com o som do meio da palavra.
(C) Incorreta, pois confundiu com o som inicial.
(D) Incorreta, pois confundiu a posição do som das letras.
SAEB: Relacionar elementos sonoros das palavras com sua representação escrita.
BNCC: EF01LP05 -- Reconhecer o sistema de escrita alfabética como representação dos sons da fala.

\item
(A) Correta, pois palavra está representada por nove sons, assim como na
palavra borboleta.
(B) Incorreta, pois a palavra possui oito sons.
(C) Incorreta, pois a palavra está representada por seis sons.
(D) Incorreta, pois a palavra possui quatro sons.
SAEB: Relacionar elementos sonoros das palavras com sua representação escrita.
BNCC: EF01LP07 -- Identificar fonemas e sua representação por letras.

\item
(A) Incorreta, pois considerou a segunda sílaba da palavra.
(B) Correta, pois a palavra começa com já, assim como o nome da janela.
(C) Incorreta, pois considerou a última sílaba.
(D) Incorreta, pois confundiu a posição da sílaba.
SAEB: Relacionar elementos sonoros das palavras com sua representação escrita.
BNCC: EF01LP08 -- Relacionar elementos sonoros (sílabas, fonemas, partes de palavras) com sua representação escrita.

\item
(A) Correta, pois a palavra não apresenta nenhum som igual a palavra moto.
(B) Incorreta, pois a palavra apresenta os sons t+o igual de moto.
(C) Incorreta, pois apresenta os sons m+o igual de moto.
(D) Incorreta, pois apresenta os sons t+o igual a sílaba final da palavra.
SAEB: Relacionar elementos sonoros das palavras com sua representação escrita.
BNCC: EF01LP09 -- Comparar palavras, identificando semelhanças e diferenças entre sons de sílabas iniciais, mediais e finais.

\item
(A) Correta, pois essa é a primeira palavra do texto.
(B) Incorreta, pois considerou a última palavra do primeiro verso.
(C) Incorreta, pois confundiu a primeira palavra com a última.
(D) Incorreta, pois considerou a primeira palavra do último verso.
SAEB: Ler palavras.
BNCC: EF01LP01 -- Reconhecer que textos são lidos e escritos da esquerda para a
direita e de cima para baixo da página.

\item
(A) Incorreta, pois não considerou a ordem das sílabas.
(B) Incorreta, pois repetiu a letra o no lugar da e.
(C) Incorreta, pois não considerou a ordem das letras.
(D) Correta, pois essa é a palavra certa.
SAEB: Ler palavras. 
Habilidades BNCC:
EF12LP01 -- Ler palavras novas com precisão na decodificação, no caso de palavras de uso frequente, ler globalmente, por memorização.

\item
(A) Incorreta, pois confundiu som de n com o do m.
(B) Incorreta, pois confundiu o som do meio.
(C) Incorreta, pois confundiu o som final.
(D) Correta, pois o som inicial da palavra macaco é m igual a de
macarrão.
SAEB: Ler palavras. 
BNCC: EF01LP13 -- Comparar palavras, identificando semelhanças e diferenças entre sons de sílabas iniciais, mediais e finais.

\item
Possíveis respostas:
telefone, fonete, lefone e tefone.
SAEB: Escrever palavras.
BNCC: EF01LP02 -- Escrever, espontaneamente ou por ditado, palavras e frases de forma alfabética -- usando letras/grafemas que representem fonemas.

\item
(A) incorreta, pois quem está no logo é a tartaruga.
(B) incorreta, pois quem está no logo é o sapo.
(C) correta, pois os patos estão nadando na lagoa.
(D) Incorreta, pois os patos estão o lago mas não estão nadando.
Saeb: Ler frases.
Habilidades BNCC: EF01LP02 -- Escrever, espontaneamente ou por ditado, palavras e frases de
forma alfabética usando letras/grafemas que representem fonemas.

\item
(A) Incorreta, pois considerou que precisa dos olhos o tinteiro para
pintar.
(B) Incorreta, pois considerou que como as letra eram pequenas os olhos
podiam ver.
(C) Correta, pois no texto está claro que dos olhos era para fazer carta
fechada.
(D) Incorreta, pois por considerar que existem várias cores de
olhos.
SAEB: Localizar informações explícitas em textos. 
BNCC: EF15LP03 -- Localizar informações explícitas em textos.

\item
(A) Incorreta, pois considerou que os jovens gostam de vaidade como a
menina.
(B) Correta, pois geralmente as crianças que gosta de ouvir contos.
(C) Incorreta, pois considerou que os adultos contam histórias para as
crianças.
(D) Incorreta, pois considerou que alguma família se reúne para contar
história.
SAEB: Reconhecer a finalidade de um texto.
BNCC: EF15LP01 -- Identificar a função social de textos que
circulam em campos da vida social dos quais participa cotidianamente (a
casa, a rua, a comunidade, a escola) e nas mídias impressa, de massa e
digital, reconhecendo para que foram produzidos, onde circulam, quem os
produziu e a quem se destinam.

\item
A) Correta, pois a rainha promoveu um baile para fazer as letras se
apresentarem
(B) Incorreta, por considerar que as letras estavam em silêncio porque
estavam tristes.
(C) Incorreta, por considerar que as letras que a rainha iria promover
esse evento.
(D) Incorreta, pois confundiu com o título do texto.
SAEB: Inferir o assunto de um texto.

\item
(A) Incorreta, pois considerou que a raposa disse que ia festejar.
(B) Correta, pois a intenção da raposa era comê-lo.
(C) Incorreta, pois considerou que ela pediu para descer.
(D) Incorreta. pois por considerar que como ia contar um segredo tinha
que ser de perto.
SAEB: Inferir informações em textos verbais.

\item
(A) Incorreta, por acreditar sem documento o carro pode ser apreendido.
(B) Incorreta, por acreditar que não passou no posto para abastecer.
(C) Correta, pois o balão de boom no texto significa que teve uma explosão.
(D) Incorreta, por acreditar que ele não estava mais interessado em viajar.
SAEB: Inferir informações em textos que articulam linguagem verbal e não verbal.
BNCC: EF15LP14 -- Construir o sentido de histórias em quadrinhos e tirinhas, relacionando imagens e palavras e interpretando recursos gráficos (tipos de balões, de letras, onomatopeias).
\end{enumerate}

\blankpage