\chapter{Respostas}
\pagestyle{plain}
\footnotesize

\pagecolor{gray!40}

\colorsec{Língua Portuguesa – Módulo 1 – Treino}

\begin{enumerate}
\item
(A) Correta. Na placa aparece apenas letras do alfabeto.
(B) Incorreta. Na placa aparece outros sinais gráficos como números e o @.
(C) Incorreta. Na placa aparece outros sinais gráficos como números \$ e \%.
(D) Incorreta. Na placa aparece outros sinais gráficos como números e o \&.

\item
(A) Está correta, pois a palavra começa com o som pa.
(B) Está incorreta, pois o ne não é o som inicial ele está no meio da palavra.
(C) Está incorreta, pois pode ter confundido os sons do pa com ga. .
(D) Está incorreta, pois essa é o som final.

\item
(A) Está incorreta, por considerar o som inicial te que também tem na palavra peteca.
(B) Está correta, pois a palavra termina com o mesmo som.
(C) Está incorreta. pois confundiu o som de da com ca.
(D) Está incorreta, pois confundiu o som medial com a final.
\end{enumerate}

\colorsec{Língua Portuguesa – Módulo 2 – Treino}

\begin{enumerate}
\item

\item

\item
\end{enumerate}

\colorsec{Língua Portuguesa – Módulo 3 – Treino}

\begin{enumerate}
\item

\item

\item
\end{enumerate}

\colorsec{Língua Portuguesa – Módulo 4 – Treino}

\begin{enumerate}
\item

\item

\item
\end{enumerate}

\colorsec{Língua Portuguesa – Módulo 5 – Treino}

\begin{enumerate}
\item

\item

\item
\end{enumerate}

\colorsec{Língua Portuguesa – Módulo 6 – Treino}

\begin{enumerate}
\item

\item

\item
\end{enumerate}

\colorsec{Língua Portuguesa – Módulo 7 – Treino}

\begin{enumerate}
\item

\item

\item
\end{enumerate}