\chapter{Respostas}
\pagestyle{plain}
\footnotesize

\pagecolor{gray!40}

\colorsec{Língua Portuguesa – Módulo 1 – Treino}

\begin{enumerate}
\item
(A) Correta. Na placa aparece apenas letras do alfabeto.
(B) Incorreta. Na placa aparece outros sinais gráficos como números e o @.
(C) Incorreta. Na placa aparece outros sinais gráficos como números \$ e \%.
(D) Incorreta. Na placa aparece outros sinais gráficos como números e o \&.

\item
(A) Está correta, pois a palavra começa com o som pa.
(B) Está incorreta, pois o ne não é o som inicial ele está no meio da palavra.
(C) Está incorreta, pois pode ter confundido os sons do pa com ga. .
(D) Está incorreta, pois essa é o som final.

\item
(A) Está incorreta, por considerar o som inicial te que também tem na palavra peteca.
(B) Está correta, pois a palavra termina com o mesmo som.
(C) Está incorreta. pois confundiu o som de da com ca.
(D) Está incorreta, pois confundiu o som medial com a final.
\end{enumerate}

\colorsec{Língua Portuguesa – Módulo 2 – Treino}

\begin{enumerate}
\item
(A) Está correta, pois a palavra começa com o som te
(B) Está incorreta, pois o ne confundiu a ordem do som das letras.
(C) Está incorreta, pois pode ter confundido os sons do te com pe
(D) Está incorreta, pois pode ter confundido os sons do te com le.

\item
(A) Está incorreta, pois essa palavra aparece no meio da frase.
(B) Está correta, pois a frase começa com essa palavra.
(C) Está incorreta, por confundido a frase final com a inicial.
(D) Está incorreta, pois não observou que essa palavra está no meio da frase.

\item
(A) Está incorreta, pois não se atentou a direção da leitura.
(B) Está incorreta, pois confundiu com primeira palavra da última frase.
(c) Está incorreta, pois confundiu com a última palavra de uma das frases.
(D) Está correta, pois o texto termina com essa palavra.
\end{enumerate}

\colorsec{Língua Portuguesa – Módulo 3 – Treino}

\begin{enumerate}
\item
(A) Está correta, pois Aa borboleta que gosta de luz é azul.
(B) Está incorreta, por acreditar que como são escuras gostam de luz.
(C) Está incorreta, por acreditar que as brancas gostam de luz por serem claras.
(D) Está incorreta, por acreditar que como a luz também é amarela essas gostam de luz.

\item
(A) Está correta, pois será usada 6 fatias de pães.
(B) Está incorreta, por acreditar que o queijo é usado em fatias.
(C) Está incorreta, pois confundiu a quantidade uma vez que o pepino também será usado em fatias.
(D) Está incorreta, por acreditar que a cenoura pode ser cortada em fatias.

\item
(A) Está incorreta, por acreditar que a foca bate palma se por uma bola
no nariz pois gosta de brincar.
(B) Está incorreta, por acreditar que se espetar a barriga ela sente
sossegas e bate palma.
(C) Está correta, pois a foca bate palmas se dar a ela uma sardinha.
(D) Está incorreta, por confundir briga com sardinha que pode se também
considerada uma brincadeira de bater.
\end{enumerate}

\colorsec{Língua Portuguesa – Módulo 4 – Treino}

\begin{enumerate}
\item

\item

\item
\end{enumerate}

\colorsec{Língua Portuguesa – Módulo 5 – Treino}

\begin{enumerate}
\item

\item

\item
\end{enumerate}

\colorsec{Língua Portuguesa – Módulo 6 – Treino}

\begin{enumerate}
\item

\item

\item
\end{enumerate}

\colorsec{Língua Portuguesa – Módulo 7 – Treino}

\begin{enumerate}
\item

\item

\item
\end{enumerate}