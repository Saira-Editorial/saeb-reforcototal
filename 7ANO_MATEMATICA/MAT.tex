%!TEX root=./LIVRO.tex

\chapter{Números racionais}
\markboth{Módulo 1}{}

\section{Habilidades do SAEB }
\begin{itemize}
\item Escrever números racionais (representação
fracionária ou decimal finita) em sua representação por algarismos ou em
língua materna ou associar o registro numérico ao registro em língua
materna.
\item
  Compor ou decompor números racionais positivos (representação decimal
  finita) na forma aditiva, ou em suas ordens, ou em adições e
  multiplicações.
\item
  Identificar números racionais ou irracionais.
\item
  Comparar ou ordenar números reais, com ou sem suporte da reta
  numérica, ou aproximar números reais para múltiplos de potência de 10
  mais próxima.
\item
  Converter uma representação de um número racional positivo para outra
  representação.
\item
  Identificar um número natural como primo, composto, ``múltiplo/fator
  de'' ou ``divisor de'' ou identificar a decomposição de um número
  natural em fatores primos ou relacionar as propriedades aritméticas
  (primo, composto, ``múltiplo/fator de'' ou ``divisor de'') de um
  número natural à sua decomposição em fatores primos.
\end{itemize}

\section{Habilidades da BNCC }
\begin{itemize}
\item EF07MA01, EF07MA03, EF07MA10
\end{itemize}

Professor, neste módulo é muito importante relembrar os alunos de cada
conjunto de números. Revisar desde os naturais e os inteiros para que
fique mais fácil de salientar as particularidades dos números racionais
e, principalmente, dos irracionais, que fogem um pouco da classe de
pertencimento dos outros conjuntos.

\section{Números naturais}

O conjunto dos números naturais é chamado de $N$. Números naturais são
todos os números que são positivos e possuem somente parte inteira. Os
números naturais têm algumas particularidades, como: um número é primo
quando possui apenas dois divisores, ele mesmo e o número $1$. O único
número primo que é par é o número dois. Números compostos são os números
que não são primos, isto é, podem ser divididos por mais de dois
números.

\subsection{Exemplos:}

\begin{itemize}
\item Números primos: $7$,$11$,$13$,$23$, etc.
\item Números compostos: $4$,$9$,$12$,$14$,$16$, etc.
\end{itemize}

\section{Números racionais}

Números racionais são números que podem ser escritos em forma de fração,
por exemplo:

$$\frac{a}{b}$$

desde que $a$ e $b$ sejam inteiros e $b$ diferente de zero.

O conjunto dos números racionais recebe o nome de $Q$. Os números que
podem ser escritos em fração estão divididos em quatro classes: as
próprias frações, que já estão no formato da definição de números
racionais, os números inteiros, que, ao serem divididos por $1$, resultam
no mesmo número, decimais finitos, que são números divididos por
potências de $10$ e as dízimas periódicas, que são decimais que têm um
período que se repete em sua parte decimal.

\subsection{Exemplos}

\textbf{Frações}: $\frac{7}{8}$, $\frac{5}{3}$, $\frac{4}{9}$ toda fração que tenha $a$,
$b$ inteiros e $b$ diferente de zero.

\textbf{Números inteiros}

 $$-\frac{9}{1},  \frac{4}{1}, -\frac{8}{1},  \frac{1057}{1}$$

\textbf{Números decimais finitos}: Qualquer número decimal finito pode ser
escrito na forma de fração, de modo que o denominador é uma potência de
$10$ elevada à quantidade de algarismos das casas decimais:

$$8,692  = \frac{8692}{10^{3}} = \frac{8692}{1000}$$

\textbf{Dízimas periódicas}: São dízimas que são infinitas, mas possuem um
período que se repete: 

$$\frac{23}{9}\  = \ 2,55555\ldots{}$$
$$\frac{22}{9} = 2,44444\ldots{}$$
$$\frac{478}{990} = 0,4828282\ldots{}.$$

Para redigir em português os números racionais, devemos relembrar do
sistema de numeração decimal. A parte inteira é separada e cada número
recebe um nome de acordo com sua posição.

\subsection{Exemplo:} 

%Paulo: Colocar cada número abaixo com uma cor diferente.

$$7,9853$$

$7$ = parte inteira. $9$ = décimos $8$ = centésimos $5$ = milésimos $3$ = décimos
de centésimos

Em todo número decimal, lemos primeiro a parte inteira, ou seja, 7
inteiros. Para os decimais, lê-se o número inteiro, indicando a última
classe, ou seja, nove mil oitocentos e cinquenta e três décimos de
milésimos.

\subsection{Exemplos:}

$0,26$ = vinte e seis centésimos;

$1,256$ = um inteiro e duzentos e cinquenta e seis milésimos;

$5,3632$ = cinco inteiros e três mil seiscentos e trinta e dois décimos de
milésimos.

Analisando os exemplos acima, a decomposição dos números racionais é feita de 
acordo com o sistema posicional, completando com zeros onde não encontramos 
números após a vírgula.

\subsection{Exemplos:}

$$0,26\  = \ 0,20 + 0,06$$
$$1,256 = \ 1 + 0,200 + 0,050 + 0,006$$
$$5,3632 = \ 5 + 0,30000 + 0,0600 + 0,0030 + 0,0002$$

\subsection{Números irracionais}

O conjunto dos números irracionais recebem o nome 𝕀. Números irracionais
são todos aqueles números que não podemos representar por frações.
Alguns exemplos são as dízimas que não possuem período e que se repetem,
e algumas raízes, como $\sqrt{2}$ \ e $\sqrt{3}$.

\subsection{Números reais}

O conjunto dos números reais recebe o nome de $R$. Os números reais são
todos os números naturais, inteiros, racionais e irracionais, excluindo
somente o conjunto dos números complexos. Para a representação e
comparação de números reais, convém transformá-los em decimais para
tornar sua visualização mais clara. Quando for necessário comparar a
parte decimal de um número, basta buscar a casa decimal que difere um
número do outro para que, assim, seja possível saber qual número é maior
e qual é menor.

\subsection{Exemplos:}

Vamos comparar 3 números decimais e colocá-los em ordem crescente.

$$33,785  - 33,784  - 33,659$$

A primeira questão a se analisar é que suas partes inteiras são iguais.
Devemos, então, comparar os números após a vírgula. Os décimos nos dois
primeiros números são iguais, mas o último é menor. Sabemos, então, que
o terceiro número é o menor deles. Passamos para os centésimos, que
também são iguais. Somente os milésimos são diferentes e, como o segundo
número é o menor entre os dois, a ordenação crescente é:

$$33,659  - 33,784  - 33,785$$

Como alternativa, podemos usar a aproximação para múltiplos de $10$. Esse
método arredonda o número para a dezena mais próxima. Quando o número é
maior ou igual a $5$, o arredondamento será feito para cima. Em todos os
outros casos, arredondamos para baixo.

\subsection{Exemplos:}

$58\cong 60$: Como o 58 está nas casas das dezenas, o
arredondamento mais próximo será uma dezena.

$459\cong 460$: Como o 459 está nas casas das centenas, o
arredondamento mais próximo será uma centena.

$7\cong 10$: Como o 7 está nas casas das unidades, o
arredondamento mais próximo será uma dezena, pois não há números nas
unidades que escritos em potências de 10. Isso ocorre pelo fato de esses
números serem escritos na potência 10, o que facilita visualizá-los em
uma reta numérica.

\section{Atividades}

\num{1} Marque Verdadeiro (V) ou falso (F) para as afirmações abaixo e
justifique sua resposta.


% boxlist :: options
\begin{boxlist}[%leftmargin=0em,
                %topsep=0pt,
                itemsep=-7pt,
                %leftmargin=0em
                ]
\setlength{\baselineskip}{-4ex}\boxitem{F} Todo número que é racional é também irracional.
{\rosa{Números irracionais não podem ser escritos em forma de fração, sendo
essa é a definição de números racionais.}}
\boxitem{F} As dízimas periódicas são irracionais.
\rosa{É possível encontrar uma fração para dízimas periódicas.}
\boxitem{V}\normalsize Dízimas periódicas podem ser escritas na forma de fração.
{\rosa{É possível encontrar uma fração geratriz.}\par}
\boxitem{V} Todos os números inteiros são racionais.
{\rosa{Todos os números inteiros podem ser escritos na forma de racionais.}\par}
\end{boxlist}

\num{2} Relacione as colunas abaixo com o número escrito por extenso.

%Paulo: criar uma tabela de duas colunas com as informações abaixo:

\begin{table}[h]
\centering\small
\begin{tabular}{ll}\toprule\midrule
0,333 & Trezentos e três milésimos                     \rosa{-- b}    \\
0,303 & Quatro inteiros e duzentos e cinquent e oito milésimos \rosa{-- c} \\
4,258 & Trezentos e trinta e três milésimos          \rosa{-- a}            \\
4,058 & Quatro inteiros e cinquenta e oito \rosa{-- d} \\\bottomrule
\end{tabular}
\end{table}

\num{3} Aproxime as partes inteiras dos números reais a potências de $10$.

\begin{multicols}{3}
\begin{enumerate}[itemsep=-4pt]
\item $1$.$039$ \rosa{-- $1$.$040$}

\item $6$ \rosa{-- $10$}

\item $782$ \rosa{-- $780$}

\item $24$ \rosa{-- $20$}

\item $3$.$563$ \rosa{-- $3$.$560$}

\item $3$  \rosa{-- $10$}
\end{enumerate}
\rosa{Para aproximar números para a potência de $10$, a parte inteira será analisada.}
\end{multicols}


\num{4} Foi feita uma competição na escola Super Saber. Durante 5 minutos, os
alunos tinham de percorrer a maior distância possível com obstáculos.
Constatou-se que a maioria percorreu a mesma distância, condicionando a
decisão aos números decimais. Observe a tabela com os resultados e
organize em um ranking decrescente de distâncias.

\begin{longtable}[]{@{}ll@{}}
\toprule
Nome do aluno~ & Distância percorrida\tabularnewline\midrule
Daniel~ & 302,006 m \rosa{6ºcolocado}\tabularnewline
Flávia~ & 303,01 m \rosa{4ºcolocado}\tabularnewline
Adriel~ & 304,05 m \rosa{1º colocado: maior distância}\tabularnewline
Larissa~ & 303,68 m \rosa{3ºcolocado}\tabularnewline
Luís Fabiano~ & 300 m~ \rosa{7º colocado: menor distância}\tabularnewline
Emanuelly~ & 302,072 m \rosa{5ºcolocado}\tabularnewline
Henrique~ & 303,809 m \rosa{2ºcolocado}\tabularnewline
\bottomrule
\end{longtable}

%Professor, lembre os alunos que o desempate acontece no maior número depois da vírgula que ocupa a mesma casa decimal do outro valor.

\num{5} Classifique os números abaixo como compostos ou primos. Indique 3
divisores para os números compostos.

\begin{escolha}
    
    \item $11$ \rosa{O número 11 é primo, só é divisível por ele mesmo e por 1.}

    \item $48$ \rosa{O número 48 é composto. Seus menores divisores são 2, 4 e 6.}

    \item $23$ \rosa{O número 23 é primo, só é divisível por ele mesmo e por 1.}

    \item $51$ \rosa{O número 51 é primo, só é divisível por ele mesmo e por 1.}

    \item $100$ \rosa{O número 100 é composto. Seus menores divisores são 2, 4 e 5.}

\end{escolha}

\num{6} Decomponha os números racionais abaixo:

\begin{escolha}

    \item 28,3569 \rosa{$28,3569 = 28 + 0,3000 + 0,0500 + 0,0060 + 0,0009$}

    \item 8,596 \rosa{$8,596 = 8 + 0,500 + 0,090 + 0,006$}
    
    \item 42,568 \rosa{$42,568 = 42 + 0,5 + 0,06 + 0,008$}
    
    \item 144,326 \rosa{$144,326 = 144 + 0,300 + 0,020 + 0,006$}

\end{escolha}

\num{7} Apresente uma representação equivalente ao número racional dado:

\begin{escolha}

    \item 5,667  \rosa{$5,667 = \frac{5667}{1000}$}

    \item $\frac{42}{99}$  \rosa{$\frac{42}{99} = 0,42424242...$}

    \item 0,36  \rosa{$0,36 = \frac{36}{100}$}

\end{escolha}

\num{8} Escreva o número decimal 0,625 como uma fração irredutível.

\reduline{$0,625 = \frac{625}{1000}= \frac{5}{8}$\hfill}

%\rosa{Para resolver este exercício, primeiro precisamos entender que um número
%decimal pode ser escrito como uma fração dividindo-se o valor decimal
%pelo valor do lugar decimal. No caso de $0,625$, o número está escrito na
%casa decimal de milésimos, logo, $0,625 = \frac{625}{1000}$.
%Simplificando a fração, temos $\frac{5}{8}$.}

\num{9} Classifique os números como racionais ou irracionais.

\begin{escolha}
    \item $7,3695216\ldots{}$
    \reduline{Irracional, pois não possui um período que se repete.\hfill}

    \item $1,28567676767\ldots{}$
    \reduline{Racional, porque é uma dízima periódica.\hfill}  
    
    \item $1,2365787878\ldots{}$
    \reduline{Racional, porque é uma dízima periódica.\hfill}
\end{escolha}

\num{10} Na sala da professora Ana Júlia, foi realizada uma atividade de
medição dos pés dos alunos. Thiago, Maria e Paula obtiveram,
respectivamente, as seguintes medidas: $23,9; 22,7 e 23,91$. Quem tem o
maior pé?

\reduline{Paula possui o maior pé, pois, como as casas centesimais são iguais,
devemos observar os milésimos.\hfill}

\section{Treino}

\num{1} Marque a alternativa em que os seguintes números são classificados
respectiva e corretamente. $\sqrt{3}$, $0,222\ldots{}$, $5,363636\ldots{}$ e $\sqrt{2}$:

\begin{escolha}
    
    \item Natural, Inteiro, Racional, Racional.
    
    \item Irracional, Racional, Racional, Irracional.
    
    \item Racional, Irracional, Irracional, Racional.
    
    \item Racional, Racional, Racional, Irracional.

\end{escolha}

%BNCC: EF07MA10 }
% -- Comparar e ordenar números racionais em diferentes
% contextos e associá-los a pontos da reta numérica..
% SAEB: Identificar números racionais ou irracionais.

% A - Incorreta, pois \sqrt{3}\\ não é Natural, 0,2222 não é Inteiro,
% \sqrt{2}\\ não é racional.
% B - Correta, pois os números foram classificados corretamente.
% C - Incorreta, pois \sqrt{3}\\ não é racional, 0,2222 não é
% irracional, \sqrt{2}\\ , 5,363636 não é irracional e \sqrt{2}\\
% não é racional.
% D - Incorreta, pois \sqrt{3}\\ não é racional.

\num{2} Um professor marcou numa reta numérica um ponto entre o $0$ e o $1$. Qual
número racional melhor representa este ponto?

\begin{escolha}
    \item $\frac{2}{8}$
    \item $\frac{5}{2}$
    \item $\frac{10}{7}$
    \item $\frac{18}{12}$
\end{escolha}

%BNCC: EF07MA03 }
% -- Comparar e ordenar números inteiros em diferentes
% contextos, incluindo o histórico, associá-los a pontos da reta numérica
% e utilizá-los em situações que envolvam adição e subtração.
% SAEB: Comparar ou ordenar números reais, com ou sem suporte da reta
% numérica, ou aproximar números reais para múltiplos de potência de 10
% mais próxima.

% A - correta, pois, ao efetuarmos a divisão, temos 0,25.
% B - incorreta, pois, ao efetuarmos a divisão, temos 2,5.
% C - incorreta, pois, ao efetuarmos a divisão, temos 1,42.
% D - incorreta, pois, ao efetuarmos a divisão, temos 1,5.

\num{3} Julgue as afirmações e marque a resposta correta.

\begin{itemize}
\item I: Todos os números inteiros são racionais.
\item II: Todo número decimal finito pode ser representado por fração.
\item III: O número 21 é primo.
\item IV: O número 2 é o único número que é par e primo.
\end{itemize}

\begin{escolha}
\item
  As afirmações II e IV são falsas.
\item
  I, II, e IV são verdadeiras.
\item
  Apenas IV é falsa.
\item
  São todas verdadeiras.
\end{escolha}

%BNCC: EF07MA10
% -- Comparar e ordenar números racionais em diferentes
% contextos e associá-los a pontos da reta numérica.
% SAEB: Converter uma representação de um número racional positivo para
% outra representação.

% A - Incorreta, pois todo número decimal finito pode ser representado por
% fração e o número 2 é o único par que é primo.
% B - Correta, pois essas são as afirmações certas.
% C - Incorreta, pois o número 2 é o único número par que é primo, então é
% uma verdade
% D - Incorreta, pois a afirmação III é falsa. O número 21 não é primo,
% ele possui 4 divisores: 1, 3,7 e 21.

\chapter{Operações aritméticas}
\markboth{Módulo 2}{}

\section{Habilidades do SAEB }

\begin{itemize}
\item Calcular o resultado de adições, subtrações,
multiplicações ou divisões envolvendo números reais.
\item
  Calcular o resultado de potenciação ou radiciação envolvendo números
  reais.
\item
  Resolver problemas de adição, subtração, multiplicação, divisão,
  potenciação ou radiciação envolvendo números reais, inclusive notação
  científica.
\item
  Resolver problemas de contagem cuja resolução envolva a aplicação do
  princípio multiplicativo.
\item
  Resolver problemas que envolvam as ideias de múltiplo, divisor, máximo
  divisor comum ou mínimo múltiplo comum.
\end{itemize}

\section{Habilidade da BNCC }
\begin{itemize}
\item EF07MA04
\end{itemize}

%Professor, tenha muita atenção ao trabalhar esse módulo com os alunos, pois trataremos das operações fundamentais envolvendo números positivos e negativos. Revise as operações de soma, subtração, multiplicação, divisão, potenciação e radiciação com números positivos. Em seguida, refaça todas elas com os números negativos. Enfatize bem as regras de sinais, distinguindo as regras da soma e multiplicação/divisão.

As operações de adição, subtração, multiplicação, potência e raiz, no
contexto dos números negativos, precisam ser realizadas com mais atenção
por envolverem as regras de sinais. Vamos relembrá-las?

\textbf{Soma}

Sinais iguais: somamos os algarismos e mantemos o sinal em comum no
resultado.

Sinais diferentes: subtraímos os algarismos e mantemos o sinal do maior
no resultado.

\textbf{Exemplo:} $$- 9 + \left( - 12 \right) = \  - 21\\
- 15 + \left( 8 \right) = \  - 7\\
- 4 + \left( 28 \right) = \  - 24\\$$

\textbf{Multiplicação/Divisão}

Sinais iguais: o sinal do resultado fica positivo.

Sinais diferentes: o sinal do resultado fica negativo.

\textbf{Exemplo:} $$- 10 \times \left( - 5 \right) = \ 50\\
- 7 \times \left( 8 \right) = \ - 56\\
- 42 \div \left( 3 \right) = - 14\\$$

\textbf{Observação:} Lembre-se de que a subtração tem a função de trocar
o sinal do número que vem após ela. Assim,

$$15 + 27 - \left( - 5 \right) = 15 + 27 + 5 = 47$$

\textbf{Potenciação} $$a^{n}$$

\rosa{Professor, use esse tópico para revisar as propriedades de potência que
já foram estudadas no 6º ano: multiplicação e divisão de bases iguais,
potência de potência, potência de expoente 0, 1, etc.}

\textbf{Base negativa:} aplicamos a regra do expoente, ou seja, quando o
expoente for par o resultado fica positivo, e quando for ímpar fica
negativo.

\textbf{\hfill\break
Exemplo:} $$\left( - 3 \right)^{2} = 9\\
\left( - 3 \right)^{3} = -27\\$$

\textbf{Observação:} para a base ser considerada negativa e aplicarmos
as regras de sinais, ela deve aparecer entre parênteses. Caso contrário,
apenas repetimos o sinal e calculamos a potência separadamente.

$$\left( - 4 \right)^{2} \neq - 4^{2}$$

$$16 \neq - 16$$
\textbf{Raiz quadrada} 
$$\sqrt{a}$$

%Professor, use esse tópico para revisar a fatoração. Revise os conceitos
%de múltiplos, divisores, números primos, m.m.c. e m.d.c.

Para resolver uma raiz quadrada, podemos utilizar três técnicas:
tentativa e erro, fatoração ou aproximação.

%Professor, faça a fatoração e explique para os alunos que agrupamos os
%termos de acordo com o índice da raiz para retirarmos os números.

\textbf{Exemplo}
$$\sqrt{576} = \sqrt{2^{2}\cdot2^{2}\cdot2^{2}\cdot3^2} = 2.2.2.3 = 24$$
$$\sqrt{7} \rightarrow \ \sqrt{4} < \sqrt{7} < \sqrt{9} \rightarrow 2 < \sqrt{7} < 3$$

\section{Atividades}

\num{1} Calcule o valor das expressões a seguir: TESTE

\begin{enumerate}

    \item $- 15 + 27 - (- 5) + 75 = $ 
    \reduline{$- 15 + 27 + 5 + 75 = 92$}

    \item $37 + (- 11) - (66) - ( - 8) =$
    \reduline{$37 - 11 - 66 + 8 = - 32$}

    \item $- (22 + ( - 33 ) - 14 + 120) =$
    \reduline{$- ( 22 - 33 - 14 + 120 ) = - ( 95) = - 95$}

    \item $200 - 88 - 14 - 144 + 75 =$
    \reduline{$ 200- 88 - 14 - 144 + 75 = 29$}

\end{enumerate}

% Professor, ao resolver os exercícios com os alunos, enfatize as regras,
% faça o passo a passo onde for preciso retirar parênteses e trabalhe a
% resolução de dois em dois números, deixando bem claros os resultados
% encontrados.

\num{2} Qual é o valor da expressão $3^4 \times \sqrt{16}$?

\reduline{Primeiro, vamos calcular a raiz quadrada de $16$, que é $4$. Substituindo
esse valor na expressão, temos: $3^4 \times \sqrt{16} = 3^4 \times 4 = 81 \times 4 = 324$. 
Portanto, o resultado da expressão é 324.}

\num{3} Efetue as operações abaixo:

\begin{escolha}

\item $- 2 \times 5 \times ( - 3 ) \times ( - 1 ) =$ \rosa{$ - 30$}
\item $(- 4 \times 8) \div ( - 2 \times ( - 8 )) =$ \rosa{( $- 32) \div 16 = - 2$}
\item $12 \times 3 \times ( - 3 ) \times 2 =$ \rosa{$-216$}
\item $( - 50 \div ( - 5 )) \times (36 \div 9) =$ \rosa{$( 10 ) \times 4 = 40$}

\end{escolha}

\num{4} Aplicando as propriedades de bases iguais, simplifique como uma única
potência e indique se seus resultados serão positivos ou negativos:

\begin{escolha}

    \item ${( - 5)}^{10} \times ( - {5})^{8} \times {( - 5)}^{3} =$ \rosa{${( - 5)}^{10} \times - {5}^{8} \times ( - 5)^{3} = ( - 5)^{10 + 8 + 3} = ( - 5 )^{21}$ Como o expoente é ímpar o resultado fica negativo}

    \item ${( - 2)}^{4} \times ( - {2}^{6} \times {( - 2)}^{12} =$ \rosa{${( - 2)}^{4} \times ( - {2}^{6} \times \left( - 2 \right)^{12} = \ \left( - 2 \right)^{4 + 6 + 12} = \left( - 2 \right)^{22}$ Como o expoente é par o resultado fica positivo.}

    \item $\frac{{( - 7)}^{15}}{{( - 7)}^{11}} =$ \rosa{$\frac{{( - 7)}^{15}}{{( - 7)}^{11}} =
{( - 7)}^{4}$  Como o expoente é par, o resultado fica
positivo.}

    \item $\frac{{( - 3)}^{33}}{{( - 3)}^{12}} =$ \rosa{$\frac{{( - 3)}^{33}}{{( - 3)}^{12}} =
{( - 3)}^{21}$  Como o expoente é ímpar, o resultado
fica negativo.}

    \item $\frac{{( - 10)}^{55} \times {( - 10)}^{25} \times {( - 10)}^{35}}{{( - 10)}^{4} \times {( - 10)}^{6} \times {( - 10)}^{32}} =$ \rosa{$\frac{{( - 10)}^{55} \times {( - 10)}^{25} \times {( - 10)}^{35}}{{( - 10)}^{4} \times {( - 10)}^{6} \times {( - 10)}^{32}} = \frac{{( - 10)}^{55 + 25 + 35}}{{( - 10)}^{4 + 6 + 32}} = \frac{{( - 10)}^{115}}{{( - 10)}^{42}} = {( - 10)}^{115 - 42} = {( - 10)}^{73}$ \rightarrow Como o expoente é ímpar, o resultado fica negativo.}

\end{escolha}

\num{5} Observe o exemplo a seguir. 

$$A = \sqrt{225} = \sqrt{3^{2} \times 5^{2}} = 3 \times 5 = 15$$

$$B = \sqrt{324} = \sqrt{2² \times 3² \times 3²} = 2 \times 3 \times 3 = 18$$

$$C = \sqrt{144} = \sqrt{2² \times 2² \times 3²} = 2 \times 2 \times 3 = 12$$

$$C - A - B$$

Como você pôde verificar, a cada letra foi associado o valor de uma raiz quadrada; depois, calculou-se
o valor delas usando a fatoração e, ao final, elas foram colocadas em ordem crescente. Faça a mesma
coisa com as raízes quadradas a seguir. \\

$A = \sqrt{441}$ \rosa{$A = \sqrt{441} = \sqrt{3^{2}.7²} = 3.\ 7 = 21$} \\

$B = \sqrt{784}$ \rosa {$B = \sqrt{784} = \sqrt{2^{2}.2^{2}.7²} = 2\ .\ 2\ .\ 7 = 28$} \\

$C = \sqrt{1225}$ \rosa{$C = \sqrt{1225} = \sqrt{5².7²} = 5\ .\ 7 = 35$} \\

$D = \sqrt{676}$ \rosa{$D = \sqrt{676} = \sqrt{2^{2}.\ 13²} = 2\ .\ 13\  = 26$} \\

$E = \sqrt{484}$ \rosa{$E = \sqrt{484} = \sqrt{2².11²} = 2\ .\ 11 = 22$} \\

\rosa {$A - E - D - B - C$}

\num{6} Indique entre quais números inteiros encontram-se as raízes abaixo,
utilizando as raízes exatas menores e maiores. Siga o exemplo:

$$\sqrt{23} \rightarrow \ \sqrt{16} < \sqrt{23} < \sqrt{25} \rightarrow \
\sqrt{23}$$ {portanto, $\sqrt{23}$ está entre 4 e 5}

\begin{escolha}
    \item $\sqrt{55}$   \rosa{$\sqrt{55} \rightarrow \ \sqrt{49} < \sqrt{55} < \sqrt{64} \rightarrow \sqrt{55}$ está entre 7 e 8}

    \item $\sqrt{66}$ \rosa{$\sqrt{66} \rightarrow \ \sqrt{64} < \sqrt{66} < \sqrt{81} \rightarrow \sqrt{66}$ está entre 8 e 9}

    \item $\sqrt{96}$ \rosa{$\sqrt{96} \rightarrow \ \sqrt{81} < \sqrt{96} < \sqrt{100} \rightarrow \sqrt{96}$ está entre 9 e 10}

    \item $\sqrt{42}$ \rosa{$\sqrt{42} \rightarrow \ \sqrt{36} < \sqrt{42} < \sqrt{49} \rightarrow \sqrt{42}$ está entre 6 e 7}
\end{escolha}


% Professor, faça com os alunos todos os produtos de fatores iguais de 1
% até 10 para facilitar a visualização das raízes a serem utilizadas.

\num{7} Pietra, Marcelo e Antônio decidiram criar um jogo para estudar para a
prova de matemática. Quem montasse a expressão com maior resultado
negativo venceria. Determine qual deles foi o vencedor: \\

Pietra:
$2^{5} + \{ 75 \div ( - 3 ) - \lbrack 6^{2} + ( 2 - 4 )^{3} \rbrack \} - \sqrt{16}$
\rosa{$ = 32 + \{ - 25 - \lbrack 36 + ( - 2 )^{3} \rbrack \} - 4 = 32 + \{ - 25 - \lbrack 36 - 8 \rbrack \} - 4 = 32 + \{ - 25 - 28 \} - 4 = 32 - 53 - 4 = - 25$} \\

Marcelo:
$230^{0} + ( - 5)^2 \{ 15 \div 5 \times \lbrack 7^{2} - ( 12 - 8 )^{3} \rbrack \} - \sqrt{121}$
\rosa{$ = 1 + 25 - \{ 3 \times \lbrack 49 - 64 \rbrack \} - 11 = 26 - \{ 3 \times ( - 15 ) \} - 11 = 26 - \{ - 45 \} - 11 = 26 + 45 - 11 = 60$} \\

Antônio:
$1^{500} \times \{ - 5 \times \lbrack ( - 9 )^{2} - ( 34 - 26 )^{2} \rbrack \} - (\sqrt{100}\  \times 700^{0})$
\rosa{$ = 1 \times \{ - 5 \times \lbrack 81 - 64 \rbrack \} - ( 10 \times 1 ) = 1 \times \{ - 5 \times 17 \} - 10 = 1 \times ( - 85 ) - 10 = - 85 - 10 = - 95$ \\ \\
Logo, quem venceu o jogo foi a Pietra, já que o maior resultado negativo
foi o dela.}

% Professor, enfatize a ordem de resoluções das expressões numéricas,
% estabelecendo as hierarquias de resoluções dos sinais de agrupamentos e
% das operações.

% Professor, revise com os alunos que a comparação dos números negativos
% funciona ao contrário dos positivos, que quanto mais longe do zero o
% número está, menor é seu valor.

\num{8} Para montar um bolo, Fabiana utiliza 2 tipos de massa, 16 tipos de
recheio, 8 tipos de cobertura e 4 tamanhos distintos, tanto para a parte
retangular quanto para a redonda. Escreva em forma de potência a
quantidade de bolos distintos que Fabiana consegue produzir combinando
as opções descritas e calcule a potência.

\reduline{Para calcular a quantidade de bolos, vamos usar o Princípio
Multiplicativo. Basta multiplicarmos todas as opções, logo:
$2 \times 16 \times 8 \times 4 \times 2$
Como devemos representar o resultado como uma única potência, vamos
transformar todos os termos para uma potência de base 2:
$2 \times 2^{4} \times 2^{3} \times 2^{2} \times 2 = 2^{1 + 4 + 3 + 2 + 1} = 2^{11} = 2048$ bolos distintos. 
\hfill}

\num{9} Em um terminal urbano, três ônibus saem às 7h com destino a bairros
distintos. De tempos em tempos, eles retornam ao terminal para refazer o
trajeto. O ônibus A retorna ao terminal de 20 em 20 minutos, o ônibus B
de 30 em 30 minutos e o ônibus C, de 45 em 45 minutos. Qual é o último
horário do dia em que os três ônibus saem da rodoviária ao mesmo tempo,
sabendo que os ônibus saem a última vez do terminal às 23h?

% Professor, mostre aos alunos que, quando queremos determinar encontros
% repetidos, usamos a ideia de Mínimo Múltiplo Comum. Revise com eles o
% cálculo pela fatoração simultânea.

\reduline{Para resolver, basta calcularmos o M.M.C dos tempos, ou seja, de 20, 30
e 45, achando de quanto em quanto tempo os ônibus vão se encontrar no
terminal. Depois, determinamos o último horário em que esse encontro
ocorrerá: 

% %Paulo: criar uma tabela com as informações abaixo: --------
% ---- ---- --- 20 30 45 2 10 15 45 2 5 15 45 3 5 5 15 3 5 5 5 5 1 1
% 1\\
% -------- ---- ---- ---

Ou seja, os ônibus se encontram de 180 em 180 minutos, isto é, 3 em 3
horas. Assim, se eles saem juntos às 7 da manhã, eles irão se encontrar
às 10h, 13h, 16h, 19h e 22h; logo, o último horário em que eles saem
juntos é às 22h.\hfill}

\num{10} Em um ateliê, existem três tecidos iguais, mas de estampas
diferentes, que são vendidos no mesmo tamanho. Em uma semana, o ateliê
recebeu rolos retangulares de 150 cm² da estampa azul, 120 cm² da
estampa vermelha e 180 cm² da estampa rosa. Para obtermos o maior
tamanho possível de retalhos sem sobrar nenhuma parte dos rolos
recebidos, quais dimensões de tecidos deverão ser cortadas? Quantos
retalhos foram obtidos de cada cor do tecido?

% Professor, mostre aos alunos que, quando necessitamos fazer divisões da
% maior forma possível, usamos a ideia de Máximo Divisor Comum. Revise com
% eles o cálculo pela fatoração simultânea.

\begin{comment}

\reduline{Para resolver, basta calcularmos o M.D.C dos tamanhos dos tecidos, ou
seja, de 120, 150 e 180, achando assim o tamanho do retalho e em seguida
quantos retalhos cada cor renderá. Assim:

\begin{longtable}[]{@{}llll@{}}
\toprule
%\endhead
120 & 150 & 180 & 2 \(\leftarrow\)\tabularnewline
60 & 75 & 90 & 2\tabularnewline
30 & 75 & 45 & 2\tabularnewline
15 & 75 & 45 & 3 \(\leftarrow\)\tabularnewline
5 & 25 & 15 & 3\tabularnewline
5 & 25 & 5 & 5 \(\leftarrow\)\tabularnewline
1 & 5 & 1 & 5\tabularnewline
1 & 1 & 1 &\tabularnewline
\bottomrule
\end{longtable}
\end{comment}

Assim, cada retalho vai ter 30cm² de tamanho e serão produzidos:

$\frac{150}{30}$ = 5 retalhos azuis

$\frac{120}{30}$ = 4 retalhos vermelhos

$\frac{180}{30}$ = 6 retalhos rosas \hfill

\section{Treino}

\num{1} Os fusos horários são feitos a partir do Meridiano de Greenwich, que
representa o marco 0. Toda localidade possui fuso negativo quando seu
horário é atrasado em relação ao meridiano. Por sua vez, o fuso é
considerado positivo quando é adiantado a partir do marco 0. Pedro vai
fazer uma viagem saindo de São Paulo, com fuso GMT-3, às 8 horas do dia
23/03 e vai para a capital da Austrália, com fuso GMT+11. Porém, o voo
de Pedro vai fazer uma escala em Dubai, local cujo fuso é GMT+4. Se a
viagem do Brasil até Dubai é de aproximadamente 14 horas e de Dubai até
a Austrália é de aproximadamente 18 horas, quando Pedro chegará na
Austrália?

\begin{escolha}

    \item 16h do dia 24/03.

    \item 6h do dia 24/03.

    \item 6h do dia 25/03.

    \item 16h do dia 25/03.

\end{escolha}


%%\section{BNCC: EF07MA04 }
% -- Resolver e elaborar problemas que envolvam operações
% com números inteiros.
% SAEB: Resolver problemas de adição, subtração, multiplicação, divisão,
% potenciação ou radiciação envolvendo números reais, inclusive notação
% científica.
%A - Incorreta, pois não foi considerada a diferença de fuso horário,
%apenas a soma das horas de maneira direta.
%B - Incorreta, pois o cálculo das horas foi feito da maneira correta,
%porém a mudança de dia não foi considerada.
%C - Correta, pois: Saída de São Paulo as 8h do dia 23/03 \rightarrow em Dubai vão ser
%15h do dia 23/03 (4 - ( - 3) = 4 + 3 = 7\text{h\ }de diferença). A viagem de São Paulo para Dubai dura 14 horas, assim Pedro vai chegar
%em Dubai às 5h do dia 24/03, que serão 12h do dia 24/03 na Austrália
%(11 - 4 = 7\text{h\ }de diferença). A viagem de Dubai até a Austrália são 18 horas, assim Pedro vai chegar
%na Austrália às 6h do dia 25/03.
%D - Incorreta, pois o dia está correto, mas foi considerado o cálculo
%sem a diferença de fuso horário.

\num{2} Quantos números de quatro algarismos diferentes podem ser formados
utilizando-se os algarismos 1, 2, 3, 4 e 5?

\begin{escolha}
    \item 24
    \item 60
    \item 120
    \item 240
\end{escolha}

%%\section{BNCC: EF07MA04 }
% -- Resolver e elaborar problemas que envolvam operações
% com números inteiros.
% SAEB: Resolver problemas de contagem cuja resolução envolva a aplicação
% do princípio multiplicativo.

% A - Incorreta, pois essa resposta não considera o fato de que cada
% algarismo deve ser diferente dos demais.
% B - Incorreta, pois essa resposta não leva em conta que a quantidade de
% opções vai diminuindo a cada algarismo escolhido.
% C - Correta, pois a quantidade de números de quatro algarismos
% diferentes que podem ser formados utilizando-se cinco algarismos é
% determinada pelo princípio multiplicativo. Portanto, o número total de
% combinações possíveis é dado pelo produto das opções disponíveis para
% cada algarismo: \ 5 \times 4\times 3\times 2  = 120
% D - Incorreta, pois essa resposta excede o número máximo de combinações
% possíveis, uma vez que estamos limitados a quatro algarismos diferentes.

\num{3} Mateus está comprando um lote para construir uma nova sede para sua
empresa. O lote tem formato quadrangular de área igual a 1296 m². Antes
de iniciar a construção, Mateus vai murar o terreno por todo seu
perímetro e colocar um portão de 12 metros de comprimento. Qual é a
extensão do muro que será construído no lote de Mateus?

\begin{escolha}
  \item 24 m
  \item 36 m
  \item 144 m
  \item 132 m
\end{escolha}

%%\section{BNCC: EF07MA04 }
% -- Resolver e elaborar problemas que envolvam operações
% com números inteiros.
% SAEB: Resolver problemas de adição, subtração, multiplicação, divisão,
% potenciação ou radiciação envolvendo números reais, inclusive notação
% científica.

% A - Incorreta, pois considerou apenas o tamanho de um lado do terreno e
% ignorou o comprimento do portão.
% B - Incorreta, pois considerou apenas o tamanho de um lado do terreno.
% C - Incorreta, pois não considerou o comprimento do portão do perímetro
% do terreno.
% D - Correta, pois, como a área de um quadrado é dado por l² - onde l é o
% tamanho do lado -, para calcular o tamanho do lado basta extrair a raiz
% quadrada da área. Assim: l^{2} = 1296 \rightarrow l = \ \sqrt{1296} \rightarrow l = \sqrt{2².2^{2}.3^{2}.3^{2}} = 2 \times 2 \times 3 \times 3 = 36m\ 

%Como o exercício pede a extensão do muro, basta calcular o perímetro do terreno e tirar o tamanho do portão. Assim:
%p\  = \ 4 \times 36 = 144 - 12 = 132

\chapter{Frações}
\markboth{Módulo 3}{}

\section{Habilidades do SAEB }
\begin{itemize}
\item Representar frações menores ou maiores que a
unidade por meio de representações pictóricas ou associar frações a
representações pictóricas.
\item
  Identificar frações equivalentes.
\item
  Determinar uma fração geratriz para uma dízima periódica.
\end{itemize}

\section{Habilidades da BNCC }
\begin{itemize}
\item EF07MA08, EF07MA09.
\end{itemize}

% Professor, neste módulo é importante ter objetos concretos e material
% dourado a disposição para trabalhar as frações. Estimule os alunos a
% pensar não só nos números, mas também em representações reais. Os alunos
% podem dividir alimentos (como pizzas e chocolates) entre si e enxergar
% as partes como frações.

\textbf{Frações próprias e impróprias}

Como já vimos antes, uma fração é representada pela sua parte no
numerador e seu inteiro no denominador. Ou seja, fração é uma
representação numérica de partes de um valor inteiro dividido em
parcelas iguais. Elas recebem o nome de frações próprias quando o
numerador é maior que o denominador.

\textbf{Exemplos}

Joana tem 6 batons. 5 deles são vermelhos e 1 é nude. Qual é a
representação fracionária do batom nude?

$$\frac{1}{6}$$

O numerador é a quantidade de batons de que estou falando. Já o
denominador representa a quantidade total.

Além disso, há representações que são maiores que as parcelas divididas.
Elas são chamadas de frações mistas ou impróprias. Elas são divididas em
partes inteiras e partes não inteiras. Isso acontece quando o numerador
é maior que o denominador. Nesses casos, é possível extrair uma parte
inteira da fração. A notação utilizada apresenta, da esquerda para a
direita, a parte inteira na frente do número e sua forma fracionária.

Professor, relembre o que é uma fração inteira, mencionando que toda
fração é uma divisão. Quando seus numeradores são iguais ou superiores
aos seus denominadores, já temos um inteiro.

\textbf{Exemplos:}

  Carlos foi a um rodízio de pizzas e comeu uma pizza e metade de outra.
  Se cada pizza fosse cortada em 8 pedaços, qual seria a sua
  representação fracionária?

$$\frac{8}{8}  + \frac{4}{8}  =  1 \frac{4}{8}$$

%\frac{29}{8} ou 3\ \frac{5}{8}

\textbf{Frações Equivalentes}

Frações equivalentes são frações escritas de forma diferentes, mas com o
mesmo valor numérico. No sexto ano, vimos o que são frações
irredutíveis, isto é, uma fração com numerador e denominador
apresentados no menor número possível. Para frações equivalentes,
podemos encontrar novas frações, dividindo ou multiplicando o numerador
e o denominador pelo mesmo número. Dessa forma, encontramos uma nova
razão para uma mesma proporção. Assim como ocorre em qualquer sequência
de números, quando comparamos duas frações, a relação só pode ser maior,
menor ou igual. Logo, podemos encontrar frações equivalentes.

Professor, nesta parte, vale lembrar os critérios de divisibilidade
mostrando como o numerador e o denominador devem ser divididos pelo
mesmo número. Durante os exercícios, convém utilizar a calculadora com
os alunos para que eles visualizem como as frações equivalentes chegam
ao mesmo quociente.

\textbf{Exemplos}

  Fábio e Michel foram a uma fábrica de chocolates. Fábio comprou um
  chocolate que era repartido em 12 quadradinhos e comeu 4 desses 12
  quadradinhos. Já Michel comprou um chocolate que vinha cortado em 36
  quadradinhos e comeu 12 quadradinhos. Considerando que as barras eram
  do mesmo tamanho, quem comeu mais chocolate?

Fábio comeu uma quantidade referente a $\frac{4}{12}$ e Michel uma
quantidade referente a $\frac{12}{36}$.

Note que é possível encontrar um padrão nos numeradores e denominadores,
pois a quantidade de que Michel comeu é igual ao triplo do numerador de
Fábio dividido pelo triplo de seu denominador, permanecendo com a mesma
razão.

Para cada fração abaixo, encontraremos duas de suas equivalentes:

$\frac{3}{7}$: Como temos dois números primos, utilizaremos seus
múltiplos, primeiro multiplicando por $2$, obtendo $\frac{6}{14}$.
Multiplicando por $3$, obtemos $\frac{9}{21}$.

$\frac{18}{64}$: Como temos dois números pares, sabemos que eles são
divisíveis por $2$, logo, temos $\frac{9}{32}$, o que resultou na fração
irredutível, logo, podemos encontrar outra fração multiplicando a
inicial por $2$, chegando a $\frac{36}{128}$.

$\frac{15}{75}$: Como temos dois números terminados em $5$, a primeira
fração equivalente será encontrada por meio da divisão por $5$, o que
resulta em $\frac{3}{15}$. Podemos encontrar outra fração dividindo
ambos por $3$, resultando em $\frac{5}{25}$.

\textbf{Observação}: É importante ressaltar, que da mesma forma que
existem infinitos números múltiplos, existem infinitas frações
equivalentes.

\textbf{Dízimas periódicas}

%Professor, nesta seção, é importante trabalhar primeiro a demonstração para encontrar frações geratrizes e, depois, o método prático para que os alunos entendam o que está acontecendo. É preciso ressaltar que as dízimas periódicas são números racionais e, por isso, têm representação em forma de fração.

Alguns números, quando divididos, resultam em algarismos que se repetem
infinitamente na mesma ordem. Essa repetição de números infinita é
chamada de período, o que explica o nome de dízima periódica. Existem
dois tipos de dízimas periódicas, as simples e compostas. As dízimas
periódicas simples apresentam um único número que se repete
indefinidamente. Já as dízimas periódicas compostas apresentam mais
números no trecho que está se repetindo.

\textbf{Exemplos}

\begin{itemize}

\item
  Dízimas periódicas simples: $0,6666\ldots$; $0,222222\ldots$;
  $0,77777\ldots$; $0,88888\ldots{}$

\item
  Dízimas periódicas compostas: $0,855555\ldots$; $0,47888888\ldots$,
  $0,985626262\ldots{}$

\end{itemize}

\textbf{Frações geratrizes}

Vamos fazer um passo a passo para encontrar a fração geratriz de dízimas
periódicas simples. Depois, aprenderemos o método prático.

\textbf{Método convencional:}

\begin{itemize}
\item
  Primeiro passo: Chamar a dízima periódica de x;
\item
  Segundo passo: Multiplicar os dois lados da igualdade por potências de
  10 a partir da quantidade de algarismos que há no período. Por
  exemplo, se temos dois números, multiplicremos por 100;
\item
  Terceiro passo: Encontrar a diferença (subtração) entre a equação
  encontrada e a equação do primeiro passo.
\item
  Quarto passo: encontrar o valor de x na equação do terceiro passo.
\end{itemize}

\textbf{Exemplo}

Vamos encontrar a fração geratriz da dízima $1,5555\ldots{}$

Primeiro passo: $x  = 1,5555\ldots$

Segundo passo: Multiplicar por 10, uma vez que só o número 5 se repete.

$$10x  =  1,5555\ldots\  \times \ 10$$ 

$$10x =  15,5555\ \ldots$$

Terceiro passo: Diferença entre as equações

$$10 x – x  =  15,5555 \ldots – 1,5555\ldots$$

$$9x  =  14$$

Quarto passo: $$\frac{14}{9}$$

\textbf{Método prático:}

\begin{itemize}
\item
  Primeiro passo: Separar a parte inteira da parte decimal;
\item
  Segundo passo: Encontrar o numerador da fração geratriz, que será dado
  pelos números da parte inteira até o período (sem a vírgula);
\item
  Terceiro passo: Encontrar o denominador da fração geratriz. Isso
  dependerá da quantidade de algarismos do período. Por exemplo, caso
  haja apenas um número, colocaremos um 9 no denominador.
\end{itemize}

\textbf{Exemplo}

Vamos encontrar a fração geratriz da dízima $1,5555\ldots{}$

Primeiro passo:

Parte inteira: $1$

Período: $5$

Segundo passo: $15 - 1 = 14$

Terceiro passo: Somente o $5$ está se repetindo, logo, no denominador
encontraremos somente um $9$.

Logo, a fração geratriz é $\frac{14}{9}$.

Agora, veremos como encontrar dízimas compostas pelos dois métodos:

\textbf{Método convencional:}

\begin{itemize}
\item
  Primeiro passo: igualar a dízima periódica a x.
\item
  Segundo passo: Multiplicar a dízima periódica composta por potências
  de 10 de modo que o antiperíodo fique antes da vírgula, isto é,
  multiplicar por 10 se encontrarmos apenas um algarismo no antiperíodo;
\item
  Terceiro passo: A dízima periódica está igual a uma dízima simples.
  Agora, basta multiplicar os dois lados da igualdade por potências de
  10, dependendo da quantidade de algarismos que há no período. Por
  exemplo, se temos dois números, multiplicaremos por 100;
\item
  Quarto passo: Encontrar a diferença (subtração) entre a equação
  encontrada no terceiro e no segundo passo.
\end{itemize}

\textbf{Método prático:}

\begin{itemize}
\item
  Primeiro passo: Separar a parte inteira, o antiperíodo e o período da
  dízima;
\item
  Segundo passo: Encontrar o numerador da fração geratriz, que será dado
  pelos números da parte inteira e do antiperíodo até o período .
\item
  Terceiro passo: Encontrar o denominador da fração geratriz, isso
  dependerá da quantidade de algarismos do período. Se encontrarmos 1
  número, colocaremos um 9 no denominador. Para cada algarismo do
  antiperíodo, acrescentamos um 0 no denominador.
\end{itemize}

\textbf{Exemplo:}

Vamos encontrar a fração geratriz da dízima 1,745555\ldots{}

Primeiro passo:

Parte inteira: $1$

Antiperíodo: $74$

Período: $5$

Segundo passo: $1745 - 174 = 1571$

Terceiro passo: Como só há um algarismo no período, no denominador,
haverá só um 9. Como há dois antiperíodos, haverá dois 0. Logo, a fração
encontrada é: $\frac{1571}{900}$.

\textbf{Observação:} também há casos em que não encontramos parte
inteira em uma dízima periódica. Quando isso ocorre, utilizamos
diretamente a regra dos 9 e dos zeros. O próprio período é o numerador.

\section{Atividades}

\num{1} Uma sala de aula tem 30 alunos, dos quais 20 são meninas. Qual das
alternativas representa a razão de meninas para o total de alunos na
sala como uma fração?

\reduline{A razão de meninas para o total de alunos na sala é expressa pela fração
que representa a parte em relação ao todo. Neste caso, temos 20 meninas
em um total de $30$ alunos. Portanto, a razão de meninas para o total de alunos é de $20/30$. Essa
fração pode ser simplificada dividindo ambos os números por $10$,
resultando em $2/3$.}

\num{2} Associe as colunas com as suas frações equivalentes:

\begin{table}[]
\begin{tabular}{c|c|c|c|}
\cline{2-2} \cline{4-4}
 & \textbf{Fração} & \textbf{} & \textbf{Fração equivalente} \\ \hline
\multicolumn{1}{|c|}{\textbf{A}} & $\frac{45}{104}$ & \textbf{I} & $\frac{40}{120}$ \\ \hline
\multicolumn{1}{|c|}{\textbf{B}} & $\frac{27}{144}$ & \textbf{II} & $\frac{90}{208}$ \\ \hline
\multicolumn{1}{|c|}{\textbf{C}} & $\frac{2}{8}$ & \textbf{III} & $\frac{14}{200}$ \\ \hline
\multicolumn{1}{|c|}{\textbf{D}} & $\frac{7}{100}$ & \textbf{IV} & $\frac{1}{4}$ \\ \hline
\multicolumn{1}{|c|}{\textbf{E}} & $\frac{1}{2}$ & \textbf{V} & $\frac{3}{16}$ \\ \hline
\multicolumn{1}{|c|}{\textbf{F}} & $\frac{1}{3}$ & \textbf{VI} & $\frac{35}{70}$ \\ \hline
\end{tabular}
\end{table}

\begin{comment} se a tabela acima não funcionar, as respostas são as seguintes

\begin{escolha}
a) $\frac{45}{104}$
b) $\frac{27}{144}$
c) $\frac{2}{8}$
d) $\frac{7}{100}$
e) $\frac{1}{2}$
f) $\frac{1}{3}$
\end{escolha}

\begin{enumerate}
\item \frac{40}{120}
\item $\frac{90}{208}$
\item $\frac{14}{200}$
\item $\frac{1}{4}$
\item $\frac{3}{16}$
\item $\frac{35}{70}$
\end{enumerate}
\end{comment}

\reduline{A-II, B-V, C-IV, D-III, E-VI, F-I. A: O numerador e denominador estão multiplicados por $2$, logo,
$\frac{90}{208}$. B. O numerador e denominador estão divididos por $9$, logo, $\frac{3}{16}$. C. O numerador 
e denominador estão divididos por $2$, logo, $\frac{1}{4}$. D. O numerador e denominador estão multiplicados 
por $2$, logo, $\frac{14}{200}$. E. O numerador e denominador estão multiplicados por $35$, logo 
$\frac{35}{70}.$ F. O numerador e denominador estão multiplicados por $40$, logo, $\frac{40}{120}$.\hfill}

\num{3} Encontre as frações geratrizes das dízimas periódicas pelo método
prático:

\begin{escolha}
  \item $1,89555555\ldots$
  \reduline{Como é uma dízima composta, devemos considerar o período e antiperíodo. Primeiro passo:
Parte inteira: $1$. Antiperíodo: $89$. Período: $5$. Segundo passo: $1895 - 189 = 1706$. Terceiro passo: 
O período é só um número (9) e o antiperíodo são dois
números ($900$), logo $\frac{1706}{900}$.\hfill}

  \item $1,44444\ldots$
  \reduline{Essa dízima é simples, então vamos usar o método prático para dízima
simples:

Primeiro passo:

Parte inteira: $1$

Período: $4$

Segundo passo: $14 - 1 = \ 13$

Terceiro passo: Como o período contém somente um número, o denominador
ficará apenas com um $9$, logo $\frac{13}{9}$.}

  \item $0,222222\ldots$
  \reduline{Basta colocar o período no numerador e um $9$ no denominador, logo
$\frac{2}{9}$.}

\end{escolha}

\num{4} Pinte o equivalente à fração $\frac{1}{2}$ no desenho abaixo.

% %Paulo: inserir a imagem: Disponível em:
% https://br.freepik.com/vetores-gratis/boho-art-tribal-doodle-esboco-linha-tracejada-quadro\_14925840.htm\#query=blank\%20rectangle\%20line\&position=0\&from\_view=search\&track=ais.
% Acesso em: 14 maio 2023.

\reduline{O aluno deve pintar a metade do desenho.\hfill}

\num{5} Encontre a fração geratriz da dízima $1,253333\ldots{}$ pelo método
convencional.

\reduline{Primeiro passo: $1,2533\ldots = x$

Segundo passo: Como o antiperíodo tem dois algarismos, vamos multiplicar
por $100 ⇒ 1,253 \cdot 100 = 100 \cdot x\  \Rightarrow \ 125,333\ldots = \ 100x$

Terceiro passo: $125,33\ldots \cdot 10 = 100x \cdot 10$ vamos multiplicar por $10$,
pois o período tem um número, então a potência de $10$ é $1$.

$1253,33\ldots = 1000x$

Quarto passo:
$1000x - 100x = \ 1253,3333 - 125,3333\ldots\ 900x = \ 1128$

$x = \frac{1128}{900}$\hfill}

\num{6} Classifique as frações em próprias ou impróprias.

\begin{escolha}
    
    \item $\frac{14}{3}$ \rosa{Imprópria (o numerador é maior que o denominador)}

    \item$\frac{1}{8}$ \rosa{própria}

    \item $\frac{8}{2}$ \rosa{Imprópria (o numerador é maior que o denominador)}

    \item $\frac{9}{30}$ \rosa{própria}

\end{escolha}

\num{7} Construa um quadrado e destaque a representação da fração
$\frac{1}{4}$.

\reduline{O aluno deverá dividir o quadrado em 4 e pintar somente uma das
partes.\hfill}

\num{8} João Marcos e Felipe, estão pintando um cômodo com paredes do mesmo
tamanho. Se João Marcos pintou $\frac{1}{4}$ da parede e Felipe pintou
$\frac{4}{16}$, quem está mais adiantado?

\reduline{Ambos os rapazes pintaram o mesmo tanto, pois as frações equivalem o
mesmo tanto, basta multiplicar o numerador e o denominador por 4 da
fração de João Marcos, obterá a fração de Felipe.\hfill}

\num{9} Ligue a fração geratriz a sua forma decimal.

a) $\frac{174}{990}$ \\ 

b) $\frac{15}{99}$ \\

c) $\frac{148}{99}$ \\

I. $0,15151515\ldots{}$ \\

II. $1,49494949\ldots{}$ \\

III. $0,17575757\ldots{}$ \\

\reduline{a-III, b-I, c-II. Para descobrir cada fração geratriz, devemos:

$0,151515\ldots$ = Quando não há parte inteira, basta colocar o período no
numerador e a quantidade de números $9$ para cada algarismo do período.

$1,49494949\ldots$ = trata-se de uma dízima simples, então, $149 - 1$ para o
numerador. No denominador, colocamos $99$, pois encontramos dois
algarismos no período.

$0,1757575\ldots$ = trata-se de uma dízima composta, então, $175 - 1 = 174$.
Para o denominador, colocamos um $9$ para cada algarismo do período e um $0$
para o antiperíodo.}

\num{10} Em uma corrida dividida em 6 partes, há 10 competidores. Ricardo,
Júlia, Marina e Flávio percorreram, respectivamente, $\frac{2}{6}$,
$\frac{4}{6}$, $\frac{3}{12}$, $\frac{8}{24}$. Algum deles está
empatado? Por quê?

\reduline{Sim, Ricardo e Flávio estão empatados, pois a fração de conclusão
equivale ao mesmo valor. Basta multiplicar o numerador e o denominador
por 4 da fração de Ricardo ou dividir a de Flávio pelo mesmo número.\hfill}

\section{Treino}

\num{1} Qual é a fração da dízima periódica 7,95959595\ldots?

\begin{escolha}
\item $\frac{795}{99}$
\item $\frac{788}{90}$
\item $\frac{95}{99}$
\item $\frac{788}{99}$
\end{escolha}

%SAEB: Determinar uma fração geratriz para uma dízima periódica.

% A - Incorreta, pois não subtraiu a parte inteira do denominador.
% B - Incorreta, pois, quando há 0 no denominador, temos um antiperíodo na
% dízima periódica.
% C - Incorreta, pois desconsiderou a parte inteira da fração.
% D - Correta, pois, para encontrar o numerador, calculamos 795 - 7 = 788
% no numerador e colocamos 99 no denominador, já que há dois algarismos no
% período.

\num{2} Fabiana, Marcelo, Maicon e Juliano participaram de uma competição de
12 etapas. O desempenho deles, respectivamente foi: $\frac{1}{12}$,
$\frac{6}{12}$, $\frac{2}{12}$,$\frac{3}{36}$. Qual foi a dupla de
competidores que ficou em último lugar?

\begin{escolha}
\item
  Fabiana e Marcelo
\item
  Fabiana e Juliano
\item
  Juliano e Maicon
\item
  Marcelo e Maicon
\end{escolha}

%%\section{BNCC: EF07MA08 }
% -- Comparar e ordenar frações associadas às ideias de
% partes de inteiros, resultado da divisão, razão e operador.
% SAEB: Identificar frações equivalentes.

% A - Incorreta, pois Marcelo foi o melhor colocado.
% B - Correta, pois, comparando as frações de mesmo denominador, Fabiana
% teve o pior desempenho e a fração de Juliano é equivalente ao resultado
% de Fabiana.
% C - Incorreta, pois Maicon ficou em segundo colocado.
% D - Incorreta, pois Marcelo e Maicon foram os primeiros colocados.

\num{3} Um terreno foi dividido em três partes iguais. João comprou $2$ dessas
partes. Qual das alternativas representa a fração que expressa a parte
do terreno que João comprou?

\begin{escolha}
\item $\frac{1}{3}$
\item $\frac{2}{3}$
\item $\frac{2}{5}$
\item $\frac{3}{4}$
\end{escolha}

%%\section{BNCC: EF07MA09 }
% -- Utilizar, na resolução de problemas, a associação
% entre razão e fração, como a fração 2/3 para expressar a razão de duas
% partes de uma grandeza para três partes da mesma ou três partes de outra
% grandeza.
% SAEB: Representar frações menores ou maiores que a unidade por meio de
% representações pictóricas ou associar frações a representações
% pictóricas.

% A - Incorreta, pois essa fração representa a parte de um todo quando uma
% parte é considerada. No caso, João comprou mais do que 1/3 do terreno.
% B - Correta, pois o terreno foi dividido em três partes iguais. João
% comprou 2 dessas partes. Para determinar a fração que expressa a parte
% do terreno que João comprou, devemos considerar que ele comprou 2 partes
% de um total de 3 partes do terreno.
% C - Incorreta, pois essa fração não corresponde à proporção de partes do
% terreno que João comprou.
% D - Incorreta, pois essa fração também não representa corretamente a
% proporção de partes do terreno que João comprou.

\chapter{Porcentagem}
\markboth{Módulo 4}{}

\section{Habilidade do SAEB }
\begin{itemize}
\item Resolver problemas que envolvam porcentagens,
incluindo os que lidam com acréscimos e decréscimos simples, aplicação
de percentuais sucessivos e determinação de taxas percentuais.
\end{itemize}

\subsection{Habilidade da BNCC }
\begin{itemize}
\item EF07MA02
\end{itemize}

% Professor, faça uma boa revisão de porcentagem com os alunos, discutindo
% representação fracionária e decimal.

Porcentagem vem de ``por cem'', ou seja, são relações com o número $100$,
mais precisamente uma divisão por $100$. Assim, definimos:

$$x\ \% = \frac{x}{100}$$

Assim, toda porcentagem pode ser representada de 3 formas distintas:
percentual, fracionária ou decimal.

\textbf{Exemplo:} 
$$5\% = \frac{5}{100} = 0,05$$
 $$17,5\% = \frac{17,5}{100} = \frac{175}{1000} = 0,175$$

Para calcular a porcentagem de um valor, basta multiplicá-lo pela
representação decimal ou fracionária, e nunca pela forma percentual.

\textbf{Exemplo:}
$$6\% de 120 = \frac{6}{100}\  \times 120 = \frac{6 \times 12}{10} = \frac{72}{10} = 7,2 ou 0,06 \times 120 = 7,2$$
Existem algumas porcentagens chamadas de ``notáveis'', que podem ser
feitas por cálculo mental:

$1\%$ de um valor \rightarrow basta dividir por $100$

$10\%$ de um valor \rightarrow basta dividir por $10$

$20\%$ de um valor \rightarrow basta dividir por $5$

$25\%$ de um valor \rightarrow basta dividir por $4$

$50\%$ de um valor \rightarrow basta dividir por $2$

A partir dessas porcentagens, podemos calcular valores relacionados,
como $5\%$ (metade de $10\%$), $40\%$ (o dobro de $20\%$), e assim por diante.

\textbf{Acréscimos e descontos}

A porcentagem é muito usada no contexto financeiro. Encontramos, por
exemplo, problemas de aumento ou diminuição no preço de itens, que, na
maioria das vezes, vão ser dados na forma de porcentagem. Quando vamos
calcular aumento e desconto em algum item, podemos usar dois
raciocínios:

1º \rightarrow calcular qual valor corresponde à porcentagem dada e,
em seguida, adicionar/subtrair do preço original do item.

2º \rightarrow relacionar o preço inicial a $100\%$ e pensar que,
quando temos um acréscimo, somamos a porcentagem a $100\%$ e, quando temos
um desconto, subtraímos. Essa nova porcentagem é calculada em cima do
valor original.

\textbf{Exemplo:} Um item que custava $R\$250,00$ recebeu um desconto de
$12\%$. Qual é o valor do item após o desconto?

1º
$\rightarrow \ 12\%\ de 250 = 0,12\  \times \ 250 = 30\  \rightarrow \ Valor final = 250 - 30 = 220\ reais$

2º
$\rightarrow \ desconto de 12\% = 100\% - 12\% = 88\%$

$\\{Valor final} = 88\%\ de 250 = 0,88\  \times 250 = 220\ reais$

\textbf{Acréscimos e descontos sucessivos}

Quando é necessário aplicar um acréscimo ou desconto em cima de outros,
é necessário multiplicar o que chamamos de ``fatores de multiplicação'',
que são os valores percentuais encontrados após subtrairmos de 100\% o
percentual de acréscimo ou desconto aplicado sucessivamente. Observe o
exemplo:

\rightarrow \ Um item que custava R\$50,00 sofreu um acréscimo de
$10\%$. Como não foi vendido, a loja deu um desconto de $15\%$ na compra à
vista. Qual é valor após o aumento e o desconto?

Aumento de $10\%\rightarrow fator de multiplicação\ é\ 110\% = 1,1$

Desconto de $15\% \rightarrow fator de multiplicação\ é\ 85\% = 0,85$

Assim, o valor final é igual a
$50 \times 1,1 \times 0,85 = 46,75\ reais$

%Professor, enfatize com os alunos o fato de que, ao aplicar aumentos e descontos sucessivos, nunca podemos somar as porcentagens. Use o exemplo anterior para mostrar como ficaria o resultado se fizéssemos dessa forma. Se necessário, faça um outro exemplo.

\section{Atividades}

\num{1} Represente as porcentagens abaixo na forma de fração e decimal:

\begin{escolha}
\item  $88\% =$ \rosa{$\frac{88}{100} = 0,88$} \\
\item  $57\% =$ \rosa{$\frac{57}{100} = 0,57$} \\
\item  $22,5\% =$ \rosa{$\frac{22,5}{100} = \frac{225}{1000} = 0,225$} \\
\item  $7,12\% =$  \rosa{$\frac{7,12}{100} = \frac{712}{10000} = 0,0712$} \\
\item  $152\% =$ \rosa{$\frac{152}{100} = 1,52$} \\
\item  $356\% =$ \rosa{$\frac{356}{100} = 3,5$} 

\end{escolha}

\num{2} Transforme o decimal abaixo em fração e em porcentagem:

\begin{escolha}
  \item 0,12 = \rosa{$\frac{12}{100} = 12\%$}  \\
  \item 0,03 =  \rosa{$\frac{3}{100} = 3\%$}  \\
  \item 5,17 =  \rosa{$\frac{517}{100} = 517\%$}  \\
  \item 1,36 =  \rosa{$\frac{136}{100} = 136\%$}  \\
  \item 0,58 =  \rosa{$\frac{58}{100} = 58\%$}  \\
  \item 0,802 =  \rosa{$\frac{802}{1000} = \frac{80,2}{100} = 80,2\%$}
\end{escolha}

\num{3} Calcule as porcentagens abaixo usando a multiplicação do valor pela
representação fracionária ou decimal:

\begin{escolha}
  \item $2\%\ de 156 =$ \rosa {$0,02 \times 156 = 3,12$} \\
  \item $77\%\ de 105 =$ \rosa {$0,77 \times 105 = 80,85$} \\
  \item $65\%\ de 178 =$ \rosa {$0,65 \times 178 = 115,7$} \\
  \item $180\%\ de 2700 =$ \rosa {$\frac{180}{100} \times 2700 = 180 \times 27 = 4860$} 
\end{escolha}

\num{4} Encontre o valor das porcentagens usando a ideia das porcentagens
notáveis:

\begin{escolha}
  \item $50\%$ de $852$ = \rosa{$ \frac{852}{2} = 426 $} \\
  \item $10\%$ de $1980$ = \rosa{$ \frac{1980}{10} = 198 $} \\
  \item $20\%$ de $365$ = \rosa{$ \frac{365}{5} = 73 $} \\
  \item $1\%$ de $4500$ = \rosa{$ \frac{4500}{100} = 45 $} \\
  \item $25\%$ de $496$ = \rosa{$ \frac{496}{4} = 124$} 
\end{escolha}

\num{5} Uma escola é composta por $1.500$ alunos. $45\%$ são meninos. Quantas
meninas há nessa escola?

\reduline{$45\%$ são meninos \rightarrow $55\%$ são meninas. 
$55\%$ de $1.500  = \frac{55}{100} \times 1.500 = 55 \times 15 = 825$ meninas.\hfill}

\num{6} Priscila foi fazer a compra do material escolar de seus dois fillhos
para o retorno das aulas. O total da compra ficou em $R\$.2100,00$, a ser
pago à vista com $7\%$ de desconto ou em 10x sem juros no cartão. Priscila
decidiu que só pagaria à vista se o desconto significasse uma diferença
de $R\$150,00$ ou mais no valor final. Qual forma de pagamento Priscila
escolheu?

\reduline{$7\%$ de desconto \rightarrow fator de multiplicação = $0,93$

$0,93 \times 2100 = 1953$

$2100 - 1953 = 147 < 150$ \rightarrow Priscila escolheu a forma parcelada. \hfill}


\num{7} Mário decidiu trocar seu carro. Ele foi a duas lojas para fazer uma
pesquisa de preço e ver qual delas estava oferecendo a melhor
oportunidade. A loja X vendia o carro por $R\$45.000,00$ à vista ou em
$12 \times 4000,00$, enquanto a loja Z vendia o carro por $R\$46.500,00$
em até 12 vezes sem juros ou com $10\%$ de desconto à vista. Mário decidiu
comprar na loja que oferecia a maior diferença do valor à vista para o
prazo. Qual loja Mário escolheu e quanto ele pagou pelo carro, sabendo
que ele pagou à vista?

\reduline{Loja A: 45.000,00 à vista

$12 \times 4000 = 48.000,00$ a prazo

$48.000 - 45.000 =  3.000,00$

Loja B: $10\%$  de desconto = $0,9 \times 46.500,00 = 41.850,00$ à vista

$46.500,00$ a prazo

$46.500 - 41.850 = 4.650,00$

Portanto, Mário escolheu a loja B e pagou $R\$41.850,00$ no carro.}

\num{8} Calcule o fator de multiplicação, com duas casas decimais, dos
aumentos e descontos sucessivos abaixo:
    
\begin{escolha}
    \item Aumento de $15\%$ seguido de um desconto de $20\%$  = \rosa{$115\% \times 80\% = 1,15 \times 0,8 = 0,92$} \\
    \item Dois aumentos de $5\%$ seguido de um desconto de $15\%$  = \rosa{$105\% \times 105\% \times 85\% = 1,05 \times 1,05 \times 0,85 \cong 0,94$} \\
    \item Dois descontos de $20\%$ seguido de dois aumentos de $8\%$  = \rosa{$80\% \times 80\% \times 108\% \times 108\% = 0,8 \times 0,8 \times 1,08 \times 1,08 \cong 0,75$} \\
    \item Desconto de $12\%$ seguido de dois aumentos de $3\%$  = \rosa{$88\% \times 103\% \times 103\% = 0,88 \times 1,03 \times 1,03 \cong 0,93$} 
\end{escolha}

\num{9} Uma geladeira que custava $R\$2.850,00$ entrou na promoção com $15\%$ de
desconto à vista. Como a procura foi grande, a loja resolveu aumentar
seu preço em $5\%$ para aproveitar as vensas. Quanto passou a custar a
geladeira após o desconto e o aumento?

\reduline{Desconto de $15\% = 0,85$

Aumento de $5\%  = 1,05$

$2850 \times 0,85 \times 1,05 = 2543,625$

No final, a geladeira saiu pelo valor de R\$ 2.543,62.\hfill}

\num{10} Uma moto que, em 2020, custava $R\$8.100,00$ sofreu uma desvalorização
de $5\%$ em 2021 e de $10\%$ em 2022. Qual é o valor da moto no início de
2023?

\reduline{
Desvalorização de $15\%  = 0,95$

Desvalorização de $10\%  =  0,9$

$8100 \times 0,95 \times 0,9 = 6.196,5$

No início de 2023 o valor da moto era de R\$ 6.925,50.\hfill}

\section{Treino}

\num{1} Um item que custava $R\$200,00$ teve um acréscimo de $15\%$. Qual das
alternativas representa o novo valor do item?

\begin{escolha}
  \item $R\$ 215,00$
  \item $R\$ 220,00$
  \item $R\$ 230,00$
  \item $R\$ 245,00$
\end{escolha}

%%\section{BNCC: EF07MA02 }
% -- Resolver e elaborar problemas que envolvam
% porcentagens, como os que lidam com acréscimos e decréscimos simples,
% utilizando estratégias pessoais, cálculo mental e calculadora, no
% contexto de educação financeira, entre outros.
% SAEB: Resolver problemas que envolvam porcentagens, incluindo os que
% lidam com acréscimos e decréscimos simples, aplicação de percentuais
% sucessivos e determinação de taxas percentuais.

% A - Incorreta, pois essa resposta não leva em consideração o acréscimo
% de 15\% no valor original.
% B - Incorreta, pois essa resposta também não considera o acréscimo
% correto. É importante lembrar que 15\% de R\$200,00 é igual a R\$30,00,
% não R\$20,00.
% C - Correta, pois o novo valor = Valor original + (Porcentagem de
% acréscimo vezes o Valor original). Nesse caso, o valor original é
% R\$200,00 e o acréscimo é de 15\%. Novo valor = 200 + \frac {15}{100}\times 200 Novo valor = 200 +
% 0,15 \times 200 Novo valor = 200 + 30 Novo valor = 230
% D - Incorreta, pois representa o valor original acrescido de 22,5\%, não
% de 15\%.

\num{2} Um apartamento é vendido por 60 parcelas de $R\$4.500,00$ ou com
$15\%$ de desconto à vista em cima do preço parcelado. A diferença do
preço à vista em relação ao preço a prazo é de:

\begin{escolha}
    \item 4.050,00

    \item 4.500,00

    \item 40.500,00

    \item 40.050,00
\end{escolha}

%%\section{BNCC: EF07MA02 }
% -- Resolver e elaborar problemas que envolvam
% porcentagens, como os que lidam com acréscimos e decréscimos simples,
% utilizando estratégias pessoais, cálculo mental e calculadora, no
% contexto de educação financeira, entre outros.
% SAEB: Resolver problemas que envolvam porcentagens, incluindo os que
% lidam com acréscimos e decréscimos simples, aplicação de percentuais
% sucessivos e determinação de taxas percentuais.

% A - Incorreta, pois a conta foi feita como se o apartamento custasse
% R\$27.000.
% B - Incorreta, pois somente considerou o valor de uma parcela como
% resposta.
% C - Correta, pois 60 \times 4500 = 270.000. 15\% \rightarrow \text{fator}\ 0,85 \rightarrow 0,85 \times 270.000 = 229.500. 270.000 - 229.500 = 40.500,00
% D - Incorreta, pois a conta 270.000 - 229.500 foi feita
% incorretamente.

\num{3} Uma loja que vende pneus recebeu uma remessa nova de mercadoria que
sofreu um aumento de $6\%$ no preço. No final do semestre, ainda restavam
alguns pneus em estoque e, para acelerar as vendas, o gerente resolveu
dar um desconto de $10\%$ no preço à vista. Sabendo que, antes do aumento,
o pneu custava $R\$270,00$, uma pessoa que comprou no final do semestre
pagou aproximadamente:

\begin{escolha}
  \item R\$ $286,00$
  \item R\$ $280,00$
  \item R\$ $260,00$
  \item R\$ $258,00$
\end{escolha}
%%\section{BNCC: EF07MA02 }
% -- Resolver e elaborar problemas que envolvam
% porcentagens, como os que lidam com acréscimos e decréscimos simples,
% utilizando estratégias pessoais, cálculo mental e calculadora, no
% contexto de educação financeira, entre outros.
% SAEB: Resolver problemas que envolvam porcentagens, incluindo os que
% lidam com acréscimos e decréscimos simples, aplicação de percentuais
% sucessivos e determinação de taxas percentuais.

% A - Incorreta, pois considerou apenas o aumento de 6\%.
% B - Incorreta, pois considerou o fator de multiplicação como sendo 1,04.
% C - Incorreta, pois calculou o fator de multiplicação como 100\% + 6\% -
% 10\%.
% D - Correta, pois Aumento de 6\%\  = \ 1,06. Desconto de 10\%\  = \ 0,9. 270 \times 1,06 \times 0,9 = 257,58. No final do semestre, o pneu saiu por aproximadamente R\$258,00.

\section{Equações de 1.º grau}
\markboth{Módulo 5}{}

\section{Habilidades do SAEB }
\begin{itemize}
\item Resolver uma equação polinomial de 1º grau.
\item
  Inferir uma equação, inequação polinomial de 1º grau ou um sistema de
  equações de 1º grau com duas incógnitas que modelam um problema.
\item
  Associar uma equação polinomial de 1º grau com duas variáveis a uma
  reta no plano cartesiano.
\item
  Resolver problemas que possam ser representados por sistema de
  equações de 1º grau com duas incógnitas.
\end{itemize}

\section{Habilidade da BNCC }
\begin{itemize}
\item EF07MA18
\end{itemize}

Box de teoria

Professor, neste módulo será muito importante que os alunos saibam que a
adição é o inverso da subtração e a divisão é o inverso da
multiplicação. É importante em todas as resoluções lembrar que a equação
é uma questão de equilíbrio e tudo que eu faço de um lado, tenho que
fazer do outro lado da igualdade para manter o mesmo. É bom trabalhar
muitos exercícios para normalizar a escrita algébrica.

\textbf{Equação do primeiro grau}

Uma equação do primeiro grau é toda expressão que tem letras e números e
o grau das letras é 1. Devemos encontrar o valor das incógnitas que são
as letras, isolando-as, para isso, utilizamos das operações básicas,
adição, subtração, multiplicação e divisão. Para isolar as letras, é
preciso fazer uma manipulação para que se mantenha o equilíbrio, para
isso, utilizamos as operações inversas.

\textbf{Exemplos de soma e subtração:}

$$x + 8 = 11$$

$$x = 11 - 8$$

$$x = 3$$

Como tinha uma soma com a letra, utilizamos da operação inversa que é a
multiplicação para descobrir o valor numérico da letra. Se fosse uma
subtração:

$n - 100 = \ 312$

$$n = 312 + 100$$

$$n = 412$$

O mesmo acontece para multiplicação e divisão.

\textbf{Exemplos multiplicação e divisão:}

$$30b = 15$$

$$b = \frac{15}{30} = \frac{1}{2}$$

Ou seja, tudo que está com a letra vai ser operado de forma inversa para
a letra ser isolada.

$$\frac{x}{12} = 12$$

$$x = \ 144$$

Quando nos deparamos com um problema de equação do primeiro grau, o que
estamos querendo descobrir, sempre será a incógnita.

\textbf{Exemplos:} Suponha que você tenha um número e multiplique ele
por 3 e depois adicione 40, como resultado você obteve 133. Qual é este
número?

Basta seguir o passo a passo, o número será x que foi multiplicado por
3, então 3x e adicionado 40 e de resultado 133:

$$3x + 40 = 133$$

$$3x = 133 - 40$$

$$x = \ 93$$

$$x = 31$$

A idade de Jéssica é o dobro da idade de Rafael. Daqui a 10 anos,
Jéssica terá 20 anos. Qual a idade de Rafael hoje?

$$2x = \ 20 - 10$$

$$2x = 10$$

$$x = 5$$

\textbf{Observação:} Além das equações do primeiro grau, há casos das
inequações do primeiro grau, que é quando ao invés de um sinal de
igualdade, é um sinal de maior ou menor. Ou seja, uma desigualdade. A
diferença nos métodos de resolução é quando um número negativo está
dividindo ou multiplicando a desigualdade será invertida, isto é, se é
maior, torna-se menor e vice-versa.

\textbf{Representações gráficas}

As equações polinomiais de grau 1 podem ser representadas por gráficos
no plano cartesiano. Ela sempre será uma reta crescente ou decrescente:

\begin{figure}
\centering
\includegraphics[width=4.82292in,height=1.69792in]{./imgSAEB_7_MAT/media/image17.png}
\caption{Diagrama Descrição gerada automaticamente com confiança média}
\end{figure}

Produzir uma imagem igual a essa, seguindo os padrões do material.

Todo ponto do plano cartesiano é dado por \((x,y)\), e com as equações
não são diferentes. Temos que, x é o valor que encontramos da incógnita
e o resultado da equação.

\textbf{Exemplo:}

Dada a equação $x + 4$, temos os seguintes pares ordenados: \\

Se $x = 1$ então $(1,5)$, pois $1 + 4 = 5$. \\

Se $x = 3$ então $(3,7)$, pois $3 + 4 = 7$.

Pela relação das grandezas, seria um gráfico crescente, pois quanto
maior o valor de x, maior o resultado obtido.

\textbf{Sistemas de equações}

É possível que se depare com um problema com duas incógnitas. Para isso,
há dois métodos comuns de solução, que é o método da adição e da
substituição.

\textbf{Método da adição:} Soma-se as duas equações a fim de eliminar
uma das incógnitas e ao descobrir uma delas, descobre a outra.

\textbf{Método da substituição:} Isola-se uma das incógnitas em função
da outra. E substitui na equação que não está isolada.

\textbf{Exemplos:}

Método da adição:

O método da adição é uma técnica utilizada para resolver sistemas de
equações lineares. Ele envolve a adição ou subtração das equações do
sistema de forma a eliminar uma das variáveis, permitindo assim resolver
a outra variável.

Considere o seguinte sistema de equações: Equação 1: 2x + 3y = 7 Equação
2: 4x - y = 1

O objetivo é encontrar os valores de x e y que satisfazem
simultaneamente as duas equações.

Passo 1: Escolha uma das variáveis para eliminar. Neste caso, vamos
eliminar a variável y.

Passo 2: Multiplique cada termo de uma das equações por um número de
forma a tornar o coeficiente da variável escolhida igual em magnitude
(mas de sinal oposto) nas duas equações. Neste exemplo, vamos
multiplicar a Equação 2 por 3, para tornar o coeficiente de y igual a 3.

Equação 1: 2x + 3y = 7 Equação 2 (multiplicada por 3): 12x - 3y = 3

Agora, as equações ficam: Equação 1: 2x + 3y = 7 Equação 2: 12x - 3y = 3

Passo 3: Some as duas equações termo a termo. Nesse caso, somamos as
equações 1 e 2 para eliminar a variável y.

(2x + 3y) + (12x - 3y) = 7 + 3

Simplificando, temos: 14x = 10

Passo 4: Isolamos a variável x dividindo ambos os lados da equação por
14:

$$\frac {14x}{14} = \frac {10}{14}$$

$$x = \frac {5}{7}$$

Agora, encontramos o valor de x.

Passo 5: Substitua o valor encontrado para x em uma das equações
originais. Vamos usar a Equação 1:

$$2(\frac {5}{7}) + 3y = 7$$

Simplificando, temos: $\frac {10}{7} + 3y = 7$

Passo 6: Isolamos a variável y:

$$3y = 7 - \frac {10}{7}$$

$$3y = \frac {49}{7} - \frac {10}{7}$$

$$3y = \frac {39}{7}$$

Dividindo ambos os lados da equação por 3: $y = \frac {13}{7}$

Agora, encontramos o valor de y.

Portanto, a solução do sistema é $x = \frac {5}{7}$ e $y =
\frac {13}{7}$.

Para a resolução, basta somar as duas equações, para isso, é necessário
deduzir que uma das incógnitas será eliminada. Em alguns casos, é
possível multiplicar ou dividir a equação toda, para que a outra seja
eliminada.

Método da substituição:

Já o método da substituição envolve a substituição de uma variável em
uma equação por uma expressão equivalente obtida a partir de outra
equação do sistema.

Considere o seguinte sistema de equações: Equação 1: 2x + y = 5 Equação
2: x - y = 1

O objetivo é encontrar os valores de x e y que satisfazem
simultaneamente as duas equações.

Passo 1: Escolha uma das equações e resolva-a em relação a uma das
variáveis. Vamos escolher a Equação 2 e resolver em relação a x:

x = 1 + y

Passo 2: Substitua a expressão encontrada para a variável escolhida na
outra equação do sistema. Nesse caso, vamos substituir x na Equação 1:

$$2(1 + y) + y = 5$$
Simplificando, temos: 

$$2 + 2y + y = 5$$

$$3y + 2 = 5$$

Passo 3: Resolva a equação resultante para encontrar o valor da
variável. Continuando com o exemplo:

$$3y + 2 = 5$$

$$3y = 5 - 2$$

$$3y = 3$$

Dividindo ambos os lados da equação por 3: y = 1

Agora, encontramos o valor de y.

Passo 4: Substitua o valor encontrado para y na expressão que
encontramos no Passo 1 para encontrar o valor da outra variável. Vamos
substituir y = 1 em x = 1 + y:

$$x = 1 + 1$$

$$x = 2$$

Agora, encontramos o valor de x.

Portanto, a solução do sistema é x = 2 e y = 1.

\section{Atividades}

\num{1} Daiane e Leandra compraram 7 balas e 4 chocolates, gastando R\$ $10,70$. Se
o valor de cada chocolate equivale a 25 balas, quanto custou cada
chocolate?

\reduline{

$7b + 4c = 10,70$

$1c = 25b$

Neste caso, como uma das variáveis já está isolada, basta substituir na
primeira equação:

$7b + 4 \cdot 25b = 10,70$

$7b + 100b = 10,70$

$107\ b = 10,70$

$b = 0,10$

$c = 2,50$
\hfill}

\num{2} Resolva as equações abaixo:

\begin{escolha}
  \item $2n - 40 = 102$ \rosa{$2n = 102 + 40\  \Rightarrow \ 2n = 142\  \Rightarrow \ n = 71$}
  \item $3p - 15 = 330$ \rosa{$3p = 330 + 15\  \Rightarrow \ 3p = 345\  \Rightarrow \ p = \ 115$}
  \item $\frac{x}{20} + 7 = 10$ \rosa{$  \ \frac{x}{20} + 7 = \ 10\  \Rightarrow \frac{x}{20} = \ 10 - 7\  \Rightarrow \ x = \ 3*20 = \ 60$}
  \item $m + 24 = 189$ \rosa{$m = 189 - 24\  \Rightarrow \ m = \ 165$}
\end{escolha}

\num{3} Os irmãos Henrique e Matheus são gêmeos e completaram 10 anos.
Considerando que, daqui 6 anos, sua mãe terá o triplo da idade deles,
qual é a idade dela?

\reduline{$3 \cdot(10 + 6) = x  \Rightarrow 30 + 18 = 48 \Rightarrow x = 48$\hfill}

\num{4} João está economizando dinheiro para comprar um novo videogame. Ele
já possui R\$ $50,00$ guardados e está economizando R\$ $10,00$ por semana.
Ele quer saber quanto tempo levará para juntar o valor total necessário
para comprar o videogame, que custa R\$ $250,00$.

\reduline{O problema pode ser resolvido utilizando uma equação polinomial de 1º
grau, na forma $ax + b = c$, onde $x$ representa o número de semanas, a é o
coeficiente de $x$ ($10$, pois João economiza R\$ $10,00$ por semana), $b$ é o
termo constante ($50$, pois ele já tem R\$ $50,00$) e $c$ é o valor total
necessário para comprar o videogame ($250$).

Ao resolver a equação, encontramos $x = 20$, o que significa que João
levará 20 semanas para juntar o valor necessário.\hfill}

\num{5} Encontre o valor das incógnitas e escreva em forma de par ordenado.

\begin{escolha}
\item $x + 10 = 28$ \rosa{$x = 28 - 10  \Rightarrow x = 18$ Logo, o par ordenado será $(18,28)$}
\item $c - 13 = \ 79$ \rosa{$c = 79 - 13  \Rightarrow x = 66$ Logo, o par ordenado será $(66,79)$}
\item $3a + 8 = \ 35$ \rosa{$3a = 35 - 8  \Rightarrow 3a = 27 \Rightarrow a = 9$. Logo, o par ordenado será $(9,35)$}
\end{escolha}

%Professor, neste momento é bom trabalhar bastante o abstrato para que os
%estudantes entendam, que independentemente do par ordenado, o conjunto
%de pontos de uma equação do primeiro grau é uma reta.

\num{6} As idades de Letícia e Gabriela somam 58 anos. Se Gabriela é 4 anos
mais velha que Letícia, qual é a idade de cad uma?

\reduline{Para resolver essas equações, precisaremos de um sistema.

$L + G = 58$

$G = L + 4$

Como Gabriela é quatro anos mais velha, ela tem a idade de Letícia
somado com 4 anos.

Um dos termos da equação já está isolado, então, basta substituir.

$L + L + 4 = 58\  \Rightarrow \ 2L = 58 - 4\  \Rightarrow \ 2L = 54\  \Rightarrow \ L = 27$

Portanto, Letícia tem 27 anos. Se Letícia tem 27 anos, vamos substituir
novamente na primeira equação.

$27 + G = 58$

$G = 58 - 27$

$G = 31$

Ou ainda, poderia pensar que como Gabriela tem 4 anos a mais que
Letícia, bastava fazer 27 + 4 = 31.\hfill}

\num{7} Fábio trabalha no Cartório Eleitoral e estava analisando a quantidade
de eleitores que já tinham feito seu voto na última eleição. No total,
são 9.536 eleitores. Na última conta feita, 7.504 já tinham votado e 900
tinham justificado a falta nas urnas. Sabendo que ainda faltavam duas
zonas de votação para serem analisadas, quantas pessoas votam em uma
delas, sabendo que as duas que sobraram possuem a mesma quantidade de
eleitores?

\reduline{

$9536 - 7504 - 900 = 2x$

$1132 = 2x$

$x = \ 566$\hfill}

\num{8} Coloque entre parênteses, o par ordenado tal qual $x = 8$ que
corresponde a equação dada:

\begin{escolha}
\item 12x + 4 - 2 (8;\rosa {$98$}) \rosa {$12 \cdot 8 + 4 - 2 =  98$}
\item 10x - 9 (8;\rosa {$71$}) \rosa {$10 \cdot 8 - 9 = 71$}
\item 5x + 2 (8;\rosa {$42$}) \rosa {$5 \cdot8 + 2 = \ 42$}
\item 7 - 2x (8;\rosa {$-9$}) \rosa {$7 - 2 \cdot 8 = - 9$}
\end{escolha}

\num{9} Uma loja vende camisetas a R\$ 30,00 cada e calças a R\$ 50,00 cada.
Em um determinado dia, um cliente comprou algumas camisetas e calças,
totalizando 7 peças. O valor total da compra foi de R\$ 280,00. Quantas
camisetas o cliente comprou?

\reduline{Vamos denotar por ``x'' o número de camisetas compradas pelo cliente.

De acordo com o enunciado, cada camiseta custa R\$ $30,00$, então o valor
total gasto em camisetas será de 30x reais.

Além disso, sabemos que o cliente comprou um total de 7 peças, ou seja,
camisetas e calças juntas. Portanto, o número de calças compradas será
dado por $7 - x$.

Cada calça custa R\$ $50,00$, então o valor total gasto em calças será de
$50 \times (7 - x)$ reais.

Somando os valores gastos em camisetas e calças, temos o valor total da
compra, que é R\$ $280,00$. Portanto, podemos escrever a seguinte equação:

$30x + 50 \times (7-x) = 280$

Agora, vamos resolver essa equação de primeiro grau.

Começamos simplificando a expressão:

$30x + 350 - 50x = 280$

Em seguida, agrupamos os termos com x:

$30x - 50x = 280 - 350$

$-20x = -70$

Dividindo ambos os lados por -20, obtemos:

$x = -70 / -20$

$x = 3,5$

Portanto, o cliente comprou 3,5 camisetas. No entanto, como não é
possível comprar meio produto, concluímos que o cliente comprou 3
camisetas.\hfill}

\num{10} Qual é o número cujo resultado da divisão por 2 somado a 10 é igual
a 28?

\reduline{$\frac{n}{2}\  + \ 10\  = \ 28\  \Rightarrow \ \frac{n}{2} = \ 28 - 10\  \Rightarrow \ n = 18 \cdot 2\  \Rightarrow \ n = 36$}

\section{Treino}

\num{1} Determine o valor de x na equação 2x + 3 = 9.

\begin{escolha}
\item x = 4 
\item x = 3 
\item x = 2 
\item x = 6
\end{escolha}

%%\section{BNCC: EF07MA18 }
% -- Resolver e elaborar problemas que possam ser
% representados por equações polinomiais de 1º grau, redutíveis à forma ax
% + b = c, fazendo uso das propriedades da igualdade.
%SAEB: Resolver uma equação polinomial de 1º grau.

%A - Incorreta, pois se substituirmos x por 4 na equação original,obtemos 2(4) + 3 = 8 + 3 = 11, que não é igual a 9.
%B - Correta, pois se substituirmos x por 3 na equação original, obtemos2(3) + 3 = 6 + 3 = 9, que é igual a 9. Portanto, x = 3 é a solução correta da equação.
%C - Incorreta, pois se substituirmos x por 2 na equação original,
%obtemos 2(2) + 3 = 4 + 3 = 7, que não é igual a 9.
%D - Incorreta, pois se substituirmos x por 6 na equação original,
%obtemos 2(6) + 3 = 12 + 3 = 15, que não é igual a 9.

\num{2} Qual é o número que, multiplicado por 3 e somado a 41 resulta em 143?

\begin{escolha}
\item 34
\item 27
\item 30
\item 33
\end{escolha}

%%\section{BNCC: EF07MA18}
%  -- Resolver e elaborar problemas que possam ser
% representados por equações polinomiais de 1º grau, redutíveis à forma ax
% + b = c, fazendo uso das propriedades da igualdade.
% SAEB: Resolver uma equação polinomial de 1º grau.

% A - Correta, pois \(3n + 41 = 143\). 3n = 143 - 41. 3n = 102. n = \ 34
% B - Incorreta, pois o triplo de 27 somado a 41 é igual a 122, que é
% diferente de 143.
% C - Incorreta, pois o triplo de 30 somado a 41 é igual a 131, que é
% diferente de 143.
% D - Incorreta, pois o triplo de 33 somado a 41 é igual a 140, que é
% diferente de 143.

\num{3} O preço de uma cartela de bingo em uma festa beneficiente é
R\(3,00. Yasmin saiu de casa com R\) 89,00 reais, gastou R\$22,00 com a
ida e a volta no aplicativo de veículos, comprou um salgado de R\$8,00,
um refrigerante de R\$5,00 e gastou uma quantia com cartelas de bingo.
Sabendo que ela retornou para casa com R\$33,00, quantas cartelas foram
compradas?

\begin{escolha}
\item 5 cartelas
\item 6 cartelas
\item 9 cartelas
\item 7 cartelas
\end{escolha}

%%\section{BNCC: EF07MA18 }
% -- Resolver e elaborar problemas que possam ser
% representados por equações polinomiais de 1º grau, redutíveis à forma ax
% + b = c, fazendo uso das propriedades da igualdade.
% SAEB: Inferir uma equação, inequação polinomial de 1º grau ou um sistema
% de equações de 1º grau com duas incógnitas que modelam um problema.

%A - Incorreta, pois, ela teria retornado com R\$39,00 se tivesse comprado 5 cartelas.
%B - Incorreta, pois ela teria retornado com R\$36,00 se tivesse comprado 6 cartelas.
%C - Incorreta, pois ela teria retornado com R\$27,00 se tivesse comprado 9 cartelas.
%D - Correta, pois, para encontrar o valor, devemos subtrair todos os gastos e igualar a 3c, que é o valor de cada cartela. 3c = \ 89 - 22 - 8 - 5 - 33. 3c = \ 21. c = \ 7 cartelas compradas.

\chapter{Expressões}
\markboth{Módulo 6}{}

\section{Habilidades do SAEB }
\begin{itemize}
\item Identificar uma representação algébrica para o
padrão ou a regularidade de uma sequência de números racionais ou
representar algebricamente o padrão ou a regularidade de uma sequência
de números racionais.
\item
  Identificar representações algébricas equivalentes.
\item
  Resolver problemas que envolvam cálculo do valor numérico de
  expressões algébricas.
\end{itemize}

Habilidades da BNCC 
\begin{itemize}
  \item EF07MA13
  \item EF07MA14
  \item EF07MA15
  \item EF07MA16
\end{itemize}

\textbf{{[}Expressões algébricas{]}}

As expressões algébricas são expressões matemáticas que relacionam
letras e números, ou apenas letras, que representam valores
desconhecidos a serem calculados.

\textbf{Exemplo:} O dobro de um valor somado com 7
\rightarrow 2a + 7.

As expressões algébricas são compostas por termos algébricos, que são
fatores da multiplicação de um \emph{{[}coeficiente{]}}, e por uma
\emph{{[}parte literal{]}}.

$$- 15x \rightarrow coeficiente = \  - 15 e parte literal = x$$

Como, em uma expressão algébrica, a letra não tem um valor fixo, ela é
chamada de \emph{{[}variável{]}}, e não incógnita como vimos na equação.
Quando atribuímos valor para a variável e encontramos um resultado para
a expressão algébrica, chamamos esse resultado de \emph{{[}valor
numérico.{]}}

\textbf{Exemplo:} Temos o dobro de um valor somado com 7. Consideremos
esse valor como 5.

$2a + 7 = 2 \times 5 + 7 = 10 + 7 = 17$ \leftarrow valor numérico

Quando a expressão algébrica possui termos com parte literal igual,
esses termos são chamados de \emph{{[}termos semelhantes{]}}, que podem
ser agrupados e reduzidos a um único termo.

$$5ab + 9ab - 2ab - 20ab + 8a = (5 + 9 - 2 - 20) ab + 8a = - 9ab + 8a$$

O termo 8a não é semelhante, já que a parte literal é diferente, por
isso não é possível realizar o agrupamento.

Existem expressões algébricas chamadas de \emph{{[}equivalentes{]}}. Ao
atribuirmos o mesmo valor às suas variáveis, encontramos o mesmo valor
numérico.

\textbf{Exemplo:} Considere as expressões 2(x\  + \ 1) + 3 e
2x + 5. Se substituirmos x por 4, temos que

$$2\left( x\  + \ 1 \right) + \ 3 = 2\left( 4 + 1 \right) + 3 = 2 \times 5 + 3 = 10 + 3 = 13$$
e 

$$2x + 5 = 2 \times 4 + 5 = 8 + 5 = 13$$ 

Portanto, elas são
expressões algébricas equivalentes.

\textbf{{[}Sequência Numérica{]}}

Uma sequência é um conjunto de elementos ordenados a partir de uma regra
ou uma lei de formação. Na matemática, o foco são as sequências
numéricas, que podem ser finitas ou infinitas.

\textbf{Exemplo:} A sequência dos múltiplos positivos de 2
(2,\ 4,\ 6,\ 8,\ 10,\ ...) é uma sequência infinita. A sequência dos
divisores positivos de 10 (1,\ 2,\ 5,\ 10) é uma sequência finita.

O termo geral de uma sequência é indicado por $a_{n}$, onde $n$  é
a posição do termo na sequência. Se ela for finita, $a_{n}$ indica o
último termo da sequência.

As sequências podem ser \emph{{[}recursivas{]}}, quando um termo é
obtido a partir do seu anterior, ou \emph{{[}não recursivas{]}}, quando
os termos são obtidos independentemente.

\textbf{Exemplo:} A sequência $(1,\ 4,\ 7,\ 10,\ \ldots{})$ é recursiva,
pois um termo é obtido pelo seu anterior + 3. A sequência dos números
primos é uma sequência não recursiva, já que os termos independem uns
dos outros.

As sequências podem ser obtidas a partir de uma \emph{{[}lei de
formação{]}}, que é uma expressão algébrica que determina qualquer termo
$a_{n}$  de uma sequência a partir da sua posição n.

\textbf{Exemplo:} Considere $a_{n} = n + 1$ uma sequência qualquer.
Qual será o 12º termo dessa sequência?

Utilizando a lei de formação, temos que

$$a_{n} = n + 1 \rightarrow a_{12} = 12 + 1 = 13$$

\section{Atividades}

\num{1} Classifique se nas expressões matemáticas abaixo temo uma incógnita
ou uma variável:

\begin{escolha}
\item $x + y = 16$ \rosa{Variável \rightarrow temos que $10 + 6 = 16, 14 + 2 = 16$, etc} \\
\item $2a - 8 = 2$ \rosa{Incógnita \rightarrow temos que $2a - 8 = 2 \rightarrow 2a = 2 + 8 \rightarrow 2a = 10 \rightarrow a = 5$} \\
\item $5w - 1$ \rosa{Variável \rightarrow é uma expressão algébrica} \\
\item $s - t = 7$ \rosa{Variável \rightarrow temos que $20 - 13 = 7, 10 - 3 = 7$, etc}
\end{escolha}

\num{2} Indique o coeficiente e a parte literal dos termos algébricos abaixo:

\begin{escolha}
\item $- 7xyz$ \rosa {$ \rightarrow C: - 7 P.L:xyz$} \\
\item $5a^2b^3$ \rosa {$ \rightarrow C: 5 P.L:a^2b^3$} \\
\item $- st^{4}$ \rosa {$ \rightarrow C:- 1 P.L:t^{4}$} \\
\item $\frac{4x^{5}}{9}$ \rosa {$ \rightarrow C:\frac{4}{9} \ P.L:x^{5}$}
\end{escolha}

\num{3} Reescreva as frases em linguagem algébrica:
\begin{escolha}
\item O dobro de um valor mais seu quadrado. \rosa{$2x + x²$} \\
\item O triplo de um valor somado com sua terça parte. \rosa{$3x + \frac{x}{3}$} \\
\item Um valor subtraído de quadruplo outro. \rosa{$x - 4y$} \\
\item A metade de um valor somado com a quinta parte de outro. \rosa{$\frac{x}{2} + \frac{y}{5}$}
\end{escolha}

\num{4} Calcule o valor numérico das expressões algébricas do exercício 3,
considerando $x = 6$ e $y = 10$.

\rosa {Item a: $2x + x^{2} = 2 \times 6 + 6^{2} = 12 + 36 = 48$} \\ \\
\rosa {Item b: $3x + \frac{x}{3} = 3 \times 6 + \frac{6}{3} = 18 + 2 = 20$} \\ \\
\rosa {Item c: $x - 4y = 6 - 4 \times 10 = 6 - 40 = - 34$} \\ \\
\rosa {Item d: $\frac{x}{2} + \frac{y}{5} = \frac{6}{2} + \frac{10}{5} = 3 + 5 = 8$} \\

\num{5} Reduza os termos semelhantes nas expressões algébricas abaixo:

\begin{escolha}
\item $5xy - 12xyz + 8xz - 6xy + 3xz - 7xyz$ \ \rosa{$(5-6)xy+(-12-7)xyz+(8+3)xz=-xy-19xyz+11xz$} \\
\item $ab + 8a - 14b - 6ab + 7b$ \ \rosa{$(1-6)ab+(-14+7)b+8a=-5ab-8ab-7b$} \\
\item $st + 2st + 11st - 35st + 12s - 9t$ \ \rosa{$(1+2+11-35)st+12s-9t=-21st+12s-9t$} \\
\item $50cd - 17cd - 25cd + 2,5cd - 12,1cd$ \ \rosa{$(50-17-25+2,5-12,1)cd=-1,6cd$}
\end{escolha}

\num{6} Utilizando n = 3, encontre quais das expressões da primeira
coluna são equivalentes às da segunda:

\begin{enumerate}
\item $n^{2} - 4$ \ \rosa{$n^{2} - 4 = 3^{2} - 4 = 9 - 4 = 5$} \\
\item $n + 7$ \ \rosa{$n + 7 = 3 + 7 = 10$} \\
\item $n^{2} - 5n$ \ \rosa{$n^{2} - 5n = 3^{2} - 5 \times 3 = 9 - 15 = - 6$} \\
\end{enumerate}

\begin{escolha}
\item $- n( - n + 5)$ \ \rosa {$-n( -n + 5 ) = -3( -3 + 5 ) = -3 \times 2 = -6$} \\
\item $(n + 2)(n - 2)$ \ \rosa {$( n + 2 )( n -2 ) = ( 3 + 2 )( 3 -1 ) = 5 \times 1 = 5$} \\
\item $2n + 2 - (n - 5)$ \ \rosa {$2n + 2 -( n -5 ) = 2 \times 3 + 2 -( 3 -5 ) = 6 + 2 -( -2 ) = 8 + 2 = 10$}
\end{escolha}

\rosa{Portanto, I é equivalente a B, II a C e III a A.} \\

\num{7} Complete os espaços que faltam seguindo a sequência. \\

\begin{escolha}
\item 12,\ 10,\ 8,\ 6,\ \_\_\_\_,\_\_\_\_,\_\_\_\_ \rosa{O padrão é o termo anterior - 2, assim os termos que faltam são 4,\ 2,\ 0.} \\
\item 1,\ 1,\ 2,\ 3,\ \_\_\_\_,\_\_\_\_,\_\_\_\_ \rosa{O padrão é a soma dos dois anteriores, assim os termos que faltam são 5,\ 8,\ 13.} \\
\item 7,\ 14,\ 21,\ 28,\ \_\_\_\_,\_\_\_\_,\_\_\_\_ \rosa{O padrão é o termo anterior + 7, assim os termos que faltam são 35,\ 42,\ 49} \\
\item 1,\ 4,\ 9,\ 16,\ \_\_\_\_,\_\_\_\_,\_\_\_\_ \rosa{O padrão é a posição do termo elevada ao quadrado, assim os termos que faltam são 25,\ 36,\ 49.} \\
\end{escolha}

%Professor, deixe os alunos analisarem as sequências com calma para que
%identifiquem qual o padrão por trás de cada uma.

\num{8} Classifique se as sequências do exercício 7 são recursivas ou não
recursivas.

\reduline{Recursiva: itens a, b, c, pois os termos dependem do anterior para serem
calculados. Não recursiva: d, pois os termos são definidos a partir da sua posição. \hfill}

\num{9} Escreva uma lei de formação para as sequências do exercício 7.

\reduline{Nas sequências recursivas, é preciso discutir com os alunos como
encontrar o termo anterior ao $a_{n}$, para que eles cheguem à ideia
de $a_{n - 1}$.

item A
  $a_{n} = a_{n - 1} - 2$
item B
  $a_{n} = a_{n - 2} + a_{n - 1}$
item C
  $a_{n} = a_{n - 1} + 7$
item D
  $a_{n} = n^2$}

\num{10} Defina os 3 primeiros termos das sequências descritas pelas leis de
formação abaixo:

\begin{escolha}
    \item $a_{n} = 2n - 3$ \\
\rosa{
$a_{1} = 2 \times 1 - 3 = 2 - 3 = - 1$ \\
${}a_{2} = 2 \times 2 - 3 = 4 - 3 = 1$ \\
${a}_{3} = 2 \times 3 - 3 = 6 - 3 = 3$ \\
}

    \item $a_{n} = n³$
\rosa{
$a_{1} = 1^{3} = 1$ \\
${a}_{2} = 2^{3} = 8$ \\
${a}_{3} = 3^{3} = 27$ \\
}

    \item $a_{n} = 5n - n²$
    \rosa{
$a_{1} = 5 \times 1 - 1^{2} = 5 - 1 = 4$ \\
${a}_{2} = 5 \times 2 - 2² = 10 - 4 = 6$ \\
${a}_{3} = 5 \times 3 - 3² = 15 - 9 = 6$ \\
}

    \item $a_{n} = - 3n^{2} + 10n$
    \rosa{
$a_{1} = - 3 \times 1^{2} + 10 \times 1 = - 3 + 10 = - 7$ \\
$a_{2} = - 3 \times 2^{2} + 10 \times 2 = - 12 + 20 = 8$ \\
${a}_{3} = - 3 \times 3^{2} + 10 \times 3 = - 27 + 30 = 3$ \\
}

\end{escolha}

\section{Treino}

\num{1} Samuel trabalha como vendedor em uma loja de roupas. Seu salário é
composto por uma parte fixa mais uma porcentagem em cima do total que
vende no mês. Em janeiro, houve uma alteração em seu salário, que passou
a ser um fixo de R\$ $2.500,00$ mais 3\% em cima de suas vendas. Assim,
considerando x como o total de vendas do mês, a expressão algébrica
que descreve o salário de Samuel é:

\begin{escolha}
  \item $2.500 + 3x$
  \item $\frac{2.500 + 3x}{100}$
  \item $2.500 \times 0,03x$
  \item $2.500 + 0,03x$
\end{escolha}

%%\section{BNCC: EF07MA13 }
% -- Compreender a ideia de variável, representada por
% letra ou símbolo, para expressar relação entre duas grandezas,
% diferenciando-a da ideia de incógnita.
% SAEB: Identificar uma representação algébrica para o padrão ou a
% regularidade de uma sequência de números racionais ou representar
% algebricamente o padrão ou a regularidade de uma sequência de números
% racionais.

% A - Incorreta, pois usou a porcentagem na forma percentual.
% B - Incorreta, pois apesar de transformar 3\% para \frac{3}{100},
% colocou o 100 como denominador da parte fixa do salário, transformando
% ela para 25 ao invés de 2500.
% C - Incorreta, pois multiplicou a comissão pelo fixo ao invés de somar.
% D - Correta, pois \text{Fixo}\  = \ 2500\ \ \ \ \ \ \ \ \ \ \ \ \ \ \ \ \ \ \ \ \ \ \ \text{Vari}á\text{vel}\  = \ 3\%\ \text{de}\ x\text{\ \ \ \ \ \ \ \ \ \ \ \ \ \ \ \ \ }\text{Sal}á\text{rio} = 2500 + 0,03x.

\num{2} O valor numérico da expressão
$\frac{a}{4} + 5b + \frac{2a^{2} - b}{5c},\ a = 8,\ b = - 3,\ c = 2$
é:

\begin{escolha}
\item $23,1$
\item $- 23,1$
\item $0,1$
\item $- 0,1$
\end{escolha}

%%\section{BNCC: EF07MA15 }
% -- Utilizar a simbologia algébrica para expressar
% regularidades encontradas em sequências numéricas.
% R: SAEB: Resolver problemas que envolvam cálculo do valor numérico de
% expressões algébricas.

% A - Incorreta, pois somou os algarismos como se fossem ambos positivos.
% B - Incorreta, pois somou os algarismos como se fossem ambos negativos.
% C - Correta, pois \frac{a}{4} + 5b + \frac{2a^{2} - b}{5c} = \frac{8}{4} + 5 \times \left( - 3 \right) + \frac{2\left( 8 \right)^{2} - \left( - 3 \right)}{5 \times 2} = = 2 - 15 + \frac{2 \times 64 + 3}{10} = \  - 13 + \frac{131}{10} = - 13 + 13,1 = 0,1.
% D - Incorreta, pois considerou o sinal do 13 como sendo o correto.

\num{3} Considerando a sequência dada por $a_{n} = n^{2} + 5n + 6$, uma
sequência que é equivalente a ela é:

\begin{escolha}
\item $a_{n} = (n + 2) + (n + 3)$
\item $a_{n} = (n + 2)(n + 3)$
\item $a_{n} = (n - 2) + (n - 3)$
\item $a_{n} = (n - 2)(n - 3)$
\end{escolha}

% SAEB: Identificar representações algébricas equivalentes. BNCC: EF07MA16
% -- Reconhecer se duas expressões algébricas obtidas para descrever a
% regularidade de uma mesma sequência numérica são ou não equivalentes.

% A - Incorreta, pois
% \left( n + 2 \right) + \left( n + 3 \right) = \left( 1 + 2 \right) + \left( 1 + 3 \right) = 3 + 4 = 7.
% B - Correta, pois \left( n + 2 \right)\left( n + 3 \right) = \left( 1 + 2 \right)\left( 1 + 3 \right) = 3 \times 4 = 12. \therefore\ \left( n + 2 \right)\left( n + 3 \right)\ é\ \text{equivalente}\ a\ n^{2} + 5n + 6.
% C - Incorreta, pois
% \left( n - 2 \right) + \left( n - 3 \right) = \left( 1 - 2 \right) + \left( 1 - 3 \right) = - 1 - 2 = - 3.
% D - Incorreta, pois
% \left( n - 2 \right)\left( n - 3 \right) = \left( 1 - 2 \right)\left( 1 - 3 \right) = - 1\  \times \ ( - 2) = 2.


\chapter{Proporcionalidade}
\markboth{Módulo 7}{}

\section{Habilidade do SAEB}
\begin{itemize}
\item Resolver problemas que envolvam variação de
proporcionalidade direta~ou inversa entre duas ou mais grandezas,
inclusive escalas, divisões proporcionais e taxa de variação.
\end{itemize}

\section{Habilidade da BNCC}
\begin{itemize}
\item EF07MA17
\end{itemize}

% Professor, trabalhe esse módulo de forma bem detalhada com os alunos,
% pois as relações de grandezas são de extrema importância. Os cálculos
% com regra de três serão muito utilizados nos próximos anos, tanto na
% matemática como em outras áreas.

\textbf{Razão e Proporção}

\textbf{{[}Razão:{]}} é uma relação de divisão entre duas grandezas de
algum sistema de medidas, indicada como $r  = \frac{A}{B}$.

\textbf{Exemplo:} a razão de 3 e 5 é
$\rightarrow \ \frac{3}{5} = 0,6$.

Algumas razões aparecem em outras áreas além da matemática. São elas:
velocidade média, escala, densidade demográfica, densidade de um corpo
etc.

Velocidade média = $\frac{distancia}{tempo}$ \\

Escala = $\frac{desenho}{real}$ \\

Densidade demográfica = $\frac{nº de habitantes}{área da região}$ \\

Densidade  corpo = $\frac{massa}{volume}$ \\

\textbf{{[}Proporção:{]}} é a igualdade entre duas razões. Assim, os
números reais não nulos A, B, C, D formam uma proporção. Temos a
seguinte relação:

$$\frac{A}{B} = \frac{C}{D} = K$$

Lemos que ``A está para B, assim como C está para D'', sendo \emph{K} a
constante de proporcionalidade, que indica quantas vezes os numeradores
das razões são maiores ou menores que os denominadores.

\textbf{{[}Propriedade fundamental das proporções:{]}} em uma proporção
$\frac{A}{B} = \frac{C}{D}$, vale a propriedade:
$A \times D\  = \ B \times C$.

\textbf{{[}Observação{]}}: Outra propriedade muito usada no contexto de
proporções, principalmente para encontrar o valor da constante de
proporcionalidade, é:

$$\frac{A}{B} = \frac{C}{D} = \frac{A \pm C}{B \pm D}$$

\textbf{Exemplo:} Qual é o valor da incógnita a seguir, sabendo que as
razões formam uma proporção?

$$\frac{x}{3} = \frac{6}{15}\  \rightarrow 15x = 18 \rightarrow \ \ x = \frac{18}{15} \rightarrow \ \ x = 1,2$$

\textbf{Exemplo:} Qual é o valor das incógnitas a seguir, sabendo que as
razões formam uma proporção e que x + y = 25?

$$\frac{x}{4} = \frac{y}{6} = \frac{x + y}{10} = \frac{25}{10} = 2,5\frac{x}{4} = 2,5 \rightarrow \ \ x = 10 \rightarrow y = 15$$

\textbf{{[}Grandezas Diretamente Proporcionais (G.D.P){]}:} duas
grandezas são diretamente proporcionais quando aumentam ou diminuem na
mesma proporção. Dizemos que são grandezas que se dividem:

$$\frac{A}{B} = K$$

\textbf{{Grandezas Inversamente Proporcionais (G.I.P)}:} duas grandezas
são inversamente proporcionais quando uma aumenta e a outra diminui na
mesma proporção. Dizemos que são grandezas que se multiplicam:

$$A \times B = K$$

\textbf{Observação:} quando vamos comparar duas grandezas, precisamos
ter alguma constante como referência para a comparação, e só assim
classificá-las em D.P ou I.P.

\textbf{Exemplo:} Comprando 15 maçãs, paga-se R\$ 7,50; comprando 8
maçãs, quanto será pago?

Maçã - Preço \rightarrow quanto mais maçãs, maior o preço,logo são G.D.P

A constante é o preço por maçã.

\textbf{Exemplo:} Com 5 funcionários, uma reforma é feita em 80 dias.
Caso sejam contratados mais 3 funcionários, em quantos dias a reforma
será concluída?

Funcionários - Dias

quanto mais pessoas trabalhando, menos dias serão gastos, logo são G.I.P.

A constante é a reforma.

\textbf{{[}Regra de três simples direta e inversa:{]}} usada para
encontrar uma grandeza desconhecida em uma proporção. Usamos a direta
quando as grandezas são diretamente proporcionais e a inversa quando
elas são inversamente proporcionais.

\textbf{Exemplo:} Comprando 15 maçãs, paga-se R\$ 7,50; comprando 8
maçãs quanto será pago?

$\frac{15}{7,5} = \frac{8}{x}\ \  \rightarrow \ \ 15x = 60 \rightarrow \ \ x = \frac{60}{15} \rightarrow \ \ x = 4$\ reais

\textbf{Exemplo:} Com 5 funcionários, uma reforma é feita em 80 dias, se
forem contratados mais 3 funcionários em quantos dias a reforma será
feita?

$5 \times 80 = 8 \times y \rightarrow \ \ 400 = 8y \rightarrow \ \ y = \frac{400}{8} = 50\ $dias

\section{Atividades}

\num{1} Represente as razões abaixo em forma de frações irredutíveis.

\begin{escolha}
\item $5$\ e\ $12$ \rosa {$= \frac{5}{12}$} \\
\item $9$\ e\ $6$ \rosa {$= \frac{9}{6} = \frac{3}{2}$} \\
\item $75$\ e\ $15$ \rosa {$= \frac{75}{35} = \frac{25}{7}$} \\
\item $120$\ e\ $45$ \rosa {$= \frac{120}{48} = \frac{5}{2}\ $} \\
\end{escolha}

\num{2} Calcule o valor das razões abaixo:

\begin{escolha}
\item 14 e 35 \rosa {$14 e 35 = \frac{14}{35} = 0,4$} \\
\item 63 e 18 \rosa {$63 e 18 = \frac{63}{18} = 3,5$} \\
\item 21 e 105 \rosa {$21 e 105 = \frac{21}{105} = 0,2$} \\
\item 212 e 84 \rosa {$848 e 212 = 4$} \\
\end{escolha}

\num{3} Verifique se as razões abaixo formam uma proporção.

\begin{escolha}
\item $\frac{12}{15} = \frac{60}{75}$ \rosa{$ 12 \times 75 = 15 \times 60\  \rightarrow \ \ 900 = 900  \rightarrow $ É uma proporção.} \\
\item $\frac{25}{16} = \frac{105}{80}$ \rosa{$ 80 \times 25 = 16 \times 105\  \rightarrow \ \ 2000 \neq 1680 \rightarrow$ Não é uma proporção.} \\
\item $\frac{7}{35} = \frac{84}{420}$ \rosa{$ 7 \times 420 = 84 \times 35\  \rightarrow \ \ 2940 = 2940 \rightarrow$ É uma proporção.} \\
\item $\frac{6}{14} = \frac{48}{128}$ \rosa{$ 6 \times 128 = 48 \times 14\  \rightarrow \ \ 768 = 672 \rightarrow$ Não é uma proporção.} \\
\end{escolha}

\num{4} Encontre o valor das incógnitas nas proporções abaixo:

\begin{escolha}
\item $\frac{x}{8} = \frac{6}{15}$ \ \rosa{$\ \ 15x = 48\  \rightarrow \ \ x = \frac{48}{15}\  \rightarrow \ \ \ x = 3,2$} \\
\item $\frac{12}{9} = \frac{20}{a}$ \ \rosa{$\ \ 12a = 180 \rightarrow \ \ a = \frac{180}{12}\  \rightarrow \ \ \ a = 15$} \\
\item $\frac{11}{b} = \frac{5}{7}$ \ \rosa{$\ \ 4b = 77\  \rightarrow \ \ b = \frac{77}{4}\  \rightarrow \ \ \ b = 19,25$} \\
\item $\frac{25}{5} = \frac{2y}{13}$ \ \rosa{$\ \ 5 \times 2y = 25 \times 13\  \rightarrow \ \ 10y = 325\  \rightarrow \ \ \ y = \frac{325}{10}\  \rightarrow \ \ \ y = 32,5$} \\
\end{escolha}

\num{5} Encontre o valor de x e y nas proporções abaixo, usando o valor das
suas somas e subtrações dadas.

\begin{escolha}

\item $\frac{x}{4} = \frac{y}{12}\  \rightarrow$ \ sabendo que $x + y = 32$ \\

\rosa{$ \frac{x + y}{4 + 12} = \frac{32}{16} = 2\  \rightarrow \ \ \frac{x}{4} = 2\  x = 8\  \rightarrow \ \ \ \frac{y}{12} = 2\  y = 24\ $} \\

\item $\frac{40}{x} = \frac{24}{y}\  \rightarrow$ \ sabendo que $x - y = 2$ \\

\rosa{$ \frac{40 - 24}{x - y} = \frac{16}{2} = 8 \rightarrow \ \ \frac{40}{x} = 8\  x = 5\  \rightarrow \ \ \ \frac{24}{y} = 8\  y = 3\ $} \\

\item $\frac{x}{58} = \frac{y}{8} \rightarrow$ \ sabendo que $x - y = 25$ \\

\rosa{$\frac{x - y}{58 - 8} = \frac{25}{50} = 0,5\  \rightarrow \ \ \frac{x}{58} = 0,5\  x = 29\  \rightarrow \ \ \ \frac{y}{8} = 0,5\  y = 4\ $} \\

\item $\frac{63}{x} = \frac{9}{y} \rightarrow$ \ sabendo que $x + y = 16$ \\

\rosa{$\frac{63 + 9}{x + y} = \frac{72}{16} = 4,5\  \rightarrow \ \ \frac{63}{x} = 4,5\  x = 14\  \rightarrow \ \ \ \frac{9}{y} = 4,5\  y = 2\ $} \\

\end{escolha}

\num{6} Classifique as grandezas abaixo como G.D.P (diretamente
proporcionais) e G.I.P (inversamente proporcionais), e indique qual deve
ser a constante da relação entre elas:

\begin{escolha}
\item Distância e tempo.

\reduline{Se a velocidade for constante, quanto maior a distância, maior será o tempo, ou seja, G.D.P.\hfill}

\item Quilometragem percorrida e consumo de combustível.

\reduline{Se o consumo do veículo for constante, quanto maior a distância, maior será a quantidade do combustível, ou seja, G.D.P.\hfill}

\item Velocidade e tempo.

\reduline{Se a distância for constante, quanto maior a velocidade, menor será o tempo, ou seja, G.I.P.
\hfill}

\item Desconto em um produto e o valor pago ao final.

\reduline{Se o preço inicial do produto for constante, quanto maior o desconto, menor será o valor pago no final, ou seja, G.I.P.\hfill}

\end{escolha}

\num{7} Sabendo que A e B; C e D são grandezas diretamente proporcionais,
calcule o valor da grandeza desconhecida:

\begin{escolha}
\item $A = 44,\ B = \ ?,\ C = 32,\ D = 8$ \\  
\rosa{$\frac{A}{B} = \frac{C}{D}\  \rightarrow \ \ \ \frac{44}{x} = \frac{32}{8}\  \rightarrow \ \ \ 32x = 352\  \rightarrow \ \ \ x = \frac{352}{32} = 11$} \\
\item $A = 150,\ B = \ 50,\ C = ?,\ D = 66$ \\
\rosa{$\frac{A}{B} = \frac{C}{D}\  \rightarrow \ \ \ \frac{150}{50} = \frac{x}{66}\  \rightarrow \ \ \ 50x = 66 \times 150\  \rightarrow \ \ \ x = \frac{66 \times 150}{50}\  \rightarrow \ \ \ x = 66 \times 3 = 198$} \\
\item $A = \ ?,\ \ B = \ 1200,\ C = 225,\ D = 300$ \\
\rosa{$\frac{A}{B} = \frac{C}{D}\  \rightarrow \ \frac{x}{1200} = \frac{225}{300}\  \rightarrow \ 300x = 225 \times 1200\  \rightarrow \ x = \frac{225 \times 1200\ }{300} \rightarrow \ x = 225 \times 4 = 900$} \\
\end{escolha}

\num{8} Sabendo que A e B; C e D são grandezas inversamente proporcionais,
calcule o valor da grandeza desconhecida:

\begin{escolha}
\item $A = 82\ ,\ B = \ 3,\ C = 41,\ D = ?$ \\ 
\rosa{$A \times B = C \times D \rightarrow \ \ 82 \times 3 = 41x\  \rightarrow 41x = 246 \rightarrow x = \frac{246\ }{41} \rightarrow \ x = 6$} \\
\item $A = 7\ ,\ B = \ ?,\ C = 49,\ D = 2$ \\
\rosa{$A \times B = C \times D \rightarrow \ \ 7x = 49 \times 2\  \rightarrow \ 7x = 98 \rightarrow x = \frac{98\ }{7} \rightarrow \ x = 14$} \\
\item $A = \ 22\ ,\ B = \ 5,\ C = ?,\ D = 10$ \\
 \rosa{$A \times B = C \times D \rightarrow \ \ 22 \times 5 = 10x\  \rightarrow 10x = 110 \rightarrow x = \frac{110\ }{10} \rightarrow \ x = 11$} \\
\end{escolha}

\num{9} Uma caixa d'água que possui 3 m de altura comporta 15.000 litros de
água. Se essa caixa d'água for substituída por uma de medidas iguais,
mas com 4 m de altura, qual será a nova capacidade em litros?

\reduline{
Quanto maior a altura, maior a capacidade da caixa d'água. Logo, são
G.D.P: \\

$\frac{3}{15\ 000} = \frac{4}{x}  \rightarrow 3x = 60\ 000\  \rightarrow  \hfill \\ 
x = \frac{60\ 000}{3}\  \rightarrow x = 20\ 000\ $ litros \hfill}

\num{10} Uma fábrica de estampar camisetas disse ao cliente que gastaria 10
dias para realizar seu pedido de última hora, utilizando as 9 máquinas
disponíveis. Porém, ao iniciar o processo, três máquinas quebraram. Qual
é o novo prazo a ser comunicado ao cliente para a entrega do pedido?

\reduline{
Como o número de máquinas trabalhando diminuiu, o tempo será maior para
concluir o pedido, logo, são G.I.P:

$10\ .\ 9 = x\ \cdot \ 6\  \rightarrow \ \ \ 6x = 90\  \rightarrow \ \ \ x = \frac{90}{6} = 15\ $ dias \hfill}

\section{Treino}

\num{1} Marcos está fazendo alguns testes para saber quanto tempo ele vai
gastar da sua casa até seu novo emprego. Em um dia sem trânsito, ele
consegue fazer todoo percurso no limite da velocidade da rodovia de 80
km/h em 40 minutos. Porém, em dias de trânsito, ele só consegue ir a uma
velocidade de 50 km/h e gasta 1 hora e 4 minutos. Marcos precisa da
constante de proporcionalidade para fazer o cálculo da velocidade e do
tempo para os outros valores. Para isso, ele deve considerar que:

\begin{escolha}
\item as grandezas são diretamente proporcionais e $k  =  2$.
\item as grandezas são inversamente proporcionais e $k\  = \ 2$.
\item as grandezas são diretamente proporcionais e $k\  = \ 3200$.
\item as grandezas são inversamente proporcionais e $k\  = \ 3200$.
\end{escolha}

%%\section{BNCC: EF07MA17 }
% -- Resolver e elaborar problemas que envolvam variação de
% proporcionalidade direta e de proporcionalidade inversa entre duas
% grandezas, utilizando sentença algébrica para expressar a relação entre
% elas.
% SAEB: Resolver problemas que envolvam variação de proporcionalidade
% direta~ou inversa entre duas ou mais grandezas, inclusive escalas,
% divisões proporcionais e taxa de variação.

% A - Incorreta, pois considerou as grandezas como diretamente
% proporcionais.
% B - Incorreta, pois ao invés de multiplicar as grandezas 80 e 40, fez a
% divisão entre elas.
% C - Incorreta, pois, apesar de encontrar a constante correta, considerou
% as grandezas como diretamente proporcionais.
% D - Correta, pois, ao reduzir a velocidade, o tempo para percorrer o
% trajeto aumentou, ou seja, são grandezas inversamente proporcionais.
% Assim: 80 \times 40 = 50 \times 64 = k = 3.200.

\num{2} Na construção de uma área comercial, para pintar 3 salas com 2
paredes de 15 m² e 2 paredes de 11 m², foi calculado que 3 pintores
gastariam 2 dias. Porém, ao terminar de levantar as paredes, os
pedreiros perceberam que as paredes de uma ssala ficaram com 22 m² e 34
m² respectivamente. Para manter o prazo de 2 dias, quantos pintores
precisam ser contratados?

\begin{escolha}
\item 5
\item 4
\item 2
\item 1
\end{escolha}

%%\section{BNCC: EF07MA17 }
% -- Resolver e elaborar problemas que envolvam variação de
% proporcionalidade direta e de proporcionalidade inversa entre duas
% grandezas, utilizando sentença algébrica para expressar a relação entre
% elas.
% SAEB: Resolver problemas que envolvam variação de proporcionalidade
% direta~ou inversa entre duas ou mais grandezas, inclusive escalas,
% divisões proporcionais e taxa de variação.

% A - Incorreta, pois colocou o total de pintores para concluir o serviço
% e não quantos tinham que ser contratados.
% B - Incorreta, pois considerou que seriam necessários 4 pintores no
% total ao invés de 5.
% C - Correta, pois inicialmente: 1 quarto \rightarrow 30 + 22 = 52m^{2} 3 quartos
% \rightarrow 3 \times 52 = 156m^{2}. No final: 1 quarto \rightarrow 44 + 68 = 112m^{2} 3 quartos \rightarrow 2 \times 52 + 112 = 216m^{2}. Se o serviço aumentou, para manter o prazo, é preciso aumentar o número de pintores, ou seja, relação diretamente proporcional. Assim: \frac{156}{3} = \frac{216}{x}\  \rightarrow \ \ 156x = 648 \rightarrow x \cong 4,1. Logo, são necessários 4,1 pintores para realizar a nova pintura, mas, como não é possível contratar 1,1 pessoas, é necessário contratar 2
% pintores para manter o prazo inicial.
% D - Incorreta, pois considerou a partir do resultado 1,1, considerou que
% bastava contratar 1 pintor.

\num{3} Uma empresa composta por 3 investidores vai fazer a divisão
proporcional dos lucros obtidos no ano de 2022. A divisão será
diretamente proporcional ao valor que cada investidor aplicou no início
do ano. O investidor A aplicou um capital de R\$50.000, o investidor B,
de R\$ 75.000,00 e o investidor C, de R\$ 115.000,00. Se o lucro da
empresa foi de R\$60.000,00, quanto o investidor B deverá receber?

\begin{escolha}
\item R\$ $15.000,00$
\item R\$ $18.750,00$
\item R\$ $20.000,00$
\item R\$ $12.500,00$
\end{escolha}

%%\section{BNCC: EF07MA17 }
% -- Resolver e elaborar problemas que envolvam variação de
% proporcionalidade direta e de proporcionalidade inversa entre duas
% grandezas, utilizando sentença algébrica para expressar a relação entre
% elas.
% SAEB: Resolver problemas que envolvam variação de proporcionalidade
% direta~ou inversa entre duas ou mais grandezas, inclusive escalas,
% divisões proporcionais e taxa de variação.

% A - Incorreta, pois pegou o total dos lucros e subtraiu do total
% investido pelo investidor B.
% B - Correta. Vamos considerar os seguintes valores recebidos de lucro pelos
% investidores: A\  = \ x\; B\  = \ y C\  = \ z. Como a divisão é diretamente proporcional ao valor investido, temos: \frac{x}{50.000} = \frac{y}{75.000}\  = \frac{z}{115.000} = k; \frac{x + y + z}{50.000 + 75.000 + 115.000} = \frac{60.000}{240.000} = 0,25; \frac{x}{50.000} = 0,25 \rightarrow x = 12.500; \frac{y}{75.000} = 0,25 \rightarrow y = 18.750; \frac{z}{115.000} = 0,25 \rightarrow z = 28.750. 
% C - Incorreta, pois pegou o total do lucro e dividiu por 3 sem
% considerar a proporcionalidade ao que cada um investiu.
% D - Incorreta, pois considerou apenas o investidor A.

\section{Polígonos}
\markboth{Módulo 8}{}

\section{Habilidades do SAEB}
\begin{itemize}
\item Identificar, no plano cartesiano, figuras obtidas
por uma ou mais transformações geométricas (reflexão, translação,
rotação).
\item
  Relacionar o número de vértices, faces ou arestas de prismas ou
  pirâmides, em função do seu polígono da base.
\item
  Relacionar objetos tridimensionais às suas planificações ou vistas.
\item
  Classificar polígonos em regulares e não regulares.
\item
  Reconhecer polígonos semelhantes ou as relações existentes entre
  ângulos e lados correspondentes nesses tipos de polígonos.
\item
  Reconhecer circunferência/círculo como lugares geométricos, seus
  elementos (centro, raio, diâmetro, corda, arco, ângulo central, ângulo
  inscrito).
\item
  Construir/desenhar figuras geométricas planas ou espaciais que
  satisfaçam as condições dadas.
\item
  Resolver problemas que envolvam relações entre os elementos de uma
  circunferência/círculo (raio, diâmetro, corda, arco, ângulo central,
  ângulo inscrito).
\end{itemize}

\section{Habilidades da BNCC}
\begin{itemize}
\item EF07MA19
\item EF07MA20
\item EF07MA21
\end{itemize}

% Professor, para este módulo, é muito importante retomar a ideia de que,
% assim como é possível desenharmos figuras planificadas, podemos criar
% figuras espaciais. Estabelecer todas as relações de planificação e
% espaço geométrico ocupado.

\textbf{Transformações geométricas no plano cartesiano}

As transformações no plano cartesiano são operações que podem ser
aplicadas a pontos, figuras geométricas ou até mesmo a todo o sistema de
coordenadas cartesianas, resultando em uma modificação na sua posição,
forma ou orientação. Essas transformações são amplamente estudadas na
geometria e são úteis para descrever movimentos e relações espaciais
entre objetos.

Existem quatro principais transformações no plano cartesiano:
translação, rotação, reflexão e ampliação/redução (também conhecida como
dilatação).

Translação: A translação consiste em mover um objeto em uma determinada
direção e distância no plano cartesiano. Esse movimento é feito mantendo
a mesma forma e tamanho do objeto original. Para realizar uma
translação, cada ponto do objeto é deslocado por uma mesma quantidade em
direção horizontal e vertical.

Rotação: A rotação envolve girar um objeto em torno de um ponto fixo
chamado centro de rotação. Nessa transformação, todos os pontos do
objeto descrevem uma trajetória circular. A rotação pode ser feita em
sentido horário (rotação negativa) ou anti-horário (rotação positiva) e
pode ser especificada pelo ângulo de rotação.

Reflexão: A reflexão é uma transformação que espelha um objeto em
relação a uma linha chamada eixo de reflexão. Todos os pontos do objeto
são refletidos em relação a essa linha, resultando em uma imagem
espelhada do objeto original. A reflexão pode ocorrer tanto
horizontalmente quanto verticalmente, dependendo da posição do eixo de
reflexão.

Ampliação/Redução: A ampliação/redução consiste em alterar o tamanho de
um objeto mantendo sua forma e proporções. Essa transformação pode ser
realizada multiplicando as coordenadas de cada ponto do objeto por um
fator de escala. Se o fator de escala for maior que 1, ocorre uma
ampliação, enquanto se for menor que 1, ocorre uma redução.

\textbf{Planificações no plano cartesiano}

No contexto da geometria, planificação refere-se a representar uma
figura tridimensional em um plano bidimensional, como o plano
cartesiano. Isso é feito por meio de projeções e mapeamentos, permitindo
que a figura tridimensional seja visualizada em uma forma plana.

As planificações são amplamente utilizadas para representar sólidos
geométricos complexos de uma maneira mais acessível e compreensível.
Elas desempenham um papel importante na engenharia, arquitetura, design
e outras áreas que lidam com objetos tridimensionais.

Existem diferentes técnicas para realizar planificações no plano
cartesiano. Alguns dos métodos mais comuns são:

Projeção Ortogonal: Nesse método, as arestas do objeto tridimensional
são projetadas perpendicularmente em um plano, criando uma representação
bidimensional. Cada face do sólido é então mapeada no plano cartesiano,
preservando as relações entre os vértices e as arestas.

Desenvolvimento: Essa técnica é usada principalmente para planificar
objetos que podem ser desdobrados, como caixas, embalagens e poliedros
regulares. Consiste em desdobrar as faces do objeto ao longo das suas
arestas e estendê-las em um plano. Isso cria uma representação
bidimensional que mostra todas as faces do objeto de forma plana e
organizada.

Mapeamento Cilíndrico: Nesse método, o objeto tridimensional é envolvido
em torno de um cilindro imaginário. A superfície do objeto é então
desenrolada e mapeada em um plano. Essa técnica é frequentemente usada
para planificar objetos com superfícies curvas, como cones e cilindros.

\textbf{Relação de Euler}

A relação de Euler diz que, para todo prisma cuja base é um polígono,
utilizamos a fórmula

$$V - A + F = 2$$

onde vértices = V, arestas = A e faces = F .

\textbf{Classificação de polígonos~}

Os polígonos recebem dois tipos de classificação. São regulares os
polígonos que possuem todos os lados e medidas de ângulos congruentes.
Já os polígonos não regulares não possuem todos os lados congruentes.
Dois polígonos são ditos semelhantes quando são congruentes os ângulos e
os lados que estão relacionados a eles. Com a prova da semelhança de
polígonos, encontramos duas propriedades: em polígonos semelhantes, os
perímetros e as diagonais também são semelhantes. Por fim, também são
semelhantes os polígonos congruentes.~

\textbf{Círculos e circunferências~}

Círculo é uma figura geométrica. Circunferência é o lugar geométrico de
todos os pontos de um plano que apresentam equidistância em relação ao
centro da circunferência.(ponto O). A circunferência é dividida em
alguns elementos essenciais, como o raio, o diâmetro, o arco e a corda.

Além disso, a circunferência oferece dois casos que facilitam o cálculo
de ângulos.

1º caso: Ângulo central, quando a abertura do ângulo coincide com o
arco.

2º caso: Ângulo inscrito, quando o ponto do ângulo faz parte da
circunferência (borda). O valor do arco é o dobro do ângulo.

\textbf{Exemplos:}

Dado um ângulo central $AÔB = \ 35{^\circ}$, qual é valor do arco $AB$?
O arco $AB$ valerá $35°$.

Quando se tem um arco de $84°$, qual é valor do ângulo inscrito na
circunferência? O ângulo valerá $42°$.

\section{Atividades}

\num{1} Considere um prisma com uma base formada por um polígono regular de n
lados. Seja V o número de vértices, F o número de faces e A o número de
arestas desse prisma.

Relacione V, F e A em função de n, o número de lados do polígono da
base.

\reduline{Um prisma é um sólido geométrico formado por duas bases paralelas
congruentes e faces laterais retangulares ou quadradas. O número de
vértices, faces e arestas de um prisma está diretamente relacionado ao
número de lados do polígono da base.

Para entender a relação entre essas quantidades, vamos analisar cada
elemento separadamente:

Número de vértices (V): Cada base possui n vértices, pois é um polígono
regular de n lados. Além disso, há um vértice em cada interseção das
arestas laterais com as bases. Portanto, o número de vértices em um
prisma é dado por $V = 2n + n = 3n$.

Número de faces (F): Cada base é uma face. Além disso, há n faces
laterais retangulares ou quadradas, cada uma correspondente a um lado do
polígono da base. Portanto, o número de faces em um prisma é dado por $F
= 2 + n$.

Número de arestas (A): Cada base possui n arestas, pois é um polígono
regular de n lados. Além disso, há n arestas laterais que conectam os
vértices das bases. Portanto, o número de arestas em um prisma é dado
por $A = 2n + n = 3n$.

Em resumo, a relação entre o número de vértices (V), faces (F) e arestas
(A) de um prisma com uma base de polígono regular de n lados é dada por:
$V = 3n F = 2 + n A = 3n$\hfill}

\num{2} Dado um polígono de n lados, determine se ele é um polígono regular
ou não regular.

\reduline{Um polígono é classificado como regular se todos os seus lados têm o
mesmo comprimento e todos os seus ângulos internos têm a mesma medida.
Caso contrário, ele é classificado como não regular.

Para determinar se um polígono é regular ou não, devemos comparar os
comprimentos dos lados e as medidas dos ângulos internos.

Verificar os comprimentos dos lados: Medir o comprimento de cada lado do
polígono. Se todos os lados tiverem o mesmo comprimento, o polígono é
regular. Caso haja pelo menos um par de lados com comprimentos
diferentes, o polígono é não regular.

Verificar as medidas dos ângulos internos: Medir cada ângulo interno do
polígono. Se todos os ângulos tiverem a mesma medida, o polígono é
regular. Caso haja pelo menos um par de ângulos com medidas diferentes,
o polígono é não regular.

Portanto, para classificar um polígono como regular ou não regular, é
necessário analisar tanto os comprimentos dos lados quanto as medidas
dos ângulos internos. Se todos os lados tiverem o mesmo comprimento e
todos os ângulos internos tiverem a mesma medida, o polígono será
classificado como regular. Caso contrário, ele será classificado como
não regular.\hfill}

\num{3} Suponha um ângulo central em uma circunferência de 31°. Qual é a
amplitude do arco do ângulo central?

\reduline{Quando o ângulo é central, vale o mesmo tanto do arco, então, 31°.\hfill}

\num{4} Considere um círculo de centro O e raio r. Seja AB uma corda desse
círculo e seja C um ponto qualquer sobre o círculo. Seja D o ponto de
interseção entre a reta AC e o círculo. Determine o diâmetro do círculo
em função do raio r.

\reduline{O diâmetro do círculo é o dobro do raio, portanto o diâmetro é igual a
2r.\hfill}

\num{5} Considere um círculo de raio 5 cm. Um ângulo central de 60 graus é
formado no centro desse círculo.

\reduline{Para calcular o arco correspondente a um ângulo central, utilizamos a
fórmula:

$s = \frac {θ}{360} \cdot 2\pi r$

Substituindo os valores conhecidos:

$θ = 60$ graus $r = 5$ cm

Podemos calcular o arco (s) correspondente utilizando a fórmula:

$s = \frac {60}{360} \cdot 2\pi \cdot 5 s =
\frac {1}{6} \cdot 2\pi \cdot 5 s = \frac {5\pi}{6}$

Portanto, o arco correspondente ao ângulo central de 60 graus é
aproximadamente igual a $\frac{5π}{6}$ cm.\hfill}

\num{6} As torres em forma de pirâmide de uma construção na cidade de Felipe
possuem base quadrangular e 5 faces. Quantos vértices essa figura
geométrica possui?

\reduline{Se a base é quadrangular, a forma é piramidal e há 5 faces, temos 8
arestas para sustentação. Pela relação de Euler,
$V - 8 + 5 = 2 \rightarrow \ 5 - 8 + 5 = 2$. Logo, temos 5 vértices.\hfill}

\num{7} Considere um prisma regular de base triangular equilátera. Determine
o número de vértices, faces e arestas desse prisma.

\reduline{Um prisma regular é um sólido geométrico que possui duas bases paralelas
congruentes, que são polígonos regulares, e suas faces laterais são
retângulos. Para determinar o número de vértices, faces e arestas desse
prisma, precisamos considerar o número de lados do polígono da base, que
é um triângulo equilátero.

Um triângulo equilátero possui três lados iguais, então cada lado do
polígono da base é igual.

Número de vértices: Um prisma tem duas bases, e cada base tem 3
vértices, sendo eles os vértices do triângulo equilátero. Além disso, há
um vértice em cada interseção das arestas laterais com as bases.
Portanto, o número de vértices em um prisma com base triangular
equilátera é dado por: $V = 3 + 3 + 1 = 7$.

Número de faces: Um prisma tem duas bases, que são triângulos
equiláteros. Além disso, há 3 faces laterais retangulares que se
conectam as arestas dos triângulos da base. Portanto, o número de faces
em um prisma com base triangular equilátera é dado por: $F = 2 + 3 = 5$.

Número de arestas: Um prisma tem 3 arestas em cada base, pois o
triângulo equilátero possui 3 lados. Além disso, há 3 arestas laterais
que conectam os vértices das bases. Portanto, o número de arestas em um
prisma com base triangular equilátera é dado por: $A = 3 + 3 = 6$.

Portanto, um prisma regular de base triangular equilátera possui 7
vértices, 5 faces e 6 arestas.\hfill}

\num{8} A professora Giovanna trabalhou com os estudantes da pré-escola o uso
de um espelho para ver que desenho formava. Qual transformação
geométrica isso implica?

\reduline{A transformação geométrica descrita é a reflexão.\hfill}

\num{9} Um objeto passou por uma transformação geométrica. Todas as
coordenadas passaram a ser negativas, sem alteraração nos números. Qual
é o seu novo quadrante?

\reduline{O novo quadrante é o terceiro quadrante.\hfill}

\num{10} Considere dois polígonos A e B. Os ângulos correspondentes de A e B
são iguais, enquanto os lados correspondentes têm comprimentos
proporcionais. Determine se os polígonos A e B são semelhantes.

\reduline{Dois polígonos são considerados semelhantes se seus ângulos
correspondentes são iguais e os lados correspondentes têm comprimentos
proporcionais.

No problema proposto, é dito que os ângulos correspondentes de A e B são
iguais, o que atende a uma das condições para a semelhança. Agora,
precisamos verificar se os lados correspondentes têm comprimentos
proporcionais.

Sejam os lados de A denotados por $a1, a2, a3, \ldots, an$, e os lados
correspondentes de B denotados por $b1, b2, b3, \ldots, bn$. Dado que os
polígonos são semelhantes, isso implica que existe uma constante de
proporcionalidade k tal que:

$$b1 = k \times a1 b2 = k \times a2 b3 = k \times a3 \ldots{}$$
$$bn = k * an k \times an$$

Se essa condição de proporcionalidade for satisfeita para todos os lados
correspondentes de A e B, podemos concluir que os polígonos são
semelhantes.

Portanto, para resolver o problema, é necessário verificar se os lados
correspondentes têm comprimentos proporcionais utilizando a relação
descrita acima. Se a relação de proporcionalidade for válida para todos
os lados, podemos concluir que os polígonos A e B são semelhantes. Caso
contrário, eles não são semelhantes.\hfill}

\section{Treino}

\num{1} Qual das seguintes opções identifica corretamente a corda de uma
circunferência?

\begin{escolha}
\item Um segmento de reta que liga dois pontos da circunferência.
\item Um segmento de reta que liga o centro da circunferência a um ponto
qualquer da circunferência.
\item Um arco da circunferência que possui a mesma medida que um ângulo
central.
\item Um segmento de reta que liga o centro da circunferência a um ponto
médio de um arco da circunferência.
\end{escolha}

%%\section{BNCC: EF07MA20}
%  -- Reconhecer e representar, no plano cartesiano, o
% simétrico de figuras em relação aos eixos e à origem.
% SAEB: Reconhecer circunferência/círculo como lugares geométricos, seus
% elementos (centro, raio, diâmetro, corda, arco, ângulo central, ângulo
% inscrito).

% A - Incorreta, pois uma corda não é apenas um segmento de reta que liga
% dois pontos da circunferência, já que não necessariamente passa pelo
% centro da circunferência.
% B - Incorreta, pois essa opção descreve o raio da circunferência, não a
% corda. O raio liga o centro da circunferência a um ponto específico na
% circunferência, enquanto a corda liga dois pontos quaisquer da
% circunferência.
% C - Incorreta, pois um arco da circunferência não pode ser considerado
% uma corda. A corda é um segmento de reta, enquanto o arco é uma parte da
% circunferência.
% D - Correta, pois a definição correta de uma corda é um segmento de reta
% que liga o centro da circunferência a um ponto médio de um arco da
% circunferência. Isso significa que a corda passa pelo centro da
% circunferência e divide o arco em duas partes iguais.

\num{2} Um prisma retangular possui 6 faces retangulares. Quantas arestas
esse prisma possui?

\begin{escolha}
\item 8
\item 10
\item 12
\item 14
\end{escolha}

%%\section{BNCC: EF07MA19}
%  -- Realizar transformações de polígonos representados no
% plano cartesiano, decorrentes da multiplicação das coordenadas de seus
% vértices por um número inteiro.
% SAEB: Relacionar o número de vértices, faces ou arestas de prismas ou
% pirâmides, em função do seu polígono da base.

% A - Incorreta, pois, se um prisma retangular tivesse 8 arestas, teria
% apenas duas arestas por face, o que não seria suficiente para formar as
% arestas laterais.
% B - Incorreta, pois, se um prisma retangular tivesse 10 arestas, teria
% três arestas por face, o que também não seria suficiente para formar as
% arestas laterais.
% C - Incorreta, pois, se um prisma retangular tivesse 12 arestas, teria
% quatro arestas por face, o que ainda não seria suficiente para formar as
% arestas laterais.
% D - Correta, pois um prisma retangular possui 12 arestas na base (4
% arestas do retângulo superior + 4 arestas do retângulo inferior + 4
% arestas verticais que conectam as bases). Além disso, existem duas
% arestas laterais que se estendem verticalmente e conectam os vértices
% das bases, totalizando 14 arestas.

\num{3} Se um poliedro tem 6 faces e 10 arestas, quantos vértices ele tem?

\begin{escolha}
\item 5
\item 6
\item 8
\item 4
\end{escolha}

%%\section{BNCC: EF07MA19 }
% Habilidade SAEB: Construir/desenhar figuras geométricas planas ou
% espaciais que satisfaçam as condições dadas.
% -- Realizar transformações de polígonos representados no
% plano cartesiano, decorrentes da multiplicação das coordenadas de seus
% vértices por um número inteiro.

% A - Incorreta, pois 5 - 10 + 6 = 1\  \neq \ 2, logo, não corresponde
% à relação de Euler.
% B - Correta, pois V - 10 + 6 = 2 \rightarrow \ 6 - 10 + 6 = 2 ,
% logo, tem 6 vértices.
% C - Incorreta, pois 8 - 10 + 6 = 4\  \neq \ 2, logo, não corresponde
% à relação de Euler.
% D - Incorreta, pois 4 - 10 + 6 = 0\  \neq \ 2, logo, não corresponde
% à relação de Euler.


\section{Triângulos}
\markboth{Módulo 9}{}

\section{Habilidades do SAEB }

\begin{itemize}
\item Identificar propriedades e relações existentes
entre os elementos de um triângulo (condição de existência, relações de
ordem entre as medidas dos lados e as medidas dos ângulos internos, soma
dos ângulos internos, determinação da medida de um ângulo interno ou
externo).
\item
  Classificar triângulos ou quadriláteros em relação aos lados ou aos
  ângulos internos.
\item
  Identificar retas ou segmentos de retas concorrentes, paralelos ou
  perpendiculares.
\item
  Identificar relações entre ângulos formados por retas paralelas
  cortadas por uma transversal.
\item
  Resolver problemas que envolvam relações entre ângulos formados por
  retas paralelas cortadas por uma transversal, ângulos internos ou
  externos de polígonos ou cevianas (altura, bissetriz, mediana,
  mediatriz) de polígonos.
\item
  Resolver problemas que envolvam relações métricas do triângulo
  retângulo, incluindo o teorema de Pitágoras.
\item
  Resolver problemas que envolvam polígonos semelhantes.
\item
  Resolver problemas que envolvam aplicação das relações de
  proporcionalidade abrangendo retas paralelas cortadas por
  transversais.
\item
  Determinar o ponto médio de um segmento de reta ou a distância entre
  dois pontos quaisquer, dadas as coordenadas desses pontos no plano
  cartesiano.
\end{itemize}

\subsection{Habilidades da BNCC}
\begin{itemize}
\item EF07MA23, EF07MA24, EF07MA25.
\end{itemize}

\textbf{{[}Retas paralelas e retas transversais{]}}

{[}Paralelas:{]} retas que possuem o mesmo sentido, mas nunca se
encontram, ou seja, não possuem pontos em comum.

{[}Transversal:{]} reta que encontra simultaneamente duas ou mais retas
em pontos diferentes.

{[}Paralelas cortadas por uma transversal:{]} nessa situação dividimos o
plano em região interna e externa das retas paralelas e lados diferentes
em relação a transversal. Assim, surgem ângulo que se relacionam de
algumas formas, chamados de correspondentes, alternos e colaterais.

\begin{itemize}
\item
  Ângulos correspondentes: ocupam uma mesma região da reta transversal e
  regiões diferentes das paralelas. Esses ângulos são congruentes.
\item
  Ângulos alternos: ocupam uma mesma região das retas paralelas e
  regiões diferentes da transversal. Podendo ser alternos internos ou
  alternos externos, esses ângulos são congruentes.
\item
  Ângulos colaterais: ocupam a mesma região da reta transversal e das
  paralelas. Podendo ser colaterais internos ou colaterais externos,
  esses ângulos são suplementares.
\end{itemize}

\textbf{{Triângulo}}

Considerando A, B e C três pontos não colineares, definimos como
triângulo a união desses pontos por meio de segmentos de reta.

\begin{figure}
\centering
\includegraphics[width=3.53174in,height=1.806in]{./imgSAEB_7_MAT/media/image41.png}
\caption{Interface gráfica do usuário, Aplicativo, Word Descrição gerada
automaticamente}
\end{figure}

$A, B, C$ \rightarrow são vértices

$a, b, c$ \rightarrow são lados

$\alpha,\beta,\gamma \rightarrow $são os ângulos

\emph{Condição de existência}: para um triângulo ser possível de
construir é preciso satisfazer relações sobre o tamanho dos lados e dos
ângulos. São elas,

$ a < b + c\ b < a + c\ c < b + c\ $
$\alpha + \beta + \gamma = 180{^\circ}$

\textbf{Observação:} dependendo do contexto basta aplicar a desigualdade
para o maior lado, se ele satisfizer já é garantido que o triângulo
existe. Assim, sendo \textbf{{a}} o maior lado do triângulo:

$\left| b - c \right| < a < b + c$

\emph{Rigidez do triângulo:} como os vértices dos triângulos definem um
único plano, isso os tornam figuras planas com estabilidade, ou seja,
são polígonos rígidos, são fortes e não se deformam de forma fácil. Por
isso, são utilizados em construções arquitetônicas, como telhados,
estruturas metálicas, em objetos de uso diário e aparecem também nas
artes plásticas.

\section{Atividades}

\num{1} Complete as frases abaixo de acordo com os ângulos gerados por duas
paralelas cortadas por uma transversal:

\begin{escolha}
\item Ângulos \rosa{colaterais internos} estão na região interna das paralelas e no mesmo lado
em relação à transversal.
\item Ângulos \rosa{alternos externos} estão na região externa das paralelas e em lados
contrários em relação à transversal.
\item Ângulos correspondentes estão em regiões \rosa{opostas} das paralelas e em
regiões \rosa{iguais} em relação à transversal.
\end{escolha}

\num{2} Considere duas retas paralelas cortadas por uma transversal.
Determine a relação entre os ângulos formados por essa configuração.

\reduline{Quando duas retas paralelas são cortadas por uma transversal, são
formados vários pares de ângulos. Esses ângulos possuem relações
específicas entre si, que podem ser classificadas em três tipos
principais:

Ângulos Correspondentes: São os pares de ângulos que ocupam a mesma
posição em relação às retas paralelas. Esses ângulos têm a mesma medida
e são correspondentes um ao outro.

Ângulos Alternos Internos: São os pares de ângulos que estão no interior
das retas paralelas, mas em lados opostos da transversal. Esses ângulos
têm a mesma medida e são chamados de ângulos alternos internos. Por
exemplo, os ângulos 3 e 6 na figura acima.

Ângulos Alternos Externos: São os pares de ângulos que estão no exterior
das retas paralelas, mas em lados opostos da transversal. Esses ângulos
têm a mesma medida e são chamados de ângulos alternos externos. Por
exemplo, os ângulos 2 e 7 na figura acima.

Portanto, ao identificar uma configuração de retas paralelas cortadas
por uma transversal, podemos determinar as relações entre os ângulos
formados utilizando os conceitos de ângulos correspondentes, ângulos
alternos internos e ângulos alternos externos. Essas relações são
fundamentais para o estudo da geometria das retas paralelas e
transversais.\hfill}

\num{3} Dois triângulos são semelhantes, e a medida do lado do primeiro
triângulo é 5 cm. Se a medida do lado correspondente do segundo
triângulo é 10 cm, qual é a razão de semelhança entre os triângulos?

\reduline{Quando dois polígonos são semelhantes, isso significa que eles possuem
ângulos correspondentes iguais e lados correspondentes proporcionais.

Nesse problema, sabemos que dois triângulos são semelhantes e que a
medida do lado do primeiro triângulo é 5 cm, enquanto a medida do lado
correspondente do segundo triângulo é 10 cm. Para determinar a razão de
semelhança entre os triângulos, devemos comparar os lados
correspondentes.

Se a medida do lado do primeiro triângulo é 5 cm, e a medida do lado
correspondente do segundo triângulo é 10 cm, podemos escrever a
proporção:

$5 cm / 10 cm = x$

Para encontrar o valor de x, devemos simplificar a proporção:

$\frac {1}{2} = x$\hfill}

\num{4} Um poste de 6 metros de altura projeta uma sombra de 3 metros no chão
em um determinado momento do dia. Ao mesmo tempo, um edifício nas
proximidades projeta uma sombra de 15 metros no chão. Sabendo que o
poste e o edifício estão em uma configuração de retas paralelas cortadas
por uma transversal, qual é a altura do edifício?

\reduline{Nesse problema, temos uma situação em que um poste e um edifício estão
em uma configuração de retas paralelas cortadas por uma transversal. As
sombras projetadas no chão nos fornecem informações sobre as relações de
proporcionalidade entre as alturas dos objetos e os comprimentos das
sombras.

Sabemos que o poste possui 6 metros de altura e projeta uma sombra de 3
metros no chão. Da mesma forma, o edifício projeta uma sombra de 15
metros no chão. Podemos usar essas informações para estabelecer uma
proporção entre as alturas dos objetos e os comprimentos das sombras.

Seja x a altura do edifício. Podemos estabelecer a proporção:

$\frac {6}{3} = \frac {x}{15}$

Simplificando a proporção, temos:

$2 = \frac {x}{15}$

Multiplicando ambos os lados por 15, obtemos:

$30 = x$

Portanto, a altura do edifício é de 30 metros.\hfill}

\num{5} Verifique se é possível construir um triângulo com os tamanhos de
lado a seguir.

\begin{escolha}
\item 11, 13, 21  \\ \\
\rosa{$| 13 - 11 | < \ 21\  < \ 13 + 11\  \rightarrow \ \ 2 < 21 < 24 $ portanto, o triângulo existe}
\item 8,5; 10,5; 20  \\ \\
\rosa{$| 10,5 - 8,5 | < \ 20\  < \ 10,5 + 8,5\  \rightarrow \ \ 2 < 20 < 19 $ portanto, o triângulo não existe}
\item 3, 6, 9  \\ \\
\rosa{$| 6 - 3 | < \ 9\  < \ 6 + 3\  \rightarrow \ \ 2 < 9 < 9 $ portanto o triângulo não existe}
\item 5, 6, 7  \\ \\
\rosa{$| 6 - 5 \right| < \ 7\  < \ 6 + 5\  \rightarrow \ \ 1 < 7 < 11 $ portanto, o triângulo existe}
\end{escolha}

%Professor, no caso da letra D, enfatize com os alunos que o lado tem que
%ser estritamente menor que a soma dos outros dois, e nesse caso é igual.

\num{6} Em um triângulo ABC, sabe-se que a medida do ângulo A é o triplo da
medida do ângulo B, e a medida do ângulo C é a metade da medida do
ângulo B. Além disso, a medida do lado AB é o dobro da medida do lado
AC. Determine as medidas dos ângulos e dos lados do triângulo ABC.

\reduline{Nesse problema, temos um triângulo ABC com informações sobre as medidas
dos ângulos e dos lados. Podemos utilizar as propriedades e relações
existentes entre os elementos de um triângulo para determinar suas
medidas.

Sabemos que a medida do ângulo A é o triplo da medida do ângulo B. Seja
x a medida do ângulo B. Portanto, a medida do ângulo A é 3x.

Também é dito que a medida do ângulo C é a metade da medida do ângulo B.
Então, a medida do ângulo C é (1/2)x.

Sabemos que a soma dos ângulos internos de um triângulo é 180 graus.
Portanto, podemos escrever a equação:

$A + B + C = 180
$
Substituindo as medidas dos ângulos, temos:

$ 3x + x + \frac {1}{2}x = 180$

Simplificando a equação, temos:

$6x + 2x + x = 360$

$9x = 360$

$x = 40$

Agora que conhecemos a medida do ângulo B, podemos encontrar as medidas
dos outros ângulos:

Ângulo $A = 3x = 3 \times 40 = 120$ graus;

 Ângulo $C =
 \frac {1}{2}x = {1}{2} \times 40 = 20$ graus.

Além disso, sabemos que o lado AB é o dobro do lado AC. Seja y a medida
do lado AC. Portanto, o lado AB tem medida 2y.

Com base nessas informações, encontramos as medidas dos ângulos e dos
lados do triângulo ABC:

Ângulo A = 120 graus Ângulo B = 40 graus Ângulo C = 20 graus Lado AB =
2y Lado AC = y

A solução é justificada utilizando as propriedades e relações existentes
entre os elementos de um triângulo. As medidas dos ângulos foram
determinadas com base nas relações dadas e na soma dos ângulos internos.
As medidas dos lados foram determinadas com base na relação dada entre
os lados AB e AC. Essas determinações são consistentes com as
propriedades e relações existentes em triângulos, permitindo obter as
medidas solicitadas.\hfill}

\num{7} Classifique o seguinte quadrilátero em relação aos seus ângulos
internos: ABCD, onde os ângulos A, B, C e D medem 90 graus cada.

\reduline{Para classificar um quadrilátero em relação aos seus ângulos internos, é
necessário analisar as medidas dos ângulos formados pelos seus lados.

No problema dado, os ângulos A, B, C e D medem 90 graus cada. Essa
medida é característica de ângulos retos, que são ângulos de 90 graus.

Um quadrilátero que possui todos os ângulos internos de 90 graus é
chamado de quadrado. Portanto, o quadrilátero ABCD com ângulos A, B, C e
D medindo 90 graus cada é um quadrado.

Essa classificação é justificada pela definição de um quadrado, que é um
quadrilátero com quatro lados congruentes e quatro ângulos retos. No
problema, os ângulos internos do quadrilátero ABCD possuem medidas de 90
graus, cumprindo a definição de um quadrado. Portanto, a resposta
correta é que ABCD é um quadrado.\hfill}

\num{8} Considere as retas r, s e t no plano cartesiano, dadas pelas
seguintes equações:

$r: y = 2x + 3$ \\
$s: y = -2x - 1$ \\
$t: y = 2x - 4$ \\

Identifique as relações existentes entre as retas r, s e t: se são
concorrentes, paralelas ou perpendiculares.

\reduline{Para identificar as relações entre as retas r, s e t, devemos analisar
as inclinações de suas equações.

A inclinação de uma reta é representada pelo coeficiente angular, que é
o valor que acompanha a variável x na equação da reta.

Na equação da reta $r: y = 2x + 3$, o coeficiente angular é 2. Na equação
da reta $s: y = -2x - 1$, o coeficiente angular é -2. Na equação da reta
$t: y = 2x - 4$, o coeficiente angular é 2.

Podemos observar que as retas r e t possuem o mesmo coeficiente angular,
que é 2. Isso indica que as retas r e t são paralelas, pois têm a mesma
inclinação.

Por outro lado, a reta s possui um coeficiente angular -2, que é o
inverso aditivo de 2. Isso indica que a reta s é perpendicular às retas
r e t.

Portanto, podemos concluir que as relações entre as retas r, s e t são
as seguintes:

A reta r e a reta t são paralelas. A reta s é perpendicular às retas r e
t.

Essa classificação é justificada pela definição de retas paralelas e
perpendiculares. Duas retas são paralelas se têm a mesma inclinação, ou
seja, seus coeficientes angulares são iguais. No caso das retas r e t,
ambas têm coeficiente angular 2, indicando que são paralelas.

Duas retas são perpendiculares quando seus coeficientes angulares são
inversos aditivos, ou seja, quando um é o oposto do recíproco do outro.
No caso da reta s, seu coeficiente angular é -2, que é o inverso aditivo
de 2, coeficiente angular das retas r e t, indicando que s é
perpendicular a r e t.\hfill}

\num{9} Em um triângulo de lados 32 e 38, e ângulos
$85{^\circ}$ e $42{^\circ}$. Qual o menor valor de lado e o tamanho de
ângulo para que esse triângulo exista?

\reduline{Em relação ao ângulo $85{^\circ} + 42{^\circ} = 127{^\circ}$, portanto
o ângulo deve medir $53{^\circ}$.

Quanto ao lado, vamos chamar de x o lado desconhecido, logo \\

$ x < 32 + 38  \\
x < 70 38 < x + 32 \\ 
6 < x 32 < x + 38 \\
- 6 < x \rightarrow 6 < x < 70 $\\
o menor valor do lado deve ser 7.

\hfill}

\num{10} Considere um paralelogramo ABCD, onde as diagonais AC e BD se
intersectam em um ponto E. Se a medida do ângulo AED é 110 graus,
determine a medida dos ângulos internos do paralelogramo ABCD.

\reduline{Para resolver esse problema, podemos utilizar as propriedades dos
ângulos formados por retas paralelas cortadas por uma transversal.

No caso do paralelogramo ABCD, as diagonais AC e BD se intersectam em um
ponto E, formando quatro ângulos internos.

Sabemos que o ângulo AED mede 110 graus. Como as diagonais de um
paralelogramo se dividem em ângulos congruentes, podemos inferir que o
ângulo BEC, que é oposto ao ângulo AED, também mede 110 graus.

Além disso, a soma dos ângulos internos de um paralelogramo é sempre 360
graus. Portanto, podemos escrever a equação:

Ângulo A + Ângulo B + Ângulo C + Ângulo D = 360

Sabendo que os ângulos AED e BEC medem 110 graus cada, podemos
substituir na equação:

110 + Ângulo B + Ângulo C + 110 = 360

Simplificando a equação, temos:

Ângulo B + Ângulo C = 360 - 110 - 110 Ângulo B + Ângulo C = 140

Sabemos que os ângulos opostos em um paralelogramo são congruentes.
Portanto, Ângulo B e Ângulo D também possuem a mesma medida.

Podemos reescrever a equação:

Ângulo B + Ângulo D = 140

A soma dos ângulos B e D é igual a 140 graus. Dividindo igualmente, cada
ângulo mede:

Ângulo B = $\frac {140}{2} = 70$ graus Ângulo $D = \frac {140}{2} =
70$ graus

Assim, encontramos a medida dos ângulos internos do paralelogramo ABCD:

Ângulo A = 110 graus Ângulo B = 70 graus Ângulo C = 140 graus Ângulo D =
70 graus \hfill}

\section{Treino}

\num{1} Em um triângulo retângulo ABC, onde o ângulo B é reto, determine a
relação correta entre as medidas dos lados:

\begin{escolha}
\item a² + b² = c² (Teorema de Pitágoras)
\item a + b = c (Propriedade da soma dos lados)
\item a = b (Lados congruentes em um triângulo retângulo)
\item a \textgreater{} b \textgreater{} c (Relação entre as medidas dos lados)
\end{escolha}

%\section{BNCC: EF07MA24 }
% -- Construir triângulos, usando régua e compasso,
% reconhecer a condição de existência do triângulo quanto à medida dos
% lados e verificar que a soma das medidas dos ângulos internos de um
% triângulo é 180°.
% SAEB: Resolver problemas que envolvam relações métricas do triângulo
% retângulo, incluindo o teorema de Pitágoras.

% A - correta, pois a² + b² = c² representa corretamente o Teorema de
% Pitágoras, que estabelece que, em um triângulo retângulo, o quadrado da
% hipotenusa (c) é igual à soma dos quadrados dos catetos (a e b).
% B - Incorreta, pois, na verdade, a soma dos quadrados dos catetos a e b
% é igual ao quadrado da hipotenusa c, como afirma o Teorema de Pitágoras.
% C - Incorreta, pois, em um triângulo retângulo, os lados a e b são os
% catetos, e eles podem ter medidas diferentes.
% D - Incorreta, pois não há uma relação específica entre as medidas dos
% lados a, b e c de um triângulo retângulo. As medidas podem variar
% dependendo do triângulo em questão.

\num{2} Considere dois polígonos, P e Q, com medidas proporcionais em todos
os lados. Determine a relação correta entre os ângulos internos dos
polígonos:

\begin{escolha}
  \item Os ângulos internos de P são congruentes aos ângulos internos de Q.
  \item Os ângulos internos de P são proporcionais aos ângulos internos de Q.
  \item Os ângulos internos de P são suplementares aos ângulos internos de Q.
  \item Os ângulos internos de P são complementares aos ângulos internos de Q.
\end{escolha}

%\section{BNCC: EF07MA23 }
% -- Verificar relações entre os ângulos formados por retas
% paralelas cortadas por uma transversal, com e sem uso de softwares de
% geometria dinâmica.
% SAEB: Resolver problemas que envolvam polígonos semelhantes.

% A - Correta, pois, em polígonos semelhantes, os ângulos internos
% correspondentes têm medidas iguais. Isso ocorre porque a semelhança
% entre os polígonos preserva a congruência dos ângulos internos.
% B - Incorreta, pois a proporção é uma relação entre as medidas dos lados
% dos polígonos semelhantes, não entre os ângulos internos.
% C - Incorreta, pois ângulos suplementares são aqueles que somam 180
% graus, mas não há uma relação de suplementaridade entre os ângulos
% internos de polígonos semelhantes.
% D - Incorreta, pois ângulos complementares são aqueles que somam 90
% graus, mas não há uma relação de complementaridade entre os ângulos
% internos de polígonos semelhantes.

\num{3} Considere duas retas paralelas cortadas por uma transversal, formando
vários ângulos. Determine a relação correta entre os ângulos formados:

\begin{escolha}
\item Ângulos alternos internos são congruentes.
\item Ângulos adjacentes são suplementares.
\item Ângulos correspondentes são congruentes.
\item Ângulos consecutivos são complementares.
\end{escolha}

%%\section{BNCC: EF07MA23 }
% -- Verificar relações entre os ângulos formados por retas
% paralelas cortadas por uma transversal, com e sem uso de softwares de
% geometria dinâmica.
% SAEB: Resolver problemas que envolvam relações entre ângulos formados
% por retas paralelas cortadas por uma transversal, ângulos internos ou
% externos de polígonos ou cevianas (altura, bissetriz, mediana,
% mediatriz) de polígonos.

% A - Incorreta, pois, embora essa afirmação seja verdadeira, ela se
% refere aos ângulos formados pelas retas paralelas e uma transversal, não
% aos ângulos internos de polígonos.
% B - Incorreta, pois ângulos adjacentes são aqueles que têm um lado em
% comum, mas não necessariamente são suplementares. Essa relação é
% verdadeira apenas para ângulos lineares ou ângulos opostos pelo vértice,
% não para todos os ângulos formados por retas paralelas cortadas por uma
% transversal.
% C - Correta, pois ângulos correspondentes são pares de ângulos que estão
% em lados opostos da transversal e em posições correspondentes em relação
% às retas paralelas. Esses ângulos têm medidas iguais.
% D - Incorreta, pois ângulos consecutivos são ângulos que possuem o mesmo
% vértice e um lado em comum, mas não necessariamente são complementares.
% Essa relação é verdadeira apenas para ângulos suplementares, não para
% todos os ângulos formados por retas paralelas cortadas por uma
% transversal.

\begin{comment}

\section{Deslocamentos}
\markboth{Módulo 10}{}

\section{Habilidade do SAEB }
\begin{itemize}
\item Descrever ou esboçar deslocamento de pessoas e/ou
de objetos em representações bidimensionais (mapas, croquis etc.),
plantas de ambientes ou vistas, de acordo com condições dadas.
\end{itemize}

\textbf{{[}Deslocamento{]}}

Compreender conceitos como trajetória, direção e sentido é essencial
para se localizar e descrever posições e deslocamentos, seja em
ambientes menores, como a sala de aula ou em nossa própria casa, quanto
em ambientes maiores, como a cidade.

Para interpretar uma representação espacial, são necessários pontos de
referência, como um edifício alto, uma praça, um cruzamento ou uma
interseção de ruas.

Para a construção da representação espacial, é necessário ter em mente
as noções de trajetória, direção e sentido para indicar com precisão a
posição e o deslocamento de objetos ou pessoas, ter ideia de como
transcrever a visualização tridimensional para a bidimensional. Também é
importante utilizar símbolos e legendas adequados para tornar a
representação compreensível.

O deslocamento é feito por meio de uma trajetória, definida como o
caminho percorrido desde o ponto de partida até o ponto de chegada,
podendo apresentar deslocamentos em linha reta ou em linha curva. A
direção indica a orientação em que o deslocamento ocorre, enquanto o
sentido indica a orientação em relação ao ponto de partida.

\section{Atividades}

\num{1} O que é um ponto de referência quando falamos de deslocamento? Dê
exemplos.

R:

Ponto de referência é um local a ser indicado para que fique mais claro
o caminho a ser percorrido. Pode ser considerado um ponto de referência
uma loja, um edifício, uma rua, uma placa etc.

\num{2} Considere um mapa de um parque, onde três amigos, Alice, Bob e Carol,
estão localizados em diferentes pontos do parque. Eles decidem se
deslocar seguindo as seguintes instruções:

I. Alice se desloca 3 unidades para o norte. II. Bob se desloca 5
unidades para o leste. III. Carol se desloca 2 unidades para o sul e
depois 4 unidades para o oeste.

R:

Alice está a noroeste de Bob e a leste de Carol. Ao se deslocar para o
norte, Alice estará acima de Bob, o que indica uma posição noroeste em
relação a ele. Além disso, ao se deslocar para o oeste, Carol estará a
oeste de Alice, indicando que Alice está a leste de Carol.

\num{3} João está em um parque e deseja chegar à lanchonete que fica do outro
lado do lago. Ele sabe que para chegar à lanchonete, precisa percorrer
200 metros para o norte até a ponte, atravessar a ponte e percorrer mais
150 metros para o sul.

Considerando que João está em um mapa do parque, qual é a melhor
representação de seu deslocamento?

R:

O aluno deverpa criar um esboço de um caminho que mostra João indo 200
metros para o norte, depois atravessando a ponte e indo 150 metros para
o sul é a correta. Essa representação descreve corretamente o
deslocamento de João, indicando que ele deve seguir para o norte até a
ponte, atravessá-la e, em seguida, ir para o sul até a lanchonete.

\num{4} Considere um mapa de um shopping center, representado em um plano
cartesiano. João está localizado no ponto (2, 5) e decide se deslocar
seguindo as seguintes instruções:

I. João se desloca 3 unidades para o norte. II. João se desloca 4
unidades para o leste. III. João se desloca 2 unidades para o sul.

R:

O deslocamento de 3 unidades para o norte a partir da posição inicial
(2, 5) coloca João na coordenada (2, 8). O deslocamento subsequente de 4
unidades para o leste o leva à coordenada (6, 8). Por fim, o
deslocamento de 2 unidades para o sul o coloca na coordenada final (6,
\num{7}.

\num{5} Considere um mapa de um labirinto representado em uma grade
bidimensional, onde cada célula representa uma posição no labirinto.
Você está localizado na célula (0, 0) e deseja chegar à célula (4, 4),
que é a saída do labirinto. No entanto, existem algumas restrições de
movimento:

Você só pode se mover para cima (C), para baixo (B), para a esquerda (E)
ou para a direita (D). Você não pode atravessar paredes, representadas
por células bloqueadas no mapa.

Determine a sequência correta de movimentos para chegar à saída do
labirinto:

R:

A sequência correta é E, E, C, C, D, D, B, B. Essa sequência de
movimentos permite que você se desloque do ponto inicial (0, 0) para o
ponto de saída (4, 4) seguindo um caminho válido pelo labirinto

\num{6} Considere um mapa de um labirinto retangular, representado por uma
grade de 5 por 5. Um objeto está inicialmente localizado no ponto A(2,
\num{3} e deseja chegar ao ponto B(5, 1) seguindo apenas movimentos para
cima, para baixo, para a esquerda ou para a direita.

Determine o caminho correto para o objeto chegar ao ponto B a partir do
ponto A.

R:

A resposta correta é A(2, 3) -\textgreater{} A(3, 3) -\textgreater{}
A(3, 2) -\textgreater{} B(4, 2) -\textgreater{} B(5, 2) -\textgreater{}
B(5, 1). Seguindo essa sequência de movimentos, o objeto avança para a
direita e para baixo até chegar ao ponto B, percorrendo o caminho
correto no labirinto.

\num{7} Considere um mapa de uma cidade, onde Maria está localizada na
posição A e precisa chegar à posição C. Ela decide seguir as seguintes
instruções:

I. Maria se desloca 3 quadras para o norte. II. Em seguida, ela se
desloca 5 quadras para o leste. III. Por fim, ela se desloca 2 quadras
para o sul.

Descreva o deslocamento de Maria em relação à posição inicial A.

R:

Ao se deslocar para o norte, Maria estará acima da posição inicial A. Em
seguida, ao se deslocar para o leste, Maria estará a leste da posição
inicial A. Por fim, ao se deslocar para o sul, Maria estará duas quadras
abaixo da posição inicial A, indicando uma posição sudeste em relação a
A.

\num{8} Uma sala retangular possui as seguintes medidas: 6 metros de largura
e 8 metros de comprimento. A escala utilizada na planta do ambiente é de
1 cm para cada 2 metros. Determine a medida do comprimento da sala na
planta.

R:

Para resolver esse problema, devemos considerar a escala utilizada na
planta do ambiente. Sabemos que 1 cm na planta representa 2 metros na
realidade.

O comprimento da sala é de 8 metros. Portanto, para determinar a medida
correspondente na planta, devemos dividir esse valor pela escala de
conversão:

\frac {8}{2} = 4 cm

Assim, a medida do comprimento da sala na planta é de 4 cm.

\num{9} Você está construindo uma nova casa e deseja projetar o layout da
cozinha. Para isso, você recebeu uma planta baixa simplificada da
cozinha, representada por um retângulo. As dimensões do retângulo são
fornecidas da seguinte forma:

Comprimento: 6 metros Largura: 4 metros

Além disso, você precisa posicionar a pia e o fogão de acordo com as
seguintes condições:

A pia deve ser colocada na parede mais próxima à porta de entrada da
cozinha. O fogão deve ser colocado na parede oposta à pia.

Determine as coordenadas (x, y) da posição correta para a pia e o fogão
na planta da cozinha:

R:

A resposta correta é Pia: (2, 0) e Fogão: (4, 4).

De acordo com a primeira condição, a pia deve ser colocada na parede
mais próxima à porta de entrada da cozinha. No retângulo que representa
a planta da cozinha, a parede mais próxima à porta de entrada é a parede
com y = 0 (ou seja, no ``chão'' da planta).

De acordo com a segunda condição, o fogão deve ser colocado na parede
oposta à pia. No retângulo, a parede oposta à parede com y = 0 é a
parede com y = 4 (ou seja, no ``teto'' da planta).

Assim, a posição correta para a pia é (2, 0) e para o fogão é (4, 4).

\num{0} Faça uma descrição do trajeto da sua casa até a escola usando o
conceito de direção, trajetória e referência.

Professor, nesse exercício, oriente cada aluno para que faça a descrição
do trajeto da forma mais clara possível.

R: Pessoal.

\section{Treino}

\num{1} Um grupo de amigos está participando de uma caminhada em uma trilha.
Cada pessoa segue um caminho específico e realiza as seguintes ações:

I. Júlia caminha 500 metros para o norte. II. Carlos caminha 300 metros
para o leste. III. Maria caminha 200 metros para o sul e depois 400
metros para o oeste.

Determine a posição final de cada pessoa na trilha:

\item
  Júlia está a nordeste de Carlos e a oeste de Maria.
\item
  Júlia está a noroeste de Carlos e a leste de Maria.
\item
  Júlia está a sudeste de Carlos e a oeste de Maria.
\item
  Júlia está a sudoeste de Carlos e a leste de Maria.
\end{enumerate}

% SAEB: Descrever ou esboçar deslocamento de pessoas e/ou de objetos em
% representações bidimensionais (mapas, croquis etc.), plantas de
% ambientes ou vistas, de acordo com condições dadas.

% A - Correta, pois, ao caminhar 500 metros para o norte, Júlia estará
% acima da posição inicial de Carlos. Isso indica que Júlia está a
% nordeste de Carlos. Carlos caminha 300 metros para o leste, o que o
% coloca à direita da posição inicial de Júlia. Maria caminha 200 metros
% para o sul e depois 400 metros para o oeste. Isso a coloca abaixo da
% posição inicial de Júlia e à esquerda da posição inicial de Carlos.
% Portanto, Maria está a oeste de Alice. Assim, a posição final de cada
% pessoa pode ser descrita da seguinte forma: Júlia está a nordeste de
% Carlos e a oeste de Maria.
% B - Incorreta, pois não seguiu corretamente as direções e posições
% finais das pessoas na trilha.
% C - Incorreta, pois não seguiu corretamente as direções e posições
% finais das pessoas na trilha.
% D - Incorreta, pois não seguiu corretamente as direções e posições
% finais das pessoas na trilha.

\num{2} Você recebeu uma planta baixa de uma sala retangular com as seguintes
dimensões: Comprimento de 10 metros e largura de 6 metros. De acordo com a planta baixa, você precisa identificar a localização correta de uma janela na sala. As seguintes condições devem ser
consideradas:

I. A janela deve estar posicionada na parede mais próxima à entrada da
sala. II. A janela deve ser colocada a uma distância igual a 4 metros da
parede oposta à entrada.

Determine a posição correta da janela na planta da sala:

a) A janela está localizada na parede de 10 metros, a 4 metros da
entrada.
b) A janela está localizada na parede de 6 metros, a 4 metros da
entrada.
c) A janela está localizada na parede de 10 metros, a 2 metros da
entrada.
d) A janela está localizada na parede de 6 metros, a 2 metros da
entrada.

% SAEB: Descrever ou esboçar deslocamento de pessoas e/ou de objetos em
% representações bidimensionais (mapas, croquis etc.), plantas de
% ambientes ou vistas, de acordo com condições dadas.

% A - Incorreta, pois não foram seguidas corretamente as condições
% especificadas para a localização da janela na planta da sala.
% B - Correta. pois, de acordo com a primeira condição, a janela deve
% estar posicionada na parede mais próxima à entrada da sala. No retângulo
% que representa a planta da sala, a parede mais próxima à entrada é a
% parede de 6 metros. De acordo com a segunda condição, a janela deve ser
% colocada a uma distância igual a 4 metros da parede oposta à entrada.
% Como a sala possui uma largura de 6 metros, a parede oposta à entrada é
% a parede de 10 metros. Portanto, a janela deve estar localizada a 4
% metros dessa parede. Assim, a posição correta da janela na planta da
% sala é na parede de 6 metros, a 4 metros da entrada, conforme descrito
% na alternativa B.
% C - Incorreta, pois não foram seguidas corretamente as condições
% especificadas para a localização da janela na planta da sala.
% D - Incorreta, pois não foram seguidas corretamente as condições
% especificadas para a localização da janela na planta da sala.

\num{3} Considere um mapa de uma cidade em escala, no qual estão
representados dois pontos A e B. João deseja se deslocar de A para B
seguindo as seguintes condições:

I. Ele deve caminhar 500 metros para o norte. II. Em seguida, deve virar
à esquerda e caminhar 300 metros para o leste. III. Por fim, deve seguir
mais 200 metros para o sul.

Determine a posição final de João no mapa:

a) João está a nordeste do ponto B.
b) João está a noroeste do ponto B.
c) João está a sudeste do ponto B.
d) João está a sudoeste do ponto B.

% SAEB: Descrever ou esboçar deslocamento de pessoas e/ou de objetos em
% representações bidimensionais (mapas, croquis etc.), plantas de
% ambientes ou vistas, de acordo com condições dadas.

% A - Incorreta, pois as direções não foram seguidas corretamente.
% B - Incorreta, pois as direções não foram seguidas corretamente.
% C - Incorreta, pois as direções não foram seguidas corretamente.
% D - Correta, pois, ao caminhar 500 metros para o norte a partir de A,
% João estará acima do ponto B. Isso indica que João está ao norte de B.
% Em seguida, João vira à esquerda e caminha 300 metros para o leste. Isso
% o coloca à direita da posição inicial de B. Por fim, João segue mais 200
% metros para o sul. Isso o coloca abaixo da posição inicial de B.
% Portanto, João está ao sul de B.

\section{Estatística e sua
Representação}
\markboth{Módulo 11}{}

\section{Habilidades do SAEB }
\begin{itemize}
\item Identificar os indivíduos (universo ou
população-alvo da pesquisa), as variáveis e os tipos de variáveis
(quantitativas ou categóricas) em um conjunto de dados.
\item
  Representar ou associar os dados de uma pesquisa estatística ou de um
  levantamento em listas, tabelas (simples ou de dupla entrada) ou
  gráficos (barras simples ou agrupadas, colunas simples ou agrupadas,
  pictóricos, de linhas, de setores, ou em histograma).
\item
  Inferir a finalidade da realização de uma pesquisa estatística ou de
  um levantamento, dada uma tabela (simples ou de dupla entrada) ou
  gráfico (barras simples ou agrupadas, colunas simples ou agrupadas,
  pictóricos, de linhas, de setores ou em histograma) com os dados dessa
  pesquisa.
  \item
  Interpretar o significado das medidas de tendência central (média,
  aritmética simples, moda e mediana) ou da amplitude.
\item
  Calcular os valores de medidas de tendência central de uma pesquisa
  estatística (média aritmética simples, moda ou mediana).
\item
  Resolver problemas que envolvam dados estatísticos apresentados em
  tabelas (simples ou de dupla entrada) ou gráficos (barras simples ou
  agrupadas, colunas simples ou agrupadas, pictóricos, de linhas, de
  setores ou em histograma).
\item
  Argumentar ou analisar argumentações/conclusões com base nos dados
  apresentados em tabelas (simples ou de dupla entrada) ou gráficos
  (barras simples ou agrupadas, colunas simples ou agrupadas,
  pictóricos, de linhas, de setores ou em histograma).
\item
  Explicar/descrever os passos para a realização de uma pesquisa
  estatística ou de um levantamento.
\end{itemize}

\section{Habilidades da BNCC }
\begin{itemize}
\item EF07MA35
\item EF07MA37
\end{itemize}

% Professor, neste módulo, é importante focar bastante nos gráficos para
% que os exemplos fiquem bem explícitos. Aliás, é o momento de fazer com
% que os alunos compreendam que todo processo científico depende de um
% rigor e uma sequência de passos.

\textbf{População e amostra}

População e amostra são partes que compõem uma pesquisa científica. A
população está relacionada ao todo, ou seja, ao público-alvo, e amostra
é uma parte desse público-alvo que seja relevante para a pesquisa. É bem
importante relacionar se a amostra representa bem o que a pesquisa
propõe.

\textbf{~Exemplo:} Será realizada uma pesquisa sobre a intenção de voto
para o próximo presidente da república no Brasil. É possível entrevistar
todos os eleitores? Como o espaço de tempo entre candidatura e eleição é
curto, não seria possível, mas é importante que a amostra seja
significativa, uma vez que influencia um grupo de pessoas. Logo, a
população da pesquisa são todos eleitores do Brasil e a amostra
corresponde ao grupo de pessoas que respondeu à pesquisa.

\textbf{Variáveis quantitativas e qualitativas~}

Variáveis quantitativas são os dados que podem ser expressos em números.
Já as variáveis qualitativas são aquelas que não aparecem dessa forma.

As variáveis qualitativas são divididas em dois grupos, ordinais e
nominais.

\textbf{Exemplos:} Variáveis quantitativas \rightarrow Peso, altura,
notas, idade, salário, pressão.

Variáveis qualitativas \rightarrow \ \ Sexo, cor dos olhos, classe
social.

\textbf{Tipos de gráficos}

Colunas

% %Paulo: inserir a imagem. Disponível em:
% https://br.freepik.com/vetores-gratis/bares-modelo-de-grafico\_763948.htm\#query=gr\%C3\%A1fico\%20de\%20colunas\&position=12\&from\_view=search\&track=aisais.
% Acesso em: 18 maio 2023.

Barras

% %Paulo: inserir a imagem. Disponível em:
% https://br.freepik.com/vetores-gratis/fundo-colorido-infografico\_938707.htm\#query=gr\%C3\%A1fico\%20de\%20barras\%20horizontais\&position=7\&from\_view=search\&track=ais
% Acesso em: 18 maio 2023.

Gráfico de setores

% %Paulo: inserir a imagem. Disponível em:
% https://br.freepik.com/vetores-gratis/carregamento-circular-grafico-progresso\_796523.htm\#query=gr\%C3\%A1fico\%20pizza\&position=5\&from\_view=search\&track=ais
% Acesso em: 18 maio 2023.

Gráfico de linhas ou histograma

% %Paulo: inserir a imagem. Disponível em:
% https://br.freepik.com/vetores-gratis/ilustracao-do-grafico-de-analise-de-dados\_2806796.htm\#query=gr\%C3\%A1fico\%20de\%20linhas\&position=17\&from\_view=search\&track=ais

Os gráficos e as tabelas auxiliam a visualização de uma pesquisa, pois,
por meio de sua leitura, compreendemos os dados apresentados. Os
gráficos de setores geralmente são utilizados para variáveis
qualitativas e os de colunas, barras e linhas, para variáveis
quantitativas. Através dos gráficos e das tabelas é possível compreender
a correlação que o autor deseja.

\textbf{Exemplos:} Fernando tem uma loja de veículos usados e pretende
organizar a venda de carros de acordo com a marca. Com uma tabela
simples, é possível ver qual marca é a mais vendida da loja. Modelo de
tabela:

%Paulo: criar uma tabela com as informações abaixo.

\begin{longtable}[]{@{}ll@{}}
\toprule
\endhead
Marcas & Unidades vendidas\tabularnewline
Honda & 16\tabularnewline
Toyota & 5\tabularnewline
Chevrolet~ & 11\tabularnewline
\bottomrule
\end{longtable}

\textbf{Medidas de tendência central}

As medidas de tendência central levam esse nome porque têm o intuito de
encontrar os valores de equilíbrio de uma pesquisa. Ou seja, os valores
que estão no meio do conjunto de dados. As medidas de tendência central
são:

Moda: É o dado que mais se repete no conjunto de dados. É possível
ocorrer mais de uma moda, ou seja, um dado que se repete a mesma
quantidade de vezes que outro. Também é possível que não haja moda.

Média aritmética: Calculada pela soma de todos os dados do conjunto
dividida pela quantidade de dados do conjunto.

Mediana: Para calcular a mediana, é preciso ordenar os dados do conjunto
em ordem crescente. Feito isso, em conjuntos de dados com quantidade
ímpar, devemos localizar o dado do meio. Para os conjuntos de dados com
quantidade par, devemos encontrar a média aritmética dos dois dados
localizados no meio.

Obs: Na maioria das vezes, não é possível calcular a média e mediana em
variáveis qualitativas.

Exemplos: Thiago quer descobrir em qual matéria ele se saiu melhor no
bimestre, Português ou Inglês. Para isso, ele reuniu as notas das quatro
avaliações e calculou a moda, a média e a mediana.

\begin{longtable}[]{@{}ll@{}}
\toprule
\endhead
Inglês &\tabularnewline
1ª avaliação & 8,2\tabularnewline
2ª avaliação & 8\tabularnewline
3ª avaliação & 8,2\tabularnewline
4ª avaliação & 9,7\tabularnewline
\bottomrule
\end{longtable}

\text{Moda} = \ 8,2

Mé\text{dia} = \frac{8,2 + 8 + 8,2 + 9,7}{4} = 8,525

\text{Mediana} = \ 8  - 8,2  - 8,2  - 9,7\ 

Como o conjunto de dados é par:

\text{Mediana}\  = \frac{\ 8,2 + 8,2}{2} = \ 8,2

\begin{longtable}[]{@{}ll@{}}
\toprule
\endhead
Português~ &\tabularnewline
1ª avaliação & 10\tabularnewline
2ª avaliação & 6,8\tabularnewline
3ª avaliação & 5\tabularnewline
4ª avaliação & 10\tabularnewline
\bottomrule
\end{longtable}

\text{Moda} = \ 10

Mé\text{dia} = \frac{10 + 6,8 + 5 + 10}{4} = 7,95

\text{Mediana} = \ 5 - \ 6,8  - 10 - \ 10\ 

Como o conjunto de dados é par:

\text{Mediana}\  = \frac{\ 6,8 + 10}{2} = \ 8,4

Podemos concluir que, ao analisar as médias, Thiago obteve melhor
rendimento em Inglês. Uma característica interessante das medidas de
tendência central é que, quanto maior a variedade dos dados, ou seja,
uns muito grandes e outros muito pequenos, menos eficazes elas serão.

\textbf{Critérios para realizar uma pesquisa estatística~}

Para realizar pesquisas com cunho estatístico, é necessário seguir
alguns critérios. São eles:

\textbf{Delimitação do problema}: Determinar como uma pesquisa de coleta
de dados pode resolver o problema.

\textbf{Planejamento:} Traçar como será feito o levantamento dos dados.

\textbf{Coleta de dados:} Coletar os dados de acordo com o planejamento
feito, seja por entrevista, por documentos do governo, por aferições ou
medidas, assim por diante.

\textbf{Organização dos dados:} Contagem de dados.

\textbf{Apresentação dos dados:} Reunião dos dados em tabelas ou
gráficos para organização dos dados.

\textbf{Análise dos dados}: Resultados e discussões. O que podemos
aprender a respeito do problema inicial a partir dos dados coletados?

\num{1} Classifique as variáveis como qualitativas ou quantitativas:

a) Profissão

b) Religião

c) Número de visualizações em vídeos

d) Cor dos cabelos

e) Batimentos cardíacos

f) Temperatura

R:

a) Qualitativa

b) Qualitativa

c) Quantitativa

d) Qualitativa

e) Quantitativa

f) Quantitativa

\num{2} O diretor da escola Brilho do Conhecimento anunciou que as eleições
para representante de sala no Ensino Médio seriam realizadas. Três
alunos dentre os 25 da turma do 1º ano se interessaram e se
candidataram. Luana, que não era candidata, resolveu perguntar para 10
amigos em quem pretendiam votar. Qual é a população e a amostra deste
experimento de Luana?

R:

A população do experimento são os 25 alunos. A amostra são os 10 amigos
de Luana.

\num{3} Uma lanchonete resolveu marcar quantos lanches eram vendidos a cada
noite para ter maior controle na hora de fazer as compras. Analise a
tabela e responda:

\begin{longtable}[]{@{}ll@{}}
\toprule
\endhead
Dias da semana & Quantidade de vendas\tabularnewline
Quinta-feira~ & 30\tabularnewline
Sexta-feira & 58\tabularnewline
Sábado & 63\tabularnewline
Domingo & 21\tabularnewline
\bottomrule
\end{longtable}

a) Qual é a média de vendas da lanchonete?

b) Em qual dia foram vendidos menos lanches?

R:

a) A média de vendas é dada por \frac{30 + 58 + 63 + 21}{4} = \ 43
lanches por dia.

b) Domingo

\begin{longtable}[]{@{}ll@{}}
\toprule
\num{4} Uma pesqui & sa foi realizada em uma escola para determinar as
preferências dos alunos em relação a diferentes esportes. Os resultados
foram registrados e agora precisam ser apresentados em forma de gráfico.
A tabela abaixo mostra o número de alunos que escolheram cada
esporte:\tabularnewline
\midrule
\endhead
Esporte & Número de Alunos\tabularnewline
Futebol & 30\tabularnewline
Basquete & 20\tabularnewline
Vôlei & 15\tabularnewline
Tênis & 10\tabularnewline
Natação & 5\tabularnewline
\bottomrule
\end{longtable}

Qual tipo de gráfico seria mais adequado para representar esses dados?

R:

O gráfico de colunas é adequado para representar dados categóricos, como
as preferências dos alunos em relação aos esportes. Cada esporte pode
ser representado por uma coluna vertical no gráfico, e a altura da
coluna corresponderá ao número de alunos que escolheram aquele esporte.

A tabela fornece os números específicos de alunos para cada esporte, e o
gráfico de colunas simples é uma maneira eficiente de visualizar e
comparar esses números. Cada coluna representará um esporte e a altura
da coluna será proporcional ao número de alunos.

\num{5} Um grupo de jovens optou por anotar a temperatura da cidade deles
todos os dias no mesmo horário. Qual é a moda, média e mediana das
temperaturas?

\begin{longtable}[]{@{}ll@{}}
\toprule
\endhead
Dia do mês~ & Temperatura\tabularnewline
01/03/2023 & 19º\tabularnewline
02/03/2023 & 18º\tabularnewline
03/03/2023 & 22º\tabularnewline
04/03/2023 & 22º\tabularnewline
05/03/2023 & 17º\tabularnewline
06/03/2023 & 29º\tabularnewline
07/03/2023 & 18º\tabularnewline
\bottomrule
\end{longtable}

R:

\text{MODA} = \ 18º\ e\ \ 22º

MÉ\text{DIA}\  = \frac{19 + 18 + 22 + 22 + 17 + 29 + 18}{7} \cong 20,7

\text{Mediana} = \ 17 - 18 - 18 - 19 - 22 - 22 - 29\ ,\ \text{logo},\ a\ \text{Mediana} = 19

\num{6} Enumere do primeiro ao último passo para se fazer uma pesquisa em
estatística.

\textbf{Análise dos dados}

\textbf{Planejamento}

\textbf{Apresentação dos dados}

\textbf{Delimitação do problema}

\textbf{Organização dos dados}

\textbf{Coleta de dados}

R:

(6) Análise dos dados

(2) Planejamento

(5) Apresentação dos dados

(1) Delimitação do problema

(4) Organização dos dados

(3) Coleta de dados

\num{7} Indique, em cada alternativa, a população e a amostra.

a) Turma do segundo ano, 3 pessoas que usam óculos do segundo ano

b) Eleitores no Brasil, eleitores de Minas Gerais

c) Macacos, fauna

R:

a) População: Turma do segundo ano;~amostra: 3 pessoas que usam óculos
do segundo ano.

b) População: Eleitores do Brasil; amostra: Eleitores de Minas Gerais.

c) População: Fauna; amostra: Eleitores de Minas Gerais

\num{8} Um professor de matemática aplicou uma prova para sua turma, e as
notas dos alunos foram as seguintes: 8, 7, 9, 6, 8, 9, 7, 10.

Considerando as medidas de tendência central, qual é a moda das notas?

R:

A média aritmética simples é calculada somando todas as notas e
dividindo pelo número de notas. No caso das notas apresentadas (8, 7, 9,
6, 8, 9, 7, 10), a soma é 64 e o número de notas é 8. Portanto, a média
aritmética simples é \frac {64}{8} = 8.

A moda é o valor que ocorre com maior frequência em um conjunto de
dados. Neste caso, a nota 9 ocorre duas vezes, enquanto as outras notas
ocorrem apenas uma vez cada. Portanto, a moda das notas é 9.

\num{9} Uma pesquisa foi realizada para determinar o número de horas que os
estudantes de uma escola dedicam aos estudos diariamente. Os resultados
obtidos foram os seguintes: 3, 4, 2, 5, 2, 3, 6, 4.

Calcule a média aritmética, a moda e a mediana.

R:

A média aritmética simples é calculada somando todas as observações e
dividindo pelo número de observações. No caso dos dados apresentados (3,
4, 2, 5, 2, 3, 6, 4), a soma é 29 e o número de observações é 8.
Portanto, a média aritmética simples é \frac {29}{8} = 3.625, que
arredondado para uma casa decimal é igual a 3.5.

A moda é o valor que ocorre com maior frequência em um conjunto de
dados. Neste caso, o número 2 ocorre duas vezes, assim como o número 3 e
o número 4. Portanto, a moda dessa pesquisa é múltipla, sendo 2, 3 e 4.

A mediana é o valor que ocupa a posição central quando os dados são
organizados em ordem crescente ou decrescente. Após ordenar os dados,
temos: 2, 2, 3, 3, 4, 4, 5, 6. Como o conjunto de dados possui um número
ímpar de observações, a mediana é o valor central, que é 3.

\num{0} Tarcísio gosta muito de assistir corridas de cavalos. Ele decidiu
anotar a velocidade de seu cavalo favorito durante uma semana. Indique a
mediana desses valores.

\begin{longtable}[]{@{}lllllll@{}}
\toprule
\endhead
60km/h & 61km/h & 55km/h & 54km/h & 71km/h & 50km/h &
65km/h\tabularnewline
\bottomrule
\end{longtable}

R:

Para encontrar a mediana, é preciso organizar o conjunto de dados e
encontrar o valor que está no meio.

\text{Mediana} = \ 50 - 54 - 55 - 60 - 61 - 65 - 71

\text{Mediana} = \ 60\text{km}/h

\section{Treino}

\num{1} Analise as afirmações e marque a correta.

I - A média é o valor que mais se repete.

II - Uma variável quantitativa pode ser expressa por números.

III - O primeiro passo para realizar uma pesquisa estatística é
delimitar o problema.

a) Apenas I está correta.
b) Todas estão erradas.
c) As afirmações II e III estão corretas.
d) As afirmações I e II estão corretas.

%\section{BNCC: EF07MA35 }
% -- Compreender, em contextos significativos, o
% significado de média estatística como indicador da tendência de uma
% pesquisa, calcular seu valor e relacioná-lo, intuitivamente, com a
% amplitude do conjunto de dados.
% SAEB: Interpretar o significado das medidas de tendência central (média,
% aritmética simples, moda e mediana) ou da amplitude.

% A - Incorreta, pois a I não está correta. mMédia, é a soma dos valores
% do conjunto de dados,dividido pela quantidade dos dados.
% B - Incorreta, pois somente a I está incorreta.
% C - Correta, pois a I é falsa. Moda é o valor que mais se repete.
% D - Incorreta, pois, a definição de moda está incorreta na I.

\num{2} Dada a tabela de salário + horas extras de Juliano durante 6 meses,
qual foi a média dos valores que ele recebeu?

\begin{longtable}[]{@{}ll@{}}
\toprule
\endhead
Meses & Salários + horas extras\tabularnewline
Janeiro & R\$1.400,00\tabularnewline
Fevereiro & R\$1.359,00\tabularnewline
Março & R\$1.260,00\tabularnewline
Junho & R\$1.300,00\tabularnewline
Julho & R\$1.500,00\tabularnewline
Agosto & R\$1.400,00\tabularnewline
\bottomrule
\end{longtable}

a) Aproximadamente R\$1.500,00
b) Aproximadamente R\$1.369,00
c) Aproximadamente R\$1.112,00
d) Aproximadamente R\$1.489,00

%\section{BNCC: EF07MA35 }
% -- Compreender, em contextos significativos, o
% significado de média estatística como indicador da tendência de uma
% pesquisa, calcular seu valor e relacioná-lo, intuitivamente, com a
% amplitude do conjunto de dados.
% SAEB: Calcular os valores de medidas de tendência central de uma
% pesquisa estatística (média aritmética simples, moda ou mediana).

% A - Incorreta, pois, os valores informados não são próximos de
% R\$1.500,00.
% B - Correta, pois
% MÉ\text{DIA} = \frac{1400 + 1359 + 1260 + 1300 + 1500 + 1400}{6} \cong 1369.
% C - Incorreta, pois o cálculo não apresenta esse resultado.
% D- Incorreta, pois os valores informados não chegam a essa média.

\num{3} Marque a alternativa que classifica as seguintes variáveis,
respectivamente: Profissão, Batimentos cardíacos, Pressão, Altura.

a) Qualitativa, Quantitativa, Quantitativa, Quantitativa
b) Qualitativa, Qualitativa, Quantitativa, Quantitativa
c) Quantitativa, Qualitativa, Quantitativa, Quantitativa
d) Quantitativa, Quantitativa, Quantitativa, Quantitativa

%\section{BNCC: EF07MA35 }
% -- Compreender, em contextos significativos, o
% significado de média estatística como indicador da tendência de uma
% pesquisa, calcular seu valor e relacioná-lo, intuitivamente, com a
% amplitude do conjunto de dados.
% SAEB: Identificar os indivíduos (universo ou população-alvo da
% pesquisa), as variáveis e os tipos de variáveis (quantitativas ou
% categóricas) em um conjunto de dados.

% A - Correta, poi, somente profissão não pode ser quantificada.
% B - Incorreta, pois batimentos cardíacos constituem uma variável
% quantitativa.
% C - Incorreta, pois profissão não é uma variável quantitativa e
% batimentos cardíacos não são qualitativos.
% D - Incorreta, pois profissão não é uma varáivel quantitativa.

\section{Grandezas}
\markboth{Módulo 12}{}

\section{Habilidades do SAEB}
\begin{itemize}
\item
  Resolver problemas que envolvam medidas de grandezas (comprimento,
  massa, tempo, temperatura, capacidade ou volume) em que haja
  conversões entre unidades mais usuais.
\item
  Resolver problemas que envolvam perímetro de figuras planas.
\item
  Resolver problemas que envolvam área de figuras planas.
\item
  Resolver problemas que envolvam volume de prismas retos ou cilindros
  retos.
\end{itemize}

\section{Habilidades da BNCC }
\begin{itemize}
\item EF07MA29
\item EF07MA30
\item EF07MA31
\item EF07MA32
\end{itemize}

% Professor, trabalhe com calma as conversões de unidade de medidas, pois
% é um conteúdo de extrema utilidade nos próximos anos.

\textbf{Unidades de medidas}

Quando precisamos medir algo, devemos partir de uma padronização. Com
isso, temos o Sistema Internacional (S.I) de medidas, que estabelece
medidas centrais de acordo com o que se quer medir. Aqui vamos focar nas
unidades de comprimento, massa, tempo, capacidade e volume e suas
respectivas transformações.

1º) {[}Unidade de comprimento{]}: o metro é a unidade central e, para
transformar os múltiplos e submúltiplos, usamos multiplicações e
divisões por 10.

2º) {[}Unidade de área{]}: o metro quadrado é a unidade central e, para
transformar os múltiplos e submúltiplos, usamos multiplicações e
divisões por 10², ou seja, 100.

3º) {[}Unidade de volume{]}: o metro cúbico é a unidade central e, para
transformar os múltiplos e submúltiplos, usamos multiplicações e
divisões por 10³, ou seja, 1000.

4º) {Unidade de capacidade:} o litro é a unidade central e, para
transformar os múltiplos e submúltiplos, usamos multiplicações e
divisões por 10.

\textbf{Observação:} as unidades de volume e capacidade se relacionam,
uma vez que o volume mede o espaço que o corpo ocupa no espaço, enquanto
a capacidade mede o espaço que pode ser preenchido no corpo. Temos que,

1L = 1\ dm^{2}1m^{3} = 1000l1cm^{3} = 1\text{ml}

5º) {Unidade de massa:} o grama é a unidade central e, para transformar
os múltiplos e submúltiplos, usamos multiplicações e divisões por 10.

6º) {Unidade de tempo:} o minuto é a unidade central e, usamos um
múltiplo e submúltiplo, cuja transformação é feita com multiplicação e
divisão por 60.

Professor, ofereça alguns exemplos de transformação de medidas para os
alunos compreenderem como usamos a multiplicação e a divisão. Use,
durante o processo de transformação, a ideia de ``andar'' com a vírgula
do número quando multiplicamos por 10, 100 ou 1000.

Na matemática, usamos as conversões de medidas principalmente no cálculo
do perímetro, da área de figuras planas e do volume de sólidos. Vamos
relembrar as principais relações para o cálculo dessas medidas.

\textbf{{[}Perímetro:{]}} é a medida do contorno de uma figura plana.

\textbf{{[}Área{]}:} é a medida do espaço interno de uma figura plana,
na qual usamos medidas área.

Fazer uma imagem como a colocada acima.

\section{Atividades}

\num{1} Encontre o valor das transformações das unidades de comprimento
abaixo:

a) 360 m em hm

b) 5,2 dam em cm

c) 72 mm em m

d) 0,003 km em dm

R:

a) \frac{360}{100} = 3,6\ hm

b) 5,2 \times 1000 = 5200\ \text{cm}

c) \frac{72}{1000} = 0,072\ m

d) 0,003 \times 10.000 = 30\ \text{dm}

\num{2} Um jardineiro deseja cercar um terreno retangular para criar uma área
de plantio. Ele mediu os lados do terreno e obteve as seguintes medidas:
lado A = 5 metros e lado B = 8 metros.

Calcule o perímetro do terreno.

R:

O perímetro de um retângulo é calculado somando todos os lados. No caso
do terreno apresentado, temos dois lados de comprimento A = 5 metros e
dois lados de comprimento B = 8 metros. Portanto, o perímetro é dado por
2A + 2B = 2(5) + 2(8) = 10 + 16 = 26 metros.

\num{3} Encontre o valor das transformações das unidades de área abaixo:

a) 0,3 dam² em dm²

b) 3.500 cm² em mm²

c) 720 hm² em km²

d) 265.000 mm² em m²

R:

Professor, enfatize com os alunos que no caso das unidades de área as
transformações são feitas utilizando o 100, ou seja, ao transformar
movimentamos a vírgula 2 casas em cada unidade que passamos.

a) 0,3 \times 10.000 = 3.000\ dm^{2}

b) 3500 \times 100 = 350.000\ \text{mm}²

c) \frac{720}{100} = 7,2\ km^{2}

d) \frac{265.000}{1\ 000\ 000} = 0,265m²

\num{4} Um fazendeiro está planejando construir um cercado retangular para
abrigar suas ovelhas. Ele mediu os lados do cercado e obteve as
seguintes medidas: base = 10 metros e altura = 6 metros.

R:

A resposta correta é Área = 60 m², pois é o resultado correto obtido ao
multiplicar a base pela altura do cercado retangular

\num{5} Um engenheiro está projetando um reservatório de água cilíndrico para
armazenar água em uma fazenda. Ele precisa calcular o volume do
reservatório para garantir que tenha capacidade suficiente. As medidas
obtidas são: altura = 4 metros e raio da base = 2 metros.

Calcule o volume do reservatório.

R:

O volume de um cilindro é calculado multiplicando a área da base pela
altura. No caso do reservatório, temos um cilindro com altura de 4
metros e raio da base de 2 metros. Portanto, o volume é dado por π *
(raio)\^{}2 * altura = π * 2\^{}2 * 4 ≈ 25,13 metros cúbicos.

π \times (raio)^{2} \times altura = π \times 2^{2} \times {4} ≈
25,13.

\num{6} Encontre o valor das transformações das unidades de massa abaixo:

a) 850 cg em dg

b) 9,3 dag em g

c) 63,9 dag em kg

d) 0,002 kg em mg

R:

a) \frac{850}{10} = 85\ \text{dg}

b) 93\  \times 10 = 930\ g

c) \frac{63,9}{100} = 0,639\ \text{kg}

d) 0,002 \times 1.000.000 = 2000\ \text{mg}

\num{7} Encontre o valor das transformações das unidades de tempo abaixo:

a) 300 minutos em horas

b) 25 minutos em segundos

c) 2 horas em segundos

d) 10.800 segundos em horas

R:

a) \frac{300}{60} = 5\ h\text{oras}

b) 25 \times 60 = 1500\ \text{segundos}

c) 2 \times 60 \times 60 = 7200\ \text{segundos}

d)
\frac{10.800}{60} = 180\ \text{minutos} = \frac{180}{60} = 3\ h\text{oras}

\num{8} Encontre o valor das transformações das unidades de volume abaixo:

a) 0,0012 hm³ em m³

b) 9804 cm³ em dm³

c) 0,651 dm³ em cm³

d) 4.200.000.000 cm³ em hm³

R:

a) 0,0012\  \times 1.000.000 = 1200\ m^{³}

b) \frac{9804}{1000} = 9,804\ dm^{³}

c) 0,651 \times 1000 = 651\ \text{cm}³

d) \frac{4.200.000.000\ }{10^{12}} = 0,0042\ hm³

\num{9} Faça a conversão entre as unidades de volume e capacidade:

a) 25 m³ em L

b) 400 L em m³

c) 725 dm³ em ml

d) 188 L em dm³

R:

% Professor, retome com os alunos quais são as unidades de volume e
% capacidade que se relacionam para depois trabalhar as transformações
% solicitadas.

a)
25,4\ m^{3}\text{em}\ L \rightarrow 1m^{3} = 1000L \rightarrow 25m^{3} = 25,4 \times 1000 = 25.400L

b)
400\ L\ \text{em}\ m^{3} \rightarrow 1m^{3} = 1000L \rightarrow 400L = \frac{400}{1000} = 0,4m³

c)
725\ dm^{3}\text{em}\ \text{ml} \rightarrow 1dm^{3} = 1L \rightarrow 725dm^{3} = 725L \rightarrow 725 \times 1000 = \mathbf{725\ 000}\mathbf{\text{ml}}

d)
188\ L\ \text{em}\ dm^{3} \rightarrow \ 1dm^{3} = 1L \rightarrow 188L = 188\text{dm}³

\num{0} Uma sala retangular possui comprimento 10 metros e largura 6 metros.
Um tapete será colocado em toda a área da sala, exceto em uma faixa de 1
metro ao redor das paredes.

Qual é o perímetro dessa faixa sem tapete?

R:

Para calcular o perímetro da faixa sem tapete, precisamos encontrar o
comprimento dos quatro lados dessa faixa. Sabendo que a sala possui
comprimento 10 metros e largura 6 metros, e que a faixa sem tapete tem
largura 1 metro, podemos calcular o comprimento dos lados.

Os dois lados mais longos da faixa serão iguais ao comprimento da sala,
descontando as faixas sem tapete, ou seja, 10 metros - 2 metros = 8
metros cada.

Os dois lados mais curtos da faixa serão iguais à largura da sala,
descontando as faixas sem tapete, ou seja, 6 metros - 2 metros = 4
metros cada.

A soma dos comprimentos dos quatro lados da faixa sem tapete é 8 + 8 + 4
+ 4 = 24 metros.

Portanto, a resposta correta é 24 metros, que representa o perímetro da
faixa sem tapete ao redor das paredes da sala retangular.

\section{Treino}

\num{1} Fernanda está reformando seu apartamento e vai trocar os pisos de
todos os cômodos, sendo que o apartamento é retangular e tem medidas
1800\ \text{cm} \times 2200\ \text{cm}. As caixas de pisos que
Fernanda vai comprar comportam 15 m² de pisos, logo, ela deve comprar:

a) 264 caixas
b) 25 caixas
c) 26 caixas
d) 27 caixas

%\section{BNCC: EF07MA29 }
% -- Resolver e elaborar problemas que envolvam medidas de
% grandezas inseridos em contextos oriundos de situações cotidianas ou de
% outras áreas do conhecimento, reconhecendo que toda medida empírica é
% aproximada.
% SAEB: Resolver problemas que envolvam medidas de grandezas (comprimento,
% massa, tempo, temperatura, capacidade ou volume) em que haja conversões
% entre unidades mais usuais.

% A - Incorreta, pois transformou errado cm² para m².
% B - Incorreta, pois fez a divisão por 16 m² ao invés de 15 m².
% C - Incorreta, pois considerou que 26,4 seria arredondado para 26
% caixas.
% D - Correta, pois área do apartamento \(= 1800 \times 2200 = 3.960.000\ cm^{2} = 396m²\). Caixas com 15m² \(= \frac{396}{15} = 26,4\ \text{caixas}\). Como não é possível comprar 1,4 caixas, Fernanda deve comprar 27 caixas.

\num{2} O professor de educação física de Mateus montou para ele um treino
aeróbico. Ele deve fazer por 5 dias na semana a sequência: 25 minutos de
esteira, 15 minutos de bicicleta e 5 minutos de pulo de corda. Assim,
Mateus faz por semana um treino aeróbico de:

a) 3 horas e 45 minutos
b) 3 horas e 35 minutos
c) 4 horas e 45 minutos
d) 4 horas e 35 minutos

%\section{BNCC: EF07MA29 }
% -- Resolver e elaborar problemas que envolvam medidas de
% grandezas inseridos em contextos oriundos de situações cotidianas ou de
% outras áreas do conhecimento, reconhecendo que toda medida empírica é
% aproximada.
% SAEB: Resolver problemas que envolvam medidas de grandezas (comprimento,
% massa, tempo, temperatura, capacidade ou volume) em que haja conversões
% entre unidades mais usuais.

% A - Correta, pois 1 dia \rightarrow 25 + 15 + 5 = 45\ \text{minutos}; 5 dias \rightarrow 45 \times 5 = 225\ \text{minutos}; \frac{225}{60} = 3,75\ h\text{oras} \rightarrow 0,75\ h\text{oras} = 0,75 \times 60 = 45\ \text{minutos}. Portanto, o tempo de treino por semana é de 3 horas e 45 minutos.
% B - Incorreta, pois calculou que 0,75 \times 60 = 35.
% C - Incorreta, pois calculou que \frac{225}{60} = 4,75.
% D - Incorreta, pois calculou que \frac{225}{60} = 4,75\ e\ 0,75 \times 60 = 35.

\num{3} A piscina da casa de Tânia tem medidas iguais a 12 m de comprimento,
7m de largura e 2,5m de profundidade. Ela precisa contratar caminhões
pipas para enchê-la. Um caminhão tem capacidade igual a 5.000 litros de
água, logo ela precisa contratar:

a) 33 caminhões
b) 34 caminhões
c) 42 caminhões
d) 43 caminhões

%\section{BNCC: EF07MA30 }
% -- Resolver e elaborar problemas de cálculo de medida do
% volume de blocos retangulares, envolvendo as unidades usuais (metro
% cúbico, decímetro cúbico e centímetro cúbico).
% SAEB: Resolver problemas que envolvam volume de prismas retos ou
% cilindros retos.

% A - Incorreta, pois calculou o volume da piscina utilizando 2 ao invés
% de 2,5 e ainda aproximou a divisão de maneira errada.
% B - Incorreta, pois calculou o volume da piscina utilizando 2 ao invés
% de 2,5.
% C - Correta, pois volume da piscina = 12 \times 7 \times 2,5 = 210\ m^{3}; Capacidade da piscina = 210 \times 1000 = 210.000\ \text{litros}; Quantidade de caminhões =
% \frac{210.000}{5.000} = 42\ \text{camin}hõ\text{es}.
% D - Incorreta, pois fez a multiplicação do volume errado, encontrando
% 215 ao invés de 210.

\chapter{Probabilidade}
\markboth{Módulo 13}{}

\section{Habilidades do SAEB }
\begin{itemize}
\item Resolver problemas que envolvam a probabilidade de
ocorrência de um resultado em eventos aleatórios equiprováveis
independentes ou dependentes.
\end{itemize}

\section{Habilidade da BNCC}
\begin{itemize}
  \item EF07MA34
  \end{itemize}

% Professor, neste módulo, é muito importante que os alunos saibam
% distinguir a dependência ou independência dos eventos.

\textbf{{[}Probabilidade condicional{]}}

\textbf{Eventos independentes:} a ocorrência de um evento não depende de
seus antecessores ou sucessores.

\textbf{Exemplo:} Qual é a probabilidade de tirar um ás de paus em um
baralho? Não há intercorrência nenhuma, ou seja, todas as cartas estão
em um baralho e só há 1 chance dentre as 52 cartas de sair um ás de
paus.

Sendo assim, sabemos que a probabilidade independente de acontecer um
evento é:

P(A)=
\frac{nú\text{mero}\ \text{de}\ ch\text{ances}\ \text{de}\ \text{ocorrer}\ o\ \text{evento}}{\text{espa}ço\ \text{amostral}(\text{todas}\ ch\text{ances}\ \text{poss}í\text{veis})}

Então temos que P(A|B) = \ P(A) e P(B|A) = \ P(B).

\textbf{Exemplo:} Tirar um número específico em um dado ou cara/coroa em
uma moeda são eventos dependentes.

\textbf{Regra da multiplicação para eventos independentes:} para
encontrar a probabilidade da ocorrência de dois eventos que não dependem
entre si, utiliza-se a regra da multiplicação:

P(A\ e\ B)\  = \ P(A).P(B/A)

P(A\ e\ B)\  = \ P(A).P(B)\ 

\textbf{Exemplo:} A probabilidade de Juliana acertar duas questões
seguidas na prova de matemática é igual a 0,82.

P\ (2\ \text{quest}õ\text{es}\ \text{corretas})\ 0,82.\ 0,82 = \ 0,6724

\textbf{Observação:} A probabilidade total é sempre igual a 1 (remetendo
a 100\%), então, no exemplo anterior, poderíamos perguntar a
probabilidade de fracasso, ou seja, 1-0,82 a chance de erro para cada
questão.

\textbf{Eventos dependentes:} é quando o evento anterior influencia na
tomada de decisão do evento atual.

\textbf{Exemplo:} em um jogo de baralho, ganha quem encontrar o 10 de
ouros primeiro. Flávia foi a primeira a retirar uma carta do baralho,
obtendo uma dama de paus que não retornou para o baralho. Qual é a
chance de Josué, o próximo jogador, encontrar a carta vencedora?

É perceptível que a ação de Flávia influencia na jogada de Josué, pois o
espaço amostral é diferente do início do jogo, já que uma das cartas de
Flávia foi retirada, restando 51 no jogo.

Para calcular a probabilidade de eventos dependentes, haverá dois
momentos: o cálculo da probabilidade do evento A, que é o primeiro
evento, e depois o cálculo do evento B, dado que A ocorreu. Depois de
calculado, multiplicam-se a probabilidade dos eventos.

P(A \cap B)\  = \ P(A).\ P(B|A)

\textbf{Exemplo:} Em uma urna, há 7 bolas brancas e 6 vermelhas. Em uma
competição entre duas pessoas, quem retirar 3 bolas vermelhas primeiro,
ganha. Na decisão de par ou ímpar, Sandro venceu e é o primeiro a tentar
retirar uma bola vermelha. Sua primeira bola é branca. Dado que as bolas
não são retornadas de volta para a urna, qual aé chance de Max, o
próximo jogador, retirar uma bola vermelha?

P(A) = \frac{6}{13}, pois, dentre todas as bolas, 6 delas são
vermelhas.

Agora, dado que Sandro já jogou, a probabilidade de B é
P(B|A) = \ \frac{6}{12}.

A probabilidade que seja vermelha é:
\frac{6}{13} \times \frac{6}{12} = \frac{12}{156} ou a fração
irredutível \frac{1}{13}.

\section{Atividades}

\num{1} Classifique os eventos como dependentes ou independentes:

a) Um jogador de futebol erra dois chutes a gol.

b) Uma pessoa escolhe o lugar no cinema antes de você.

c) Retirar cartas do baralho e não devolvê-las antes de passar a vez
para o próximo jogador.

R:

a) Eventos independentes, pois o fato de o jogador errar o primeiro
chute não impõe que ele errará o segundo.

b) Dependente, pois, quando você for escolher, terá menos opções.

c) Dependente, porque quando a próxima pessoa for retirar uma carta,
terá menos opções.

\num{2} Felipe é muito bom em geografia. As chances de acertar questões em
provas objetivas é de 0,9, ou seja, 90\%. Qual é a chance de Felipe
acertar 3 questões consecutivas?

R:

São eventos independentes, pois errar ou acertar a questão anterior não
influencia na próxima questão. Então, utilizando a regra da
multiplicação, temos
:\ P(\ 3\ \text{acertos}\ \text{em}\ \text{quest}õ\text{es}) = \ 0,9 \times 0,9 \times 0,9 = \ 0,729.

\num{3} Qual é a probabilidade de o lançamento de um dado cair 4 vezes
seguidas no número 2?

R:

São eventos independentes, uma vez que o resultado anterior do dado não
diminui as chances de cair na mesma face. Então, pela regra da
multiplicação:

P(A) = \ \frac{1}{6} \times \frac{1}{6} \times \frac{1}{6} \times \frac{1}{6} = \ \frac{1}{1296}

Professor, nesta questão, é importante trabalhar com o número decimal e
ver o quão longe do 1, está. Logo, a chance de acontecer é muito
pequena.

\num{4} Um dado de seis faces é lançado duas vezes consecutivas. Qual é a
probabilidade de obter um número par no primeiro lançamento e um número
ímpar no segundo lançamento?

R:

Para determinar a probabilidade desejada, devemos analisar cada
lançamento separadamente e considerar que o resultado de um lançamento
não influencia o resultado do outro, tornando os eventos independentes.

No primeiro lançamento, metade das faces do dado é par e a outra metade
é ímpar, então a probabilidade de obter um número par é de
\frac {1}{2}.

No segundo lançamento, mesmo após o primeiro, a probabilidade de obter
um número ímpar ainda é de \frac {1}{2}, pois os eventos são
independentes.

Para determinar a probabilidade conjunta dos dois eventos independentes,
multiplicamos as probabilidades individuais:
\frac {1}{2} \times \frac {1}{2} = \frac {1}{4}

\num{5} Qual é a chance de eu virar cartas de um baralho ao acaso e retirar
um 5, não retorná-lo, e tirar uma dama logo em seguida?

R:

Trata-se de um evento dependente, então é preciso calcular o primeiro
evento e depois o segundo acontecimento, dado o primeiro já ter
ocorrido, e multiplicar as probabilidades.

P\left( A \right) = \frac{4}{52}\ e\ P\left( A \right) = \ \frac{4}{51},\ P\left( A \right)\text{.\ }P\left( A \right) = \ \frac{4}{52} \times \ \frac{4}{51} = \frac{16}{2652}\ \text{ou}\ a\ \text{fra}ção\ \text{irredut}í\text{vel}\ \frac{4}{663}.

\num{6} A chance de um processador de 7ª geração estragar antes de um ano de
garantia equivale a 0,2. Qual é a chance de dois processadores não
estragarem nesse um ano?

R:

São eventos independentes, pois o fato de um processador ter estragado
não implica que o outro também estrague. Se a chance de estragar é 0,2,
a de não estragar é de 0,8, pois 1 - 0,2 = 0,8. Para eventos
independentes, utilizamos a regra da multiplicação,
P(2\ \text{processad}\text{ores}\ não\ \text{estragarem}) = \ 0,8.0,8 = \ 0,64,\ \text{ou}\ \text{seja},\ 64\%.

\num{7} Calcule a probabilidade de encontrarmos um número par ao jogarmos um
dado.

R:

Os números pares de um dado são 2,4 ou 6. Logo, a probabilidade de ser
um número par é de \frac{3}{6}.

\num{8} Um saco contém 5 bolas numeradas de 1 a 5. Uma pessoa retira uma bola
do saco sem olhar e registra o número. Em seguida, a bola é colocada de
volta no saco e a pessoa retira outra bola. Qual é a probabilidade de
ambas as bolas retiradas terem números ímpares?

R:

Para calcular a probabilidade de ambas as bolas retiradas terem números
ímpares, vamos considerar cada retirada de bola como um evento
independente.

Existem três bolas com números ímpares no saco: 1, 3 e 5. O número total
de bolas no saco é 5.

Na primeira retirada, a probabilidade de escolher uma bola ímpar é de
\frac {3}{5}, pois existem três bolas ímpares entre as cinco
disponíveis.

Após a primeira retirada, a bola é colocada de volta no saco, e o número
de bolas e a proporção de bolas ímpares permanecem as mesmas.

Na segunda retirada, a probabilidade de escolher uma bola ímpar
novamente é de \frac {3}{5}.

Portanto, para calcular a probabilidade de ambas as bolas retiradas
terem números ímpares, multiplicamos as probabilidades de cada evento
independente: \frac {3}{5} \times \frac {3}{5} = \frac {9}{25}

\num{9} Qual é a probabilidade de eu tirar uma carta de um baralho que seja
preta e de ouro, sendo que uma carta vermelha já foi retirada do baralho
e não retornada?

R:

A probabilidade acima é igual a 0, pois as cartas de ouro de um baralho
são vermelhas.

\num{0} Ruan tem que selecionar dois alunos de uma classe de 20 meninas e 18
meninos. Qual é a probabilidade de os dois alunos escolhidos serem
meninos?

R:

Temos que P(A) = \frac{18}{38}, ou seja, 18 meninos para uma turma
de total 38. Considerando que o primeiro menino já foi retirado, vamos
calcular a probabilidade do segundo menino:
P(B|A) = \ \frac{17}{37}. Utilizando a regra da multiplicação,
temos:

P(A).P(B|A) = \frac{18}{38} \times \frac{17}{37} = \ \frac{306}{1406} = \ \frac{153}{703}

\section{Treino}

\num{1} Jogando uma moeda, qual é a chance de sair, 3 vezes seguidas, a face
coroa?

\item
  \ \frac{3}{8}

b) \frac{1}{8}

c) \frac{3}{2}

\item
  \ \frac{1}{6}


%%\section{BNCC: EF07MA34 }
% -- Planejar e realizar experimentos aleatórios ou
% simulações que envolvem cálculo de probabilidades ou estimativas por
% meio de frequência de ocorrências.
% SAEB: Resolver problemas que envolvam a probabilidade de ocorrência de
% um resultado em eventos aleatórios equiprováveis independentes ou
% dependentes.

% A - Incorreta, pois usou as três jogadas no numerador, porém, o correto
% seria a possibilidade de cair coroa.
% B - Correta, pois são de eventos independentes, ou seja, eles não
% dependem do evento anterior para ocorrer. P(A).\ P(B).\ P(C) = \ \frac{1}{2} \times \frac{1}{2} \times \frac{1}{2} = \ \frac{1}{8}.
% C - Incorreta, pois considerou a quantidade vezes de jogadas como o
% numerador e desconsiderou que se trata de eventos independentes.
% D - Incorreta, pois multiplicou o número de jogadas pelo número de
% possibilidades em uma moeda, desconsiderando que são eventos
% independentes.

\num{2} Em um globo de bingo existem 75 bolas numeradas de 1 a 75. Qual é a
chance de tirar a bola 18, não recolocá-la e tirar a bola 4 em seguida?

a) \frac{2}{75}
b) \frac{5}{75}
c) \frac{1}{5550}
d) \frac{1}{74}

%\section{BNCC: EF07MA34 }
% -- Planejar e realizar experimentos aleatórios ou
% simulações que envolvem cálculo de probabilidades ou estimativas por
% meio de frequência de ocorrências.
% SAEB: Resolver problemas que envolvam a probabilidade de ocorrência de
% um resultado em eventos aleatórios equiprováveis independentes ou
% dependentes.

% A - Incorreta, pois não foram considerados eventos dependentes.
% B - Incorreta, pois não foram considerados eventos dependentes para 5
% jogadas.
% C - Correta, pois são eventos dependentes. O primeiro evento é retirar a
% bola 18, que é dada por: P(A) = \ \frac{1}{75}\ . Retirando a
% primeira bola e não devolvendo, restam 74 bolas, assim a probabilidade
% de retirar a bola 4 de primeira, no segundo evento é
% P(A|B) = \ \frac{1}{74}. Assim, a probabilidade do evento ocorrer é:
% P(A).P(A|B) = \ \frac{1}{5550}.
% D - Incorreta, pois retirou uma das bolas do espaço amostral, mas não
% considerou o evento ocorrido anteriormente.

\num{3} O melhor arremessador do mundo tem uma porcentagem de erro igual a
0,18. Qual é a chance de o arremessador acertar três arremessos
consecutivos?

a) Aproximadamente 100\%
b) Aproximadamente 21\%
c) Aproximadamente 55\%
d) Aproximadamente 1\%

%\section{BNCC: EF07MA34 }
% -- Planejar e realizar experimentos aleatórios ou
% simulações que envolvem cálculo de probabilidades ou estimativas por
% meio de frequência de ocorrências.
% SAEB: Resolver problemas que envolvam a probabilidade de ocorrência de
% um resultado em eventos aleatórios equiprováveis independentes ou
% dependentes.

% A - Incorreta, pois, para que o arremessador acertasse todas as bolas, a
% porcentagem de erro seria igual a 0.
% B - Incorreta, pois 21\% é a aproximadamente a chance de erro para um
% arremesso.
% C - Correta. pois são eventos independentes. Devemos multiplicar as
% probabilidades. P(1) P(2) P(3) são as chances de acerto, então, basta
% descobrirmos as chances de acertos dado por \(1 - 0,18 = \ 0,82.\). P(1).P(2).P(3) = \ 0,551\ \text{aproximadamente}\ 55\%.
% D- Incorreta, pois, para que os arremessos fossem perto de 1\%, a
% probabilidade de erro do arremessador seria muito perto de 0,98, ou
% melhor, 98\%.

\chapter{Simulado 1}

\num{1} Uma escola organizou uma corrida com obstáculos. Respectivamente,
Júlio, Tiago, Isabel e Fábio obtiveram os tempos:
13,02  - 14,89  - 14,62  - 14,7. Marque a opção em que os
tempos são ordenados numa crescente.

a) Tiago - Fábio- Júlio - Isabel
b) Fábio - Isabel - Júlio - Tiago
c) Júlio - Isabel - Fábio - Tiago
d) Júlio - Isabel - Tiago - Fábio

%\section{BNCC: EF07MA10 }
% -- Comparar e ordenar números racionais em diferentes
% contextos e associá-los a pontos da reta numérica.
% SAEB: Comparar ou ordenar números reais, com ou sem suporte da reta
% numérica, ou aproximar número reais para múltiplos de potência de 10
% mais próxima.

% A - Incorreta, pois Fábio fez em um tempo menor que Tiago e veio depois
% na listagem.
% B - Incorreta, pois Isabel fez em um tempo menor que Fábio e veio depois
% na listagem.
% C - Correta, pois os números que possuem as partes inteiras iguais foram
% analisados pelas casas decimais.
% D - Incorreta, pois, apesar de Júlio e Isabel estarem nas posições
% corretas, o Fábio fez um tempo menor que Tiago, logo, ele não seria o
% último.

\num{2} A solução para a expressão
17 + ( - 2) \times \left\{ 5 - 9 + 2 \times \left\lbrack 36 \div 4 - \left( 7 \times 3 \right) \right\rbrack \right\}
é:

a) -420

b) 420

c) - 79

d) 73

%\section{BNCC: EF07MA04 }
% -- Resolver e elaborar problemas que envolvam operações
% com números inteiros.
% SAEB: Calcular o resultado de adições, subtrações, multiplicações ou
% divisões envolvendo números reais.

% A - Incorreta, pois resolveu
% 17 + ( - 2) \times \left( - 28 \right) = 15 \times ( - 28) = - 420.
% B - Incorreta, pois resolveu
% 17 + ( - 2) \times \left( - 28 \right) = - 15 \times ( - 28) = 420. 
% C - Incorreta, errou no jogo de sinal e resolveu pois resolveu que
% ( - 2) \times \left( - 28 \right) = \  - 96.
% D - Correta, pois
% 17 + ( - 2) \times \left\{ - 4 + 2 \times \left\lbrack 9 - \left( 21 \right) \right\rbrack \right\} = 17 + ( - 2) \times \left\{ - 4 - 24 \right\} = 17 + ( - 2) \times \left( - 28 \right) = 73.

\num{3} O tio de Juliana é 20 anos mais velho que ela. Considerando que
Leandro foi pai de Juliana com 23 anos e hoje está com 39 anos, qual é a
idade do tio de Juliana?

a) 36

b) 22

c) 25

d) 35

%\section{BNCC: EF07MA18 }
% -- Resolver e elaborar problemas que possam ser
% representados por equações polinomiais de 1º grau, redutíveis à forma ax
% + b = c, fazendo uso das propriedades da igualdade.
% SAEB: Resolver uma equação polinomial de 1º grau.

% A - Correta, pois J + 20 = \ \text{Tio}; J = 39 - 23 = 16; 16 + 20 = 36.
% B - Incorreta, pois, se Juliana tivesse 22 anos, seu tio teria 42 anos,
% o que não condiz com o enunciado.
% C - Incorreta, pois, se Juliana tivesse 25 anos, seu tio teria 45 anos,
% o que não condiz com o enunciado.
% D - Incorreta, pois, se Juliana tivesse 35 anos, seu tio teria 55 anos,
% o que não condiz com o enunciado.

\num{4} Marcela estava acompanhando o preço de um produto desde janeiro,
quando ele custava R\$2.100,00. Em março, ela observou que houve um
aumento de 10\% do valor. Em maio, foi dado um desconto de 12\%. Marcela
comprou o produto em junho, quando percebeu um novo aumento de 5\%.
Comparado com o preço de janeiro, Marcela comprou o produto com

a) aumento de, aproximadamente, 16\%.

b) desconto de, aproximadamente, 16\%.

c) aumento de, aproximadamente, 1,6\%.

d) desconto de, aproximadamente, 1,6\%.


%\section{BNCC: EF07MA02 }
% -- Resolver e elaborar problemas que envolvam
% porcentagens, como os que lidam com acréscimos e decréscimos simples,
% utilizando estratégias pessoais, cálculo mental e calculadora, no
% contexto de educação financeira, entre outros.
% SAEB: Resolver problemas que envolvam porcentagens, incluindo os que
% lidam com acréscimos e decréscimos simples, aplicação de percentuais
% sucessivos e determinação de taxas percentuais.

% A - Incorreta, pois considerou que 0,0164\cong 16\%
% B - Incorreta, pois além de considerar 0,0164\cong 16\%,
% interpretou como desconto.
% C - Correta, pois, fazendo os cálculos com os fatores de multiplicação,
% temos 1,1 \times 0,88 \times 1,05 = 1,0164 - 1 = 0,0164\cong 1,6\%.
% D - Incorreta, pois fez as contas de forma correta, mas interpretou como
% desconto ao invés de aumento.

\num{5} Rogério, Fernanda, Otávio foram em um rodízio de pizza e comeram o
equivalente ao que está pintado na figura. Qual fração representa essa
quantidade?

\includegraphics[width=3.625in,height=1.72917in]{./imgSAEB_7_MAT/media/image95.png}

a) \frac{5}{6}

b) 3\frac{1}{4}

c) 1\frac{1}{4}

d) 1\frac{4}{4}


%\section{BNCC: EF07MA09 }
% -- Utilizar, na resolução de problemas, a associação
% entre razão e fração, como a fração 2/3 para expressar a razão de duas
% partes de uma grandeza para três partes da mesma ou três partes de outra
% grandeza.
% SAEB: Representar frações menores ou maiores que a unidade por meio de
% representações pictóricas ou associar frações a representações
% pictóricas.

% A - Incorreta, pois a quantidade de fatias pintadas está correta, porém,
% o denominador não condiz, uma vez, que é dividido em 4 partes.
% B - Incorreta, pois, embora a fração \frac{1}{4} apareça, a parte
% inteira tem somente um círculo todo pintado.
% C - Correta, pois apresenta 1 círculo todo pintado, que é a parte
% inteira e \frac{1}{4} de outro.
% D- Incorreta, pois a representação da parte inteira e da parte decimal,
% corresponde a 1 inteiro, logo 2 inteiros, e na imagem somente um círculo
% está todo pintado.

\num{6} Na aula de matemática de Sofia, a professora escreveu a sequência
abaixo no quadro e pediu para os alunos escreverem a lei de formação de
acordo com a posição dos termos.

2,\ 6,\ 12,\ 20,\ldots

Sofia conseguiu encontrar a lei de formação correta, sendo ela:

\item
  a_{n} = 2n + 2
\item
  a_{n} = n² + n
\item
  a_{n} = 4n
\item
  a_{n} = n(n - 1)


%%\section{BNCC: EF07MA15 }
% -- Utilizar a simbologia algébrica para expressar
% regularidades encontradas em sequências numéricas.
% SAEB: Identificar uma representação algébrica para o padrão ou a
% regularidade de uma sequência de números racionais ou representar
% algebricamente o padrão ou a regularidade de uma sequência de números
% racionais.

% A - Incorreta, pois o primeiro termo seria
% a_{1} = 2 \times 1 + 2 = 2 + 2 = 4, o que não confere.
% B - Correta., pois temos, a_{1} = 1^{2} + 1 = 1 + 1 = 2,
% a_{2} = 2^{2} + 2 = 4 + 2 = 6, a_{3} = 3^{2} + 3 = 9 + 3 = 12 e
% a_{4} = 4² + 4 = 16 + 4 = 20.
% C - Incorreta, pois o primeiro termo seria a_{1} = 4 \times 1 = 4, o
% que não confere.
% D - Incorreta, pois o primeiro termo seria
% a_{1} = 1\left( 1 - 1 \right) = 1 \times 0 = 0, o que não confere.

\num{7} Em uma brincadeira, ganha aquele que encontrar o sete de copas em um
baralho. Lucas começou jogando e tirou duas cartas, nenhuma delas sendo
o sete de copas. Qual é a probabilidade de o próximo jogador ganhar?

a) \frac{3}{52}
b) \frac{1}{2650}


\item
  \ \frac{1}{52}

d) \frac{1}{2600}

%%\section{BNCC: EF07MA34 }

% -- Planejar e realizar experimentos aleatórios ou
% simulações que envolvem cálculo de probabilidades ou estimativas por
% meio de frequência de ocorrências.
% SAEB: Resolver problemas que envolvam a probabilidade de ocorrência de
% um resultado em eventos aleatórios equiprováveis independentes ou
% dependentes.

% A - Incorreta, pois calculou a probabilidade com 3 cartas,
% desconsiderando os eventos e a independência entre eles.
% B - Incorreta, pois não considerou que foram retiradas 2 cartas de uma
% vez.
% C - Incorreta, pois calculou apenas a probabilidade do primeiro evento
% acontecer.
% D - Correta, pois são eventos dependentes. A primeira jogada influencia
% na segunda jogada: P\left( A \right) = \frac{1}{52};\ P\left( A \right) = \frac{1}{50}; P\left( A \right) \times P\left( A \right) = \frac{1}{52} \times \frac{1}{50} = \frac{1}{2600}.

\num{8} Um carro faz uma viagem em 5 horas andando a 120km/h. Porém, o
trajeto foi modificado e o limite de velocidade passou a ser de 100km/h.
Quanto tempo esse carro gastará no trajeto com essa nova velocidade?

a) 5 horas

b) 6 horas

c) 4 horas e 9 minutos

d) 4 horas e 10 minutos


%\section{BNCC: EF07MA17 }
% -- Resolver e elaborar problemas que envolvam variação de
% proporcionalidade direta e de proporcionalidade inversa entre duas
% grandezas, utilizando sentença algébrica para expressar a relação entre
% elas.
% SAEB: Resolver problemas que envolvam variação de proporcionalidade
% direta~ou inversa entre duas ou mais grandezas.

% A - Incorreta, pois considerou que, mesmo mudando a velocidade, o tempo
% permaneceria o mesmo.
% B - Correta, pois velocidade e tempo são grandezas inversamente
% proporcionais, logo, o produto entre elas é constante. Assim,
% 5 \times 120 = 100x \rightarrow 100x = 600 \rightarrow x = \frac{600}{100} = 6\ h\text{oras}.
% C - Incorreta, pois considerou as grandezas como diretamente
% proporcionais e ainda arredondou 4,16 horas para 4 horas e 9 minutos.
% D - Incorreta, pois considerou as grandezas como diretamente
% proporcionais.

\num{9} Em uma pesquisa qualitativa nominal, foram apresentados dados da cor
dos olhos da amostra escolhida. O gráfico era todo colorido dentro de um
círculo dividido em fatias. Qual tipo de gráfico é esse?

\item
  Histograma
\item
  Gráfico de barras
\item
  Gráfico de colunas
\item
  Gráfico de linhas

%%\section{BNCC: EF07MA37}
%  -- Interpretar e analisar dados apresentados em gráfico
% de setores divulgados pela mídia e compreender quando é possível ou
% conveniente sua utilização.
% SAEB: Representar ou associar os dados de uma pesquisa estatística ou de
% um levantamento em listas, tabelas (simples ou de dupla entrada) ou
% gráficos (barras simples ou agrupadas, colunas simples ou agrupadas,
% pictóricos, de linhas, de setores, ou em histograma).

% A - Incorreta, pois o histograma é feito por linhas e barras.
% B - Incorreta, pois o gráfico de barras é formado por barras
% retangulares e com base maior na horizontal.
% C - Correta, pois esse tipo de gráico apresenta setores de uma figura
% geométrica, geralmente, um círculo.
% D - Incorreta, pois o gráfico de linhas é representado por pontos unidos
% por linhas.

\num{10} No plano cartesiano, considere o triângulo ABC com vértices A(2, 4),
B(5, 6) e C(7, 2). A figura obtida após aplicar uma transformação de
reflexão em relação ao eixo x nesse triângulo será:

a) Triângulo A'B'C' com vértices A'(-2, 4), B'(5, -6) e C'(7, -2).

b) Triângulo A'B'C' com vértices A'(-2, -4), B'(5, -6) e C'(7, -2).

c) Triângulo A'B'C' com vértices A'(-2, -4), B'(5, 6) e C'(7, 2).

d) Triângulo A'B'C' com vértices A'(2, -4), B'(-5, -6) e C'(-7, -2).


%\section{BNCC: EF07MA20 }
% -- Reconhecer e representar, no plano cartesiano, o
% simétrico de figuras em relação aos eixos e à origem.
% SAEB: Identificar, no plano cartesiano, figuras obtidas por uma ou mais
% transformações geométricas (reflexão, translação, rotação).

% A - Incorreta, pois os valores dos vértices não condizem com o processo
% de reflexão.
% B - Correta, pois, ao realizar uma reflexão em relação ao eixo x, os
% pontos mantêm a mesma coordenada x, mas têm sua coordenada y negativa.
% No triângulo ABC original, o ponto A(2, 4) terá a mesma coordenada x,
% mas sua coordenada y será negativa, resultando em A'(-2, -4). Da mesma
% forma, os pontos B(5, 6) e C(7, 2) terão suas coordenadas y negativas
% após a reflexão, resultando em B'(5, -6) e C'(7, -2), respectivamente.
% C - Incorreta, pois os valores dos vértices não condizem com o processo
% de reflexão.
% D - Incorreta, pois os valores dos vértices não condizem com o processo
% de reflexão.

\chapter{Simulado 2}

\num{1} No triângulo ABC, o ângulo A é um ângulo reto e o lado AC mede 5 cm.
Se o ângulo B mede 45 graus, qual é o comprimento do lado BC?

\item
  5 cm
\item
  5 \sqrt 2 \; cm
\item
  5 \sqrt 3\;cm
\item
  10 cm
\end{enumerate}


%\section{BNCC: EF07MA24 }
% -- Construir triângulos, usando régua e compasso,
% reconhecer a condição de existência do triângulo quanto à medida dos
% lados e verificar que a soma das medidas dos ângulos internos de um
% triângulo é 180°.
% SAEB: Identificar propriedades e relações existentes entre os elementos
% de um triângulo.

% A - Correta, pois pois o comprimento do lado BC é igual ao comprimento
% do lado AC em um triângulo retângulo com ângulos de 45 graus.
% B - Incorreta, pois essa resposta seria correta se o triângulo fosse um
% triângulo isósceles retângulo de 45-45-90 graus, mas no problema não foi
% mencionado que os ângulos são iguais.
% C - Incorreta, pois essa resposta seria correta se o triângulo fosse um
% triângulo equilátero de 60 graus, mas no problema não foi mencionado que
% os ângulos são iguais.
% D - Incorreta, pois essa resposta é o dobro do comprimento do lado AC, o
% que não é possível em um triângulo retângulo com ângulos de 45 graus.

\num{2} Observe a imagem abaixo onde ocorreu o deslocamento do retângulo
ABCD.

\includegraphics{./imgSAEB_7_MAT/media/image98.png}

Considerando que cada quadrado da malha tem 1 cm de lado, o deslocamento
do retângulo foi de:

\item
  1 cm para a direita
\item
  3 cm para a direita
\item
  1 cm para a esquerda
\item
  3 cm para a esquerda
\end{enumerate}

% SAEB: Descrever ou esboçar deslocamento de pessoas e/ou de objetos em
% representações bidimensionais (mapas, croquis etc.), plantas de
% ambientes ou vistas, de acordo com condições dadas.

% A - Incorreta, pois considerou a distância das duas imagens como o total
% do deslocamento.
% B - Correta, pois, pegando o vértice A como referência, podemos observar
% que ele foi deslocado em três espaços para a direita, gerando o vértice
% A'. Como cada linha da malha é 1 cm, o total do deslocamento foi de 3 cm
% para a direita.
% C - Incorreta, pois, além de considerar a distância das duas imagens
% como o total do deslocamento, confundiu a direção.
% D - Incorreta, pois fez o deslocamento correto, mas confundiu a direção.

\num{3} Frederico e Paulo estão pesquisando o preço de motos. A alternativa
que eles encontraram foi comparar a venda de quatro marcas diferentes
durante três meses. O critério de escolha será a maior média. Qual marca
eles compraram?

\begin{longtable}[]{@{}llll@{}}
\toprule
aulo: cria & r uma tabela com as infor & mações abaixo: &\tabularnewline
\midrule
\endhead
Marcas~ & Primeiro mês de vendas & Segundo mês de vendas & Terceiro mês
de vendas\tabularnewline
Yamaha & 12 & 8 & 13\tabularnewline
Suzuki & 7 & 4 & 10\tabularnewline
BMW & 2 & 6 & 14\tabularnewline
Honda & 8 & 3 & 7\tabularnewline
\bottomrule
\end{longtable}

a) Yamaha

b) Suzuki

c) BMW

d) Honda


%\section{BNCC: EF07MA35 }
% -- Compreender, em contextos significativos, o
% significado de média estatística como indicador da tendência de uma
% pesquisa, calcular seu valor e relacioná-lo, intuitivamente, com a
% amplitude do conjunto de dados.
% SAEB: Calcular os valores de medidas de tendência central de uma
% pesquisa estatística (média aritmética simples, moda ou mediana).

% A - Correta, pois,
% Mé\text{dia}\ \text{yama}ha = \ \frac{12 + 8 + 13}{3} = 11.
% B - Incorreta, pois,
% \ Mé\text{dia}\ \text{Suzuki}\  = \ \frac{7 + 4 + 10}{3} = 7.
% C - Incorreta, pois,
% Mé\text{dia}\ \text{BMW} = \ \frac{2 + 6 + 14}{3} = 7,3.
% D - Incorreta, pois,
% Mé\text{dia}\ \text{Honda} = \ \frac{8 + 3 + 7}{3} = 6.

\num{4} Um recipiente de formato cúbico tem aresta de tamanho 9m. A
capacidade desse recipiente é de:

a) 729 litros

b) 7.290 litros

c) 72.900 litros

d) 729.000 litros

%\section{BNCC: EF07MA29 }
% -- Resolver e elaborar problemas que envolvam medidas de
% grandezas inseridos em contextos oriundos de situações cotidianas ou de
% outras áreas do conhecimento, reconhecendo que toda medida empírica é
% aproximada.
% SAEB: Resolver problemas que envolvam medidas de grandezas em que haja
% conversões entre unidades mais usuais.

% A - Incorreta, pois considerou que a relação m³ é direta ao litro.
% B - Incorreta, pois converteu m³ para litro multiplicando por 10.
% C - Incorreta, pois converteu m³ para litro multiplicando por 100.
% D - Correta, pois o volume do cubo cprresponde ao tamanho da aresta
% elevado à terceira potência. Logo, V = 9^{3} = 729m³ . Como
% 1m^{3} = \ 1000\ L \rightarrow \ 729\ m³\  = \ 729.000 litros.

\num{5} Considerando a expressão algébrica \(5c = d\), o valor de c para
d = 75\ é:

a) \(70\)

b) \(375\)

c)\(\ 80\)

d) \(15\)


%\section{BNCC: EF07MA13 }
% -- Compreender a ideia de variável, representada por
% letra ou símbolo, para expressar relação entre duas grandezas,
% diferenciando-a da ideia de incógnita.
% SAEB: Resolver problemas que envolvam cálculo do valor numérico de
% expressões algébricas.

% A - Incorreta, pois ao calcular 5c = 75, passou o 5 subtraindo ao
% invés de dividindo.
% B - Incorreta, pois substituiu o valor de d no lugar de c.
% C - Incorreta, pois ao calcular 5c = 75, passou o 5 somando ao invés
% de dividindo.
% D - Correta, pois, para encontrar o valor de s, basta substituir o valor
% de t. Assim, 5c = d\  \rightarrow \ 5c = 75\  \rightarrow \ c = \frac{75}{5} = 15

\chapter{Simulado 3}

\num{1} Marcela vai fazer uma viagem para o exterior e precisa levar pelo
menos 300 combinações diferentes de looks para um ensaio fotográfico. Na
separação de roupas ela escolheu 5 blusas, 4 calças, 3 saias, 2 sapatos
e 3 blusas de frio. Usando essas opções Marcela vai conseguir a
quantidade de looks diferentes que precisa?

a) Sim. Ainda vão sobrar 60 combinações de roupa.

b) Não. Ainda vão faltar 60 combinações de roupa.

c) Não. Ainda vão faltar 180 combinações de roupa..

d) Não, pois ela terá apenas 17 combinações possíveis de roupa.

%\section{BNCC: EF07MA04 }
% -- Resolver e elaborar problemas que envolvam operações
% com números inteiros.
% SAEB: Resolver problemas de contagem cuja resolução envolva a aplicação
% do princípio multiplicativo.

% A - Correta, pois, usando o princípio multiplicativo, as combinações
% distintas são 5 \times 4 \times 3 \times 2 \times 3 = 360. Como Marcela precisa de 300 combinações distintas, ainda vão sobrar 60
% opções de combinações.
% B - Incorreta, pois não percebeu que o número mínimo de combinações foi
% atingido.
% C - Incorreta, pois não considerou todas as peças de roupa.
% D - Incorreta, pois somou as opções de escolha ao invés de multiplicar.

\num{2} Mateus está vendendo seu telefone com 1 ano de uso. Fazendo uma
pesquisa, ele descobriu que o celular desvalorizou 20\% em relação ao
preço inicial. Sabendo que ele pagou R\$1.800,00, o preço atual é:

a) R\$1.780,00

b) R\$1.440,00

c) R\$360,00

d) R\$1.400,00


%\section{BNCC: EF07MA02 }
% -- Resolver e elaborar problemas que envolvam
% porcentagens, como os que lidam com acréscimos e decréscimos simples,
% utilizando estratégias pessoais, cálculo mental e calculadora, no
% contexto de educação financeira, entre outros.
% SAEB: Resolver problemas que envolvam porcentagens, incluindo os que
% lidam com acréscimos e decréscimos simples, aplicação de percentuais
% sucessivos e determinação de taxas percentuais.

% A - Incorreta, pois considerou uma desvalorização de 20 reais.
% B - Correta, pois, como houve uma desvalorização de 20\%, vamos utilizar
% o fator multiplicativo 100\% - 20\% = 80\% = 0,8, assim 0,8 \times 1800 = 1440.
% C - Incorreta, pois considerou o valor da desvalorização e não o valor
% final.
% D - Incorreta, pois, ao fazer 1800 - 360, encontrou 1400 ao invés de
% 1440.

\num{3} Uma sequência equivalente a a_{n} = n²(n + 1) seria

a) a_{n} = n³(n + 1)

b) a_{n} = n(n² + 1)

c) a_{n} = n^{3} + n²

d) a_{n} = \ n³ + 1

%\section{BNCC: EF07MA16 }
% -- Reconhecer se duas expressões algébricas obtidas para
% descrever a regularidade de uma mesma sequência numérica são ou não
% equivalentes.
%SAEB: Identificar representações algébricas equivalentes.

% A - Incorreta, pois considerou que, para duas sequências serem
% equivalentes, basta ter uma parte em comum.
% B - Incorreta, pois considerou que, ao trocar o n com n² de lugar, uma
% sequência semelhante seria gerada.
% C - Correta., pois, aplicando a propriedade distributiva, temos:
% n^{2}\left( n + 1 \right) = n^{3} + n^{2}.
% D - Incorreta, pois, ao fazer a distributiva, apenas multiplicou o n
% pelo n².

\num{4} Em um mapa, as instruções eram de que 10 cm no desenho correspondiam
a 10 km do real. Dessa forma, a escala utilizada é de:

a) 1\ :10

b) 10\ :100000

c) 1\ :1000000

d) 1\ :100000


%\section{BNCC: EF07MA17 }
% -- Resolver e elaborar problemas que envolvam variação de
% proporcionalidade direta e de proporcionalidade inversa entre duas
% grandezas, utilizando sentença algébrica para expressar a relação entre
% elas.
% SAEB: Resolver problemas que envolvam variação de proporcionalidade
% direta ou inversa entre duas ou mais grandezas.

% A - Incorreta, pois considerou que basta igualar 1 cm no desenho ao
% mundo real.
% B - Incorreta, pois, ao simplificar \frac{10}{1000000}, foi colocado
% um 0 a mais no numerador.
% C - Incorreta, pois, ao simplificar \frac{10}{1000000}, foi colocado
% um 0 a mais no denominador.
% D - Correta, pois, como escala é a razão
% \frac{\text{desen}ho}{\text{real}} e ambos possuem a mesma unidade
% de medida, temos \frac{\text{desen}ho}{\text{real}} = \frac{10\text{cm}}{10\text{km}} = \frac{10}{1000000} = \frac{1}{100000}

\num{5} Um marceneiro está construindo uma mesa de madeira e precisa decidir
qual será o formato dos pés da mesa. Sabendo que ele optou pelo formato
com maior rigidez, os pés vão ter uma forma:

a) circular

b) triangular

c) retangular

d) hexagonal

%\section{BNCC: EF07MA25}
%  -- Reconhecer a rigidez geométrica dos triângulos e suas
% aplicações, como na construção de estruturas arquitetônicas (telhados,
% estruturas metálicas e outras) ou nas artes plásticas.
% SAEB: Identificar propriedades e relações existentes entre os elementos
% de um triângulo.

% A - Incorreta, pois considerou que o círculo é uma figura rígida por não
% possuir pontas.
% B - Correta, pois, como estudado, a forma geométrica que possui maior
% rigidez é o triângulo. Assim, a mesa que apresenta a maior rigidez na
% base é a de base triangular.
% C - Incorreta, pois considerou a estrutura que costuma ver com maior
% frequência no dia a dia.
% D - Incorreta, pois considerou o hexágono com maior rigidez que o
% triângulo.

\num{6} Em uma planta de um apartamento, a sala de estar possui 6 metros de
comprimento e 4 metros de largura. O quarto principal tem o dobro da
área da sala de estar. Qual das opções abaixo representa corretamente a
área do quarto principal?

a) 10 metros quadrados.

b) 12 metros quadrados.

c) 16 metros quadrados.

d) 24 metros quadrados.

% SAEB: Descrever ou esboçar deslocamento de pessoas e/ou de objetos em
% representações bidimensionais (mapas, croquis etc.), plantas de
% ambientes ou vistas, de acordo com condições dadas.

% A - Incorreta, pois o valor encontrado a partir do cálculo da área não é
% esse.
% B - Incorreta, pois o valor encontrado a partir do cálculo da área não é
% esse.
% C - Incorreta, pois o valor encontrado a partir do cálculo da área não é
% esse.
% D - Correta, pois a área da sala de estar é calculada multiplicando o
% comprimento pela largura:
% {6 \; metros} \times {4\; metros} = 24\;m^2 Como o quarto principal
% tem o dobro da área da sala de estar, a área do quarto principal será 48
% metros quadrados.

\num{7} Calcule o volume da caixa representada a seguir:

\includegraphics{./imgSAEB_7_MAT/media/image99.jpg}

a) 24 cm³ b) 24 cm² c) 420 cm³ d) 420 cm²

%\section{BNCC: EF07MA30 }
% -- Resolver e elaborar problemas de cálculo de medida do
% volume de blocos retangulares, envolvendo as unidades usuais (metro
% cúbico, decímetro cúbico e centímetro cúbico).
% SAEB: Resolver problemas que envolvam volume de prismas retos ou
% cilindros retos.

% A - Incorreta, pois considerou que o volume de um bloco retangular é a
% soma das três medidas.
% B - Incorreta, pois além de somar as três medidas, a unidade está
% errada.
% C - Correta, pois, como o volume de um bloco retangular é o produto das
% medidas dos seus lados, temos que
% V = 12 \times 7 \times 5 = 420\text{cm}³.
% D - Incorreta, pois, apesar de o valor do volume ter sido calculado
% corretamente, a unidade de medida está errada.

\num{8} Ao analisar uma tabela de dupla entrada que apresenta a relação entre
o nível de escolaridade e a renda média mensal dos indivíduos, é
possível inferir a finalidade da pesquisa estatística realizada. Qual
das opções abaixo representa corretamente a finalidade dessa pesquisa?

a) Investigar a relação entre a idade e a renda dos indivíduos.

b) Comparar a escolaridade entre diferentes faixas etárias.

c) Identificar os fatores determinantes do nível de escolaridade.

d) Avaliar a influência da escolaridade na renda dos indivíduos.

%\section{BNCC: EF07MA37 }
% -- Interpretar e analisar dados apresentados em gráfico
% de setores divulgados pela mídia e compreender quando é possível ou
% conveniente sua utilização.
% SAEB: Inferir a finalidade da realização de uma pesquisa estatística ou
% de um levantamento, dada uma tabela (simples ou de dupla entrada) ou
% gráfico, (barras simples ou agrupadas, colunas simples ou agrupadas,
% pictóricos, de linhas, de setores ou em histograma) com os dados dessa
% pesquisa.

% A - Incorreta, pois o gráfico não faz essa relação.
% B - Correta, pois o gráfica não apresenta faixas etárias.
% C - Incorreta, pois o gráfico não apresenta tais fatores.
% D - Correta, pois a tabela de dupla entrada apresenta a relação entre
% duas variáveis: o nível de escolaridade e a renda média mensal. A
% finalidade dessa pesquisa é analisar e inferir como o nível de
% escolaridade influencia a renda dos indivíduos. Ao cruzar os dados da
% tabela, é possível observar se há uma relação entre a escolaridade e a
% renda e, assim, avaliar a influência dessa variável na determinação da
% renda média mensal.

\num{9} Dia de quinta-feira é dia de jogar RPG na casa de Pedro. Considerando
que um dado de RPG tem 12 lados, qual é a chance de Pedro tirar o número
4 duas vezes seguidas?

a)\frac{1}{2}

b)\frac{2}{12}

c)\frac{1}{12}

d)\frac{1}{144}

%\section{BNCC: EF07MA34 }
% -- Planejar e realizar experimentos aleatórios ou
% simulações que envolvem cálculo de probabilidades ou estimativas por
% meio de frequência de ocorrências.
% SAEB: Resolver problemas que envolvam a probabilidade de ocorrência de
% um resultado em eventos aleatórios equiprováveis independentes ou
% dependentes.

% A - Incorreta, pois foi feito o cálculo de probabilidade normal, sem
% considerar os eventos e considerando que o dado havia sido jogado uma
% única vez.
% B - Incorreta, pois foi feito o cálculo de uma probabilidade normal sem
% considerar os eventos.
% C - Incorreta, pois foi calculada a probabilidade de acertar a face
% número 4 uma vez.
% D - Correta, pois são eventos independentes. Desse modo:
% P\left( A \right) \times P\left( B \right) = \frac{1}{12} \times \frac{1}{12}\  = \frac{1}{144}.

\num{10} Em uma festa na escola, cada aluno levou um cupcake para comemorar.
Juliana comeu somente a metade de seu doce e cedeu o resto a Leonardo.
Como Leonardo já havia comido o dele, qual fração representa o que ele
consumiu?

a)\frac{1}{8}

b)\frac{1}{2}

c)1\frac{1}{2}

d)5\frac{1}{5}

%\section{BNCC: EF07MA09 }
% -- Utilizar, na resolução de problemas, a associação
% entre razão e fração, como a fração 2/3 para expressar a razão de duas
% partes de uma grandeza para três partes da mesma ou três partes de outra
% grandeza.
% SAEB: Representar frações menores ou maiores que a unidade por meio de
% representações pictóricas ou associar frações a representações
% pictóricas.

% A - Incorreta, pois, o cupcake não estava dividido em oito pedaços.
% B - Incorreta, pois Leonardo consumiu mais de um cupcake.
% C - Correta, pois foi consumido 1 cupcake inteiro e metade do de
% Juliana.
% D - Incorreta, pois Leonardo não consumiu 5 cupcakes inteiros e 1 parte
% entre 4 de outro.

\chapter{Simulado 4}

\num{1} A idade de Alice triplicada e somada a 19 é igual a 40. Quantos anos
ela tem?

a) 3

b) 7

c) 20

d) 40

%\section{BNCC: EF07MA18 }
% -- Resolver e elaborar problemas que possam ser
% representados por equações polinomiais de 1º grau, redutíveis à forma ax
% + b = c, fazendo uso das propriedades da igualdade.
% SAEB: Resolver uma equação polinomial de 1º grau.

% A - Incorreta, pois, se a idade de Alice fosse 3, a idade seria igual a
% 18.
% B - Correta, pois:
% 3x + 19 = 40 \Rightarrow \ 3x = 40 - 19 - \Rightarrow 3x = 21 \Rightarrow x = 7.
% C - Incorreta, pois, se a idade de Alice fosse 20, a soma teria que ser
% 79.
% D - Incorreta, pois, se a idade de Alice fosse 40, a soma teria que ser
% 139.

\num{2} Uma bailarina está no centro de um círculo na primeira posição do
balé, ou seja, com os pés em formato de V. Se o ângulo aberto é de 30º,
quanto vale o arco correspondente?

a) 15º

b) 17º

c) 30º

d) 60º

%\section{BNCC: EF07MA20 }
% -- Reconhecer e representar, no plano cartesiano, o
% simétrico de figuras em relação aos eixos e à origem.
% SAEB: Reconhecer circunferência/círculo como lugares geométricos, seus
% elementos (centro, raio, diâmetro, corda, arco, ângulo central, ângulo
% inscrito).

% A - Incorreta, pois o ângulo não equivale à metade de 30º.
% B - Incorreta, pois esse valor não condiz ao que foi pedido.
% C - Correta, pois, quando o ponto está no centro da circunferência, ele
% apresenta o mesmo tamanho do arco.
% D - Incorreta, pois não equivale o dobro do ângulo.

\num{3} Considere a figura A no plano cartesiano dada por A(2,3). Após a
aplicação da seguinte transformação geométrica, a figura A se transforma
na figura B:

Reflexão em relação ao eixo y, seguida de uma translação de 4 unidades
para a esquerda e 2 unidades para baixo.

Qual é a coordenada da figura B no plano cartesiano?

a) B(-6,1)

b) B(2,-5)

c) B(-2,1)

d) B(6,-1)

%\section{BNCC: EF07MA20 }
% -- Reconhecer e representar, no plano cartesiano, o
% simétrico de figuras em relação aos eixos e à origem.
% SAEB: Identificar, no plano cartesiano, figuras obtidas por uma ou mais
% transformações geométricas (reflexão, translação, rotação).

% A - Correta, pois a reflexão em relação ao eixo y faz com que o ponto
% A(2,3) se transforme no ponto A'(-2,3). Em seguida, a translação de 4
% unidades para a esquerda e 2 unidades para baixo transforma o ponto
% A'(-2,3) no ponto B(-6,1).
% B - Incorreta, pois não leva em consideração a reflexão em relação ao
% eixo y.
% C - Incorreta, pois não leva em consideração a reflexão em relação ao
% eixo y.
% D - Incorreta, pois não leva em consideração a translação de 4 unidades
% para a esquerda e 2 unidades para baixo.

\num{4} Foi pedido a Fernanda que desenhasse um gráfico de uma equação de
primeiro grau. O gráfico de Fernanda tem de parecer com qual dos
gráficos abaixo?

a) \includegraphics{./imgSAEB_7_MAT/media/image102.png}

b) \includegraphics{./imgSAEB_7_MAT/media/image103.png}

c) \includegraphics{./imgSAEB_7_MAT/media/image104.png}

d) \includegraphics{./imgSAEB_7_MAT/media/image105.png}

%\section{BNCC: EF07MA18 }
% -- Resolver e elaborar problemas que possam ser
% representados por equações polinomiais de 1º grau, redutíveis à forma ax
% + b = c, fazendo uso das propriedades da igualdade.
% SAEB: Associar uma equação polinomial de 1º grau com duas variáveis a
% uma reta no plano cartesiano.

% A - Incorreta, pois é um gráfico de equação de segundo grau.
% B - Incorreta, pois é um gráfico de logarítmica
% C - Incorreta, pois é um gráfico modular
% D - Correta: pois se trata de uma reta.

\num{5} Considerando a expressão algébrica s - 15 = 2t, o valor de s
para t = 8\  é:

a) - 5

b) 25

c) 31

d) 1

%\section{BNCC: EF07MA13 }
% -- Compreender a ideia de variável, representada por
% letra ou símbolo, para expressar relação entre duas grandezas,
% diferenciando-a da ideia de incógnita.
% SAEB: Resolver problemas que envolvam cálculo do valor numérico de
% expressões algébricas.

% A - Incorreta, pois, ao invés de multiplicar o valor de t por 2, fez a
% soma e subtraiu o 15 ao invés de somar.
% B - Incorreta, pois ao invés de multiplicar o valor de t por 2 fez a
% soma.
% C - Correta, pois, para encontrar o valor de s, basta substituir o valor
% de t dado. Assim, s - 15 = 2t\  \rightarrow \ s - 15 = 2\  \times \ 8\  \rightarrow \ s - 15 = 16\  \rightarrow \ \ s = 16 + 15 = 31.
% D - Incorreta, pois ao invés de fazer 16 + 15, fez 16-15.

\chapter{Simulado 5}

\num{1} Fabrício colocou em uma tabela o lucro de todos os dias da semana do
seu restaurante. Em quais dias da semana o lucro foi superior a R\$400
reais?

\begin{longtable}[]{@{}ll@{}}
\toprule
\endhead
Dias da semana & Lucro\tabularnewline
Segunda-feira & R\$100\tabularnewline
Terça-feira & R\$80\tabularnewline
Quarta-feira & R\$407\tabularnewline
Quinta-Feira & R\$100\tabularnewline
Sexta-feira & R\$140\tabularnewline
Sábado & R\$500\tabularnewline
Domingo & R\$700\tabularnewline
\bottomrule
\end{longtable}

a) Quarta feira, Sábado e Domingo.

b) Somente domingo

c) Segunda feira, Quarta, Domingo

d) Terça feira, Sexta-feira

%\section{BNCC: EF07MA37 }
% -- Interpretar e analisar dados apresentados em gráfico
% de setores divulgados pela mídia e compreender quando é possível ou
% conveniente sua utilização.
% SAEB: Inferir a finalidade da realização de uma pesquisa estatística ou
% de um levantamento, dada uma tabela (simples ou de dupla entrada) ou
% gráfico, (barras simples ou agrupadas, colunas simples ou agrupadas,
% pictóricos, de linhas, de setores ou em histograma) com os dados dessa
% pesquisa.

% A - Correta, pois, nesses dias, o lucro foi de 407, 500 e 700,
% respectivamente.
% B - Incorreta, pois não levou em conta todo os dias em que o lucro foi
% maior.
% C - Incorreta, pois não levou em conta todo os dias em que o lucro foi
% maior.
% D - Incorreta, pois não levou em conta todo os dias em que o lucro foi
% maior.

\num{2} Luana e Gabriel estavam brincando de cara ou coroa. Qual é a chance
de Luana lançar a moeda e obter duas vezes cara?

a) \frac{1}{2}

b) \frac{2}{4}

c)  \frac{1}{8}

d) \frac{1}{4}


%\section{BNCC: EF07MA34 }
% -- Planejar e realizar experimentos aleatórios ou
% simulações que envolvem cálculo de probabilidades ou estimativas por
% meio de frequência de ocorrências.
% SAEB: Resolver problemas que envolvam a probabilidade de ocorrência de
% um resultado em eventos aleatórios equiprováveis independentes ou
% dependentes.

% A - Incorreta, pois foi feito o cálculo da probabilidade de uma única
% moeda.
% B - Incorreta, pois não considerou cada lance como um evento
% independente.
% C - Incorreta, pois o espaço amostral não pode ser 8.
% D - Correta, pois são eventos independentes, logo
% P\left( A \right) \times P\left( B \right) = \frac{1}{2}\  \times \frac{1}{2} = \frac{1}{4}.

\num{3} 31 pessoas consumiram 8 pizzas de 8 pedaços cada e outros 2 pedaços.
Como podemos representar essa quantidade em fração?

a) \frac{1}{8}

b) \ 4\frac{1}{3}

c) \ 8\frac{1}{4}

d) \ 7\frac{1}{2}


%\section{BNCC: EF07MA09}
%  -- Utilizar, na resolução de problemas, a associação
% entre razão e fração, como a fração 2/3 para expressar a razão de duas
% partes de uma grandeza para três partes da mesma ou três partes de outra
% grandeza.
% SAEB: Representar frações menores ou maiores que a unidade por meio de
% representações pictóricas ou associar frações a representações
% pictóricas.

% A - Incorreta, pois essa fração corresponde a comer somente 1 pedaço da
% pizza.
% B - Incorreta, pois essa fração corresponde a 4 pizzas inteiras e um
% pedaço em uma pizza de corte diferente.
% C - Correta, pois é uma fração imprópria com a parte inteira 8 e mais os
% dois pedaços.
% D - Incorreta, pois essa fração corresponde a 7 pizzas inteiras e metade
% de outra.

\num{4} Pensei em um número. Multipliquei por 3 e somei com 70, obtendo 103.
Qual é esse número?

a) 12

b) 11

c) 20

d) 103

%\section{BNCC: EF07MA18 }
% -- Resolver e elaborar problemas que possam ser
% representados por equações polinomiais de 1º grau, redutíveis à forma ax
% + b = c, fazendo uso das propriedades da igualdade.
% SAEB: Resolver uma equação polinomial de 1º grau.

% A- Incorreta, pois, se o número fosse 12, a soma seria 106.
% B - Correta, pois 3x + 70 = 103 \rightarrow \ 3x = 103 - 70 \rightarrow \ 3x = 33 \rightarrow x = 11.
% C - Incorreta, pois, se o número fosse 20, a soma teria que ser 130.
% D - Incorreta, pois, se o número fosse 103, a soma teria que ser 379.

\num{5} Considere uma circunferência de raio 8 cm. Determine o arco
correspondente a um ângulo central de 45 graus.

a) 2 cm

b) 6 cm

c) 8 cm

d) 16 cm


%\section{BNCC: EF07MA21 }
% -- Reconhecer e construir figuras obtidas por simetrias
% de translação, rotação e reflexão, usando instrumentos de desenho ou
% softwares de geometria dinâmica e vincular esse estudo a representações
% planas de obras de arte, elementos arquitetônicos, entre outros.
% SAEB: Reconhecer circunferência/círculo como lugares geométricos, seus
% elementos (centro, raio, diâmetro, corda, arco, ângulo central, ângulo
% inscrito).

% A - Incorreta, pois essa resposta não corresponde ao cálculo correto do
% arco.
% B - Correta, pois o arco correspondente ao ângulo central de 45 graus
% possui um comprimento de aproximadamente 6.28 cm.
% C - Incorreta, pois essa resposta corresponde ao perímetro da
% circunferência, não ao comprimento do arco de um ângulo central
% específico.
% D - Incorreta, pois essa resposta corresponde ao dobro do perímetro da
% circunferência, não ao comprimento do arco de um ângulo central
% específico.

\num{6} Dois corredores saem da linha de partida ao mesmo tempo. Sabendo que
um deles percorre toda a pista em 15 minutos e o outro, em 25 minutos,
dentro de quanto tempo eles se encontram novamente na linha de partida?

a) 5 minutos b) 40 minutos. c) 1 hora e 5 minutos d) 1 hora e 15 minutos

%\section{BNCC: EF07MA04 }
% -- Resolver e elaborar problemas que envolvam operações
% com números inteiros.
% SAEB: Resolver problemas que envolvam as ideias de múltiplo, divisor,
% máximo divisor comum ou mínimo múltiplo comum.

% A - Incorreta, pois considerou que o tempo de encontro seria pelo M.D.C
% ao invés do M.M.C. . B - Incorreta, pois apenas foi feita a soma do
% tempo de cada corredor. C - Incorreta, pois foi associado um número
% errado de minutos. D - Correta, pois devemos usar a ideia de mínimo
% múltiplo comum. Assim, calculando o M.M.C dos tempos, temos: 15, 25, 3; 5, 25, 5; 1, 5, 5; 1, 1, (3x5x5 = 75). Logo, os atletas se encontram depois de 75 minutos, ou seja, 1 hora e 15
% minutos.

\num{7} A porcentagem possui diversas representações: a percentual, a
fracionária e a decimal. Considerando a porcentagem, 25,3\% as suas
representações são:

a) 2,53\ e\frac{253}{100} b) 0,253\ e\frac{253}{1000} c)
0,253\ e\frac{253}{100} d) 2,53\ e\frac{253}{1000}

%\section{BNCC: EF07MA02 }
% --: Resolver e elaborar problemas que envolvam
% porcentagens, como os que lidam com acréscimos e decréscimos simples,
% utilizando estratégias pessoais, cálculo mental e calculadora, no
% contexto de educação financeira, entre outros.
% SAEB: - Resolver problemas que envolvam porcentagens, incluindo os que
% lidam com acréscimos e decréscimos simples, aplicação de percentuais
% sucessivos e determinação de taxas percentuais.

% A - Incorreta, pois considerou que o numerador da fração centesimal é a
% forma percentual sem a vírgula.
% B - Correta, pois
% 25,3\% = \frac{25,3}{100} = \frac{253}{1000} = 0,253.
% C - Incorreta, pois esqueceu de adicionar um 0 no denominador da
% representação fracionária ao andar com a vírgula.
% D - Incorreta, pois se esqueceu de levar em conta as 3 casas decimais no
% denominador.

\num{8} A corrida de um táxi na cidade de São Paulo custa
R\(1,50 por km rodado mais a bandeira de R\) 6,50. Uma pessoa que pagou
R\$ 18,50 por uma corrida percorreu um total de:

a) 18\ \text{km} b) \ 16,6\ \text{km} c) 12,3\ \text{km} d)
8\ \text{km}

%\section{BNCC: EF07MA13 }
% -- Compreender a ideia de variável, representada por
% letra ou símbolo, para expressar relação entre duas grandezas,
% diferenciando-a da ideia de incógnita.
% SAEB: Resolver problemas que envolvam cálculo do valor numérico de
% expressões algébricas.

% A - Incorreta, pois ao invés de fazer \frac{12}{1,5}\ calculou 12 x
% 1,5. B - Incorreta, pois ao invés de fazer 18,5 - 6,5 fez uma soma. C -
% Incorreta, pois considerou apenas o preço pago por km sem o valor da
% bandeira. D - Correta, pois, como a corrida tem um preço por km mais um
% fixo, podemos escrever de maneira geral que a corrida custa
% c = 1,5q + 6,5. Assim, 18,5 = 1,5q + 6,5 \rightarrow 1,5q = 12 \rightarrow \ q = \frac{12}{1,5} \rightarrow q = 8\text{km}.

\num{9} Um atendente de um banco gasta 1 horas e 40 minutos para atender 12
clientes. No quinto dia útil o atendimento acaba sendo maior, então para
atender 18 clientes, o atendente vai gastar:

a) 2 horas e 30 minutos b) 1 hora e 30 minutos c) 1 hora e 7 minutos d)
1 hora e 50 minutos

%\section{BNCC: EF07MA17 }
% -- Resolver e elaborar problemas que envolvam variação de
% proporcionalidade direta e de proporcionalidade inversa entre duas
% grandezas, utilizando sentença algébrica para expressar a relação entre
% elas.
% SAEB: Resolver problemas que envolvam variação de proporcionalidade
% direta~ou inversa entre duas ou mais grandezas.

% A - Correta, pois, quanto mais pessoas para atender, mais tempo será
% gasto, logo são G.D.P. Para fazer o cálculo, precisamos do tempo em uma
% unidade apenas, ou seja, 1 hora e 40 minutos será 100 minutos. Assim, \frac{100}{12} = \frac{x}{18} \rightarrow 12x = 1800 \rightarrow x = \frac{1800}{12} = 150\ \text{minutos} = 2\ h\text{oras}\ e\ 30\ \text{minutos}. B - Incorreta, pois, ao converter 150 minutos para horas, considerou 1
% hora ao invés de 2. C - Incorreta, pois considerou as grandezas como sendo inversamente
% proporcionais, fazendo 100\ .\ 12\  = \ 18x. D - Incorreta, pois considerou que 1 hora tem 100 minutos.

\num{10} Considere um triângulo ABC, onde AB = 8 cm, AC = 6 cm e BC = 10 cm.
Qual é a medida da altura relativa ao lado AB?

a) 2 cm

b) 3 cm

c) 4 cm

d) 5 cm


%\section{BNCC: EF07MA23 }
% -- Verificar relações entre os ângulos formados por retas
% paralelas cortadas por uma transversal, com e sem uso de softwares de
% geometria dinâmica.
% SAEB: Resolver problemas que envolvam relações entre ângulos formados
% por retas paralelas cortadas por uma transversal, ângulos internos ou
% externos de polígonos ou cevianas (altura, bissetriz, mediana,
% mediatriz) de polígonos.

% A - Incorreta, pois 2 cm é a medida da altura relativa ao lado BC.
% B - Incorreta, pois 3 cm é a medida da altura relativa ao lado AC.
% C - Correta, pois a altura relativa ao lado AB divide o triângulo ABC em
% dois triângulos retângulos, onde a altura é a hipotenusa e os catetos
% são os segmentos AH e BH. Podemos utilizar o teorema de Pitágoras para
% calcular a medida da altura e chegar ao resultado.
% D - Incorreta, pois 5 cm é a medida da mediana relativa ao lado AB.

\chapter{Simulado 6}

\num{1} Qual das seguintes afirmações melhor descreve a diferença entre
mapa, planta e croqui?

\item
  Um mapa é uma representação de uma região ou área, enquanto uma planta
  é um desenho esquemático de uma construção, e um croqui é uma vista
  aérea de uma cidade ou paisagem.
\item
  Um mapa é uma vista aérea de uma cidade ou paisagem, enquanto uma
  planta é uma representação de uma região ou área, e um croqui é um
  desenho esquemático de uma construção.
\item
  Um mapa é um desenho esquemático de uma construção, enquanto uma
  planta é uma representação de uma região ou área, e um croqui é uma
  vista aérea de uma cidade ou paisagem.
\item
  Um mapa, uma planta e um croqui são todos sinônimos e podem ser usados
  de forma intercambiável.

% SAEB: Descrever ou esboçar deslocamento de pessoas e/ou de objetos em
% representações bidimensionais (mapas, croquis etc.), plantas de
% ambientes ou vistas, de acordo com condições dadas.

% A - Correta, pois essa definição está correta.
% B - Incorreta, pois as primeiras definições estão incorretas.
% C - Incorreta, pois a definição de planta está incorreta.
% D - Incorreta, pois há diferenças entre esses conceitos.

\num{2} Qual é a área de um triângulo retângulo com base de 10 cm e altura
de 6 cm?

a) 30 cm²

b) 36 cm²

c) 45 cm²

d) 60 cm²

%\section{BNCC: EF07MA31 }
% -- Estabelecer expressões de cálculo de área de
% triângulos e de quadriláteros.
% SAEB: Resolver problemas que envolvam área de figuras planas.

% A - Incorreta, pois essa resposta corresponde ao cálculo incorreto da
% área.
% B - Correta, pois, substituindo os valores fornecidos na fórmula, temos:
% Área do triângulo = \frac {(10 \times 6)}{2} = \frac {60}{2} =
% 30 \;cm^2. Portanto, a área do triângulo retângulo é de 30 cm².
% A = \frac{\left( b + B \right)h}{2} = \frac{(6 + 18) \times 10}{2} = 24 \times 5 = 120m².
% C - Incorreta, pois essa resposta não corresponde ao cálculo correto da
% área do triângulo com os valores fornecidos.
% D - Incorreta, pois essa resposta corresponde à multiplicação da base
% pela altura, sem dividir por 2, o que resulta em uma área incorreta.

\num{3} Em um plano cartesiano, foi desenhada uma letra L no primeiro
quadrante. Se eu desenhar a mesma letra com as mesmas dimensões, mas sem
utilizar da simetria dos pares ordenados, com duas unidades para cima,
qual transformação geométrica ocorrerá?

a) Translação

b) Rotação

c) Reflexão

d) Imersão

%\section{BNCC: EF07MA20 }
% -- Reconhecer e representar, no plano cartesiano, o
% simétrico de figuras em relação aos eixos e à origem.
% SAEB: Identificar, no plano cartesiano, figuras obtidas por uma ou~ mais
% transformações geométricas (reflexão, translação, rotação).

% A - Correta, pois como não saiu do eixo e só subiu duas unidades da
% simetria, ocorreu a translação.
% B - Incorreta, pois, para ocorrer a rotação, teríamos uma inclinação.
% C - Incorreta, pois a transformação geométrica reflexão apenas repetiria
% a figura.
% D - Incorreta, pois a imersão não é uma transformação geométrica.

\num{4} Em uma competição de natação, Renato, Yan, Cristian e Rafael
completaram as provas, respectivamente, nos seguintes tempos: 7,573
7,563 7,482 7,568. Quem foi o terceiro colocado entre eles?

a) Yan

b) Renato

c) Cristian

d) Rafael

%\section{BNCC: EF07MA10 }
% -- Comparar e ordenar números racionais em diferentes
% contextos e associá-los a pontos da reta numérica.
% SAEB: Comparar ou ordenar números reais, com ou sem suporte da reta
% numérica, ou aproximar número reais para múltiplos de potência de 10
% mais próxima.

% A - Incorreta, pois Yan é o segundo colocado.
% B- Incorreta, pois Renato é o último colocado.
% C - Incorreta, pois Cristian ficou em primeiro lugar.
% D - Correta, pois Rafael ficou na terceira posição.

\num{5} Considerando a expressão algébrica x - y + 20 = 0, o valor de x
para y = \  - 2\ é:

a) 22

b) - 22

c) - 18

d) 18

%\section{BNCC: EF07MA13 }
% -- Compreender a ideia de variável, representada por
% letra ou símbolo, para expressar relação entre duas grandezas,
% diferenciando-a da ideia de incógnita.
% SAEB: Resolver problemas que envolvam cálculo do valor numérico de
% expressões algébricas.

% A - Incorreta, pois não trocou a operação do 22 ao mudar de lado na
% igualdade.
% B - Correta, pois, para encontrar o valor de x, basta substituir o valor
% de y. Assim, x - y + 20 = 0 \rightarrow x - ( - 2) + 20 = 0 \rightarrow \ x + 22 = 0 \rightarrow \ x = - 22.
% C - Incorreta, pois, ao substituir o y, a mudança de sinal não foi
% feita.
% D - Incorreta, pois, além de não fazer a mudança do sinal com o y, a
% operação do 18 não foi feita.

\chapter{Simulado 7}

\num{1} Em uma corrida, o primeiro lugar ganhou com diferença de 0,035 de
tempo para o segundo lugar. A diferença de tempo por extenso é:

a)~35 segundos

b) 35 milésimos de segundos

c) 35 centésimos de segundos

d) 35 minutos

%\section{BNCC: EF07MA10 }
% -- Comparar e ordenar números racionais em diferentes
% contextos e associá-los a pontos da reta numérica.
% SAEB: Escrever números racionais (representação fracionária ou decimal
% finita) em sua representação por algarismos ou em língua materna ou
% associar o registro numérico ao registro em língua materna.

% A - Incorreta, pois 35 segundos não é representado por números decimais.
% B - Correta, pois, quando há três casas decimais, temos milésimos de
% segundos.
% C - Incorreta, pois 35 centésimos de segundos apareceria com duas casas
% decimais após a vírgula.
% D - Incorreta, pois 3 minutos é uma quantidade representada por um
% número inteiro.

\num{2} A data para a eleição do novo diretor da escola foi marcada para o
final do mês de novembro. A escola possui 9 turmas, sendo que somente
uma turma de 9º ano ainda não pode votar. Duas alunas tiveram a ideia de
pesquisar a intenção de voto de algumas pessoas, angariando a opinião de
3 turmas completas. Marque a opção que indica, respectivamente, a
população e a amostra dessa pesquisa.

a) 9 turmas, 3 turmas

b) 3 turmas, 9 turmas

c) 8 turmas, 3 turmas

d) 3 turmas, 8 turmas


%\section{BNCC: EF07MA35 }
% -- Compreender, em contextos significativos, o
% significado de média estatística como indicador da tendência de uma
% pesquisa, calcular seu valor e relacioná-lo, intuitivamente, com a
% amplitude do conjunto de dados.
% SAEB: Identificar os indivíduos (universo ou população-alvo da
% pesquisa), as variáveis e os tipos de variáveis (quantitativas ou
% categóricas) em um conjunto de dados.

% A - Incorreta, pois, apesar de a escola ter 9 turmas, uma delas é
% excluída da votação.
% B - Incorreta, pois o valor de amostra e população estão invertidos.
% C - Correta, pois temos 8 turmas como população e 3 turmas de amostra.
% D - Incorreta, pois o valor da amostra e população estão invertidos.

\num{3} Ana Clara encheu \frac{4}{8} do galão de água da casa dela. Um
valor equivalente seria

a) Metade

b) Inteiro

c) \frac{1}{4}

d) \frac{1}{3}


%\section{BNCC: EF07MA08}
%  -- Comparar e ordenar frações associadas às ideias de
% partes de inteiros, resultado da divisão, razão e operador.
% SAEB:~Identificar frações equivalentes.

% A - Correta, pois \frac{4}{8} é equivalente a \frac{1}{2}.
% B - Incorreta, pois a fração não corresponde ao galão inteiro, uma vez
% que encheu 4 partes de 8 divididas.
% C - Incorreta, pois a fração irredutível encontrada está errada.
% D - Incorreta, pois essa fração não representa a parte cheia do galão.

\num{4} Yuri pensou em um número que, somado a 1.200, é igual a quatro vezes
si mesmo. O número que Yuri pensou é:

a) 355

b) 400

c) 200

d) 420

%\section{BNCC: EF07MA18}
%  -- Resolver e elaborar problemas que possam ser
% representados por equações polinomiais de 1º grau, redutíveis à forma ax
% + b = c, fazendo uso das propriedades da igualdade.
% SAEB: Resolver uma equação polinomial de 1º grau.

% A - Incorreta, pois, se o número fosse 355, o resultado final seria
% diferente.
% B - Correta, pois
% x + 1200 = 4x \rightarrow \ 4x - x = 1200 \rightarrow 3x = 1200 \rightarrow x = 400.
% C - Incorreta, pois, se o número fosse 200, o resultado final seria
% diferente.
% D - Incorreta, pois, se o número fosse 420, o resultado final seria
% diferente.

\num{5} Qual é o arco de uma circunferência de raio 10 cm, considerando um
ângulo central de 60°?

a) 10π cm

b) 6π cm

c) 3π cm

d) 2π cm


%\section{BNCC: EF07MA19 }
% -- Realizar transformações de polígonos representados no
% plano cartesiano, decorrentes da multiplicação das coordenadas de seus
% vértices por um número inteiro.
% SAEB: Reconhecer circunferência/círculo como lugares geométricos, seus
% elementos (centro, raio, diâmetro, corda, arco, ângulo central, ângulo
% inscrito).

% A - Incorreta, pois o valor do ângulo central não foi considerado
% corretamente.
% B - Incorreta, pois o valor do ângulo central não foi considerado
% corretamente.
% C - Correta, pois Arco = \frac {Ângulo}{360º} \times 2π \times raio.
% No caso do problema, substituindo os valores na fórmula, temos 3π cm.
% D - Incorreta, pois o valor do ângulo central não foi considerado
% corretamente.

\num{6} A transformação geométrica que ocorre com objetos bidimensionais
posicionados diante de um espelho perpendicular é:

a) Rotação

b) Reflexão

c) Translação

d) Inchaço

%\section{BNCC: EF07MA20}
%  -- Reconhecer e representar, no plano cartesiano, o
% simétrico de figuras em relação aos eixos e à origem.
% SAEB: Identificar, no plano cartesiano, figuras obtidas por uma ou~ mais
% transformações geométricas (reflexão, translação, rotação).

% A - Incorreta, pois a rotação acontece quando o objeto inclina-se.
% B - Correta, pois o ato de posicionar um espelho reflete a imagem.
% C - Incorreta, pois a translação ocorre quando o objeto segue pelas
% direções norte, sul, leste e oeste.
% D - Incorreta, pois no inchaço há deformação na imagem.

\num{7}~ O professor de Matemática resolveu fazer uma brincadeira e contar as
notas dos alunos em forma de fração, Lúcio tirou \(\frac{19}{3}\). Onde
está localizado este número na reta numérica?

\includegraphics{./imgSAEB_7_MAT/media/image109.png}

a) Entre o 5 e o 6

b) Entre o 9 e o 10

c) Próximo ao 1

d) Entre o 6 e o 7


%\section{BNCC: EF07MA01 }
% -- Resolver e elaborar problemas com números naturais,
% envolvendo as noções de divisor e de múltiplo, podendo incluir máximo
% divisor comum ou mínimo múltiplo comum, por meio de estratégias
% diversas, sem a aplicação de algoritmos.
% SAEB: Comparar ou ordenar números reais, com ou sem suporte da reta
% numérica, ou aproximar número reais para múltiplos de potência de 10
% mais próxima.

% A - Incorreta, pois a divisão de 19 por 3 é maior que 6.
% B - Incorreta, pois a divisão de 19 por 3 é menor que 7.
% C - Incorreta, pois a divisão de 19 por 3 é maior que 1.
% D - Correta, pois, analisando a fração como divisão, temos que o 3 está
% dentro do 9 aproximadamente 6 vezes, logo, a fração é maior do que 6 e
% menor do que 7.

\num{8} Marcos está montando uma lembrança para seus alunos e resolveu fazer
saquinhos com doces. Ele comprou 75 balas, 60 pirulitos e 45 chocolates,
e quer dividi-los com a maior quantidade possível de cada doce por
lembrancinha. Assim, em cada saquinho, o número de lembranças será

a) 5

b) 8

c) 15

d) 20


%\section{BNCC: EF07MA04 }
% -- Resolver e elaborar problemas que envolvam operações
% com números inteiros.
% SAEB: Resolver problemas que envolvam as ideias de múltiplo, divisor,
% máximo divisor comum ou mínimo múltiplo comum.

% A - Incorreta, pois esqueceu de considerar que o 3 também dividiu os
% valores simultaneamente.
% B - Incorreta, pois, ao invés de multiplicar o 3 e o 5, fez a soma
% deles.
% C - Correta, pois, para dividir as lembranças na maior quantidade
% possível, usamos o cálculo do M.D.C; assim, fatorando a quantidade de
% cada doce, temos 75, 60, 45, 2; 75, 30, 45, 2; 75, 15, 45, 3; 25, 5, 15, 3; 25, 5, 5, 5; 5, 1, 1, 5; 1, 1, 1. Logo, em cada saquinho terá 15 doces.
% D - Incorreta, pois, ao fatorar, considerou que usaria todos os fatores
% como se fosse M.M.C.

\num{9} Larissa decidiu comprar um videogame de presente para seu sobrinho.
Fazendo uma pesquisa na internet, encontrou um valor de R\$3.500,00 com
desconto de 5\% à vista. Ela já havia procurado na cidade e encontrado o
mesmo aparelho por R\$3.800,00 com desconto de 10\% à vista. Sabendo que
a loja virtual cobra um frete de R\$35,00 e que Larissa vai comprar à
vista, a melhor opção dela é:

a) Na loja virtual, pagando R\$3.360,00 b) Na loja virtual, pagando
R\$3.325,00 c) Na loja física, pagando R\$3.360,00 d) Na loja física,
pagando R\$3.420,00


%\section{BNCC: EF07MA02 }
%-- Resolver e elaborar problemas que envolvam
% porcentagens, como os que lidam com acréscimos e decréscimos simples,
% utilizando estratégias pessoais, cálculo mental e calculadora, no
% contexto de educação financeira, entre outros.
% SAEB: - Resolver problemas que envolvam porcentagens, incluindo os que
% lidam com acréscimos e decréscimos simples, aplicação de percentuais
% sucessivos e determinação de taxas percentuais.

% A - Correta, pois \text{Na}\ \text{loja}\ \text{virtual} \rightarrow 3500 \times 0,95 = 3.325 + 35 = 3360. \text{Na}\ \text{loja}\ fí\text{sica} \rightarrow 3800 \times 0,9 = 3420. Portanto, a melhor opção de Larissa é a loja virtual.
% B - Incorreta, pois não considerou o valor do frete da loja virtual.
% C - Incorreta, pois confundiu as informações da loja física com a
% virtual.
% D - Incorreta, pois, ao realizar as operações, o resultado obtido é
% outro.

\num{10} Analisando as sequências abaixo, podemos classificá-las como
recursivas ou não recursivas na seguinte ordem:

\text{I.} \; \left( 1,\ 4,\ 7,\ 10,\ 13,\ \ldots \right)

\text{II}.\ ( - 3,\  - 5\ , - \ 7,\  - 9,\  - 11,\ \ldots)

\text{III}.\ (2,\ 3,\ 5,\ 7,\ 11,\ \ldots)

a) Recursiva, não recursiva, recursiva

b) Recursiva, recursiva, recursiva

c) Recursiva, não recursiva, não recursiva

d) Recursiva, recursiva, não recursiva

%\section{BNCC: EF07MA14}
%  -- Classificar sequências em recursivas e não recursivas,
% reconhecendo que o conceito de recursão está presente não apenas na
% matemática, mas também nas artes e na literatura.
% SAEB: Identificar uma representação algébrica para o padrão ou a
% regularidade de uma sequência de números racionais ou representar
% algebricamente o padrão ou a regularidade de uma sequência de números
% racionais.

% A - Incorreta, pois considerou que a sequência 2 não apresenta uma
% relação entre os termos e que os números primos possuem algum padrão.
% B - Incorreta, pois considerou que a sequência dos números primos pode
% ser encontrada por alguma relação entre os termos.
% C - Incorreta, pois não enxergou que a sequência 2 apresenta uma relação
% entre os termos por meio do antecessor subtraído de 2.
% D - Correta, pois podemos observar que a sequência I é determinada pelo
% termo anterior mais 3, a sequência II pelo termo anterior menos 2, e a
% III é a sequência dos números primos. Como a sequência recursiva é
% aquela em que um termo depende do anterior, temos recursiva, recursiva e
% não recursiva.

\chapter{Simulado 8}

\num{1} Sabendo que x + y + z = 38, o valor de x, y, z na proporção
abaixo é, respectivamente:

\frac{x}{42} = \frac{y}{16} = \frac{z}{18}

a) 84, 32, 36

b) 21, 8, 9

c) 84, 36, 32

d) 21, 9, 8

%\section{BNCC: EF07MA17 }
% -- Resolver e elaborar problemas que envolvam variação de
% proporcionalidade direta e de proporcionalidade inversa entre duas
% grandezas, utilizando sentença algébrica para expressar a relação entre
% elas.
% SAEB: Resolver problemas que envolvam variação de proporcionalidade
% direta~ou inversa entre duas ou mais grandezas.

% A - Incorreta, pois, ao determinar o valores de x, y, z, foi feita uma
% multiplicação por 2 ao invés de uma divisão.
% B - Correta, pois \frac{x}{42} = \frac{y}{16} = \frac{z}{18} = \frac{x + y + z}{42 + 16 + 18} = \frac{38}{76} = \frac{1}{2}. \frac{x}{42} = \frac{1}{2} \rightarrow \ \ x = 21. \frac{y}{16} = \frac{1}{2} \rightarrow \ \ x = 8\ \ \ \ \ \ \ \ \ \ \ \ \frac{z}{18} = \frac{1}{2} \rightarrow \ \ x = 9.
% C - Incorreta, pois, além de calcular o valor das incógnitas usando a
% multiplicação, os valores que seriam y e z foram invertidos.
% D - Incorreta, pois, apesar de fazer os cálculos corretos, a ordem dos
% valores de y e z foi invertida.

\num{2} Em um triângulo retângulo, a hipotenusa mede 10 cm e um dos catetos
mede 6 cm. Qual é a medida do ângulo agudo oposto a esse cateto?

a) 30°

b) 45°

c) 60°

d) 90°


%\section{BNCC: EF07MA23 }
% -- Verificar relações entre os ângulos formados por retas
% paralelas cortadas por uma transversal, com e sem uso de softwares de
% geometria dinâmica.
% SAEB: Resolver problemas que envolvam relações entre ângulos formados
% por retas paralelas cortadas por uma transversal, ângulos internos ou
% externos de polígonos ou cevianas (altura, bissetriz, mediana,
% mediatriz) de polígonos.

% A - Incorreta, pois não corresponde à medida correta do ângulo agudo.
% B - Incorreta, pois não corresponde à medida correta do ângulo agudo.
% C - Correta, pois, no triângulo retângulo, o ângulo agudo oposto ao
% cateto é sempre o complementar do ângulo formado entre a hipotenusa e o
% cateto. Esse ângulo pode ser encontrado usando a função trigonométrica
% seno.
% D - Incorreta, pois corresponde ao ângulo reto, não ao ângulo agudo
% oposto ao cateto.

\num{3} Qual é o perímetro de um retângulo com comprimento de 10 cm e
largura de 5 cm?

a) 10 cm

b) 15 cm

c) 20 cm

d) 30 cm

%SAEB: Resolver problemas que envolvam perímetro de figuras planas.

% A - Incorreta, pois corresponde à soma apenas dos comprimentos das duas
% arestas menores.
% B - Incorreta, pois corresponde à soma apenas dos comprimentos das duas
% arestas maiores.
% C - Correta, pois o perímetro de um retângulo é dado pela soma dos
% comprimentos de todos os lados, ou seja, a soma dos comprimentos das
% quatro arestas.
% No caso do retângulo descrito no problema, temos duas arestas de
% comprimento 10 cm e duas arestas de comprimento 5 cm. Portanto, o
% perímetro é dado por: P = 10 cm + 10 cm + 5 cm + 5 cm P = 20 cm + 10 cm P = 30 cm. 
% D - Incorreta, pois corresponde à soma de todos os lados do retângulo,
% incluindo os comprimentos das arestas duas vezes.

\num{4} Qual é a área de um triângulo retângulo com base de 8 cm e altura de
6 cm?

a) 12 cm²

b) 24 cm²

c) 30 cm²

d) 48 cm²

%\section{BNCC: EF07MA32}
%  -- Resolver e elaborar problemas de cálculo de medida de
% área de figuras planas que podem ser decompostas por quadrados,
% retângulos e/ou triângulos, utilizando a equivalência entre áreas.
%SAEB: Resolver problemas que envolvam área de figuras planas.

% A - Incorreta, pois corresponde à metade da área correta do triângulo.
% B - Correta, pois a fórmula para calcular a área de um triângulo é dada
% pela metade do produto da base pela altura: A =
% \frac {(base \times altura)}{2}. Substituindo os valores do problema, temos: A = 24 cm².
% C - Incorreta, pois não corresponde à área correta do triângulo.
% D - Incorreta, pois corresponde à área do retângulo formado pela base e
% altura do triângulo, não à área do próprio triângulo.

\num{5} Considerando a expressão algébrica 2a + 8 - 3b = 5, o valor de b
para a = - 3\ é:

a) - 1

b) 1

c) 3

d) - \frac{7}{3}

%\section{BNCC: EF07MA13 }
% -- Compreender a ideia de variável, representada por
% letra ou símbolo, para expressar relação entre duas grandezas,
% diferenciando-a da ideia de incógnita.
% SAEB: Resolver problemas que envolvam cálculo do valor numérico de
% expressões algébricas.

% A - Correta, pois, para encontrar o valor de b, basta substituir o valor
% de a. Assim, 2a + 8 - 3b = 5\  \rightarrow \ 2\  \times \ \left( - 3 \right) + 8 - 3b = 5 \rightarrow \  - 6 + 8 - 3b = 5 - 3b = 3\  \rightarrow \ b = \  - 1.
% B - Incorreta, pois, ao dividir 3 por -3, a regra de sinal não foi
% respeitada.
% C - Incorreta, pois ao fazer 2\  \times \ \left( - 3 \right), foi
% encontrado 6 ao invés de -6.
% D - Incorreta, pois, ao passar o 2 de lado, a regra do sinal foi
% ignorada.

\end{comment}