%!TEX root=./LIVRO.tex

\chapter{Respostas}

\footnotesize

\pagecolor{gray!40}

\section*{Matemática — Módulo 1 — Treino}

\begin{enumerate}
\item a) Incorreta. A raiz quadrada de 3 não é natural; 0,2222 não é inteiro; a raiz quadrada de 2 não é racional.
b) Correta. Os números foram classificados corretamente.
c) Incorreta. A raiz quadrada de 3 não é racional; 0,2222 não é irracional; 5,363636 não é irracional; a raiz quadradad de 2 não é racional.
d) Incorreta. A raiz quadrada de 3 não é racional.


\item a) Correta. Ao efetuarmos a divisão, temos 0,25.
b) Incorreta. Ao efetuarmos a divisão, temos 2,5.
c) Incorreta. Ao efetuarmos a divisão, temos 1,42.
d) Incorreta. Ao efetuarmos a divisão, temos 1,5.


\item a) Incorreta. Todo número decimal finito pode ser representado por fração e o número 2 é o único par que é primo.
b) Correta. Essas são as afirmações certas.
c) Incorreta. O número 2 é o único número par que é primo, então é uma verdade.
d) Incorreta. A afirmação III é falsa. O número 21 não é primo, ele possui 4 divisores: 1, 3,7 e 21.
\end{enumerate}

\section*{Matemática — Módulo 2 — Treino}

\begin{enumerate}
\item a) Incorreta. Não foi considerada a diferença de fuso horário, apenas a soma das horas de maneira direta.
b) Incorreta. O cálculo das horas foi feito da maneira correta, porém a mudança de dia não foi considerada.
c) Correta. Saída de São Paulo as 8h do dia 23/03 $\rightarrow$ em Dubai vão ser
15h do dia 23/03, $(4 - ( - 3) = 4 + 3 = 7h$ de diferença). A viagem de São Paulo para Dubai dura 14 horas, assim Pedro vai chegar
em Dubai às 5h do dia 24/03, que serão 12h do dia 24/03 na Austrália
$(11 - 4 = 7{h}$ de diferença). A viagem de Dubai até a Austrália são 18 horas, assim Pedro vai chegar
na Austrália às 6h do dia 25/03.
d) Incorreta. O dia está correto, mas foi considerado o cálculo sem a diferença de fuso horário.


\item a) Incorreta. Essa resposta não considera o fato de que cada algarismo deve ser diferente dos demais.
b) Incorreta. Essa resposta não leva em conta que a quantidade de opções vai diminuindo a cada algarismo escolhido.
c) Correta. A quantidade de números de quatro algarismos
diferentes que podem ser formados utilizando-se cinco algarismos é determinada pelo princípio multiplicativo. Portanto, o número total de
combinações possíveis é dado pelo produto das opções disponíveis para
cada algarismo: $5 \times 4\times 3\times 2  = 120$
d) Incorreta. Essa resposta excede o número máximo de combinações possíveis, uma vez que estamos limitados a quatro algarismos diferentes.


\item a) Incorreta. Considerou-se apenas o tamanho de um lado do terreno e ignorou-se o comprimento do portão.
b) Incorreta. Considerou-se apenas o tamanho de um lado do terreno.
c) Incorreta. Não se considerou o comprimento do portão do perímetro do terreno.
d) Correta. Como a área de um quadrado é dado por $l^2$ - onde l é o tamanho do lado -, para calcular o tamanho do lado basta extrair a raiz quadrada da área. Assim
\end{enumerate}

\section*{Matemática — Módulo 3 — Treino}


\begin{enumerate}
\item a) Incorreta. Não se subtraiu a parte inteira do denominador.
b) Incorreta. Quando há 0 no denominador, temos um antiperíodo na dízima periódica.
c) Incorreta. Desconsiderou-se a parte inteira da fração.
d) Correta. Para encontrar o numerador, calculamos $795 - 7 = 788$ no numerador e colocamos 99 no denominador, já que há dois algarismos no período.

\rosa{ Correta, pois: $(2^10) = 1024$; $1024 \cdot 32 = 32.768$ megabytes.}
\rosa{ Incorreta, o aluno chegaria a esse resultado ao considerar que $(2^{10})$ seja 1.000 ao invés de 1.024.}
\rosa{ Incorreta, pois, ao considerar um 2 a menos na expressão, o aluno chegaria a esse resultado.}
\rosa{ Incorreta, pois, ao realizar apenas a multiplicação ao invés de realizar o cálculo da potência, o aluno chegaria a esse resultado.}
\item a) Incorreta. Marcelo foi o melhor colocado.
b) Correta. Comparando-se as frações de mesmo denominador, Fabiana teve o pior desempenho e a fração de Juliano é equivalente ao resultado de Fabiana.
c) Incorreta. Maicon ficou em segundo colocado.
d) Incorreta. Marcelo e Maicon foram os primeiros colocados.

\item a) Incorreta. Essa fração representa a parte de um todo quando uma
parte é considerada. No caso, João comprou mais do que 1/3 do terreno.
b) Correta. O terreno foi dividido em três partes iguais. João
comprou 2 dessas partes. Para determinar a fração que expressa a parte
do terreno que João comprou, devemos considerar que ele comprou 2 partes
de um total de 3 partes do terreno.
c) Incorreta. Essa fração não corresponde à proporção de partes do terreno que João comprou.
d) Incorreta. Essa fração também não representa corretamente a proporção de partes do terreno que João comprou.

\end{enumerate}


\section*{Matemática — Módulo 4 — Treino}

\begin{enumerate}
\item a) Incorreta. Essa resposta não leva em consideração o acréscimo de 15\% no valor original.
b) Incorreta. Essa resposta também não considera o acréscimo correto. É importante lembrar que 15\% de R\$200,00 é igual a R\$30,00, não R\$20,00.
c) Correta. O novo valor = Valor original + (Porcentagem de acréscimo vezes o Valor original). Nesse caso, o valor original é
R\$200,00 e o acréscimo é de 15\%. Novo valor $= 200 + \frac {15}{100}\times 200.$ Novo valor $= 200 +
0,15 \times 200$. Novo valor $= 200 + 30$. $Novo valor = 230$
d) Incorreta. Representa o valor original acrescido de 22,5\%, não de 15\%.

\item a) Incorreta. A conta foi feita como se o apartamento custasse R\$27.000.
b) Incorreta, pois somente considerou o valor de uma parcela como
resposta.
c) Correta. $60 \times 4500 = 270.000$. 15\% $\rightarrow$ fator $0,85 \rightarrow 0,85 \times 270.000 = 229.500. 270.000 - 229.500 = 40.500,00$
d) Incorreta. A conta $270.000 - 229.500$ foi feita
incorretamente.

\item a) Incorreta. Considerou-se apenas o aumento de 6\%.
b) Incorreta. Considerou-se o fator de multiplicação como sendo 1,04.
c) Incorreta. Calculou-se o fator de multiplicação como 100\% + 6\% - 10\%.
d) Correta. Aumento de 6\%\  = \ 1,06. Desconto de $10\%\  = \ 0,9. 270 \times 1,06 \times 0,9 = 257,58$. No final do semestre, o pneu saiu por aproximadamente R\$258,00.
\end{enumerate}

\section*{Matemática — Módulo 5 — Treino}

\begin{enumerate}
\item a) Incorreta. Se substituirmos x por 4 na equação original, obtemos 2(4) + 3 = 8 + 3 = 11, que não é igual a 9.
b) Correta. Se substituirmos x por 3 na equação original, obtemos 2(3) + 3 = 6 + 3 = 9, que é igual a 9. Portanto, x = 3 é a solução correta da equação.
c) Incorreta. Se substituirmos x por 2 na equação original, obtemos 2(2) + 3 = 4 + 3 = 7, que não é igual a 9.
d) Incorreta. Se substituirmos x por 6 na equação original, obtemos 2(6) + 3 = 12 + 3 = 15, que não é igual a 9.

\item a) Correta. \(3n + 41 = 143\). 3n = 143 - 41. 3n = 102. n = \ 34
b) Incorreta. O triplo de 27 somado a 41 é igual a 122, que é diferente de 143.
c) Incorreta. O triplo de 30 somado a 41 é igual a 131, que é diferente de 143.
d) Incorreta. O triplo de 33 somado a 41 é igual a 140, que é diferente de 143.

\item a) Incorreta. Ela teria retornado com R\$39,00 se tivesse comprado 5 cartelas.
b) Incorreta. Ela teria retornado com R\$36,00 se tivesse comprado 6 cartelas.
c) Incorreta. Ela teria retornado com R\$27,00 se tivesse comprado 9 cartelas.
d) Correta. Para encontrar o valor, devemos subtrair todos os gastos e igualar a 3c, que é o valor de cada cartela. 3c = \ 89 - 22 - 8 - 5 - 33. 3c = \ 21. c = \ 7 cartelas compradas.
\end{enumerate}

\section*{Matemática — Módulo 6 — Treino}

\begin{enumerate}
\item a) Incorreta. Usou-se a porcentagem na forma percentual.
b) Incorreta. Apesar de transformar 3\% para $\frac{3}{100}$, colocou-se o 100 como denominador da parte fixa do salário, transformando ela para 25 ao invés de 2500.
c) Incorreta. Multiplicou-se a comissão pelo fixo ao invés de somar.
d) Correta. 


\item a) Incorreta. Somaram-se os algarismos como se fossem ambos positivos.
b) Incorreta. Somaram-se os algarismos como se fossem ambos negativos.
c) Correta. $\frac{a}{4} + 5b + \frac{2a^{2} - b}{5c} = \frac{8}{4} + 5 \times \left( - 3 \right) + \frac{2\left( 8 \right)^{2} - \left( - 3 \right)}{5 \times 2} = = 2 - 15 + \frac{2 \times 64 + 3}{10} = \  - 13 + \frac{131}{10} = - 13 + 13,1 = 0,1$.
d) Incorreta. Considerou-se o sinal do 13 como sendo o correto.


\item a) Incorreta.
$\left( n + 2 \right) + \left( n + 3 \right) = \left( 1 + 2 \right) + \left( 1 + 3 \right) = 3 + 4 = 7$.
b) Correta.  $\left( n + 2 \right)\left( n + 3 \right) = \left( 1 + 2 \right)\left( 1 + 3 \right) = 3 \times 4 = 12. \therefore\ \left( n + 2 \right)\left( n + 3 \right)$ é equivalente} a $n^{2} + 5n + 6$.
c) Incorreta. 
$\left( n - 2 \right) + \left( n - 3 \right) = \left( 1 - 2 \right) + \left( 1 - 3 \right) = - 1 - 2 = - 3$.
d) Incorreta.
$\left( n - 2 \right)\left( n - 3 \right) = \left( 1 - 2 \right)\left( 1 - 3 \right) = - 1\  \times \ ( - 2) = 2$.
\end{enumerate}

\section*{Matemática — Módulo 7 — Treino}

\begin{enumerate}
\item a) Incorreta. Foram consideradas as grandezas como diretamente proporcionais.
b) Incorreta. Ao invés de multiplicar as grandezas 80 e 40, fez-se a divisão entre elas.
c) Incorreta. Apesar de encontrar a constante correta, considerou as grandezas como diretamente proporcionais.
d) Correta. Ao se reduzir a velocidade, o tempo para percorrer o trajeto aumentou, ou seja, são grandezas inversamente proporcionais.
Assim: 80. 40 = 50 . 64 = k = 3.200.


\item a) Incorreta. Colocou o total de pintores para concluir o serviço e não quantos tinham que ser contratados.
b) Incorreta. Cconsiderou-se que seriam necessários 4 pintores no total ao invés de 5.
c) Correta. Inicialmente: 1 quarto -> 30 + 22 = 52m^{2} 3 quartos
-> $3 \times 52 = 156m^{2}$. No final: 1 quarto -> 44 + 68 = 112m^{2} 3 quartos -> $2 \times 52 + 112 = 216m^{2}$. Se o serviço aumentou, para manter o prazo, é preciso aumentar o número de pintores, ou seja, relação diretamente proporcional. Assim: $\frac{156}{3} = \frac{216}{x}$  -> \ \ 156x = 648 -> $x \cong 4,1$. Logo, são necessários 4,1 pintores para realizar a nova pintura, mas, como não é possível contratar 1,1 pessoas, é necessário contratar 2
pintores para manter o prazo inicial.
d) Incorreta. A partir do resultado 1,1, considerou-se que bastava contratar 1 pintor.

\item a) Incorreta. O total dos lucros foi subtraído do total investido pelo investidor B.
b) Correta. Vamos considerar os seguintes valores recebidos de lucro pelos
investidores: A\  = \ x\; B\  = \ y C\  = \ z. Como a divisão é diretamente proporcional ao valor investido, temos: $\frac{x}{50.000} = \frac{y}{75.000}\  = \frac{z}{115.000} = k$; $\frac{x + y + z}{50.000 + 75.000 + 115.000} = \frac{60.000}{240.000} = 0,25$; $\frac{x}{50.000} = 0,25 \rightarrow x = 12.500$; $\frac{y}{75.000} = 0,25 \rightarrow y = 18.750$; $\frac{z}{115.000} = 0,25 \rightarrow z = 28.750$. 
c) Incorreta. Dividiu-se o total do lucro e  por 3 sem
considerar a proporcionalidade ao que cada um investiu.
d) Incorreta. Considerou-se apenas o investidor A.
\end{enumerate}

\section*{Matemática — Módulo 8 — Treino}

\begin{enumerate}
\item a) Incorreta. Uma corda não é apenas um segmento de reta que liga dois pontos da circunferência, já que não necessariamente passa pelo centro da circunferência.
b) Incorreta. Essa opção descreve o raio da circunferência, não a corda. O raio liga o centro da circunferência a um ponto específico na
circunferência, enquanto a corda liga dois pontos quaisquer da circunferência.
c) Incorreta. Um arco da circunferência não pode ser considerado uma corda. A corda é um segmento de reta, enquanto o arco é uma parte da circunferência.
d) Correta. A definição correta de uma corda é um segmento de reta que liga o centro da circunferência a um ponto médio de um arco da
circunferência. Isso significa que a corda passa pelo centro da circunferência e divide o arco em duas partes iguais.

\item a) Incorreta. Se um prisma retangular tivesse 8 arestas, teria apenas duas arestas por face, o que não seria suficiente para formar as arestas laterais.
b) Incorreta. Se um prisma retangular tivesse 10 arestas, teria três arestas por face, o que também não seria suficiente para formar as arestas laterais.
c) Incorreta. Se um prisma retangular tivesse 12 arestas, teria quatro arestas por face, o que ainda não seria suficiente para formar as arestas laterais.
d) Correta. Um prisma retangular possui 12 arestas na base (4 arestas do retângulo superior + 4 arestas do retângulo inferior + 4 arestas verticais que conectam as bases). Além disso, existem duas
arestas laterais que se estendem verticalmente e conectam os vértices
das bases, totalizando 14 arestas.

\item a) Incorreta. $5 - 10 + 6 = 1\  \neq \ 2$, logo, não corresponde à relação de Euler.
b) Correta. $V - 10 + 6 = 2 \rightarrow \ 6 - 10 + 6 = 2$ , logo, tem 6 vértices.
c) Incorreta. $8 - 10 + 6 = 4\  \neq \ 2, logo$, não corresponde à relação de Euler.
d) Incorreta. $4 - 10 + 6 = 0\  \neq \ 2$, logo, não corresponde à relação de Euler.
\end{enumerate}

\section*{Matemática — Módulo 9 — Treino}

\begin{enumerate}
\item a) Correta. a^2 + b^2 = c^2 representa corretamente o Teorema de Pitágoras, que estabelece que, em um triângulo retângulo, o quadrado da
hipotenusa (c) é igual à soma dos quadrados dos catetos (a e b).
b) Incorreta. Na verdade, a soma dos quadrados dos catetos a e b é igual ao quadrado da hipotenusa c, como afirma o Teorema de Pitágoras.
c) Incorreta. Em um triângulo retângulo, os lados a e b são os catetos, e eles podem ter medidas diferentes.
d) Incorreta. Não há uma relação específica entre as medidas dos lados a, b e c de um triângulo retângulo. As medidas podem variar dependendo do triângulo em questão.

\item a) Correta. Em polígonos semelhantes, os ângulos internos
correspondentes têm medidas iguais. Isso ocorre porque a semelhança entre os polígonos preserva a congruência dos ângulos internos.
b) Incorreta. A proporção é uma relação entre as medidas dos lados dos polígonos semelhantes, não entre os ângulos internos.
c) Incorreta. Ângulos suplementares são aqueles que somam 180 graus, mas não há uma relação de suplementaridade entre os ângulos internos de polígonos semelhantes.
d) Incorreta. Ângulos complementares são aqueles que somam 90 graus, mas não há uma relação de complementaridade entre os ângulos internos de polígonos semelhantes.

\item a) Incorreta. Embora essa afirmação seja verdadeira, ela se refere aos ângulos formados pelas retas paralelas e uma transversal, não aos ângulos internos de polígonos.
b) Incorreta. Ângulos adjacentes são aqueles que têm um lado em comum, mas não necessariamente são suplementares. Essa relação é verdadeira apenas para ângulos lineares ou ângulos opostos pelo vértice,
não para todos os ângulos formados por retas paralelas cortadas por uma transversal.
c) Correta. Ângulos correspondentes são pares de ângulos que estão em lados opostos da transversal e em posições correspondentes em relação às retas paralelas. Esses ângulos têm medidas iguais.
d) Incorreta. Ângulos consecutivos são ângulos que possuem o mesmo vértice e um lado em comum, mas não necessariamente são complementares.
Essa relação é verdadeira apenas para ângulos suplementares, não para todos os ângulos formados por retas paralelas cortadas por uma transversal.
\end{enumerate}

\section*{Matemática — Módulo 10 — Treino}

\begin{enumerate}
\item a) Correta. Ao caminhar 500 metros para o norte, Júlia estará acima da posição inicial de Carlos. Isso indica que Júlia está a nordeste de Carlos. Carlos caminha 300 metros para o leste, o que o
coloca à direita da posição inicial de Júlia. Maria caminha 200 metros
para o sul e depois 400 metros para o oeste. Isso a coloca abaixo da
posição inicial de Júlia e à esquerda da posição inicial de Carlos.
Portanto, Maria está a oeste de Alice. Assim, a posição final de cada
pessoa pode ser descrita da seguinte forma: Júlia está a nordeste de
Carlos e a oeste de Maria.
b) Incorreta. Não foram seguidas corretamente as direções e posições finais das pessoas na trilha.
c) Incorreta. Não foram seguidas corretamente as direções e posições finais das pessoas na trilha.
d) Incorreta. Não se seguiram corretamente as direções e posições finais das pessoas na trilha.

\item a) Incorreta, pois não foram seguidas corretamente as condições
especificadas para a localização da janela na planta da sala.
b) Correta. De acordo com a primeira condição, a janela deve
estar posicionada na parede mais próxima à entrada da sala. No retângulo
que representa a planta da sala, a parede mais próxima à entrada é a
parede de 6 metros. De acordo com a segunda condição, a janela deve ser
colocada a uma distância igual a 4 metros da parede oposta à entrada.
Como a sala possui uma largura de 6 metros, a parede oposta à entrada é
a parede de 10 metros. Portanto, a janela deve estar localizada a 4
metros dessa parede. Assim, a posição correta da janela na planta da
sala é na parede de 6 metros, a 4 metros da entrada, conforme descrito
na alternativa B.
c) Incorreta. Não foram seguidas corretamente as condições
especificadas para a localização da janela na planta da sala.
d) Incorreta. Não foram seguidas corretamente as condições
especificadas para a localização da janela na planta da sala.

\item a) Incorreta. As direções não foram seguidas corretamente.
b) Incorreta. As direções não foram seguidas corretamente.
c) Incorreta. As direções não foram seguidas corretamente.
d) Correta. Ao caminhar 500 metros para o norte a partir de A,
João estará acima do ponto B. Isso indica que João está ao norte de B.
Em seguida, João vira à esquerda e caminha 300 metros para o leste. Isso
o coloca à direita da posição inicial de B. Por fim, João segue mais 200
metros para o sul. Isso o coloca abaixo da posição inicial de B.
Portanto, João está ao sul de B.
\end{enumerate}

\section*{Matemática — Módulo 11 — Treino}

\begin{enumerate}
\item a) Incorreta. I não está correta. Média é a soma dos valores
do conjunto de dados dividida pela quantidade dos dados.
b) Incorreta. Somente I está incorreta.
c) Correta. I é falsa. Moda é o valor que mais se repete.
d) Incorreta. A definição de moda está incorreta na I.

\item a) Incorreta. Os valores informados não são próximos de R\$1.500,00.
b) Correta. Média = (1400 + 1359 + 1260 + 1300 + 1500 + 1400)/6 = 1.369.
c) Incorreta. O cálculo não apresenta esse resultado.
d) Incorreta. Os valores informados não chegam a essa média.

\item a) Correta. Somente profissão não pode ser quantificada.
b) Incorreta. Batimentos cardíacos constituem uma variável
quantitativa.
c) Incorreta. Profissão não é uma variável quantitativa e 
batimentos cardíacos não são qualitativos.
d) Incorreta. Profissão não é uma varáivel quantitativa.
\end{enumerate}

\section*{Matemática — Módulo 12 — Treino}

\begin{enumerate}
\item a) Incorreta. Transformou-se errado cm^2 para m^2.
b) Incorreta. Fez-se a divisão por 16 m^2 ao invés de 15 m^2.
c) Incorreta. Considerou-se que 26,4 seria arredondado para 26
caixas.
d) Correta. Área do apartamento = $$1800 \times 2200 = 3.960.000 cm^2 = 396 m^2$$. Caixas com 15m^2 = $$\frac{396}{15} = 26,4$$ caixas. Como não é possível comprar 0,4 caixa, Fernanda deve comprar 27 caixas.

\item a) Correta. 1 dia -> 25 + 15 + 5 = 45 minutos; 5 dias -> 45 . 5 = 225 minutos; $\frac{225}{60}$ = 3,75 horas} -> 0,75 horas = 0,75 . 60 = 45 minutos. Portanto, o tempo de treino por semana é de 3 horas e 45 minutos.
b) Incorreta. Calculou que 0,75 . 60 = 35.
c) Incorreta. Calculou que $\frac{225}{60}$ = 4,75.
d) Incorreta. Calculou que $\frac{225}{60}$ = 4,75 e 0,75 . 60 = 35.

\item a) Incorreta. Calculou o volume da piscina utilizando 2 ao invés
de 2,5 e ainda aproximou a divisão de maneira errada.
b) Incorreta. Calculou o volume da piscina utilizando 2 ao invés
de 2,5.
c) Correta. Volume da piscina = $12 \times 7 \times 2,5 = 210\ m^{3}$; Capacidade da piscina = $210 \times 1000 = 210.000$ litros; Quantidade de caminhões =
$\frac{210.000}{5.000}$ = 42 caminhões.
d) Incorreta. Fez a multiplicação do volume errado, encontrando 215 ao invés de 210.
\end{enumerate}

\section*{Matemática — Módulo 13 — Treino}

\begin{enumerate}
\item a) Incorreta. Usou as três jogadas no numerador;  porém, o correto
seria a possibilidade de cair coroa.
b) Correta. São de eventos independentes, ou seja, eles não
dependem do evento anterior para ocorrer. $P(A).\ P(B).\ P(C) = \ \frac{1}{2} \times \frac{1}{2} \times \frac{1}{2} = \ \frac{1}{8}$.
c) Incorreta. Considerou a quantidade vezes de jogadas como o
numerador e desconsiderou que se trata de eventos independentes.
d) Incorreta. Multiplicou o número de jogadas pelo número de
possibilidades em uma moeda, desconsiderando que são eventos
independentes.

\item a) Incorreta. Não foram considerados eventos dependentes.
b) Incorreta. Não foram considerados eventos dependentes para 5
jogadas.
c) Correta. São eventos dependentes. O primeiro evento é retirar a
bola 18, que é dada por: $P(A) = \ \frac{1}{75}$ . Retirando a
primeira bola e não devolvendo, restam 74 bolas, assim a probabilidade
de retirar a bola 4 de primeira, no segundo evento é
$P(A|B) = \ \frac{1}{74}$. Assim, a probabilidade do evento ocorrer é:
$P(A).P(A|B) = \ \frac{1}{5550}$.
d) Incorreta. Retirou uma das bolas do espaço amostral, mas não
considerou o evento ocorrido anteriormente.

\item a) Incorreta. Para que o arremessador acertasse todas as bolas, a
porcentagem de erro seria igual a 0.
b) Incorreta. 21\% é a aproximadamente a chance de erro para um
arremesso.
c) Correta. São eventos independentes. Devemos multiplicar as
probabilidades. P(1) P(2) P(3) são as chances de acerto, então, basta
descobrirmos as chances de acertos dado por $(1 - 0,18 = 0,82.). P(1).P(2).P(3)$ = 0,551 aproximadamente 55\%.
d) Incorreta. Para que os arremessos fossem perto de 1\%, a
probabilidade de erro do arremessador seria muito perto de 0,98, ou
melhor, 98\%.
\item 
\end{enumerate}


\section*{Matemática — Simulado 1}

\begin{enumerate}
\item a) Incorreta. Fábio fez em um tempo menor que Tiago e veio depois na listagem.
b) Incorreta. Isabel fez em um tempo menor que Fábio e veio depois na listagem.
c) Correta. Os números que possuem as partes inteiras iguais foram analisados pelas casas decimais.
d) Incorreta. Apesar de Júlio e Isabel estarem nas posições corretas, o Fábio fez um tempo menor que Tiago, logo, ele não seria o último.

\item a) Incorreta. O que se fez foi
$17 + ( - 2) \times \left( - 28 \right) = 15 \times ( - 28) = - 420$.
b) Incorreta. O que se fez foi
$17 + ( - 2) \times \left( - 28 \right) = - 15 \times ( - 28) = 420$. 
c) Incorreta. Errou-se no jogo de sinal:
$( - 2) \times \left( - 28 \right) = \  - 96$.
d) Correta. 
$17 + ( - 2) \times \left\{ - 4 + 2 \times \left\lbrack 9 - \left( 21 \right) \right\rbrack \right\} = 17 + ( - 2) \times \left\{ - 4 - 24 \right\} = 17 + ( - 2) \times \left( - 28 \right) = 73$.


\item a) Correta. $J + 20$ = Tio; $J = 39 - 23 = 16$; $16 + 20 = 36$.
b) Incorreta. Se Juliana tivesse 22 anos, seu tio teria 42 anos, o que não condiz com o enunciado.
c) Incorreta. Se Juliana tivesse 25 anos, seu tio teria 45 anos, o que não condiz com o enunciado.
d) Incorreta. Se Juliana tivesse 35 anos, seu tio teria 55 anos, o que não condiz com o enunciado.


\item a) Incorreta. Considerou-se que 0,0164\cong 16\%
b) Incorreta. Além de se considerar 0,0164\cong 16\%,
interpretou como desconto.
c) Correta. Fazendo-se os cálculos com os fatores de multiplicação,
temos $1,1 \times 0,88 \times 1,05 = 1,0164 - 1 = 0,0164\cong 1,6\%$.
d) Incorreta. Fizeram-se as as contas de forma correta, mas interpretou como desconto ao invés de aumento.

\item a) Incorreta. A quantidade de fatias pintadas está correta, porém, o denominador não condiz, uma vez, que é dividido em 4 partes.
b) Incorreta. Embora a fração $\frac{1}{4}$ apareça, a parte inteira tem somente um círculo todo pintado.
c) Correta. Apresenta um círculo todo pintado, que é a parte inteira e $\frac{1}{4}$ de outro.
d) Incorreta. A representação da parte inteira e da parte decimal, corresponde a 1 inteiro, logo 2 inteiros, e na imagem somente um círculo está todo pintado.


\item a) Incorreta. O primeiro termo seria
$a_{1} = 2 \times 1 + 2 = 2 + 2 = 4$, o que não confere.
b) Correta. Temos $a_{1} = 1^{2} + 1 = 1 + 1 = 2$,
$a_{2} = 2^{2} + 2 = 4 + 2 = 6, a_{3} = 3^{2} + 3 = 9 + 3 = 12 e a_{4} = 4^2 + 4 = 16 + 4 = 20$.
c) Incorreta. O primeiro termo seria $a_{1} = 4 \times 1 = 4$, o que não confere.
d) Incorreta. O primeiro termo seria
$a_{1} = 1\left( 1 - 1 \right) = 1 \times 0 = 0$, o que não confere.


\item a) Incorreta. Calculou-se a probabilidade com 3 cartas, desconsiderando os eventos e a independência entre eles.
b) Incorreta. Não se considerou que foram retiradas 2 cartas de uma vez.
c) Incorreta. Calculou-se apenas a probabilidade de o primeiro evento acontecer.
d) Correta. São eventos dependentes. A primeira jogada influencia na segunda jogada: $P\left( A \right) = \frac{1}{52};\ P\left( A \right) = \frac{1}{50}; P\left( A \right) \times P\left( A \right) = \frac{1}{52} \times \frac{1}{50} = \frac{1}{2600}$.


\item a) Incorreta. Considerou-se que, mesmo mudando a velocidade, o tempo permaneceria o mesmo.
b) Correta. Velocidade e tempo são grandezas inversamente proporcionais, logo, o produto entre elas é constante. Assim,
$5 \times 120 = 100x \rightarrow 100x = 600 \rightarrow x = \frac{600}{100} = 6$ horas.
c) Incorreta. Foram consideradas as grandezas como diretamente proporcionais e ainda arredondou 4,16 horas para 4 horas e 9 minutos.
d) Incorreta. Foram consideradas as grandezas como diretamente proporcionais.


\item a) Incorreta. O histograma é feito por linhas e barras.
b) Incorreta. O gráfico de barras é formado por barras
retangulares e com base maior na horizontal.
c) Correta. Esse tipo de gráico apresenta setores de uma figura geométrica, geralmente, um círculo.
d) Incorreta. O gráfico de linhas é representado por pontos unidos por linhas.


\item a) Incorreta. Os valores dos vértices não condizem com o processo de reflexão.
b) Correta. Ao se realizar uma reflexão em relação ao eixo x, os pontos mantêm a mesma coordenada x, mas têm sua coordenada y negativa.
No triângulo ABC original, o ponto A(2, 4) terá a mesma coordenada x, mas sua coordenada y será negativa, resultando em A'(-2, -4). Da mesma
forma, os pontos B(5, 6) e C(7, 2) terão suas coordenadas y negativas
após a reflexão, resultando em B'(5, -6) e C'(7, -2), respectivamente.
c) Incorreta. Os valores dos vértices não condizem com o processo de reflexão.
d) Incorreta. Os valores dos vértices não condizem com o processo de reflexão.
\end{enumerate}

\section*{Matemática — Simulado 2}

\begin{enumerate}
\item a) Correta. O comprimento do lado BC é igual ao comprimento do lado AC em um triângulo retângulo com ângulos de 45 graus.
b) Incorreta. Essa resposta seria correta se o triângulo fosse um triângulo isósceles retângulo de 45-45-90 graus, mas no problema não foi mencionado que os ângulos são iguais.
c) Incorreta. Essa resposta seria correta se o triângulo fosse um triângulo equilátero de 60 graus, mas no problema não foi mencionado que os ângulos são iguais.
d) Incorreta. Essa resposta é o dobro do comprimento do lado AC, o que não é possível em um triângulo retângulo com ângulos de 45 graus.


\item a) Incorreta. Considerou-se a distância das duas imagens como o total do deslocamento.
b) Correta. Pegando o vértice A como referência, podemos observar que ele foi deslocado em três espaços para a direita, gerando o vértice
A'. Como cada linha da malha é 1 cm, o total do deslocamento foi de 3 cm para a direita.
c) Incorreta. Além de se considerar a distância das duas imagens como o total do deslocamento, confundiu-se a direção.
d) Incorreta. Fez-se o deslocamento correto, mas confundiu-se a direção.

\item a) Correta.
Média yamaha =  $\frac{12 + 8 + 13}{3} = 11$.
b) Incorreta.
Média Suzuki  =  $\frac{7 + 4 + 10}{3} = 7$.
c) Incorreta.
Média BMW =  $\frac{2 + 6 + 14}{3} = 7,3$.
d) Incorreta.
Média Honda = $\frac{8 + 3 + 7}{3} = 6$.


\item a) Incorreta. Considerou-se que a relação m³ é direta ao litro.
b) Incorreta. Converteu-se m³ para litro multiplicando por 10.
c) Incorreta. Converteu-se m³ para litro multiplicando por 100.
d) Correta. O volume do cubo cprresponde ao tamanho da aresta elevado à terceira potência. Logo, $V = 9^{3} = 729m³$ . Como
$1m^{3} = \ 1000\ L \rightarrow \ 729\ m³\  = \ 729.000 litros$.


\item a) Incorreta. Ao se calcular $5c = 75$, passou o 5 subtraindo ao invés de dividindo.
b) Incorreta. Substituiu-se o valor de d no lugar de c.
c) Incorreta. Ao se calcular $5c = 75$, passou o 5 somando ao invés de dividindo.
d) Correta. Para encontrar o valor de s, basta substituir o valor de t. Assim, $5c = d\  \rightarrow \ 5c = 75\  \rightarrow \ c = \frac{75}{5} = 15$
\end{enumerate}

\section*{Matemática — Simulado 3}

\begin{enumerate}
\item a) Correta. Usand0-se o princípio multiplicativo, as combinações
distintas são $5 \times 4 \times 3 \times 2 \times 3 = 360$. Como Marcela precisa de 300 combinações distintas, ainda vão sobrar 60 opções de combinações.
b) Incorreta. Não se percebeu que o número mínimo de combinações foi atingido.
c) Incorreta. Não se consideraram todas as peças de roupa.
d) Incorreta. Somaram-se as opções de escolha ao invés de multiplicar.


\item a) Incorreta. Considerou-se uma desvalorização de 20 reais.
b) Correta. Como houve uma desvalorização de 20\%, vamos utilizar
o fator multiplicativo $100\% - 20\% = 80\% = 0,8$, assim $0,8 \times 1800 = 1440$.
c) Incorreta. Considerou-se o valor da desvalorização e não o valor final.
d) Incorreta. Ao se fazer $1800 - 360$, encontrou 1400 ao invés de 1440.


\item a) Incorreta. Considerou-se que, para duas sequências serem equivalentes, basta ter uma parte em comum.
b) Incorreta. Considerou-se que, ao trocar o n com n^2 de lugar, uma sequência semelhante seria gerada.
c) Correta. Aplicando a propriedade distributiva, temos:
$n^{2}\left( n + 1 \right) = n^{3} + n^{2}$.
d) Incorreta. Ao se fazer a distributiva, apenas multiplicou o n pelo n^2.

\item a) Incorreta. Considerou-se que basta igualar 1 cm no desenho ao mundo real.
b) Incorreta. Ao se simplificar $\frac{10}{1000000}$, foi colocado um 0 a mais no numerador.
c) Incorreta. Ao se simplificar $\frac{10}{1000000}$, foi colocado um 0 a mais no denominador.
d) Correta. Como escala é a razão
$\frac{\text{desen}ho}{\text{real}}$ e ambos possuem a mesma unidade
de medida, temos $\frac{\text{desen}ho}{\text{real}} = \frac{10\text{cm}}{10\text{km}} = \frac{10}{1000000} = \frac{1}{100000}$


\item a) Incorreta. Considerou-se que o círculo é uma figura rígida por não possuir pontas.
b) Correta. Como estudado, a forma geométrica que possui maior rigidez é o triângulo. Assim, a mesa que apresenta a maior rigidez na base é a de base triangular.
c) Incorreta. Considerou-se a estrutura que costuma ver com maior frequência no dia a dia.
d) Incorreta. Considerou-se o hexágono com maior rigidez que o triângulo.

\item a) Incorreta. O valor encontrado a partir do cálculo da área não é esse.
b) Incorreta. O valor encontrado a partir do cálculo da área não é esse.
c) Incorreta. O valor encontrado a partir do cálculo da área não é
esse.
d) Correta. A área da sala de estar é calculada multiplicando o comprimento pela largura:
${6 \; metros} \times {4\; metros} = 24\;m^2$ Como o quarto principal
tem o dobro da área da sala de estar, a área do quarto principal será 48 metros quadrados.

\item a) Incorreta. Considerou-se que o volume de um bloco retangular é a soma das três medidas.
b) Incorreta. Além de se somar as três medidas, a unidade está errada.
c) Correta. Como o volume de um bloco retangular é o produto das medidas dos seus lados, temos que
$V = 12 \times 7 \times 5 = 420\text{cm}³$.
d) Incorreta. Apesar de o valor do volume ter sido calculado corretamente, a unidade de medida está errada.

\item a) Incorreta. O gráfico não faz essa relação.
b) Correta. O gráfica não apresenta faixas etárias.
c) Incorreta. O gráfico não apresenta tais fatores.
d) Correta, pois a tabela de dupla entrada apresenta a relação entre duas variáveis: o nível de escolaridade e a renda média mensal. A finalidade dessa pesquisa é analisar e inferir como o nível de escolaridade influencia a renda dos indivíduos. Ao cruzar os dados da
tabela, é possível observar se há uma relação entre a escolaridade e a renda e, assim, avaliar a influência dessa variável na determinação da renda média mensal.

\item a) Incorreta. Foi feito o cálculo de probabilidade normal, sem considerar os eventos e considerando que o dado havia sido jogado uma
única vez.
b) Incorreta. Foi feito o cálculo de uma probabilidade normal sem considerar os eventos.
c) Incorreta. Foi calculada a probabilidade de acertar a face número 4 uma vez.
d) Correta. São eventos independentes. Desse modo:
$P\left( A \right) \times P\left( B \right) = \frac{1}{12} \times \frac{1}{12}\  = \frac{1}{144}$.

\item a) Incorreta. O cupcake não estava dividido em oito pedaços.
b) Incorreta. Leonardo consumiu mais de um cupcake.
c) Correta. Foi consumido 1 cupcake inteiro e metade do de Juliana.
d) Incorreta. Leonardo não consumiu 5 cupcakes inteiros e 1 parte entre 4 de outro.
\end{enumerate}

\section*{Matemática — Simulado 4}

\begin{enumerate}
\item a) Incorreta, pois, se a idade de Alice fosse 3, a idade seria igual a 18.
b) Correta.
$3x + 19 = 40 \Rightarrow \ 3x = 40 - 19 - \Rightarrow 3x = 21 \Rightarrow x = 7$.
c) Incorreta. Se a idade de Alice fosse 20, a soma teria que ser 79.
d) Incorreta. Se a idade de Alice fosse 40, a soma teria que ser 139.


\item a) Incorreta. O ângulo não equivale à metade de 30º.
b) Incorreta. Esse valor não condiz ao que foi pedido.
c) Correta. Quando o ponto está no centro da circunferência, ele apresenta o mesmo tamanho do arco.
d) Incorreta. Não equivale ao dobro do ângulo.


\item a) Correta. A reflexão em relação ao eixo y faz com que o ponto
A(2,3) se transforme no ponto A'(-2,3). Em seguida, a translação de 4 unidades para a esquerda e 2 unidades para baixo transforma o ponto A'(-2,3) no ponto B(-6,1).
b) Incorreta. Não se leva em consideração a reflexão em relação ao eixo y.
c) Incorreta. Não se leva em consideração a reflexão em relação ao eixo y.
d) Incorreta. Não se leva em consideração a translação de 4 unidades para a esquerda e 2 unidades para baixo.


\item a) Incorreta. É um gráfico de equação de segundo grau.
b) Incorreta. É um gráfico de logarítmica.
c) Incorreta. É um gráfico modular.
d) Correta. Trata-se de uma reta.


\item a) Incorreta. Ao invés de multiplicar o valor de t por 2, fez a soma e subtraiu o 15 ao invés de somar.
b) Incorreta. Ao invés de multiplicar o valor de t por 2 fez a soma.
c) Correta. Para encontrar o valor de s, basta substituir o valor de t dado. Assim, $s - 15 = 2t\  \rightarrow \ s - 15 = 2\  \times \ 8\  \rightarrow \ s - 15 = 16\  \rightarrow \ \ s = 16 + 15 = 31$.
d) Incorreta. Ao invés de fazer $16 + 15$, fez $16-15$.
\end{enumerate}

\section*{Matemática — Simulado 5} 

\begin{enumerate}
\item a) Correta. Nesses dias, o lucro foi de 407, 500 e 700, respectivamente.
b) Incorreta. Não se levaram em conta todos os dias em que o lucro foi maior.
c) Incorreta. Não se levaram em conta todos os dias em que o lucro foi maior.
d) Incorreta. Não se levaram em conta todo os dias em que o lucro foi maior.


\item a) Incorreta. Foi feito o cálculo da probabilidade de uma única moeda.
b) Incorreta. Não se considerou cada lance como um evento independente.
b) Incorreta. O espaço amostral não pode ser 8.
d) Correta. São eventos independentes, logo
$P\left( A \right) \times P\left( B \right) = \frac{1}{2}\  \times \frac{1}{2} = \frac{1}{4}$.


\item a) Incorreta. Essa fração corresponde a comer somente 1 pedaço da pizza.
b) Incorreta. Essa fração corresponde a 4 pizzas inteiras e um pedaço em uma pizza de corte diferente.
c) Correta. É uma fração imprópria com a parte inteira 8 e mais os dois pedaços.
d) Incorreta. Essa fração corresponde a 7 pizzas inteiras e metade de outra.


\item a) Incorreta. Se o número fosse 12, a soma seria 106.
b) Correta. $3x + 70 = 103 \rightarrow \ 3x = 103 - 70 \rightarrow \ 3x = 33 \rightarrow x = 11$.
c) Incorreta. Se o número fosse 20, a soma teria que ser 130.
d) Incorreta. Se o número fosse 103, a soma teria que ser 379.


\item a) Incorreta. Essa resposta não corresponde ao cálculo correto do arco.
b) Correta. O arco correspondente ao ângulo central de 45 graus possui um comprimento de aproximadamente 6.28 cm.
c) Incorreta. Essa resposta corresponde ao perímetro da
circunferência, não ao comprimento do arco de um ângulo central específico.
d) Incorreta. Essa resposta corresponde ao dobro do perímetro da circunferência, não ao comprimento do arco de um ângulo central específico.


\item a) Incorreta. Considerou-se que o tempo de encontro seria pelo M.D.C ao invés do M.M.C. . 
b) Incorreta. Apenas foi feita a soma do
tempo de cada corredor. 
c) Incorreta. Foi associado um número
errado de minutos. 
d) Correta. Devemos usar a ideia de mínimo múltiplo comum. Assim, calculando o M.M.C dos tempos, temos: 15, 25, 3; 5, 25, 5; 1, 5, 5; 1, 1, $(3x5x5 = 75)$. Logo, os atletas se encontram depois de 75 minutos, ou seja, 1 hora e 15 minutos.


\item a) Incorreta. Considerou-se que o numerador da fração centesimal é a forma percentual sem a vírgula.
b) Correta.
$25,3\% = \frac{25,3}{100} = \frac{253}{1000} = 0,253$.
c) Incorreta. Esqueceu-se de adicionar um 0 no denominador da representação fracionária ao andar com a vírgula.
d) Incorreta. Esqueceu-se de levar em conta as 3 casas decimais no denominador.


\item a) Incorreta. Ao invés de fazer $\frac{12}{1,5}$ calculou $12 x
1,5$. 
b) Incorreta. Ao invés de fazer $18,5 - 6,5$ fez uma soma. 
c) Incorreta. Considerou-se apenas o preço pago por km sem o valor da bandeira. 
d) Correta. Como a corrida tem um preço por km mais um
fixo, podemos escrever de maneira geral que a corrida custa
$c = 1,5q + 6,5. Assim, 18,5 = 1,5q + 6,5 \rightarrow 1,5q = 12 \rightarrow \ q = \frac{12}{1,5} \rightarrow q = 8\text{km}$.


\item a) Correta. Quanto mais pessoas para atender, mais tempo será gasto, logo são G.D.P. Para fazer o cálculo, precisamos do tempo em uma
unidade apenas, ou seja, 1 hora e 40 minutos será 100 minutos. Assim, $\frac{100}{12} = \frac{x}{18} \rightarrow 12x = 1800 \rightarrow x = \frac{1800}{12} = 150$ minutos = 2 horas e 30 minutos. 
b) Incorreta. Ao converter 150 minutos para horas, considerou 1 hora ao invés de 2. 
c) Incorreta. Foram consideradas as grandezas como sendo inversamente proporcionais, fazendo 100\ .\ 12\  = \ 18x. 
d) Incorreta. Considerou-se que 1 hora tem 100 minutos.


\item a) Incorreta. 2 cm é a medida da altura relativa ao lado BC.
b) Incorreta, pois 3 cm é a medida da altura relativa ao lado AC.
C - Correta, pois a altura relativa ao lado AB divide o triângulo ABC em
dois triângulos retângulos, onde a altura é a hipotenusa e os catetos
são os segmentos AH e BH. Podemos utilizar o teorema de Pitágoras para
calcular a medida da altura e chegar ao resultado.
D - Incorreta, pois 5 cm é a medida da mediana relativa ao lado AB.
\end{enumerate}

\section*{Matemática — Simulado 6} 

\begin{enumerate}
\item a) Correta. Essa definição está correta.
b) Incorreta. As primeiras definições estão incorretas.
c) Incorreta. A definição de planta está incorreta.
d) Incorreta, pois há diferenças entre esses conceitos.


\item a) Incorreta. Essa resposta corresponde ao cálculo incorreto da área.
b) Correta. Substituindo os valores fornecidos na fórmula, temos:
Área do triângulo = $\frac {(10 \times 6)}{2} = \frac {60}{2} =
30 \;cm^2$. Portanto, a área do triângulo retângulo é de 30 cm^2.
$A = \frac{\left( b + B \right)h}{2} = \frac{(6 + 18) \times 10}{2} = 24 \times 5 = 120m^2$.
c) Incorreta. Essa resposta não corresponde ao cálculo correto da área do triângulo com os valores fornecidos.
d) Incorreta. Essa resposta corresponde à multiplicação da base pela altura, sem dividir por 2, o que resulta em uma área incorreta.

\item a) Correta. Como não saiu do eixo e só subiu duas unidades da simetria, ocorreu a translação.
b) Incorreta. Para ocorrer a rotação, teríamos uma inclinação.
c) Incorreta. A transformação geométrica reflexão apenas repetiria a figura.
d) Incorreta. A imersão não é uma transformação geométrica.

\item a) Incorreta. Yan é o segundo colocado.
b) Incorreta. Renato é o último colocado.
c) Incorreta. Cristian ficou em primeiro lugar.
d) Correta. Rafael ficou na terceira posição.


\item a) Incorreta. Não se trocou a operação do 22 ao mudar de lado na igualdade.
b) Correta. Para encontrar o valor de x, basta substituir o valor
de y. Assim, $x - y + 20 = 0 \rightarrow x - ( - 2) + 20 = 0 \rightarrow \ x + 22 = 0 \rightarrow \ x = - 22$.
c) Incorreta. Ao substituir o y, a mudança de sinal não foi feita.
d) Incorreta. Além de não fazer a mudança do sinal com o y, a operação do 18 não foi feita.
\end{enumerate}

\section*{Matemática — Simulado 7} 

\begin{enumerate}
\item a) Incorreta. Ao determinar o valores de x, y, z, foi feita uma multiplicação por 2 ao invés de uma divisão.
b) Correta. $\frac{x}{42} = \frac{y}{16} = \frac{z}{18} = \frac{x + y + z}{42 + 16 + 18} = \frac{38}{76} = \frac{1}{2}. \frac{x}{42} = \frac{1}{2} \rightarrow \ \ x = 21. \frac{y}{16} = \frac{1}{2} \rightarrow \ \ x = 8\ \ \ \ \ \ \ \ \ \ \ \ \frac{z}{18} = \frac{1}{2} \rightarrow \ \ x = 9$.
c) Incorreta. Além de calcular o valor das incógnitas usando a multiplicação, os valores que seriam y e z foram invertidos.
d) Incorreta. Apesar de fazer os cálculos corretos, a ordem dos valores de y e z foi invertida.

\item a) Incorreta. Apesar de a escola ter 9 turmas, uma delas é excluída da votação.
b) Incorreta. O valor de amostra e população estão invertidos.
c) Correta. Temos 8 turmas como população e 3 turmas de amostra.
d) Incorreta. O valor da amostra e população estão invertidos.


\item a)  Correta.  \frac{4}{8} é equivalente a \frac{1}{2}.
b)  Incorreta. A fração não corresponde ao galão inteiro, uma vez que encheu 4 partes de 8 divididas.
c) Incorreta. A fração irredutível encontrada está errada.
d) Incorreta. Essa fração não representa a parte cheia do galão.


\item a) Incorreta. Se o número fosse 355, o resultado final seria diferente.
b) Correta.
$x + 1200 = 4x \rightarrow \ 4x - x = 1200 \rightarrow 3x = 1200 \rightarrow x = 400$.
c) Incorreta. Se o número fosse 200, o resultado final seria diferente.
d) Incorreta. Se o número fosse 420, o resultado final seria diferente.

\item a) Incorreta. O valor do ângulo central não foi considerado corretamente.
b) Incorreta. O valor do ângulo central não foi considerado corretamente.
c) Correta. Arco = $\frac {Ângulo}{360º} \times 2π \times$ raio.
No caso do problema, substituindo os valores na fórmula, temos 3π cm.
d) Incorreta. O valor do ângulo central não foi considerado corretamente.

\item a) Incorreta. A rotação acontece quando o objeto inclina-se.
b) Correta. O ato de posicionar um espelho reflete a imagem.
c) Incorreta. A translação ocorre quando o objeto segue pelas direções norte, sul, leste e oeste.
d) Incorreta. No inchaço há deformação na imagem.


\item a) Incorreta. A divisão de 19 por 3 é maior que 6.
b) Incorreta. A divisão de 19 por 3 é menor que 7.
c) Incorreta. A divisão de 19 por 3 é maior que 1.
d) Correta. Analisando a fração como divisão, temos que o 3 está dentro do 9 aproximadamente 6 vezes, logo, a fração é maior do que 6 e menor do que 7.

\item a) Incorreta. Esqueceu-se de se considerar que o 3 também dividiu os valores simultaneamente.
b) Incorreta. Ao invés de multiplicar o 3 e o 5, fez a soma deles.
c) Correta. Para dividir as lembranças na maior quantidade possível, usamos o cálculo do M.D.C; assim, fatorando a quantidade de cada doce, temos 75, 60, 45, 2; 75, 30, 45, 2; 75, 15, 45, 3; 25, 5, 15, 3; 25, 5, 5, 5; 5, 1, 1, 5; 1, 1, 1. Logo, em cada saquinho terá 15 doces.
d) Incorreta. Ao fatorar, considerou que usaria todos os fatores como se fosse M.M.C.

\item a) Correta. Na loja virtual $\rightarrow 3500 \times 0,95 = 3.325 + 35 = 3360. \text{Na}\ \text{loja}\ fí\text{sica} \rightarrow 3800 \times 0,9 = 3420$. Portanto, a melhor opção de Larissa é a loja virtual.
b) Incorreta. Não se considerou o valor do frete da loja virtual.
c) Incorreta. Confundiram-se as informações da loja física com a virtual.
d) Incorreta. Ao realizar as operações, o resultado obtido é outro.

\item a) Incorreta. Considerou-se que a sequência 2 não apresenta uma relação entre os termos e que os números primos possuem algum padrão.
b) Incorreta. Considerou-se que a sequência dos números primos pode ser encontrada por alguma relação entre os termos.
c) Incorreta. Não se enxergou que a sequência 2 apresenta uma relação entre os termos por meio do antecessor subtraído de 2.
d) Correta. Podemos observar que a sequência I é determinada pelo termo anterior mais 3, a sequência II pelo termo anterior menos 2, e a
III é a sequência dos números primos. Como a sequência recursiva é aquela em que um termo depende do anterior, temos recursiva, recursiva e não recursiva.
\end{enumerate}

\section*{Matemática — Simulado 8} 

\begin{enumerate}
\item a) Incorreta. Ao determinar o valores de x, y, z, foi feita uma
multiplicação por 2 ao invés de uma divisão.
b) Correta. $\frac{x}{42} = \frac{y}{16} = \frac{z}{18} = \frac{x + y + z}{42 + 16 + 18} = \frac{38}{76} = \frac{1}{2}. \frac{x}{42} = \frac{1}{2} \rightarrow \ \ x = 21. \frac{y}{16} = \frac{1}{2} \rightarrow \ \ x = 8\ \ \ \ \ \ \ \ \ \ \ \ \frac{z}{18} = \frac{1}{2} \rightarrow \ \ x = 9$.
c) Incorreta. Além de calcular o valor das incógnitas usando a
multiplicação, os valores que seriam y e z foram invertidos.
d) Incorreta. Apesar de fazer os cálculos corretos, a ordem dos
valores de y e z foi invertida.

\item a) Incorreta. Não corresponde à medida correta do ângulo agudo.
b) Incorreta. Não corresponde à medida correta do ângulo agudo.
c) Correta. No triângulo retângulo, o ângulo agudo oposto ao
cateto é sempre o complementar do ângulo formado entre a hipotenusa e o
cateto. Esse ângulo pode ser encontrado usando a função trigonométrica
seno.
d) Incorreta. Corresponde ao ângulo reto, não ao ângulo agudo
oposto ao cateto.

\item a) Incorreta. Corresponde à soma apenas dos comprimentos das duas
arestas menores.
b) Incorreta. Corresponde à soma apenas dos comprimentos das duas
arestas maiores.
c) Correta. O perímetro de um retângulo é dado pela soma dos
comprimentos de todos os lados, ou seja, a soma dos comprimentos das
quatro arestas.
No caso do retângulo descrito no problema, temos duas arestas de
comprimento 10 cm e duas arestas de comprimento 5 cm. Portanto, o
perímetro é dado por: $P = 10 cm + 10 cm + 5 cm + 5 cm P = 20 cm + 10 cm P = 30 cm$. 
d) Incorreta. Corresponde à soma de todos os lados do retângulo,
incluindo os comprimentos das arestas duas vezes.

\item a) Incorreta. Corresponde à metade da área correta do triângulo.
b) Correta. A fórmula para calcular a área de um triângulo é dada
pela metade do produto da base pela altura: $A =
\frac {(base \times altura)}{2}$. Substituindo os valores do problema, temos: $A = 24 cm^2$.
c) Incorreta. Não corresponde à área correta do triângulo.
d) Incorreta. Corresponde à área do retângulo formado pela base e
altura do triângulo, não à área do próprio triângulo.

\item a) Correta, pois, para encontrar o valor de b, basta substituir o valor
de a. Assim, $2a + 8 - 3b = 5\  \rightarrow \ 2\  \times \ \left( - 3 \right) + 8 - 3b = 5 \rightarrow \  - 6 + 8 - 3b = 5 - 3b = 3\  \rightarrow \ b = \  - 1$.
b) Incorreta. Ao dividir 3 por -3, a regra de sinal não foi
respeitada.
c) Incorreta. Ao fazer 2 vezes -3, foi
encontrado 6 ao invés de -6.
d) Incorreta. Ao passar o 2 de lado, a regra do sinal foi
ignorada.
\end{enumerate}

\end{comment}