%!TEX root=./LIVRO.tex

\chapter{Respostas}

\footnotesize

\pagecolor{gray!40}

\section*{Matemática — Módulo 1 — Treino}

\begin{itemize}
\item A - Incorreta, pois \sqrt{3}\\ não é Natural, 0,2222 não é Inteiro,
\sqrt{2}\\ não é racional.
B - Correta, pois os números foram classificados corretamente.
C - Incorreta, pois \sqrt{3}\\ não é racional, 0,2222 não é
irracional, \sqrt{2}\\ , 5,363636 não é irracional e \sqrt{2}\\
não é racional.
D - Incorreta, pois \sqrt{3}\\ não é racional.
\item A - correta, pois, ao efetuarmos a divisão, temos 0,25.
B - incorreta, pois, ao efetuarmos a divisão, temos 2,5.
C - incorreta, pois, ao efetuarmos a divisão, temos 1,42.
D - incorreta, pois, ao efetuarmos a divisão, temos 1,5.
\item A - Incorreta, pois todo número decimal finito pode ser representado por
fração e o número 2 é o único par que é primo.
B - Correta, pois essas são as afirmações certas.
C - Incorreta, pois o número 2 é o único número par que é primo, então é
uma verdade
D - Incorreta, pois a afirmação III é falsa. O número 21 não é primo,
ele possui 4 divisores: 1, 3,7 e 21.
\end{itemize}

\section*{Matemática — Módulo 2 — Treino}

\begin{itemize}
\item A - Incorreta, pois não foi considerada a diferença de fuso horário,
apenas a soma das horas de maneira direta.
B - Incorreta, pois o cálculo das horas foi feito da maneira correta,
porém a mudança de dia não foi considerada.
C - Correta, pois: Saída de São Paulo as 8h do dia 23/03 \rightarrow em Dubai vão ser
15h do dia 23/03 (4 - ( - 3) = 4 + 3 = 7\text{h\ }de diferença). A viagem de São Paulo para Dubai dura 14 horas, assim Pedro vai chegar
em Dubai às 5h do dia 24/03, que serão 12h do dia 24/03 na Austrália
(11 - 4 = 7\text{h\ }de diferença). A viagem de Dubai até a Austrália são 18 horas, assim Pedro vai chegar
na Austrália às 6h do dia 25/03.
D - Incorreta, pois o dia está correto, mas foi considerado o cálculo
sem a diferença de fuso horário.
\item A - Incorreta, pois essa resposta não considera o fato de que cada
algarismo deve ser diferente dos demais.
B - Incorreta, pois essa resposta não leva em conta que a quantidade de
opções vai diminuindo a cada algarismo escolhido.
C - Correta, pois a quantidade de números de quatro algarismos
diferentes que podem ser formados utilizando-se cinco algarismos é
determinada pelo princípio multiplicativo. Portanto, o número total de
combinações possíveis é dado pelo produto das opções disponíveis para
cada algarismo: \ 5 \times 4\times 3\times 2  = 120
D - Incorreta, pois essa resposta excede o número máximo de combinações
possíveis, uma vez que estamos limitados a quatro algarismos diferentes.
\item A - Incorreta, pois considerou apenas o tamanho de um lado do terreno e
ignorou o comprimento do portão.
B - Incorreta, pois considerou apenas o tamanho de um lado do terreno.
C - Incorreta, pois não considerou o comprimento do portão do perímetro
do terreno.
D - Correta, pois, como a área de um quadrado é dado por l² - onde l é o
tamanho do lado -, para calcular o tamanho do lado basta extrair a raiz
quadrada da área. Assim
\end{itemize}

\section*{Matemática — Módulo 3 — Treino}

\begin{itemize}
	\item A - Incorreta, pois não subtraiu a parte inteira do denominador.
B - Incorreta, pois, quando há 0 no denominador, temos um antiperíodo na
dízima periódica.
C - Incorreta, pois desconsiderou a parte inteira da fração.
D - Correta, pois, para encontrar o numerador, calculamos 795 - 7 = 788
no numerador e colocamos 99 no denominador, já que há dois algarismos no
período.
\item A - Incorreta, pois Marcelo foi o melhor colocado.
B - Correta, pois, comparando as frações de mesmo denominador, Fabiana
teve o pior desempenho e a fração de Juliano é equivalente ao resultado
de Fabiana.
C - Incorreta, pois Maicon ficou em segundo colocado.
D - Incorreta, pois Marcelo e Maicon foram os primeiros colocados.
\item A - Incorreta, pois essa fração representa a parte de um todo quando uma
parte é considerada. No caso, João comprou mais do que 1/3 do terreno.
B - Correta, pois o terreno foi dividido em três partes iguais. João
comprou 2 dessas partes. Para determinar a fração que expressa a parte
do terreno que João comprou, devemos considerar que ele comprou 2 partes
de um total de 3 partes do terreno.
C - Incorreta, pois essa fração não corresponde à proporção de partes do
terreno que João comprou.
D - Incorreta, pois essa fração também não representa corretamente a
proporção de partes do terreno que João comprou.
\end{itemize}

\section*{Matemática — Módulo 4 — Treino}

\begin{itemize}
\item A - Incorreta, pois essa resposta não leva em consideração o acréscimo
de 15\% no valor original.
B - Incorreta, pois essa resposta também não considera o acréscimo
correto. É importante lembrar que 15\% de R\$200,00 é igual a R\$30,00,
não R\$20,00.
C - Correta, pois o novo valor = Valor original + (Porcentagem de
acréscimo vezes o Valor original). Nesse caso, o valor original é
R\$200,00 e o acréscimo é de 15\%. Novo valor = 200 + \frac {15}{100}\times 200 Novo valor = 200 +
0,15 \times 200 Novo valor = 200 + 30 Novo valor = 230
D - Incorreta, pois representa o valor original acrescido de 22,5\%, não
de 15\%.
\item A - Incorreta, pois a conta foi feita como se o apartamento custasse
R\$27.000.
B - Incorreta, pois somente considerou o valor de uma parcela como
resposta.
C - Correta, pois 60 \times 4500 = 270.000. 15\% \rightarrow \text{fator}\ 0,85 \rightarrow 0,85 \times 270.000 = 229.500. 270.000 - 229.500 = 40.500,00
D - Incorreta, pois a conta 270.000 - 229.500 foi feita
incorretamente.
\item A - Incorreta, pois considerou apenas o aumento de 6\%.
B - Incorreta, pois considerou o fator de multiplicação como sendo 1,04.
C - Incorreta, pois calculou o fator de multiplicação como 100\% + 6\% -
10\%.
D - Correta, pois Aumento de 6\%\  = \ 1,06. Desconto de 10\%\  = \ 0,9. 270 \times 1,06 \times 0,9 = 257,58. No final do semestre, o pneu saiu por aproximadamente R\$258,00.
\end{itemize}

\section*{Matemática — Módulo 5 — Treino}

\begin{itemize}
\item A - Incorreta, pois se substituirmos x por 4 na equação original,
obtemos 2(4) + 3 = 8 + 3 = 11, que não é igual a 9.
B - Correta, pois se substituirmos x por 3 na equação original, obtemos
2(3) + 3 = 6 + 3 = 9, que é igual a 9. Portanto, x = 3 é a solução
correta da equação.
C - Incorreta, pois se substituirmos x por 2 na equação original,
obtemos 2(2) + 3 = 4 + 3 = 7, que não é igual a 9.
D - Incorreta, pois se substituirmos x por 6 na equação original,
obtemos 2(6) + 3 = 12 + 3 = 15, que não é igual a 9.
\item A - Correta, pois \(3n + 41 = 143\). 3n = 143 - 41. 3n = 102. n = \ 34
B - Incorreta, pois o triplo de 27 somado a 41 é igual a 122, que é
diferente de 143.
C - Incorreta, pois o triplo de 30 somado a 41 é igual a 131, que é
diferente de 143.
D - Incorreta, pois o triplo de 33 somado a 41 é igual a 140, que é
diferente de 143.
\item A - Incorreta, pois, ela teria retornado com R\$39,00 se tivesse
comprado 5 cartelas.
B - Incorreta, pois ela teria retornado com R\$36,00 se tivesse comprado
6 cartelas.
C - Incorreta, pois ela teria retornado com R\$27,00 se tivesse comprado
9 cartelas.
D - Correta, pois, para encontrar o valor, devemos subtrair todos os
gastos e igualar a 3c, que é o valor de cada cartela. 3c = \ 89 - 22 - 8 - 5 - 33. 3c = \ 21. c = \ 7 cartelas compradas.
\end{itemize}

\section*{Matemática — Módulo 6 — Treino}

\begin{itemize}
\item A - Incorreta, pois usou a porcentagem na forma percentual.
B - Incorreta, pois apesar de transformar 3\% para \frac{3}{100},
colocou o 100 como denominador da parte fixa do salário, transformando
ela para 25 ao invés de 2500.
C - Incorreta, pois multiplicou a comissão pelo fixo ao invés de somar.
D - Correta, pois \text{Fixo}\  = \ 2500\ \ \ \ \ \ \ \ \ \ \ \ \ \ \ \ \ \ \ \ \ \ \ \text{Vari}á\text{vel}\  = \ 3\%\ \text{de}\ x\text{\ \ \ \ \ \ \ \ \ \ \ \ \ \ \ \ \ }\text{Sal}á\text{rio} = 2500 + 0,03x.
\item A - Incorreta, pois somou os algarismos como se fossem ambos positivos.
B - Incorreta, pois somou os algarismos como se fossem ambos negativos.
C - Correta, pois \frac{a}{4} + 5b + \frac{2a^{2} - b}{5c} = \frac{8}{4} + 5 \times \left( - 3 \right) + \frac{2\left( 8 \right)^{2} - \left( - 3 \right)}{5 \times 2} = = 2 - 15 + \frac{2 \times 64 + 3}{10} = \  - 13 + \frac{131}{10} = - 13 + 13,1 = 0,1.
D - Incorreta, pois considerou o sinal do 13 como sendo o correto.
\item A - Incorreta, pois
\left( n + 2 \right) + \left( n + 3 \right) = \left( 1 + 2 \right) + \left( 1 + 3 \right) = 3 + 4 = 7.
B - Correta, pois \left( n + 2 \right)\left( n + 3 \right) = \left( 1 + 2 \right)\left( 1 + 3 \right) = 3 \times 4 = 12. \therefore\ \left( n + 2 \right)\left( n + 3 \right)\ é\ \text{equivalente}\ a\ n^{2} + 5n + 6.
C - Incorreta, pois
\left( n - 2 \right) + \left( n - 3 \right) = \left( 1 - 2 \right) + \left( 1 - 3 \right) = - 1 - 2 = - 3.
D - Incorreta, pois
\left( n - 2 \right)\left( n - 3 \right) = \left( 1 - 2 \right)\left( 1 - 3 \right) = - 1\  \times \ ( - 2) = 2.
\end{itemize}

\section*{Matemática — Módulo 7 — Treino}

\begin{itemize}
\item A - Incorreta, pois considerou as grandezas como diretamente
proporcionais.
B - Incorreta, pois ao invés de multiplicar as grandezas 80 e 40, fez a
divisão entre elas.
C - Incorreta, pois, apesar de encontrar a constante correta, considerou
as grandezas como diretamente proporcionais.
D - Correta, pois, ao reduzir a velocidade, o tempo para percorrer o
trajeto aumentou, ou seja, são grandezas inversamente proporcionais.
Assim: 80 \times 40 = 50 \times 64 = k = 3.200.
\item A - Incorreta, pois colocou o total de pintores para concluir o serviço
e não quantos tinham que ser contratados.
B - Incorreta, pois considerou que seriam necessários 4 pintores no
total ao invés de 5.
C - Correta, pois inicialmente: 1 quarto \rightarrow 30 + 22 = 52m^{2} 3 quartos
\rightarrow 3 \times 52 = 156m^{2}. No final: 1 quarto \rightarrow 44 + 68 = 112m^{2} 3 quartos \rightarrow 2 \times 52 + 112 = 216m^{2}. Se o serviço aumentou, para manter o prazo, é preciso aumentar o número de pintores, ou seja, relação diretamente proporcional. Assim: \frac{156}{3} = \frac{216}{x}\  \rightarrow \ \ 156x = 648 \rightarrow x \cong 4,1. Logo, são necessários 4,1 pintores para realizar a nova pintura, mas, como não é possível contratar 1,1 pessoas, é necessário contratar 2
pintores para manter o prazo inicial.
D - Incorreta, pois considerou a partir do resultado 1,1, considerou que
bastava contratar 1 pintor.
\item A - Incorreta, pois pegou o total dos lucros e subtraiu do total
investido pelo investidor B.
B - Correta. Vamos considerar os seguintes valores recebidos de lucro pelos
investidores: A\  = \ x\; B\  = \ y C\  = \ z. Como a divisão é diretamente proporcional ao valor investido, temos: \frac{x}{50.000} = \frac{y}{75.000}\  = \frac{z}{115.000} = k; \frac{x + y + z}{50.000 + 75.000 + 115.000} = \frac{60.000}{240.000} = 0,25; \frac{x}{50.000} = 0,25 \rightarrow x = 12.500; \frac{y}{75.000} = 0,25 \rightarrow y = 18.750; \frac{z}{115.000} = 0,25 \rightarrow z = 28.750. 
C - Incorreta, pois pegou o total do lucro e dividiu por 3 sem
considerar a proporcionalidade ao que cada um investiu.
D - Incorreta, pois considerou apenas o investidor A.
\end{itemize}

\section*{Matemática — Módulo 8 — Treino}

\begin{itemize}
\item A - Incorreta, pois uma corda não é apenas um segmento de reta que liga
dois pontos da circunferência, já que não necessariamente passa pelo
centro da circunferência.
B - Incorreta, pois essa opção descreve o raio da circunferência, não a
corda. O raio liga o centro da circunferência a um ponto específico na
circunferência, enquanto a corda liga dois pontos quaisquer da
circunferência.
C - Incorreta, pois um arco da circunferência não pode ser considerado
uma corda. A corda é um segmento de reta, enquanto o arco é uma parte da
circunferência.
D - Correta, pois a definição correta de uma corda é um segmento de reta
que liga o centro da circunferência a um ponto médio de um arco da
circunferência. Isso significa que a corda passa pelo centro da
circunferência e divide o arco em duas partes iguais.
\item A - Incorreta, pois, se um prisma retangular tivesse 8 arestas, teria
apenas duas arestas por face, o que não seria suficiente para formar as
arestas laterais.
B - Incorreta, pois, se um prisma retangular tivesse 10 arestas, teria
três arestas por face, o que também não seria suficiente para formar as
arestas laterais.
C - Incorreta, pois, se um prisma retangular tivesse 12 arestas, teria
quatro arestas por face, o que ainda não seria suficiente para formar as
arestas laterais.
D - Correta, pois um prisma retangular possui 12 arestas na base (4
arestas do retângulo superior + 4 arestas do retângulo inferior + 4
arestas verticais que conectam as bases). Além disso, existem duas
arestas laterais que se estendem verticalmente e conectam os vértices
das bases, totalizando 14 arestas.
\item A - Incorreta, pois 5 - 10 + 6 = 1\  \neq \ 2, logo, não corresponde
à relação de Euler.
B - Correta, pois V - 10 + 6 = 2 \rightarrow \ 6 - 10 + 6 = 2 ,
logo, tem 6 vértices.
C - Incorreta, pois 8 - 10 + 6 = 4\  \neq \ 2, logo, não corresponde
à relação de Euler.
D - Incorreta, pois 4 - 10 + 6 = 0\  \neq \ 2, logo, não corresponde
à relação de Euler.
\end{itemize}

\section*{Matemática — Módulo 9 — Treino}

\begin{itemize}
\item A - correta, pois a² + b² = c² representa corretamente o Teorema de
Pitágoras, que estabelece que, em um triângulo retângulo, o quadrado da
hipotenusa (c) é igual à soma dos quadrados dos catetos (a e b).
B - Incorreta, pois, na verdade, a soma dos quadrados dos catetos a e b
é igual ao quadrado da hipotenusa c, como afirma o Teorema de Pitágoras.
C - Incorreta, pois, em um triângulo retângulo, os lados a e b são os
catetos, e eles podem ter medidas diferentes.
D - Incorreta, pois não há uma relação específica entre as medidas dos
lados a, b e c de um triângulo retângulo. As medidas podem variar
dependendo do triângulo em questão.
\item A - Correta, pois, em polígonos semelhantes, os ângulos internos
correspondentes têm medidas iguais. Isso ocorre porque a semelhança
entre os polígonos preserva a congruência dos ângulos internos.
B - Incorreta, pois a proporção é uma relação entre as medidas dos lados
dos polígonos semelhantes, não entre os ângulos internos.
C - Incorreta, pois ângulos suplementares são aqueles que somam 180
graus, mas não há uma relação de suplementaridade entre os ângulos
internos de polígonos semelhantes.
D - Incorreta, pois ângulos complementares são aqueles que somam 90
graus, mas não há uma relação de complementaridade entre os ângulos
internos de polígonos semelhantes.
\item A - Incorreta, pois, embora essa afirmação seja verdadeira, ela se
refere aos ângulos formados pelas retas paralelas e uma transversal, não
aos ângulos internos de polígonos.
B - Incorreta, pois ângulos adjacentes são aqueles que têm um lado em
comum, mas não necessariamente são suplementares. Essa relação é
verdadeira apenas para ângulos lineares ou ângulos opostos pelo vértice,
não para todos os ângulos formados por retas paralelas cortadas por uma
transversal.
C - Correta, pois ângulos correspondentes são pares de ângulos que estão
em lados opostos da transversal e em posições correspondentes em relação
às retas paralelas. Esses ângulos têm medidas iguais.
D - Incorreta, pois ângulos consecutivos são ângulos que possuem o mesmo
vértice e um lado em comum, mas não necessariamente são complementares.
Essa relação é verdadeira apenas para ângulos suplementares, não para
todos os ângulos formados por retas paralelas cortadas por uma
transversal.
\end{itemize}

\section*{Matemática — Módulo 10 — Treino}

\begin{itemize}
\item A - Correta, pois, ao caminhar 500 metros para o norte, Júlia estará
acima da posição inicial de Carlos. Isso indica que Júlia está a
nordeste de Carlos. Carlos caminha 300 metros para o leste, o que o
coloca à direita da posição inicial de Júlia. Maria caminha 200 metros
para o sul e depois 400 metros para o oeste. Isso a coloca abaixo da
posição inicial de Júlia e à esquerda da posição inicial de Carlos.
Portanto, Maria está a oeste de Alice. Assim, a posição final de cada
pessoa pode ser descrita da seguinte forma: Júlia está a nordeste de
Carlos e a oeste de Maria.
B - Incorreta, pois não seguiu corretamente as direções e posições
finais das pessoas na trilha.
C - Incorreta, pois não seguiu corretamente as direções e posições
finais das pessoas na trilha.
D - Incorreta, pois não seguiu corretamente as direções e posições
finais das pessoas na trilha.
\item A - Incorreta, pois não foram seguidas corretamente as condições
especificadas para a localização da janela na planta da sala.
B - Correta. pois, de acordo com a primeira condição, a janela deve
estar posicionada na parede mais próxima à entrada da sala. No retângulo
que representa a planta da sala, a parede mais próxima à entrada é a
parede de 6 metros. De acordo com a segunda condição, a janela deve ser
colocada a uma distância igual a 4 metros da parede oposta à entrada.
Como a sala possui uma largura de 6 metros, a parede oposta à entrada é
a parede de 10 metros. Portanto, a janela deve estar localizada a 4
metros dessa parede. Assim, a posição correta da janela na planta da
sala é na parede de 6 metros, a 4 metros da entrada, conforme descrito
na alternativa B.
C - Incorreta, pois não foram seguidas corretamente as condições
especificadas para a localização da janela na planta da sala.
D - Incorreta, pois não foram seguidas corretamente as condições
especificadas para a localização da janela na planta da sala.
\item A - Incorreta, pois as direções não foram seguidas corretamente.
B - Incorreta, pois as direções não foram seguidas corretamente.
C - Incorreta, pois as direções não foram seguidas corretamente.
D - Correta, pois, ao caminhar 500 metros para o norte a partir de A,
João estará acima do ponto B. Isso indica que João está ao norte de B.
Em seguida, João vira à esquerda e caminha 300 metros para o leste. Isso
o coloca à direita da posição inicial de B. Por fim, João segue mais 200
metros para o sul. Isso o coloca abaixo da posição inicial de B.
Portanto, João está ao sul de B.
\end{itemize}

\section*{Matemática — Módulo 11 — Treino}

\begin{itemize}
\item A - Incorreta, pois a I não está correta. mMédia, é a soma dos valores
do conjunto de dados,dividido pela quantidade dos dados.
B - Incorreta, pois somente a I está incorreta.
C - Correta, pois a I é falsa. Moda é o valor que mais se repete.
D - Incorreta, pois, a definição de moda está incorreta na I.
\item A - Incorreta, pois, os valores informados não são próximos de
R\$1.500,00.
B - Correta, pois
MÉ\text{DIA} = \frac{1400 + 1359 + 1260 + 1300 + 1500 + 1400}{6} \cong 1369.
C - Incorreta, pois o cálculo não apresenta esse resultado.
D- Incorreta, pois os valores informados não chegam a essa média.
\item A - Correta, poi, somente profissão não pode ser quantificada.
B - Incorreta, pois batimentos cardíacos constituem uma variável
quantitativa.
C - Incorreta, pois profissão não é uma variável quantitativa e
batimentos cardíacos não são qualitativos.
D - Incorreta, pois profissão não é uma varáivel quantitativa.
\end{itemize}

\section*{Matemática — Módulo 12 — Treino}

\begin{itemize}
\item A - Incorreta, pois transformou errado cm² para m².
B - Incorreta, pois fez a divisão por 16 m² ao invés de 15 m².
C - Incorreta, pois considerou que 26,4 seria arredondado para 26
caixas.
D - Correta, pois área do apartamento \(= 1800 \times 2200 = 3.960.000\ cm^{2} = 396m²\). Caixas com 15m² \(= \frac{396}{15} = 26,4\ \text{caixas}\). Como não é possível comprar 1,4 caixas, Fernanda deve comprar 27 caixas.
\item A - Correta, pois 1 dia \rightarrow 25 + 15 + 5 = 45\ \text{minutos}; 5 dias \rightarrow 45 \times 5 = 225\ \text{minutos}; \frac{225}{60} = 3,75\ h\text{oras} \rightarrow 0,75\ h\text{oras} = 0,75 \times 60 = 45\ \text{minutos}. Portanto, o tempo de treino por semana é de 3 horas e 45 minutos.
B - Incorreta, pois calculou que 0,75 \times 60 = 35.
C - Incorreta, pois calculou que \frac{225}{60} = 4,75.
D - Incorreta, pois calculou que \frac{225}{60} = 4,75\ e\ 0,75 \times 60 = 35.
\item A - Incorreta, pois calculou o volume da piscina utilizando 2 ao invés
de 2,5 e ainda aproximou a divisão de maneira errada.
B - Incorreta, pois calculou o volume da piscina utilizando 2 ao invés
de 2,5.
C - Correta, pois volume da piscina = 12 \times 7 \times 2,5 = 210\ m^{3}; Capacidade da piscina = 210 \times 1000 = 210.000\ \text{litros}; Quantidade de caminhões =
\frac{210.000}{5.000} = 42\ \text{camin}hõ\text{es}.
D - Incorreta, pois fez a multiplicação do volume errado, encontrando
215 ao invés de 210.
\end{itemize}

\section*{Matemática — Módulo 13 — Treino}

\begin{itemize}
\item A - Incorreta, pois usou as três jogadas no numerador, porém, o correto
seria a possibilidade de cair coroa.
B - Correta, pois são de eventos independentes, ou seja, eles não
dependem do evento anterior para ocorrer. P(A).\ P(B).\ P(C) = \ \frac{1}{2} \times \frac{1}{2} \times \frac{1}{2} = \ \frac{1}{8}.
C - Incorreta, pois considerou a quantidade vezes de jogadas como o
numerador e desconsiderou que se trata de eventos independentes.
D - Incorreta, pois multiplicou o número de jogadas pelo número de
possibilidades em uma moeda, desconsiderando que são eventos
independentes.
\item A - Incorreta, pois não foram considerados eventos dependentes.
B - Incorreta, pois não foram considerados eventos dependentes para 5
jogadas.
C - Correta, pois são eventos dependentes. O primeiro evento é retirar a
bola 18, que é dada por: P(A) = \ \frac{1}{75}\ . Retirando a
primeira bola e não devolvendo, restam 74 bolas, assim a probabilidade
de retirar a bola 4 de primeira, no segundo evento é
P(A|B) = \ \frac{1}{74}. Assim, a probabilidade do evento ocorrer é:
P(A).P(A|B) = \ \frac{1}{5550}.
D - Incorreta, pois retirou uma das bolas do espaço amostral, mas não
considerou o evento ocorrido anteriormente.
\item A - Incorreta, pois, para que o arremessador acertasse todas as bolas, a
porcentagem de erro seria igual a 0.
B - Incorreta, pois 21\% é a aproximadamente a chance de erro para um
arremesso.
C - Correta. pois são eventos independentes. Devemos multiplicar as
probabilidades. P(1) P(2) P(3) são as chances de acerto, então, basta
descobrirmos as chances de acertos dado por \(1 - 0,18 = \ 0,82.\). P(1).P(2).P(3) = \ 0,551\ \text{aproximadamente}\ 55\%.
D- Incorreta, pois, para que os arremessos fossem perto de 1\%, a
probabilidade de erro do arremessador seria muito perto de 0,98, ou
melhor, 98\%.
\item 
\end{itemize}


\section*{Matemática — Simulado 1}
\begin{itemize}
\item A - Incorreta, pois Fábio fez em um tempo menor que Tiago e veio depois
na listagem.
B - Incorreta, pois Isabel fez em um tempo menor que Fábio e veio depois
na listagem.
C - Correta, pois os números que possuem as partes inteiras iguais foram
analisados pelas casas decimais.
D - Incorreta, pois, apesar de Júlio e Isabel estarem nas posições
corretas, o Fábio fez um tempo menor que Tiago, logo, ele não seria o
último.
\item A - Incorreta, pois resolveu
17 + ( - 2) \times \left( - 28 \right) = 15 \times ( - 28) = - 420.
B - Incorreta, pois resolveu
17 + ( - 2) \times \left( - 28 \right) = - 15 \times ( - 28) = 420. 
C - Incorreta, errou no jogo de sinal e resolveu pois resolveu que
( - 2) \times \left( - 28 \right) = \  - 96.
D - Correta, pois
17 + ( - 2) \times \left\{ - 4 + 2 \times \left\lbrack 9 - \left( 21 \right) \right\rbrack \right\} = 17 + ( - 2) \times \left\{ - 4 - 24 \right\} = 17 + ( - 2) \times \left( - 28 \right) = 73.
\item A - Correta, pois J + 20 = \ \text{Tio}; J = 39 - 23 = 16; 16 + 20 = 36.
B - Incorreta, pois, se Juliana tivesse 22 anos, seu tio teria 42 anos,
o que não condiz com o enunciado.
C - Incorreta, pois, se Juliana tivesse 25 anos, seu tio teria 45 anos,
o que não condiz com o enunciado.
D - Incorreta, pois, se Juliana tivesse 35 anos, seu tio teria 55 anos,
o que não condiz com o enunciado.
\item A - Incorreta, pois considerou que 0,0164\cong 16\%
B - Incorreta, pois além de considerar 0,0164\cong 16\%,
interpretou como desconto.
C - Correta, pois, fazendo os cálculos com os fatores de multiplicação,
temos 1,1 \times 0,88 \times 1,05 = 1,0164 - 1 = 0,0164\cong 1,6\%.
D - Incorreta, pois fez as contas de forma correta, mas interpretou como
desconto ao invés de aumento.
\item A - Incorreta, pois a quantidade de fatias pintadas está correta, porém,
o denominador não condiz, uma vez, que é dividido em 4 partes.
B - Incorreta, pois, embora a fração \frac{1}{4} apareça, a parte
inteira tem somente um círculo todo pintado.
C - Correta, pois apresenta 1 círculo todo pintado, que é a parte
inteira e \frac{1}{4} de outro.
D- Incorreta, pois a representação da parte inteira e da parte decimal,
corresponde a 1 inteiro, logo 2 inteiros, e na imagem somente um círculo
está todo pintado.
\item A - Incorreta, pois o primeiro termo seria
a_{1} = 2 \times 1 + 2 = 2 + 2 = 4, o que não confere.
B - Correta., pois temos, a_{1} = 1^{2} + 1 = 1 + 1 = 2,
a_{2} = 2^{2} + 2 = 4 + 2 = 6, a_{3} = 3^{2} + 3 = 9 + 3 = 12 e
a_{4} = 4² + 4 = 16 + 4 = 20.
C - Incorreta, pois o primeiro termo seria a_{1} = 4 \times 1 = 4, o
que não confere.
D - Incorreta, pois o primeiro termo seria
a_{1} = 1\left( 1 - 1 \right) = 1 \times 0 = 0, o que não confere.
\item A - Incorreta, pois calculou a probabilidade com 3 cartas,
desconsiderando os eventos e a independência entre eles.
B - Incorreta, pois não considerou que foram retiradas 2 cartas de uma
vez.
C - Incorreta, pois calculou apenas a probabilidade do primeiro evento
acontecer.
D - Correta, pois são eventos dependentes. A primeira jogada influencia
na segunda jogada: P\left( A \right) = \frac{1}{52};\ P\left( A \right) = \frac{1}{50}; P\left( A \right) \times P\left( A \right) = \frac{1}{52} \times \frac{1}{50} = \frac{1}{2600}.
\item A - Incorreta, pois considerou que, mesmo mudando a velocidade, o tempo
permaneceria o mesmo.
B - Correta, pois velocidade e tempo são grandezas inversamente
proporcionais, logo, o produto entre elas é constante. Assim,
5 \times 120 = 100x \rightarrow 100x = 600 \rightarrow x = \frac{600}{100} = 6\ h\text{oras}.
C - Incorreta, pois considerou as grandezas como diretamente
proporcionais e ainda arredondou 4,16 horas para 4 horas e 9 minutos.
D - Incorreta, pois considerou as grandezas como diretamente
proporcionais.
\item A - Incorreta, pois o histograma é feito por linhas e barras.
B - Incorreta, pois o gráfico de barras é formado por barras
retangulares e com base maior na horizontal.
C - Correta, pois esse tipo de gráico apresenta setores de uma figura
geométrica, geralmente, um círculo.
D - Incorreta, pois o gráfico de linhas é representado por pontos unidos
por linhas.
\item A - Incorreta, pois os valores dos vértices não condizem com o processo
de reflexão.
B - Correta, pois, ao realizar uma reflexão em relação ao eixo x, os
pontos mantêm a mesma coordenada x, mas têm sua coordenada y negativa.
No triângulo ABC original, o ponto A(2, 4) terá a mesma coordenada x,
mas sua coordenada y será negativa, resultando em A'(-2, -4). Da mesma
forma, os pontos B(5, 6) e C(7, 2) terão suas coordenadas y negativas
após a reflexão, resultando em B'(5, -6) e C'(7, -2), respectivamente.
C - Incorreta, pois os valores dos vértices não condizem com o processo
de reflexão.
D - Incorreta, pois os valores dos vértices não condizem com o processo
de reflexão.
\end{itemize}

\section*{Matemática — Simulado 2}
\begin{itemize}
\item A - Correta, pois pois o comprimento do lado BC é igual ao comprimento
do lado AC em um triângulo retângulo com ângulos de 45 graus.
B - Incorreta, pois essa resposta seria correta se o triângulo fosse um
triângulo isósceles retângulo de 45-45-90 graus, mas no problema não foi
mencionado que os ângulos são iguais.
C - Incorreta, pois essa resposta seria correta se o triângulo fosse um
triângulo equilátero de 60 graus, mas no problema não foi mencionado que
os ângulos são iguais.
D - Incorreta, pois essa resposta é o dobro do comprimento do lado AC, o
que não é possível em um triângulo retângulo com ângulos de 45 graus.
\item A - Incorreta, pois considerou a distância das duas imagens como o total
do deslocamento.
B - Correta, pois, pegando o vértice A como referência, podemos observar
que ele foi deslocado em três espaços para a direita, gerando o vértice
A'. Como cada linha da malha é 1 cm, o total do deslocamento foi de 3 cm
para a direita.
C - Incorreta, pois, além de considerar a distância das duas imagens
como o total do deslocamento, confundiu a direção.
D - Incorreta, pois fez o deslocamento correto, mas confundiu a direção.
\item A - Correta, pois,
Mé\text{dia}\ \text{yama}ha = \ \frac{12 + 8 + 13}{3} = 11.
B - Incorreta, pois,
\ Mé\text{dia}\ \text{Suzuki}\  = \ \frac{7 + 4 + 10}{3} = 7.
C - Incorreta, pois,
Mé\text{dia}\ \text{BMW} = \ \frac{2 + 6 + 14}{3} = 7,3.
D - Incorreta, pois,
Mé\text{dia}\ \text{Honda} = \ \frac{8 + 3 + 7}{3} = 6.
\item A - Incorreta, pois considerou que a relação m³ é direta ao litro.
B - Incorreta, pois converteu m³ para litro multiplicando por 10.
C - Incorreta, pois converteu m³ para litro multiplicando por 100.
D - Correta, pois o volume do cubo cprresponde ao tamanho da aresta
elevado à terceira potência. Logo, V = 9^{3} = 729m³ . Como
1m^{3} = \ 1000\ L \rightarrow \ 729\ m³\  = \ 729.000 litros.
\item A - Incorreta, pois ao calcular 5c = 75, passou o 5 subtraindo ao
invés de dividindo.
B - Incorreta, pois substituiu o valor de d no lugar de c.
C - Incorreta, pois ao calcular 5c = 75, passou o 5 somando ao invés
de dividindo.
D - Correta, pois, para encontrar o valor de s, basta substituir o valor
de t. Assim, 5c = d\  \rightarrow \ 5c = 75\  \rightarrow \ c = \frac{75}{5} = 15
\end{itemize}

\section*{Matemática — Simulado 3}
\begin{itemize}
\item A - Correta, pois, usando o princípio multiplicativo, as combinações
distintas são 5 \times 4 \times 3 \times 2 \times 3 = 360. Como Marcela precisa de 300 combinações distintas, ainda vão sobrar 60
opções de combinações.
B - Incorreta, pois não percebeu que o número mínimo de combinações foi
atingido.
C - Incorreta, pois não considerou todas as peças de roupa.
D - Incorreta, pois somou as opções de escolha ao invés de multiplicar.
\item A - Incorreta, pois considerou uma desvalorização de 20 reais.
B - Correta, pois, como houve uma desvalorização de 20\%, vamos utilizar
o fator multiplicativo 100\% - 20\% = 80\% = 0,8, assim 0,8 \times 1800 = 1440.
C - Incorreta, pois considerou o valor da desvalorização e não o valor
final.
D - Incorreta, pois, ao fazer 1800 - 360, encontrou 1400 ao invés de
1440.
\item A - Incorreta, pois considerou que, para duas sequências serem
equivalentes, basta ter uma parte em comum.
B - Incorreta, pois considerou que, ao trocar o n com n² de lugar, uma
sequência semelhante seria gerada.
C - Correta., pois, aplicando a propriedade distributiva, temos:
n^{2}\left( n + 1 \right) = n^{3} + n^{2}.
D - Incorreta, pois, ao fazer a distributiva, apenas multiplicou o n
pelo n².
\item A - Incorreta, pois considerou que basta igualar 1 cm no desenho ao
mundo real.
B - Incorreta, pois, ao simplificar \frac{10}{1000000}, foi colocado
um 0 a mais no numerador.
C - Incorreta, pois, ao simplificar \frac{10}{1000000}, foi colocado
um 0 a mais no denominador.
D - Correta, pois, como escala é a razão
\frac{\text{desen}ho}{\text{real}} e ambos possuem a mesma unidade
de medida, temos \frac{\text{desen}ho}{\text{real}} = \frac{10\text{cm}}{10\text{km}} = \frac{10}{1000000} = \frac{1}{100000}
\item A - Incorreta, pois considerou que o círculo é uma figura rígida por não
possuir pontas.
B - Correta, pois, como estudado, a forma geométrica que possui maior
rigidez é o triângulo. Assim, a mesa que apresenta a maior rigidez na
base é a de base triangular.
C - Incorreta, pois considerou a estrutura que costuma ver com maior
frequência no dia a dia.
D - Incorreta, pois considerou o hexágono com maior rigidez que o
triângulo.
\item A - Incorreta, pois o valor encontrado a partir do cálculo da área não é
esse.
B - Incorreta, pois o valor encontrado a partir do cálculo da área não é
esse.
C - Incorreta, pois o valor encontrado a partir do cálculo da área não é
esse.
D - Correta, pois a área da sala de estar é calculada multiplicando o
comprimento pela largura:
{6 \; metros} \times {4\; metros} = 24\;m^2 Como o quarto principal
tem o dobro da área da sala de estar, a área do quarto principal será 48
metros quadrados.
\item A - Incorreta, pois considerou que o volume de um bloco retangular é a
soma das três medidas.
B - Incorreta, pois além de somar as três medidas, a unidade está
errada.
C - Correta, pois, como o volume de um bloco retangular é o produto das
medidas dos seus lados, temos que
V = 12 \times 7 \times 5 = 420\text{cm}³.
D - Incorreta, pois, apesar de o valor do volume ter sido calculado
corretamente, a unidade de medida está errada.
\item A - Incorreta, pois o gráfico não faz essa relação.
B - Correta, pois o gráfica não apresenta faixas etárias.
C - Incorreta, pois o gráfico não apresenta tais fatores.
D - Correta, pois a tabela de dupla entrada apresenta a relação entre
duas variáveis: o nível de escolaridade e a renda média mensal. A
finalidade dessa pesquisa é analisar e inferir como o nível de
escolaridade influencia a renda dos indivíduos. Ao cruzar os dados da
tabela, é possível observar se há uma relação entre a escolaridade e a
renda e, assim, avaliar a influência dessa variável na determinação da
renda média mensal.
\item A - Incorreta, pois foi feito o cálculo de probabilidade normal, sem
considerar os eventos e considerando que o dado havia sido jogado uma
única vez.
B - Incorreta, pois foi feito o cálculo de uma probabilidade normal sem
considerar os eventos.
C - Incorreta, pois foi calculada a probabilidade de acertar a face
número 4 uma vez.
D - Correta, pois são eventos independentes. Desse modo:
P\left( A \right) \times P\left( B \right) = \frac{1}{12} \times \frac{1}{12}\  = \frac{1}{144}.
\item A - Incorreta, pois, o cupcake não estava dividido em oito pedaços.
B - Incorreta, pois Leonardo consumiu mais de um cupcake.
C - Correta, pois foi consumido 1 cupcake inteiro e metade do de
Juliana.
D - Incorreta, pois Leonardo não consumiu 5 cupcakes inteiros e 1 parte
entre 4 de outro.
\end{itemize}

\section*{Matemática — Simulado 4}
\begin{itemize}
\item A - Incorreta, pois, se a idade de Alice fosse 3, a idade seria igual a
18.
B - Correta, pois:
3x + 19 = 40 \Rightarrow \ 3x = 40 - 19 - \Rightarrow 3x = 21 \Rightarrow x = 7.
C - Incorreta, pois, se a idade de Alice fosse 20, a soma teria que ser
79.
D - Incorreta, pois, se a idade de Alice fosse 40, a soma teria que ser
139.
\item A - Incorreta, pois o ângulo não equivale à metade de 30º.
B - Incorreta, pois esse valor não condiz ao que foi pedido.
C - Correta, pois, quando o ponto está no centro da circunferência, ele
apresenta o mesmo tamanho do arco.
D - Incorreta, pois não equivale o dobro do ângulo.
\item A - Correta, pois a reflexão em relação ao eixo y faz com que o ponto
A(2,3) se transforme no ponto A'(-2,3). Em seguida, a translação de 4
unidades para a esquerda e 2 unidades para baixo transforma o ponto
A'(-2,3) no ponto B(-6,1).
B - Incorreta, pois não leva em consideração a reflexão em relação ao
eixo y.
C - Incorreta, pois não leva em consideração a reflexão em relação ao
eixo y.
D - Incorreta, pois não leva em consideração a translação de 4 unidades
para a esquerda e 2 unidades para baixo.
\item A - Incorreta, pois é um gráfico de equação de segundo grau.
B - Incorreta, pois é um gráfico de logarítmica
C - Incorreta, pois é um gráfico modular
D - Correta: pois se trata de uma reta.
\item A - Incorreta, pois, ao invés de multiplicar o valor de t por 2, fez a
soma e subtraiu o 15 ao invés de somar.
B - Incorreta, pois ao invés de multiplicar o valor de t por 2 fez a
soma.
C - Correta, pois, para encontrar o valor de s, basta substituir o valor
de t dado. Assim, s - 15 = 2t\  \rightarrow \ s - 15 = 2\  \times \ 8\  \rightarrow \ s - 15 = 16\  \rightarrow \ \ s = 16 + 15 = 31.
D - Incorreta, pois ao invés de fazer 16 + 15, fez 16-15.
\end{itemize}

\section*{Matemática — Simulado 5} 
\begin{itemize}
\item A - Correta, pois, nesses dias, o lucro foi de 407, 500 e 700,
respectivamente.
B - Incorreta, pois não levou em conta todo os dias em que o lucro foi
maior.
C - Incorreta, pois não levou em conta todo os dias em que o lucro foi
maior.
D - Incorreta, pois não levou em conta todo os dias em que o lucro foi
maior.
\item A - Incorreta, pois foi feito o cálculo da probabilidade de uma única
moeda.
B - Incorreta, pois não considerou cada lance como um evento
independente.
C - Incorreta, pois o espaço amostral não pode ser 8.
D - Correta, pois são eventos independentes, logo
P\left( A \right) \times P\left( B \right) = \frac{1}{2}\  \times \frac{1}{2} = \frac{1}{4}.
\item A - Incorreta, pois essa fração corresponde a comer somente 1 pedaço da
pizza.
B - Incorreta, pois essa fração corresponde a 4 pizzas inteiras e um
pedaço em uma pizza de corte diferente.
C - Correta, pois é uma fração imprópria com a parte inteira 8 e mais os
dois pedaços.
D - Incorreta, pois essa fração corresponde a 7 pizzas inteiras e metade
de outra.
\item A- Incorreta, pois, se o número fosse 12, a soma seria 106.
B - Correta, pois 3x + 70 = 103 \rightarrow \ 3x = 103 - 70 \rightarrow \ 3x = 33 \rightarrow x = 11.
C - Incorreta, pois, se o número fosse 20, a soma teria que ser 130.
D - Incorreta, pois, se o número fosse 103, a soma teria que ser 379.
\item A - Incorreta, pois essa resposta não corresponde ao cálculo correto do
arco.
B - Correta, pois o arco correspondente ao ângulo central de 45 graus
possui um comprimento de aproximadamente 6.28 cm.
C - Incorreta, pois essa resposta corresponde ao perímetro da
circunferência, não ao comprimento do arco de um ângulo central
específico.
D - Incorreta, pois essa resposta corresponde ao dobro do perímetro da
circunferência, não ao comprimento do arco de um ângulo central
específico.
\item A - Incorreta, pois considerou que o tempo de encontro seria pelo M.D.C
ao invés do M.M.C. . B - Incorreta, pois apenas foi feita a soma do
tempo de cada corredor. C - Incorreta, pois foi associado um número
errado de minutos. D - Correta, pois devemos usar a ideia de mínimo
múltiplo comum. Assim, calculando o M.M.C dos tempos, temos: 15, 25, 3; 5, 25, 5; 1, 5, 5; 1, 1, (3x5x5 = 75). Logo, os atletas se encontram depois de 75 minutos, ou seja, 1 hora e 15
minutos.
\item A - Incorreta, pois considerou que o numerador da fração centesimal é a
forma percentual sem a vírgula.
B - Correta, pois
25,3\% = \frac{25,3}{100} = \frac{253}{1000} = 0,253.
C - Incorreta, pois esqueceu de adicionar um 0 no denominador da
representação fracionária ao andar com a vírgula.
D - Incorreta, pois se esqueceu de levar em conta as 3 casas decimais no
denominador.
\item A - Incorreta, pois ao invés de fazer \frac{12}{1,5}\ calculou 12 x
1,5. B - Incorreta, pois ao invés de fazer 18,5 - 6,5 fez uma soma. C -
Incorreta, pois considerou apenas o preço pago por km sem o valor da
bandeira. D - Correta, pois, como a corrida tem um preço por km mais um
fixo, podemos escrever de maneira geral que a corrida custa
c = 1,5q + 6,5. Assim, 18,5 = 1,5q + 6,5 \rightarrow 1,5q = 12 \rightarrow \ q = \frac{12}{1,5} \rightarrow q = 8\text{km}.
\item A - Correta, pois, quanto mais pessoas para atender, mais tempo será
gasto, logo são G.D.P. Para fazer o cálculo, precisamos do tempo em uma
unidade apenas, ou seja, 1 hora e 40 minutos será 100 minutos. Assim, \frac{100}{12} = \frac{x}{18} \rightarrow 12x = 1800 \rightarrow x = \frac{1800}{12} = 150\ \text{minutos} = 2\ h\text{oras}\ e\ 30\ \text{minutos}. B - Incorreta, pois, ao converter 150 minutos para horas, considerou 1
hora ao invés de 2. C - Incorreta, pois considerou as grandezas como sendo inversamente
proporcionais, fazendo 100\ .\ 12\  = \ 18x. D - Incorreta, pois considerou que 1 hora tem 100 minutos.
\item A - Incorreta, pois 2 cm é a medida da altura relativa ao lado BC.
B - Incorreta, pois 3 cm é a medida da altura relativa ao lado AC.
C - Correta, pois a altura relativa ao lado AB divide o triângulo ABC em
dois triângulos retângulos, onde a altura é a hipotenusa e os catetos
são os segmentos AH e BH. Podemos utilizar o teorema de Pitágoras para
calcular a medida da altura e chegar ao resultado.
D - Incorreta, pois 5 cm é a medida da mediana relativa ao lado AB.
\end{itemize}

\section*{Matemática — Simulado 6} 
\begin{itemize}
\item A - Correta, pois essa definição está correta.
B - Incorreta, pois as primeiras definições estão incorretas.
C - Incorreta, pois a definição de planta está incorreta.
D - Incorreta, pois há diferenças entre esses conceitos.
\item A - Incorreta, pois essa resposta corresponde ao cálculo incorreto da
área.
B - Correta, pois, substituindo os valores fornecidos na fórmula, temos:
Área do triângulo = \frac {(10 \times 6)}{2} = \frac {60}{2} =
30 \;cm^2. Portanto, a área do triângulo retângulo é de 30 cm².
A = \frac{\left( b + B \right)h}{2} = \frac{(6 + 18) \times 10}{2} = 24 \times 5 = 120m².
C - Incorreta, pois essa resposta não corresponde ao cálculo correto da
área do triângulo com os valores fornecidos.
D - Incorreta, pois essa resposta corresponde à multiplicação da base
pela altura, sem dividir por 2, o que resulta em uma área incorreta.
\item A - Correta, pois como não saiu do eixo e só subiu duas unidades da
simetria, ocorreu a translação.
B - Incorreta, pois, para ocorrer a rotação, teríamos uma inclinação.
C - Incorreta, pois a transformação geométrica reflexão apenas repetiria
a figura.
D - Incorreta, pois a imersão não é uma transformação geométrica.
\item A - Incorreta, pois Yan é o segundo colocado.
B- Incorreta, pois Renato é o último colocado.
C - Incorreta, pois Cristian ficou em primeiro lugar.
D - Correta, pois Rafael ficou na terceira posição.
\item A - Incorreta, pois não trocou a operação do 22 ao mudar de lado na
igualdade.
B - Correta, pois, para encontrar o valor de x, basta substituir o valor
de y. Assim, x - y + 20 = 0 \rightarrow x - ( - 2) + 20 = 0 \rightarrow \ x + 22 = 0 \rightarrow \ x = - 22.
C - Incorreta, pois, ao substituir o y, a mudança de sinal não foi
feita.
D - Incorreta, pois, além de não fazer a mudança do sinal com o y, a
operação do 18 não foi feita.
\end{itemize}

\section*{Matemática — Simulado 7} 
\begin{itemize}
\item A - Incorreta, pois, ao determinar o valores de x, y, z, foi feita uma
multiplicação por 2 ao invés de uma divisão.
B - Correta, pois \frac{x}{42} = \frac{y}{16} = \frac{z}{18} = \frac{x + y + z}{42 + 16 + 18} = \frac{38}{76} = \frac{1}{2}. \frac{x}{42} = \frac{1}{2} \rightarrow \ \ x = 21. \frac{y}{16} = \frac{1}{2} \rightarrow \ \ x = 8\ \ \ \ \ \ \ \ \ \ \ \ \frac{z}{18} = \frac{1}{2} \rightarrow \ \ x = 9.
C - Incorreta, pois, além de calcular o valor das incógnitas usando a
multiplicação, os valores que seriam y e z foram invertidos.
D - Incorreta, pois, apesar de fazer os cálculos corretos, a ordem dos
valores de y e z foi invertida.
\item 
A - Incorreta, pois, apesar de a escola ter 9 turmas, uma delas é
excluída da votação.
B - Incorreta, pois o valor de amostra e população estão invertidos.
C - Correta, pois temos 8 turmas como população e 3 turmas de amostra.
D - Incorreta, pois o valor da amostra e população estão invertidos.
\item A - Correta, pois \frac{4}{8} é equivalente a \frac{1}{2}.
B - Incorreta, pois a fração não corresponde ao galão inteiro, uma vez
que encheu 4 partes de 8 divididas.
C - Incorreta, pois a fração irredutível encontrada está errada.
D - Incorreta, pois essa fração não representa a parte cheia do galão.
\item A - Incorreta, pois, se o número fosse 355, o resultado final seria
diferente.
B - Correta, pois
x + 1200 = 4x \rightarrow \ 4x - x = 1200 \rightarrow 3x = 1200 \rightarrow x = 400.
C - Incorreta, pois, se o número fosse 200, o resultado final seria
diferente.
D - Incorreta, pois, se o número fosse 420, o resultado final seria
diferente.
\item A - Incorreta, pois o valor do ângulo central não foi considerado
corretamente.
B - Incorreta, pois o valor do ângulo central não foi considerado
corretamente.
C - Correta, pois Arco = \frac {Ângulo}{360º} \times 2π \times raio.
No caso do problema, substituindo os valores na fórmula, temos 3π cm.
D - Incorreta, pois o valor do ângulo central não foi considerado
corretamente.
\item A - Incorreta, pois a rotação acontece quando o objeto inclina-se.
B - Correta, pois o ato de posicionar um espelho reflete a imagem.
C - Incorreta, pois a translação ocorre quando o objeto segue pelas
direções norte, sul, leste e oeste.
D - Incorreta, pois no inchaço há deformação na imagem.
\item A - Incorreta, pois a divisão de 19 por 3 é maior que 6.
B - Incorreta, pois a divisão de 19 por 3 é menor que 7.
C - Incorreta, pois a divisão de 19 por 3 é maior que 1.
D - Correta, pois, analisando a fração como divisão, temos que o 3 está
dentro do 9 aproximadamente 6 vezes, logo, a fração é maior do que 6 e
menor do que 7.
\item A - Incorreta, pois esqueceu de considerar que o 3 também dividiu os
valores simultaneamente.
B - Incorreta, pois, ao invés de multiplicar o 3 e o 5, fez a soma
deles.
C - Correta, pois, para dividir as lembranças na maior quantidade
possível, usamos o cálculo do M.D.C; assim, fatorando a quantidade de
cada doce, temos 75, 60, 45, 2; 75, 30, 45, 2; 75, 15, 45, 3; 25, 5, 15, 3; 25, 5, 5, 5; 5, 1, 1, 5; 1, 1, 1. Logo, em cada saquinho terá 15 doces.
D - Incorreta, pois, ao fatorar, considerou que usaria todos os fatores
como se fosse M.M.C.
\item A - Correta, pois \text{Na}\ \text{loja}\ \text{virtual} \rightarrow 3500 \times 0,95 = 3.325 + 35 = 3360. \text{Na}\ \text{loja}\ fí\text{sica} \rightarrow 3800 \times 0,9 = 3420. Portanto, a melhor opção de Larissa é a loja virtual.
B - Incorreta, pois não considerou o valor do frete da loja virtual.
C - Incorreta, pois confundiu as informações da loja física com a
virtual.
D - Incorreta, pois, ao realizar as operações, o resultado obtido é
outro.
\item A - Incorreta, pois considerou que a sequência 2 não apresenta uma
relação entre os termos e que os números primos possuem algum padrão.
B - Incorreta, pois considerou que a sequência dos números primos pode
ser encontrada por alguma relação entre os termos.
C - Incorreta, pois não enxergou que a sequência 2 apresenta uma relação
entre os termos por meio do antecessor subtraído de 2.
D - Correta, pois podemos observar que a sequência I é determinada pelo
termo anterior mais 3, a sequência II pelo termo anterior menos 2, e a
III é a sequência dos números primos. Como a sequência recursiva é
aquela em que um termo depende do anterior, temos recursiva, recursiva e
não recursiva.
\end{itemize}

\section*{Matemática — Simulado 8} 
\begin{itemize}
\item A - Incorreta, pois, ao determinar o valores de x, y, z, foi feita uma
multiplicação por 2 ao invés de uma divisão.
B - Correta, pois \frac{x}{42} = \frac{y}{16} = \frac{z}{18} = \frac{x + y + z}{42 + 16 + 18} = \frac{38}{76} = \frac{1}{2}. \frac{x}{42} = \frac{1}{2} \rightarrow \ \ x = 21. \frac{y}{16} = \frac{1}{2} \rightarrow \ \ x = 8\ \ \ \ \ \ \ \ \ \ \ \ \frac{z}{18} = \frac{1}{2} \rightarrow \ \ x = 9.
C - Incorreta, pois, além de calcular o valor das incógnitas usando a
multiplicação, os valores que seriam y e z foram invertidos.
D - Incorreta, pois, apesar de fazer os cálculos corretos, a ordem dos
valores de y e z foi invertida.
\item A - Incorreta, pois não corresponde à medida correta do ângulo agudo.
B - Incorreta, pois não corresponde à medida correta do ângulo agudo.
C - Correta, pois, no triângulo retângulo, o ângulo agudo oposto ao
cateto é sempre o complementar do ângulo formado entre a hipotenusa e o
cateto. Esse ângulo pode ser encontrado usando a função trigonométrica
seno.
D - Incorreta, pois corresponde ao ângulo reto, não ao ângulo agudo
oposto ao cateto.
\item A - Incorreta, pois corresponde à soma apenas dos comprimentos das duas
arestas menores.
B - Incorreta, pois corresponde à soma apenas dos comprimentos das duas
arestas maiores.
C - Correta, pois o perímetro de um retângulo é dado pela soma dos
comprimentos de todos os lados, ou seja, a soma dos comprimentos das
quatro arestas.
No caso do retângulo descrito no problema, temos duas arestas de
comprimento 10 cm e duas arestas de comprimento 5 cm. Portanto, o
perímetro é dado por: P = 10 cm + 10 cm + 5 cm + 5 cm P = 20 cm + 10 cm P = 30 cm. 
D - Incorreta, pois corresponde à soma de todos os lados do retângulo,
incluindo os comprimentos das arestas duas vezes.
\item A - Incorreta, pois corresponde à metade da área correta do triângulo.
B - Correta, pois a fórmula para calcular a área de um triângulo é dada
pela metade do produto da base pela altura: A =
\frac {(base \times altura)}{2}. Substituindo os valores do problema, temos: A = 24 cm².
C - Incorreta, pois não corresponde à área correta do triângulo.
D - Incorreta, pois corresponde à área do retângulo formado pela base e
altura do triângulo, não à área do próprio triângulo.
\item A - Correta, pois, para encontrar o valor de b, basta substituir o valor
de a. Assim, 2a + 8 - 3b = 5\  \rightarrow \ 2\  \times \ \left( - 3 \right) + 8 - 3b = 5 \rightarrow \  - 6 + 8 - 3b = 5 - 3b = 3\  \rightarrow \ b = \  - 1.
B - Incorreta, pois, ao dividir 3 por -3, a regra de sinal não foi
respeitada.
C - Incorreta, pois ao fazer 2\  \times \ \left( - 3 \right), foi
encontrado 6 ao invés de -6.
D - Incorreta, pois, ao passar o 2 de lado, a regra do sinal foi
ignorada.
\end{itemize}