\addcontentsline{toc}{chapter}{Simulado 1}
\markboth{Simulado 1}{}

\num{1} Observe o cartaz abaixo para responder à questão.

\includegraphics[width=5.90551in,height=2.125in]{./imgSAEB_7_POR/media/image16.png}

\fonte{Secretaria do Meio Ambiente do Estado do Amazonas. Campanha Floresta faz a diferença. 
Disponível em: https://meioambiente.am.gov.br/campanha-floresta-faz-a-diferenca/.
Acesso em: 22 mai. 23}

Acima, pode-se observar um cartaz de conscientização sobre o desmatamento da
Amazônia. O uso e repetição de determinadas palavras e imagens tem como
finalidade sensibilizar os leitores para o problema do desmatamento.
Sobre o uso da palavra ``floresta'', no cartaz, assinale a alternativa
correta.

\begin{escolha}

  \item Destacada à esquerda em letras grandes, uma parte da palavra ``floresta'' compõe um jogo com 
  palavras do texto no centro do cartaz.   

  \item A parte destacada da palavra ``floresta'', à esquerda, remete aos malefícios das queimadas e desmatamento
  que o cartaz pretende combater. 

  \item O destaque da palavra ``floresta'' não dialoga com ``diferença'' e ``diferente'', que se articulam para convidar a uma mudança de atitude.

  \item O uso da palavra ``floresta'', em destaque, à esquerda, dialoga com as imagens à direita sem remeter ao texto central do cartaz da campanha. 

\end{escolha}

\coment{SAEB: Identificar o uso de recursos persuasivos em textos verbais e
não verbais.

BNCC: EF67LP07 -- Identificar o uso de recursos persuasivos em textos
argumentativos diversos (como a elaboração do título, escolhas lexicais,
construções metafóricas, a explicitação ou a ocultação de fontes de
informação) e perceber seus efeitos de sentido.

a) Correta. A parte de palavra ``resta'' dialoga com a frase ``se você fizer diferente, a floresta faz o
restante'': \textit{resta} e \textit{restante} dizem respeito ao papel da \textit{floresta}.     
b)Incorreta. A parte destacada da palavra ``floresta'' não remete aos malefícios das queimadas e desmatamento, 
mas ao contrário: \textit{resta} e \textit{restante} dizem respeito ao papel da \textit{floresta}. 
c)Incorreta. O destaque da palavra ``floresta'' dialoga com ``diferença'' e ``diferente'': se as pessoas fizerem
diferente, a floresta faz o \textit{restante}, palavra que remete à parte da palavra ``floresta'', destacada 
à esquerda. 
d)Incorreta. O uso da palavra ``floresta'' certamente dialoga com as imagens à direita e com o texto central do cartaz da campanha.}

\num{2} Leia o texto abaixo para responder à questão. 

\begin{quote}
\textbf{Fumaça de queimadas do Amazonas chega a São Paulo; moradores
relatam ``cheiro forte''}

\textit{Segundo a Climatempo, a nuvem de fumaça foi gerada por incêndios em
parte do Amazonas, Acre e Mato Grosso.}

Moradores da cidade de São Paulo relataram sentir cheiro de fumaça na
manhã desta sexta-feira (9).

Na quinta (8), a Climatempo havia alertado para a nuvem gerada por
queimadas em parte do Amazonas, Acre e Mato Grosso, que se espalhou sobre o
Brasil, e que chegaria também em grande parte do Centro-Oeste, no Paraná
e até em parte do estado de São Paulo.

A Companhia Ambiental do Estado de São Paulo (CETESB) disse que a
Divisão de Qualidade do Ar da agência está investigando as fumaças e
monitorando a qualidade do ar.

``A divisão está fazendo uma apuração detalhada dos dados das
estações de monitoramento da qualidade do ar distribuídas pela RMSP
Informaremos assim que tivermos um dado mais consolidado sobre a
ocorrência'', diz a nota.
\end{quote}

\fonte{Ingrid Oliveira e Carolina Figueiredo. Fumaça de queimadas do Amazonas chega a São Paulo; 
moradores relatam “cheiro forte”. Disponível em:
https://www.cnnbrasil.com.br/nacional/fumaca-de-queimadas-do-amazonas-chega-a-sao-paulo-moradores-relatam-cheiro-forte/.
Acesso em: 22 mai. 2023.}

Assinale a alternativa que contém a parte da notícia denominada linha fina.

\begin{escolha}

\item Fumaça de queimadas do Amazonas chega a São Paulo; moradores relatam ``cheiro forte''.

\item A divisão está fazendo uma apuração detalhada dos dados das estações de monitoramento.

\item Moradores da cidade de São Paulo relataram sentir cheiro de fumaça na manhã desta sexta-feira (9).

\item Segundo a Climatempo, a nuvem de fumaça foi gerada por incêndios em parte do Amazonas, Acre e Mato Grosso.

\end{escolha}

\coment{SAEB: Identificar elementos constitutivos de textos pertencentes ao
domínio jornalístico/midiático.

a) Incorreta. O trecho corresponde ao título da notícia.
b) Incorreta. O trecho pertence ao corpo da notícia.
c) Incorreta. O trecho pertence ao corpo da notícia.
d) Correta. A linha fina aparece como uma introdução ao assunto da notícia, logo depois do título.}

\num{3} Leia o texto abaixo para responder à questão.

\begin{quote}
Uma parceria entre a Universidade Federal do Rio de Janeiro e a
Universidade Federal do Estado do Rio de Janeiro (Unirio) vem estudando
o vírus-T linfotrópico humano do tipo 1, o HTLV-1, retrovírus da mesma
família do HIV, associado a casos de leucemia e doenças
neurodegenerativas. A pesquisa recebeu investimento da Organização
Pan-Americana de Saúde (Opas/OMS) junto com outros estudos
latino-americanos e caribenhos que investigam doenças negligenciadas.
Embora descrito, na década de 1980, como um vírus associado ao câncer, o
HTLV ainda é pouco estudado e bastante desconhecido do público em geral.
\end{quote}

\fonte{Carol Correia e Luana Reis.Conexão UFRJ. 
Pesquisa investiga vírus que pode levar à leucemia e neurodegeneração.
Disponível em: https://conexao.ufrj.br/2023/04/pesquisa-investiga-virus-negligenciado-que-pode-levar-a-doencas-neurodegenerativas-e-leucemia/.
Acesso em: 22 mai. 2023.}

O texto acima pertence ao gênero de divulgação científica, pois:

\begin{escolha}
  
  \item traz informações técnicas e de difícil compreensão.
  
  \item é escrito de maneira formal e dirigido a cientistas e estudiosos.
  
  \item traz informações sobre fatos do cotidiano.
  
  \item traz informações científicas em linguagem clara e acessível ao público
  geral.

\end{escolha}

\coment{SAEB: Identificar elementos constitutivos de gêneros de divulgação científica

a) Incorreta. O texto não apresenta linguagem técnica. 
b) Incorreta. O texto não é destinado apenas a cientistas.
c) Incorreta. O trecho não aborda fatos cotidianos. 
d) Correta. O texto faz uso de linguagem clara e acessível para divulgar estudos científicos
para o público em geral.}

\num{4} Leia o texto abaixo para responder à questão.

\begin{quote}

Ilma. Sra. Diretora da Escola Estadual Ulysses Guimarães:

Mariana Amaral Santos, aluna regularmente matriculada no sétimo ano do
Ensino Fundamental desta escola, vem respeitosamente solicitar à V.S.\textsuperscript{a} a
expedição dos documentos necessários à sua transferência para outro
estabelecimento de ensino.

Nestes termos, pede deferimento.

Curitiba, 24 de setembro de 2022.

\end{quote}

\fonte{Texto adaptado para este material.}

Os gêneros textuais reivindicatórios são textos que circulam no campo da
vida pública e possuem função social e estrutura específicas. O texto
acima é um modelo de Requerimento pois tem como objetivo:

\begin{escolha} 

\item convencer o destinatário a realizar uma ação importante.

\item solicitar formalmente ao destinatário a realização de uma ação.

\item mudar o comportamento do destinatário a partir da realização de ações.

\item informar formalmente o destinatário acerca da solicitação de ações.

\end{escolha}

\coment{SAEB: Identificar formas de organização de textos normativos, legais e/ou
reivindicatórios.

BNCC: EF67LP17 -- Analisar, a partir do contexto de produção, a forma de
organização das cartas de solicitação e de reclamação (datação, forma de
início, apresentação contextualizada do pedido ou da reclamação, em
geral, acompanhada de explicações, argumentos e/ou relatos do problema,
fórmula de finalização mais ou menos cordata, dependendo do tipo de
carta e subscrição) e algumas das marcas linguísticas relacionadas à
argumentação, explicação ou relato de fatos, como forma de possibilitar
a escrita fundamentada de cartas como essas ou de postagens em canais
próprios de reclamações e solicitações em situações que envolvam
questões relativas à escola, à comunidade ou a algum dos seus membros.


a) Incorreta. O texto não tem a pretensão de convencer o leitor.
b) Correta. O texto contém solicitação de transferência de instituição.
c) Incorreta. Não há referência sobre expctativa de mudança de atitude do leitor. 
d) Incorreta. O objetivo da carta não é informar, mas solicitar transferência de instituição.}

\num{5} Leia o texto abaixo para responder à questão.

\begin{quote}
\textbf{ATO I}

\textbf{Cena I}

\textit{Veneza. Uma rua. Entram Antônio, Salarino e Salânio.}

ANTÔNIO -- Não sei, realmente, porque estou tão triste. Isso me enfara; e
a vós também, dissestes. Mas como começou essa tristeza, de que modo a
adquiri, como me veio, onde nasceu, de que matéria é feita, ainda estou
por saber. E de tal modo obtuso ela me deixa, que mui dificilmente me
conheço.

SALARINO -- Vosso espírito voga em pleno oceano, onde vossos galeões de
altivas velas -- como burgueses ricos e senhores das ondas, ou qual vista
aparatosa distendida no mar -- olham por cima da multidão de humildes
traficantes que os saúdam, modestos, inclinando-se, quando perpassam com
tecidas asas.

\end{quote}

\fonte{William Shakespeare. O Mercador de Venza. 
Disponível em: http://www.dominiopublico.gov.br/download/texto/cv000094.pdf.
Acesso em: 22 mai. 2023.}

O trecho acima apresenta certas características, como discurso direto,
indicações de falas de personagens e rubricas, que são características 

\begin{escolha}

  \item das peças de teatro, textos escritos para encenação.

  \item dos poemas, compostos por versos e estrofes.

  \item dos contos, narrações curtas e de alto impacto.

  \item dos romances, narrativa longas e complexas.

\end{escolha}

\coment{SAEB: Analisar elementos constitutivos de textos pertencentes ao domínio literário.

a) Correta. O texto do exercício é parte de uma peça de teatro, com discurso direto, indicação
de personagens e rubricas.
b) Incorreta. O texto do exercício é parte de uma peça de teatro, com discurso direto, indicação
de personagens e rubricas.
c) Incorreta. O texto do exercício é parte de uma peça de teatro, com discurso direto, indicação
de personagens e rubricas.
d) Incorreta. O texto do exercício é parte de uma peça de teatro, com discurso direto, indicação
de personagens e rubricas.}

\num{6} Leia o texto abaixo para responder à questão. 

\begin{quote}

Rogério Ceni deixou o São Paulo. Este é só mais um capítulo na história
de um clube que caminha a passos largos rumo ao buraco. A verdade é uma
só e dói para o apaixonado torcedor admitir: aquele São Paulo que era
modelo de administração e que num intervalo de três anos ganhou
Paulista, Libertadores, Mundial de Clubes e o tricampeonato brasileiro
acabou. Virou apenas lembrança.

\end{quote}

\fonte{Marcelo Prado. G1. Opinião: vaidade e soberba de dirigentes afundam o São Paulo rumo a um futuro preocupante. Disponível em: https://ge.globo.com/futebol/times/sao-paulo/noticia/2023/04/20/opiniao-vaidade-e-soberba-de-dirigentes-afundam-o-sao-paulo-rumo-a-um-futuro-preocupante.ghtml.
Acesso em: 23 mai. 2023.}

O texto acima contém muitas opiniões e poucos fatos. Há um fato
expresso no texto em:

\begin{escolha}

  \item ``Este é só mais um capítulo na história de um clube''.
  
  \item ``A verdade é uma só e dói para o apaixonado torcedor''.
  
  \item ``um clube que caminha a passos largos rumo ao buraco''.
  
  \item ``O São Paulo num intervalo de três anos ganhou Paulista''.

\end{escolha}

\coment{Saeb: Distinguir fatos de opiniões em textos.
 
a) Incorreta. Este trecho representa a opinião do jornalista.
b) Incorreta. Este trecho representa a opinião do jornalista.
c) Incorreta. Este trecho representa a opinião do jornalista.
d) Correta. Este trecho apresenta um fato.}

\num{7} Leia o texto abaixo para responder à questão.

%Alterei a questão. Rogério, 24/5/23, 10h13
%\includegraphics[width=4.16667in,height=1.57292in]{./imgSAEB_7_POR/media/image17.png}
%\fonte{http://www.arionaurocartuns.com.br/search?q=moradia}{\uline{http://www.arionaurocartuns.com.br/search?q=moradia}}.
%Acesso em 20 de Abr de 2023.

\begin{quote}

É inútil descrever o quarto de um estudante: aí nada se
encontra de novo. Ao muito acharão uma estante, onde ele guarda
os seus livros, um cabide, onde pendura a casaca, o moringue, o
castiçal, a cama, uma até duas canastras de roupa, o chapéu, a
bengala e a bacia, a mesa, onde escreve e que só apresenta de
recomendável a gaveta cheia de papéis, de cartas de família, de
flores e fitinhas misteriosas: é pouco mais ou menos assim o
quarto de Augusto.

Agora ele está só. Às sete horas, desse quarto saíram três
amigos: Filipe, Leopoldo e Fabrício. Trataram da viagem no dia 
seguinte e retiraram-se descontentes.

\end{quote}

\fonte{Joaquim Manuel de Macedo. A moreninha. 
Disponível em: http://objdigital.bn.br/Acervo_Digital/Livros_eletronicos/a_moreninha.pdf.
Acesso em: 24 mai. 2023.}

Assinale a alternativa correta quanto aos trechos do texto.

\begin{escolha}
    
    \item ``\textbf{aí} nada se encontra de novo'': o termo destacado se refere a ``um estudante''.
    
    \item ``onde escreve'': o sentido da frase equivale a ``onde eles escrevem''.  
    
    \item ``o quarto de Augusto'': Augusto é o estudante cujo quarto é citado no início do parágrafo.  
    
    \item ``Agora \textbf{ele} está só'': o termo destacado refere-se a ``o quarto de Augusto''.   

\end{escolha}

\coment{SAEB: Inferir, em textos multissemióticos, efeitos de humor, ironia e/ou
crítica.

BNCC: EF69LP03 -- Inferir e justificar, em textos multissemióticos --
tirinhas, charges, memes, gifs etc. --, o efeito de humor, ironia e/ou
crítica pelo uso ambíguo de palavras, expressões ou imagens ambíguas, de
clichês, de recursos iconográficos, de pontuação etc.

a) Incorreta. No trecho ``\textbf{aí} nada se encontra de novo'', o termo destacado se refere a 
``o quarto de um estudante''.
b) Incorreta. No trecho ``onde escreve'': o sentido da frase não equivale a ``onde eles escrevem''.
Pela coesão e coerência do texto, nota-se que a frase equivale a ``onde ele (o estudante) escreve''.
c) Correta. Pela coesão e coerência do texto, verifica-se que, no trecho ``o quarto de Augusto'', 
Augusto é o estudante cujo quarto é citado no início do parágrafo.  
d) Incorreta. No trecho ``Agora \textbf{ele} está só'': o termo destacado refere-se a Augusto.}   

\num{8} Leia o texto abaixo para responder à questão. 

%Tirei a imangem abaixo e alterei a questão (Rogério, 24/5/23, 9h52)
%\includegraphics[width=5.90551in,height=1.72222in]{./imgSAEB_7_POR/media/image18.png}
%\fonte{https://tirasarmandinho.tumblr.com/search/amor}{\uline{https://tirasarmandinho.tumblr.com/search/amor}}.
%Acesso em 21 de Abr.

\begin{quote}

--- Não diga isso, Camilo. Se você soubesse como eu tenho andado, por sua
causa. Você sabe; já lhe disse. Não ria de mim, não ria \ldots{}

Camilo pegou-lhe nas mãos, e olhou para ela sério e fixo. Jurou que lhe
queria muito, que os seus sustos pareciam de criança; em todo o caso,
quando tivesse algum receio, a melhor cartomante era ele mesmo. Depois,
repreendeu-a; disse-lhe que era imprudente andar por essas casas.

\end{quote}

\fonte{Machado de Assis. A cartomante. Disponível em: 
http://www.dominiopublico.gov.br/download/texto/bv000257.pdf.
Acesso em: 23 mai. 2023.}

O diálogo acima pertence ao conto ``A cartomante'', de Machado de Assis.
No primeiro parágrafo, Rita lamenta ao amante Camilo o que vem 
sofrendo por ele. Assinale a alternativa correta quanto aos 
referentes dos pronomes destacados.  

\begin{escolha}

  \item ``\textbf{Você} sabe; já \textbf{lhe} disse'': os pronomes se referem a Rita.
  
  \item ``Camilo pegou-\textbf{lhe} nas mãos'': o pronome se refere às mãos de Camilo.
  
  \item ``Jurou que \textbf{lhe} queria muito'': o pronome se refere a Camilo.
  
  \item ``disse-\textbf{lhe} que era imprudente'': o pronome se refere a Rita.

\end{escolha}

\coment{SAEB: Analisar a intertextualidade entre textos literários ou entre
estes e outros textos verbais ou não verbais.

Bncc: EF67LP27 -- Analisar, entre os textos literários e entre estes e
outras manifestações artísticas (como cinema, teatro, música, artes
visuais e midiáticas), referências explícitas ou implícitas a outros
textos, quanto aos temas, personagens e recursos literários e semióticos

a) Incorreta. No trecho ``\textbf{Você} sabe; já \textbf{lhe} disse'', os pronomes se referem a Camilo.
b) Incorreta. No trecho ``Camilo pegou-\textbf{lhe} nas mãos'', o pronome se refere às mãos de Rita.
c) Incorreta. No trecho ``Jurou que \textbf{lhe} queria muito'', o pronome se refere a Rita.
d) Correta. No trecho ``disse-\textbf{lhe} que era imprudente'', o pronome se refere a Rita.}

\num{9} Leia o texto abaixo para responder à questão.

\begin{quote}

O Rio Doce, que nós, Krenak, chamamos de Watu, nosso avô, é uma pessoa,
não um recurso, como dizem os economistas.

\end{quote}

\fonte{Ailton Krenak. Ideias para adiar o fim do mundo. 2ª Ed. São Paulo:
Companhia da Letras, 2020, p.40.}

No texto acima, o autor

\begin{escolha}

  \item contou narrativas tradicionais da cultura indígena.

  \item comparou pontos de vista de indígenas e não indígenas.

  \item apresentou a língua falada pelos povo indígena Krenak.

  \item conciliou a cosmovisão de indígenas e não indígenas.

\end{escolha}

\coment{SAEB: Inferir a presença de valores sociais, culturais e humanos em
textos literários.

BNCC:EF69LP44 -- Inferir a presença de valores sociais, culturais e
humanos e de diferentes visões de mundo, em textos literários,
reconhecendo nesses textos formas de estabelecer múltiplos olhares sobre
as identidades, sociedades e culturas e considerando a autoria e o
contexto social e histórico de sua produção.
 
a) Incorreta. O trecho não contém narrtiva.
b) Correta. No trecho, Ailton Krenak explica que, para o povo Krenak, o 
Rio Doce é uma pessoa, ponto de vista bastante diferente dos não indígenas,
que veem o rio como recurso natural. 
c) Incorreta. Apesar de apresentar um vocábulo utilizado para designar o rio, 
Krenak apresenta sua língua no trecho. 
d) Incorreta. O trecho não contém uma conciliação de cosmovisões, mas a
oposição entre elas: indígenas e não indígenas veem o rio de maneiras 
bastante distintas.} 

\num{10} Leia o texto abaixo para responder à questão.

\begin{quote}

\textbf{Faxina correta da casa evita crise alérgica e possíveis problemas
respiratórios}

\textit{Pessoas alérgicas devem evitar acúmulo de objetos dentro de casa,
principalmente no quarto.}

Alérgica desde criança, a aposentada Glória Pordeus, 53 anos, viu os
problemas respiratórios serem agravados após se tornar portadora de
Lúpus, devido à baixa imunidade relacionada à doença. \textbf{Com isso,}
ela precisou redobrar os cuidados na hora de organizar a casa.
Especialistas alertam que pessoas alérgicas devem evitar acúmulo de
certos objetos, principalmente no quarto, e invés de usar vassoura para
limpar o chão, deve-se usar um pano úmido, e repassam dicas de como
manter o domicílio limpo sem comprometer a saúde.

\fonte{Jornal da Paraíba. Disponível em: https://jornaldaparaiba.com.br/bichos/faxina-correta-da-casa-evita-crise-alergica-e-possiveis-problemas-respiratorios/.
Acesso em: 22 mai. 2023.}

No texto acima, a expressão destacada se refere:

\begin{escolha}
  
  \item à limpeza redobrada do ambiente.
  
  \item ao agravamento dos problemas respiratórios.
  
  \item à orientação médica sobre os cuidados com as crianças.
  
  \item à idade da paciente que precisa de cuidados.

\end{escolha}

\coment{SAEB: Analisar os mecanismos que contribuem para a progressão textual.
a) Incorreta. O termo se refere ao agravamento dos problemas de saúde de 
Glória Pordeus. 
b) Correta. O termo se refere ao agravamento dos problemas de saúde de 
Glória Pordeus. 
c) Incorreta. O termo se refere ao agravamento dos problemas de saúde de 
Glória Pordeus. 
d) Incorreta. O termo se refere ao agravamento dos problemas de saúde de 
Glória Pordeus.} 

\num{11} Leia a notícia a seguir e responda à questão.

\begin{quote}

\textbf{Sedentários e grudados a uma tela. O
que mostra o maior estudo mundial sobre atividade física de
jovens}

Os especialistas já há algum tempo vêm alertando que os jovens não fazem todo o exercício físico que deveriam. Agora temos a confirmação: 80\% dos adolescentes (11 a 17 anos) em todo o mundo não fazem a atividade diária mínima para estarem saudáveis. E os especialistas não falam só de praticar esportes, e sim de ações básicas como caminhar até a escola ou jogar bola com os amigos no parque. Os padrões da Organização Mundial da Saúde (OMS) falam de uma hora diária de movimento. Estes dados ganham agora uma nova relevância se levarmos em conta a epidemia de obesidade que alcançou praticamente todos os países do mundo.

\end{quote}

\fonte{Patrícia Peiró. El País Brasil. Disponível em: https://brasil.elpais.com/brasil/2019/11/18/actualidad/1574086350_697117.html.
Acesso em: 23 mai. 2023.}

A frase que contém uso de figura de linguagem é:

\begin{escolha}
    
    \item ``Sedentários e grudados a uma tela''.
    
    \item ``não fazem todo o exercício físico que deveriam''.
    
    \item ``80\% dos adolescentes em todo o mundo não fazem a atividade''.
    
    \item ``se levarmos em conta a epidemia de obesidade''.

\end{escolha}

\coment{SAEB:Analisar o uso de figuras de linguagem como estratégia
argumentativa.

a) Correta. Na expressão ``grudados a uma tela'' ocorre metáfora.
b) Incorreta. Não há figura de linguagem na expressão destacada neste item.
c) Incorreta. Não há figura de linguagem na expressão destacada neste item.
d) Incorreta. Não há figura de linguagem na expressão destacada neste item.}

\num{12} Leia o texto abaixo para responder á questão.

\begin{quote}

Os especialistas consultados consideram que o mais previsível durante as
próximas semanas é ``um crescimento substancial das detecções da
ômicron, que já começa a ser percebido, até que a nova variante
substitua a delta, \textbf{algo que poderia ocorrer} em cerca de três
semanas'', explica Federico García, chefe de Microbiologia do Hospital
San Cecilio (Granada), um centro de referência para a Andaluzia
oriental.

\end{quote}

\fonte{Oriol Güell. El País Brasil. Agência de saúde pública da UE avisa
que a ômicron se propaga pelo continente mais rápido do que é detectada.
Disponível em: https://brasil.elpais.com/internacional/2021-12-13/agencia-de-saude-publica-da-ue-avisa-que-a-omicron-se-propaga-pelo-continente-mais-rapido-do-que-e-detectada.html.
Acesso em: 23 mai. 2023.}

Na expressão em destaque, a palavra ``algo'' se refere a:

\begin{escolha}
  
  \item acontecimentos inesperados.
  
  \item crescimento das detecções.
  
  \item substituição da variante ômicron pela delta.
  
  \item mortes por covid.

\end{escolha}

\coment{SAEB: Analisar os processos de referenciação lexical e pronominal.

a) Incorreta. ``Algo'' se refere à frase imediatamente anterior: ``até que a
nova variante substitua a delta''.
b) Incorreta. ``Algo'' se refere à frase imediatamente anterior: ``até que a
nova variante substitua a delta''.
c) Correta. O ``Algo'' se refere à frase imediatamente anterior: ``até que a
nova variante substitua a delta''.
d) Incorreta. ``Algo'' se refere à frase imediatamente anterior: ``até que a
nova variante substitua a delta''.}

\num{13} Leia o texto abaixo para responder à questão. 

\begin{quote}

\textbf{A prática da leitura é o grande desafio da escola e das famílias
para os alunos e filhos. É comum encontrarmos aqueles que não gostam de
ler --} mas isso só até descobrirem o prazer dessa atividade. Se as
crianças fossem estimuladas no lado prazeroso da leitura, teríamos mais
leitores e não apenas compradores de livros. \textbf{A literatura é um
dos principais meios de transmissão cultural e,} como nos mantém em
contato com a imaginação, estimula a nossa criatividade.

\end{quote}

\fonte{Ana Cássia Maturano. G1. Opinião: como fazer de seu filho um leitor. 
Disponível em: https://g1.globo.com/Noticias/Vestibular/0,,MUL723077-5604,00-OPINIAO+COMO+FAZER+DE+SEU+FILHO+UM+LEITOR.html.
Acesso em: 22 mai. 2023.}

Os trechos em destaque apresentam argumentação do tipo

\begin{escolha}

    \item Argumento de autoridade

    \item Argumento de exemplo

    \item Argumento de consenso

    \item Argumento por ilustração

\end{escolha}

\coment{SAEB: Avaliar a eficácia das estratégias argumentativas em textos de
diferentes gêneros.

a) Incorreta. Como não há citação da autoria das afirmações, não se 
pode afirmar que ocorra argumento de autoridade.
b) Incorreta. O trecho não contém exemplos.
c) Correta. O trecho contém argumentos que são unânimes.
d) Incorreta. Neste caso, não são citados exemplos como estratégia de
argumentação.}

\num{14} Observe a imagem a seguir para responder à questão.

\includegraphics[width=5.90551in,height=5.90278in]{./imgSAEB_7_POR/media/image19.png}

\fonte{Ulysses Gusmão de Oliveira e Luiz Carlos Pereira Borges.
Prefeitura de Jatái. Separe seu lixo de forma adequada. 
Disponível em: https://www.jatai.go.gov.br/separe-seu-lixo-de-forma-adequada/.
Acesso em: 23 mai. 2023.}

Na imagem acima, pode-se perceber diversos recursos para que a mensagem
seja transmitida de maneira eficiente. No caso da escolha dos
verbos utilizados, é correto afirmar que:

\begin{escolha}

    \item as formas verbais no imperativo afirmativo sugerem ações.

    \item as formas verbais no infinitivo expressam instruções.

    \item as formas verbais no subjuntivo representam advertências.

    \item as formas verbais no infinitivo indicam ordens.

\end{escolha}

\coment{SAEB: Analisar os efeitos de sentido dos tempos, modos e/ou
vozes verbais com base no gênero textual e na intenção comunicativa.
 
a) Correta. O modo imperativo tem, de forma geral, a função de expressar
ordem, sugestão ou instrução. No caso específico do cartaz, sugere mudanças
de atitude por parte do leitor.
b) Incorreta. As formas verbais do cartaz não estão no infinitivo.
c) Incorreta. As formas verbais do cartaz não estão flexionadas no modo subjuntivo.
d) Incorreta. As formas verbais do cartaz não estão no infinitivo.}

\num{15} Observe a imagem a seguir para responder à questão.

\includegraphics[width=5.90551in,height=3.15278in]{./imgSAEB_7_POR/media/image20.png}

\fonte{Tribunal Regional Eleitoral de São Paulo. 
Tuitaço #RolêdasEleições alcança trending topics no Twitter.
Disponível em: https://www.tre-sp.jus.br/comunicacao/noticias/2022/Marco/tuitaco-roledaseleicoes-alcanca-trending-topics-no-twitter.
Acesso em: 22 mai. 2023.}

Pode-se afirmar que o público-alvo da campanha acima é

\begin{escolha}

    \item o público em geral, devido ao uso da linguagem formal combinada com imagens vivazes.
    
    \item o público adulto, daí a neutralidade sóbria do texto e a austeridade das imagens. 
    
    \item o público jovem, que se identificará com a imagem e que usa redes sociais.  
    
    \item a público infantil, pelo chamado lúdico do texto e das imagens à participação política. 

\end{escolha}

\coment{Avaliar a adequação das variedades linguísticas em contextos de uso.

a) Incorreta. A campanha é voltada ao público jovem: na imagem, é retratada 
uma adolescente, com a qual esse público se identifica; o texto de 
chamada contém uma hashtag, o que sugere uso das redes sociais, que é mais
recorrente entre os jovens.  
b) Incorreta. A campanha é voltada ao público jovem: na imagem, é retratada 
uma adolescente, com a qual esse público se identifica; o texto de 
chamada contém uma hashtag, o que sugere uso das redes sociais, que é mais
recorrente entre os jovens.  
c) Correta. A campanha é voltada ao público jovem: na imagem, é retratada 
uma adolescente, com a qual esse público se identifica; o texto de 
chamada contém uma hashtag, o que sugere uso das redes sociais, que é mais
recorrente entre os jovens.  
d) Incorreta. A campanha é voltada ao público jovem: na imagem, é retratada 
uma adolescente, com a qual esse público se identifica; o texto de 
chamada contém uma hashtag, o que sugere uso das redes sociais, que é mais
recorrente entre os jovens.}

\addcontentsline{toc}{chapter}{Simulado 2}
\markboth{Simulado 2}{}

\num{1} Leia o texto abaixo para responder à questão.

\begin{quote}

No período entre julho e setembro, a ocorrência de queimadas se torna
mais frequente, por conta da estiagem e da baixa umidade relativa do ar.
A Especialista CNN em agronegócio Carmen Perez falou sobre as técnicas
empregadas nas fazendas para evitar que o fogo se alastre e cause
destruição.

Ela explicou que a seca traz dois principais alertas para produtores
rurais: a necessidade de garantir alimento aos animais e a atenção ao
risco de queimadas.``Cada fazenda tem diversas estratégias para coibir e
enfrentar incêndios'', explicou. Dentre elas, estão os aceiros, áreas
livres em que circulam parte da plantação para interromper a propagação do
fogo, reservatórios de água estratégicos e equipamentos específicos.

``Estamos no início da seca, mas já estamos alertas para evitar
possíveis riscos para a natureza, plantações e animais'', disse Perez.

\end{quote}

\fonte{Fernanda Pinotti. CNN Brasil. Carmen Perez: Fazendas têm diversas 
estratégias para coibir incêndios. Disponível em:
https://www.cnnbrasil.com.br/economia/carmen-perez-fazendas-tem-diversas-estrategias-para-coibir-incendios/
Acesso em: 23 mai. 2023.}

Segundo a especialista em agronegócio, qual fator requer maior atenção
dos produtores rurais no que diz respeito ao risco de queimadas?

\begin{escolha}
    
    \item As altas temperaturas.
    
    \item O período de estiagem.
    
    \item A alimentação dos animais.
    
    \item O descaso com questões ambientais.

\end{escolha}

\coment{SAEB: Identificar teses, opiniões, posicionamentos explícitos e
argumentos em textos.  
a) Incorreta. Não há alusão ao aumento da temperatura como fator de risco.
b) Correta. Segundo a especialista, o período de estiagem e a baixa
umidade do ar são fatores de alerta para risco de queimadas.
c) Incorreta. Segundo a especialista, a seca traz duas preocupações para 
os produtores rurais: a necessidade de garantir alimento aos animais e a 
atenção ao risco de queimadas. A alimentação dos animais não é, portanto,
fator de risco de queimadas: é \textit{outra preocupação} dos produtores. 
d)Incorreta. No texto, não há alusão ao descaso com questões ambientais.}
  
\num{2} Leia o artigo 5 da Constituição Federal, de 1988, da República
Federativa do Brasil.

\begin{quote}

Art. 5º Todos são iguais perante a lei, sem distinção de qualquer
natureza, garantindo-se aos brasileiros e aos estrangeiros residentes no
País, a inviolabilidade do direito à vida, à liberdade, à igualdade, à
segurança e à propriedade.

\end{quote}

\fonte{Presidência da República. Constituição da República Federativa do Brasil 
de 1988. Disponível em: http://www.planalto.gov.br/ccivil_03/constituicao/constituicao.htm.
Acesso em: 23 mai. 2023.}

A divisão em artigos caracteriza o trecho como:

\begin{escolha}

    \item O texto de uma lei.

    \item Uma carta de solicitação.

    \item Uma reivindicação.

    \item Uma petição.

\end{escolha}

\coment{SAEB: Identificar formas de organização de textos normativos, legais e/ou reivindicatórios.

BNCC: EF69LP27 -- Analisar a forma composicional de textos pertencentes a
gêneros normativos/ jurídicos e a gêneros da esfera política, tais como
propostas, programas políticos (posicionamento quanto a diferentes ações
a serem propostas, objetivos, ações previstas etc.), propaganda política
(propostas e sua sustentação, posicionamento quanto a temas em
discussão) e textos reivindicatórios: cartas de reclamação, petição
(proposta, suas justificativas e ações a serem adotadas) e suas marcas
linguísticas, de forma a incrementar a compreensão de textos
pertencentes a esses gêneros e a possibilitar a produção de textos mais
adequados e/ou fundamentados quando isso for requerido

a) Correta. A divisão em artigos e parágrafos é uma característica
composicional de textos do gênero normativo/jurídico.
b) Incorreta. A divisão e forma composicional do trecho não correspondem
a cartas de solicitação.
c) Incorreta. A divisão e forma composicional do trecho não correspondem
a uma reivindicação.
d) Incorreta. A divisão e forma composicional do trecho não correspondem
a uma petição.} 

\num{3} Linguagem direta e curta, envolvendo poucos personagens, em espaço
delimitado e poucos conflitos e ações de personagens. Um texto que se
compõe de uma situação inicial, um conflito, clímax e desfecho. Estas
características estão presentes em:

\begin{escolha}

    \item Romance

    \item Contos

    \item Cartas pessoais

    \item Crônicas

\end{escolha}

\coment{SAEB: Analisar elementos constitutivos de textos pertencentes ao domínio
literário.

Bncc: EF69LP47 Analisar, em textos narrativos ficcionais, as diferentes
formas de composição próprias de cada gênero, os recursos coesivos que
constroem a passagem do tempo e articulam suas partes, a escolha lexical
típica de cada gênero para a caracterização dos cenários e dos
personagens e os efeitos de sentido decorrentes dos tempos verbais, dos
tipos de discurso, dos verbos de enunciação e das variedades
linguísticas (no discurso direto, se houver) empregados, identificando o
enredo e o foco narrativo e percebendo como se estrutura a narrativa nos
diferentes gêneros e os efeitos de sentido decorrentes do foco narrativo
típico de cada gênero, da caracterização dos espaços físico e
psicológico e dos tempos cronológico e psicológico, das diferentes vozes
no texto (do narrador, de personagens em discurso direto e indireto), do
uso de pontuação expressiva, palavras e expressões conotativas e
processos figurativos e do uso de recursos linguístico-gramaticais
próprios a cada gênero narrativo.

a) Incorreta. Os romances, embora apresentem algumas das
características elencadas no enunciado, em geral, são textos longos, 
com vários personagens que desenvolvem suas ações em vários cenários 
e conflitos.
b) Correta. As características elencadas no enunciado descrevem os 
contos.
c) Incorreta. Cartas pessoais possuem outras formas composicionais, tais
como saudação, corpo do texto e despedida.
d) Incorreta. As crônicas em geral possuem outras características tais
como temas do cotidiano e efeitos de humor.} 

\num{4} Leia o texto abaixo para responder à questão.

\begin{quote}

Uma pesquisa realizada pelo Instituto de Estudos em Saúde Coletiva
(Iesc) e pela Faculdade de Medicina (FM), atualmente em revisão na
revista Scientific Reports, indicou que os programas sociais de
transferência de renda foram essenciais durante o período crítico da
covid-19. A investigação também ressaltou que a população negra teve um
maior índice de mortalidade no mesmo recorte temporal.

O estudo, que analisou dados de contágio da doença colhidos entre março
de 2020 e setembro de 2021 em todo o Brasil, detectou uma relação
inversa entre as taxas de mortalidade e infecção e o número de pessoas
de uma mesma família que eram beneficiárias de algum dos programas
governamentais.

\end{quote}

\fonte{Carol Correia. Conexão UFRJ. Programas sociais foram fundamentais durante fase crítica da covid-19.
https://conexao.ufrj.br/2023/03/programas-sociais-foram-fundamentais-durante-fase-critica-da-covid-19/.
Acesso em: 24 mai 2023.}

No texto acima, a linguagem objetiva e a apresentação de dados de pesquisa são traços do 
gênero:

\begin{escolha}

    \item artístico-literário.

    \item de divulgação científica.

    \item jornalístico midiático.

    \item jurídico.

\end{escolha}

\coment{SAEB: Identificar elementos constitutivos de gêneros de divulgação
científica.

a) Incorreta. A linguagem objetiva e a apresentação de dados de pesquisa são 
traços do gênero de divulgação científica.
b) Correta.  Incorreta. A linguagem objetiva e a apresentação de dados de pesquisa são 
traços do gênero de divulgação científica.
c) Incorreta. A linguagem objetiva e a apresentação de dados de pesquisa são 
traços do gênero de divulgação científica.
d) Incorreta. A linguagem objetiva e a apresentação de dados de pesquisa são 
traços do gênero de divulgação científica.}

\num{5} Leia o texto abaixo para responder à questão. 

\begin{quote}

O livro \textit{A queda do Céu}, de Davi Kopenawa e Bruce Albert, é um modelo
inovador de produção textual, que combina a auto-etnografia de uma
cultura, o manifesto político das culturas tradicionais, os relatos de
vidas não ocidentais e uma visão cosmológica e espiritual do mundo,
quase extinta na sociedade moderna.

A obra retrata a vida do narrador (Kopenawa), desde sua iniciação
religiosa até alcançar o ápice como líder Yanomami.

Segundo sua cultura e tradições, os Yanomamis são os responsáveis por
assegurar que o céu não caia. Kopenawa, xamã da tribo, cita diversos
momentos em sua vida que exemplificam essas situações, onde ele ou os
antigos xamãs mobilizaram os espíritos para que a Floresta permanecesse
em equilíbrio.

\end{quote}

\fonte{Akil Alexandre Costa Silvério da Silva. Resenha do texto: \textit{A queda do Céu}: Davi Kopenawa e
Bruce Albert. Disponível em: https://edisciplinas.usp.br/pluginfile.php/3395103/mod_resource/content/1/T6\%20aperfei\%C3\%A7oado.pdf.
Acesso em: 24 mai. 2023.}

No texto acima, notam-se características da

\begin{escolha}

    \item crônica (temas do cotidiano de forma crítica e bem humorada).

    \item resenha crítica (características e explicações de outra obra).

    \item notícia (informações ao leitor sobre fatos ocorridos).

    \item peça de teatro (texto escrito para ser encenado). 

\end{escolha}

\coment{SAEB: Analisar elementos constitutivos de textos pertencentes ao 
domínioliterário.

a) Incorreta. O texto analisado apresenta características e explicações de outra obra.
b) Correta. O texto analisado apresenta características e explicações de outra obra.
c) Incorreta. O texto analisado apresenta características e explicações de outra obra.
d) Incorreta. O texto analisado apresenta características e explicações de outra obra.}

\num{6} Leia o texto abaixo para responder à questão. 

\begin{quote}

Na última década, o diabetes cresceu 54\% nos homens e 28,5\% nas
mulheres. Outra doença que tem crescido entre os brasileiros e que está
relacionada com o alto consumo de açúcar é a obesidade, que atinge
mais de 25\% da população adulta do país.

\end{quote}

\fonte{Ministério da Saúde. Saúde promove conscientização sobre o consumo de açúcar em webinário. 
Disponível em: https://aps.saude.gov.br/noticia/15359\#:~:text=Os\%20brasileiros\%20consomem\%2050\%25\%20a,adulto\%20\%C3\%A9\%20de\%2012\%20colheres.
Acesso em: 24 mai. 2023.}

De acordo com as informações acima, o consumo de açúcar pode ser
responsável pelos altos índices de obesidade. Além da obesidade, qual
outra doença pode estar relacionada ao consumo de açúcar? Assinale a
alternativa correta:

\begin{escolha}

  \item Diabetes.
  
  \item doenças crônicas não transmissíveis.
  
  \item doenças comuns em homens.
  
  \item doenças crônicas em mulheres. 

\end{escolha}

\coment{SAEB: Inferir informações implícitas em distintos textos.

a) Correta. O texto contém referências a apenas duas doenças relacionadas 
ao consumo de açúcar: diabetes e obesidade.
b) Incorreta. O texto contém referências a apenas duas doenças relacionadas 
ao consumo de açúcar: diabetes e obesidade.
c) Incorreta. O texto contém referências a apenas duas doenças relacionadas 
ao consumo de açúcar: diabetes e obesidade.
d) Incorreta. O texto contém referências a apenas duas doenças relacionadas 
ao consumo de açúcar: diabetes e obesidade.}


\num{7} Observe o meme abaixo para responder à questão.

\includegraphics[width=4.16667in,height=3.03125in]{media/image21.png}

O efeito de sentido produzido pela imagem é bem claro. Assinale
alternativa que apresenta tais efeitos de sentido:

\begin{escolha}

    \item crítica ao desmatamento por meio de oposição.

    \item divulgação de informações por meio de imagem.

    \item persuasão por meio de frases injuntivas. 

    \item sensibilização por meio de frases motivacionais. 

\end{escolha}

\coment{SAEB: Inferir, em textos multissemiótico, efeitos de humor, ironia e/ou
crítica.

Bncc: EF69LP05 -- Inferir e justificar, em textos multissemióticos -- tirinhas, charges,
memes, gifs etc. --, o efeito de humor, ironia e/ou crítica pelo uso
ambíguo de palavras, expressões ou imagens ambíguas, de clichês, de
recursos iconográficos, de pontuação etc.

a) Correta. A crítica ao desmatamento se dá por meio da oposição entre o
desolamento causado pela destruição da natureza e o tom entusiástico da
propaganda do empreendimento imobiliário. 
b) Incorreta. O meme não contém divulgação de informações. 
c) Incorreta. O meme não contém frases injuntivas, isto é, imperativas.
d) Incorreta. O meme não contém frases motivacionais.} 

\num{8} Leia o texto abaixo para responder à questão. 

\begin{quote}

A teórica Kirin Narayan questiona esse dualismo e foca na qualidade das
relações que mantemos com as pessoas que buscamos representar em nossos
textos. Nesse sentido, a autora propõe desconstruir a ideia de que os
interlocutores poderiam ser classificados como meros indivíduos para
declarações acerca de um \textbf{``outro generalizado}'', impulsionando
assim um entendimento desses sujeitos como portadores de
\textbf{``vozes, perspectivas e dilemas''}.

\end{quote}

\fonte{Leonardo Bomfim. Desafios e potencialidades do ``pesquisador nativo'':
perspectivas etnográficas em um festival musical.
Disponível em: https://econtents.bc.unicamp.br/inpec/index.php/muspop/article/view/17627/12474.
Acesso em: 24 mai. 2023.}

Os termos destacados que aparecem entre aspas representam:

    \item as falas do autor sobre o tema.

    \item ideias que devem ser reforçadas.

    \item partes principais do argumento.

    \item citação das palavras de Kirin Naryan.

\end{escolha}

\coment{SAEB: Analisar efeitos de sentido produzido pelo uso de formas de
apropriação textual (paráfrase, citação etc.)

a) Incorreta. Os trechos entre aspas indicam citações das palavras da teórica Kirin Narayan.
b) Incorreta. Os trechos entre aspas indicam citações das palavras da teórica Kirin Narayan.
c) Incorreta. Os trechos entre aspas indicam citações das palavras da teórica Kirin Narayan.
d) Correta. Os trechos entre aspas indicam citações das palavras da teórica Kirin Narayan.}

\num{9} Leia o texto abaixo para responder à questão. 

\begin{quote}

Falando sobre o jogo Minecraft, Pedro Rezende se consolidou como o
maior YouTuber de games do Brasil. ``A ideia inicial começou quando eu
precisava de ajuda para passar de uma fase em um jogo e procurei na
internet''. A fala é de um dos maiores youtubers da atualidade no Brasil. O jovem
paranaense Pedro Rezende (19) mudou sua vida da água para o vinho depois
que abandonou a profissão de goleiro em um time de futebol na Itália
para seguir a carreira de youtuber de games por aqui.

\end{quote}

\fonte{Jadson Falcão. A União. Geração de youtubers faz sucesso entre o público jovem. 
Disponível em: https://auniao.pb.gov.br/noticias/caderno_diversidade/geracao-de-youtubers-faz-sucesso-entre-os-jovens-do-pais.
Acesso em: 24 mai. 2023.}

No contexto em que se insere, a expressão ``mudou sua vida da água para o vinho''
significa que 

\begin{escolha}

    \item a vida do garoto era miserável.

    \item o garoto trocou água por vinho.

    \item a vida do garoto continua a mesma.

    \item a vida do garoto mudou drasticamente.

\end{escolha}

\coment{SAEB: Analisar o uso de figuras de linguagem como estratégia
argumentativa.

a) Incorreta. No contexto em que se insere, a expressão significa que a vida 
do garoto mudou radicalmente. Não há elementos que indiquem que ele levava
uma vida miserável. 
b) Incorreta. No contexto em que se insere, a expressão significa que a vida 
do garoto mudou radicalmente.
c) Incorreta. No contexto em que se insere, a expressão significa que a vida 
do garoto mudou radicalmente.
d) Correta. No contexto em que se insere, a expressão significa que a vida 
do garoto mudou radicalmente.}

\num{10} Leia o texto abaixo para responder à questão.

\begin{quote}

\textbf{"Ariel, a Pequena Sereia" é opção de teatro para as crianças neste 
final de semana}

A programação cultural deste final de semana traz para o público infantil 
a peça teatral "Ariel, a Pequena Sereia". O espetáculo ganha sessões no Teatro 
Barreto Júnior, neste sábado (22) e domingo (23), às 16h30.

\end{quote}

\fonte{Folha de Pernambuco. 
Disponível em:
https://www.folhape.com.br/cultura/ariel-a-pequena-sereia-e-opcao-de-teatro-para-as-criancas-neste/267280/.
Acesso em: 24 mai. 2023}

Na manchete e no corpo do texto as aspas foram usadas para

\begin{escolha}
    
    \item citar a fala de um entrevistado.
    
    \item citar o nome de uma obra,
    
    \item citar uma gíria.
    
    \item dar ênfase ao termo.

\end{escolha}

\coment{SAEB: Analisar os efeitos de sentido decorrentes dos mecanismos de construção
de textos jornalísticos/midiáticos.

a) Incorreta. Na manchete e no corpo do texto as aspas foram usadas para citar o nome da peça.
b) Correta. Na manchete e no corpo do texto as aspas foram usadas para citar o nome da peça.
c) Incorreta. Na manchete e no corpo do texto as aspas foram usadas para citar o nome da peça.
d) Incorreta. Na manchete e no corpo do texto as aspas foram usadas para citar o nome da peça.}

\num{11} Leia os textos abaixo para responder à questão. 

\begin{quote}

\textbf{Texto 1}

Às 14h, ocorreu o acidente. Um desmoronamento interno fez com que uma
enorme rocha se desprendesse da montanha e caísse sobre o túnel,
fechando completamente a ligação com a superfície. Dentro da mina,
ficaram 33 mineiros -- 32 chilenos e 1 boliviano.

\end{quote}

\fonte{Rogério Simões. BBC News Brasil. 
Mineiros do Chile: a incrível e dramática saga acompanhada pelo mundo ao vivo na TV.
Disponível em: https://www.bbc.com/portuguese/internacional-55926799.
Acesso em: 24 mai. 2023.}

\textbf{Texto 2}

\begin{quote}

No dia 5 de agosto, um desmoronamento deixou 33 operários presos na mina
de San José, situada no deserto do Atacama, no Chile. Eles ficaram
incomunicáveis, a 700 metros de profundidade, durante 17 dias, até serem
descobertos pelas equipes de sondagem.

\end{quote}

\fonte{José Renato Salatiel. Mineiros do Chile: O resgate que emocionou o mundo.
Disponível em: https://vestibular.uol.com.br/resumo-das-disciplinas/atualidades/mineiros-do-chile-o-resgate-que-emocionou-o-mundo.htm.
Acesso em: 24 mai. 2023.}

Assinale a única afirmação comum aos dois textos. 

\begin{escolha}

    \item O acidente ocorreu às 14 horas.

    \item O acidente deixou 33 operários presos.

    \item Os mineiros ficaram presos por 17 dias.

    \item O acidente atingiu 32 chilenos e 1 boliviano.

\end{escolha}

\coment{Avaliar a fidedignidade de informações sobre um mesmo fato divulgado em
diferentes veículos e mídias.

a) Incorreta. A informação de que o acidente deixou 33 operários presos é a única comum aos dois textos.
b) Correta. A informação de que o acidente deixou 33 operários presos é a única comum aos dois textos.
c) Incorreta. A informação de que o acidente deixou 33 operários presos é a única comum aos dois textos.
d) Incorreta. A informação de que o acidente deixou 33 operários presos é a única comum aos dois textos.}

\num{12} Leia o texto abaixo para responder à questão. 

\begin{quote}

\textbf{Nordeste poderia crescer mais que o Brasil até 2030}

Combinar aumento da produtividade com redução das desigualdades seria a
melhor alternativa para elevar o PIB per capita na região.

\fonte{Instituto de Pesquisa Econômica Aplicada. Nordeste poderia crescer mais que o Brasil até 2030. 
Disponível em: 
https://www.ipea.gov.br/portal/categorias/45-todas-as-noticias/noticias/9506-nordeste-poderia-crescer-mais-que-o-brasil-ate-2030?highlight=WyJub3JkZXN0ZSIsIm5vcmRlc3RlJy4iXQ==.
Acesso em: 24 mai. 2023.}

De acordo com o texto, o Nordeste

\begin{escolha}
    
    \item vai crescer mais que o Brasil até 2030 e aumentar a produtividade para diminuir a desigualdade.
    
    \item só crescerá mais que o Brasil se combinar aumento da produtividade e redução da desigualdade. 
    
    \item já reduziu a desigualdade social e tem alcançado elevação proporcional do PIB.
    
    \item cresceu sensivelmente em relação ao Brasil, de modo que já conseguiu elevar o PIB.

\end{escolha}

\coment{SAEB: Identificar os recursos de modalização em textos diversos.

a) Incorreta. De acordo com o texto, o Nordeste crescerá mais do que o Brasil se combinar aumento
da produtividade com redução da desigualdade. 
b) Correta. De acordo com o texto, o Nordeste crescerá mais do que o Brasil se combinar aumento
da produtividade com redução da desigualdade. 
c) Incorreta. De acordo com o texto, o Nordeste crescerá mais do que o Brasil se combinar aumento
da produtividade com redução da desigualdade. 
d) Incorreta. De acordo com o texto, o Nordeste crescerá mais do que o Brasil se combinar aumento
da produtividade com redução da desigualdade.}

\num{13} Leia o texto abaixo para responder à questão. 

\begin{quote}

Sua história tem pouca coisa de notável. Fora Leonardo
algibebe em Lisboa, sua pátria; aborrecera-se porém do negócio,
e viera ao Brasil. Aqui chegando, não se sabe por proteção de
quem, alcançou o emprego de que o vemos empossado, e que
exercia, como dissemos, desde tempos remotos. Mas viera com
ele no mesmo navio, não sei fazer o quê, uma certa Maria da
Hortaliça, quitandeira das praças de Lisboa, saloia rechonchuda e
bonitota. O Leonardo, fazendo-se-lhe justiça, não era nesse tempo
de sua mocidade mal apessoado, e sobretudo era maganão. Ao
sair do Tejo, estando a Maria encostada à borda do navio, o
Leonardo fingiu que passava distraído por junto dela, e com o
ferrado sapatão assentou-lhe uma valente pisadela no pé direito.
A Maria, como se já esperasse por aquilo, sorriu-se como
envergonhada do gracejo, e deu-lhe também em ar de disfarce um
tremendo beliscão nas costas da mão esquerda. Era isto uma
declaração em forma, segundo os usos da terra: levaram o resto
do dia de namoro cerrado; ao anoitecer passou-se a mesma cena
de pisadela e beliscão, com a diferença de serem desta vez um
pouco mais fortes; e no dia seguinte estavam os dois amantes tão
eXtremosos e familiares, que pareciam sê-lo de muitos anos.

\end{quote}

\fonte{Manuel Antônio de Almeida. Memórias de um sargento de milícias.
Disponível em: https://digital.bbm.usp.br/handle/bbm/1601.
Acesso em: 24 mai. 2023.}

Assinale a alternativa correta no que se refere à coesão e à coerência 
do texto. 

\begin{escolha}

\item
  Em ``\textbf{Sua história} tem pouca coisa de notável'', o termo destacado
  se refere à história de Maria da Hortaliça.
\item
  Em ``assentou-\textbf{lhe} uma valente pisadela no pé direito'', o termo destacado
  se refere ao pé de Maria da Hortaliça.
\item
  Em ``deu-\textbf{lhe} também em ar de disfarce um tremendo beliscão'', o termo 
  destacado se refere ao pé de Maria da Hortaliça.
\item
  A expressão ``um pouco mais fortes'' refere-se à condição física de Leonardo e 
  Maria da Hortaliça.

\end{escolha}

\coment{SAEB: Analisar os processos de referenciação lexical e pronominal.

a) Incorreta.   Em ``\textbf{Sua história} tem pouca coisa de notável'', o termo destacado 
se refere à história de Leonardo.
a) Correta.   Em ``assentou-\textbf{lhe} uma valente pisadela no pé direito'', o termo destacado 
se refere ao pé de Maria da Hortaliça.
a) Incorreta.   Em ``deu-\textbf{lhe} também em ar de disfarce um tremendo beliscão'', o termo  
destacado se refere às costas da mão de Leonardo.
a) Incorreta.   A expressão ``um pouco mais fortes'' refere-se às ações de Leonardo e  
Maria da Hortaliça.}

\num{14} Leia o texto abaixo para responder à questão. 

\begin{quote}

A vacina salva vidas. Doenças que causavam milhares de vítimas no
passado, como varíola e poliomielite, foram erradicadas. Outras doenças
transmissíveis também deixaram de ser problema de saúde pública porque
foram eliminadas no Brasil e nas Américas, como o sarampo, rubéola e
rubéola congênita.

\end{quote}

\fonte{Ministério da Saúde. Programa Nacional de Imunizações -- Vacinação. Disponível em: https://www.gov.br/saude/pt-br/acesso-a-informacao/acoes-e-programas/programa-nacional-de-imunizacoes-vacinacao. Acesso em: 24 mai. 2023.}

Qual das seguintes opções melhor descreve a estratégia argumentativa
utilizada para enfatizar a importância da vacinação?

\begin{escolha}

    \item Emoção.
    
    \item Generalização.
    
    \item Lógica.
    
    \item Autoridade.

\end{escolha}

\coment{SAEB: Avaliar a eficácia das estratégias argumentativas em textos de
diferentes gêneros.

a) Incorreta. O autor não apela diretamente para as emoções do leitor,
mas para argumentos lógicos e fatos concretos.
b) Incorreta. O autor apresenta exemplos concretos de doenças
erradicadas e reduzidas graças à vacinação, sem generalizações.
c) Correta. O autor usa uma abordagem lógica para argumentar a favor da
vacinação, apresentando exemplos de doenças erradicadas e reduzidas
graças à vacinação, argumentando que a vacinação é estratégia mais
eficaz para prevenir doenças e salvar vidas.
d) Incorreta. O autor não se apoia em afirmações de especialistas para
argumentar a favor da vacinação.}

\num{15} Observe a imagem abaixo para responder à questão. 

\begin{quote}

\includegraphics[width=5.90551in,height=3.18056in]{media/image22.png}

\fonte{https://www.tre-sp.jus.br/comunicacao/noticias/2022/Junho/campanha-201ctamo-junto201d-do-tre-sp-quer-incentivar-voto-consciente-dos-jovens.
Acesso em: 24 mai. 2023.}

A imagem acima faz parte de uma campanha incentivando a votação. Ao analisar
as escolhas de linguagem utilizadas podemos afirmar que a campanha se dirige: 

\begin{escolha}

    \item ao público idoso.

    \item ao público infantil.

    \item a mulheres.

    \item ao público jovem.

\end{escolha}

\coment{SAEB: Avaliar a adequação das variedades linguísticas em contextos de uso

a) Incorreta. O uso de recursos utilizados em redes sociais, como hashtags, 
não é o mais indicado ao público idoso.
b) Incorreta. O uso de gírias, símbolos e siglas de internet não é eficaz 
para o público infantil.
c) Incorreta. Não há nada na imagem que remeta ao público feminino.
d) Correta. O uso de gírias e hashtags utilizados no meio
digital é recurso eficaz para atingir o público jovem.}

\addcontentsline{toc}{chapter}{Simulado 3}
\markboth{Simulado 3}{}

\num{1} Leia o texto abaixo para responder à questão. 

\begin{quote}

\textbf{Alimentos que prejudicam o meio ambiente são também os piores à
saúde}

\textit{Comer cereais, frutas, verduras, batatas e azeite de oliva protege o
planeta e previne doenças}

Aproximadamente 60\% dos fatores de risco responsáveis por todas as
doenças são o resultado de uma dieta de má qualidade. Esse fato está
ligado à saúde do planeta. Um estudo publicado na revista PNAS demonstra
que os alimentos mais prejudiciais ao ser humano também o são para a
Terra.

Os pesquisadores analisaram 15 alimentos que fazem parte da dieta diária
ocidental. Ligaram a maneira como são produzidos (a água que se gasta, a
superfície implicada e os produtos químicos utilizados, entre outros)
aos resultados de estudos anteriores sobre o impacto desses mesmos
alimentos sobre a saúde. E tudo se encaixava. As frutas, verduras, a
batata, o azeite de oliva, os legumes, as frutas secas e os cereais são
os alimentos mais saudáveis e que, além disso, têm impacto mínimo sobre
o planeta.

A carne vermelha processada e não processada, por outro lado, é um
produto que deveria sair da lista de compras.

O peixe é um dilema. É uma opção saudável, como a maioria das pessoas
sabe, mas tem uma pegada ambiental maior, ao lado do frango e dos
laticínios, do que as dietas baseadas em vegetais, de acordo com os
resultados do estudo. Basulto afirma que um produto é benéfico quando
impede o consumidor de comer alimentos mais prejudiciais para sua saúde.
``Se o cliente come peixe e não consome carne vermelha, portanto, é bom
para ele e para o planeta'', acrescenta.

\fonte{Agathe Cortes. El País Brasil. 
Disponível em: https://brasil.elpais.com/brasil/2019/10/29/ciencia/1572344750_688431.html.
Acesso em: 22 mai. 2023.}

Segundo a reportagem, assinale a alternativa que contém os alimentos
mais saudáveis e que menos impactam o planeta:

\begin{escolha}
  
    \item cereais, ovos, legumes.
  
    \item frutas, verduras, cereais.
  
    \item frutas, verduras, frango.
  
    \item frutas secas, azeite, carnes.

\end{escolha}

\coment{SAEB: Identificar teses, opiniões, posicionamentos explícitos e
argumentos em textos.

a) Incorreta. A reportagem não cita os ovos.
b)Correta. Estes estão entre os alimentos mais saudáveis e de menor
impacto junto com azeite e frutas secas.
c) Incorreta. Segundo o texto, o consumo de frango tem ``pegada ambiental
maior''.
d) Incorreta. As carnes, segundo a reportagem, deveriam ser retiradas
da dieta.}

\num{2} Leia o texto abaixo para responder à questão. 

\begin{quote}

No caso, a quantidade de comida industrializada ingerida parece
influenciar no aparecimento de doenças e na quantidade de mortes
prematuras. Esses estudos, no entanto, não conseguem demonstrar qual
seria o mecanismo por trás dessa aparente correlação.

A geriatra Claudia Suemoto, da Faculdade de Medicina da USP, que
coordenou o estudo do Elsa sobre ultraprocessados e desempenho
cognitivo, espera superar essa limitação em breve. Serão feitas imagens
do cérebro de voluntários para ver se o alto consumo de ultraprocessados
pode causar eventos isquêmicos ou pequenos derrames cerebrais, que, ao
longo do tempo, poderiam comprometer as funções cognitivas. ``Dessa
forma, poderemos investigar possíveis mecanismos que expliquem a
associação do ponto de vista estrutural'', conta Suemoto.

\end{quote}

Segundo o texto, a limitação a ser superada pela pesquisa é:

\begin{escolha}

    \item diminuir a quantidade de ingestão de ultraprocessados para evitar derrames e melhorar o desempenho cognitivo.

    \item impossibilidade de demonstrar o mecanismo por trás da relação entre o consumo de ultraprocessados e as mortes prematuras.
  
    \item conseguir imagens do cérebro de voluntários que sofreram derrame cerebral causado pelo consumo excessivo de ultraprocessados.
  
    \item estudar as funções cognitivas e os eventos isquêmicos de quem sofreu derrame cerebral causado pelo consumo excessivo de ultraprocessados.

\end{escolha}

\coment{SAEB: Identificar elementos constitutivos de gêneros de divulgação científica.

a) Incorreta. Como se observa no primeiro parágrafo, impossibilidade de demonstrar o mecanismo por trás da relação entre o consumo de ultraprocessados e as mortes prematuras.
b) Correta. Como se observa no primeiro parágrafo, impossibilidade de demonstrar o mecanismo por trás da relação entre o consumo de ultraprocessados e as mortes prematuras.
c) Incorreta. Como se observa no primeiro parágrafo, impossibilidade de demonstrar o mecanismo por trás da relação entre o consumo de ultraprocessados e as mortes prematuras.
d) Incorreta. Como se observa no primeiro parágrafo, impossibilidade de demonstrar o mecanismo por trás da relação entre o consumo de ultraprocessados e as mortes prematuras.}

\num{3} Leia o texto abaixo para responder à questão. 

\begin{quote}

\textbf{Título I}

Das Disposições Preliminares

Art. 1º Esta Lei dispõe sobre a proteção integral à criança e ao
adolescente.

Art. 2º Considera-se criança, para os efeitos desta Lei, a pessoa até
doze anos de idade incompletos, e adolescente aquela entre doze e
dezoito anos de idade.

Parágrafo único. Nos casos expressos em lei, aplica-se excepcionalmente
este Estatuto às pessoas entre dezoito e vinte e um anos de idade.

\end{quote}

\fonte{Presidência da República. 
Disponível em: https://www.planalto.gov.br/ccivil_03/leis/l8069.htm.
Acesso em: 24 mai. 2023.}

Estão presentes no texto acima e são características de textos legais ou
jurídicos

\begin{escolha}

\item linguagem impessoal e organização em títulos, capítulos e sessões.
\item uso de linguagem informal e organização em parágrafos e estrofes.
\item linguagem rebuscada e organização em títulos e subtítulos.
\item uso da primeira pessoa e organização em livre.
\end{escolha}

\coment{SAEB:  Identificar formas de organização de textos normativos, legais
e/ou reivindicatórios.

BNCC: EF69LP20 -- Identificar, tendo em vista o contexto de produção, a
forma de organização dos textos normativos e legais, a lógica de
hierarquização de seus itens e subitens e suas partes: parte inicial
(título -- nome e data -- e ementa), blocos de artigos (parte, livro,
capítulo, seção, subseção), artigos (caput e parágrafos e incisos) e
parte final (disposições pertinentes à sua implementação) e analisar
efeitos de sentido causados pelo uso de vocabulário técnico, pelo uso do
imperativo, de palavras e expressões que indicam circunstâncias, como
advérbios e locuções adverbiais, de palavras que indicam generalidade,
como alguns pronomes indefinidos, de forma a poder compreender o caráter
imperativo, coercitivo e generalista das leis e de outras formas de
regulamentação.

a) Incorreta. O texto se caracteriza pela linguagem impessoal e organização em 
títulos, capítulos e sessões.
b) Correta. O texto se caracteriza pela linguagem impessoal e organização em 
títulos, capítulos e sessões. 
c) Incorreta. O texto se caracteriza pela linguagem impessoal e organização em 
títulos, capítulos e sessões. 
d) Incorreta. O texto se caracteriza pela linguagem impessoal e organização em 
títulos, capítulos e sessões.} 

\num{4} Leia o texto abaixo para responder à questão. 

\begin{quote}

\textbf{Faciap promove encontros para jovens empreendedores em Curitiba}

Entre quinta (25) e sexta-feira (26) acontecem dois eventos promovidos
pelos integrantes da ala jovem da Federação das Associações Comerciais e
Empresariais do Paraná (Faciap).

A Assembleia Geral Ordinária (AGO) da Confederação Nacional de Jovens
Empresários (Conaje), que começa nesta quinta-feira (25) é um evento
nacional e vai reunir jovens empreendedores de todo o Brasil. Já na
sexta-feira (26) acontece o Encontro Paranaense de Jovens
Empreendedores.

\end{quote}

\fonte{CBN Curitiba. Faciap promove encontros para jovens empreendedores 
em Curitiba.
Disponível em: https://cbncuritiba.com.br/materias/faciap-promove-encontros-para-jovens-empreendedores-em-curitiba/.
Acesso em: 24 mai. 2023.}

Considerando os elementos constitutivos dos textos jornalísticos, a
informação sobre a cidade onde ocorre o evento está localizada:

\begin{escolha}
  
  \item no corpo da notícia.
  
  \item na lide.
  
  \item no título auxiliar.
  
  \item na manchete.

\end{escolha}

\coment{Identificar elementos constitutivos de textos pertencentes ao domínio
jornalístico/midiático.

a) Incorreta. A informação sobre Curitiba está localizada na manchete.
b) Incorreta. A informação sobre Curitiba está localizada na manchete.
c) Incorreta. A informação sobre Curitiba está localizada na manchete.
d) Correta. A informação sobre Curitiba está localizada na manchete.}

\num{5} Leia o poema abaixo para responder à questão. 

\begin{quote}
\begin{verse}

Mas a minha tristeza é sossego \\
Porque é natural e justa \\
E é o que deve estar na alma \\
Quando já pensa que existe \\
E as mãos colhem flores sem ela dar por isso.

\end{verse}
\end{quote}

\fonte{Alberto Caeiro (heterônimo de Fernando Pessoa. O Guardador de Rebanhos. 
Disponível em: http://www.dominiopublico.gov.br/download/texto/pe000001.pdf.
Acesso em: 24 mai. 2023.}

Pode-se inferir que, para o eu lírico do poema, sua tristeza

\begin{escolha}

    \item causa confusão ao colher as flores.

    \item impede que a justiça natural ocorra.

    \item repousa distante dele mesmo. 

    \item traz quietude por ser natual e justa. 

\end{escolha}

\coment{SAEB: Inferir informações implícitas em distintos textos.

a) Incorreta. Para o eu lírico, sua tristeza é sossego, sem causa confusão.
b) Incorreta. Para o eu lírico, sua tristeza é natural e justa. 
c) Incorreta. Para o eu lírico, sua tristeza está na alma.
d) Correta. Para o eu lírico, sua tristeza é sossego (portanto, quietude),
por ser natural e justa.}

\num{6} Leia o texto abaixo para responder à questão. 

%Excluí essa imagem
%\includegraphics[width=5.90551in,height=1.70833in]{media/image3.png}
%\fonte{https://tirasarmandinho.tumblr.com/}{{https://tirasarmandinho.tumblr.com/}}.
%Acesso em 26 de Abr de 2023

\begin{quote}

Dito isto, expirei às duas horas da tarde de uma
sexta-feira do mês de agosto de 1869, na minha bela
chácara de Catumbi. Tinha uns sessenta e quatro anos,
rijos e prósperos, era solteiro, possuía cerca de trezentos
contos e fui acompanhado ao cemitério por onze amigos.
Onze amigos! Verdade é que não houve cartas nem
anúncios. Acresce que chovia -- peneirava uma chuvinha
miúda, triste e constante, tão constante e tão triste \ldots{}

\end{quote}

\fonte{Machado de Assis. Memórias Póstumas de Brás Cubas. 
Disponível em: https://digital.bbm.usp.br/handle/bbm/4826.
Acesso em: 24 mai. 2023.}

Nos dois últimos períodos do parágrafo, o narrador pretende 

\begin{escolha}

    \item justificar o pequeno número de presentes a seu enterro. 

    \item lamentar a injustiça de ter morrido cedo e com saúde.

    \item exaltar a popularidade que tinha entre os amigos.

    \item agradecer sinceramente aos amigos que o acompanharam.

\end{escolha}

\coment{Saeb: Analisar elementos constitutivos de textos pertencentes ao domínio literário.

a) Correta. Os dois últimos períodos do parágrafo pretendem justificar a baixa
adesão, de apenas onze amigos, ao enterro do narrador. Note-se: ao iniciar o 
penúltimo período com ``verdade é que'' ele pretende explicar por que tão 
poucas pessoas compareceram: porque não houve anúncio de sua morte e porque chovia.
b) Incorreta. Os dois últimos períodos do parágrafo pretendem justificar a baixa
adesão, de apenas onze amigos, ao enterro do narrador. Note-se: ao iniciar o 
penúltimo período com ``verdade é que'' ele pretende explicar por que tão 
poucas pessoas compareceram: porque não houve anúncio de sua morte e porque chovia.
c) Incorreta. Os dois últimos períodos do parágrafo pretendem justificar a baixa
adesão, de apenas onze amigos, ao enterro do narrador. Note-se: ao iniciar o 
penúltimo período com ``verdade é que'' ele pretende explicar por que tão 
poucas pessoas compareceram: porque não houve anúncio de sua morte e porque chovia.
d) Correta. Os dois últimos períodos do parágrafo pretendem justificar a baixa
adesão, de apenas onze amigos, ao enterro do narrador. Note-se: ao iniciar o 
penúltimo período com ``verdade é que'' ele pretende explicar por que tão 
poucas pessoas compareceram: porque não houve anúncio de sua morte e porque chovia.}

\num{7} Leia os textos abaixo para responder à questão. 

\textbf{Exemplo 1}

\begin{quote}

\textit{Câncer de mama}
É o tipo de câncer mais frequente na mulher brasileira. Nesta doença,
ocorre um desenvolvimento anormal das células da mama, que
multiplicam-se repetidamente até formarem um tumor maligno.

\textit{Como descobrir a doença mais cedo?}
Toda mulher com 40 anos ou mais de idade deve procurar um ambulatório,
centro ou posto de saúde para realizar o exame clínico das mamas
anualmente, além disso, toda mulher, entre 50 e 69 anos deve fazer pelo
menos uma mamografia a cada dois anos. O serviço de saúde deve ser
procurado mesmo que não tenha sintomas!

\textit{O que é o exame clínico das mamas?}
É o exame das mamas realizado por médico ou enfermeiro treinado para
essa atividade. Neste exame poderão ser identificadas alterações nas
mesmas. Se for necessário, será indicado um exame mais específico, como
a mamografia.

\textit{O auto-exame previne a doença?}
O exame das mamas realizado pela própria mulher, apalpando os seios,
ajuda no conhecimento do próprio corpo, entretanto, esse exame não
substitui o exame clínico das mamas realizado por um profissional de
saúde treinado. Caso a mulher observe alguma alteração deve procurar
imediatamente o serviço de saúde mais próximo de sua residência. Mesmo
que não encontre nenhuma alteração no auto-exame, as mamas devem ser
examinadas uma vez por ano por um profissional de saúde!

\end{quote}

\fonte{Biblioteca Virtual em Saúde. Câncer de mama. Disponível em:
https://bvsms.saude.gov.br/cancer-de-mama/.
Acesso em: 24 mai. 2023.}

\textbf{Exemplo 2}

\begin{quote}

\includegraphics[width=3.73839in,height=3.73839in]{media/image4.png}

\end{quote}

\fonte{Ministério da Gestão e da Inovação em Serviços Públicos. 
Câncer de mama: é hora de falar sobre isso. 
Disponbível em: https://www.gov.br/arquivonacional/pt-br/canais_atendimento/imprensa/copy_of_noticias/cancer-de-mama-e-hora-de-falar-sobre-isso.
Acesso em: 24 mai. 2023.}

Os dois exemplos trazem informações sobre o câncer de mama e os exames
necessários para detectar a doença. Nos dois casos, o que faz com que as
informações sejam confiáveis é a presença de

\begin{escolha}
    
    \item textos explicativos.
    
    \item imagens e textos.
    
    \item citação de fontes oficiais.
    
    \item linguagem clara.

\end{escolha}

\coment{SAEB: Avaliar a fidedignidade de informações sobre um mesmo fato
divulgado em diferentes veículos e mídias.

a) Incorreta. Embora os textos sejam bastante explicativos, este não é
um critério de confiabilidade.
b) Incorreta. As figuras e textos só aparecem no Exemplo 2 e não podem
ser considerados recursos para avaliar a confiabilidade das informações.
c) Correta. Em ambos os casos a fonte das informações é o INCA, órgão
oficial que trata das questões relativas ao câncer de mama e demais
formas da doença. 
d) Incorreta. Embora a linguagem utilizada seja clara, este não é um
critério de confiabilidade.}

\num{8} Leia o texto abaixo para responder à questão. 

\begin{quote}

\textbf{O que a ciência diz sobre os remédios naturais para dormir}

\textit{Se você tomar um chazinho à base de ervas ou borrifar o quarto com
aromas agradáveis... será que vai conseguir dormir como um bebê?}

O sono raramente é mais desejado do que quando não conseguimos dormir.

E há uma grande variedade de remédios caseiros que prometem nos ajudar a
pegar no sono sem recorrer a produtos farmacêuticos.

\end{quote}

\fonte{G1. Disponível em: https://g1.globo.com/saude/noticia/2023/04/26/o-que-a-ciencia-diz-sobre-os-remedios-naturais-para-dormir.ghtml. 
Acesso em: 24 mai. 2023.}

A expressão ``raramente'', para se referir ao desejo de dormir, expressa:

\begin{escolha}
  
  \item que o sono é mais desejado quando não se consegue dormir.
  
  \item que na maioria das vezes o sono é desejado em qualquer ocasião.
  
  \item que sempre sentimos sono quando precisamos dormir.
  
  \item que só sentimos sono quando não desejamos dormir. 

\end{escolha}

\coment{SAEB: Analisar os efeitos de sentido produzidos pelo uso de modalizadores em
textos diversos.

a) Correta. A explicação desta alternativa corresponde às afirmações contidas
em ``O sono raramente é mais desejado do que quando não conseguimos dormir''.
b) Incorreta.A explicação desta alternativa não corresponde às afirmações contidas
em ``O sono raramente é mais desejado do que quando não conseguimos dormir''.
c) Incorreta.A explicação desta alternativa não corresponde às afirmações contidas
em ``O sono raramente é mais desejado do que quando não conseguimos dormir''.
d) Incorreta.A explicação desta alternativa não corresponde às afirmações contidas
em ``O sono raramente é mais desejado do que quando não conseguimos dormir''.}

\num{9} Leia o texto abaixo para responder à questão. 

\begin{quote}

\textbf{-- Por que não ser também um chato?}

\begin{verse}

Mas de repente percebi \\
Sem ser doutor \\
Que a chatice é a única doença \\
que não dói no portador \\
E resolvi ser chato. 

\end{verse}
\end{quote}

\fonte{Millor Fernandes. Poeminha Maçante, no livro Circo de palavras: histórias, poemas, pensamentos.
São Paulo: Ática, 2007. p. 63.}

Segundo o poema, o eu lírico escolheur ser chato porque 

\begin{escolha}

  \item a chatice é uma doença.
  
  \item ser chato não dói.
  
  \item ele queria ser doutor.
  
  \item se percebeu chato de repente.

\end{escolha}

\coment{SAEB: Avaliar a eficácia das estratégias argumentativas em textos de
diferentes gêneros.

a) Incorreta. O eu lírico escolheu ser chato porque ``a chatice é a única doença / 
que não dói no portador''. 
b) Correta. O eu lírico escolheu ser chato porque ``a chatice é a única doença / 
que não dói no portador''. 
c) Incorreta. O eu lírico escolheu ser chato porque ``a chatice é a única doença / 
que não dói no portador''. 
chatice é uma doença=
d) Incorreta. O eu lírico escolheu ser chato porque ``a chatice é a única doença / 
que não dói no portador''.} 

\num{10} Leia o texto abaixo para responder à questão. 

\begin{quote}

O casamento fez-se onze meses depois, e foi a mais bela festa das
relações dos noivos. Amigas de Clara, menos por amizade que por
inveja, tentaram arredá-la do passo que ia dar. Não negavam a gentileza
do noivo, nem o amor que lhe tinha, nem ainda algumas virtudes; diziam
que era dado em demasia a patuscadas.

\end{quote}

\fonte{Machado de Assis. Pai contra mãe. 
Disponível em: http://www.dominiopublico.gov.br/download/texto/bv000245.pdf.
Acesso em: 24 mai. 2023.}

De acordo com o texto, as amigas de Clara

\begin{escolha}

    \item tentaram impedir o casamento dela porque lhe nutriam amizade honesta.

    \item persuadiram-na a desistir por causa da falsidade dos sentimentos do noivo.  

    \item desaprovavam o casamento porque não lhe nutriam amizade verdadeira.

    \item tentaram convencê-la a desistir do casamento por inveja, apesar da amizade.

\end{escolha}

\coment{SAEB: Analisar os processos de referenciação lexical e pronominal.

a) Incorreta. As amigas de Clara eram evidentemente mais invejosas do que amigas, 
como se verifica na expressão ``menos por amizade que por inveja''.
b) Incorreta. Os sentimentos do noivo não são questionados. As amigas de Clara, 
invejosas, dizem-lhe que ele, embora apaixonado e virtuoso, era festeiro demais.
c) Incorreta. Embora invejosas, as amigas de Clara tinham-lhe alguma amizade.
d) Correta. A chave para escolher esta alternativa é compreender que, por meio da  
expressão ``menos por amizade que por inveja'' o narrador evidencia que, por mais 
houvesse amizade entre Clara e as amigas, estas tentaram convencê-la a desistir 
do casamento por inveja.} 

\num{11} Observe os períodos a seguir.

\begin{itemize}

    \item \textbf{Já que} o vento soprava muito forte, decidiram não sair
naquela noite.

    \item Eu não consegui ir ao escritório \textbf{porque} estava muito doente

    \item Os jovens terão suas necessidades ouvidas, \textbf{exceto} se se
afastarem dos responsáveis.

    \item \textbf{Embora} estivesse triste, não chorou.

\end{itemize}

Assinale a alternativa que indica corretamente as relações de sentido
expressas pelos conectivos em destaque.

\begin{escolha}

    \item causa, causa, condição, concessão.

    \item comparação, condição, finalidade, oposição.

    \item causa, oposição, condição, finalidade

    \item finalidade, comparação, tempo, causa.

\end{escolha}

\coment{SAEB: Analisar os mecanismos que contribuem para a progressão textual.

a) Correta. Na primeira afirmação, a causa da decisão de não sair é o vento soprar forte;
na segunda, estar doente é a causa da impossibilidade de ir ao escritório; na terceira, 
as necessidades dos jovens só não serão ouvidas em uma condição: se eles se afastarem dos
responsáveis; na última, apesar da tristeza, não houve lágrimas. 
b) Incorreta. Na primeira afirmação, a causa da decisão de não sair é o vento soprar forte;
na segunda, estar doente é a causa da impossibilidade de ir ao escritório; na terceira, 
as necessidades dos jovens só não serão ouvidas em uma condição: se eles se afastarem dos
responsáveis; na última, apesar da tristeza, não houve lágrimas.  
c) Incorreta. Na primeira afirmação, a causa da decisão de não sair é o vento soprar forte;
na segunda, estar doente é a causa da impossibilidade de ir ao escritório; na terceira, 
as necessidades dos jovens só não serão ouvidas em uma condição: se eles se afastarem dos
responsáveis; na última, apesar da tristeza, não houve lágrimas. 
d) Incorreta. Na primeira afirmação, a causa da decisão de não sair é o vento soprar forte;
na segunda, estar doente é a causa da impossibilidade de ir ao escritório; na terceira, 
as necessidades dos jovens só não serão ouvidas em uma condição: se eles se afastarem dos
responsáveis; na última, apesar da tristeza, não houve lágrimas.}  

\num{12} Leia o texto abaixo para responder à questão. 

\begin{quote}

Em relação ao consumo de agrotóxicos no Brasil, hoje paira uma grande
incerteza sobre os números exatos. No ano de 2015, só foram divulgados
os valores em dólares dos ganhos da indústria. Neste sentido, houve uma
forte queda, de 21\%, em relação a 2014. No entanto, se considerarmos a
variação do câmbio, vemos que na verdade o faturamento em reais subiu de
R\$28 para R\$32 bilhões. Como uma boa parte do custo dos agrotóxicos é
importada, não é possível saber se aumentou ou diminuiu a quantidade de
agrotóxicos em 2015.

\end{quote}

\fonte{Alan Tygel. Agrotóxicos no Brasil: O veneno ainda está na mesa.  
Disponível em:
https://www.brasildefato.com.br/2016/11/23/agrotoxicos-no-brasil-o-veneno-ainda-esta-na-mesa.
Acesso em: 24 mai. 2023.}

Assinale a alternativa que contém a figura de linguagem correta e o
trecho em que essa figura aparece.

\begin{escolha}

    \item Hipérbole em ``Agrotóxicos do Brasil''. 

    \item Metáfora em ``hoje paira uma grande incerteza sobre os números exatos''. 

    \item Metonímia em ``O veneno ainda está na mesa''. 

    \item Hipérbole em ``Neste sentido, houve uma forte queda, de 21\%''. 

\end{escolha}

\coment{SAEB: Analisar o uso de figuras de linguagem como estratégia argumentativa.

a) Incorreta. Não ocorre metáfora na expressão indicada.
b) Incorreta. Não ocorre metáfora na expressão indicada.
c) Correta. Ocorre metonímia no uso do termo ``o veneno'' para se
referir à alimentação contaminada por agrotóxicos. 
d) Incorreta. Não ocorre hipérbole na expressão indicada.}

\num{13} Leia o texto abaixo para responder à questão. 

\begin{quote}

Uma noite destas, vindo da cidade para o Engenho
Novo, encontrei no trem da Central um rapaz aqui do
bairro, que eu conheço de vista e de chapéu. 
Cumprimentou-me, sentou-se ao pé de mim, falou da lua
e dos ministros, e acabou recitando-me versos. A viagem 
era curta, e os versos \textbf{pode ser que não fossem} 
inteiramente maus. Sucedeu, porém, que, como eu estava 
cansado, fechei os olhos três ou quatro vezes; tanto 
bastou para que ele interrompesse a leitura e metesse 
os versos no bolso.

\end{quote}

\fonte{Machado de Assis. Dom Casmurro. 
Disponível em: https://www.ic.unicamp.br/~stolfi/misc/2012-02-13-domine-casmurrus.pdf.
Acesso em: 24 mai. 2023.}

A escolha dos verbos no trecho em destaque dá a entender que o narrador

\begin{escolha}

  \item julga ruins os versos do rapaz e interrompe-lhe a leitura.

  \item considera bons os versos do rapaz, apesar de cansativos.

  \item não tolera a falta de qualidade da poesia do rapaz.

  \item avalia que os versos não eram ruins, sem considerá-los bons.

\end{escolha}

\coment{SAEB: Analisar os efeitos de sentido dos tempos, modos e/ou vozes verbais com
base no gênero textual e na intenção comunicativa.

a) Incorreta. O narrador não deixa claro que considerava os versos ruins, como se
pode verificar pela escolha das formas verbais destacadas no texto.
b) Incorreta. O narrador não deixa claro que considera bons os versos do rapaz,como se
pode verificar pela escolha das formas verbais destacadas no texto.
c) Incorreta. A escolha das formas verbais destacadas no texto não permite fazer uma
afirmação tão categórica quanto a desta alternativa.
d) Correta. A escolha das formas verbais destacadas no texto sugere que os versos do
rapaz, embora não pudessem ser chamados de completamente ruins, também não poderiam
ser chamados de bons.}

\num{14} Leia o texto abaixo para responder à questão. 

\begin{quote}

\textbf{Os cosplays mais irados de 2021}

Apesar da pandemia, a criatividade na cultura pop não parou,
por isso vamos ver os cosplays que mais se destacaram em 2021 até
agora!

\end{quote}

\fonte{Terra. Os cosplays mais irados de 2021. 
Disponível em: https://www.terra.com.br/gameon/geek/os-cosplays-mais-irados-de-2021,b409f1697bb7db34500d0614340e6423966k7p1p.html.
Acesso em: 24 mai. 2023.}

No exemplo acima, observam-se algumas variações linguísticas. Assinale a
alternativa que contém a correta descrição de termos do texto.

\begin{escolha}
  
  \item cosplays: estrangeirismo.
  
  \item cultura pop: regionalismo.
  
  \item irados: norma-padrão.
  
  \item criatividade: gíria.

\end{escolha}

\coment{SAEB: Analisar as variedades linguísticas em textos.

a) Correta. O termo ``cosplay'' vem do inglês.
b) Incorreta. A expressão ``'cultura pop'' é amplamente utilizada nas mídias,
sem se restringir a uma região específica.
c) Incorreta. ``Irado'', no contexto, é gíria. 
d) Incorreta. O substantivo ``criatividade'' não é gíria.} 

\num{15} Leia o texto abaixo para responder à questão. 

\begin{quote}

\textbf{Os ``causos'' de Joseli Dias}

O ofício de jornalista de Joseli Dias deu-lhe a agilidade de lidar com
as palavras. As narrativas das lendas e dos ``causos'' desta região,
reunidas no livro \textit{Mitos e Lendas do Amapá}, se constituem em um trabalho
de suma importância para todos nós, que valorizamos os elementos mais
puros e autênticos da nossa cultura popular.

\fonte{Joseli Dias. Mitos e lendas no Amapá. 
Disponível em: https://www2.senado.leg.br/bdsf/bitstream/handle/id/576836/Mitos_lendas_Amapa.pdf.
Acesso em: 24 mai. 2023.}

O uso do termo ``causos'' se justifica, pois, nos textos de Joseli Dias,

\begin{escolha}
    
    \item era necessário o uso da linguagem formal para retratar os mitos da região do Amapá..
    
    \item o ofício do jornalismo impunha a seu estilo a objetividade breve dos cronistas.
    
    \item as gírias da juventude de sua época ganham expressão literária significativa.
    
    \item a pureza e a autenticidade da cultura popular se manifestam por meio do vocabulário.

\end{escolha}

\coment{SAEB: Avaliar a adequação das variedades linguísticas em contextos de uso.

a) Incorreta. O uso do termo ``causos'' alude ao vocabulário da cultura popular, 
não ao uso da linguagem formal.
b) Incorreta. O uso do termo ``causos'' alude ao vocabulário da cultura popular, 
não à objetividade breve dos cronistas.
c) Incorreta. O uso do termo ``causos'' alude ao vocabulário da cultura popular, 
não às gírias da juventude.
d) Correta. O uso do termo ``causos'' alude ao vocabulário da cultura popular.}

\addcontentsline{toc}{chapter}{Simulado 4}
\markboth{Simulado 4}{}

\num{1} Leia o texto abaixo para responder à questão. 

\begin{quote}

\textbf{França limitará o aumento do preço da eletricidade até 2025}

\textit{Recuperação da atividade econômica após a pandemia do coronavírus
provocou uma alta nos preços da energia na Europa}

A França vai prolongar até 2025 os subsídios adotados em outubro de 2021
para limitar o aumento do preço da eletricidade, em um contexto de
preocupação com a perda de poder de compra, anunciou o governo nesta
sexta-feira (21).

Le Maire alertou, por outro lado, que encerrará este ano o programa para
limitar o aumento dos preços do gás, já que estes ``voltaram ao nível
anterior à crise, de 50 euros por megawatt-hora''.

A recuperação da atividade econômica após a pandemia do coronavírus
provocou uma alta nos preços da energia na Europa no final de 2021, que
disparou meses depois com o início da invasão russa à Ucrânia.

\end{quote}

\fonte{Folha de Pernambuco. França limitará o aumento do preço da eletricidade 
até 2025
Disponível em:
https://www.folhape.com.br/noticias/franca-limitara-o-aumento-do-preco-da-eletricidade-ate-2025/267267/.
Acesso em: 24 mai. 2023.}

Segundo a matéria os aumentos do preço da energia se devem:

\begin{escolha}
    
    \item à recuperação econômica pós pandemia e falta economia dos franceses.
    
    \item ao programa do governo que se encerra.
    
    \item à recuperação da atividade econômica e à guerra na Ucrânia.
    
    \item à preocupação com o poder de compra dos franceses.

\end{escolha}

\coment{SAEB: Identificar teses, opiniões, posicionamentos explícitos e argumentos em
textos.

a) Incorreta. O texto não contém referências à atitude dos consumidores.
b) Incorreta. A interrupção do programa governamental não é fator que 
tenha interferido no aumento.
c) Correta. Segundo o texto, estes são os principais fatores que levaram
ao aumento dos preços da energia na Europa. 
d) Incorreta. O texto se refere ao poder de compra dos franceses como uma
preocupação, mas não trata a questão como fator para o aumento da energia.}

\num{2} Leia o texto abaixo para responder à questão. 

\begin{quote}

\includegraphics[width=4.89702in,height=6.12128in]{media/image24.png}

\fonte{Prefeitura Municipal de Eunápolis. Vacinação: Covid-19 para crianças de 5 a 11 anos. 
Disponível em: https://www.eunapolis.ba.gov.br/site/Noticias/noticia-060220221925071633-Vacina-o-Covid-19-para-crian-as-de-5-a-11-anos.
Acesso em: 24 mai. 2023.}

A força expressiva da campanha acima advém sobretudo 

\begin{escolha}
    
    \item da imagem inesperada que ilustra o cartaz.
    
    \item do contraste entre as cores de fundo. 
    
    \item do destaque da lista de locais de vacinação. 
    
    \item do jogo de palavras em ``não vacile, vacine''.

\end{escolha}

\coment{SAEB: Identificar o uso de recursos persuasivos em textos verbais e não
verbais

a) Incorreta. A imagem que ilustra o cartaz não é inesperada.
b) Incorreta. O contraste entre as cores de fundo não alcança
grande força expressiva.
c) Incorreta. A lista de locais de vacinação não está em destaque.
d) Correta. O jogo com palavras de sons semelhantes, mas de sentidos
distintos confere força expressiva ao cartaz.}

\num{3} Leia o texto abaixo para responder à questão. 

\begin{quote}

CAPÍTULO I

DISPOSIÇÕES GERAIS

Art. 1º É instituída a Lei Brasileira de Inclusão da Pessoa com
Deficiência (Estatuto da Pessoa com Deficiência), destinada a assegurar
e a promover, em condições de igualdade, o exercício dos direitos e das
liberdades fundamentais por pessoa com deficiência, visando à sua
inclusão social e cidadania.

Parágrafo único. Esta Lei tem como base a Convenção sobre os Direitos
das Pessoas com Deficiência e seu Protocolo Facultativo, ratificados
pelo Congresso Nacional por meio do Decreto Legislativo nº 186, de 9 de
julho de 2008 , em conformidade com o procedimento previsto no § 3º do
art. 5º da Constituição da República Federativa do Brasil , em vigor
para o Brasil, no plano jurídico externo, desde 31 de agosto de 2008, e
promulgados pelo Decreto nº 6.949, de 25 de agosto de 2009 , data de
início de sua vigência no plano interno.

\end{quote}

\fonte{Presidência da República. 
Lei Brasileira de Inclusão da Pessoa com Deficiência (Estatuto da Pessoa com Deficiência).
Disponível em: https://www.planalto.gov.br/ccivil_03/_ato2015-2018/2015/lei/l13146.htm.
Acesso em: 24 mai. 2023.}

O texto acima pertence ao domínio dos textos normativos e tem como
objetivo assegurar os direitos das pessoas com deficiência. Esta
afirmação pode ser comprovada pelo

\begin{escolha}
    
    \item artigo primeiro das disposições gerais.
    
    \item artigo quinto da Constituição Federal.
    
    \item capítulo I.
    
    \item parágrafo único.

\end{escolha}

\coment{SAEB: Identificar formas de organização de textos normativos, 
legais e/ou reivindicatórios.

a) Correta. Logo no primeiro artigo aparecem as disposições gerais e os
temas a serem normalizados pela lei.
b) Incorreta. O artigo quinto da Constituição aparece como reforço e
justificativa da importância da Lei.
c) Incorreta. O capítulo primeiro compreende demais partes e
peculiaridades da Lei.
d) Incorreta. O parágrafo único apresenta as justificativas sobre
pertinência da regulamentação do Estatuto da Pessoa com Deficiência.}

\num{4} Leia o texto abaixo para responder à questão. 

\begin{quote}

Aqui em Manaus faço uma parte desse grande projeto que é desenvolvido
por diversas instituições japonesas e conta com financiamento da GHIT
Funding, tendo à frente o doutor Shigeto Yoshida, da Universidade de
Kanazawa, no Japão, que é o desenvolvedor dessa formulação vacinal.
Essa vacina atua contra o parasita no hospedeiro humano e, também,
tentando evitar a infecção do hospedeiro que é o vetor, que transmite a
doença de uma pessoa para outra. Ela tem na sua forma a proteína CSP,
presente na vacina que já está em uso em diversos países da África e na
vacina desenvolvida pela Universidade de Oxford, no Reino Unido. Então
ela tem esse pedaço do parasita, essa proteína, que é um alvo estudado
já há muitos anos e com poder de proteção para as infecções nos humanos.
Essa proteína tem um papel importante para impedir que o parasita chegue
ao fígado, que é o primeiro local em que ele se instala, e, por causa
disso, os anticorpos e a resposta celular produzidos por uma vacina
poderiam impedir a entrada do parasita e seu desenvolvimento nos
humanos.

\end{quote}

\fonte{Ciça Guedes. Ciência Hoje. Malária: uma vacina contra um desafio amazônico.
Disponível em: https://cienciahoje.org.br/artigo/malaria-uma-vacina-contra-um-desafio-amazonico/
Acesso em: 24 mai. 2023.}

Neste texto, podemos ver a explicação de como devem funcionar as vacinas
contra a malária e como estão sendo desenvolvidas. Assinale a alternativa
cuja expressão destacada seja marcador discursivos de causa.

\begin{escolha}

    \item \ldots ``conta com financiamento da GHIT Funding, \textbf{tendo à frente} o doutor Shigeto Yoshida''.

    \item ``\textbf{e, também}, tentando evitar a infecção do hospedeiro''.

    \item ``\textbf{Então} ela tem esse pedaço do parasita''.

    \item ``e, \textbf{por causa disso}, os anticorpos e a resposta celular''.

\end{escolha}

\coment{SAEB: Identificar elementos constitutivos de gêneros de divulgação científica

a) Incorreta. A expressão destacada expressa a circunstância de modo.
b) Incorreta. A expressão destacada expressa a circunstância de adição.
c) Incorreta. No contexto em que se insere, a expressão destacada é marca de oralidade.
d) Correta. A expressão destacada expressa a circunstância de causa.}

\num{5} Leia os textos abaixo para responder à questão. 

\begin{quote}

\textbf{Exemplo 1}

A relação do ser humano com os animais de estimação existe há mais de 10
mil anos. Os animais domésticos preenchem várias necessidades emocionais
dos homens e dessa forma esses bichinhos, principalmente cães e gatos,
se tornam cada vez mais parte da nossa casa e de nossa família.

Mas infelizmente, muita gente ainda possui animais de estimação, mas não
possui a menor condição de criá-los. E quando falamos em condição de
criar, refiro-me principalmente a condições psicológicas.

Tanto é verdade que os maus-tratos aos ``pets'' são evidentes em todas as
cidades do país: animais famintos, torturados, feridos covardemente,
confinados em espaços minúsculos ou abandonados nas ruas ou estradas
Brasil afora.

E se quisermos mudar essa situação, precisamos perder o medo e o receio
de nos envolver e denunciar os maus-tratos, que só cessarão
quando aqueles que cometem esses crimes começarem a ser exemplarmente
punidos.

\end{quote}

\fonte{Hugo Xavier. Jornal Cidade. Os maus tratos e o abandono de animais. 
Disponível em: https://www.jornalcidademg.com.br/artigo-os-maus-tratos-e-o-abandono-de-animais/. 
Acesso em: 24 mai. 2023.}

\textbf{Exemplo 2}

\begin{quote}

\textbf{Polícia resgata 10 cachorros em situação de maus-tratos em dois
imóveis de Curitiba}

Resgate aconteceu no bairro Abranches e no Barreirinha. Polícia chegou
aos casos após denúncias de vizinhos.

A Polícia Civil e a Rede de Proteção Animal de Curitiba resgataram 10
cachorros que estavam em situação de maus-tratos na capital paranaense,
na quinta-feira (23).

O resgate aconteceu em dois imóveis, um no bairro Abranches e outro no
Barreirinha.

O proprietário do imóvel do Abranches foi multado em R\$ 12 mil, por
criação e venda irregulares e maus-tratos aos animais. A suspeita é que
no local existia um criadouro ilegal.

\end{quote}

\fonte{G1. Polícia resgata 10 cachorros em situação de maus-tratos em dois imóveis de Curitiba.
Disponível em: https://g1.globo.com/pr/parana/noticia/2023/02/24/policia-resgata-10-cachorros-em-situacao-de-maus-tratos-em-dois-imoveis-de-curitiba.ghtml
Acesso em: 24 mai. 2023.}

Os dois exemplos foram extraídos de meios de comunicação e representam
diferentes gêneros do campo jornalístico. Com base nas diferenças entre
os dois textos é correto afirmar que:

\begin{escolha}
    
    \item O Exemplo 1 é uma entrevista e o Exemplo 2 é uma notícia.
    
    \item O Exemplo 1 é um artigo de opinião e o Exemplo 2 uma notícia.
    
    \item O Exemplo 1 é uma reportagem e o Exemplo 2 um artigo de opinião.
    
    \item O Exemplo 1 é uma carta de leitor e o Exemplo 2 é um artigo de opinião.

\end{escolha}

\coment{SAEB: Analisar a relação temática entre diferentes gêneros jornalísticos.

a) Incorreta. O Exemplo 1 é um artigo de opinião; o Exemplo 2, uma notícia.
b) Correta. O Exemplo 1 é artigo de opinião, pois apresenta
argumentação acerca do tema; o Exemplo 2 apresenta claramente estrutura
de notícia, dividido em título, linha fina, lide e corpo, além de 
restringir-se a noticiar o resgate dos animais, sem expressão de 
opinião. 
c) Incorreta. O Exemplo 1 é um artigo de opinião; o Exemplo 2, uma notícia.
d) Incorreta. O Exemplo 1 é um artigo de opinião; o Exemplo 2, uma notícia.}

\num{6} Leia o soneto abaixo para responder à questão. 

\begin{quote}
\begin{verse}

Apartava-se Nise de Montano, \\
em cuja alma partindo se ficava; \\
que o pastor na memória a debuxava, \\
por poder sustentar-se deste engano.

Pelas praias do Índico Oceano \\
sobre o curvo cajado se encostava, \\
e os olhos pelas águas alongava, \\
que pouco se doíam de seu dano.

Pois com tamanha mágoa e saudade \\
(dizia) quis deixar me a que eu adoro, \\
por testemunhas tomo Céu e estrelas.

Mas se em vós, ondas, mora piedade, \\
levai também as lágrimas que choro, \\
pois assim me levais a causa delas!

\end{verse}
\end{quote}

\fonte{Luís de Camões. Sonetos.  
Disponível em: http://www.dominiopublico.gov.br/download/texto/bv000164.pdf.
Acesso em: 24 mai. 2023.}

Considerando os elementos que caracterizam os textos literários, o texto
acima pode ser considerado

\begin{escolha}

    \item conto, com personagens de ações restritas a espaço limitado.
    
    \item texto dramático, com falas de personagens e rubricas.  
    
    \item poema, baseado na sonoridade de versos divididos em estrofes.  
    
    \item romance, extensa narração de ações de diversos conflitos. 

\end{escolha}

\coment{SAEB: Analisar elementos constitutivos de textos pertencentes ao domínio
literário.

a) Incorreta. O texto é um poema.
b) Incorreta. O texto é um poema.
c) Correta. O texto é um soneto, forma poemática de quatro estrofes, 
dois quartetos e dois tercetos, cujos versos rimados e metrificados 
imprimem sonoridade regular ao conjunto. 
d) Incorreta. O texto é um poema.}

\num{7} Leia o texto abaixo para responder à questão. 

\begin{quote}

\textbf{Falta de iluminação em rua completa 1 ano e tem ``festa de
aniversário'' como protesto em Ourinhos}

Manifestação foi na Rua Narciso Nicolosi, no Jardim Paulista. Prefeitura
informou na manhã seguinte ao ato que fez a troca de lâmpada queimada.

Moradores de Ourinhos (SP) protestaram contra a falta de iluminação
pública na Rua Narciso Nicolosi, no Jardim Paulista, na noite desta
quarta-feira (26) com uma ``festa de aniversário'' para marcar um ano de
espera pela troca de lâmpadas em um poste.

A ``comemoração'' teve bolo com vela, refrigerante, bexigas, cartazes e
parabéns para você em coro e palmas em meio a críticas pela escuridão.

\end{quote}

\fonte{G1. Falta de iluminação em rua completa 1 ano e tem 'festa de 
aniversário' como protesto em Ourinhos. 
Disponível em: https://g1.globo.com/sp/bauru-marilia/noticia/2023/04/27/falta-de-iluminacao-em-rua-completa-1-ano-e-tem-festa-de-aniversario-como-protesto-em-ourinhos.ghtml.
Acesso em: 24 mai. 2023.}

Os termos entre aspas na notícia expressam

\begin{escolha}
    
    \item ironia. 
    
    \item citação de falas de autoridades da cidade.
    
    \item citação de argumentos de especialistas.
    
    \item enfase nas palavras e termos destacados.

\end{escolha}

\coment{SAEB: Analisar efeitos de sentido produzido pelo uso de formas
de apropriação textual (paráfrase, citação etc.).

a) Correta. Os termos estão sendo usados de forma irônica para protestar
contra a falta de energia.
b) Incorreta. O uso de aspas neste caso não indica a fala de
autoridades.
c) Incorreta. O uso das aspas neste caso não introduz falas
de especialistas.
d) Incorreta. Os termos não estão sendo puramente enfatizados, o uso de
aspas se dá principalmente pelo contexto irônico da notícia.}

\num{8} Leia o texto abaixo para responder à questão. 

\begin{quote}

\textbf{Telegram diz que Justiça ordenou entrega de dados ``impossíveis''
de serem obtidos; PF afirma que lentidão permitiu exclusão de
informações}

\textit{Cofundador do aplicativo afirmou que empresa está recorrendo da
suspensão do serviço, que começou a valer na noite da última quarta
(26).}

O cofundador do Telegram Pavel Durov afirmou nesta quinta-feira (27) que
a Justiça brasileira ordenou a entrega de dados ``impossíveis'' de serem
coletados. Em seu canal no aplicativo, ele afirmou que a empresa vai
recorrer da decisão.

\end{quote}

\fonte{G1. Telegram diz que Justiça ordenou entrega de dados impossíveis 
de serem obtidos; PF afirma que lentidão permitiu exclusão de informações. 
Disponível em: https://g1.globo.com/tecnologia/noticia/2023/04/27/telegram-diz-que-justica-ordenou-coleta-de-dados-impossiveis-de-serem-obtidos-pf-afirma-que-lentidao-permitiu-exclusao-de-informacoes.ghtml.
Acesso em: 24 mai. 2023.}

Considerando a notícia acima, pode-se afirmar que o título da notícia expressa

\begin{escolha}

    \item concordância do autor com a justificativa da empresa. 

    \item simpatia do autor pela decisão da Polícia Federal.  

    \item intenção de influenciar a opinião do leitor.  

    \item tentativa de apresentar dois pontos de vista opostos. 

\end{escolha}

\coment{SAEB: Analisar os efeitos de sentido decorrentes dos mecanismos de 
construção de textos jornalísticos/midiáticos.

a) Incorreta. O título não contém concordância do autor com a justificativa da empresa.
b) Incorreta. O título não contém simpatia do autor pela decisão da Polícia Federal.
c) Incorreta. Não há elementos que permitam afirma que o autor tenha tentado
influenciar a opinião do leitor.
d) Correta. A extensão do título da notícia já revela a intenção do autor: 
apresentar dois pontos de vista opostos (o do Telegram e o da Polícia Federal),
separados por ponto e vírgula.}

\num{9} Leia o texto abaixo para responder à questão. 

\begin{quote}

\textbf{O rato do mato e o rato da cidade}

Um ratinho da cidade foi uma vez convidado para ir à casa
de um rato do campo. Vendo que seu companheiro vivia pobremente de
raízes e ervas, o rato da cidade convidou-o a ir morar com ele:

-- Tenho muita pena da pobreza em que você vive --
disse. -- Venha morar comigo na cidade e você verá como lá a
vida é mais fácil.

Lá se foram os dois para a cidade, onde se acomodaram
numa casa rica e bonita.

Foram logo à despensa e estavam muito bem, se
empanturrando de comidas fartas e gostosas, quando entrou uma
pessoa com dois gatos, que pareceram enormes ao ratinho do
campo.

Os dois ratos correram espavoridos para se esconder.

-- Eu vou para o meu campo -- disse o rato do campo
quando o perigo passou. -- Prefiro minhas raízes e ervas na
calma, às suas comidas gostosas com todo esse susto.

\textit{Mais vale magro no mato que gordo na boca do gato.} 

\end{quote}

\fonte{Ana Rosa Abreu e outros autores. Alfabetização: livro do aluno. Vol.2: 
contos tradicionais, fábulas, lendas e mitos. Disponível em: 
http://www.dominiopublico.gov.br/download/texto/me001614.pdf.
Acesso em: 24 mai. 2023.}

A frase final, destacada em itálico, tem a finalidade de 

\begin{escolha}
    
    \item adicionar elementos à fábula, explicando-lhe o sentido. 
    
    \item resumir de forma expressiva o sentido da fábula. 
    
    \item propor ao leitor uma pergunta sobre o sentido da fábula.
    
    \item remeter a conclusão da fábula a um outro texto.  

\end{escolha}

\coment{SAEB: Inferir informações implícitas em distintos textos

a) Correta. O trecho final, em destaque, chama-se \textit{moral da história}
e serve para sintetizar, de maneira expressiva (observe-se a rima entre
\textit{mato} e \textit{gato}) o sentido da fábula. 
b) Incorreta. A moral da história resume de forma expressiva o sentido da 
fábula.
c) Incorreta. A moral da história resume de forma expressiva o sentido da 
fábula.
d) Incorreta. A moral da história resume de forma expressiva o sentido da 
fábula.}

\num{10} Leia o texto abaixo para responder à questão. 

\begin{quote}

Os recifes de coral são ambientes extremamente diversos, abrigando cerca
de 25\% de toda a biodiversidade marinha. Suas estruturas rígidas são
constituídas por organismos marinhos de esqueleto calcário, os corais.
Mas há outros tipos de organismos que vivem nos recifes, como algas,
moluscos e peixes.

Além de sua importância biológica, recifes também são valiosos do ponto
de vista socioeconômico. A elevada biodiversidade produz bastante
pescado, de modo que, hoje, cerca de 10\% da proteína animal consumida
no planeta provem de recifes de coral. Outro fator a considerar é que a
beleza natural estimula o turismo, e, consequentemente, o
estabelecimento de operadoras de mergulho, pousadas e restaurantes.

\end{quote}

\fonte{Tássia Biazon. Ciência Hoje. Recifes sob estresse. 
Disponível em: https://cienciahoje.org.br/artigo/recifes-sob-estresse/. 
Acesso em: 24 mai. 2023. com adaptações.}

A importância biológica dos recifes de coral pode ser comprovada por:

\begin{escolha}
    
    \item conterem beleza natural que estimula o turismo e o comércio de
  pousadas e restaurantes.
    
    \item serem estruturas rígidas constituídas por organismos marinhos de
  esqueleto calcário, chamadas de corais.
    
    \item serem ambientes extremamente diversos, abrigando cerca de 25\% de
  toda a biodiversidade marinha.
    
    \item contribuírem com a presença de peixes favorecendo a produção de
  pescado.

\end{escolha}

\coment{SAEB: Distinguir fatos de opiniões em textos.

a) Incorreta. O turismo e o comércio não se referem à importância 
\textit{biológica} dos recifes de coral: essas atividades dizem respeito a
seu valor socioeconômico.
b) Incorreta. A alternativa contém a descrição objetiva da estrutura dos
corais, não sua importância biológica.
c) Correta. Segundo o texto, os recifes de corais são importantes
biologicamente por abrigarem grande biodiversidade.
d) Incorreta. A produção de pescado não se refere à importância 
\textit{biológica} dos recifes de coral: essa atividade diz respeito
a seu valor socioeconômico.}

\num{11} Leia o texto abaixo para responder à questão. 

\begin{quote}

Algumas atitudes da sociedade têm colaborado para que as crianças
desenvolvam hábitos alimentares saudáveis. É o caso de escolas que
procuram controlar o tipo de comida consumida pelos seus alunos. Na
cantina ou lancheira são proibidos doces e frituras. Caso a criança os
traga de casa, esses alimentos são mandados de volta e uma outra solução
é dada para a refeição daquele dia.

Mesmo assim, alguns pais acabam infringindo as regras e mandando como
lanche para seus filhos alguns alimentos proibidos. Muitos justificam
que, se não for aquilo, a criança não come, sendo melhor que coma algo
sem qualidade que ficar de estômago vazio.

\end{quote}

\fonte{Ana Cássia Maturano. G1. Alimentação saudável na mira das escolas e dos pais.
Disponível em: https://g1.globo.com/educacao/noticia/2010/04/opiniao-alimentacao-saudavel-na-mira-das-escolas-e-dos-pais.html.
Acesso em: 24 mai. 2023.}

No último parágrafo do texto o autor deixa claro que

\begin{escolha}
    
    \item as escolas não se empenham em promover uma alimentação saudável para os
  estudantes.
    
    \item os pais se empenham em promover uma alimentação saudável para os
  filhos.
    
    \item pais e escolas estão empenhados em oferecer uma alimentação saudável para
  os estudantes.
    
    \item alguns pais não se empenham em oferecer uma alimentação saudável para os filhos.

\end{escolha}

\coment{SAEB: Analisar os processos de referenciação lexical e pronominal.

a) Incorreta. No último parágrafo, a autora afirma que, apesar das restrições
e atitudes de algumas escolas, há pais que não se empenham em oferecer uma alimentação saudável para os filhos.
b) Incorreta. No último parágrafo, a autora afirma que, apesar das restrições
e atitudes de algumas escolas, há pais que não se empenham em oferecer uma alimentação saudável para os filhos.
c) Incorreta. No último parágrafo, a autora afirma que, apesar das restrições
e atitudes de algumas escolas, há pais que não se empenham em oferecer uma alimentação saudável para os filhos.
d) Correta. No último parágrafo, a autora afirma que, apesar das restrições
e atitudes de algumas escolas, há pais que não se empenham em oferecer uma alimentação saudável para os filhos.}

\num{12} Leia o texto abaixo para responder à questão. 

\begin{quote}

\textbf{Direito da criança de brincar na rua é negligenciado}

\emph{Carimba, cabra-cega, passa anel, boca-de forno, batata-quente:
crianças desconhecem muitas brincadeiras}

Simplesmente brincar. Na rua, nas praças, quadras, parquinho, com
amigos, com vizinhos. É direito constitucional, está nos estatutos,
declarações e lei. Mas, efetivamente, a brincadeira está bem longe de
ser uma prioridade, tanto para muitas crianças, quanto para os pais.

\end{quote}

\fonte{Diário do Nordeste. Direito da criança de brincar na rua é 
negligenciado. Disponível em: https://diariodonordeste.verdesmares.com.br/metro/direito-da-crianca-de-brincar-na-rua-e-negligenciado-1.621061.
Acesso em: 24 mai. 2023.}

A leitura atenta da reportagem permite afirmar que há marca de parcialidade
no trecho:

\begin{escolha}
  
    \item Simplesmente brincar.
  
    \item Direito da criança de brincar na rua é negligenciado.
  
    \item Na rua, nas praças, quadras, parquinho, com amigos, com vizinhos.
  
    \item É direito constitucional, está nos estatutos, declarações e lei.

\end{escolha}

\coment{SAEB: Avaliar diferentes graus de parcialidade em textos jornalísticos.

a) Incorreta. O trecho não apresenta parcialidade.
b) Correta. A opinião expressa no título da notícia está evidente 
no uso do adjetivo \textit{negligenciado}.
c) Incorreta. O trecho não apresenta parcialidade.
d) Incorreta. O trecho não apresenta parcialidade.}

\num{13} Leia o texto abaixo para responder à questão. 

\begin{quote}

\textbf{Strava mostra aumento do número de ciclistas, pós pandemia, em
seis capitais}

A pandemia trouxe algumas mudanças positivas que têm tudo para se tornar
permanentes para muitas pessoas. Uma delas foi o maior número de
ciclistas pelas cidades, que começaram a utilizar o meio de locomoção
com o receio de compartilhar viagens em carros e transportes urbanos.

\end{quote}

\fonte{Daniel Ottoni. Strava mostra aumento do número de ciclistas, pós pandemia, em seis capitais. 
Disponível em: https://www.otempo.com.br/opiniao/esportivamente/strava-mostra-aumento-do-numero-de-ciclistas-pos-pandemia-em-seis-capitais-1.2571161.
Acesso em: 24 mai. 2023.}

Na primeira parte do texto o uso da expressão ``têm tudo para se
tornar'' representa:

\begin{escolha}

    \item otimismo do autor em relação manuntenção das mudanças positivas
  impostas pela pandemia.

    \item pessimismo do autor em relação à continuidade das atitudes
  positivas impulsionadas pela pandemia.

    \item posição neutra em relação às mudanças positivas que chegaram com a
  pandemia.

    \item condição para que as mudanças positivas impostas pela pandemia se
  perpetuem.

\end{escolha}

\coment{SAEB: Identificar os recursos de modalização em textos diversos.

a) Correta. O autor expressa otimismo por meio desse recurso de modalização.
b) Incorreta. A expressão o uso da expressão ``têm tudo para se
tornar'' expressa otimismo.
c) Incorreta.  Incorreta. A expressão o uso da expressão ``têm tudo para se
tornar'' expressa otimismo.
d) Incorreta. Incorreta. A expressão o uso da expressão ``têm tudo para se
tornar'' expressa otimismo.}

\num{14} Leia o texto abaixo para responder à questão. 

\begin{quote}

A Academia Americana de Medicina do Sono recomenda que crianças e
adolescentes entre 6 e 12 anos durmam ao menos nove horas por dia, mas
muitos não seguem a recomendação e os pesquisadores queriam entender
melhor o impacto de menos horas de descanso no desenvolvimento desses
jovens.

Os cientistas analisaram os dados de 8.323 crianças de 9 e 10 anos. Eles
dividiram esse conjunto em dois grupos das crianças que dormiam menos de
nove horas (4.181 participantes) e daquelas que dormiam pelo menos nove
horas (4.142 participantes) -- e avaliaram seus dados neurais e
comportamentais no início da pesquisa e após dois anos.

Os cientistas notaram também uma diferença significativa no volume de
massa cinzenta em 12 das 184 regiões avaliadas, padrão que se repetiu
depois de dois anos, sugerindo que algumas medidas estruturais são
suscetíveis a períodos insuficientes de sono.

\end{quote}

\fonte{O Tempo. Dormir menos de nove horas afeta cérebro e comportamento das crianças.
Disponível em:
https://www.otempo.com.br/mundo/dormir-menos-de-nove-horas-afeta-cerebro-e-comportamento-das-criancas-1.2707918
Acesso em: 24 mai. 2023.}

No texto acima a principal estratégia de argumentação se dá por meio de:

\begin{escolha}

    \item falas de especialistas e opiniões de pais.

    \item discurso de autoridade.

    \item exemplos históricos.

    \item dados de pesquisas.

\end{escolha}

\coment{SAEB: Avaliar a eficácia das estratégias argumentativas em textos de
diferentes gêneros.

a) Incorreta. O texto não contém falas de especialistas nem opinião de pais.
b) Incorreta. Não há citação de discursos de autoridade no texto.
c) Incorreta. Não há exemplos históricos no texto.
d) Correta. O texto se apoia em dados de pesquisas.}

\num{15} Leia o texto abaixo para responder à questão. 

%\includegraphics[width=2.84494in,height=2.84494in]{media/image6.png}

\begin{quote}

Desgostoso com a existência medíocre na sua pequena cidade natal, 
um belo dia, aí pelos seus vinte e dois anos, aceitara o convite de um
engenheiro inglês que, por aquelas bandas, andava a explorar
terras e terrenos diamantíferos. Todos julgavam que o ``seu'' mister 
andasse fazendo isso; a verdade, porém, é que o sábio inglês fazia estudos
desinteressados. Fazia puras e platônicas pesquisas geológicas e mineralógicas.
O diamante não era o fim dos seus trabalhos; mas o povo, que teimava em ver,
pelos arredores da cidade, o ventre da terra cheio de diamantes, não podia 
supor que um inglês que levava a catar pedras, pela manhã e até à noite,
tomando notas e com uns instrumentos rebarbativos, não estivesse com tais 
gatimonhas a caçar diamantes. Não havia meio do mister convencer à simplória 
gente do lugar que ele não queria saber de diamantes. 

\end{quote}

\fonte{Lima Barreto. Clara dos Anjos. 
Disponível em: http://www.dominiopublico.gov.br/download/texto/bn000048.pdf.
Acesso em: 25 mai. 2023.}

Levando em consideração o conjunto do parágrafo, a expressão que melhor 
sintetiza o respeito do povo para com o estrangeiro sonhador é

\begin{escolha}

    \item engenheiro inglês.

    \item ``seu'' mister.

    \item o sábio inglês. 

    \item um inglês que levava a catar pedras

\end{escolha}

\coment{SAEB: Avaliar a adequação das variedades linguísticas em contextos de uso.

a) Incorreta. A expressão ``engenheiro inglês'' é usada pelo narrador para
referir-se ao estrangeiro, não pela população local.
b) Correta. A expressão \textit{``seu'' mister} é usada pela população local
para referir-se ao estrangeiro. O uso de ``seu'' já manifesta o respeito 
a ele, que era considerado um sábio. 
c) Incorreta. A expressão ``o sábio inglês'' é usada pelo narrador para
referir-se ao estrangeiro, não pela população local.
d) Incorreta. A expressão ``um inglês que levava a catar pedras'', meramente
descritiva, é usada pelo narrador para referir-se ao estrangeiro e não expressa
o respeito da população a ele.}
