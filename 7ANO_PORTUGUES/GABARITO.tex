\chapter{Respostas}
\pagestyle{plain}
\footnotesize

\pagecolor{gray!40}

\colorsec{Módulo 1 – Treino}

\begin{enumerate}
\item
SAEB: Identificar teses, opiniões, posicionamentos explícitos e
argumentos em textos.

BNCC: EF67LP05 -- Identificar e avaliar teses/opiniões/posicionamentos explícitos 
e argumentos em textos argumentativos (carta de leitor, comentário, artigo de opinião,
resenha crítica etc.), manifestando concordância ou discordância.

a)Correta. Segundo o texto, motivadas por questões econômicas, as pessoas migram de áreas 
assoladas pela doença para outras, espalhando a doença.
b)Incorreta. No texto, alude-se à meta dos governos de erradicar a doença.
c)Incorreta. No texto, afirma-se que o mosquito pode migrar de uma área para outra.
d)Incorreta. No texto, pode-se ler a afirmação de que o mosquito não conhece fronteiras 
e por isso é capaz de migrar de uma região para outra.

\item
SAEB: Identificar o uso de recursos persuasivos em textos verbais e não
verbais.

BNCC: EF67LP07 -- Identificar o uso de recursos persuasivos em
textos argumentativos diversos (como a elaboração do título, escolhas
lexicais, construções metafóricas, a explicitação ou a ocultação de
fontes de informação) e perceber seus efeitos de sentido.

a) Incorreta. A afirmação apresentada não faz referência a nenhuma autoridade.
b) Correta. De acordo com a opinião expressa no texto, o argumento entre
aspas refere-se à atitude do governo em relação aos problemas escolares
que, segundo a autora, são insuficientes para enfrentar o problema.
c) Incorreta. As aspas não indicam destaque. 
d) Incorreta. O termo entre aspas não aparece em outros momentos do texto nem
faz referência a outros trechos, por isso, não representa ambiguidade.

\item
SAEB: Identificar teses, opiniões, posicionamentos explícitos e
argumentos em textos.

BNCC: EF67LP07 -- Identificar o uso de recursos persuasivos em textos
argumentativos diversos (como a elaboração do título, escolhas lexicais,
construções metafóricas, a explicitação ou a ocultação de fontes de
informação) e perceber seus efeitos de sentido.

a) Correta. Segundo o texto, os aspectos culturais e afetivos devem ser
levados em conta para que a produção e consumo de alimentos saudáveis
seja estimulada.
b) Incorreta. Se forem levados em conta os aspectos culturais e regionais
para a produção de legumes, verduras e frutas, as práticas alimentares
podem ser melhoradas. 
c) Incorreta. O texto não contém referência a essa afirmação, apenas cita a 
produção regional como uma saída possível.
d) Incorreta. As práticas alimentares devem ser resgatadas a fim de
promover a produção e distribuição de alimentos que compõem a cultura
local.

\end{enumerate}

\colorsec{Módulo 2 – Treino}

\begin{enumerate}

\item
SAEB: Identificar elementos constitutivos de gêneros de divulgação
científica.

BNCC: EF69LP02 -- Analisar e comparar peças publicitárias variadas
(cartazes, folhetos, outdoor, anúncios e propagandas em diferentes
mídias, spots, jingle, vídeos etc.), de forma a perceber a articulação
entre elas em campanhas, as especificidades das várias semioses e
mídias, a adequação dessas peças ao público-alvo, aos objetivos do
anunciante e/ou da campanha e à construção composicional e estilo dos
gêneros em questão, como forma de ampliar suas possibilidades de
compreensão (e produção) de textos pertencentes a esses gêneros.

a) Incorreta. Embora o texto contenha alguns vocábulos ligados à área médica, não se pode 
afirmar que as informações veiculadas sejam voltadas apenas ao público especializado.
b) Correta. O texto contém explicações sobre os benefícios do exercício físico para idosos.
c) Incorreta. O texto não contém citações de artigos científicos. Há apenas falas curtas de um
especialista no assunto.
d) Incorreta. O texto não contém narrativas pessoais sobre os processos inerentes ao envelhecimento.

\item
SAEB:Analisar a relação temática entre diferentes gêneros jornalísticos.

a) Incorreta. O primeiro exemplo traz uma notícia sobre o número de casos
e segundo exemplo traz indicações de ações para prevenir a
proliferação do mosquito.

b) Correta. O primeiro exemplo é uma
notícia sobre o número de casos de dengue em 2023; o segundo contém
indicações de ações de prevenção e combate à proliferação do
mosquito que transmite a doença.

c) Incorreta. Embora o segundo exemplo possa ser considerado um texto
informativo, o primeiro exemplo não é um texto de opinião.

d) Incorreta. O primeiro exemplo não traz dados científicos e nem possui
linguagem científica e técnica sobre o assunto.

\item
SAEB: Identificar formas de organização de textos normativos, legais
e/ou reivindicatórios.

BNCC: EF69LP27 -- Analisar a forma composicional de textos
pertencentes a gêneros normativos/ jurídicos e a gêneros da esfera
política, tais como propostas, programas políticos (posicionamento
quanto a diferentes ações a serem propostas, objetivos, ações previstas
etc.), propaganda política (propostas e sua sustentação, posicionamento
quanto a temas em discussão) e textos reivindicatórios: cartas de
reclamação, petição (proposta, suas justificativas e ações a serem
adotadas) e suas marcas linguísticas, de forma a incrementar a
compreensão de textos pertencentes a esses gêneros e a possibilitar a
produção de textos mais adequados e/ou fundamentados quando isso for
requerido.

a) Incorreta. Os trechos não são contraditórios. O segundo trecho
apenas trata de uma parcela da população em particular enquanto o
primeiro trata do conjunto da população brasileira.

b) Incorreta. Os trechos apresentam relação direta ao passo que a
população indígena faz parte da população brasileira e como tal também
tem direitos civis garantidos pela lei.

c) Incorreta. Os dois trechos versam sobre direitos essenciais, portanto
não tratam de questões distintas.

d)Correta. Os dois trechos apresentam relação de complementaridade, pois
as disposições presentes no segundo trecho apenas especificam direitos
dos povos indígenas e deveres do Estado brasileiro para com essa
população a fim de garantir as obrigações do Estado brasileiro dispostas
no Artigo 3º.

\end{enumerate}

\colorsec{Módulo 3 – Treino}

\begin{enumerate}

\item
SAEB: Inferir a presença de valores sociais, culturais e humanos em
textos literários.

BNCC: EF69LP44 -- Inferir a presença de valores sociais,
culturais e humanos e de diferentes visões de mundo, em textos
literários, reconhecendo nesses textos formas de estabelecer múltiplos
olhares sobre as identidades, sociedades e culturas e considerando a
autoria e o contexto social e histórico de sua produção.

a) Correta. O trecho faz alusão à passagem das quatro estações do ano
exemplificadas pelo ciclo natural da fruta murici, além de referir-se
a períodos de seca e cheia, que permitem inferir a passagem de cerca de um ano.
b) Incorreta. Por meio das características citadas, tais como ocorrência
de chuvas e seca, pode-se inferir que um ano se passou.
c) Incorreta. O trecho faz alusão a diversas características das estações
do ano, portanto mais de uma estação se passou, indicando a passagem
de um ano.
d) Incorreta. O trecho faz alusão à passagem das quatro estações do ano
exemplificadas pelo ciclo natural da fruta murici, além de referir-se
a períodos de seca e cheia, que permitem inferir a passagem de cerca de um ano.

\item
SAEB: Analisar a intertextualidade entre textos literários ou entre
estes e outros textos verbais ou não verbais.

BNCC: EF67LP27 -- Analisar, entre os textos literários e entreestes e outras manifestações artísticas (como cinema, teatro, música,
artes visuais e midiáticas), referências explícitas ou implícitas a
outros textos, quanto aos temas, personagens e recursos literários e
semióticos

a) Incorreta. Os traços de intertextualidade não se configuram como plágio.
b) Incorreta. Além do título, existem diversas alusões de Murilo Mendes ao 
poema de Gonçalves Dias.
c) Correta. O tipo de intertextualidade entre os dois poemas é chamado de
paródia, que pode satirizar, de criticar ou de homenagear o texto de referência.
Neste caso, Murilo Mendes critica o nacionalismo no Romantismo
brasileiro e junto ao movimento modernista propõe uma nova forma de
nacionalismo que valoriza a cultura brasileira.
d)  O poema de Gonçalves Dias foi escrito cerca de 100 anos antes,
portanto não pode ser baseado no poema de Murilo Mendes.

\item
SAEB: Analisar elementos constitutivos de textos pertencentes ao domínio
literário.

a) Incorreta. Na poesia concreta, pode haver combinação de palavras e imagens 
para a criação dos sentidos do poema.
b) Incorreta. Não se verificam versos rimados e estrofes tradicionais na poesia
visual. 
c) Incorreta. As imagens podem fazer parte da constituição do poema e contribuir
para a produção de sentido.
d) Correta. A disposição de letras e palavras na página é, de fato, recurso de
produção de sentidos do poema.
\end{enumerate}


\colorsec{Módulo 4 – Treino}

\begin{enumerate}

\item
SAEB: Analisar efeitos de sentido produzido pelo uso de formas de
apropriação textual (paráfrase, citação etc.).

BNCC: EF69LP43 -- Identificar e utilizar os modos de introdução de
outras vozes no texto -- citação literal e sua formatação e paráfrase
--, as pistas linguísticas responsáveis por introduzir no texto a
posição do autor e dos outros autores citados (``Segundo X; De acordo
com Y; De minha/nossa parte, penso/amos que''...) e os elementos de
normatização (tais como as regras de inclusão e formatação de citações e
paráfrases, de organização de referências bibliográficas) em textos
científicos, desenvolvendo reflexão sobre o modo como a
intertextualidade e a retextualização ocorrem nesses textos.

a)Incorreta. O texto não apresenta a voz dos usuários de smartphones.
b)Incorreta. O texto apresenta as vozes do criador dos usuários de
smartphones. 
c) Correta. O texto apresenta duas vozes: a do autor e a do criador
do primeiro telefone móvel.
d) Incorreta. O texto não apresenta a voz dos usuários de smartphones.

\item
SAEB: Analisar efeitos de sentido produzido pelo uso de formas de
apropriação textual (paráfrase, citação etc.).

BNCC: EF69LP43 -- Identificar e utilizar os modos de introdução de
outras vozes no texto -- citação literal e sua formatação e paráfrase
--, as pistas linguísticas responsáveis por introduzir no texto a
posição do autor e dos outros autores citados (``Segundo X; De acordo
com Y; De minha/nossa parte, penso/amos que''...) e os elementos de
normatização (tais como as regras de inclusão e formatação de citações e
paráfrases, de organização de referências bibliográficas) em textos
científicos, desenvolvendo reflexão sobre o modo como a
intertextualidade e a retextualização ocorrem nesses textos.

a) Incorreta. A voz do autor não pede uso de aspas. 
b) Incorreta. No texto, as aspas não servem para dar destaque e o trecho isolado entre 
elas não é uma explicação.
c) Incorreta. A citação a um artigo não é citada entre aspas.
d) Correta. O trecho entre aspas indica a citação de um especialista em 
vida saudável na revista The Lancet.

\item
SAEB: Analisar os efeitos de sentido decorrentes dos mecanismos de construção
de textos jornalísticos/midiáticos.

a)Incorreta. A necessidade de uso massivo de máscaras não produz o efeito de sentido
solicitado no enunciado. 
b)Correta. A possibilidade de impedir a segunda onda pode sensibilizar o leitor.
c)Incorreta. A segunda onda é apresentada como fato; a possibilidade de
impedi-la é que sensibiliza o leitor.
d)Incorreta. A referência a um estudo confere credibilidade ao texto, mas 
não sensibiliza o leitor quanto à necessidade prevenção contra a Covid-19.

\end{enumerate}

\colorsec{Módulo 5 – Treino}

\begin{enumerate}

\item
SAEB: Distinguir fatos de opiniões em textos.

a) Incorreta. O trecho não contém opinião. 
b) Correta. O texto apresenta dados de pesquisas que comprovam o fato de
que o valor das cestas básicas diminuiu.
c) Incorreta. Apesar de apresentar os dados de pesquisas, o trecho não
contém opinião do autor.
d) Incorreta. O trecho não contém opinião do autor sobre o tema.

\item
SAEB:Inferir informações implícitas em distintos textos.
BNCC: EF67LP04 -- Distinguir, em segmentos descontínuos de
textos, fato da opinião enunciada em relação a esse mesmo fato.

a) Incorreta. Não há referências no texto às desigualdades sociais como
opinião fundamentada em pesquisa científica.  
b) Incorreta. Na coerência interna do texto, as desigualdades sociais são
apresentadas como fato. 
c) Correto. Na coerência interna do texto, as desigualdades sociais são
apresentadas como fato que dificulta a garantia de direitos.
d) Incorreta. Na coerência interna do texto, as desigualdades sociais são
apresentadas como fato.

\item
SAEB: Distinguir fatos de opiniões em textos.
BNCC: EF67LP04 -- Distinguir, em segmentos descontínuos de textos,
fato da opinião enunciada em relação a esse mesmo fato.

a)Incorreta. É fato que crianças, grávidas e idosos são mais suscetíveis a
doenças. 
b)Incorreta. O entrevistado entende que o carnaval é uma celebração 
da vida, mas essa é apenas sua opinião: outras pessoas podem entender essa
festa popular de outras maneiras. Da mesma maneira, é fato que 700 mil
pessoas perderam a vida por covid no Brasil.  
c) Correta. A matéria apresenta dados sobre os óbitos por covid no
país. Trata-se, portanto, de fatos. No trecho entre aspas, ao afirmar que 
``não é razoável que o carnaval signifique um risco para as coletividades'',
o entrevistado emite sua opinião. 
d) Incorreta. É fato que 700 mil pessoas perderam a vida por covid 
no Brasil.

\end{enumerate}

\colorsec{Módulo 6 – Treino}

\begin{enumerate}

\item
SAEB: Inferir, em textos multissemiótico, efeitos de humor, ironia e/ou
crítica.

a) Correta. Nos primeiros períodos do parágrafo, a repetição de 
``nem todos gostavam'', seguida de expressões que se referem à brutalidade
com que os escravizados eram tratados, dirige o olhar do leitor à violência
por eles vivida, naturalizando a linguagem da brutalidade e resultando na 
ironia, por meio da qual Machado de Assis critica a escravidão.
b) Incorreta. Embora haja passagens descritivas da realidade no fragmento, 
a crítica de Machado de Assis só se constitui por meio das ironias no conjunto. 
c) Incorreta. A nostalgia observável em alguns trechos não é suficiente para
estabelecer a crítica, que só terá lugar por meio da naturalização da linguagem
violenta da escravidão. 
d) Incorreta. Não se observa, no texto, a valorização dos escravizados. Evidentemente
Machado de Assis reproduz, no conjunto, a linguagem que os depreciava, evidenciando
a naturalização da brutalidade contra eles.

\item
SAEB: Inferir, em textos multissemiótico, efeitos de humor, ironia e/ou
crítica

BNCC: EF69LP05 -- Inferir e justificar, em textos multissemióticos --
tirinhas, charges, memes, gifs etc. --, o efeito de humor, ironia e/ou
crítica pelo uso ambíguo de palavras, expressões ou imagens ambíguas, de
clichês, de recursos iconográficos, de pontuação etc.

a) Incorreta. A força da campanha é dada pelo jogo entre as palavras
``alvo'' e ``foco''.
b)Incorreta. A força da campanha é dada pelo jogo entre as palavras
``alvo'' e ``foco''.
c)Incorreta. A força da campanha é dada pelo jogo entre as palavras
``alvo'' e ``foco''.
d) Correta. A força da campanha é dada pelo jogo entre as palavras
``alvo'' e ``foco''.

\item
SAEB: Inferir, em textos multissemiótico, efeitos de humor, ironia e/ou
crítica

Bncc: EF69LP05 -- Inferir e justificar, em textos multissemióticos --
tirinhas, charges, memes, gifs etc. --, o efeito de humor, ironia e/ou
crítica pelo uso ambíguo de palavras, expressões ou imagens ambíguas, de
clichês, de recursos iconográficos, de pontuação etc.

a) Correta. No meme, a palavra ``casa'' pode assumir dois sentidos. O primeiro
é figurado: ``casa'' equivale a ``família''; o segundo é literal: a educação
viria da \textit{casa} propriamente dita, que se opõe ao apartamento. A criança 
estaria, assim, ironizando e relativizando, de forma sagaz, a máxima de que 
``a educação vem de casa'' (expressão na qual ``casa'' assume o primeiro dos 
sentidos). Morando em apartamento, o garoto considera o segundo sentido da palavra
``casa'', dispensa a si mesmo da educação e faz a bagunça retratada na imagem.    
b) Incorreta. O meme não contém crítica à falta de moradia.
c) Incorreta. Não há elementos que permitam inferir que o meme contém propaganda 
subliminar. 
d)Incorreta. Não há elementos que permitam inferir que o meme questiona a qualidade 
de vida dos moradores de apartamentos.
\end{enumerate}

\colorsec{Módulo 7 – Treino}

\begin{enumerate}

\item
SAEB: Analisar marcas de parcialidade em textos jornalísticos.

a) Incorreta. Entrevistas com especialistas pretendem obter o efeito da 
imparcialidade.
b) Incorreta. A apresentação de diferentes pontos de vista pretende obter
o efeito da imparcialidade. 
c) Correta. A presença de adjetivos valorativos é marca de parcialidade.
d) Incorreta. A presença de fontes e fatos pretende obter
o efeito da imparcialidade.

\item
SAEB:Avaliar diferentes graus de parcialidade em textos jornalísticos.

BNCC: EF67LP04 -- Distinguir, em segmentos descontínuos de textos,
fato da opinião enunciada em relação a esse mesmo fato.

a) Incorreta. A opinião do jornalista não é um recurso de imparcialidade.
b)Incorreta. O autor não se opõe ao Acordo de Paris, citado como referência às metas
para reversão da situação climática.
c)Correta. O uso de gráficos e recursos similares trazem maior clareza, possibilitando
ao leitor formar suas próprias opiniões
d)Incorreta. O uso de expressões que pretendem sensibilizar o leitor são recursos de
persuasão importantes, mas não conferem imparcialidade ao texto.

\item
SAEB:Analisar marcas de parcialidade em textos jornalísticos.

BNCC: EF67LP04 -- Distinguir, em segmentos descontínuos de textos,
fato da opinião enunciada em relação a esse mesmo fato.

a)Correta. Já no título a notícia apresenta argumentos que levam a crer na
necessidade de reajuste.
b) Incorreta. Essas informações não representam a opinião do veículo e 
podem ser considerados recursos para atribuir veracidade à notícia.
c) Incorreta. Essas informações não representam a opinião do veículo e 
podem ser considerados recursos para atribuir veracidade à notícia.
d) Incorreta. Essas informações não representam a opinião do veículo e 
podem ser considerados recursos para atribuir veracidade à notícia.

\end{enumerate}

\colorsec{Módulo 8 – Treino}

\begin{enumerate}

\item
SAEB: Analisar os efeitos de sentido produzidos pelo uso de modalizadores em textos diversos.

BNCC: EF07LP14 -- Identificar, em textos, os efeitos de sentido do
uso de estratégias de modalização e argumentatividade.

a) Incorreta. No trecho em que se insere, a expressão indicada opõe a afirmação anterior, sobre
a obrigatoriedade do uso de máscaras decretada pelo governo, e a posterior, ``ainda há dúvidas 
de parte da população quanto à necessidade e ao benefício do seu uso''. 
b) Incorreta. No trecho em que se insere, a expressão indicada opõe a afirmação anterior, sobre
a obrigatoriedade do uso de máscaras decretada pelo governo, e a posterior, ``ainda há dúvidas 
de parte da população quanto à necessidade e ao benefício do seu uso''.
c) Correta. No trecho em que se insere, a expressão indicada opõe a afirmação anterior, sobre
a obrigatoriedade do uso de máscaras decretada pelo governo, e a posterior, ``ainda há dúvidas 
de parte da população quanto à necessidade e ao benefício do seu uso''.
c)Incorreta. No trecho em que se insere, a expressão indicada opõe a afirmação anterior, sobre
a obrigatoriedade do uso de máscaras decretada pelo governo, e a posterior, ``ainda há dúvidas 
de parte da população quanto à necessidade e ao benefício do seu uso''.

\item
SAEB:Analisar os efeitos de sentido dos tempos, modos e/ou vozes verbais
com base no gênero textual e na intenção comunicativa.

a) Incorreta. A forma verbal ``deve'', no contexto em que se insere, expressa possibilidade.
b) Incorreta. A forma verbal ``deve'', no contexto em que se insere, expressa possibilidade.
c) Incorreta. A forma verbal ``deve'', no contexto em que se insere, expressa possibilidade.
d) Correta. A forma verbal ``deve'', no contexto em que se insere, expressa possibilidade.

\item
SAEB: Analisar os efeitos de sentido dos tempos, modos e/ou vozes
verbais com base no gênero textual e na intenção comunicativa.

BNCC: EF69LP04 -- Identificar e analisar os efeitos de sentido
que fortalecem a persuasão nos textos publicitários, relacionando as
estratégias de persuasão e apelo ao consumo com os recursos
linguístico-discursivos utilizados, como imagens, tempo verbal, jogos de
palavras, figuras de linguagem etc., com vistas a fomentar práticas de
consumo conscientes.

a) Incorreta. No contexto em que se insere, a forma verbal no gerúndio tem valor condicional: 
``Água: se soubermos usar, não vai faltar''. 
b) Incorreta. No contexto em que se insere, a forma verbal no gerúndio tem valor condicional: 
``Água: se soubermos usar, não vai faltar''. 
c) Incorreta. No contexto em que se insere, a forma verbal no gerúndio tem valor condicional: 
``Água: se soubermos usar, não vai faltar''.
d) Incorreta. No contexto em que se insere, a forma verbal no gerúndio tem valor condicional: 
``Água: se soubermos usar, não vai faltar''.

\end{enumerate}

\colorsec{Módulo 9 – Treino}

\begin{enumerate}

\item
SAEB:Avaliar a eficácia das estratégias argumentativas em textos de
diferentes gêneros.

a) Incorreta. Na frase em destaque ocorre metonímia.

b) Incorreta. Na frase em destaque ocorre metonímia.

c) Incorreta. Na frase em destaque ocorre metonímia.

d) Correta. Na frase em destaque ocorre metonímia, isto é, o uso de um nome no lugar de outro. 
No caso, ao usar ``Brasil'', o autor se referia ao governo brasileiro.

\item
SAEB: Avaliar a eficácia das estratégias argumentativas em textos de
diferentes gêneros.

a) Incorreta. A palavra que mais se repete no texto é a conjunção ``como'', que contribui
para as comparações entre a personagem e a natureza.
b) Incorreta. As hipérboles do texto contribuem
para as comparações entre a personagem e a natureza.
c) Correta. As descições de Iracema são todas compostas por meio de comparações:
seus cabelos são ``mais negros que a asa da graúna e mais longos
que seu talhe de palmeira''. Seu sorriso é mais doce que o favo da jati,
e seu hálito perfumado cheira mais do que a baunilha. 
d) Incorreta. No texto, a natureza brasileira não é depreciada.

\item
SAEB: Analisar o uso de figuras de linguagem como estratégia argumentativa.

a) Incorreta. O narrador insiste nas ambições de João Romão, não em sua simplicidade.
b) Incorreta. As galinhas de João Romão e sua companheira davam muitos ovos, o que significa 
que ele e a companheira não viviam em miséria.
c) Incorreta. João Romão não se resigna: ele é tomado pela ``a febre de possuir'' para ``aumentar 
os bens''.
d) Correta. O conjunto do trecho insiste em que ``a febre de possuir'' é a personificação da 
ambiçao de João Romão, cujos atos ``visavam um interesse pecuniário'', que só queria ``aumentar 
os bens'' e ``reduzir tudo a moeda''.

\end{enumerate}

\colorsec{Módulo 10 – Treino}

\begin{enumerate}

\item
SAEB: Analisar os processos de referenciação lexical e pronominal.

BNCC: EF07LP12 -- Reconhecer recursos de coesão referencial:
substituições lexicais (de substantivos por sinônimos) ou pronominais
(uso de pronomes anafóricos -- pessoais, possessivos, demonstrativos.

a) Incorreta. O termo se refere à expressão ``ensaios clínicos randomizados controlados''.
b) Incorreta. O termo se refere à expressão ``ensaios clínicos randomizados controlados''.
c) Correta. O termo se refere à expressão ``ensaios clínicos randomizados controlados''. 
d) Incorreta. O termo se refere à expressão ``ensaios clínicos randomizados controlados''.

\item
SAEB:Analisar os mecanismos que contribuem para a progressão textual.

BNCC: EF07LP12 -- Reconhecer recursos de coesão referencial:
substituições lexicais (de substantivos por sinônimos) ou pronominais
(uso de pronomes anafóricos -- pessoais, possessivos, demonstrativos.

a) Correta. O termo ``o texto'' se refere a ``a lei''.
b) Incorreta. O termo ``o texto'' se refere a ``a lei''.  
c) Incorreta. O termo ``o texto'' se refere a ``a lei''. 
d) Incorreta. O termo ``o texto'' se refere a ``a lei''.

\item
SAEB: Analisar os processos de referenciação lexical e pronominal.

BNCC: EF07LP12 -- Reconhecer recursos de coesão referencial:
substituições lexicais (de substantivos por sinônimos) ou pronominais
(uso de pronomes anafóricos -- pessoais, possessivos, demonstrativos.

a) Incorreta. ``Eles'' e ``-los'' se referem ao antecedente ``contos tradicionais''.
b) Incorreta. ``Eles'' e ``-los'' se referem ao antecedente ``contos tradicionais''.
c) Correta. ``Eles'' e ``-los'' se referem ao antecedente ``contos tradicionais''.
d) Incorreta. ``Eles'' e ``-los'' se referem ao antecedente ``contos tradicionais''.

\end{enumerate}

\colorsec{Módulo 11 – Treino}

\begin{enumerate}

\item
SAEB: Avaliar a adequação das variedades linguísticas em contextos de uso.

a) Incorreta. A transcrição da fala popular se observa em ``fio dela'', equivalente a ``filho dela''.
b) Incorreta. A transcrição da fala popular se observa em ``fio dela'', equivalente a ``filho dela''.
c) Incorreta. A transcrição da fala popular se observa em ``fio dela'', equivalente a ``filho dela''.
d) Correta. A transcrição da fala popular se observa em ``fio dela'', equivalente a ``filho dela''.

\item
Saeb:Avaliar a adequação das variedades linguísticas em contextos de
uso.

a) Incorreta. O texto da questão é jornalístico. 
b) Incorreta. O texto da questão é jornalístico. 
c) Correta. O texto da questão é jornalístico, por isso o autor optou pelo uso da norma-padrão. 
d) Incorreta. O texto da questão é jornalístico.

\item
SAEB: Analisar as variedades linguísticas em textos.

a) Incorreta. A transcrição da fala popular se observa em ``A modo qui tô conhecendo mecê''.
b) Incorreta. A transcrição da fala popular se observa em ``A modo qui tô conhecendo mecê''. 
c) Incorreta. A transcrição da fala popular se observa em ``A modo qui tô conhecendo mecê''. 
d) Incorreta. A transcrição da fala popular se observa em ``A modo qui tô conhecendo mecê''.

\end{enumerate}

\colorsec{Simulado 1}



\colorsec{Simulado 2}

\colorsec{Simulado 3}

\colorsec{Simulado 4}