\chapter{Respostas}
\pagestyle{plain}
\footnotesize

\pagecolor{gray!40}

\colorsec{Matemática – Módulo 1 – Treino}

\begin{enumerate}
\item

BNCC: EF67LP05 -- Identificar e avaliar teses/opiniões/posicionamentos
explícitos e argumentos em textos argumentativos (carta de leitor,
comentário, artigo de opinião, resenha crítica etc.), manifestando
concordância ou discordância.

a) Incorreta. O texto afirma que as crianças aprendem a consumir de
forma inconsequente, porém não é essa a principal razão do desequilíbrio
global.

b) Incorreta. Critérios e valores distorcidos de consumo são passados
para as pessoas desde a infância, e não critérios e valores distorcidos
de modo geral, como dá a entender a afirmativa.

c) Correta. O fato de explorarmos de forma irresponsável o meio ambiente
ao longo de décadas é o principal responsável pelo desequilíbrio global.

d) Incorreta. Critérios e valores distorcidos com relação ao consumo (e
não de modo geral) são de fato problemas de ordem ética, econômica e
social, porém não são os responsáveis diretos pelo desequilíbrio global.


