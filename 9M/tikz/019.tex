\documentclass[border=3.14mm,tikz]{standalone}
\usepackage{tikz}
\usetikzlibrary{calc}
\begin{document}

\def\r{4}
\begin{tikzpicture}
\draw[fill=red!30,draw=red,line width=1.5pt] (0,0) circle (\r) node[inner sep=1pt,fill] (O) {};
\draw[-latex] (0,0) -- (\r,0) node[midway,below] {\huge Raio};
\end{tikzpicture}


\begin{tikzpicture}
\draw[fill=red!30,draw=red,line width=1.5pt] (0,0) circle (\r);
\draw[latex-latex] (0:\r) -- (180:\r) node[midway,above] {\huge Diâmetro};
\end{tikzpicture}


\begin{tikzpicture}
\draw[fill=red!30,draw=red,line width=1.5pt] (0,0) circle (\r) node[inner sep=1pt,fill] (O) {};
\draw[latex-latex] (30:\r) -- (160:\r) node[midway,above] {\huge Corda};
\end{tikzpicture}


\begin{tikzpicture}
\draw[fill=red!30,draw=red,line width=1.5pt] (0,0) circle (\r) node[inner sep=1pt,fill] (O) {};
\draw[latex-latex] ($(30:\r)!-1cm!(160:\r)$) -- ($(225:\r)!-1cm!(45:\r)$) node[midway,above left] {\huge Secante};
\end{tikzpicture}


\begin{tikzpicture}

% definir o centro do círculo
\coordinate (O) at (0, 0);

% desenhar o círculo com raio 3
\draw (O) circle (\r);
\draw[fill=red!30,draw=red,line width=1.5pt] (0,0) circle (\r) node[inner sep=1pt,fill] (O) {};

% desenhar a tangente. Primeiro, definimos um ponto na borda do círculo.
\coordinate (A) at (30:\r);

% em seguida, calcula-se o ponto onde a tangente toca a borda do círculo
\coordinate (T) at ($(A)!3cm!90:(O)$);
\coordinate (T2) at ($(A)!-3cm!90:(O)$);

% e finalmente desenha-se a tangente
\draw[-latex] (A) -- (T);
\draw[-latex] (A) -- (T2);

% Colocar marcadores nos pontos importantes
%\fill[black] (O) circle (2pt) node[below left] {$O$};
\fill[black] (A) circle (2pt) node[above right] {\huge Tangente};

\end{tikzpicture}



\end{document}