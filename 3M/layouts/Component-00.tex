\starttext

\pagebreak
\page[blank]

\startpositioning
   \position(0,0){\externalfigure
                 [grafismos_PDF/pag3.pdf][width=1.02\paperwidth]}
\stoppositioning

\blank[4cm]

\section[title={ARESENTAÇÃO DO VOLUME},reference={aresentação-do-volume}]

Neste volume, a proposta é discutir, de maneira mais geral, o
{\em bullying} e, de maneira mais específica, o {\em cyberbullying},
práticas que têm se tornado cada vez mais evidentes em nossa sociedade.
Duas são as colunas que sustentam a proposta deste volume. Em primeiro
lugar, parte fundamental da conscientização com o {\em bullying} é
compreendê-lo e entender quão problemático é. Por outro lado, os alunos
que são vítimas de atitudes desse tipo devem se sentir confortáveis e
acolhidos para falar do assunto sem constrangimento.

\section[title={MENSAGEM DO AUTOR},reference={mensagem-do-autor}]

Sou uma psicopedagoga especializada em comportamento na infância e na
adolescência, com experiência fora e dentro da sala de aula. Escrevi
este livro com o objetivo de trazer informações valiosas e ferramentas
práticas para da área de saúde mental, a fim de combater a violência que
tanto prejudica nossas crianças e nossos adolescentes.


\startMPclip{circleclip}
     clip currentpicture to fullcircle shifted (.5,.5)
       xscaled \width yscaled \height ;
\stopMPclip
   

\inright{\clip[width=3.5cm,height=3.5cm,mp=circleclip]{\externalfigure[media/celina.png][width=3.5cm]}}


%\inmargin{
%\externalfigure
%              [media/celina.png]
%              [width=4cm,
%              frame=on,
%              framecolor=Orange,
%              frameoffset=3pt,
%              rulethickness=4pt,
%              framecorner=round]
%}

Como
profissional, tenho vasta experiência no tratamento de comportamentos
agressivos e violentos em crianças e jovens. Minha abordagem é baseada
em uma compreensão aprofundada da psicologia infantil e juvenil, e busco
sempre me manter atualizada sobre as mais recentes pesquisas e técnicas.

\pagebreak
\page[blank]

\stoptext