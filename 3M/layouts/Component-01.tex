\title[title={O que é o {\em bullying}?},reference={capítulo-1-o-que-é-o-bullying}]

{\externalfigure[media/image1.jpeg][width=1.6\textwidth]}

%{\bf https://br.freepik.com/fotos-gratis/pessoa-que-sofre-de-bullying_20146576.htm\#query=bullying&position=2&from_view=search&track=robertav1_2_sidr}

\startitemize[asterisk,packed][color=Red2]
\item
  Como você descreveria a cena que aparece na imagem desta página?
\item
  O que você acha que as quatro pessoas ao fundo estão dizendo?
\item
  Como você imagina que a garota que aparece à frente está se sentindo?
\item
  Você já vivenciou ou presenciou uma situação como a representada? Como
  foi?
\stopitemize

\page

\section[title={Definindo o {\em bullying}},reference={definindo-o-bullying}]

Embora a definição do {\em bullying} e do {\em cyberbullying} seja
relativamente simples, a compreensão completa desses fenômenos requer um
exame mais aprofundado de suas causas, consequências e formas de
prevenção. Isso porque eles não são fenômenos restritor ou autolimitados
-- pelo contrário, eles se manifestam ao redor de todo o mundo e geram
consequências, muitas vezes, gravíssimas.

\startmarginblock
\startPalavra
{\bf bullying:} Comportamento intencional e repetido de agressão
física, verbal ou psicológica, que ocorre em um contexto desigual de
poder, em que uma ou mais pessoas causam dor, angústia ou sofrimento a
outra pessoa de forma intencional e constante.

{\bf cyberbullying:} Uma forma de {\em bullying} que ocorre por
meio da internet e de outras tecnologias digitais. Consiste em
comportamentos agressivos, intencionais e repetidos, que visam a causar
dano, dor ou constrangimento a uma pessoa ou grupo, usando meios como
redes sociais, mensagens de texto, e-mails, fóruns, jogos {\em on-line}
e outras plataformas digitais. O {\em cyberbullying} pode incluir
ameaças, difamação, exclusão social, divulgação de informações pessoais
ou imagens constrangedoras, entre outras formas de violência virtual.
\stopPalavra
\stopmarginblock

Em primeiro lugar, é importante destacar que o {\em bullying} e o
{\em cyberbullying} podem ocorrer em qualquer ambiente onde haja
interação social, como escolas, locais de trabalho, comunidades virtuais
e até mesmo em famílias. Esses comportamentos podem ser desencadeados
por uma variedade de fatores, como diferenças culturais, preconceitos de
gênero ou orientação sexual, rivalidades pessoais, entre muitos outros
fatores e situações.

Por isso, é fundamental abordar essas questões de maneira ampla, com uma
visão que vá além do indivíduo e considere os aspectos sociais,
políticos e culturais envolvidos. Todas as pessoas -- especialmente
aquelas em idade escolar -- provavelmente já presenciaram situações de
{\em bullying}. Isso mostra quão disseminada é a prática -- além de
mostrar a gravidade do problema.

\placefigure{No {\em cyberbullying}, a rede mundial de computadores é
usada como uma espécie de fortaleza por trás da qual os agressores se
escondem.}{\externalfigure[media/image2.jpeg][width=\textwidth]}

%<https://br.freepik.com/vetores-gratis/conceito-de-cyber-bullying_8967116.htm\#query=bullying&position=4&from_view=search&track=robertav1_2_sidr>>

Outro ponto importante é que o {\em bullying} e o {\em cyberbullying}
não são eventos isolados, mas sim um padrão de comportamento que se
repete ao longo do tempo. Isso significa que os agressores não agem
impulsivamente ou sem intenção, mas com uma estratégia bem definida para
atacar e prejudicar suas vítimas. Por essa razão, é fundamental prestar
atenção aos sinais: mudanças no comportamento, diminuição no rendimento
escolar e exclusão de determinadas atividades, por exemplo. A partir
disso, é preciso agir rapidamente para interromper o comportamento
agressivo.

Além disso, é importante enfatizar que o {\em bullying} e o
{\em cyberbullying} têm um impacto significativo na saúde mental e
emocional das vítimas. Esses efeitos podem ser especialmente intensos
para crianças e adolescentes em fase de desenvolvimento, cujas
habilidades sociais e emocionais ainda não estão completamente formadas.
Por essa razão, é fundamental fornecer apoio e intervenção apropriados
para as vítimas, a fim de minimizar os efeitos negativos dessas
experiências.

\startmarginblock
{\placefigure{Os jovens tendem a querer participar de comportamentos de
massa na tentativa de se encaixar em padrões socialmente
estabelecidos.}{\externalfigure[media/image3.jpeg][width=4cm]}}
\stopmarginblock

O {\em bullying} é um assunto sério e delicado que deve ser abordado com
cuidado e sensibilidade, especialmente com alunos do sexto ano, que
estão em uma fase importante de desenvolvimento social e emocional.
Trata-se de uma abordagem extremamente importante para ajudá-los a
entender o que é, como identificar e como evitar comportamentos
inadequados que possam ser prejudiciais a si mesmos ou aos outros.
Estabeleça uma relação de confiança com os alunos antes de iniciar a
conversa. Isso pode ser feito por meio de uma roda de conversa em que
eles compartilhem experiências, sentimentos e sensações em ambiente
seguro e de acolhimento. Encoraje os alunos a serem empáticos e a se
colocarem no lugar das vítimas de {\em bullying} -- se é que nunca o
foram, de fato. Explique que a empatia é fundamental para prevenir o
{\em bullying} e criar um ambiente escolar mais respeitoso.

Finalmente, é importante destacar que a prevenção do {\em bullying} e do
{\em cyberbullying} requer uma abordagem multidisciplinar, que envolva
pais, educadores, profissionais de saúde e segurança, além da sociedade
em geral. Isso inclui a implementação de políticas e programas de
prevenção eficazes, a criação de espaços seguros e inclusivos para as
vítimas, a promoção da empatia e da resolução pacífica de conflitos e o
uso responsável da tecnologia e das redes sociais. Somente por meio de
um esforço conjunto e coordenado poderemos superar o {\em bullying} e o
{\em cyberbullying} e construir um mundo mais justo e compassivo, ou
seja, em que haja mais compaixão.

\startPAPEL
\color[darkred]{\bfb Papel e caneta}


Os adolescentes muitas vezes sentem uma grande pressão para se encaixar
nos padrões sociais estabelecidos pela sociedade. Essa pressão pode vir
de muitas fontes diferentes, incluindo os amigos, a mídia, a família e a
escola. A necessidade de pertencer a um grupo e de ser aceito pelos
outros pode ser tão forte que muitos adolescentes se conformam com os
comportamentos e as atitudes do grupo, mesmo que isso signifique ir
contra suas próprias crenças e valores.

As razões por trás dessa necessidade de se encaixar são complexas. Em
parte, pode ser devido ao desenvolvimento do cérebro adolescente, que
muitas vezes valoriza a aprovação social acima de outras formas de
recompensa. Além disso, muitos adolescentes ainda estão descobrindo quem
são e o que querem da vida, o que pode torná-los mais vulneráveis à
influência dos outros.

\MyIcon{pen-alt}{darkred}

Como você acha que essa pressão para se encaixar nos padrões pode se
relacionar com o problema do {\em bullying} e do {\em cyberbullying}?
Numa folha pautada, escreva uma breve reflexão sobre isso, em um texto
de, no máximo, 20 linhas.

Sobre a seção Papel e caneta, vale ressaltar que a necessidade
adolescente de se encaixar pode ter consequências negativas para os
adolescentes. Eles podem se sentir pressionados a fazer o que não querem
fazer ou a adotar comportamentos perigosos a fim de se encaixar em um
grupo. Eles também podem se sentir isolados ou rejeitados se não
conseguirem se encaixar, o que pode levar a problemas de saúde mental,
como depressão e ansiedade. É importante que os adolescentes saibam que
eles são únicos e têm direito a respeito e proteção, independentemente
de sua conformidade com os padrões sociais. Os adultos que trabalham com
adolescentes, como pais e educadores, podem ajudar a promover a
autoestima e a autoconfiança dos adolescentes, encorajando-os a
expressar suas opiniões e a serem fiéis a si mesmos. Isso pode ajudar a
reduzir a pressão de se encaixar e a promover um ambiente escolar e
social mais saudável e inclusivo para todos.
\stopPAPEL

\section[title={Os tipos da
agressão},reference={os-tipos-da-agressão}]

O {\em bullying} e o {\em cyberbullying} são formas de violência que
podem assumir muitas facetas diferentes. Embora a maioria das pessoas
associe o {\em bullying} com a agressão física, como empurrões ou socos,
ele também pode assumir formas verbais e psicológicas, como insultos,
difamação e exclusão social. Da mesma forma, o {\em cyberbullying} pode
assumir diferentes formas, como assédio {\em on-line}, compartilhamento
de informações privadas sem consentimento ou disseminação de rumores e
boatos prejudiciais.

Explique o que é {\em bullying} de maneira clara e objetiva, enfatizando
que é um comportamento intencional e repetido que pode ser físico,
verbal ou psicológico. Ajude os alunos a identificarem exemplos de
comportamentos que podem ser considerados {\em bullying}, como
xingamentos, exclusão social, agressão física ou virtual, entre outros.
Discuta com os alunos estratégias para prevenir o {\em bullying}, como
comunicar-se de forma assertiva, buscar ajuda de adultos confiáveis e
desenvolver habilidades sociais e emocionais. Reforce a importância de
respeitar as diferenças e promover a inclusão de todos os alunos,
independentemente de suas características pessoais. Finalmente, esteja
disponível para seus alunos caso precisem de ajuda ou apoio em relação
ao {\em bullying}.

<Organizar essas definições em uma espécie de boxe com fundo levemente
colorido, que se destaque do texto geral, mas que também pareça parte
dele>

\startitemize[bookmark][color=Orange1,afterhead={\blank[small]}]
\starthead{\bf {\em Bullying} verbal}

é uma forma comum de {\em bullying} que
envolve o uso de palavras ou comportamentos que são destinados a magoar
ou humilhar a vítima. Exemplos incluem insultos, difamação, gozação,
ameaças e xingamentos. O {\em bullying} verbal pode ser direto, quando o
agressor ataca a vítima diretamente, ou indireto, quando o agressor usa
outras pessoas para intimidar a vítima.

\starthead{\bf {\em bullying} físico} envolve comportamentos físicos que são
destinados a ferir ou machucar a vítima. Isso pode incluir empurrões,
socos, chutes, beliscões, puxões de cabelo e outros tipos de agressão
física. O {\em bullying} físico também pode ser direto ou indireto.

\starthead{\bf {\em bullying} social} ou {\em bullying} relacional, envolve a
exclusão social da vítima e a disseminação de boatos e rumores
prejudiciais. Exemplos incluem recusa em incluir a vítima em atividades
em grupo, difamação nas redes sociais e outras formas de afastamento
social. O {\em bullying} social pode ser especialmente prejudicial
porque muitas vezes é difícil de identificar e abordar.

\starthead{\bf {\em sexting}} é uma forma de {\em cyberbullying} que envolve o
compartilhamento de imagens ou vídeos constrangedores sem o
consentimento da vítima. Isso pode ter consequências graves, incluindo a
disseminação de imagens íntimas em toda a internet, assédio e humilhação
pública, e até mesmo perda de emprego ou oportunidades educacionais.

\starthead{\bf {\em stalking}} ou assédio virtual, é uma forma de
{\em cyberbullying} que envolve o monitoramento e a perseguição
constante de uma vítima, tanto {\em on-line} como {\em off-line}. Isso
pode incluir o envio de mensagens excessivas e ameaçadoras, a criação de
perfis falsos para entrar em contato com a vítima e até mesmo
perseguição real à vítima. O {\em stalking} pode ser extremamente
assustador e invasivo para as vítimas e pode afetar negativamente sua
saúde mental e física.

\starthead{\bf {\em flaming}} é uma forma de {\em cyberbullying} que envolve a
postagem de comentários agressivos e insultuosos em fóruns de discussão,
grupos de {\em chat} e outras plataformas {\em on-line}. O objetivo do
{\em flaming} é provocar uma resposta emocional da vítima e desencadear
uma discussão acalorada. O {\em flaming} pode ser especialmente
prejudicial quando as vítimas são alvo de ataques em massa, como quando
um grupo de pessoas se une para atacar uma pessoa ou uma comunidade.

\starthead{\bf {\em outcasting}} é uma forma de {\em bullying} social que
envolve a exclusão social deliberada de uma pessoa ou um grupo de
pessoas. Isso pode incluir a recusa em permitir que a vítima participe
de atividades em grupo, a exclusão social e a recusa em fornecer ajuda
ou apoio quando necessário. O {\em outcasting} pode ser extremamente
prejudicial para as vítimas, que se sentem progressivamente solitárias e
não aceitas.
\stopitemize

Como se pode notar, existem muitas formas diferentes de {\em bullying} e
{\em cyberbullying}, e cada uma delas pode ter um impacto significativo
na saúde física ou mental das vítimas. É importante que as pessoas
estejam cientes desses diferentes tipos de violência e que saibam como
identificá-los e abordá-los de forma eficaz. Isso pode incluir a criação
de políticas escolares e da iniciativa privada para lidar com o problema
e a conscientização pública sobre os efeitos prejudiciais do
{\em bullying} e do {\em cyberbullying}, além da promoção de uma cultura
de respeito e inclusão nas comunidades.

\startOFICINA
\color[darkgreen]{\bfb Oficina digital}

Junte-se a um grupo de mais dois ou três colegas e, juntos, criem um
roteiro, em tópicos, para um breve {\em podcast} em que se faça uma
conscientização sobre o uso responsável das redes sociais.

\MyIcon{keyboard}{darkgreen}

Na seção Oficina digital, o roteiro pode variar muito. O importante é
que tenha conteúdo coerente de conscientização sobre o uso das redes
sociais. A seguir, apresenta-se uma sugestão. Introdução: apresentação
do tema do {\em podcast} e do objetivo de conscientizar os ouvintes
sobre o uso consciente das redes sociais. Desenvolvimento: explorar os
impactos do uso excessivo das redes sociais na saúde mental e emocional
dos usuários, destacar a importância de não se comparar com as vidas
"perfeitas" que muitas vezes são exibidas nas redes sociais, discutir
como as {\em fake news} e a desinformação podem ser perigosas e
prejudiciais para a sociedade, abordar as consequências negativas do
{\em cyberbullying} e incentivar a promoção de uma cultura de respeito e
empatia na internet, oferecer dicas práticas para um uso saudável e
consciente das redes sociais, como limitar o tempo de uso e praticar a
autenticidade {\em on-line}. Conclusão: reiterar a importância do uso
consciente das redes sociais para uma vida mais equilibrada e saudável,
encorajar os ouvintes a refletirem sobre seu próprio uso das redes
sociais e a fazerem mudanças positivas em sua rotina {\em on-line}.
\stopOFICINA

\section[title={O que dizem as estatísticas},reference={o-que-dizem-as-estatísticas}]

O {\em bullying} e o {\em cyberbullying} são problemas sérios que afetam
crianças, adolescentes e até mesmo adultos em todo o mundo. De acordo
com estatísticas da Organização das Nações Unidas (ONU), cerca de 1 em
cada 3 estudantes em todo o mundo é vítima de {\em bullying} em algum
momento de sua vida escolar. No Brasil, esse número é ainda mais
alarmante: uma pesquisa realizada pelo Instituto Brasileiro de Geografia
e Estatística (IBGE) mostrou que mais de 70\letterpercent{} dos
estudantes do Ensino Médio já foram vítimas de algum tipo de
{\em bullying}.

\placefigure{O combate ao {\em bullying} é uma responsabilidade de
todos.}{\externalfigure[media/image4.jpeg][width=\textwidth]}

No que se refere ao {\em cyberbullying}, a situação não é diferente.
Segundo uma pesquisa realizada pelo Comitê Gestor da Internet no Brasil,
19\letterpercent{} das crianças e dos adolescentes brasileiros já foram
vítimas de algum tipo de violência {\em on-line}. Esse número é ainda
maior entre os jovens de 15 a 17 anos, chegando a 29\letterpercent{}.
Além disso, a pesquisa também mostrou que as meninas são mais
vulneráveis ao {\em cyberbullying} do que os meninos.

\startmarginblock
\startPalavra
{\bf Ideação suicida:} Presença perigosa de pensamentos recorrentes sobre o suicídio ou sobre a
própria morte.
\stopPalavra
\stopmarginblock

\startmarginblock
\startMarca
{\bf Marca-texto}

Em termos globais, estima-se que cerca de 1 em cada 4 crianças em todo o
mundo seja vítima de {\em cyberbullying}. Essa estatística alarmante
destaca a necessidade urgente de ações preventivas e de políticas
públicas que abordem esses problemas. É importante que governos,
escolas, empresas e comunidades em geral trabalhem juntos para combater
o {\em bullying} e o {\em cyberbullying}.
\stopMarca
\stopmarginblock

%<https://br.freepik.com/fotos-gratis/comportamento-agressivo-sem-icone-de-intimidacao_17056343.htm\#query=Fighting\letterpercent{}20against\letterpercent{}20bullying&position=32&from_view=search&track=robertav1_2_sidr>>

Além disso, é importante destacar que o {\em bullying} e o
{\em cyberbullying} podem ter efeitos negativos duradouros sobre a saúde
mental e física das vítimas. Estudos mostram que as vítimas de
{\em bullying} têm maior probabilidade de sofrer de depressão,
ansiedade, baixa autoestima e até mesmo ter {\bf ideação suicida}. O
{\em cyberbullying} pode ter efeitos ainda mais graves, incluindo o
isolamento social, a ansiedade crônica e o desenvolvimento de
transtornos mentais graves.

É essencial que a conscientização sobre esses problemas continue a
crescer e que sejam tomadas medidas para combater o {\em bullying} e o
{\em cyberbullying} em todo o mundo. É importante que as vítimas saibam
que elas não estão sozinhas e que existem recursos disponíveis para
ajudá-las a superar essas situações difíceis.

Leia o texto a seguir.

\startblockquote
{\bf Um em cada dez estudantes já foi ofendido nas redes sociais}

{\em Dado é da Pesquisa Nacional de Saúde do Escolar}

Aproximadamente um em cada dez adolescentes (13,2\letterpercent{}) já se
sentiu ameaçado, ofendido e humilhado em redes sociais ou aplicativos.
Consideradas apenas as meninas, esse percentual é ainda maior,
16,2\letterpercent{}. Entre os meninos é 10,2\letterpercent{}. Os dados
fazem parte da Pesquisa Nacional de Saúde do Escolar (PeNSE) 2019,
divulgada {[}em 2021, a mais recente{]} pelo Instituto Brasileiro de
Geografia e Estatística (IBGE).

Ao todo, foram entrevistados quase 188 mil estudantes, com idade entre
13 e 17 anos, em 4.361 escolas de 1.288 municípios de todo o país. O
grupo representa 11,8 milhões de estudantes brasileiros. A coleta dos
dados foi feita antes da pandemia, entre abril e setembro de 2019. A
partir de 2020, com a suspensão das aulas presenciais, o uso das redes
sociais, até mesmo como ferramenta de estudos, foi intensificado.

As agressões existem também fora da internet, nas escolas, onde
23\letterpercent{} dos estudantes afirmaram ter sido vítimas de
{\em bullying}, ou seja, sentiram-se humilhados por provocações feitas
por colegas nos 30 dias anteriores à pesquisa. Quando perguntados sobre
o motivo de sofrerem {\em bullying}, os três maiores percentuais foram
para aparência do corpo (16,5\letterpercent{}), aparência do rosto
(11,6\letterpercent{}) e cor ou raça (4,6\letterpercent{}).

%\inmargin{
\startplacefigure[location=leftmargin,title={Ao criar barreiras e afastamentos, o {\em bullying} é
prejudicial ao ambiente escolar como um
todo.}]
\externalfigure[../06/media/image5.jpeg][width=\marginwidth]
\stopplacefigure
%}

%https://br.freepik.com/fotos-gratis/garota-de-vista-traseira-sendo-intimidada-na-escola_32519225.htm\#query=bullying\letterpercent{}20at\letterpercent{}20school&position=30&from_view=search&track=robertav1_2_sidr

Em relação à saúde mental dos estudantes, metade (50,6\letterpercent{})
disse se sentir muito preocupado com as coisas comuns do dia a dia. Um
em cada cinco estudantes (21,4\letterpercent{}) afirmou que a vida não
valia a pena ser vivida. Entre as meninas, esse percentual é
29,6\letterpercent{} e, entre os meninos, 13\letterpercent{}.

Os resultados mostram ainda insatisfação com o próprio corpo. Menos da
metade (49,8\letterpercent{}) achava o corpo normal,
28,9\letterpercent{} se achavam magros ou muito magros e
20,6\letterpercent{}, gordos ou muito gordos.

{[}...{]}

Mariana Tokarnia. Agência Brasil. IBGE: um em
cada dez estudantes já foi ofendido nas redes sociais. Disponível em:
https://agenciabrasil.ebc.com.br/geral/noticia/2021-09/ibge-um-
em-cada-dez-estudantes-ja-foi-ofendido-nas-redes-sociais.
\MyMicroIcon{paperclip}
Acesso em: 29 abr. 2023.
\stopblockquote

A preocupação com os dados alarmantes sobre {\em bullying} e
{\em cyberbullying} é uma questão que afeta profundamente a sociedade
atual. Além disso, com o aumento do uso da tecnologia, o
{\em cyberbullying} se tornou uma forma cada vez mais comum de violência
contra os jovens.

Um dos problemas mais graves é que esses comportamentos também afetam
negativamente o ambiente escolar e social como um todo, promovendo uma
cultura de medo, intimidação e exclusão. Para combater essas práticas, é
necessário educar os jovens sobre a importância da empatia, do respeito
e da tolerância, bem como ensiná-los a reconhecer e reportar
comportamentos de {\em bullying}. Também é fundamental envolver os pais,
os educadores e a comunidade em geral para criar uma rede de apoio e
proteção para as vítimas.

Além disso, as empresas de tecnologia têm um papel importante a
desempenhar na prevenção do {\em cyberbullying}. Elas devem ser
incentivadas a desenvolver tecnologias e políticas que promovam a
segurança {\em on-line} e a proteção dos usuários. Isso inclui a
implementação de medidas de segurança robustas e ações claras e rápidas
em resposta a comportamentos de {\em cyberbullying}.

\placefigure{}{
\externalfigure[][width=1.4\textwidth]}

\subject[title={Atividades},reference={atividades}]

\startitemize[n][color=Red1,style=bf]
\item Qual das seguintes ações é um exemplo de {\em cyberbullying}?

\startitemize[a][color=Blue2,style=bf]
\item Compartilhar uma foto divertida com amigos.

\mar{\symbol[awesome5-solid][arrow-right]} Enviar uma mensagem ameaçadora para um colega na internet.

\item Fazer uma busca na rede para encontrar informações sobre um assunto
escolar.

\item Jogar um jogo on-line com amigos.
\stopitemize

\item Explique como o {\em bullying} pode afetar a saúde mental e emocional
das vítimas, incluindo consequências que têm longa duração.

\thinrules[n=8,color=Blue2]

%O {\em bullying} pode afetar a saúde mental e emocional das vítimas de
%diversas maneiras. Algumas consequências imediatas incluem ansiedade,
%depressão, baixa autoestima, isolamento social e desinteresse escolar. A
%longo prazo, o {\em bullying} pode levar a problemas de saúde mental,
%como transtornos de ansiedade e depressão crônicos, bem como
%dificuldades de relacionamento e de ajuste social.

\item Discuta as possíveis causas do {\em cyberbullying} e como a
tecnologia pode ser usada para prevenir e combater esse problema.

\thinrules[n=8,color=Blue2]

%O {\em cyberbullying} pode ser causado por diversos fatores, como o
%anonimato proporcionado pela internet, a falta de supervisão dos pais e
%educadores, a facilidade de disseminação de conteúdo na rede e a
%possibilidade de se esconder atrás de perfis falsos. Para prevenir e
%combater o {\em cyberbullying}, é importante educar os jovens sobre o
%uso seguro e responsável da tecnologia, monitorar seus comportamentos
%{\em on-line} e denunciar os casos de agressão virtual.

\page

\item Observe a imagem.

\placefigure{}
{\externalfigure[../06/media/image6.jpeg][width=\textwidth]}

%\https://br.freepik.com/fotos-gratis/menina-triste-em-tons-de-cinza-sendo-vitima-de-cyberbullying-nas-redes-sociais_13463115.htm\#query=cyberbullying&position=2&from_view=search&track=robertav1_2_sidr>

O {\em cyberbullying} é especialmente perigoso porque não depende de
espaço, localização, tempo disponível ou grandes recursos tecnológicos,
em um mundo no qual a internet está na palma da mão de quase todas as
pessoas, de todas as idades. Descreva a imagem, associando-a a esse
risco que o {\em cyberbullying} representa por sua simplicidade de ação.

\thinrules[n=14,color=Blue2]

%A facilidade de acesso à internet trouxe inúmeros benefícios para a
%sociedade moderna, como a democratização do conhecimento, a ampliação do
%acesso a serviços e produtos e a melhoria na comunicação entre pessoas e
%empresas. Essa mesma facilidade, no entanto, também criou um ambiente
%propício para a prática do {\em cyberbullying}. A facilidade de
%anonimato e a falta de barreiras para disseminar conteúdo agressivo na
%internet tornam o {\em cyberbullying} especialmente perigoso, pois, de
%certa forma, funciona como uma proteção para os agressores. É importante
%ressaltar que o fácil acesso à internet e às redes sociais não é o único
%fator responsável pelo aumento do {\em cyberbullying}. A falta de
%educação digital e a falta de supervisão por parte de pais e educadores
%também contribuem para o problema. Muitos jovens não entendem os limites
%do que é aceitável no mundo virtual e, por isso, acabam se envolvendo em
%situações de {\em cyberbullying} sem sequer perceber.

\page

\item Qual é o papel de pais e educadores na prevenção do {\em bullying} e
do {\em cyberbullying}?

\startitemize[a][color=Blue2,style=bf]
\item Fornecer tecnologia para as crianças e adolescentes usarem
livremente.

\item Subestimar comportamentos inadequados dos jovens.

\mar{\symbol[awesome5-solid][arrow-right]} Ensinar valores como empatia, respeito e tolerância.

\item Deixar a responsabilidade da prevenção do {\em bullying} e do
{\em cyberbullying} com educadores na escola.
\stopitemize

\item Explique a importância da educação sobre empatia, respeito e
tolerância na prevenção do {\em bullying} e do {\em cyberbullying}.

\thinrules[n=8,color=Blue2]

%A educação sobre empatia, respeito e tolerância é fundamental para
%prevenir o {\em bullying} e o {\em cyberbullying}, pois ajuda os jovens
%a desenvolver habilidades socioemocionais importantes, como a capacidade
%de se colocar no lugar do outro, de reconhecer a diversidade e de lidar
%com conflitos de forma saudável. Essas habilidades contribuem para a
%formação de uma cultura escolar mais positiva e inclusiva.

\item Discuta a responsabilidade das empresas de tecnologia na prevenção do
{\em cyberbullying} e quais medidas elas podem tomar para promover a
segurança {\em on-line}.

\thinrules[n=8,color=Blue2]

%As empresas de tecnologia têm uma grande responsabilidade na prevenção
%do {\em cyberbullying}, já que suas plataformas são frequentemente
%usadas para disseminar conteúdo agressivo e prejudicar a reputação de
%outras pessoas. Algumas medidas que essas empresas podem tomar incluem a
%criação de políticas claras de uso da plataforma, o monitoramento de
%comportamentos inadequados e a disponibilização de recursos para
%denunciar casos de {\em cyberbullying}.

\page

\item Analise a relação entre {\em bullying} e violência escolar e como
isso pode ser abordado pelas políticas públicas.

\thinrules[n=8,color=Blue2]

%A relação entre {\em bullying} e violência escolar é complexa e
%multifacetada. O {\em bullying} pode ser um precursor da violência, já
%que muitos casos de agressão física começam com comportamentos de
%intimidação. Por outro lado, a violência escolar pode ser uma forma de
%{\em bullying}, quando ocorre repetidamente e de forma intencional. Para
%abordar esses problemas, é necessário implementar políticas públicas que
%promovam a prevenção do {\em bullying} e da violência, bem como o
%diálogo aberto e respeitoso entre os jovens.
\stopitemize