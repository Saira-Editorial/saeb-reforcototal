\chapter{Respostas}
\pagestyle{plain}
\footnotesize

\pagecolor{gray!40}

\section*{Módulo 1 -- Treino}

\begin{enumerate}
\item
(A) Correta. A decomposição 700 + 30 + 4 equivale a 734.
(B) Incorreta. 70 + 3 + 4 = 707.
(C) Incorreta. 700 + 40 + 3 = 743.
(D) Incorreta. 70 + 300 + 4 = 374.
SAEB: Compor ou decompor números naturais de até 6 ordens na forma aditiva, ou em suas ordens, ou em adições e multiplicações.
BNCC: EF03MA04 -- Estabelecer a relação entre números naturais e pontos da reta numérica para
utilizá-la na ordenação dos números naturais e também na construção de fatos da adição e da
subtração, relacionando-os com deslocamentos para a direita ou para a esquerda.

\item
(A) Incorreta. Neste ponto estará o número 550.
(B) Incorreta. Neste ponto estará o número 560.
(C) Correta. O 570 estará no ponto S, pois, como no ponto P está o 540 e cada
repartição é de 10 unidades, ele deve estar no terceiro ponto após o P,
sem contar o ponto S.
(D) Incorreta. Neste ponto estará o número 580.
SAEB: Identificar a ordem ocupada por um algarismo ou seu valor posicional (ou valor relativo) em um número natural de até 6 ordens.
BNCC: EF03MA01 -- Ler, escrever e comparar números naturais de até a ordem de unidade de milhar, estabelecendo relações entre os registros numéricos e em língua materna.

\item
(A) Incorreta. Repetem-se os algarismos 5 e 9, mas o número formado está errado.
(B) Incorreta. Não se trata do número formado.
(C) Correta. 5 x 100 + 1 x 9 = 509.
(D) Incorreta. 1.509 tem 1000 unidades a mais do que 509.
SAEB: Compor ou decompor números naturais de até 6 ordens na forma aditiva, ou em suas ordens, ou em adições e multiplicações.
BNCC: EF03MA01 -- Ler, escrever e comparar números naturais de até a ordem de unidade de milhar, estabelecendo relações entre os registros numéricos e em língua materna.
\end{enumerate}

\section*{Módulo 2 -- Treino}

\begin{enumerate}
\item
(A) Incorreta. Trata-se somente do número de botões.
(B) Incorreta. Trata-se, simplesmente, do número de camisas.
(C) Incorreta. Trata-se da soma de 6 com 9.
(D) Correta. A conta é a seguinte: 6 x 9 = 54 botões.
SAEB: Calcular o resultado de multiplicações ou divisões envolvendo números naturais de até 6 ordens.
BNCC: EF03MA07 – Resolver e elaborar problemas de multiplicação (por 2, 3, 4, 5 e 10) com os
significados de adição de parcelas iguais e elementos apresentados em disposição retangular,
utilizando diferentes estratégias de cálculo e registros.

\item
(A) Incorreta. Trata-se do número de garrafas em um único fardo.
(B) Incorreta. Trata-se da quantidade de fardos.
(C) Correta. Os cálculos são os seguintes: 6 x 12 = 72 garrafas.
(D) Incorreta. Trata-se do dobro da quantidade real de garrafas de suco.
SAEB: Resolver problemas de multiplicação ou de divisão, envolvendo números naturais de até 6 ordens, com os significados de formação de grupos iguais (incluindo repartição equitativa e medida), proporcionalidade ou disposição retangular.
BNCC: EF03MA07 – Resolver e elaborar problemas de multiplicação (por 2, 3, 4, 5 e 10) com os
significados de adição de parcelas iguais e elementos apresentados em disposição retangular,
utilizando diferentes estratégias de cálculo e registros.

\item
(A) Incorreta. Trata-se do número de times multiplicado pelo número de reservas.
(B) Incorreta. Nesse caso, não se consideraram os reservas.
(C) Incorreta. Trata-se do número de times multiplicado por ele mesmo.
(D) Correta. O cálculo é o seguinte: (5 + 3) x 6 = 48 alunos.
SAEB: Resolver problemas de multiplicação ou de divisão, envolvendo números naturais de até 6 ordens, com os significados de formação de grupos iguais (incluindo repartição equitativa e medida), proporcionalidade ou disposição retangular.
BNCC: EF03MA07 – Resolver e elaborar problemas de multiplicação (por 2, 3, 4, 5 e 10) com os
significados de adição de parcelas iguais e elementos apresentados em disposição retangular,
utilizando diferentes estratégias de cálculo e registros.
\end{enumerate}

\section*{Módulo 3 -- Treino}

\begin{enumerate}
\item
(A) Incorreta. O número 25 será o primeiro.
(B) Incorreta. O número 48 ficará antes de 54.
(C) Correta. Colocando os números em ordem crescente, teremos: 25; 48; 54; 63; 68; 76; 95. Sendo assim, o número que aparece imediatamente antes do 63 é o 54.
(D) Incorreta. O número 68 aparecerá antes de 76.
SAEB: Inferir o padrão ou a regularidade de uma sequência de números naturais ordenados, objetos ou figuras.
BNCC: EF03MA10 -- Identificar regularidades em sequências ordenadas de números naturais,
resultantes da realização de adições ou subtrações sucessivas, por um mesmo número,
descrever uma regra de formação da sequência e determinar elementos faltantes ou seguintes.

\item
(A) Incorreta. O número não faz parte da sequência.
(B) Incorreta. Trata-se do número de bolinhas da figura 5.
(C) Incorreta. O número não faz parte da sequência.
(D) Correta. (4; 7; 10; 13; 16; 19; ou (6 x 3) + 1 = 19.). Em cada imagem, 3 é multiplicado pelo número da posição e, ao valor multiplicado, soma-se um.
SAEB: Inferir o padrão ou a regularidade de uma sequência de números naturais ordenados, objetos ou figuras.
BNCC: EF03MA10 -- Identificar regularidades em sequências ordenadas de números naturais,
resultantes da realização de adições ou subtrações sucessivas, por um mesmo número,
descrever uma regra de formação da sequência e determinar elementos faltantes ou seguintes.

\item
(A) Incorreta. Nesse caso, a sequência aumenta de 5 em 5. Para haver antecessor e sucessor, esse aumento precisa ser de 1 em 1.
(B) Incorreta. Nesse caso, a sequência aumenta de 100 em 100. Para haver antecessor e sucessor, esse aumento precisa ser de 1 em 1.
(C) Incorreta. Nesse caso, a sequência aumenta de 10 em 10. Para haver antecessor e sucessor, esse aumento precisa ser de 1 em 1.
(D) Correta. Antecessor de um número: Número imediatamente anterior a um número na sequência dos números naturais. Sucessor de um número: Número imediatamente a frente, ou após, de um número na sequência dos números naturais.
SAEB: Inferir ou descrever atributos ou propriedades comuns que os elementos que constituem uma sequência recursiva de números naturais apresentam.
BNCC: EF03MA10 -- Identificar regularidades em sequências ordenadas de números naturais,
resultantes da realização de adições ou subtrações sucessivas, por um mesmo número,
descrever uma regra de formação da sequência e determinar elementos faltantes ou seguintes.
\end{enumerate}


\section*{Módulo 4 -- Treino}

\begin{enumerate}
\item
(A) Incorreta. Nesse caso, ele cumpriria apenas 3h.
(B) Incorreta. Nesse caso, ele cumpriria apenas 3h30min.
(C) Incorreta. Nesse caso, ele cumpriria apenas 4h.
(D) Correta. Como pela manhã ele entra às 8:00 e deve cumprir nesse período 4 horas e
meia de trabalho antes de sair para o almoço, conclui-se que ele sairá para o almoço às 12:30.
SAEB: - Determinar o horário de início, o horário de término ou a duração de um acontecimento. 
BNCC: EF03MA22 -- Ler e registrar medidas e intervalos de tempo, utilizando relógios (analógico e
digital) para informar os horários de início e término de realização de uma atividade e sua
duração.

\item
(A) Incorreta. 1 garrafinha daria para 3 dias apenas.
(B) Incorreta. 2 garrafinhas dariam para 6 dias apenas.
(C) Incorreta. 3 garrafinhas dariam para 9 dias apenas.
(D) Correta. 4 x 40 x 10- = 1600 mL. Como cada garrafinha contém 500 mL, ela terá que
preparar 4 garrafinhas e haverá uma sobra de chá.
SAEB: Resolver problemas que envolvam medidas de grandezas (comprimento, massa, tempo e capacidade) em que haja conversões entre as unidades mais usuais. 
BNCC: EF03MA22 -- Ler e registrar medidas e intervalos de tempo, utilizando relógios (analógico e
digital) para informar os horários de início e término de realização de uma atividade e sua
duração.

\item
(A) Correta. Saída: 8 horas e 15 minutos. Chegada: 11 horas e 30 minutos. Tempo de voo: 3 horas e 15 minutos.
(B) Incorreta. Se o voo tivesse tido 2h de duração, ela chegaria às 10h15min.
(C) Incorreta. Se o voo tivesse tido 2h45min de duração, ela chegaria às 11h.
(D) Incorreta. Se o voo tivesse tido 3h50min de duração, ela chegaria às 12h05min.
SAEB: Resolver problemas que envolvam medidas de grandezas (comprimento, massa, tempo e capacidade) em que haja conversões entre as unidades mais usuais. 
BNCC: EF03MA22 -- Ler e registrar medidas e intervalos de tempo, utilizando relógios (analógico e
digital) para informar os horários de início e término de realização de uma atividade e sua
duração.
\end{enumerate}

\section*{Módulo 5 -- Treino}

\begin{enumerate}
\item
(A) Incorreta. Trata-se de apenas um terço do caminho.
(B) Incorreta. Faltariam, ainda, 5 m.
(C) Incorreta. Trata-se de quase a totalidade, mas ainda faltariam 3m.
(D) Correta. Ele deverá andar 5 lados de quadrado. Como cada lado de quadrado possui
medida igual a 3 m, ele deverá andar 15 metros.
SAEB: Medir ou comparar perímetro ou área de figuras planas desenhadas em malha quadriculada. 
BNCC: EF03MA19 -- Estimar, medir e comparar comprimentos, utilizando unidades de medida
não padronizadas e padronizadas mais usuais (metro, centímetro e milímetro) e diversos
instrumentos de medida.

\item
(A) Incorreta. O L visivelmente ocupa mais do que 3 quadradinhos.
(B) Incorreta. 14 quadradinhos representakm a área aproximada de pouco mais da metade do L.
(C) Correta. O L não ocupa espaços exatos, mas, de forma aproximada, ocupa 26 quadradinhos. É importante imaginá-lo mais ajustado à malha.
(D) Incorreta. O L visivelmente ocupa muito menos do que 120 quadradinhos.
SAEB: Resolver problemas que envolvam área de figuras planas. 
BNCC: EF03MA19 -- Estimar, medir e comparar comprimentos, utilizando unidades de medida
não padronizadas e padronizadas mais usuais (metro, centímetro e milímetro) e diversos
instrumentos de medida.

\item
Resposta: D
A nova pista de caminhada tem o quíntuplo da extensão da anterior.
SAEB: Resolver problemas que envolvam medidas de grandezas (comprimento, massa, tempo e capacidade) em que haja conversões entre as unidades mais usuais. 
BNCC: EF03MA19 -- Estimar, medir e comparar comprimentos, utilizando unidades de medida
não padronizadas e padronizadas mais usuais (metro, centímetro e milímetro) e diversos
instrumentos de medida.
\end{enumerate}

\section*{Módulo 6 -- Treino}

\begin{enumerate}
\item
(A) Incorreta. Nesse caso, não se considerou o desconto.
(B) Incorreta. Nesse caso, simplesmente se repetiu o valor do desconto.
(C) Correta.Resposta: R\$ 985,00 -- R\$ 75,00 = R\$ 910,00.
(D) Incorreta. Nesse caso, somou-se o valor do desconto, em vez de se descontar.
SAEB: Resolver problemas que envolvam moedas e/ou cédulas do sistema monetário brasileiro. 
BNCC: EF03MA24 -- Resolver e elaborar problemas que envolvam a comparação e a equivalência de
valores monetários do sistema brasileiro em situações de compra, venda e troca.

\item
(A) Incorreta. Há apenas 6 moedas de 1 real representadas.
(B) Incorreta. Há apenas 8 moedas de 1 real representadas.
(C) Correta. Ela tem R\$ 12,00; então, receberá 12 moedas de 1 real.
(D) Incorreta. Há 20 moedas de 1 real representadas.
SAEB: Resolver problemas que envolvam moedas e/ou cédulas do sistema monetário brasileiro. 
BNCC: EF03MA24 -- Resolver e elaborar problemas que envolvam a comparação e a equivalência de
valores monetários do sistema brasileiro em situações de compra, venda e troca.

\item
(A) Incorreta. Trata-se de valor inferior ao real.
(B) Incorreta. Trata-se de 13 reais a menos do que o esperado.
(C) Incorreta. Trata-se de 11 reais a menos do que o esperado.
(D) Correta.
1 x 5,00 + 2 x 6,00 + 2 x 12,00 = 5,00 + 12,00 + 24,00 =R\$ 41,00.
SAEB: Resolver problemas que envolvam moedas e/ou cédulas do sistema monetário brasileiro. 
BNCC: EF03MA24 -- Resolver e elaborar problemas que envolvam a comparação e a equivalência de
valores monetários do sistema brasileiro em situações de compra, venda e troca.
\end{enumerate}

\section*{Módulo 7 -- Treino}

\begin{enumerate}
\item
(A) Incorreta. Essa seria a probabilidade caso o alvo fosse qualquer caminhão.
(B) Correta. Se metade dos 10 caminhões da coleção tem 6 rodas, então eles são 5 no total.
(C) Incorreta. Há mais do que um caminhão de 6 rodas no total.
(D) Incorreta. Nesse caso, estariam sendo considerados todos os carrinhos da coleção como alvo.
SAEB: Identificar, entre eventos aleatórios, aqueles que têm menos, maiores ou iguais chances de ocorrência, sem utilizar frações. 
BNCC: EF03MA25 -- Identificar, em eventos familiares aleatórios, todos os resultados possíveis,
estimando os que têm maiores ou menores chances de ocorrência.

\item
(A) Correta. Somando-se as fantasias de maçã, uva e kiwi, são 30 no total (e o total são 100).
(B) Incorreta. Há menos do que 50 pessoas fantasiadas de frutas.
(C) Incorreta. Se há pessoas fantasiadas de frutas, então essa chance existe.
(D) Incorreta. Como há outras fantasias, que não são de frutas, existe a chance de a pessoa escolhida não usar uma fantasia de fruta.
SAEB: Determinar a probabilidade de ocorrência de um resultado em eventos aleatórios, quando todos os resultados possíveis têm a mesma chance de ocorrer (equiprováveis).  
BNCC: EF03MA25 -- Identificar, em eventos familiares aleatórios, todos os resultados possíveis,
estimando os que têm maiores ou menores chances de ocorrência.

\item
(A) Incorreta. Nesse caso, consideraram-se os três candidatos ou as três iniciais.
(B) Correta. São 7 alunos no total com iniciais A, C e T. Porém, os candidatos não votam. Restam, então, 4 alunos possíveis (Abigail, Aparecida, Carla e Coralina.)
(C) Incorreta. Nesse caso, consideraram-se os candidatos como alunos votantes.
(D) Incorreta. Se há alunos com as iniciais indicadas como alvos, então existe chance.
SAEB: Determinar a probabilidade de ocorrência de um resultado em eventos aleatórios, quando todos os resultados possíveis têm a mesma chance de ocorrer (equiprováveis). 
BNCC: EF03MA25 -- Identificar, em eventos familiares aleatórios, todos os resultados possíveis,
estimando os que têm maiores ou menores chances de ocorrência.
\end{enumerate}

\section*{Módulo 8 -- Treino}

\begin{enumerate}
\item
(A) Correta. Pela análise da tabela, percebe-se que o candidato A teve todas as notas
acima de 30 e é o que teve mais notas iguais. Portanto, o candidato A
deverá ser aprovado.
(B) Incorreta. O candidato B teve apenas duas notas 32 repetidas.
(C) Incorreta. O candidato C, além de ter uma nota abaixo de 30, não tem notas repetidas.
(D) Incorreta. O candidato D, além de ter uma nota abaixo de 30, não tem notas repetidas.
SAEB: Ler/identificar ou comparar dados estatísticos expressos em tabelas (simples ou de dupla entrada). 
BNCC: EF03MA27 -- Ler, interpretar e comparar dados apresentados em tabelas de dupla entrada,
gráficos de barras ou de colunas, envolvendo resultados de pesquisas significativas, utilizando
termos como maior e menor frequência, apropriando-se desse tipo de linguagem para
compreender aspectos da realidade sociocultural significativos.

\item
(A) Correta.  A soma 440 + 320 + 270 é igual a 1.030 alunos.
(B) Incorreta. Trata-se da soma 440 + 320.
(C) Incorreta. Trata-se da soma 440 + 270.
(D) Incorreta. Trata-se da soma 320 + 270.
SAEB: Ler/identificar ou comparar dados estatísticos expressos em gráficos (barras simples ou agrupadas, colunas simples ou agrupadas, pictóricos ou de linhas). 
BNCC: EF03MA27 -- Ler, interpretar e comparar dados apresentados em tabelas de dupla entrada,
gráficos de barras ou de colunas, envolvendo resultados de pesquisas significativas, utilizando
termos como maior e menor frequência, apropriando-se desse tipo de linguagem para
compreender aspectos da realidade sociocultural significativos.

\item
(A) Incorreta. Há mais do que 4 alunos com notas maiores ou iguais a 7,0.
(B) Correta. Entre as notas fornecidas temos 10 notas maiores ou iguais a 7,0.
(C) Incorreta. Há mais do que 6 alunos com notas maiores ou iguais a 7,0.
(D) Incorreta. Há menos do que 16 alunos com notas maiores ou iguais a 7,0.
SAEB: Resolver problemas que envolvam dados apresentados tabelas (simples ou de dupla entrada) ou gráficos estatísticos (barras simples ou agrupadas, colunas simples ou agrupadas, pictóricos ou de linhas). 
BNCC: EF03MA27 -- Ler, interpretar e comparar dados apresentados em tabelas de dupla entrada,
gráficos de barras ou de colunas, envolvendo resultados de pesquisas significativas, utilizando
termos como maior e menor frequência, apropriando-se desse tipo de linguagem para
compreender aspectos da realidade sociocultural significativos.
\end{enumerate}

\section*{Simulado 1}

\begin{enumerate}
\item
(A) Incorreta. 623 equivaleria a 6 x 100 + 2 x 10 + 3 x 1 = 623.
(B) Incorreta. 423 equivaleria a 4 x 100 + 2 x 10 + 3 x 1 = 423.
(C) Incorreta. 503 equivaleria a 5 x 100 + 3 x 1 = 503.
(D) Correta. 5 x 100 + 2 x 10 + 3 x 1 = 523.
SAEB: Compor ou decompor números naturais de até 6 ordens na forma aditiva, ou em suas ordens, ou em adições e multiplicações.
BNCC: EF03MA04 -- Estabelecer a relação entre números naturais e pontos da reta numérica para
utilizá-la na ordenação dos números naturais e também na construção de fatos da adição e da
subtração, relacionando-os com deslocamentos para a direita ou para a esquerda.

\item
(A) Incorreta. Trata-se de um número sem o último zero à direita; logo, dez vezes menor.
(B) Incorreta. A posição dos algarismos está invertida.
(C) Incorreta. A posição dos algarismos está invertida.
(D) Correta. 2 x 1.000 + 3 x 100 + 1 x 10 = 2.310.
SAEB: Compor ou decompor números naturais de até 6 ordens na
forma aditiva, ou em suas ordens, ou em adições e multiplicações.
BNCC: EF03MA06 – Resolver e elaborar problemas de adição e subtração com os significados de
juntar, acrescentar, separar, retirar, comparar e completar quantidades, utilizando diferentes
estratégias de cálculo exato ou aproximado, incluindo cálculo mental.

\item
(A) Incorreta. De fato está como 4, mas deveria ser 400, não 40.
(B) Incorreta. De fato deveria ser 400, mas está como 4, não como 40.
(C) Correta. Como o número apresentado no enunciado está com o primeiro e o último
algarismos trocados, conclui-se que o número correto seria 483. Na placa,
o último algarismo é o 4, que tem valor relativo de 4 unidades, mas no
número correto ele estaria na centena comum, apresentando, então, um valor
relativo de 400 (quatrocentos).
(D) Incorreta. Ambas as análises estão erradas.
SAEB: Identificar a ordem ocupada por um algarismo ou seu
valor posicional (ou valor relativo) em um número natural de até 6
ordens.
BNCC: EF03MA04 -- Estabelecer a relação entre números naturais e pontos da reta numérica para
utilizá-la na ordenação dos números naturais e também na construção de fatos da adição e da
subtração, relacionando-os com deslocamentos para a direita ou para a esquerda.

\item
(A) Incorreta. A soma não totaliza 513.
(B) Incorreta. Não se trata do total de árvores.
(C) Correta. 359 + 246 = 605.
(D) Incorreta. Pode ter havido um erro na soma, para 100 unidades a mais.
SAEB: Resolver problemas de adição ou de subtração,
envolvendo números naturais de até 6 ordens, com os significados de
juntar, acrescentar, separar, retirar, comparar ou completar.
BNCC: EF03MA06 – Resolver e elaborar problemas de adição e subtração com os significados de
juntar, acrescentar, separar, retirar, comparar e completar quantidades, utilizando diferentes
estratégias de cálculo exato ou aproximado, incluindo cálculo mental.

\item
(A) Incorreta. Trata-se do dobro da resposta correta.
(B) Correta. 4 054 -- 2 843 = 1 211. Como o aumento foi o mesmo nnas duas escolas, não
precisamos fazer a soma do aumento aos números antigos, já que a diferença entre eles irá se manter.
(C) Incorreta. Trata-se de um número menor que a diferença real.
(D) Incorreta. Não é esse o número que quantifica a diferença.
SAEB: Resolver problemas de adição ou de subtração,
envolvendo números naturais de até 6 ordens, com os significados de
juntar, acrescentar, separar, retirar, comparar ou completar.
BNCC: EF03MA06 – Resolver e elaborar problemas de adição e subtração com os significados de
juntar, acrescentar, separar, retirar, comparar e completar quantidades, utilizando diferentes
estratégias de cálculo exato ou aproximado, incluindo cálculo mental.

\item
(A) Incorreta. A sala 301 é a que fica na lateral mais perto da árvore.
(B) Incorreta. A sala 302 está com a janela entreaberta.
(C) Correta. A sala será a 303 seguindo as instruções do enunciado.
(D) Incorreta. Seguindo a lógica, nota-se que não há sala 304.
SAEB: Inferir o padrão ou a regularidade de uma sequência de
números naturais ordenados, objetos ou figuras.
BNCC: EF03MA10 -- Identificar regularidades em sequências ordenadas de números naturais,
resultantes da realização de adições ou subtrações sucessivas, por um mesmo número,
descrever uma regra de formação da sequência e determinar elementos faltantes ou seguintes.

\item
(A) Correta. 42 -- 14 + 5 = 33 pontos.
(B) Incorreta. Nesse caso, somaram-se 4 pontos a mais.
(C) Incorreta. Nesse caso, foi considerada uma pontuação quase dobrada.
(D) Incorreta. Nesse caso, foram considerados apenas os pontos da terceira rodada.
Habilidade Saeb: Resolver problemas de adição ou de subtração,
envolvendo números naturais de até 6 ordens, com os significados de
juntar, acrescentar, separar, retirar, comparar ou completar.
BNCC: EF03MA06 – Resolver e elaborar problemas de adição e subtração com os significados de
juntar, acrescentar, separar, retirar, comparar e completar quantidades, utilizando diferentes
estratégias de cálculo exato ou aproximado, incluindo cálculo mental.

\item
(A) Incorreta. Consideraram-se, nesse caso, 4 quadradinhos a menos.
(B) Incorreta. Consideraram-se, nesse caso, 3 quadradinhos a menos.
(C) Correta. Observando a figura e realizando a contagem do número de quadradinhos
pintados, temos que esse número é igual a 29.
Incorreta. Consideraram-se, nesse caso, 5 quadradinhos a mais.
SAEB: Medir ou comparar perímetro ou área de figuras planas
desenhadas em malha quadriculada.
BNCC: EF03MA19 -- Estimar, medir e comparar comprimentos, utilizando unidades de medida
não padronizadas e padronizadas mais usuais (metro, centímetro e milímetro) e diversos
instrumentos de medida.

\item
(A) Incorreta. Trata-se do valor apenas de Camila.
(B) Incorreta. Trata-se do valor apenas de Ana Beatriz.
(C) Correta. Ana Beatriz possui R\$ 37,70; Camila possui R\$ 36,40. Então, elas possuem juntas: 37,70 + 36,40 = R\$ 74,10.
(D) Incorreta. Nesse caso, pode ter havido um erro de soma.
SAEB: Resolver problemas que envolvam moedas e/ou cédulas do
sistema monetário brasileiro.
BNCC: EF03MA24 -- Resolver e elaborar problemas que envolvam a comparação e a equivalência de
valores monetários do sistema brasileiro em situações de compra, venda e troca.

\item
(A) Incorreta. A ocupa 3 quadradinhos a mais que C.
(B) Incorreta. D ocupa 1 quadradinho a menos que E.
(C) Incorreta. D ocupa 2 quadradinhos a mais que C.
(D) Correta. Letra A: 14 quadradinhos; letra C: 11 quadradinhos; letra D: 13 quadradinhos; letra E: 14 quadradinhos. Portanto, as duas letras que ocupam as superfícies de mesmo tamanho são A e E.
SAEB: Medir ou comparar perímetro ou área de figuras planas
desenhadas em malha quadriculada.
BNCC: EF03MA20 -- Estimar e medir capacidade e massa, utilizando unidades de medida não
padronizadas e padronizadas mais usuais (litro, mililitro, quilograma, grama e miligrama),
reconhecendo-as em leitura de rótulos e embalagens, entre outros.

\item
(B) Incorreta. Não se trata da quantidade correta.
(B) Correta. Número de alunos do 4º ano: 32 + 29 + 25 = 86 
(C) Incorreta. Há 5 alunos a mais nessa soma.
(D) Incorreta. Não se trata da quantidade correta.
SAEB: Ler/identificar ou comparar dados estatísticos
expressos em gráficos (barras simples ou agrupadas, colunas simples ou
agrupadas, pictóricos ou de linhas).
BNCC: EF03MA27 -- Ler, interpretar e comparar dados apresentados em tabelas de dupla entrada,
gráficos de barras ou de colunas, envolvendo resultados de pesquisas significativas, utilizando
termos como maior e menor frequência, apropriando-se desse tipo de linguagem para
compreender aspectos da realidade sociocultural significativos.

\item
(A) Incorreta. A sequência é crescente.
(B) Incorreta. Não se trata de sequência ordenada.
(C) Incorreta. Não se trata de sequência ordenada.
(D) Correta. (400; 326; 280; 153; 120; 88; 25).
SAEB: Inferir ou descrever atributos ou propriedades comuns
que os elementos que constituem uma sequência recursiva de números
naturais apresentam.
BNCC: EF03MA10.

\item
(A) Correta. 273 x 3 = 819
(B) Incorreta. Não seria o cálculo de multiplicação correto.
(C) Incorreta. Nesse caso, a multiplicação seria por 2.
(D) Incorreta. Trata-se, simplesmente, da repetição de um dos números que aparece na lousa.
Habilidade Saeb: Calcular o resultado de multiplicações ou divisões
envolvendo números naturais de até 6 ordens.
BNCC: EF03MA07 – Resolver e elaborar problemas de multiplicação (por 2, 3, 4, 5 e 10) com os
significados de adição de parcelas iguais e elementos apresentados em disposição retangular,
utilizando diferentes estratégias de cálculo e registros.

\item
(A) Correta. Observando a posição dos ponteiros, conclui-se que a hora marcada pelo
relógio é 8 horas e 30 minutos.
(B) Incorreta. Nesse caso, o ponteiro menos deveria apontar para perto do número 7 e o maior deveria apontar para o número 3.
(C) Incorreta. Nesse caso, o ponteiro menos deveria apontar para perto do número 9 e o maior deveria apontar para o número 10.
(D) Incorreta. Nesse caso, o ponteiro menos deveria apontar para perto do número 8 e o maior deveria apontar para o número 10.
SAEB: Identificar horas em relógios analógicos ou associar
horas em relógios analógicos e digitais.
BNCC: EF03MA23 – Ler horas em relógios digitais e em relógios analógicos e reconhecer a relação
entre hora e minutos e entre minuto e segundos.

\item
(A) Incorreta. O lápis menor mede menos da metade do maior.
(B) Correta. Pela análise da figura, percebe-se que o lápis maior mede quatro vezes a medida do lápis menor.
(C) Incorreta. Isso significaria uma diferença de comprimeiro bem maior entre os lápis.
(D) Incorreta. Isso significaria que os dois lápis seriam do mesmo tamanho.
SAEB: Estimar/inferir medida de comprimento, capacidade ou
massa de objetos, utilizando unidades de medida convencionais ou não ou
medir comprimento, capacidade ou massa de objetos.
BNCC: EF03MA19 -- Estimar, medir e comparar comprimentos, utilizando unidades de medida
não padronizadas e padronizadas mais usuais (metro, centímetro e milímetro) e diversos
instrumentos de medida.
\end{enumerate}

\section*{Simulado 2}

\begin{enumerate}
\item
(A) Incorreta. O número formado seria 646.
(B) Incorreta. O número formado seria 634.
(C) Incorreta. O número formado seria 743.
(D) Correta. 724 = 7 x 100 + 2 x 10 + 4 (7 placas, 2 barras e 4 cubinhos).
SAEB: Compor ou decompor números naturais de até 6 ordens na forma aditiva, ou em suas ordens, ou em adições e multiplicações.
BNCC: EF03MA01 -- Ler, escrever e comparar números naturais de até a ordem de unidade de milhar, estabelecendo relações entre os registros numéricos e em língua materna.

\item
(A) Incorreta. O número 380 aparece depois de 200, não entre 150 e 200.
(B) Incorreta. O número 380 aparece depois de 300, não entre 250 e 300.
(C) Correta. Seguindo a sequência da reta numérica, conclui-se que o número 380 deverá ser colocado entre o 350 e o 400.
(D) Incorreta. O número 380 aparece antes de 450, não entre 450 e 500.
SAEB: Comparar ou ordenar números racionais (naturais de até 6 ordens, representação fracionária ou decimal finita até a ordem dos milésimos), com ou sem suporte da reta numérica.
BNCC: EF03MA04 -- Estabelecer a relação entre números naturais e pontos da reta numérica para
utilizá-la na ordenação dos números naturais e também na construção de fatos da adição e da
subtração, relacionando-os com deslocamentos para a direita ou para a esquerda.

\item
(A) Incorreta. Com mais 3 placas, ele teria 10 placas; ainda faltariam 10.
(B) Incorreta. Com mais 10 placas, ele teria 17 placas; ainda faltariam 3.
(C) Correta. Pela análise da figura dada, nota-se que ela tem 7 barras e, com isso,
precisa de mais 13 barras para trocar por 2 placas, uma vez que cada placa é equivalente a 10 barras.
(D) Incorreta. Com mais 20 placas, ele teria 27 placas; sobrariam 7.
SAEB: Compor ou decompor números naturais de até 6 ordens na forma aditiva, ou em suas ordens, ou em adições e multiplicações.
BNCC: EF03MA01 -- Ler, escrever e comparar números naturais de até a ordem de unidade de milhar, estabelecendo relações entre os registros numéricos e em língua materna.

\item
(A) Incorreta. 26 é o início do número 26.104.
(B) Incorreta. 28 contém os algarismos 2 e 8, mas pode ser resultado de uma operação realizada de forma errada.
(C) Incorreta. 208 contém os algarismos 2, 0 e 8, mas pode ser resultado de uma operação realizada de forma errada
(D) Correta. A quantidade de caixas que serão produzidas é numericamente igual ao número de
escolas que receberão as caixas. Sendo assim, 26.104/13 = 2.008 escolas.
SAEB: Resolver problemas de multiplicação ou de divisão, envolvendo números naturais de até 6 ordens, com os significados de formação de grupos iguais (incluindo repartição equitativa e medida), proporcionalidade ou disposição retangular.
BNCC: EF03MA08 -- Resolver e elaborar problemas de divisão de um número natural por outro (até
10), com resto zero e com resto diferente de zero, com os significados de repartição equitativa
e de medida, por meio de estratégias e registros pessoais.

\item
(A) Incorreta. 20 não é metade de 30.
(B) Correta. O próximo número da sequência será o 15, pois ela tem a lógica de o próximo
elemento ser metade do antecessor.
(C) Incorreta. 10 é um terço de 30, não metade.
(D) Incorreta. 5 é um sexto de 30, não metade.
SAEB: Inferir o padrão ou a regularidade de uma sequência de números naturais ordenados, objetos ou figuras.
BNCC: EF03MA10 -- Identificar regularidades em sequências ordenadas de números naturais,
resultantes da realização de adições ou subtrações sucessivas, por um mesmo número,
descrever uma regra de formação da sequência e determinar elementos faltantes ou seguintes.

\item
(A) Correta. 5 x 7 + 2 = 37 dias.
(B) Incorreta. 27 dias são 3 semanas e 6 dias.
(C) Incorreta. 17 dias são 2 semanas e 3 dias.
(D) Incorreta. 7 dias é o tempo de duração de uma semana.
SAEB: Reconhecer a unidade de medida ou o instrumento mais apropriado para medições de comprimento, área, massa, tempo, capacidade
ou temperatura.
BNCC: EF03MA22 -- Ler e registrar medidas e intervalos de tempo, utilizando relógios (analógico e
digital) para informar os horários de início e término de realização de uma atividade e sua
duração.

\item
(A) Incorreta. 1,10 m são 5 palmos e mais 5 cm.
(B) Incorreta. 1,30 m são 6 palmos e mais 4 cm. 
(C) Correta. 7 x 21 = 147 cm. Portanto, aproximadamente 1,5 m.
(D) Incorreta. 1,60 m são quase 8 palmos.
SAEB: Estimar/inferir medida de comprimento, capacidade ou massa de objetos, utilizando unidades de medida convencionais ou não ou medir comprimento, capacidade ou massa de objetos.
BNCC: EF03MA19 -- Estimar, medir e comparar comprimentos, utilizando unidades de medida
não padronizadas e padronizadas mais usuais (metro, centímetro e milímetro) e diversos
instrumentos de medida.

\item
(A) Incorreta. O mês de janeiro tem 31 dias.
(B) Correta. O mês de fevereiro é o mês que pode ter 28 ou 29 dias.
(C) Incorreta.O mês de junho tem 30 dias.
(D) Incorreta. O mês de agosto tem 31 dias.
SAEB: Reconhecer a unidade de medida ou o instrumento mais apropriado para medições de comprimento, área, massa, tempo, capacidade ou temperatura.
BNCC: EF03MA22 -- Ler e registrar medidas e intervalos de tempo, utilizando relógios (analógico e
digital) para informar os horários de início e término de realização de uma atividade e sua
duração.

\item
(A) Correta. 12 x 0,50 + 8 x 0,25 = 6 + 2 = R\$ 8,00. Portanto, 4 notas de 2 reais.
(B) Incorreta. 6 notas de 2 reais totalizariam 12 reais.
(C) Incorreta. 8 notas de 2 reais totalizariam 16 reais.
(D) Incorreta. 20 notas de 2 reais totalizariam 40 reais.
SAEB: Resolver problemas que envolvam moedas e/ou cédulas do sistema monetário brasileiro.
BNCC: EF03MA24 -- Resolver e elaborar problemas que envolvam a comparação e a equivalência de
valores monetários do sistema brasileiro em situações de compra, venda e troca.

\item
(A) Incorreta. Esse seria o resultado de 200 -- 2 x (64).
(B) Incorreta. Esse seria o resultado de 200 -- 2 x (53).
(C) Incorreta. Esse seria o resultado de 200 -- 2 x (37).
(D) Correta. 200 -- 2 x (1 x 5 + 2 x 7) = 200 -- 38 = 162 peças.
SAEB: Resolver problemas de multiplicação ou de divisão, envolvendo números naturais de até 6 ordens, com os significados de formação de grupos iguais (incluindo repartição equitativa e medida), proporcionalidade ou disposição retangular.
BNCC: EF03MA07 – Resolver e elaborar problemas de multiplicação (por 2, 3, 4, 5 e 10) com os
significados de adição de parcelas iguais e elementos apresentados em disposição retangular, utilizando diferentes estratégias de cálculo e registros.

\item
(A) Incorreta. Em abril foram vendidas 24 bolas, o menor dos índices.
(B) Incorreta. Em maio foram vendidas 45 bolas, que não é o dobro de nenhum outro valor.
(C) Incorreta. Em junho foram vendidas 62 bolas, número que não representa o dobro de outro mês.
(D) Correta. Em julho, pois tivemos uma venda de 72 bolas, que é o triplo das 24 unidades vendidas em abril.
SAEB: Ler/identificar ou comparar dados estatísticos
expressos em gráficos (barras simples ou agrupadas, colunas simples ou agrupadas, pictóricos ou de linhas).
BNCC: EF03MA27 -- Ler, interpretar e comparar dados apresentados em tabelas de dupla entrada,
gráficos de barras ou de colunas, envolvendo resultados de pesquisas significativas, utilizando
termos como maior e menor frequência, apropriando-se desse tipo de linguagem para compreender aspectos da realidade sociocultural significativos.

\item
(A) Incorreta. No mês de maio o número foi menor do que no mês anterior.
(B) Incorreta. Na maioria dos meses, os números de fato só cresceram.
(C) Incorreta. Houve bastante variação nos números.
(D) Correta. Se no mês de maio o número de livros retirados fosse entre 206 e 209, de fato haveria crescimento mês a mês.
SAEB: Ler/identificar ou comparar dados estatísticos
expressos em gráficos (barras simples ou agrupadas, colunas simples ou
agrupadas, pictóricos ou de linhas).
BNCC: EF03MA27 -- Ler, interpretar e comparar dados apresentados em tabelas de dupla entrada,
gráficos de barras ou de colunas, envolvendo resultados de pesquisas significativas, utilizando
termos como maior e menor frequência, apropriando-se desse tipo de linguagem para compreender aspectos da realidade sociocultural significativos.

\item
(A) Incorreta. 2 colheres seriam o bastante para fazer 500 mL de café.
(B) Correta. Como 750 mL é igual a 3 x 250, conclui-se que, pela proporção, ela
precisará do triplo de pó de café. Portanto, 3 x 1 = 3; três colheres de sopa de pó de café serão necessárias.
(C) Incorreta. 4 colheres seriam o bastante para fazer 1.000 mL de café.
(D) Incorreta. 5 colheres seriam o bastante para fazer 1.250 mL de café.
SAEB: Resolver problemas de multiplicação ou de divisão, envolvendo números naturais de até 6 ordens, com os significados de formação de grupos iguais (incluindo repartição equitativa e medida), proporcionalidade ou disposição retangular.
BNCC: EF03MA07 – Resolver e elaborar problemas de multiplicação (por 2, 3, 4, 5 e 10) com os
significados de adição de parcelas iguais e elementos apresentados em disposição retangular,
utilizando diferentes estratégias de cálculo e registros.

\item
(A) Incorreta. A poltrona de número 30 apareceria à direita de quem olha em relação à poltrona 29.
(B) Incorreta. A poltrona de número 19 apareceria à esquerda de quem olha em relação à poltrona 20.
(C) Incorreta. A poltrona de número 35 apareceria  6 poltronas à direita de quem olha em relação à poltrona 29, a última que aparece na imagem.
(D) Correta. Seguindo a sequência dos números naturais, o número que se encontra entre o 24 e o 26 é o 25.
SAEB: Inferir os elementos ausentes em uma sequência de
números naturais ordenados, objetos ou figuras.
BNCC: EF03MA10 -- Identificar regularidades em sequências ordenadas de números naturais,
resultantes da realização de adições ou subtrações sucessivas, por um mesmo número,
descrever uma regra de formação da sequência e determinar elementos faltantes ou seguintes.

\item
(A) Incorreta. O número 6.079 é o antecessor do antecessor de 6.081.
(B) Incorreta. O número 6.080 é o antecessor de 6.081.
(C) Incorreta. O número 6.082 é o sucessor de 6.081.
(D) Correta. Sucessor do sucessor de 6.081 = 6.081 +1 +1 = 6.083.
SAEB: Inferir os elementos ausentes em uma sequência de números naturais ordenados, objetos ou figuras.
BNCC: EF03MA10 -- Identificar regularidades em sequências ordenadas de números naturais,
resultantes da realização de adições ou subtrações sucessivas, por um mesmo número,
descrever uma regra de formação da sequência e determinar elementos faltantes ou seguintes.
\end{enumerate}

\section*{Simulado 3}

\begin{enumerate}
\item
(A) Incorreta. 4 x 1.000 + 3 x 10 + 5 x 1.
(B) Correta. 4 x 1.000 + 3 x 100 + 3 x 10 + 5 x 1 = 4.335.
(C) Incorreta. 5 x 1.000 + 3 x 100 + 3 x 10 + 4 x 1.
(D) Incorreta. 5 x 1.000 + 3 x 100 + 4 x 1.
SAEB: Compor ou decompor números naturais de até 6 ordens na forma aditiva, ou em suas ordens, ou em adições e multiplicações.
BNCC: EF03MA01 -- Ler, escrever e comparar números naturais de até a ordem de unidade de milhar, estabelecendo relações entre os registros numéricos e em língua materna.

\item
(A) Incorreta. 750.421 inclui o algarismo 0, que não aparece entre as opções de Ricardo.
(B) Correta. 75.421 é o maior número, porque basta colocar os algarismos dados em ordem decrescente para formá-lo.
(C) Incorreta. No número 45.257 aparece repetido o algarismo 5. 
(D) Incorreta. No número 17.545 aparece repetido o algarismo 5. 
SAEB: Compor ou decompor números naturais de até 6 ordens na forma aditiva, ou em suas ordens, ou em adições e multiplicações.
BNCC: EF03MA01 -- Ler, escrever e comparar números naturais de até a ordem de unidade de milhar, estabelecendo relações entre os registros numéricos e em língua materna.

\item
(A) Incorreta. 13 = 1 barra e 3 cubinhos.
(B) Incorreta. 76 = 7 barras e 6 cubinhos.
(C) Correta. 6 barras: 6 x 10 = 60; 7 cubinhos: 7; 60 + 7 = 67.
(D) Incorreta. 21 = 2 barras e 1 cubinho.
SAEB: Compor ou decompor números naturais de até 6 ordens na forma aditiva, ou em suas ordens, ou em adições e multiplicações.
BNCC: EF03MA01 -- Ler, escrever e comparar números naturais de até a ordem de unidade de milhar, estabelecendo relações entre os registros numéricos e em língua materna.

\item
(A) Incorreta. 660 é o número de pessoas que ainda vão conseguir entrar.
(B) Correta. 1.200 -- 540 = 660; 932 -- 660 = 272 pessoas não conseguirão assistir a essa seção.
(C) Incorreta. 268 é o resultado da subtração 1.200 -- 932.
(D) Incorreta. 1.472 é o resultado da soma 540 + 932.
SAEB: Resolver problemas de adição ou de subtração,
envolvendo números naturais de até 6 ordens, com os significados de
juntar, acrescentar, separar, retirar, comparar ou completar.
BNCC: EF03MA06 – Resolver e elaborar problemas de adição e subtração com os significados de
juntar, acrescentar, separar, retirar, comparar e completar quantidades, utilizando diferentes
estratégias de cálculo exato ou aproximado, incluindo cálculo mental.

\item
(A) Incorreta. Essa soma só funciona entre os dois primeiros termos.
(B) Incorreta. Nesse caso, teríamos uma sequência decrescente.
(C) Correta. Basta fazer as divisões de 729 por 3, 243 por 3, 81 por 3 etc. e notar que o resultado será sempre o termo anterior.
(D) Incorreta. Essa razão não existe entre quaisquer dois termos da sequência.
SAEB: Inferir o padrão ou a regularidade de uma sequência de números naturais ordenados, objetos ou figuras. Observando a sequência observa-se que para descobrir um termo basta multiplicar por 3 seu antecessor.
BNCC: EF03MA10 -- Identificar regularidades em sequências ordenadas de números naturais,
resultantes da realização de adições ou subtrações sucessivas, por um mesmo número,
descrever uma regra de formação da sequência e determinar elementos faltantes ou seguintes.

\item
(A) Incorreta. Se o zoológico fechasse às 16 horas e trinta minutos ele ficaria aberto 7h e 30 min.
(B) Correta. 9 + 8,5 = 17,5 = 17 horas e 30 minutos.
(C) Incorreta. Se o zoológico fechasse às 17 horas e 45 minutos ele ficaria aberto 8h e 45 min.
(D) Incorreta. Se o zoológico fechasse às 18 horas e 30 minutos ele ficaria aberto 9h e 30 min.
SAEB: Determinar o horário de início, o horário de término ou a duração de um acontecimento.
BNCC: EF03MA22 -- Ler e registrar medidas e intervalos de tempo, utilizando relógios (analógico e
digital) para informar os horários de início e término de realização de uma atividade e sua duração.

\item
(A) Incorreta. Aumentar apenas os pratos não garantiria bolo a todos.
(B) Incorreta. No caso de aumentar apenas alguns ingredientes, a receita daria errado pela falta de proporção.
(C) Incorreta. Nesse caso, haveria diminuição da quantidade, além de erro na proporção.
(D) Correta. Deve-se ampliar a quantidade de todos os ingredientes seguindo a mesma proporção inicial.
SAEB: Resolver problemas que envolvam medidas de grandezas (comprimento, massa, tempo e capacidade) em que haja conversões entre as
unidades mais usuais.
BNCC: EF03MA20 -- Estimar e medir capacidade e massa, utilizando unidades de medida não
padronizadas e padronizadas mais usuais (litro, mililitro, quilograma, grama e miligrama),
reconhecendo-as em leitura de rótulos e embalagens, entre outros.

\item
(A) Incorreta. Nesse caso, o espaço ocupado seria de 12 quadradinhos, 6 a menos que a realidade.
(B) Incorreta. Nesse caso, o espaço ocupado seria de 17 quadradinhos, 1 a menos que a realidade.
(C) Correta. Contando o númerode quadradinhos que representa o carpete, chega-se a 18 e,
assim, tem-se: 18 x 1 = 18 metros quadrados de carpete.
(D) Incorreta. Nesse caso, o espaço ocupado seria de 20 quadradinhos, 2 a mais que a realidade.
SAEB: Medir ou comparar perímetro ou área de figuras planas desenhadas em malha quadriculada.
BNCC: EF03MA19 -- Estimar, medir e comparar comprimentos, utilizando unidades de medida
não padronizadas e padronizadas mais usuais (metro, centímetro e milímetro) e diversos
instrumentos de medida.

\item
(A) Incorreta. Essa é a soma apenas dos livros de aventura e dos livros de histórias em quadrinhos.
(B) Incorreta. Essa é a soma apenas dos livros de aventura e dos livros de romance.
(C) Incorreta. Essa é a soma apenas dos livros de romance e dos livros de histórias em quadrinhos.
(D) Correta. 12 + 6 + 14 = 32 livros.
SAEB: Resolver problemas de adição ou de subtração,
envolvendo números naturais de até 6 ordens, com os significados de juntar, acrescentar, separar, retirar, comparar ou completar.
BNCC: EF03MA06 – Resolver e elaborar problemas de adição e subtração com os significados de
juntar, acrescentar, separar, retirar, comparar e completar quantidades, utilizando diferentes
estratégias de cálculo exato ou aproximado, incluindo cálculo mental.

\item
(A) Incorreta. Nesse dia foram vendidos 18 sanduíches, o segundo maior número.
(B) Incorreta. Nesse dia foram vendidos 16 sanduíches, o segundo menor número.
(C) Correta. O dia com a maior quantidade vendas foi o dia 07/04. com 25 produtos vendidos.
(D) Incorreta. Nesse dia foram vendidos 14 sanduíches, o menor número.
SAEB: Ler/identificar ou comparar dados estatísticos
expressos em gráficos (barras simples ou agrupadas, colunas simples ou
agrupadas, pictóricos ou de linhas).
BNCC: EF03MA27 -- Ler, interpretar e comparar dados apresentados em tabelas de dupla entrada,
gráficos de barras ou de colunas, envolvendo resultados de pesquisas significativas, utilizando
termos como maior e menor frequência, apropriando-se desse tipo de linguagem para
compreender aspectos da realidade sociocultural significativos.

\item
(A) Incorreta. Nesse caso, foram computados R\$ 49,00 a menos do que o valor correto.
(B) Incorreta. Nesse caso, foram computados R\$ 36,00 a menos do que o valor correto.
(C) Correta. 1 x 100 + 1 x 50 + 1 x 20 + 1 x 5 + 5 x 2 = 100 + 50 + 20 + 5 + 10 = R\$ 185,00.
(D) Incorreta. Nesse caso, foram computados R\$ 15,00 a mais do que o valor correto.
SAEB: Relacionar valores de moedas e/ou cédulas do sistema monetário brasileiro, com base nas imagens desses objetos.
BNCC: EF03MA24 -- Resolver e elaborar problemas que envolvam a comparação e a equivalência de
valores monetários do sistema brasileiro em situações de compra, venda e troca.

\item
(A) Correta. O relógio que está marcando 3 horas e 15 minutos é o da alternativa A.
(B) Incorreta. O relógio, nesse caso, marca 12 horas e 20 minutos - o bolo ainda não estaria pronto.
(C) Incorreta. O relógio, nesse caso, marca 3 horas em ponto - o bolo ainda estaria cru.
(D) Incorreta. O relógio, nesse caso, marca 4 horas em ponto - o bolo teria queimado.
SAEB: Identificar horas em relógios analógicos ou associar horas em relógios analógicos e digitais.
BNCC: EF03MA23 – Ler horas em relógios digitais e em relógios analógicos e reconhecer a relação
entre hora e minutos e entre minuto e segundos.

\item
(A) Incorreta. A lontra e o chimpanzé são os animais com maior tempo de vida.
(B) Incorreta. O hipopótamo vive o dobro do tempo do tigre e do leão; a girafa vive um pouco mais que tigre e leão.
(C) Correta. Analisando o gráfico, percebe-se que o mico-leão-dourado e o lobo-guará são os que vivem menos do que os dois animais especificados no enunciado.
(D) Incorreta. De fato, o mico-leão-dourado vive menos; mas a vida da lontra é das mais longas.
SAEB: Ler/identificar ou comparar dados estatísticos
expressos em gráficos (barras simples ou agrupadas, colunas simples ou agrupadas, pictóricos ou de linhas).
BNCC: EF03MA27 -- Ler, interpretar e comparar dados apresentados em tabelas de dupla entrada,
gráficos de barras ou de colunas, envolvendo resultados de pesquisas significativas, utilizando
termos como maior e menor frequência, apropriando-se desse tipo de linguagem para
compreender aspectos da realidade sociocultural significativos.

\item
(A) Incorreta. Amanda apresenta a segunda menor massa.
(B) Correta. Pela balança, percebe-se que Raquel tem a maior medida de massa corpórea.
(C) Incorreta. Carlos tem a segunda maior massa.
(D) Incorreta. Alex tem a menor massa de todos.
SAEB: Estimar/inferir medida de comprimento, capacidade ou
massa de objetos, utilizando unidades de medida convencionais ou não ou
medir comprimento, capacidade ou massa de objetos.
BNCC: EF03MA20 -- Estimar e medir capacidade e massa, utilizando unidades de medida não
padronizadas e padronizadas mais usuais (litro, mililitro, quilograma, grama e miligrama),
reconhecendo-as em leitura de rótulos e embalagens, entre outros.

\item
(A) Incorreta. Trata-se do número de folhas que cada aluno pegou.
(B) Incorreta. Trata-se do número de alunos.
(C) Incorreta. Trata-se da soma do número de alunos com o número de folhas que cada um pegou.
(D) Correta. 6 x 8 = 48 folhas.
SAEB: Resolver problemas de multiplicação ou de divisão, envolvendo números naturais de até 6 ordens, com os significados de formação de grupos iguais (incluindo repartição equitativa e medida), proporcionalidade ou disposição retangular.
BNCC: EF03MA07 – Resolver e elaborar problemas de multiplicação (por 2, 3, 4, 5 e 10) com os
significados de adição de parcelas iguais e elementos apresentados em disposição retangular,
utilizando diferentes estratégias de cálculo e registros.
\end{enumerate}

\section*{Simulado 4}

\begin{enumerate}
\item
(A) Correta. O algarismo que está no valor posicional de centenas é o 7.
(B) Incorreta. O algarismo 6 está no valor posicional das unidades.
(C) Incorreta. O algarismo 5 está no valor posicional das dezenas.
(D) Incorreta. O algarismo 3 está no valor posicional dos milhares.
SAEB: Compor ou decompor números naturais de até 6 ordens na forma aditiva, ou em suas ordens, ou em adições e multiplicações.
BNCC: EF03MA04 -- Estabelecer a relação entre números naturais e pontos da reta numérica para
utilizá-la na ordenação dos números naturais e também na construção de fatos da adição e da
subtração, relacionando-os com deslocamentos para a direita ou para a esquerda.

\item
(A) Incorreta. Esse é o valor de uma única pizza.
(B) Incorreta. Esse é o valor que seria pago inicialmente, por 2 pizzas.
(C) incorreta. Esse valor representa metade do valor real.
(D) Correta. Valor de cada pizza: R\$ 81,60/2 = R\$ 40,80. Valor de 6 pizzas: 6 x 40,80 = R\$ 244,80.
SAEB: Resolver problemas que envolvam moedas e/ou cédulas do sistema monetário brasileiro.
BNCC: EF03MA24 -- Resolver e elaborar problemas que envolvam a comparação e a equivalência de
valores monetários do sistema brasileiro em situações de compra, venda e troca.

\item
(A) Incorreta. 4 dezenas seriam 40 páginas.
(B) Incorreta. 5 dezenas seriam 50 páginas.
(C) Correta. 60 = 6 x 10; portanto, 6 dezenas.
(D) Incorreta. 5 dezenas seriam 50 páginas.
SAEB: Resolver problemas de multiplicação ou de divisão, envolvendo números naturais de até 6 ordens, com os significados de formação de grupos iguais (incluindo repartição equitativa e medida),
proporcionalidade ou disposição retangular.
BNCC: EF03MA08 -- Resolver e elaborar problemas de divisão de um número natural por outro (até
10), com resto zero e com resto diferente de zero, com os significados de repartição equitativa
e de medida, por meio de estratégias e registros pessoais.

\item
(A)Incorreta. Nesse caso, foram contadas 2 unidades a mais.
(B) Incorreta. Nesse caso, foram contadas 3 unidades a menos.
(C) Correta. Temos representadas na figura 12 unidades de uvas e 11 unidades de morangos; portanto: 12 + 11 = 23 unidades no total que ainda faltam ser adicionadas.
(D) Incorreta. Nesse caso, foram contadas apenas as uvas.
SAEB: Resolver problemas de adição ou de subtração,
envolvendo números naturais de até 6 ordens, com os significados de
juntar, acrescentar, separar, retirar, comparar ou completar.
BNCC: EF03MA06 – Resolver e elaborar problemas de adição e subtração com os significados de
juntar, acrescentar, separar, retirar, comparar e completar quantidades, utilizando diferentes
estratégias de cálculo exato ou aproximado, incluindo cálculo mental.

\item
(A) Incorreta. O número 300, que é o dobro de 150, não é o antecesor de 302, número que falta.
(B) Correta. O número que está faltando na sequência é o 302 (antecessor de 303), pois a lógica embutida é a soma de 100 unidades de um número para o outro.
(C) Incorreta. O número 252, sucessor de 251, não é o número que falta.
(D) Incorreta. O número 250, metade de 500, não é o número que falta.
SAEB: Inferir os elementos ausentes em uma sequência de
números naturais ordenados, objetos ou figuras.
BNCC: EF03MA10 -- Identificar regularidades em sequências ordenadas de números naturais,
resultantes da realização de adições ou subtrações sucessivas, por um mesmo número,
descrever uma regra de formação da sequência e determinar elementos faltantes ou seguintes.

\item
(A) Incorreta. A figura 1 é composta por 6 quadradinhos, não sendo a menor delas.
(B) Correta. A figura 1 é composta por 6 quadradinhos. A figura 2 é composta por 4 quadradinhos. A figura 3 é composta por 5 quadradinhos. A figura 4 é composta por 7 quadradinhos. Como os quadradinhos são de mesmo tamanho, pode-se concluir que a figura que possui a menor área é a figura 2, por ser composta por um número menor de quadradinhos.
(C) Incorreta. A figura 3 é composta por 5 quadradinhos, sendo a segunda maior.
(D) Incorreta. A figura 4 é composta por 7 quadradinhos, o que significa que é a maior de todas.
SAEB: Medir ou comparar perímetro ou área de figuras planas desenhadas em malha quadriculada.
BNCC: EF03MA19 -- Estimar, medir e comparar comprimentos, utilizando unidades de medida
não padronizadas e padronizadas mais usuais (metro, centímetro e milímetro) e diversos
instrumentos de medida.

\item
(A) Incorreta. Se o relógio estivesse marcando 11 horas e 50 minutos, ela teria levado, para se arrumar, 15 minutos.
(B) Incorreta. Se o relógio estivesse marcando 12 horas, ela teria levado, para se arrumar, 25 minutos.
(C) Incorreta. Se o relógio estivesse marcando 12 horas e 5 minutos, ela teria levado, para se arrumar, 30 minutos.
(D) Correta. O relógio está marcando 11 horas e 35 minutos; se acrescentarmos a esse horário 35 minutos, teremos no relógio 12 horas e 10 minutos.
SAEB: Determinar o horário de início, o horário de término ou a duração de um acontecimento.
BNCC: EF03MA23 – Ler horas em relógios digitais e em relógios analógicos e reconhecer a relação
entre hora e minutos e entre minuto e segundos.

\item
(A) Correta. Valor gasto por Rafael: 35 + 3 = R\$ 38,00; dentre as alternativas a que que nos dá esse valor exato é a alternativa A.
(B) Incorreta. 1 cédula de 10 reais, 4 cédulas de 5 reais e 3 moedas de 1 real totalizariam R\$ 33,00.
(C) Incorreta. 2 cédulas de 10 reais, 1 cédula de 5 reais e 3 moedas de 1 real totalizariam R\$ 28,00.
(D) Incorreta. 2 cédulas de 10 reais, 2 cédulas de 5 reais e 2 moedas de 1 real totalizariam R\$ 32,00.
SAEB: Resolver problemas que envolvam moedas e/ou cédulas do sistema monetário brasileiro.
BNCC: EF03MA24 -- Resolver e elaborar problemas que envolvam a comparação e a equivalência de
valores monetários do sistema brasileiro em situações de compra, venda e troca.

\item
(A) Incorreta. Trata-se de um quinto do que Alana tinha.
(B) Incorreta. Trata-se de metade do que Alana tinha.
(C) Correta. 10 x 0,05 + 5 x 0,50 + 70 x 0,10 = 0,50 + 2,50 + 7 = R\$ 10,00.
(D) Incorreta. Trata-se do dobro do valor que Alana tinha.
SAEB: Resolver problemas que envolvam moedas e/ou cédulas do sistema monetário brasileiro.
BNCC: EF03MA24 -- Resolver e elaborar problemas que envolvam a comparação e a equivalência de
valores monetários do sistema brasileiro em situações de compra, venda e troca.

\item
(A) Incorreta. 8 é o número de bexigas brancas, apenas.
(B) Incorreta. 16 é o número de bexigas brancas ou vermelhas multiplicadas por 2.
(C) Incorreta. 24 é o número total de bexigas vermelhas, apenas.
(D) Correta. Bexigas brancas: 8. Bexigas vermelhas: 8 x 3 = 24. Total de bexigas: 8 + 24 = 32.
SAEB: Resolver problemas de multiplicação ou de divisão, envolvendo números naturais de até 6 ordens, com os significados de formação de grupos iguais (incluindo repartição equitativa e medida), proporcionalidade ou disposição retangular.
BNCC: EF03MA07 – Resolver e elaborar problemas de multiplicação (por 2, 3, 4, 5 e 10) com os significados de adição de parcelas iguais e elementos apresentados em disposição retangular, utilizando diferentes estratégias de cálculo e registros.

\item
(A) Incorreta. 7 foram as crianças de 4 a 6 anos.
(B) Incorreta. 12 foram as crianças de 7 a 9 anos.
(C) Incorreta. 16 não é uma quantidade de crianças representada no gráfico.
(D) Correta. Segundo o gráfico apresentado, 12 + 9 = 21 crianças de 7 a 12 anos (que visitaram a loja).
SAEB: Ler/identificar ou comparar dados estatísticos
expressos em gráficos (barras simples ou agrupadas, colunas simples ou agrupadas, pictóricos ou de linhas).
BNCC: EF03MA27 -- Ler, interpretar e comparar dados apresentados em tabelas de dupla entrada,
gráficos de barras ou de colunas, envolvendo resultados de pesquisas significativas, utilizando
termos como maior e menor frequência, apropriando-se desse tipo de linguagem para
compreender aspectos da realidade sociocultural significativos.

\item
(A) Correta. A boneca foi a menos vendida, com apenas 10 unidades.
(B) Incorreta. O tambor foi o brinquedo mais vendido, com 32 unidades.
(C) Incorreta. O carrinho, com 18 unidades, foi o segundo menos vendido.
(D) Incorreta. A bola, com 25 unidades, foi o segundo item mais vendido.
SAEB: Ler/identificar ou comparar dados estatísticos
expressos em gráficos (barras simples ou agrupadas, colunas simples ou agrupadas, pictóricos ou de linhas).
BNCC: EF03MA27 -- Ler, interpretar e comparar dados apresentados em tabelas de dupla entrada,
gráficos de barras ou de colunas, envolvendo resultados de pesquisas significativas, utilizando
termos como maior e menor frequência, apropriando-se desse tipo de linguagem para
compreender aspectos da realidade sociocultural significativos.

\item
(A) Incorreta. Nesse caso, a soma considerou o tênis mais caro e o boné mais barato, apenas.
(B) Incorreta. Nesse caso, a soma considerou o tênis mais caro e o boné mais caro, apenas.
(C) Incorreta. Nesse caso, a soma considerou o tênis mais barato, o boné mais barato e a camiseta mais cara.
(D) Correta. 105 + 18 + 22 = R\$ 145,00.
SAEB: Resolver problemas que envolvam moedas e/ou cédulas do sistema monetário brasileiro.
BNCC: EF03MA24 -- Resolver e elaborar problemas que envolvam a comparação e a equivalência de
valores monetários do sistema brasileiro em situações de compra, venda e troca.

\item
(A) Incorreta. 3 é o número de bolinhas em uma embalagem apenas.
(B) Incorreta. 12 é a quantidade de bolinhas em 4 embalagens.
(C) Incorreta. 15 é a quantidade de bolinhas em 5 embalagens.
(D) Correta. 7 x 3 = 21 bolinhas.
SAEB: Resolver problemas de multiplicação ou de divisão, envolvendo números naturais de até 6 ordens, com os significados de formação de grupos iguais (incluindo repartição equitativa e medida), proporcionalidade ou disposição retangular.
BNCC: EF03MA07 – Resolver e elaborar problemas de multiplicação (por 2, 3, 4, 5 e 10) com os
significados de adição de parcelas iguais e elementos apresentados em disposição retangular,
utilizando diferentes estratégias de cálculo e registros.

\item
(A) Incorreta. 239 é o resultado da subtração 417 -- 178.
(B) Correta. 417 -- 105 = 312.
(C) Incorreta. 595 é o resultado da soma 417 + 178.
(D) Incorreta. 417 é o número da primeira parcela.
SAEB: Calcular o resultado de adições ou subtrações
envolvendo números naturais de até 6 ordens.
BNCC: EF03MA06 – Resolver e elaborar problemas de adição e subtração com os significados de
juntar, acrescentar, separar, retirar, comparar e completar quantidades, utilizando diferentes
estratégias de cálculo exato ou aproximado, incluindo cálculo mental.
\end{enumerate}