\chapter{Respostas}
\pagestyle{plain}
\footnotesize

\pagecolor{gray!40}

\section*{Língua Portuguesa – Módulo 1 – Treino}

\begin{enumerate}
\item SAEB: Localizar informação explícita.
BNCC: EF15LP03 – Localizar informações explícitas em textos.
 a) Incorreta. O texto apenas menciona a Lua para comparar a distância
entre nosso planeta e esse satélite e a rota do asteroide.
b) Incorreta. O texto afirma que não há motivo para alarde na Terra.
c) Correta. De acordo com o texto, o asteroide não representa nenhuma
ameaça para os habitantes do planeta Terra.
d) Incorreta. Segundo o texto, o asteroide apresenta cerca de 1,2 km de
extensão.

\item
SAEB: Inferir informações implícitas em textos.
BNCC: EF35LP04 – Inferir informações implícitas nos textos lidos.
a) Correta. O fato de o texto salientar a preservação da múmia implica
que, normalmente, os achados não apresentam as mesmas características.
b) Incorreta. As informações encontradas no texto implicam exatamente o
contrário.
c) Incorreta. O texto menciona outros achados arqueológicos.
d) Incorreta. O texto destaca a relevância da descoberta.


\item
SAEB: Identificar a ideia central do texto.
BNCC: EF35LP03 – Identificar a ideia central do texto, demonstrando compreensão
global.
a) Incorreta. O texto não menciona danos causados pelo entrada do
asteroide na atmosfera da Terra.
b) Incorreta. O texto faz referência a apenas um asteroide.
c) Incorreta. O texto apenas menciona o fato de o fato ter sido
noticiado pela agência.
d) Correta. O primeiro parágrafo do texto já explica seu tema.
\end{enumerate}

\section*{Língua Portuguesa – Módulo 2 – Treino}

\begin{enumerate}
\item 
SAEB: Identificar elementos constitutivos de textos narrativos.
BNCC: EF35LP26 – Ler e compreender, com certa autonomia, narrativas ficcionais
que apresentem cenários e personagens, observando os elementos da
estrutura narrativa: enredo, tempo, espaço, personagens, narrador e a
construção do discurso indireto e discurso direto.
a) Incorreta. Em alguns casos, as aspas indicam falas em textos narrativos, mas não é o que ocorre com o texto em análise.
b) Correta. Os travessões são usados para introduzir as falas dos
personagens.
c) Incorreta. Vírgulas são usadas para separar frases.
d) Incorreta. Pontos finais são usados para finalizar frases.

\item
SAEB: Analisar os efeitos de sentido de verbos de enunciação.
BNCC: EF05LP09 – Ler e compreender, com autonomia, textos instrucionais de
regras de jogo, dentre outros gêneros do campo da vida cotidiana, de
acordo com as convenções do gênero e considerando a situação
comunicativa e a finalidade do texto.
a) Correta. O verbo “dizer” demarca uma neutralidade na forma de pronunciar a fala.
b) Incorreta. Não há indicação de que o personagem gritou.
c) Incorreta. Não há indicação de que o personagem sussurrou.
d) Incorreta. Não há indicação no texto da posição do personagem que fala ou da posição de seu interlocutor.


\item
SAEB: Identificar as marcas de organização de textos dramáticos.
BNCC: EF35LP24 – Identificar funções do texto dramático (escrito para ser
encenado) e sua organização por meio de diálogos entre personagens e
marcadores das falas das personagens e de cena.
a) Incorreta. Trata-se de um texto dramático; portanto não há narrador.
b) Incorreta. A informação não se relaciona com o cenário da peça.
c) Incorreta. A indicação entre parênteses, apesar de aparecer antes da fala, após a nomeação do personagem, não abre uma fala.
d) Correta. A indicação marca que o personagem fala enquanto caminha em direção à outra personagem.
\end{enumerate}

\section*{Língua Portuguesa – Módulo 3 – Treino}

\begin{enumerate}
\item
SAEB: Analisar elementos constitutivos de gêneros textuais diversos.
BNCC: EF35LP16 – Identificar e reproduzir, em notícias, manchetes, lides
e corpo de notícias simples para público infantil e cartas de reclamação
(revista infantil), digitais ou impressos, a formatação e diagramação
específica de cada um desses gêneros, inclusive em suas versões orais.
a) Correta. É comum, em notícias, o uso do presente do indicativo para noticiar fatos já ocorridos, o que aproxima o fato do leitor.
b) Incorreta. Pelo contrário, cria-se uma aproximação temporal.
c) Incorreta. O presente do indicativo, nesses casos, substitui a forma no pretérito.
d) Incorreta. O fato ocorreu antes do momento da publicação da notícia.

\item
SAEB: Analisar elementos constitutivos de gêneros textuais diversos.
BNCC: EF35LP16 – Identificar e reproduzir, em notícias, manchetes, lides
e corpo de notícias simples para público infantil e cartas de reclamação
(revista infantil), digitais ou impressos, a formatação e diagramação
específica de cada um desses gêneros, inclusive em suas versões orais.
a) Correta. A marca “nesta quinta-feira (19)” demonstra que se trata de um fato que, na época de publicação da notícia, era recente.
b) Incorreta. Demonstra-se, no texto, que o acontecimento é recente.
c) Incorreta. Não há um juízo de valor quanto ao fato.
d) Incorreta. Está claro no texto que o fato é passado.

\item
SAEB: Analisar os efeitos de sentido decorrentes do uso da pontuação.
BNCC: EF05LP04 – Diferenciar, na leitura de textos, vírgula, ponto e
vírgula, dois-pontos e reconhecer, na leitura de textos, o efeito de
sentido que decorre do uso de reticências, aspas, parênteses.
a) Correta. O ponto de interrogação é usado para indicar uma pergunta.
b) Incorreta. O ponto de exclamação é que é usado para destacar uma
informação.
c) Incorreta. A vírgula é usada para dividir duas frases.
d) Incorreta. O final de um texto é indicado por um ponto final.
\end{enumerate}

\section*{Língua Portuguesa – Módulo 4 – Treino}

\begin{enumerate}
\item
SAEB: Analisar o uso de recursos de persuasão em textos verbais e/ou
multimodais. BNCC: EF05LP20 – Analisar a validade e força de argumentos
em argumentações sobre produtos de mídia para público infantil (filmes,
desenhos animados, HQs, games etc.), com base em conhecimentos sobre os
mesmos.
a) Incorreta. O texto não faz menção ao processo de fabricação das
vacinas.
b) Incorreta. O texto desenvolve justamente o argumento oposto.
c) Incorreta. O texto não cita uma doença específica.
d) Correta. O texto salienta a importância da vacinação.

\item
SAEB: Julgar a eficácia de argumentos em textos. BNCC: EF05LP20 -
Analisar a validade e força de argumentos em argumentações sobre
produtos de mídia para público infantil (filmes, desenhos animados, HQs,
games etc.), com base em conhecimentos sobre os mesmos.
a) Correta. O texto menciona os testes pelos quais uma vacina deve
passar.
b) Incorreta. O texto menciona doenças, mas não cria uma relação quanto ao surgimento de doenças em determinado espaço de tempo.
c) Incorreta. Na reaidade, é dito exatamente o contrário no texto.
d) Incorreta. O texto menciona a possibilidade de uma criança ficar
doente caso não seja vacinada.

\item
SAEB: Analisar os efeitos de sentido de recursos multissemióticos em
textos que circulam em diferentes suportes. BNCC: EF05LP20 – Analisar a
validade e força de argumentos em argumentações sobre produtos de mídia
para público infantil (filmes, desenhos animados, HQs, games etc.), com
base em conhecimentos sobre os mesmos.
a) Incorreta. A imagem não faz referência à vacinação.
b) Incorreta. Embora a imagem apresente a imagem de duas mãos, não há
referência a cumprimentos.
c) Correta. A imagem apresenta a imagem de duas mãos sendo higienizadas.
d) Incorreta. A imagem não faz referência a aglomerações.
\end{enumerate}

\section*{Língua Portuguesa – Módulo 5 – Treino}
\begin{enumerate}
\item
SAEB: Reconhecer diferentes modos de organização composicional de textos
em versos. BNCC: EF35LP27 – Ler e compreender, com certa autonomia,
textos em versos, explorando rimas, sons e jogos de palavras, imagens
poéticas (sentidos figurados) e recursos visuais e sonoros.
a) Correta. Cada estrofe apresenta oito versos.
b) Incorreta. O trecho acima apresenta duas estrofes, e não dois versos.
c) Incorreta. O poema apresenta rimas.
d) Incorreta. Há expressão de sentimentos nesse trecho do poema.

\item
SAEB: Reconhecer diferentes modos de organização composicional de textos
em versos. BNCC: EF35LP27 – Ler e compreender, com certa autonomia,
textos em versos, explorando rimas, sons e jogos de palavras, imagens
poéticas (sentidos figurados) e recursos visuais e sonoros.
a) Incorreta. A sonoridade das palavras não é semelhante.
b) Incorreta. Embora os dois trechos apresentem palavras iguais, não é
estabelecida uma rima.
c) Correta. Os dois verbos são utilizados ao final de dois versos
diferentes, criando uma rima.
d) Incorreta. As duas palavras aparecem uma após a outra na estrofe.

\item
SAEB: Analisar a construção de sentidos de textos em versos com base em
seus elementos constitutivos. BNCC: EF35LP31 – Identificar, em textos
versificados, efeitos de sentido decorrentes do uso de recursos rítmicos
e sonoros e de metáforas.
a) Incorreta. A presença de termos negativos impede que o sentimento de
alegria seja identificado.
b) Correta. O uso de um termo como “pavor” expõe o sentimento de
desespero.
c) Incorreta. A voz que fala no poema não aparenta aceitar os sentimentos negativos que
a acometem.
d) Incorreta. Menciona-se “Este frio que anda em mim”,
o que demonstra sofrimento.
\end{enumerate}

\section*{Língua Portuguesa – Módulo 6 – Treino}

\begin{enumerate}
\item
SAEB: Identificar as variedades linguísticas em textos. BNCC:EF35LP05- Inferir o sentido de palavras ou expressões desconhecidas em textos, com base no contexto da frase ou do texto.
a) Incorreta. O termo mais apropriado para o significado descrito seria
“mesinha”.
b) Correta. O termo “mezinha” refere-se a uma medicação natural.
c) Incorreta. O termo mais apropriado para o significado descrito seria “mezena”.
d) Incorreta. O termo mais apropriado para o significado descrito seria
“mezanino”.

\item
SAEB: Identificar as variedades linguísticas em textos. BNCC:EF35LP05- Inferir o sentido de palavras ou expressões desconhecidas em textos, com base no contexto da frase ou do texto.
a) Correta. A expressão “numa relancina” equivale ao advérbio “rapidamente”.
b) Incorreta. O termo seria antônimo de “numa relancina”.
c) Incorreta. Como “fazer algo numa relancina” refere-se a realizar algo rapidamente,
não há o sentido de dificuldade.
d) Incorreta. A expressão “numa relancina” tem o sentido de velocidade e prontidão;
então não há sentido de pausa.

\item
SAEB: Identificar as variedades linguísticas em textos. BNCC: EF35LP30 – Diferenciar discurso indireto e discurso direto, determinando o efeito de sentido de verbos de enunciação e explicando o uso de variedades linguísticas no discurso direto, quando for
o caso.
a) Incorreta. Sendo discurso direto, o narrador reproduz exatamente a
fala do personagem.
b) Correta. Como é discurso direto, o personagem falou exatamente dessa
forma.
c) Incorreta. A definição de discurso direto se relaciona com a fala
exata do personagem, e não o contrário.
d) Incorreta. Sendo discurso direto, sua definição pressupõe a fala
exata do personagem.
\end{enumerate}

\section*{Língua Portuguesa – Módulo 7 – Treino}

\begin{enumerate}
\item
SAEB: Analisar os efeitos de sentido decorrentes do uso dos adjetivos.
a) Incorreta. Pela adjetivação, fica clara a ideia de que a avó está bastante fragilizada.
b) Correta. Todos os adjetivos empregados constroem a ideia de que a avó descrita não se encontra mais no viço da idade.
c) Incorreta. Pelo contrário, ela aparece no poema muito mais frácil que uma jovem.
d) Incorreta. Nas duas primeiras estrofes do poema, não estão explicitados os sentimentos da própria avó.

\item
SAEB: Analisar os efeitos de sentido decorrentes do uso dos adjetivos.
a) Incorreta. Pelo contrário, a avó, que estava dormindo, acorda alegre ao veros netos invadindo a sala.
b) Incorreta. O fato de seu rosto ficar mais lindo mostra que ela se importa positivamente com os netos ao vê-los chegar.
c) Correta. Com a chegada dos netos à sala, a avó se transforma, e o adjetivo “lindo” marca essa transformação.
d) Incorreta. Apesar de cansada e debilitada, a avó ainda reage ao mundo ao seu redor.

\item
SAEB: Analisar os efeitos de sentido decorrentes do uso dos advérbios.
a) Incorreta. Na realidade, os advérbios reforçam a ideia contrária.
b) Correta. Em mais um momento do poema, fica claro que a idade da avó a debilita, o que a faz ter movimentos trêmulos e lentos.
c) Incorreta. Os advérbios não apontam para sentimentos de saudosismo e alegria, apesar de a avó estar contente na cena.
d) Incorreta. A avó não está nem infeliz nem raivosa; peo contrário, alegra-se com a presença dos netos.
\end{enumerate}

\section*{Língua Portuguesa – Módulo 8 – Treino}

\begin{enumerate}
\item
SAEB: Avaliar a fidedignidade de informações sobre um mesmo fato veiculadas em diferentes mídias. BNCC:
EF05LP16 – Comparar informações sobre um mesmo fato veiculadas em diferentes mídias e concluir sobre qual é mais confiável e por quê.
a) Incorreta. Essa afirmação não é sustentada pelo texto, pois não há
nenhuma menção às empresas farmacêuticas.
b) Incorreta. A disseminação da notícia errada sobre a vacina contra novo coronavírus se deve principalmente à rede antivacina que usa táticas
específicas para promover conteúdo desinformativo.
c) Correta. A notícia falsa foi compartilhada por uma rede de ativistas
antivacina, ou seja, foi algo intencional.
d) Incorreta. Não há nenhuma menção, no texto, a políticos que se opõem à vacinação
contra a covid-19.

\item
SAEB: Avaliar a fidedignidade de informações sobre um mesmo fato veiculadas em diferentes mídias. BNCC:
EF05LP16 – Comparar informações sobre um mesmo fato veiculadas em diferentes mídias e concluir sobre qual é mais confiável e por quê.
a) Incorreta. As vacinas são descritas como sendo seguras, por terem
sido desenvolvidas segundo normas científicas.
b) Correta. O artigo se concentra em explicar como uma notícia errada
foi compartilhada no Facebook e se tornou viral devido a uma rede
comprometida de ativistas antivacina.
c) Incorreta. O texto mostra que as informações sobre a insegurança das
vacinas são falsas; portanto as pessoas podem confiar na ciência.
d) Incorreta. O texto mostra que, pelo contrário, foram os ativistas
anti-vacina que divulgaram notícias falsas.

\item
SAEB: Distinguir fatos de opiniões em textos. BNCC: EF05LP16 – Comparar
informações sobre um mesmo fato veiculadas em diferentes mídias e concluir sobre
qual é mais confiável e por quê.
a) Incorreta. O Centro defende a eficácia das vacinas e, por isso, incentiva
que as pessoas se vacinem.
b) Incorreta. O Centro busca divulgar notícias baseadas em fatos para
evitar a disseminção de notícias falsas.
c) Correto. O Centro busca encorajar o conhecimento e a propagação das
vacinas, já que são comprovadamente eficazes.
d) Incorreta. O Centro baseia-se em pesquisas científicas organizadas em
dados para defender as vacinas – entre as pesquisas, as com dados populacionais.
\end{enumerate}

\section*{Língua Portuguesa – Módulo 9 – Treino}

\begin{enumerate}
\item
SAEB: Analisar informações apresentadas em gráficos,
infográficos ou tabelas. Habilidades da BNCC: EF05LP23 – Comparar
informações apresentadas em gráficos ou tabelas.
a) Incorreto. A taxa se refere ao estado do Piauí.
b) Incorreto. A taxa se refere ao estado do Maranhão.
c) Correto. A taxa se refere ao estado de São Paulo (68,4\%).
d) Incorreto. A taxa se refere aos estados do Pará e do Maranhão.

\item
SAEB: Analisar informações apresentadas em gráficos,
infográficos ou tabelas. Habilidades da BNCC: EF05LP23 – Comparar
informações apresentadas em gráficos ou tabelas.
a) Incorreta. A diferença é de 13,6 pontos percentuais, que é maior que a menor diferença (2 pontos percentuais).
b) Incorreta. A diferença é de 4,1 pontos percentuais, que é maior que a menor diferença (2 pontos percentuais).
c) Correta. A diferença de 2 pontos percentuais é a menor que aparece entre os dados do gráfico.
d) Incorreta. A diferença é de 18,7 pontos percentuais, que é maior que a menor diferença (2 pontos percentuais).

\item
SAEB: Analisar informações apresentadas em gráficos,
infográficos ou tabelas. Habilidades da BNCC: EF05LP23 – Comparar
informações apresentadas em gráficos ou tabelas.
a) Incorreta. O Pará teve taxa de 82,6\%, menor que a do Maranhão, com
96,2\%.
b) Incorreta. Roraima teve taxa de 60,2\%, menor que a do Maranhão, com
96,2\%.
c) Incorreta. São Paulo teve taxa de 68,4\%, menor que a do Maranhão, com
96,2\%.
d) Correta. A taxa do Maranhão (maior faixa do gráfico) foi de 96,2\%. Também é facilmente verificável, visualmente, que a barra atribuída ao estado do Maranhão é a maior entre as apresentadas no gráfico.
\end{enumerate}

\section*{Língua Portuguesa – Módulo 10 – Treino}

\begin{enumerate}
\item
SAEB: Analisar relações de causa e consequência. BNCC:
EF35LP26 – Ler e compreender, com certa autonomia, narrativas ficcionais que apresentem
cenários e personagens, observando os elementos da estrutura narrativa: enredo, tempo, espaço,
personagens, narrador e a construção do discurso indireto e discurso direto.
a) Correta. A mentira do pai da jovem a obrigou a sustentar a mentira
diante do rei, fingindo saber transformar palha em ouro.
b) Incorreta. O rei decidiu se casar em consequência do ouro fabricado
pelo duende.
c) Incorreta. O duende apareceu buscando o que a moça lhe daria em troca
do favor de salvar sua vida.
d) Incorreta. O acordo de entregar o primeiro filho veio em consequência
de a moça não ter mais nada de valor para dar em troca do favor do
duende.

\item
SAEB: Identificar os mecanismos de progressão textual. BNCC:
EF05LP07 – Identificar, em textos, o uso de conjunções e a relação
que estabelecem entre partes do texto: adição, oposição, tempo, causa,
condição, finalidade.
a) Incorreta. A preposição “sem”, nesse caso, introduz uma oração reduzida de infinitivo que cria uma circunstância de causa.
b) Correta. A locução conjuntiva “em que” estabelece uma circunstância de tempo.
c) Incorreta. A preposição “para” introduz uma oração reduzida de infinitivo que cria uma circunstância e finalidade.
d) Incorreta. A conjunção “e” estabelece uma adição (relação de coordenação).

\item
SAEB: Identificar os mecanismos de referenciação lexical e pronominal.
a) Incorreta. O “o” que antecede “duende” é um artigo.
b) Correta. O pronome mencionado retoma e substitui o filho da moça e aparece no feminino porque concorda com o substantivo “criança”.
c) Incorreta. O pronome “sua”, ligado a “liberdade”, refere-se à moça.
d) Incorreta. O pronome “seu”, ligado a “nome”, refere-se ao duende.
\end{enumerate}

\section*{Arte – Módulo 1 –  Treino}

\begin{enumerate}
\item
SAEB: Reconhecer elementos constitutivos das artes visuais, dança,
música e teatro. BNCC: EF15AR02 – Explorar e reconhecer elementos constitutivos das artes
visuais (ponto, linha, forma, cor, espaço, movimento etc.).
a) Incorreta. Em geral, as instalações são obras passageiras, com
duração determinada.
b) Incorreta. No espaço dimensional, a obra possui duas dimensões: altura
e largura.
c) Correta. É uma obra com três dimensões: altura, largura e
profundidade.
d) Incorreta. A legenda informa o tamanho da obra: 15 m x 15 m x 4,5 m, ou
seja, é uma obra de grandes dimensões.

\item
SAEB: Identificar características do sistema de circulação das artes
visuais, dança, música e teatro em diferentes contextos (teatros,
palcos, museus, galerias, artistas, artesãos, curadores, produtores
etc.). BNCC: EF15AR01 – Identificar e apreciar formas distintas das artes
visuais tradicionais e contemporâneas, cultivando a percepção, o
imaginário, a capacidade de simbolizar e o repertório imagético.
a)  Incorreta. O educador é o responsável por dar informações sobre o que
  está exposto no museu.
b) Incorreta. O restaurador é o profissional responsável pela conservação
  e pela restauração dos objetos.
c) Incorreta. O historiador é o responsável pela pesquisa, pela classificação
  e pela análise de documentos e objetos do passado.
d) Correta. O orientador de público é o responsável por supervisionar as
  áreas expositivas, garantido que as regras de comportamento do museu
  sejam seguidas.

\item
SAEB: Identificar distintas formas e/ou gêneros de expressão da dança,
da música e do teatro em diferentes contextos e práticas. BNCC: EF15AR13 – Identificar e apreciar criticamente diversas formas e
gêneros de expressão musical, reconhecendo e analisando os usos e as
funções da música em diversos contextos de circulação, em especial,
aqueles da vida cotidiana.
a)  Incorreta. O hip-hop surgiu nas comunidades afro-americanas na década
  de 1970. Além da música, esse movimento encontra-se representado na
  dança e no grafite.
b) Incorreta. O rock teve sua origem nos Estados Unidos, no final da
  década de 1940.
c) Correta. O samba surgiu nas comunidades afro-brasileiras, no começo do
  século XX. Além de gênero musical, também encontra-se representado na
  dança.
d) Incorreta. O sertanejo é um gênero musical brasileiro, com sua origem
  no início do século XX. Como o próprio nome indica, esse gênero surgiu
  no sertão do Brasil.
\end{enumerate}

\section*{Arte – Módulo 2 –  Treino}

\begin{enumerate}
\item
SAEB: Analisar relações entre as partes corporais e seu todo na
estética da dança.
a) Incorreta. A alternativa refere-se a problemas de saúde vinculados
à prática excessiva ou incorreta das técnicas do balé.
b) Incorreta. As sapatilhas são peças do vestuário clássico dos
dançarinos de balé e auxiliam na execução correta dos movimentos.
c) Incorreta. A alternativa trata do contexto histórico e social para a prática do
balé naquela época.
d) Correta. O ensino tradicional do balé clássico valoriza o domínio dos
movimentos e das técnicas exaustivamente.

\item
SAEB: Identificar as características de instrumentos musicais
variados, bem como o potencial musical do corpo humano.
a) Incorreta. São os idiofones que produzem som pela vibração de
seu próprio corpo. Exemplo: sino.
b) Incorreta. São os cordofones que produzem som pela vibração das
cordas. Exemplo: violino.
c) Incorreta. São os membrafones que produzem som por meio de uma
membrana. Exemplo: tambor.
d) Correta. A flauta é classificada como aerofone, ou seja, produz som
pela vibração do ar.

\item
SAEB: Reconhecer a influência de distintas matrizes estéticas e
culturais nas manifestações das artes visuais, dança, música e teatro na
cultura brasileira.
BNCC: EF15AR03 – Reconhecer e analisar a influência de distintas
matrizes estéticas e culturais das artes visuais nas manifestações
artísticas das culturas locais, regionais e nacionais.
a) Correta. A principal matriz estética e cultural do Samba de Roda do
Recôncavo Baiano é a africana.
b) Incorreta. Fazem parte da cultura asiática países como China,
Japão, Índia, Coreia do Norte e do Sul, entre outros.
c) Incorreta. Apesar de a cultura europeia fazer parte das matrizes
estéticas e culturais do Samba de Roda do Recôncavo Baiano, a matriz
principal é a africana.
d) Incorreta. O texto não inclui a matriz estética e cultural indígena.
\end{enumerate}

\section*{Arte – Módulo 3 –  Treino}

\begin{enumerate}
\item
SAEB: Avaliar nas linguagens artísticas a diversidade do patrimônio
cultural da humanidade (material e imaterial), em especial o brasileiro,
a partir de suas diferentes matrizes. BNCC: EF15AR25 – Conhecer e valorizar
o patrimônio cultural, material e imaterial, de culturas diversas, em especial
a brasileira, incluindo-se suas matrizes indígenas, africanas e europeias, de
diferentes épocas, favorecendo a construção de vocabulário e repertório relativos
às diferentes linguagens artísticas.
a) Correta. As bonecas apresentadas na imagem foram feitas com retalhos
de tecido, apresentam nós na sua confecção e foram feitas sem costura.
Além disso, nas bonecas abayomi, em sua origem, não havia a
possibilidade de demarcação de olho, nariz nem boca.
b) Incorreta. Trata-se de uma boneca feita de retalhos de pano, mas nela
a tecnologia da costura se faz presente.
c) Incorreta. É uma boneca feita com retalhos de pano, com vestimenta
complexa, colares de contas, pulseiras, e nela a tecnologia da costura se
faz presente.
d) Incorreta. São bonecos de pano feitos com retalhos de tecidos, mas seus
corpos e as vestimentas foram costurados.

\item
SAEB: Avaliar nas linguagens artísticas a diversidade do patrimônio
cultural da humanidade (material e imaterial), em especial o brasileiro,
a partir de suas diferentes matrizes.
BNCC: EF15AR25 – Conhecer e valorizar o patrimônio cultural, material e
imaterial, de culturas diversas, em especial a brasileira, incluindo-se
suas matrizes indígenas, africanas e europeias, de diferentes épocas,
favorecendo a construção de vocabulário e repertório relativos às
diferentes linguagens artísticas.
a)  Incorreta. A cidade de Ouro Preto (MG) é um bem de natureza material.
  Os bens de natureza material podem ser imóveis como as cidades
  históricas.
b) Incorreta. O Parque Estadual Veredas do Peruaçu (MG) é um bem imóvel
  e, por isso, de natureza material. Trata-se de um monumento natural,
  constituído por formações físicas e biológicas, detentor de valor do
  ponto de vista estético e científico.
c) Correta. Os conhecimentos, as técnicas e as práticas se constituem
  patrimônios imateriais. Mestre de capoeira é um ofício e, por isso,
  enquadra-se na classificação de bem imaterial.
d) Incorreta. A Matriz de Nossa Senhora da Conceição é um bem imóvel, ou
  seja, de natureza material.

\item
SAEB: Avaliar nas linguagens artísticas a diversidade do patrimônio
cultural da humanidade (material e imaterial), em especial o brasileiro,
a partir de suas diferentes matrizes.
BNCC: EF15AR25 – Conhecer e valorizar o patrimônio cultural, material e
imaterial, de culturas diversas, em especial a brasileira, incluindo-se
suas matrizes indígenas, africanas e europeias, de diferentes épocas,
favorecendo a construção de vocabulário e repertório relativos às
diferentes linguagens artísticas.
a) Incorreta. A literatura de cordel caracteriza-se por ser composta de versos.
b) Incorreta. A literatura de cordel possui uma variedade de temas
  populares e da cultura nacional.
c) Correta. Apesar de ser apreciada e ter representantes em todo o
  Brasil, a literatura de cordel é de tradição nordestina.
d) Incorreta. A linguagem da literatura de cordel é popular, oral e
  informal.
\end{enumerate}

\section*{Ciências Humanas – Módulo 1 – Treino}

\begin{enumerate}
\item
BNCC: EF05HI07 - Identificar os processos de produção,
hierarquização e difusão dos marcos de memória e discutir a presença
e/ou a ausência de diferentes grupos que compõem a sociedade na nomeação
desses marcos de memória.
a) Incorreta. Os presidentes de países importantes têm impacto em
questões de grande escala mundial e suas decisões situam-se no nível
macro.
b) Incorreta. Guerras entre grandes sociedades afetam também as
populações ao redor em grande escala, compondo o nível macro.
c) Correta. A micro-história é a história do cotidiano e de personagens
que, originalmente, não estão nos livros de história, como as empregadas
domésticas.
d) Incorreta. A ocupação de continentes, como a da América pelos
europeus, faz parte da macro-história.

\item
BNCC: EF05HI01 - Identificar os processos de formação das culturas e dos povos, relacionando-os com o espaço geográfico ocupado.
a) Incorreta. Não há indícios sobre a nomeação baseada na alimentação ou vestimenta dos moradores.
b) Incorreta. Não é dada importância à economia na rua, nem à cultura popular de forma geral.
c) Correta. De fato, a origem nordestina dos moradores e Inhuma como símbolo da cidade de Guarulhos são os fatores responsáveis pela
nomeação.
d) Incorreta. Não há menção à localização como fator, nem à arquitetura
das moradias.

\item
BNCC: EF05HI08 - Identificar formas de marcação da passagem do
tempo em distintas sociedades, incluindo os povos indígenas originários
e os povos africanos.
a) Correta. De fato, o autor faz referência à mudança do tempo enquanto
ligada aos acontecimentos naturais, como a chuva e a queima da terra.
b) Incorreta. Não é mencionada a guerra, nem conflitos com outros
povos.
c) Incorreta. Não é falado sobre doenças ou sobre a cura delas.
d) Incorreta. Apesar de mencionar os avós e os pais, o indígena não os
relaciona com a mudança do tempo, somente os situa dentro dele.
\end{enumerate}

\section*{Ciências Humanas – Módulo 2 – Treino}

\begin{enumerate}
\item
BNCC: EF05GE10 - Reconhecer e comparar atributos da qualidade ambiental e algumas formas de poluição dos cursos de água e dos oceanos (esgotos,
efluentes industriais, marés negras etc.).
a) Incorreta. A prefeitura não visa a aumentar o número de turistas, mas
controlar o impacto da vinda de turistas sobre a natureza.
b) Incorreta. Não existe relação entre a taxa ambiental e a
criminalidade na cidade.
c) Incorreta. Não existem evidências no texto sobre o consumo de
alimentos.
d) Correta. De fato, a taxa visa a compensar os impactos dos turistas
sobre as praias locais.

\item
BNCC: EF05GE11 - Identificar e descrever problemas ambientais que ocorrem no
entorno da escola e da residência (lixões, indústrias poluentes,
destruição do patrimônio histórico etc.), propondo soluções (inclusive
tecnológicas) para esses problemas.
a) Incorreta. A imagem não se relaciona ao saneamento, mas sim à
reciclagem.
b) Incorreta. A imagem não influi no sistema industrial, mas no descarte
do lixo.
c) Correta. A imagem mostra o processo de separação do lixo, parte da
reciclagem.
d) Incorreta. A imagem não dá sinais de impactar a ambientação urbana.

\item
BNCC: EF05GE03 - Identificar as formas e funções das cidades e
analisar as mudanças sociais, econômicas e ambientais provocadas pelo
seu crescimento.
a) Incorreta. Esse trecho da lei não menciona as escolas quando aborda o
contexto urbano.
b) Incorreta. O trecho não fala sobre crescimento populacional.
c) Correta. De fato, o trecho busca incentivar sistemas e tecnologias
construtivas no ambiente urbano que diminuam os impactos à natureza.
d) Incorreta. O trecho não menciona políticas de produção alimentícia.
\end{enumerate}

\section*{Ciências Humanas – Módulo 3 – Treino}

\begin{enumerate}
\item
BNCC: EF05GE02 - Identificar diferenças étnico-raciais e étnico-culturais e
desigualdades sociais entre grupos em diferentes territórios.
a) Incorreta. O texto não fala sobre a região habitada pelos grupos.
b) Correta. De fato, a diferença da condição financeira dos grupos
afetou um mais do que o outro no combate à pandemia.
c) Incorreta. O texto não menciona uma diferença entre os grupos em relação à
higienização.
d) Incorreta. O texto não fala sobre hábitos alimentares.

\item
BNCC: EF05HI03 - Analisar o papel das culturas e das religiões na composição
identitária dos povos antigos.
a) Incorreta. O texto não fala sobre a colheita acontecer na Ágora,
somente o comércio.
b) Correta. O texto fala sobre os debates públicos que ocorriam na
Ágora, um espaço de expressão dos considerados cidadãos na Grécia
antiga.
c) Incorreta. O texto não registra rituais acontecidos na Ágora.
d) Incorreta. O texto não fala sobre combates entre as cidades, mas da
estrutura interna de cada cidade.

\item
BNCC: EF05HI10 - Inventariar os patrimônios materiais e imateriais da
humanidade e analisar mudanças e permanências desses patrimônios ao
longo do tempo.
a) Incorreta. O texto não faz menção a uma importância ambiental nem
religiosa.
b) Incorreta. O texto não aponta intenções políticas nem alimentares
relacionadas às bonecas.
c) Incorreta. O texto não fala sobre a escola nem sobre o comportamento
das mulheres que produzem as bonecas.
d) Correta. O texto enfatiza a importância das bonecas para a renda das
famílias e sua relevância no contexto cultural familiar.
\end{enumerate}

\section*{Ciências Humanas – Módulo 4 – Treino}

\begin{enumerate}
\item
BNCC: EF05HI02 - Identificar os mecanismos de organização do
poder político com vistas à compreensão da ideia de Estado e/ou de
outras formas de ordenação social.
a) Incorreta. Um bairro está dentro de um município.
b) Incorreta. Um estado tem vários municípios dentro dele.
c) Correta. Os municípios equivalem às cidades; assim, Campinas é um
município.
d) Incorreta. Uma região é composta por vários estados, que têm vários
municípios.

\item
BNCC: EF05HI02 - Identificar os mecanismos de organização do
poder político com vistas à compreensão da ideia de Estado e/ou de
outras formas de ordenação social.
a) Incorreta. Não há indícios no texto sobre a relação da nomeação com
proteção.
b) Correta. Ao nomear um território, estamos expressando as
características identitárias de seu povo.
c) Incorreta. Não há relação no texto ente nome e valor econômico.
d) Incorreta. Não há relação no texto entre nomeação e beleza natural.

\item
BNCC: EF05HI02 - Identificar os mecanismos de organização do
poder político com vistas à compreensão da ideia de Estado e/ou de
outras formas de ordenação social.
a) Incorreta. Não é o professor, mas o estado o responsável pela criação
de políticas públicas.
b) Incorreta. O estado é o responsável pela criação de políticas
públicas, cabe à escola segui-las.
c) Incorreta. Não é a família, mas o estado o responsável pela criação
de políticas públicas.
d) Correta. O estado e o poder público são responsáveis pela criação de
políticas sociais.
\end{enumerate}

\section*{Ciências Humanas – Módulo 5 – Treino}

\begin{enumerate}
\item
BNCC: EF05HI04 - Associar a noção de cidadania com os
princípios de respeito à diversidade, à pluralidade e aos direitos
humanos.
a) Correta. Vital fala que estamos escolhendo quem vai tomar as decisões
do futuro do nosso país por nós
b) Incorreta. Vital não fala sobre decidir sua profissão.
c) Incorreta. Vital não fala que o voto serve para disfarçar nossos
erros.
d) Incorreta. Vital não coloca o voto como diretamente relacionado com a
saúde corporal.

\item
BNCC: EF05HI05 - Associar o conceito de cidadania à conquista
de direitos dos povos e das sociedades, compreendendo-o como conquista
histórica.
a) Incorreta. O texto não fala sobre a intenção de ganhar dinheiro
b) Incorreta. O texto não fala sobre tentativa de prejudicar o outro.
c) Correta. De fato, o texto fala que os jovens utilizam as redes para
dar visibilidade a suas lutas e a sua cultura.
d) Incorreta. Não há menção a um pedido de separação do território
brasileiro.

\item
BNCC: EF05HI04 - Associar a noção de cidadania com os
princípios de respeito à diversidade, à pluralidade e aos direitos
humanos.
a) Correta. A construção de habitações populares é eficaz para melhorar
o acesso à moradia de pessoas em situação de pobreza.
b) Incorreta. Essa medida não atua diretamente sobre o acesso à moradia.
c) Incorreta. Essa medida garante a segurança, não o acesso à moradia.
d) Incorreta. Essa medida garante a segurança alimentar, não o acesso
à moradia.
\end{enumerate}

\section*{Ciências Humanas – Módulo 6 – Treino}

\begin{enumerate}
\item
BNCC: EF05GE05 - Identificar e comparar as mudanças dos tipos de trabalho
e desenvolvimento tecnológico na agropecuária, na indústria, no comércio
e nos serviços.
a) Incorreto. Não se trata de um pasto, mas de uma indústria.
b) Incorreto. Não se trata de uma plantação, mas de uma indústria.
c) Correto. A imagem mostra um ambiente industrial.
d) Incorreto. Não se trata de uma mina, mas de uma indústria.

\item
BNCC: EF05GE05 - Identificar e comparar as mudanças dos tipos de trabalho
e desenvolvimento tecnológico na agropecuária, na indústria, no comércio
e nos serviços.
a) Incorreta. O avião é um transporte atual e não antigo.
b) Correta. A carruagem puxada por cavalo era utilizada antigamente para o
transporte de mercadorias.
c) Incorreta. O caminhão carreta é um transporte moderno.
d) Incorreta. As motos são veículos modernos, além de não serem considerados veículos de carga.

\item
BNCC: EF05GE05 - Identificar e comparar as mudanças dos tipos de trabalho
e desenvolvimento tecnológico na agropecuária, na indústria, no comércio
e nos serviços.
a) Incorreta. O texto não fala sobre a falta de estudos, mas sim sobre
os avanços neles.
b) Incorreta. O texto não fala sobre técnicas tradicionais, mas sobre
novas técnicas de produção.
c) Incorreta. O texto fala justamente sobre a importância do
desenvolvimento de novas técnicas para a produção do vidro.
d) Incorreta. O texto não fala sobre a vida em sociedades antigas, nem a
qualifica.
\end{enumerate}

\section*{Simulado 1}

\begin{enumerate}

\item
SAEB: Localizar informação explícita. BNCC: EF15LP03 – Localizar
informações explícitas em textos.
a) Incorreta. O texto menciona somente o plástico.
b) Incorreta. O texto não trata da praticidade do processo.
c) Correta. Afirma-se explicitamente que o novo procedimento
aumenta a eficiência da reciclagem.
d) Incorreta. O texto parte de um pressuposto oposto.

\item
SAEB: Identificar elementos constitutivos de textos narrativos. BNCC:
EF35LP26 – Ler e compreender, com certa autonomia, narrativas ficcionais
que apresentem cenários e personagens, observando os elementos da
estrutura narrativa: enredo, tempo, espaço, personagens, narrador e a
construção do discurso indireto e discurso direto.
a) Correta. Ao longo do texto encontramos esses três personagens.
b) Incorreta. Falta mencionar o homem misterioro.
c) Incorreta. A vaca não pode ser considerada um personagem.
d) Incorreta. Os feijões não podem ser considerados personagens.

\item
SAEB: Analisar elementos constitutivos de gêneros textuais diversos.
BNCC: EF35LP16 – Identificar e reproduzir, em notícias, manchetes, lides
e corpo de notícias simples para público infantil e cartas de reclamação
(revista infantil), digitais ou impressos, a formatação e diagramação
específica de cada um desses gêneros, inclusive em suas versões orais.
a) Incorreta. Não há imagem associada à notícia.
b) Correta. O primeiro parágrafo do trecho configura o lide, que resume as principais informações da notícia.
c) Incorreta. Há um subtítulo, mas ele reforça uma informação que existe no corpo da notícia.
d) Incorreta. O título da notícia esclarece bem o assunto do texto.

\item
SAEB: Analisar o uso de recursos de persuasão em textos verbais e/ou
multimodais. BNCC: EF05LP20 – Analisar a validade e força de argumentos
em argumentações sobre produtos de mídia para público infantil (filmes,
desenhos animados, HQs, games etc.), com base em conhecimentos sobre os
mesmos.
a) Correta. O texto expõe formas para manutenção da natureza brasileira.
b) Incorreta. O texto não trabalha com a possibilidade de expansão.
c) Incorreta. O texto não procura convencer o leitor dessa característica.
d) Incorreta. O texto apenas constata essa diminuição.

\item
SAEB: Reconhecer diferentes modos de organização composicional de textos em versos.
BNCC: EF35LP27 – Ler e compreender, com certa autonomia,
textos em versos, explorando rimas, sons e jogos de palavras, imagens
poéticas (sentidos figurados) e recursos visuais e sonoros.
a) Incorreta. O terceiro e o quarto verso não rimam entre si.
b) Incorreta. As rimas não são alternadas.
c) Incorreta. O primeiro e o segundo verso rimam entre si.
d) Correta. O primeiro verso rima com o segundo; o terceiro rima com o sexto; o quarto e o quinto rimam entre si.

\item
SAEB: Identificar as variedades linguísticas em textos. BNCC: EF35LP22
- Perceber diálogos em textos narrativos, observando o efeito de sentido
de verbos de enunciação e, se for o caso, o uso de variedades
linguísticas no discurso direto.
a) Incorreta. A palavra “cachola” não tem relação com “cachoeira”, apesar de haver semelhança fonética.
b) Correta. As palavras mencionadas são variações linguísticas de “cabeça” e “enrascada”, respectivamente.
c) Incorreta. As palavras “rascada” e “risco” não têm relação semântica.
d) Incorreta. As palavras mencionadas se assemelham apenas foneticamente às variações no texto.

\item
SAEB: Analisar os efeitos de sentido decorrentes do uso da pontuação.
BNCC: EF05LP04 – Diferenciar, na leitura de textos, vírgula, ponto e
vírgula, dois-pontos e reconhecer, na leitura de textos, o efeito de
sentido que decorre do uso de reticências, aspas, parênteses.
a) Incorreta. O ponto de exclamação é que é usado para destacar ideias.
b) Incorreta. A vírgula é que é usada para dividir duas frases.
c) Incorreta. O ponto final é que é usado para marcar o final de uma frase.
d) Correta. O ponto de interrogação é usado para marcar perguntas.

\item
SAEB: Reconhecer elementos constitutivos das artes visuais, dança,
música e teatro.
BNCC: EF15AR02 – Explorar e reconhecer elementos constitutivos das artes
visuais (ponto, linha, forma, cor, espaço, movimento etc.).
a)  Incorreta. A pintura rupestre apresenta duas dimensões: altura e largura.
b)  Incorreta. O relevo é uma característica de obras tridimensionais.
c)  Incorreta. A arte rupestre é um exemplo de arte bidimensional.
d)  Correta. Por se tratar de uma arte dimensional, apresenta como
  característica a superfície plana.

\item
SAEB: Identificar características do sistema de circulação das artes
visuais, dança, música e teatro em diferentes contextos (teatros,
palcos, museus, galerias, artistas, artesãos, curadores, produtores
etc.).
BNCC: EF15AR01 – Identificar e apreciar formas distintas das artes
visuais tradicionais e contemporâneas, cultivando a percepção, o
imaginário, a capacidade de simbolizar e o repertório imagético.
a)  Incorreta. Na dança, o responsável pela ação é o dançarino.
b)  Incorreta. Na escultura, o responsável pela ação é o artista plástico.
c)  Correta. No teatro, o responsável pela ação é o ator.
d)  Incorreta. No artesanato, o responsável pela ação é o artesão.

\item
SAEB: Reconhecer elementos constitutivos das artes visuais, dança,
música e teatro.
BNCC: EF15AR08 – Experimentar e apreciar formas distintas de
manifestações da dança presentes em diferentes contextos, cultivando a
percepção, o imaginário, a capacidade de simbolizar e o repertório
corporal.
a) Incorreta. Uma partitura não é uma obra de arte visual.
b) Correta. Está representada uma partitura, que é uma forma de notação musical.
c) Incorreta. Não se trata de um texto dramático, inclusive porque não há texto verbal.
d) Incorreta. A partitura é uma linguagem não verbal, mas não contém linguagem verbal.

\item
BNCC: EF05HI07 - Identificar os processos de produção, hierarquização e
difusão dos marcos de memória e discutir a presença e/ou a ausência de
diferentes grupos que compõem a sociedade na nomeação desses marcos de
memória.
a) Incorreta. A bússola nos ajuda na localização geográfica.
b) Correta. A ampulheta serve para medir o tempo.
c) Incorreta. O termômetro serve para medir a temperatura.
d) Incorreta. A luneta serve para observar em grandes distâncias.

\item
BNCC: EF05GE10 - Reconhecer e comparar atributos da qualidade
ambiental e algumas formas de poluição dos cursos de água e dos oceanos
(esgotos, efluentes industriais, marés negras etc.).
a) Incorreta. O desmatamento das florestas pode piorar a poluição nos
rios.
b) Incorreta. O aumento da mineração pode aumentar os níveis de metais
nos rios.
c) Incorreta. A exploração do solo pode agravar a poluição dos rios a
partir do uso de agrotóxicos.
d) Correta. A reciclagem do lixo pode de fato contribuir para diminuir a
poluição dos rios.

\item
BNCC: EF05HI10 - Inventariar os patrimônios materiais e imateriais da
humanidade e analisar mudanças e permanências desses patrimônios ao
longo do tempo.
a) Incorreto. A prática não deve necessariamente ter retorno financeiro
para ser considerada patrimônio.
b) Correto. Como o maracatu representa um aspecto da cultura
afro-brasileira, para ser nomeado Patrimônio Imaterial o proponente deve
também representar a cultura de seu grupo.
c) Incorreto. A prática não precisa ser necessariamente dentro da
família, mas pode ser de um grupo maior de pessoas.
d) Incorreto. Existem práticas muito antigas que podem ser consideradas
Patrimônio Imaterial e devem ser preservadas.

\item
BNCC: EF05HI04 - Associar a noção de cidadania com os
princípios de respeito à diversidade, à pluralidade e aos direitos
humanos.
a) Incorreta. O ensino de história da África não garante a melhoria das
merendas na escola.
b) Correta. A obrigatoriedade do ensino de história da África contribui
para a diversidade do ensino da cultura de múltiplas etnias e raças no
Brasil.
c) Incorreta. Não há impacto sobre as crianças em situação de rua.
d) Incorreta. Já foi instituído o fim da escravidão no Brasil, o impacto
é na memória sobre este fenômeno.

\item
BNCC: EF05GE05 - Identificar e comparar as mudanças dos tipos
de trabalho e desenvolvimento tecnológico na agropecuária, na indústria,
no comércio e nos serviços.
a) Correta. Trabalhar em casa só é possível porque o trabalhador
consegue se comunicar com seus colegas pela tecnologia.
b) Incorreta. O aumento de empregos em indústrias de produção não
influencia sobre o home office.
c) Incorreta. Não é mencionada uma dificuldade em encontrar empregos
presenciais.
d) Incorreta. Não é mencionada a criação de novas tarefas domésticas.
\end{enumerate}

\section*{Simulado 2}

\begin{enumerate}
\item
SAEB: Identificar as marcas de organização de textos dramáticos. BNCC:
EF35LP24 – Identificar funções do texto dramático (escrito para ser
encenado) e sua organização por meio de diálogos entre personagens e
marcadores das falas das personagens e de cena.
a) Incorreta. O trecho faz parte de um diálogo.
b) Incorreta. O trecho é uma fala da personagem.
c) Correta. O trecho contém instruções de cena específicas para o ator.
d) Incorreta. O trecho faz parte da fala de D. Carlota.

\item
SAEB: Analisar os efeitos de sentido de verbos de enunciação. BNCC:
EF05LP10 – Ler e compreender, com autonomia, anedotas, piadas e cartuns,
dentre outros gêneros do campo da vida cotidiana, de acordo com as
convenções do gênero e considerando a situação comunicativa e a
finalidade do texto.
a) Correta. O verbo “perguntar” está relacionado a uma fala do personagem.
b) Incorreta. O verbo “ficar” não é um verbo de enunciação.
c) Incorreta. O verbo “ver” não faz referência a uma fala.
d) Incorreta. O verbo “poder” não remete a um diálogo.

\item
SAEB: Analisar os efeitos de sentido de verbos de enunciação. BNCC:
EF05LP10 – Ler e compreender, com autonomia, anedotas, piadas e cartuns,
dentre outros gêneros do campo da vida cotidiana, de acordo com as
convenções do gênero e considerando a situação comunicativa e a
finalidade do texto.
a) Correta. A enfermeira disse algo ao médico, mas não foi uma pergunta.
b) Incorreta. A enfermeira não fez uma pergunta, mas, de qualquer modo, o médido fez um comentário com ela relacionado ao que ela tinha dito.
c) Incorreta. Não há indícios de que o médico tenha ficado nervoso, pois o verbo “responder” não dá essa indicação.
d) Incorreta. De fato, o médico dirigia-se à enfermeira, fazendo um comentário ao que ela tinha dito, e isso se indica por meio do verbo “responder”.

\item
SAEB: Reconhecer diferentes gêneros textuais. BNCC: EF35LP29 -
Identificar, em narrativas, cenário, personagem central, conflito
gerador, resolução e o ponto de vista com base no qual histórias são
narradas, diferenciando narrativas em primeira e terceira pessoas.
a) Incorreta. Ambos os tipos de texto apresentam personagens.
b) Incorreta. Os dois tipos de texto podem apresentar a mesma temática.
c) Correta. O texto narrativo, ao contrário do dramático, apresenta um narrador.
d) Incorreta. Os registros linguísticos podem ser os mesmos nos dois tipos de texto.

\item
SAEB: Analisar os efeitos de sentido decorrentes do uso dos adjetivos.
Não há correspondência com a BNCC do quinto ano.
a) Incorreta. O adjetivo revela uma visão clara a respeito do assunto.
b) Incorreta. Uma visão neutra a respeito do assunto não utilizaria um adjetivo positivo.
c) Correta. O adjetivo “competitiva” revela um atributo positivo.
d) Incorreta. O adjetivo atribui um sentido oposto.

\item
SAEB: Analisar informações apresentadas em gráficos, infográficos ou
tabelas. BNCC: EF05LP23 – Comparar informações apresentadas em gráficos
ou tabelas.
a) Incorreta. O 1º ano é o segundo colocado na ordem crescente.
b) Correta. O número de matriculados no 2º ano, 17.850, é o menor que aparece na tabela.
c) Incorreta. O 3º ano é o terceiro colocado na ordem crescente.
d) Incorreta. O 4º ano é o segundo colocado na ordem decrescente.

\item
SAEB: Distinguir fatos de opiniões em textos. BNCC: EF05LP16 – Comparar
informações sobre um mesmo fato veiculadas em diferentes mídias e
concluir sobre qual é mais confiável e por quê.
a) Correta. De fato, essa é uma opinião em que acredita o especialista citado no texto.
b) Incorreta. O Ingenuity, de fato, realizou o voo mais longo na superfície de Marte; trata-se de um fato.
c) Incorreta. Segundo o texto, o Ingenuity é mesmo um mini-helicóptero da Nasa.
d) Incorreta. É um fato expresso no texto que o robô Perseverance pousou em Marte no ano de 2021.

\item
SAEB: Identificar as características de instrumentos musicais
variados, bem como o potencial musical do corpo humano.
a) Correta. O afoxé é um instrumento musical classificado como idiofone,
ou seja, que produz som por meio da vibração de seu próprio corpo.
b) Incorreta. Os instrumentos musicais que produzem som por meio da
vibração das cordas são os cordofones, como o violão.
c) Incorreta. Os instrumentos musicais que produzem som por meio de uma
membrana são os membrafones, como o tamborim.
d) Incorreta. Os instrumentos que produzem som por meio da vibração do ar
são os aerofones, como o trombone.

\item
SAEB: Reconhecer a influência de distintas matrizes estéticas e
culturais nas manifestações das artes visuais, dança, música e teatro na
cultura brasileira.
BNCC: EF15AR03 – Reconhecer e analisar a influência de distintas
matrizes estéticas e culturais das artes visuais nas manifestações
artísticas das culturas locais, regionais e nacionais.
a) Incorreta. A máscara dos yakas, apesar de fazer parte do acervo do
Museu Afro-Brasil, apresenta estética de matriz africana, mas faz parte
do contexto sociocultural do negro na África.
b) Correta. O maracatu é uma manifestação cultural afro-brasileira que
apresenta elementos estéticos de matriz africana e que surgiu no estado
de Pernambuco, no século XVIII, no contexto sociocultural do período
colonial brasileiro.
c) Incorreta. O painel de porta do povo iorubá apresenta estética de
matriz africana, mas não faz parte do contexto sociocultural do negro no
Brasil.
d) Incorreta. A escultura yumbe apresenta estética de matriz africana,
mas faz parte do contexto sociocultural do negro na África.

\item
SAEB: Analisar relações entre as partes corporais e seu todo na
estética da dança.
a) Incorreta. Trata-se de um movimento corporal característico da dança do frevo.
b) Incorreta. Trata-se de um movimento corporal característico da dança do tango.
c) Correta. Trata-se de um movimento corporal característico da estética da dança \textit{hip
hop}: pernas para cima, cabeça para baixo, como apoio as mãos, cabeça ou ombros.
d) Incorreta. Trata-se de movimento característico da dança balé.

\item
BNCC: EF05GE12 - Identificar órgãos do poder público e canais de participação
social responsáveis por buscar soluções para a melhoria da qualidade de
vida (em áreas como meio ambiente, mobilidade, moradia e direito à
cidade) e discutir as propostas implementadas por esses órgãos que
afetam a comunidade em que vive.
a) Correta. A medida é importante para avaliar o impacto dos transportes
no ar e no meio ambiente do município.
b) Incorreta. Não há um interesse de diminuição da circulação de
produtos na região.
c) Incorreta. Não existe uma política ambiental sendo avaliada, mas sim
um impacto do uso de transportes.
d) Incorreta. Não há impactos sobre a qualidade da educação nas escolas.

\item
BNCC: EF05HI02 - Identificar os mecanismos de organização do
poder político com vistas à compreensão da ideia de Estado e/ou de
outras formas de ordenação social.
A alternativa correta é a d), cuja ordem corresponde às regiões.

\item
BNCC: EF05HI05 - Associar o conceito de cidadania à conquista
de direitos dos povos e das sociedades, compreendendo-o como conquista
histórica.
a) Incorreta. O texto fala sobre transporte público e não particular.
b) Incorreta. O texto não fala sobre as condições dos motoristas.
c) Correta. O texto mostra que o movimento queria a gratuidade do
transporte público.
d) Incorreta. O texto não fala sobre a qualidade das vias.

\item
BNCC: EF05HI0 - Identificar os processos de
formação das culturas e dos povos, relacionando-os com o espaço
geográfico ocupado.
a) Correta. O texto fala sobre sua boa localização - limites do
município - e sobre o interesse de empresários em seu potencial de
crescimento.
b) Incorreta. Não há menção à educação nem à beleza da paisagem.
c) Incorreta. Não há menção ao turismo ou ao passado do bairro.
d) Incorreta. Não há menção a riquezas minerais nem a um polo
gastronômico.

\item
BNCC: EF05GE05 - Identificar e comparar as mudanças dos tipos de trabalho
e desenvolvimento tecnológico na agropecuária, na indústria, no comércio
e nos serviços.
a) Correta. Precisamos da pecuária para as carnes e da agricultura para
os demais ingredientes.
b) Incorreta. Não precisamos do turismo.
c) Incorreta. Não precisamos da mineração.
d) Incorreta. Não precisamos do turismo.
\end{enumerate}

\section*{Simulado 3}

\begin{enumerate}
\item
a) Incorreta. Apenas com base no título, essa informação não pode ser analisada.
b) Incorreta. No título, não se faz menção a esse aspecto do acontecimento.
c) Correta. Segundo o título, o Acre precisou ser socorrido; portanto fica entendido que o estado estava com dificuldades no momento de publicação da notícia.
d) Incorreta. Está claro, por meio de “mais de” que o valor expresso não é o exato.

\item
a) Correta. Entre parênteses, apresenta-se uma informação acessária relacionada à Estação Espacial Internacional, que ajuda o leitor a reconhecer a sigla pela qual ela é normalmente identificada.
b) Incorreta. O termo “Estação Espacial Internacional”, a que o trecho entre parênteses se relaciona, não é explicado por ele, mas tem revelado um aspecto acessório.
c) Incorreta. A expressão não é traduzida, apesar de aparecer a sigla em inglês.
d) Incorreta. A informação não tinha sido dada até aquele ponto.

\item
a) Incorreta. Não se trata de um advérbio de lugar.
b) Incorreta. Não se trata de um advérbio de negação.
c) Incorreta. Não se trata de um advérbio de modo.
d) Correta. De fato, trata-se de um advérbio de tempo.

\item
a) Incorreta. Não se trata de poemas de amor, mas orações.
b) Incorreta. Não há elementos para se afirmar que se trata de lamentações.
c) Incorreta. O tom da estrofe não é de alegria; portanto não há comemoração.
d) Correta. O eu lírico dirige-se a um “querubim” (provavelmente, sua amada), pedindo-lhe que reze por si.

\item
a) Incorreta. A locução “no entanto” não é consecutiva.
b) Incorreta. A locução “no entanto” não é conclusiva.
c) Correta. A locução “no entanto” é de oposição.
d) Incorreta. A locução “no entanto” não é aditiva.

\item
a) Incorreta. A imagem reforça o texto verbal.
b) Correta. O artifício visual reforça o ponto de vista construído no texto.
c) Incorreta. O cartaz cumpre a função de complementar, e não substituir o texto verbal.
d) Incorreta. O cartaz, como um todo, defende o ponto de vista oposto.

\item
a) Incorreta. O texto não menciona a expansão desses programas.
b) Correta. O texto cita explicitamente esse argumento, e ele é bem forte em relação ao ponto de vista que se quer defender.
c) Incorreta. O texto não trata do processo de fabricação das vacinas.
d) Incorreta. O texto menciona o calendário, mas não afirma que podemos nos vacinar apenas durante sua vigência.

\item
a) Correta. É um patrimônio imaterial, pois diz respeito a práticas e
domínios da vida social dos povos indígenas do Xingu.
b) Incorreta. É um patrimônio imaterial.
c) Incorreta. É transmitido de geração a geração, mas é um patrimônio
imaterial.
d) Incorreta. É um patrimônio cultural imaterial, mas pertence ao
patrimônio de povos indígenas do centro-oeste brasileiro (Mato Grosso do
Sul).

\item
a) Incorreta. A dança é uma linguagem artística representativa do forró,
mas a linguagem das artes visuais não se faz presente no forró.
b) Incorreta. A música é uma linguagem artística presente no forró,
porém a linguagem das artes visuais não se encontra representada nessa
manifestação artística.
c) Correta. A dança e a música são linguagens artísticas presentes no
forró.
d) Incorreta. A música é uma linguagem representativa do forró, mas a
linguagem do teatro não é característica dessa manifestação cultural.

\item
a) Incorreta. O maracá é de matriz indígena.
b) Correta. De fato, o maracá é um instrumento de origem indígena.
c) Incorreta. O maracá não foi trazido para o Brasil pelos escravizados.
d) Incorreta. A origem do maracá é bem conhecida.

\item
a) Incorreta. O texto não fala que os objetos ameaçam o meio ambiente.
b) Incorreta. O texto não trata como lixo, mas como objetos de memória.
c) Correta. Os objetos são tratados como objetos de memória, ou seja,
documentos sobre a história de Hugo.
d) Incorreta. Não é falado sobre o valor material dos objetos.

\item
a) Incorreta. A imagem não mostra o corte de árvores.
b) Incorreta. A imagem não mostra reciclagem de papéis, além dela ser
benéfica ao meio ambiente.
c) Incorreta. A imagem não mostra campos queimados.
d) Correta. A imagem mostra lixos jogados no chão da escola, local
inadequado.

\item
a) Incorreta. A escrita não foi inventada pelos fenícios.
b) Incorreta. A escola não foi inventada pelos fenícios.
c) Correta. O texto fala sobre a utilização pelos gregos do alfabeto
criado pelos fenícios.
d) Incorreta. Não foram os fenícios que criaram o livro.

\item
a) Incorreta. Não é uma organização proibida, mas sim a base da
sociedade.
b) Incorreta. Não é uma lenda, mas a organização principal das aldeias.
c) Correta. É uma junção de vários grupos familiares por diversos
interesses.
d) Incorreta. Não pode ser utilizada para denominar os povos nativos.

\item
a) Incorreta. A secretária não trabalha no ambiente doméstico.
b) Incorreta. A padeira não trabalha no ambiente doméstico, mas na
padaria.
c) Incorreta. A vendedora trabalha na loja e não na casa.
d) Correta. A faxineira é uma trabalhadora doméstica responsável pela
limpeza da casa.
\end{enumerate}

\section*{Simulado 4}

\begin{enumerate}

\item
a) Incorreta. Ambos os textos foram publicados em portais bastante conhecidos e respeitados.
b) Incorreta. Ambos os textos tratam de um assunto sério.
c) Correta. Apresentar a fala de um especialista é uma vantagem importante para um texto desse gênero.
d) Incorreta. A extensão do texto também não implica informações mais confiáveis, necessariamente.

\item
a) Incorreta. O substantivo “mão” não é retomado no texto.
b) Correta. A pronome “ele” retoma a palavra “dono”.
c) Incorreta. A conjunção “mas” não poderia ser retomada pelo pronome “ele”.
d) Incorreta. O verbo “era” não poderia ser retomado pelo pronome “ele”.

\item
a) Correta. O texto cita esse fato como a causa do fenômeno.
b) Incorreta. O texto menciona o fenômeno sendo observado somente no Brasil.
c) Incorreta. O texto apenas afirma que muitas pessoas registraram o fenômeno.
d) Incorreta. O aumento da população de esturjão é responsável pelo nome “superlua de esturjão”, mas não com a ocorrência da superlua.

\item
a) Correta. Esse sufixo é utilizado para indicar algo diminuto.
b) Incorreta. O sufixo \textbf{-ão}, por exemplo, é utilizado para indicar algo grande.
c) Incorreta. Esse sufixo não representa um juízo de valor.
d) Incorreta. Não há um afixo para demonstrar irrelevância.

\item
a) Incorreta. O verbo “vir” não é retomado ao final do diálogo.
b) Incorreta. A palavra “comigo” não reaparece posteriormente.
c) Incorreta. O verbo “arranjar” não é relacionada à palavra mencionada.
d) Correta. A sequência do diálogo permite que o leitor infira o sentido da palavra, relacionado ao sentido de “dinheiro”.

\item
a) Incorreta. O texto apenas descreve uma iniciativa relacionada às abelhas.
b) Incorreta. O texto menciona somente uma startup.
c) Incorreta. O texto não menciona a agressividade das abelhas.
d) Correta. O trecho “uma \textit{startup} {[}...{]} voltada a melhorar a precisão da polinização do café por abelhas” expõe o tema do texto.

\item
a) Incorreta. Podemos inferir que o instituto já existia.
b) Incorreta. Segundo o texto, a pesquisa já existia.
c) Correta. O texto menciona essa informação explicitamente.
d) Correta. Na realidade, afirma-se que os furacões se intensificam com o aquecimento do planeta.

\item
a) Incorreta. O balé clássico é um gênero de dança que possui técnicas e
  vocabulário próprios. Na dança contemporânea, não existem técnicas nem
  vocabulário predefinidos.
b) Correta. A dança contemporânea não se define em movimentos específicos,
  e o bailarino/intérprete tem autonomia para construir sua coreografia.
c) Incorreta. Na dança contemporânea, não há limitações para roupas e
  acessórios, inexistindo um padrão.
d) Incorreta. A postura reta e verticalizada é uma característica do balé
  clássico. A dança contemporânea tem como uma de suas características a
  maior mobilidade da coluna.

\item
a) Incorreta. O pandeiro, apesar de fazer parte dos instrumentos
característicos da roda de capoeira, foi introduzido no Brasil pelos
portugueses.
b) Correta. O berimbau foi introduzido no Brasil pelos escravizados
africanos e é um dos instrumentos característicos da roda de capoeira.
c) Incorreta. O cavaquinho é de origem portuguesa e marca presença em
ritmos como chorinho, samba e pagode.
d) Incorreta. A cuíca é outro instrumento trazido pelos escravizados para o
Brasil, mas é utilizada no samba.

\item
a) Incorreta. O Centro Histórico de Ouro Preto é um conjunto arquitetônico e
urbanístico e, por isso, constitui um exemplo de patrimônio material.
b) Incorreta. As ruínas das missões jesuíticas constituem um
conjunto arquitetônico e, por isso, de natureza material.
c) Incorreta. A Praça São Francisco é um conjunto arquitetônico,
composto de edifícios construídos no período durante o qual as coroas de
Portugal e Espanha estiveram unidas, entre 1580 e 1640; é, portanto, de
natureza material.
d) Correta. As festas são exemplos de patrimônio cultural imaterial.

\item
a) Incorreta. Não é um relógio.
b) Incorreta. Não é uma ampulheta.
c) Incorreta. Não é um diário.
d) Correta. É um calendário.

\item
a) Incorreta. O texto não fala sobre a ocupação de áreas agrícolas, mas
de biomas preservados.
b) Incorreta. O texto não fala que a melhora na vida dos humanos piora a
vida animal.
c) Correta. A expansão urbana invade biomas e áreas preservadas e
destinadas à vida animal.
d) Incorreta. O aumento das cidades não baixa a destruição das
florestas, pelo contrário, aumenta.

\item
a) Incorreta. O texto mostra que existem muitas denúncias.
b) Incorreta. A diversidade ensinada nas escolas melhoraria o problema,
não seria o motivo dele.
c) Incorreta. O texto não fala que essas comunidades são violentas, mas
sim violentadas.
d) Correta. O texto fala que ainda há resquícios de racismo em nossa
sociedade.

\item
a) Incorreta. Um condomínio residencial não é considerado patrimônio
histórico e artístico pelo IPHAN.
b) Incorreta. Uma plantação de arroz não é considerada patrimônio
histórico e artístico pelo IPHAN.
c) Correta. Uma igreja muito antiga pode fazer parte do patrimônio
protegido pelo IPHAN.
d) Incorreta. Uma floresta não é considerada patrimônio histórico e
artístico pelo IPHAN mas sim patrimônio da natureza.

\item
a) Incorreta. O texto não fala sobre o ferimento do direito de proteção
contra a tortura.
b) Incorreta. O texto não fala que as pessoas não têm direito à moradia.
c) Incorreta. O texto não fala que existem pessoas sem nacionalidade no
Catar.
d) Correta. O texto fala que as mulheres são discriminadas e que certas
comunidades não têm direitos iguais a outras.
\end{enumerate}

%\afterpage{\nopagecolor}