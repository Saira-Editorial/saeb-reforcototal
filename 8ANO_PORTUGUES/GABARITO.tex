%!TEX root=./LIVRO.tex
\chapter{Respostas}
\pagestyle{plain}
\footnotesize

\pagecolor{gray!40}

\section*{Língua Portuguesa – Módulo 1 – Treino}

\num{1}

SAEB: Identificar teses, opiniões, posicionamentos explícitos e
argumentos em textos.

BNCC: EF89LP04 -- Identificar e avaliar
teses/opiniões/posicionamentos explícitos e implícitos, argumentos e
contra-argumentos em textos argumentativos do campo (carta de leitor,
comentário, artigo de opinião, resenha crítica etc.), posicionando-se
frente à questão controversa de forma sustentada.

(A) Incorreta. O texto não é do gênero crônica, pois a situação
explicitada não está situada no tempo, embora o texto de fato expresse
uma experiência literária da resenhista. 

(B) Incorreta. O texto não é do
gênero diário, pois seu objetivo não é falar sobre o cotidiano literário
da resenhista nem sobre outro tipo de situação cotidiana. 

(C) Correta. O
texto é do gênero resenha crítica porque apresenta a avaliação e a
opinião da resenhista sobre um livro que ela leu, para indicar a leitura
ao leitor. 

(D) Incorreta. O texto não é do gênero anúncio publicitário,
pois a avaliação que a resenhista trará sobre o livro não tem o objetivo
de vendê-lo como mercadoria para o leitor, e sim dizer-lhe se deve
experimentar lê-lo ou não.

\num{2}

SAEB: Identificar o uso de recursos persuasivos em textos verbais e não
verbais. 

BNCC: EF89LP04 -- Identificar e avaliar
teses/opiniões/posicionamentos explícitos e implícitos, argumentos e
contra-argumentos em textos argumentativos do campo (carta de leitor,
comentário, artigo de opinião, resenha crítica etc.), posicionando-se
frente à questão controversa de forma sustentada.

(A) Incorreta. O trecho expressa um convite para que o leitor continue
acompanhando a resenha para saber se a autora gostou ou não do livro.

(B) Incorreta. O trecho expressa um gosto da resenhista por um gênero
literário específico e não se refere ainda ao livro em questão, que
nesse ponto da resenha não tinha sido citado. 

(C) Incorreta. O trecho
expressa uma hipótese ou dúvida que a resenhista deixa em suspense, como
forma de incentivar o leitor a continuar lendo a resenha para saber a
opinião dela sobre o livro. 

(D) Correta. O trecho expressa o diferencial
do livro em questão, pois a resenhista diz que ele é diferente dos
outros livros do gênero distopia futurista por mostrar um apocalipse
financeiro nos Estados Unidos, em vez do mais comum, que é uma invasão
zumbi ou alienígena, e isso a deixa ainda mais instigada a lê-lo e pode
também incentivar o leitor.

\section*{Língua Portuguesa – Módulo 2 – Treino}

\num{1}

SAEB: Identificar elementos constitutivos de textos pertencentes ao
domínio jornalístico/midiático. 

BNCC: EF08LP01 -- Identificar e comparar
as várias editorias de jornais impressos e digitais e de sites
noticiosos, de forma a refletir sobre os tipos de fato que são
noticiados e comentados, as escolhas sobre o que noticiar e o que não
noticiar e o destaque/enfoque dado e a fidedignidade da informação.

(A) Incorreta. O gênero textual é notícia e, assim, não se encontram no
texto marcas de valoração ou pontos de vista. 

(B) Incorreta. Embora o
texto possa gerar a sensação de alerta na população, por antecipar
grande volume de chuva, a finalidade comunicativa primária de uma
notícia não é alertar. 

(C) Incorreta. O gênero notícia não tem como
característica o didatismo, nem se encontram no texto marcas de
instrução ao leitor. 

(D) Correta. O gênero notícia tem a finalidade de
informar o leitor sobre acontecimentos passados e futuros relevantes
para a sociedade.

\num{2}

SAEB: Analisar efeitos de sentido produzido pelo uso de formas de
apropriação textual (paráfrase, citação etc.). 

BNCC: EF08LP01 --
Identificar e comparar as várias editorias de jornais impressos e
digitais e de \emph{sites} noticiosos, de forma a refletir sobre os
tipos de fato que são noticiados e comentados, as escolhas sobre o que
noticiar e o que não noticiar e o destaque/enfoque dado e a
fidedignidade da informação.

(A) Incorreta. Outros tipos de destaque poderiam ser usados para chamar
a atenção do leitor, tais como negrito, itálico, sublinhado, os quais
não carregariam, entretanto, o mesmo valor do uso das aspas nesse texto.

(B) Incorreta. O trecho entre aspas não traz dados no sentido estrito;
trata-se apenas de parte de um texto maior da Organização Pan-Americana
de Saúde, ao qual o leitor não tem acesso na notícia em questão. 

(C) Incorreta. Do ponto de vista semântico, o trecho entre aspas realmente
significa que o vírus se espalha de uma maneira nova, diferentemente do
modo até então conhecido. Porém, as aspas não foram usadas simplesmente
para informar isso. Seu valor é externo ao texto, pois está relacionado
aos princípios de isenção e objetividade do jornalismo. 

(D) Correta. O
texto jornalístico, em geral, preza pela objetividade e pela isenção.
Uma das formas de atender a esses requisitos é citar entre aspas as
palavras de outrem. No texto em questão, as aspas foram usadas para
evitar assumir a responsabilidade por uma afirmação categórica feita, na
realidade, por uma instituição que tem legitimidade para tal (a
Organização Pan-Americana de Saúde). Isso significa que tal instituição
tem maior credibilidade, de modo que as aspas servem, ao mesmo tempo,
para comprovar que a notícia veicula informações reais já confirmadas
por instituição competente.

\num{3}

SAEB: Identificar elementos constitutivos de gêneros de divulgação
científica.

BNCC: Sem correspondência exata.

(A) Incorreta. O texto não é construído com linguagem técnica e, por
isso, seu público-alvo não são estudiosos nem pessoas com grande
conhecimento científico.

(B) Incorreta. O texto não é construído com linguagem técnica e, por
isso, seu público-alvo não são biólogos, que, para aprenderem conteúdos
de sua área, provavelmente procurarão textos científicos especializados,
normalmente divulgados em forma de artigos científicos, teses,
dissertações.

(C) Correta. O texto tem um tom bastante informal e utiliza vocabulário
e linguagem simples e cotidianos, sem perder seu caráter científico,
pois seu objetivo é fazer divulgação científica correta para o grande
público não especializado.

(D) Incorreta. Não é o objetivo do texto divulgar o ditado popular, que
nesse caso serve apenas para ativar um conhecimento prévio e corriqueiro
do leitor, de modo a estabelecer com ele uma interação e ganhar sua
atenção. Isso se confirma ainda pelo fato de apenas na introdução o
texto citar o ditado e não mais abordar o assunto.

\section*{Língua Portuguesa – Módulo 3 – Treino}

\num{1}

SAEB: Analisar elementos constitutivos de textos pertencentes ao domínio
literário. 

BNCC: EF69LP44 -- Inferir a presença de valores sociais,
culturais e humanos e de diferentes visões de mundo, em textos
literários, reconhecendo nesses textos formas de estabelecer múltiplos
olhares sobre as identidades, sociedades e culturas e considerando a
autoria e o contexto social e histórico de sua produção.

(A) Incorreta. O apelo moral não está presente no poema.

(B) Correta. O apelo sensorial está presente no poema por meio de
referências à sensação térmica no Rio de Janeiro, à temperatura e à
degustação do café, ao paladar (``boca adocicada''), ao sabor doce do
caramelo e à própria boca.

(C) Incorreta. O apelo crítico não está presente no poema.

(D) Incorreta. O apelo humorístico não está presente no poema.

\num{2}

SAEB: Analisar a intertextualidade entre textos literários ou entre
estes e outros textos verbais ou não verbais. 

BNCC: EF89LP32 -- Analisar
os efeitos de sentido decorrentes do uso de mecanismos de
intertextualidade (referências, alusões, retomadas) entre os textos
literários, entre esses textos literários e outras manifestações
artísticas (cinema, teatro, artes visuais e midiáticas, música), quanto
aos temas, personagens, estilos, autores etc., e entre o texto original
e paródias, paráfrases, pastiches, trailer honesto, vídeos-minuto,
vidding, dentre outros.

(A) Correta. O eu lírico exalta o momento em que toma café expresso com
caramelo em uma tarde do Rio de Janeiro, a qual ele considera rara, que,
por isso mesmo, se torna um momento de degustação e deleite. Isso
comprova que o eu lírico valoriza coisas simples como um singelo café
com caramelo.

(B) Incorreta. O poema descreve as sensações de um momento específico, e
não pensamentos.

(C) Incorreta. O poema descreve as sensações de um momento específico, e
não a saúde ou a sabedoria.

(D) Incorreta. O poema descreve as sensações de um momento específico, e
não a busca da felicidade.

\num{3}

SAEB: Inferir a presença de valores sociais, culturais e humanos em
textos literários. 

BNCC: EF69LP44 -- Inferir a presença de valores
sociais, culturais e humanos e de diferentes visões de mundo, em textos
literários, reconhecendo nesses textos formas de estabelecer múltiplos
olhares sobre as identidades, sociedades e culturas e considerando a
autoria e o contexto social e histórico de sua produção.

(A) Correta. A personagem é discriminada por sua origem social, pois ela
é uma \emph{dalit}, um grupo social excluído da sociedade indiana o qual
não participa sequer do sistema de castas, conforme diz a narrativa.

(B) Incorreta. A personagem é discriminada por ser uma \emph{dalit,} um
grupo social excluído da sociedade na Índia. O fato de ela ter uma filha
não é mencionado como causa da discriminação sofrida, mesmo porque não é
possível, no trecho, saber se é mãe solo.

(C) Incorreta. O gênero da personagem, inicialmente, parece ter relação
com a proibição de ir à escola, mas a narrativa logo esclarece que todo
o grupo social dos \emph{dalits}, do qual ela faz parte, vive à margem
da sociedade.

(D) Incorreta. O texto cita uma fala de Gandhi sobre os \emph{dalits}
serem filhos de Deus, mas isso não é mencionado com o objetivo de
situá-los em alguma religião. O texto não cita a religião desse grupo
social.

\section*{Língua Portuguesa – Módulo 4 – Treino}

\num{1}

SAEB: Analisar efeitos de sentido produzido pelo uso de formas de
apropriação textual (paráfrase, citação etc.). 

BNCC: EF69LP43 --
Identificar e utilizar os modos de introdução de outras vozes no texto
-- citação literal e sua formatação e paráfrase --, as pistas
linguísticas responsáveis por introduzir no texto a posição do autor e
dos outros autores citados (``Segundo X; De acordo com Y; De minha/nossa
parte, penso/amos que''\ldots) e os elementos de normatização (tais como
as regras de inclusão e formatação de citações e paráfrases, de
organização de referências bibliográficas) em textos científicos,
desenvolvendo reflexão sobre o modo como a intertextualidade e a
retextualização ocorrem nesses textos.

(A) Incorreta. As vozes dos personagens estão presentes no texto; apenas
não estão marcadas graficamente, por se tratar de discurso indireto
livre. 
(B) Correta. A técnica narrativa empregada é o discurso indireto
livre, caracterizado por não marcar graficamente as vozes dos
personagens, inserindo-as dentro da voz do narrador. 
(C) Incorreta. As
vozes das personagens mantêm sua importância expressiva, embora estejam
inseridas na voz do narrador. 
(D) Incorreta. Os sinais gráficos de
marcação do discurso direto, como travessão e aspas, não são usados no
discurso indireto livre.

\num{2}

SAEB: Analisar efeitos de sentido produzido pelo uso de formas de
apropriação textual (paráfrase, citação etc.). 

BNCC: EF89LP05 --
Analisar o efeito de sentido produzido pelo uso, em textos, de recurso a
formas de apropriação textual (paráfrases, citações, discurso direto,
indireto ou indireto livre).

(A) Incorreta. As angústias do menino e da galinha são claramente
perceptíveis nos trechos que marcam a presença de suas vozes no texto.
Essas angústias são escritas como perguntas retóricas feitas para si
mesmo.

(B) Incorreta. As vozes dos personagens aparecem tal como são ditas ou
pensadas por eles, embora integradas à voz do narrador, e o narrador não
tem o poder de exagerá-las, pois não são reportadas em discurso
indireto.

(C) Correta. O narrador conhece os pensamentos do menino e da galinha
porque suas vozes estão integradas à voz narrativa, pois é como se o
narrador as reproduzisse para o leitor. Entretanto, vale lembrar que as
vozes dos personagens são reproduzidas tal como são ditas ou pensadas
por eles.

(D) Incorreta. O narrador, no discurso indireto livre, reproduz as vozes
do menino e da galinha tal como são ditas ou pensadas, embora não
estejam marcadas graficamente como discurso direto.

\num{3}

SAEB: Analisar os efeitos de sentido decorrentes dos mecanismos de
construção de textos jornalísticos/midiáticos. 

BNCC: EF89LP05 --
Analisar o efeito de sentido produzido pelo uso, em textos, de recurso a
formas de apropriação textual (paráfrases, citações, discurso direto,
indireto ou indireto livre).

(A) Incorreta. A análise da entrevista mostra que o nível de formalidade
entre os participantes foi baixo, o que se percebe pelo emprego de
linguagem informal.

(B) Incorreta. O trecho transcrito da entrevista ocorre entre dois
participantes apenas, o entrevistador e o entrevistado.

(C) Correta. Ao dar voz ao entrevistado, esse gênero textual permite um
contato direto do leitor com as respostas dele. Na entrevista
transcrita, o fã passa a conhecer os detalhes do encontro entre os
artistas Nando Reis e Pitty pelo ponto de vista do próprio cantor.
Informações de bastidores geralmente são dadas pelo próprio artista ou
por pessoas muito próximas a ele, pois estão no universo de sua vida
privada.

(D) Incorreta. A entrevista não traz perguntas que exijam emissão de
opinião; trata-se de assunto do cotidiano dos artistas.

\section*{Língua Portuguesa – Módulo 5 – Treino}

\num{1}

SAEB: Distinguir fatos de opiniões em textos.

(A) Correta. O texto é do gênero artigo científico e, como tal,
privilegia a transmissão de informação com objetividade. Porém, há nele
um traço sutil de opinião. No trecho ``Isso era previsível'' avalia o
fato como uma decorrência óbvia, pois ambos estabelecem entre si uma
relação necessária de causa e consequência. 

(B) Incorreta. O trecho
expressa um fato. 

(C) Incorreta. O trecho expressa um fato. 

(D) Incorreta. O trecho expressa um fato.

\num{2}

Saeb: Inferir informações implícitas em distintos textos.

(A) Incorreta. O narrador busca interagir com o leitor para estabelecer
com ele a correspondência entre suas vivências, de modo que não se sabe
se tal intenção provoca semelhanças ou descompasso entre as
subjetividades de narrador e leitor. 

(B) Incorreta. O narrador expõe
situações de sua vivência para buscar interação com as vivências do
leitor, de modo que as ideias são voltadas para ambas as subjetividades,
do narrador e do leitor. 

(C) Incorreta. O narrador expõe situações de
sua vivência para buscar interação com as vivências do leitor, de modo
que as ideias são voltadas para ambas as subjetividades, do narrador e
do leitor. 

(D) Correta. A conexão expressa no título é buscada quando o
narrador expõe sua subjetividade por meio do relato de vivências e
experiências pessoais, ao mesmo tempo fazendo perguntas retóricas ao
leitor para buscar uma interação entre ambas as vivências, do narrador e
do leitor.

\num{3}

SAEB: Inferir informações implícitas em distintos textos.

(A) Correta. O leitor precisa, antes, compreender que o conceito de
cidadania é amplo e abrange diferentes direitos, dentre os quais o
direito ao voto. Assim, ao citar o exemplo do Brasil, fica claro que
esse direito de cidadania foi conquistado tardiamente no país. 

(B) Incorreta. Pelo contrário, o texto enfatiza que o conceito de cidadania
é amplo e abrange os diferentes direitos a que se pode ter acesso. Os
direitos específicos listados no texto são apenas exemplificações dos
tantos outros abrangidos. 

(C) Incorreta. O texto diz que a cidadania foi
uma conquista árdua da humanidade, porém não é possível deduzir daí que
ela seja garantida em todos os países do mundo, pois um conhecimento
prévio básico nos diz que há países autoritários. 

(D) Incorreta. O texto
nega essa afirmativa ao dizer que o conceito correto de cidadania ainda
é desconhecido, embora o termo seja muito usado no cotidiano.

\section*{Língua Portuguesa – Módulo 6 – Treino}

\num{1}

SAEB: Inferir, em textos multissemiótico, efeitos de humor, ironia e/ou
crítica. 

BNCC: EF69LP05 -- Inferir e justificar, em textos
multissemióticos --- tirinhas, charges, memes, gifs etc. ----, o efeito
de humor, ironia e/ou crítica pelo uso ambíguo de palavras, expressões
ou imagens ambíguas, de clichês, de recursos iconográficos, de pontuação
etc.

(A) Incorreta. O texto não apresenta contradição como forma de gerar
humor. 

(B) Correta. A situação humorística é criada pela brincadeira com
a expressão ``reunião de pais'', com a palavra ``pais'' ora significando
progenitor masculino, ora significando ambos os progenitores, pai e mãe.

(C) Incorreta. A situação narrada não é absurda, mas corriqueira, pois a
reunião de pais faz parte do cotidiano de famílias com filhos em idade
escolar. 

(D) Incorreta. Não há quebra de expectativa no texto.

\num{2}

SAEB: Inferir, em textos multissemiótico, efeitos de humor, ironia e/ou
crítica.

BNCC: EF69LP05 -- Inferir e justificar, em textos
multissemióticos --- tirinhas, charges, memes, gifs etc. ----, o efeito
de humor, ironia e/ou crítica pelo uso ambíguo de palavras, expressões
ou imagens ambíguas, de clichês, de recursos iconográficos, de pontuação
etc.

(A) Correta. O texto ironiza a postura de procrastinação da classe
política, que ano após ano deixa de atuar na resolução dos problemas
causados pelo grande volume de chuvas que já é conhecido e esperado todo
ano. 

(B) Incorreta. A denúncia não se faz presente no texto, pois se
trata de um texto humorístico que se utiliza de outras estratégias
argumentativas mais aptas do que a denúncia para gerar o humor, tais
como a ironia. 

(C) Incorreta. Por ser um texto voltado para o humor, a
objetividade não está presente nele, já que ela pode prejudicar a
expressividade necessária aos efeitos de humor. 

(D) Incorreta. Por ser
um texto voltado para o humor, pressupõe-se o posicionamento crítico
frente ao fato, isto é, a parcialidade é que se faz presente.

\num{3}

Saeb: Inferir, em textos multissemióticos, efeitos de humor, ironia e/ou
crítica. 

BNCC: EF69LP05 -- Inferir e justificar, em textos
multissemióticos --- tirinhas, charges, memes, gifs etc. ---, o efeito
de humor, ironia e/ou crítica pelo uso ambíguo de palavras, expressões
ou imagens ambíguas, de clichês, de recursos iconográficos, de pontuação
etc.


(A) Incorreta. O autor não apresenta passividade frente à clara
inquietude juvenil, pois seu texto apresenta questionamentos que indicam
a permanência da postura inquieta frente ao tema em questão. 

(B)
Incorreta. O autor chama o leitor para um diálogo voltado para a vida
prática, de modo a ter alguém em quem se espelhar, tal como o autor
fazia na juventude. 

(C) Correta. O autor relata que buscava exemplos
práticos para se espelhar e fugia de diálogos com viés acadêmico. 

(D)
Incorreta. Ao referir-se à abordagem acadêmica, o autor está, na
realidade, refutando-a, pois prefere exemplos práticos que sirvam de
espelho.

\section*{Língua Portuguesa – Módulo 7 – Treino}

\num{1}
(A) Incorreta. Não se trata de linguagem neutra, pois a escolha entre os
diferentes termos citados não se refere à abordagem de gênero. 

(B)
Incorreta. O texto não aborda o tema sob o ponto de vista da compaixão.

(C) Correta. O texto traz o tema sob um ponto de vista do problema
social que assola o país. A expressão traz um sentido implícito de
condição temporária que se opõe a ``morador de rua'' ou ``mendigo'',
formas linguísticas categóricas que expressam o sentido de condição
permanente. 

(D) Incorreta. O texto não aborda a solidariedade nem
campanhas de doação.

\num{2}

(A) Incorreta. O texto I foi publicado após a assembleia, não podendo
ser uma divulgação dela. O texto II critica a decisão tomada na
assembleia, e não o evento propriamente dito. 

(B) Correta. O texto I,
sendo uma notícia, informa os leitores de que a greve foi mantida. O
texto II, sendo um texto opinativo, avalia negativamente a decisão de
manter a greve. 

(C) Incorreta. O texto I apenas noticia um fato, e o
texto II de fato faz uma reclamação sobre a greve, por ser opinativo.

(D) Incorreta. O texto I apenas noticia um fato, e o texto II de fato se
solidariza com as pessoas que dependem do transporte público, avaliando
negativamente a greve.

\num{8}

(A) Incorreta. Os dois textos mencionam o ano de 2022 como o período
principal de referência das informações. 

(B) Incorreta. O fato certo
presente em ambos os textos, independentemente do viés, é a ocorrência
do desmatamento. 

(C) Correta. Os dois textos apresentam dados distintos
sobre o desmatamento, e o leitor não consegue, à primeira vista,
identificar qual é mais real, pois provavelmente os critérios por trás
do levantamento desses dados são completamente diferentes entre os
veículos de informação e voltados para o interesse do ponto de vista que
se quer transmitir no texto. 

(D) Incorreta. Ambos os textos expressam
posicionamentos políticos claros. O texto I opõe-se ao governo,
apresentando a notícia de modo desfavorável a ele, enquanto o texto II é
favorável ao governo, buscando um viés que favoreça a imagem do
presidente.

\section*{Língua Portuguesa – Módulo 8 – Treino}

\num{1}
(A) Correta. A notícia privilegia o relato de fatos relevantes do
cotidiano de uma sociedade e, para manter o interesse do leitor ou
ganhar sua atenção, precisa sempre parecer nova e recente, já que os
fatos novos sempre se sobrepõem aos antigos. Para isso, recorre ao
emprego do tempo verbal presente do indicativo. 

(B) Incorreta. A
veracidade dos fatos não depende exclusivamente do tempo verbal
presente, pois há trechos de notícia narrados com o tempo verbal
pretérito que não deixam de ser verdadeiros. 

(C) Incorreta. A
imparcialidade se constrói por meio de outras estratégias, como o uso de
terceira pessoa. 

(D) Incorreta. O gênero notícia busca, geralmente,
abordar os fatos com objetividade, embora possa apresentar traço de
opinião. Entretanto, o tempo verbal presente não é um dos elementos
linguísticos usados para marcar opinião.

\num{2}
(A) Incorreta. A veracidade está presente no texto, mas não se trata de
uma característica influenciada pelas expressões em questão. 

(B)
Correta. A expressão ``expansão urbana'' é atribuída à ONG MapBiomas e
revela uma postura menos avaliativa e mais neutra sobre o crescimento da
cidade de Manaus. Por outro lado, referindo-se ao mesmo fato, a
expressão ``crescimento desordenado'', presente em trechos que
correspondem à voz do veículo de notícias, revela uma apreciação sobre o
crescimento da cidade de Manaus. 

(C) Incorreta. Embora a objetividade
seja uma característica desejável no texto em questão, a expressão
``crescimento desordenado'' revela certo grau de subjetividade por
trazer em si uma postura avaliativa do veículo de notícias em relação ao
fato. 

(D) Incorreta. Embora a imparcialidade seja uma característica
desejável no texto em questão, apenas uma das expressões, o termo
``expansão urbana'', traz um efeito semântico mais imparcial.

\num{3}

(A) Incorreta. A linguagem do texto é comprometida com o jornalismo e
com a postura do veículo de notícias frente ao fato. 

(B) Incorreta. A
linguagem empregada é sempre em terceira pessoa para marcar a
impessoalidade, característicadesejável no texto em questão. 

(C)
Incorreta. A citação do estudo feito pela ONG tem como objetivo aumentar
a credibilidade do texto, e não emitir apreciação. 

(D) Correta. O
veículo de notícia refere-se ao fato noticiado empregando a expressão
``crescimento desordenado'', citando ainda os respectivos prejuízos para
Manaus. O título demonstra esse posicionamento ao empregar o verbo
``sofre''. Por outro lado, a ONG citada emprega um termo mais neutro,
``expansão urbana'', para designar o mesmo fato.

\section*{Língua Portuguesa – Módulo 9 – Treino}

\num{1}

(A) Incorreta. O menino acompanhou o velório da mãe, que foi realizado
na casa da família; portanto ele sabia da morte dela. 

(B) Correta. A
expressão se caracteriza como eufemismo, o que busca amenizar a dor do
menino, embora ele não tivesse a real dimensão do fato. Sendo ele uma
criança, a tia preocupou-se em confortá-lo pela perda da mãe. 

(C)
Incorreta. Por se tratar de uma criança, de modo algum a tia, que lhe
tinha afeto, teria a intenção de aumentar no menino a dor da perda da
mãe. 

(D) Incorreta. A expressão não explica o sentido real, mas dá ao
menino uma percepção que ele conseguiria apreender.

\num{2}

(A) Correta. O emparelhamento de ideias opostas é feito pela figura de
linguagem chamada antítese, que no texto se identifica pelos versos em
que se acham as seguintes ideias: feliz/infeliz, sol/chuva,
felicidade/infelicidade, montanhas/planícies, rochedos/erva. 

(B)
Incorreta. A alternativa se refere ao eufemismo, que não é a principal
figura de linguagem no poema, pois não há recorrência dela. 

(C)
Incorreta. O exagero é obtido por meio da hipérbole, que não se faz
presente no poema. 

(D) Incorreta. Trata-se da ironia, figura de
linguagem que significa o contrário do que se diz, característica que
não se encontra no poema.

\num{3}

(A) Incorreta. O texto não menciona a participação de representante da
OMS no programa. 

(B) Incorreta. O argumento de autoridade empregado ao
citar a OMS pode relacionar-se com a necessidade do combate à obesidade
infantil, mas seu objetivo no texto é justificar a escolha do tema. 

(C)
Incorreta. O texto não visa influenciar a opinião ou formar a opinião do
leitor, mas influenciar sua decisão de acompanhar o programa divulgado.

(D) Correta. A menção à OMS, tendo em vista o objetivo do texto de
convidar o espectador para acompanhar o programa, visa convencer o
público de que o tema é relevante. O texto chega a dizer que a escolha
do tema não foi por acaso, justificando-a com dados estatísticos da OMS
sobre a doença.

\section*{Língua Portuguesa – Módulo 10 – Treino}

\num{1}

(A) Incorreta. O termo ``nenê'' se refere à boneca.

(B) Incorreta. O termo ``ligeira'' descreve a forma como a menina se
antecipou ao castigo da mãe naquela situação específica.

(C) Incorreta. O termo ``boneca'' se refere ao brinquedo propriamente
dito.

(D) Correta. O termo ``pequenita'' retoma o nome Georgeana e, ao mesmo
tempo, qualifica a menina, indicando que era uma criança.

\num{2}

(A) Incorreta. Os termos não apresentam sentidos opostos no texto, e sim
se complementam semanticamente.

(B) Incorreta. Os termos se relacionam semanticamente, mas um não é
conclusão do outro.

(C) Incorreta. Os termos se relacionam semanticamente, mas um não é a
explicação do outro.

(D) Correta. Os termos estabelecem entre si uma relação de coesão e
progressão textual, na medida em que o segundo e o terceiro retomam a
ideia do primeiro, porém especificando-o, numa relação semântica de todo
(recursos naturais) e parte (folhas e frutos).

\num{3}

(A) Incorreta. Brasil e mundo globalizado não são localidades opostas,
pois ambos são impactados pos
(B) Incorreta. Redes sociais e mídias
virtuais são citados como exemplos de tecnologias que promoveram
benefícios para a humanidade e não se opõem; pelo contrário,
complementam-se ao ampliarem o acesso ao conhecimento. 
(C) Correta. Os
avanços tecnológicos são trabalhados no texto sob a ótica da oposição
entre seus benefícios e seus prejuízos. A ampliação do acesso ao
conhecimento é um benefício e a manipulação comportamental é um
prejuízo, e estão em lados opostos. 
(D) Incorreta. A alternativa
apresenta exemplos não opostos, pois ambos são prejuízos da tecnologia,
segundo o texto.

\section*{Língua Portuguesa – Módulo 11 – Treino}

\num{1}

(A) Correta. As duas expressões promovem uma interação informal entre os
participantes da entrevista, pois estão presentes em ambas as falas, do
entrevistado e do entrevistador, de modo a aumentar o engajamento de
ambos no diálogo.

(B) Incorreta. Os termos não exemplificam nenhum tipo de variação
linguística.

(C) Incorreta. Os termos não exemplificam nenhum tipo de variação
linguística.

(D) Incorreta. Os termos não exemplificam nenhum tipo de variação
linguística.

\num{2}

Saeb: Avaliar a adequação das variedades linguísticas em contextos de uso.

BNCC: EF69LP55

(A) Correta. O texto tem a finalidade de interagir com o leitor, o que
se percebe pelas perguntas pessoais que faz para buscar aproximar-se
dele. Para isso, emprega linguagem informal, que se percebe na colocação
pronominal diferente do que prescreve a norma-padrão -- ``me sinto'' --
e no emprego de expressões do dia a dia, como ``bater um papo''.

(B) Incorreta. A linguagem do texto pode ser considerada como um uso
culto da língua, pois o autor é alguém que reside em centro urbano e
teve acesso à escolaridade, o que se percebe por sua boa articulação em
língua escrita. Entretanto, o erro está em dizer que o autor visa exibir
seu domínio da língua para o leitor.

(C) Incorreta. O texto não visa a um público intelectual, e sim aos
diferentes tipos de leitores que podem ter acesso ao texto. Isso fica
claro na crítica feita ao viés acadêmico com que se aborda o tema. A
linguagem empregada no texto busca atingir o maior número de tipos de
leitores possível, daí seu caráter mais acessível.

(D) Incorreta. O texto visa a um público leitor diversificado. Além
disso, a linguagem coloquial não pressupõe que o leitor deva ser menos
escolarizado.

\num{3}

Saeb: Analisar as variedades linguísticas em textos.

BNCC: EF69LP55

(A) Incorreta. O desvio ortográfico presente no texto não é proveniente
da idade da autora, pois não é marca de distinção de uma geração
específica.

(B) Correta. O desvio ortográfico presente no texto provém do nível de
escolaridade da autora, cuja escrita se mostra influenciada pela
oralidade diante de certas relações entre som e letra que são aprendidas
somente com ensino formal. O contexto também permite essa constatação,
pois a autora é moradora da periferia, onde normalmente está a maior
parte da população pouco escolarizada.

(C) Incorreta. O desvio de escrita em questão não é distintivo de
regionalidade.

(D) Incorreta. No texto, predomina a colocação pronominal enclítica
(atender-me/disse-lhe/maltrata-os), uma marca característica da
linguagem formal. Além disso, os desvios ortográficos em questão são
provenientes do baixo domínio das relações representativas entre som e
letra.


%\afterpage{\nopagecolor}