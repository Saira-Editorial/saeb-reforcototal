\section{Simulado 2}

\num{1} Leia o texto.

\textbf{Mudança}

\begin{quote}
Levantei as 6 hóras preparando as roupas, fazendo trouxas para zarpar da
favela. Fiz o café e fui comprar pão, pedi ao Chico para atender-me logo
porque eu ia mudar.

--- Para onde?

--- Vou ressidir em Osasco.

Ele serviu logo, paguei e sai correndo.

Estava preparando os trastes quando chegou o senhor Paulino de Moura
dono da livraria Boulevard. Vêio convidar-me para eu ir na sua livraria
autografar os meus livros.

Eu disse-lhe que irei depois que agêitar a vida dos meus filhos porque,
quando eu os deixo na favela os favelados maltrata-os. {[}...{]}
\end{quote}

\fonte{Carolina Maria de Jesus. \emph{Casa de alvenaria, volume 1}: Osasco.
1.ed.
São Paulo: Companhia das Letras, 2021 (fragmento).}

O texto deve ser considerado como contendo exemplo de

\begin{escolha}
\item variação linguística, devendo ser respeitado como possibilidade de
uso da língua.

\item ignorância em relação à gramática, o que se percebe pela falta de
habilidade na escrita.

\item descuido com a língua, pois a autora não se importou com o público
que leria seu livro.

\item uso incorreto da língua, pois o desrespeito à norma-padrão
comprometeu a coerência.
\end{escolha}

\num{2} Analise o anúncio.

\begin{figure}[H]
\centering
\includegraphics[width=1.81474in,height=2.66429in]{./imgSAEB_8_POR/media/image35.png}
%\caption{https://commons.wikimedia.org/wiki/File:Col\%C3\%ADrio\_Moura\_Brasil\_1944\_propaganda.png}
\end{figure}


Em relação à vantagem de se comprar o produto anunciado, tanto o texto verbal
quanto o não verbal da propaganda enfatizam

\begin{escolha}
\item seu preço acessível.

\item sua marca renomada.

\item sua eficiência no tratamento.

\item sua disponibilidade no mercado.
\end{escolha}

\num{3} Leia o texto.

\begin{quote}
No filme ``Matrix'', clássico do gênero ficção científica, o
protagonista Neo é confrontado pela descoberta de que o mundo em que
vive é, na realidade, uma ilusão construída a fim de manipular o
comportamento dos seres humanos, que, imersos em máquinas que mantêm
seus corpos sob controle, são explorados por um sistema distópico
dominado pela tecnologia. Embora seja uma obra ficcional, o filme
apresenta características que se assemelham ao atual contexto
brasileiro, pois, assim como na obra, os mecanismos tecnológicos têm
contribuído para a alienação dos cidadãos, sujeitando-os aos filtros de
informações impostos pela mídia, o que influencia negativamente seus
padrões de consumo e sua autonomia intelectual.

\fonte{Disponível em: \url{http://portal.mec.gov.br/images/stories/noticias/2019/outubro/24.10.2019redacaolink7.pdf}.
Acesso em: 19 fev. 2023 (fragmento).}
\end{quote}

O texto constrói sua argumentação a partir

\begin{escolha}
\item da negação da influência da tecnologia.

\item da análise do perfil do protagonista Neo.

\item do convite para assistir ao filme ``Matrix''.

\item da comparação entre a ficção e a realidade.
\end{escolha}

Texto para as questões 4 e 5.

\begin{quote}
Devia ser proibido debochar de quem se aventura em língua estrangeira.
Certa manhã, ao deixar o metrô por engano numa estação azul igual à
dela, com um nome semelhante à estação da casa dela, telefonei da rua e
disse: aí estou chegando quase. Desconfiei na mesma hora que tinha
falado besteira, porque a professora me pediu para repetir a sentença.
Aí estou chegando quase... havia provavelmente algum problema com a
palavra quase. Só que, em vez de apontar o erro, ela me fez repeti-lo,
repeti-lo, repeti-lo, depois caiu numa gargalhada que me levou a bater o
fone. Ao me ver à sua porta teve novo acesso, e quanto mais prendia o
riso na boca, mais se sacudia de rir com o corpo inteiro. Disse enfim
ter entendido que eu chegaria pouco a pouco, primeiro o nariz, depois
uma orelha, depois um joelho, e a piada nem tinha essa graça toda.
\end{quote}

\fonte{Chico Buarque de Holanda. \emph{Budapeste.} São Paulo: Companhia das
Letras, 2003 (fragmento).}

\num{4} A expressão ``devia ser proibido'' revela, da parte do
narrador-personagem,

\begin{escolha}
\item uma postura autoritária em relação a uma brincadeira feita pela
professora com o erro cometido.

\item uma indignação com a atitude da professora, de quem esperava um
comportamento diferente.

\item um arrependimento por ter se arriscado a falar uma língua estrangeira
que ainda não dominava.

\item uma decepção perante o erro que cometeu e que realçou seu fraco
domínio da língua estrangeira.
\end{escolha}

\num{5} As duas ocorrências da forma ``dela'' dependem de outro
termo presente no texto para terem seu sentido completo. Esse termo é

\begin{escolha}
\item ``rua''.

\item ``estação''.

\item ``professora''.

\item ``língua estrangeira''.
\end{escolha}

\num{6} Leia o texto.

\begin{quote}
Num passado não muito distante...

A neblina cobria parte dos limites da cidade, enquanto ele caminhava
rumo ao seio urbano. Seus passos firmes amassavam a grama amarelada que
crescia à beira da rodovia. Usava uma capa negra com capuz e sapatos de
couro, ligeiramente sujos de barro avermelhado. {[}...{]}

\fonte{VENÂNCIO, Gabrielle. \emph{Angellore}, a divina conspiração: essência,
volume II. Ribeirão Preto: Selo Jovem, 2016 (fragmento).}
\end{quote}

A apresentação do cenário e do personagem contribui para dar a essa
narrativa uma feição

\begin{escolha}
\item lúdica.

\item realista.

\item fantástica.

\item enigmática.
\end{escolha}

\num{7} Leia o texto.

\begin{quote}
\textbf{Torto arado}

Quando retirei a faca da mala de roupas, embrulhada em um pedaço de
tecido antigo e encardido, com nódoas escuras e um nó no meio, tinha
pouco mais de sete anos. Minha irmã, Belonísia, que estava comigo, era
mais nova um ano. Pouco antes daquele evento estávamos no terreiro da
casa antiga, brincando com bonecas feitas de espigas de milho colhidas
na semana anterior. Aproveitávamos as palhas que já amarelavam para
vestir feito roupas nos sabugos. Falávamos que as bonecas eram nossas
filhas, filhas de Bibiana e Belonísia. Ao percebermos nossa avó se
afastar da casa pela lateral do terreiro, nos olhamos em sinal de que o
terreno estava livre, para em seguida dizer que era hora de descobrir o
que Donana escondia na mala de couro, em meio às roupas surradas com
cheiro de gordura rançosa. Donana notava que crescíamos e, curiosas,
invadíamos seu quarto para perguntar sobre as conversas que escutávamos
e sobre as coisas que nada sabíamos, como os objetos no interior de sua
mala. A todo instante éramos repreendidas por nosso pai ou nossa mãe.
Minha vó, em particular, só precisava nos olhar com firmeza para
sentirmos a pele arrepiar e arder, como se tivéssemos nos aproximado de
uma fogueira. {[}...{]}

\fonte{VIEIRA JUNIOR, Itamar. \emph{Torto arado}. 1.ed. São Paulo: Todavia,
2019 (fragmento).}
\end{quote}

O modo de sequenciação dos acontecimentos pode tornar uma narrativa
bastante interessante para o leitor. No decorrer da leitura desse trecho
de narrativa, o leitor descobre que a primeira frase

\begin{escolha}
\item introduz a situação inicial da história, a qual mais tarde entra em
desequilíbrio.

\item descreve o cenário em que vivem a protagonista e os demais
personagens da história.

\item apresenta a protagonista e o contexto, auxiliando na compreensão das
ações seguintes.

\item antecipa uma ação que, à frente, será compreendida como conflito por
narrar uma travessura da protagonista.
\end{escolha}

Texto para as questões 8 e 9.

\begin{quote}
As primeiras duas décadas do século XXI, no Brasil e no mundo
globalizado, foram marcadas por consideráveis avanços científicos,
dentre os quais destacam-se as tecnologias de informação e comunicação
(TICs). Nesse sentido, tal panorama promoveu a ampliação do acesso ao
conhecimento, por intermédio das redes sociais e mídias virtuais. Em
contrapartida, nota-se que essa realidade impôs novos desafios às
sociedades contemporâneas, como a possibilidade de manipulação
comportamental via dados digitais. Desse modo, torna-se premente
analisar os principais impactos dessa problemática: a perda da autonomia
de pensamento e a sabotagem dos processos políticos democráticos.

\fonte{Disponível em: \url{http://portal.mec.gov.br/images/stories/noticias/2019/outubro/24.10.2019redacaolink6.pdf.}
Acesso em: 21 fev. 2023 (fragmento).}
\end{quote}

\num{8} A ideia que o autor pretende sustentar no texto é a de que

\begin{escolha}
\item a ciência ajudou a construir uma sociedade livre de problemas e
desafios.

\item a tecnologia trouxe vantagens, mas também desvantagens para a
humanidade.

\item os prejuízos trazidos pelos avanços científicos não compensaram seus
benefícios.

\item o acesso ao conhecimento facilitado pela tecnologia foi prejudicial
para a sociedade.
\end{escolha}

\num{9} 
O articulador ``Em contrapartida'' introduz no texto uma

\begin{escolha}
\item mudança de assunto, avançando no tema.

\item reformulação, para retificar uma informação.

\item concordância conclusiva, para encerrar o assunto.

\item avaliação pessimista, oposta ao que foi dito antes.
\end{escolha}

\num{10} Leia o texto.

\begin{quote}
Estava exausto, cansado, mas não conseguia dormir. A adrenalina do show
percorria todo o meu corpo dolorido, ainda sentia as batidas da banda
atrás de mim, o calor do público e o grito das meninas. Observava as
luzes da cidade a perder de vista atrás da janela do hotel enquanto me
secava, tentando me lembrar onde estava. Isto acontecia com frequência
após os shows. A turnê atingia a metade e, depois de duas horas tocando
e cantando, memorizar o nome do lugar em que me apresentei era algo
confuso. Após um tempo respirando profundamente, eu me lembrei que
estava em Curitiba e, no dia seguinte, iria para Porto Alegre. {[}...{]}

\fonte{MAYRINK, Graciela. \emph{O som de um coração vazio}. 1.ed. Rio de
Janeiro: Bambolê, 2018 (fragmento).}
\end{quote}

A narrativa traz, já na introdução, um fato que causa desequilíbrio na
situação inicial apresentada, manifestado

\begin{escolha}
\item na pressão que o personagem sofre para ser um artista perfeito o
tempo todo.

\item na insônia que atrapalha o descanso do personagem após as
apresentações no palco.

\item na insatisfação do personagem com a recepção do público nas
apresentações musicais.

\item nos lapsos de memória após o personagem experimentar momentos de
euforia nos shows.
\end{escolha}

\num{11} Um anúncio de uma campanha de doação de sangue traz o texto reproduzido a seguir.

\begin{quote}
\centering Não é só o sangue de Cristo que tem poder.
\end{quote}


A campanha de doação de sangue remete ao texto bíblico com o objetivo de

\begin{escolha}
\item sensibilizar as pessoas com a lembrança da morte e ressurreição de
Cristo.
\item expressar religiosidade e divulgar a mensagem de Cristo para a
sociedade.
\item valorizar o doador, que com seu sangue pode salvar a vida de outra
pessoa.
\item restringir o público doador, que para doar precisa ser membro de uma
igreja.
\end{escolha}

\num{12} Leia o texto.

\begin{quote}
Biosfera. O conjunto de todos os ecossistemas terrestres é chamado de
biosfera, que significa a camada de vida que envolve a Terra. É a área
do nosso planeta em que é possível a sobrevivência dos organismos vivos,
devido à existência de diversas condições que permitem a sustentação da
vida. Compreende não só a superfície terrestre, mas também uma parte da
atmosfera, do meio aquático e do subsolo. Tem aproximadamente 18
quilômetros, sendo 7 quilômetros para cima da superfície, na atmosfera,
e 11 quilômetros para baixo, até as profundezas marinhas. A biosfera
pode ser considerada o maior dos ecossistemas conhecidos.

\fonte{RIOS, E. P.; THOMPSON, M. \emph{Biomas brasileiros}. São Paulo:
Melhoramentos, 2013 (fragmento).}
\end{quote}

Em razão das características do gênero, emprega-se no texto uma
linguagem que preza

\begin{escolha}
\item pela criatividade e autenticidade, com o objetivo de demarcar a autoria.

\item pelo sentido literal das palavras, de modo a restringir a interpretação
das informações.

\item pela alternância da objetividade e da subjetividade, quando a opinião do
autor aparece.

\item pela sequenciação dos fatos no tempo e no espaço, a qual leva a um
desfecho inevitável.
\end{escolha}


Texto para as questões 13 e 14.

\begin{quote}
Merecer é um verbo que vem sendo muito utilizado ultimamente. Todo mundo
acha que merece uma porção de coisas. Tem gente que compra roupa nova
toda semana. Tem aqueles que adquirem relógios, computadores e outros
bens de consumo disponíveis no mercado.

Conheço mães que presenteiam seus filhos pequenos com brinquedos novos
num curto espaço de tempo, sem qualquer critério. Vejo homens que
costumam trocar de carro todos os anos e assim por diante.

Essas aquisições são feitas sob a desculpa do ``eu mereço''. E cada um
tem seus motivos para justificar esse merecimento. No entanto, muitas
vezes o tempo passa e as pessoas continuam extasiadas com seus pequenos
prazeres cotidianos, sem perceber que se contentam, na verdade, com
muito pouco.

\fonte{DOMINGOS, Reinaldo. \emph{Como reduzir o impulso de comprar}. São Paulo:
DSOP Educação Financeira, 2013 (fragmento).}
\end{quote}

\num{13} No texto, desenvolve-se uma crítica ao comportamento

\begin{escolha}
\item egoísta, representado na fala ``eu mereço''.

\item consumista, justificado pela lei da compensação.

\item materialista, percebido no apego aos bens de consumo.

\item conformista, manifestado no contentamento com o pouco.
\end{escolha}

\num{14} A finalidade comunicativa do texto indica que foi organizado em forma de

\begin{escolha}
\item relato pessoal.

\item nota de repúdio.

\item artigo de opinião.

\item carta de reclamação.
\end{escolha}


\num{15} Leia os textos.

\textbf{TEXTO I}

\begin{quote}
\textbf{Com redução de crimes e outras ocorrências, Governo de Minas
promove o Carnaval mais seguro do Brasil}

\textit{Atuação conjunta das secretarias e planejamento estratégico das
forças de segurança garantiram sucesso da folia no estado}

\fonte{AGÊNCIA MINAS, 23 fev. 2023. Disponível em:
\url{https://www.agenciaminas.mg.gov.br/noticia/com-reducao-de-crimes-e-outras-ocorrencias-governo-de-minas-promove-o-carnaval-mais-seguro-do-brasil.}
Acesso em: 04 mar. 2023 (fragmento).}
\end{quote}


\textbf{TEXTO II}

\begin{quote}
\textbf{Carnaval de Minas Gerais registra queda expressiva de
criminalidade}

\textit{Os dados são em comparação ao último Carnaval, em 2020, e mostram
queda de assaltos, furtos e importunação sexual}

\fonte{CAVALCANTI, M. \emph{Rádio Itatiaia,} 23 fev. 2023. Disponível em:
\url{https://www.itatiaia.com.br/editorias/cidades/2023/02/23/carnaval-de-minas-gerais-registra-queda-expressiva-de-criminalidade-veja-numeros}.
Acesso em: 04 mar. 2023 (fragmento).}
\end{quote}

Considerando-se as características do gênero textual, a diferença entre
os textos I e II se verifica

\begin{escolha}
\item na apresentação das informações fundamentais, tais como ``o quê'',
``quando'', ``quem'' e ``onde''.

\item na especificação dos tipos de crimes que foram cometidos e de
ocorrências que foram registradas.

\item nas marcas de parcialidade e imparcialidade com que se referem à
queda na criminalidade.

\item na veracidade das informações divulgadas, o que levanta dúvidas sobre
a credibilidade de um dos veículos.
\end{escolha}