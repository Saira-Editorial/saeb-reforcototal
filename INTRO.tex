%!TEX root=./LIVRO.tex

\chapter*{Apresentação}

O \textbf{Sistema de Avaliação da Educação Básica (SAEB)} é um conjunto
de avaliações externas em larga escala que permite ao Instituto Nacional
de Estudos e Pesquisas Educacionais Anísio Teixeira (Inep) realizar um
diagnóstico da educação básica brasileira e de fatores que podem
interferir no desempenho dos estudantes.

Por meio de testes e questionários, aplicados a cada dois anos na rede
pública e em uma amostra da rede privada, o SAEB reflete os níveis de
aprendizagem demonstrados pelos estudantes avaliados, explicando esses
resultados a partir de uma série de informações contextuais.

O SAEB permite que as escolas e as redes municipais e estaduais de
ensino avaliem a qualidade da educação oferecida aos estudantes. O
resultado da avaliação é um indicativo da qualidade do ensino brasileiro
e oferece subsídios para elaboração, monitoramento e aprimoramento de
políticas educacionais, sempre com base em evidências.

As médias de desempenho dos estudantes, apuradas no SAEB, juntamente com
as taxas de aprovação, reprovação e abandono, apuradas no Censo Escolar,
compõem o Índice de Desenvolvimento da Educação Básica (Ideb).

Realizado desde 1990, o SAEB passou por uma série de aprimoramentos
teórico-metodológicos ao longo das edições. A edição de 2019 marca o
início de um período de transição entre as matrizes de referência
utilizadas desde 2001 e as novas matrizes elaboradas em conformidade com
a Base Nacional Comum Curricular (BNCC).

\section*{Revisa SAEB: Recomposição de Aprendizagem}

Como o próprio nome diz, a educação básica é aquela em que se promove a
formação mais essencial dos alunos. O Sistema de Avaliação da Educação
Básica (SAEB) ajuda na detecção dos pontos fortes e fracos na formação
dos alunos em estados, em municípios, em escolas, funcionando como um
parâmetro para que problemas sejam solucionados e para que, ano após
ano, essa formação evolua e ajude no crescimento dessas escolas, desses
municípios e desses estados.

A preparação adequada para avaliações em larga escala, como a do SAEB, é
importante para que, no momento da prova, os alunos possam estar atentos
e tranquilos para darem o melhor possível de seu potencial. Assim, um
material didático de apoio que, de fato, promova essa preparação é o
maior dos aliados para professores e gestores. Este material tem
exatamente esta intenção: garantir um melhor aproveitamento de cada
estudante na resolução de atividades, para que consigam resultados
excelentes nessa trajetória de avaliação.

O \textbf{Revisa SAEB} está dividido em 18 volumes, distribuídos, ao
longo do Ensino Fundamental, da seguinte maneira:

\begin{itemize}
\item
  Nos Anos Iniciais, para o \textbf{primeiro}, o \textbf{segundo}, o
  \textbf{terceiro} e o \textbf{quarto ano}, e nos Anos Finais, para o
  \textbf{sexto}, o \textbf{sétimo} e o \textbf{oitavo ano}, existe um
  volume por ano de Língua Portuguesa e existe um volume por ano de
  Matemática.
\item
  O \textbf{quinto ano} apresenta uma estrutura especial. Em um volume,
  aparecem estes componentes: Língua Portuguesa, Arte e Ciências
  Humanas. Em outro volume, aparecem estes componentes: Matemática,
  Educação Física e Ciências da Natureza.
\item
  O \textbf{nono ano} também apresenta uma estrutura especial. São dois
  volumes: um contém Língua Portuguesa, Arte, Língua Inglesa e Ciências
  Humanas; o outro contém Matemática, Educação Física e Ciências da
  Natureza.
\end{itemize}

Cada volume, em cada componente, está dividido em módulos temáticos (de
uma ou duas aulas), e cada módulo conta com a estrutura descrita a
seguir.

\begin{itemize}
\item
  A \textbf{abertura do módulo} apresenta um resumo teórico de
  contextualização, vinculado, principalmente, a habilidades das
  matrizes do SAEB, mas também a algumas habilidades da BNCC que têm
  relação com essas primeiras habilidades.
\item
  Na sequência, abre-se uma seção de \textbf{atividades}, que contém
  exercícios de modelos variados para retomada e fixação
  do conteúdo trazido pelo módulo.
\item
  No fechamento de cada módulo, surge uma seção chamada \textbf{Treino},
  que contém três itens (questões no modelo do Inep, que é o mesmo
  utilizado nos testes cognitivos do SAEB). Cada item checa o
  desenvolvimento de uma habilidade das matrizes do SAEB e, sempre que
  há conexão, está vinculado a uma habilidade da BNCC do ano
  correspondente.
\item
  Os volumes se encerram com quatro \textbf{simulados}, para serem
  aplicados bimestralmente. Ao longo dos quatros simulados, todas as
  habilidades das matrizes do SAEB, em cada ano, são trabalhadas, desde
  que vinculadas a conteúdos previstos pela BNCC para o ano em
  específico. Igualmente compostos de itens, em números variados, os
  simulados também apresentam, sempre que há conexão, habilidades da
  BNCC.
\end{itemize}
