%!TEX root=./LIVRO.tex

\pagebreak

{\hfill\Large\bfseries Respostas}

\footnotesize

\pagecolor{gray!40}

\section*{Matemática — Matemática Módulo — 1 Treino}

\begin{enumerate}


\item Alternativa A: incorreta, pois o aluno pode ter uma mal interpretação e
contar as classes ao invés das ordens.
Alternativa B: incorreta, pois o aluno pode ter uma mal interpretação e
considerar que as ordens são um conjunto de $3$ números após o ponto.
Alternativa C: correta, pois são $4$ ordens ao total.
Alternativa D: incorreta, pois o aluno pode compreender que ordens são a
soma de todos os números descritos.

\item Alternativa A: incorreta, o aluno pode esquecer de somar um ``X''.
Alternativa B: correta, pois essa é a representação em numerais romanos.
Alternativa C: incorreta, o aluno pode compreender que IX é $11$ ao invés
de $9$.
Alternativa D: incorreta, o aluno pode esquecer de somar a idade do
irmão mais novo.

\item Alternativa A: incorreta, pois o aluno pode considerar que a letra ``D''
representa ``Dezena'', logo o valor seria esse.
Alternativa B: correta, pois essa é a representação dos números.
Alternativa C: incorreta, pois o aluno pode confundir e contar um ``I''
a mais e considerar que o valor correto é esse.
Alternativa D: incorreta, pois o aluno pode considerar que a letra D
signifique Duzentos, logo o resultado seria esse.
\end{enumerate}

\section*{Matemática — Matemática Módulo — 2 Treino}

\begin{enumerate}
\item Alternativa A: incorreta, pois o aluno pode realizar a somar a soma ao
invés de calcular o M.\,M.\,C.
Alternativa B: incorreta, pois o aluno pode considerar que o valor do
M.\,M.\,C. em si já é a resposta.
Alternativa C: incorreta, pois o aluno pode confundir M.\,M.\,C. com M.\,D.\,C.
nos cálculos e chegar a esse resultado.
Alternativa D: correta, pois somando o resultado do M.\,M.\,C. com o ano de
1984 obtemos este valor.

\item Alternativa A: incorreta, pois O aluno pode confundir o resultado do
M.\,D.\,C. dos valores como resposta.
Alternativa B: incorreta, pois o aluno pode calcular incorretamente o
M.\,D.\,C. esquecendo do valor $5$ no final, onde o resultado seria esse.
Alternativa C: incorreta, pois o aluno pode esquecer de contar um número
``2'' no cálculo do M.\,D.\,C.
Alternativa D: correta, pois, calculando o M.\,D.\,C., obtemos $24$,
realizando a operação $1080$: $24$ obtemos $45$.

\item Alternativa A: incorreta, pois o aluno pode considerar correta essa
alternativa caso ele considere que se encontram no mesmo dia de todo
mês.
Alternativa B: incorreta, pois o aluno pode chegar a esse resultado se
considerar que outubro tenha $30$ dias.
Alternativa C: correta, pois realizando o M.\,M.\,C., temos $42$ dias. $20$ dias
depois de $20$ de setembro cairá no dia $1$ de novembro, lembrando que
outubro tem $31$ dias.
Alternativa D: incorreta, pois o aluno pode considerar que setembro
tenha $31$ dias.
\end{enumerate}

\section*{Matemática — Módulo 3 — Treino}

\begin{enumerate}
\item Alternativa A: incorreta, pois o aluno provavelmente efetuou a operação
de maneira incorreta.
Alternativa B: correta, pois $2/4 = (2 x 2) / (4 x 2) = 4/8$
$3/8$; logo, Maria comeu mais torta.
Alternativa C: incorreta, pois a operação demonstra que Maria e João
comeram porções diferentes.
Alternativa D: incorreta, pois o aluno deve saber comparar frações
diferentes.

\item Alternativa A: correta, pois: Para comparar frações elas devem possuir os denominadores iguais. Para isso, calculamos o MMC entre $5$, $4$, $3$ e $9$, que são os denominadores das frações sorteadas.
Alternativa B: incorreta, O aluno pode considerar que quanto maior o
denominador, maior o valor fracionário, assim $1/4$ seria uma fração maior
que $2/3$.
Alternativa C: incorreta, o aluno pode se confundir na forma de calcular
o M.\,M.\,C. e colocar erroneamente as frações de forma incorreta.
Alternativa D: incorreta, o aluno pode se confundir e colocar as frações
em forma decrescente ao invés de crescente.

\item Alternativa A: incorreta, pois o aluno pode erroneamente considerar que
ambos os potes de sorvete foram divididos em $3$ partes.
Alternativa B: incorreta, pois o aluno erroneamente pode considerar que
somando as partes de chocolates de ambos os potes sem calcular o M.\,M.\,C.
pode se tornar uma resposta correta.
Alternativa C: correta, pois: o primeiro pote continha $3$ sabores em
iguais quantidades: $1/3$ de chocolate, $1/3$ de baunilha e $1/3$ de morango.
No segundo pote, havia $1/2$ de chocolate e $1/2$ de baunilha. Considerando
os dois potes de sorvete, dividimos os dois potes em partes iguais.
Fazendo então o M.\,M.\,C. de (2,3), obtemos que cada pote foi dividido em $6$
partes iguais. Portanto nos dois potes temos $12$ partes iguais. Sendo que
destas, $5$ partes correspondem ao sabor chocolate.
Alternativa D: incorreta, o aluno pode considerar dividir os potes em $3$
partes iguais e somar sem calcular o M.\,M.\,C., que chegará a esse
resultado erroneamente.
\end{enumerate}

\section*{Matemática — Módulo 4 — Treino}

\begin{enumerate}
\item Alternativa A: incorreta, pois o aluno pode considerar que aumentar $25\%$
signifique aumentar $25$ centavos.
Alternativa B: incorreta, pois o aluno pode calcular erroneamente $1,20$ x
0,025, chegando a esse resultado equivocado.
Alternativa C: incorreta, poiso aluno pode considerar que $25\%$ tenha
relação com o valor $R\$1,25$, pela semelhança.
Alternativa D: correta, pois $R\$1,20 x 0,25 = 0,3$, logo somando $R\$1,20$
+ $R\$0,30$ temos $1,50$
\item Alternativa A: incorreta, pois aluno pode deduzir que ao multiplicar o
valor por $0,68$ que o valor logo terá $68\,\%$ de desconto.
Alternativa B: incorreta, pois o aluno pode deduzir que, ao multiplicar
o valor por $0,68$, os livros terão $6,8\,\%$ devido à semelhança dos termos.
Alternativa C: incorreta, pois o cálculo pode ser feito corretamente mas
a semelhança de $3,2\%$ para $32\%$ pode confundir o aluno na hora de
decisão de assinalar a resposta correta.
Alternativa D: correta, pois, ao multiplicar qualquer valor de livro por
68\%, obtém-se um desconto de $32\%$.
\item Alternativa A: incorreta, pois o aluno pode realizar o cálculo
corretamente, mas confundir os valores próximos de R\$1.555, $00$ com
R\$1.545, $00$ devido à semelhança.
Alternativa B: incorreta, pois o aluno, por meio de dedução, pode
considerar que, somando $3\%$ ao valor inicial e subtraindo $3\%$, o valor
inicial fique inerte.
Alternativa C: Correta. Cálculo do acréscimo: $1500\times 0,03 = 45$; $1.550 + 45 = 1.545$. Cálculo do desconto: $1.545\times 0,03 = 46,35$; $1.545 - 46,35 = 1.498,65$.
Alternativa D: incorreta, pois o aluno, por meio de dedução, pode
considerar que, somando $3\%$ ao valor inicial e subtraindo $3\%$, o valor
inicial fique inerte.
\end{enumerate}

\section*{Matemática — Módulo 5 — Treino}

\begin{enumerate}
\item Alternativa A: incorreta, pois o aluno pode chegar à conclusão de que o
número de bolas vermelhas é $72$, dividindo por $4$ ao tentar encontrar o
número de bolas amarelas.
Alternativa B: incorreta, pois o aluno pode considerar que o enunciado
pede o número de bolas amarelas.
Alternativa C: incorreta, pois o aluno pode realizar a operação $360$:4,
obtendo um resultado incorreto.
Alternativa D: correta, pois, realizando o sistema, temos que: $X + Y = 360$; $X = 4y$. Inserindo o valor de X na primeira equação, temos que: $4y + y = 360$; $5y = 360$; $y = 72$. Realizando $360 - 72 = 288$, temos o valor correto de bolas vermelhas.

\item Alternativa A: incorreta, pois o aluno durante a resolução pode
confundir e ao invés de dividir $8$ por $0,05$, realizar a multiplicação.
Alternativa B: incorreta, pois o aluno pode resolver a equação
erroneamente, calculando $0,05$ : $8$.
Alternativa C: correta, pois, ao substituir t por $8$, temos: 
$$8 = 0,05 . x;$$ 
$$8/0,05 = x;$$ 
$$x = 160.$$
Alternativa D: incorreta, pois o aluno pode erroneamente colocar o valor
8 na incógnita x.

\item Alternativa A: Correta. Considerando que x = quantidade do produto A em gramas; y = quantidade do produto B em gramas. $x + y = 100$ (I); $x·A + y·B = 3,60$ (II). De (I), deduzimos: $y = 100 - x$. Que aplicamos em (II): $x·A + (100-x)·B = 3,60$. Substituindo A e B pelos seus custos em reais: $x·0,03 + (100-x)y·0,05 = 3,60$. Multiplicando toda a equação acima por $100$, a fim de tornar inteiros
seus coeficientes: $x·3 + (100-x)·5 = 360$; $3x + 500 - 5x = 360$; $-2x = 360 - 500$; $-2x = -140$; $x = -140/-2$; $x = 70$ gramas. 
Alternativa B: incorreta, pois o aluno pode simplesmente retirar a
quantidade de gramas do enunciado e considerar como resposta correta.
Alternativa C: incorreta, pois o aluno pode considerar o preço final do
produto como resposta correta.
Alternativa D: incorreta, pois o aluno pode esquecer de dividir a
equação final por $2$, chegando a esse resultado.
\end{enumerate}

\section*{Matemática — Módulo 6 — Treino}

\begin{enumerate}
\item Alternativa A: correta, pois, ao realizar a regra de $3$ simples, obtemos
o valor de $2$ máquinas.
Alternativa B: incorreta, pois o aluno pode esquecer de realizar a
conversão de uma hora para meia hora, chegando a esse resultado
erroneamente.
Alternativa C: incorreta, pois o aluno pode, ao invés de realizar o
cruzamento na regra de três, multiplicar linearmente, chegando a esse
resultado.
Alternativa D: O aluno pode realizar a conversão corretamente, mas errar
o cruzamento no cálculo de regra de três, chegando a esse valor
erroneamente.

\item Alternativa A: incorreta, pois o aluno pode realizar o cruzamento de
dados erroneamente na regra de $3$ e chegar a esse valor.
Alternativa B: correta, pois, realizando a regra de $3$ composta, obtemos
o valor $35$.
Alternativa C: incorreta, pois, ao esquecer que as apostilas não tem
mais $27$ folhas e sim $35$, o aluno chega a esse resultado erroneamente.
Alternativa D: incorreta, pois, caso o aluno esqueça de ler todo o
enunciado, ele não compreenderá que os minutos de funcionamento da
impressora $2$ diminuem, chegando a essa resposta.

\item Alternativa A: incorreta, pois, caso o aluno realize a multiplicação da
regra de $3$ sem cruzamentos, chegará a esse valor.
Alternativa B: incorreta, pois, ao calcular o cruzamento da regra de $3$
erroneamente, o alunochegará a esse valor.
Alternativa C: incorreta, pois, ao confundir o resultado em horas com
minutos, o aluno acabará assinalando essa alternativa erroneamente.
Alternativa D: correta, pois realizando a regra de três composta,
obtemos esse valor.
\end{enumerate}

\section*{Matemática — Módulo 7 — Treino}

\begin{enumerate}
\item Alternativa A: incorreta, pois o aluno pode esquecer que o valor do raio
é a metade do diâmetro, chegando nesse valor.
Alternativa B: incorreta, pois ao confundir a fórmula do perímetro da
circunferência com a fórmula da área da circunferência chegará a esse
valor.
Alternativa C: incorreta, pois, ao realizar uma soma ao invés de uma
multiplicação na fórmula, obterá esse valor.
Alternativa D: correta, pois ao considerar $\pi = 3$, temos que $2.3.6 = 36\,cm$.

\item Alternativa A: correta, pois, ao calcular a fórmula da área do círculo,
temos que A= $3\times 10^2 = 300$m²
Alternativa B: incorreta, pois, ao realizar o cálculo de perímetro da
circunferência, ao invés do cálculo da área chegaremos a esse valor.
Alternativa C: incorreta, pois o aluno pode esquecer de trocar o
diâmetro pelo raio na fórmula e chegará a esse valor.
Alternativa D: incorreta, pois o aluno pode esquecer de verificar que o
valor do enunciado se trata de m² e não cm².

\item Alternativa A: correta, pois $V + F = A + 2$; $8 + 6 = A + 2$; $14 = A + 2$; $A = 14 - 2$; $A = 12\,arestas$.
Alternativa B: incorreta, pois o aluno pode somar todos os números do
poliedro e chegar a essa conclusão equivocada.
Alternativa C: incorreta pois o aluno pode realizar uma subtração ao
invés de utilizar a fórmula.
Alternativa D: incorreta o aluno pode realizar uma multiplicação ao
invés de utilizar a fórmula.
\end{enumerate}

\section*{Matemática — Módulo 8 — Treino}

\begin{enumerate}
\item Alternativa A: incorreta, pois o aluno pode chegar a essa conclusão
somando os valores literais e dividindo em seguida pelos números
restantes.
Alternativa B: incorreta, pois o aluno pode considerar multiplicar os
valores e dividir após o resultado para chegar a esse valor.
Alternativa C: correta, pois, Utilizando os conhecimentos sobre
bissetriz obtemos que as medidas dos Ângulos BÔP e PÔA são iguais, logo $2x+8=3x-10$; $2x-3x= -10 - 8$; $-x = -18$; $x = 18$.
Alternativa D: incorreta, pois o aluno, pela falta de conhecimento sobre
bissetriz, pode relembrar que $45$ seja o valor da bissetriz do ângulo
reto e chegar a essa conclusão mesmo não se tratando de um ângulo reto.

\item Alternativa A: incorreta, pois o aluno errou na soma dos ângulos.
Alternativa B: incorreta, pois o aluno não soube aplicar a soma dos
ângulos internos.
Alternativa C: incorreta, pois o aluno não somou $80$º para encontrar o
resultado correto.
Alternativa D: correta, pois a soma dos ângulos internos de um triângulo
é sempre igual a $180$ graus. Se um dos ângulos mede $90$ graus, a soma dos
outros dois ângulos deve ser igual a $180 - 90 = 90$ graus.

\item Alternativa A: incorreta, pois o aluno pode considerar cortar mais peças
em relação àquilo que o enunciado recomenda.
Alternativa B: incorreta, pois o aluno pode considerar cortar mais peças
em relação àquilo que o enunciado recomenda.
Alternativa C: correta, pois, para cortar apenas $2$ peças de madeira, o
brinquedo deverá ser um triangulo de lado $50\,cm$. Como será um triangulo
equilátero, terá $3$ lados iguais.
Alternativa D: incorreta, pois o aluno pode considerar cortar mais peças
em relação àquilo que o enunciado recomenda.
\end{enumerate}

\section*{Matemática — Módulo 9 — Treino}

\begin{enumerate}
\item Alternativa A: incorreta, pois o aluno pode esquecer de somar um
quadrinho e chegar a esse número.
Alternativa B: correta: pois, é utilizada a rota $20 + 10 + 30 + 20 + 10 = 90$.
Alternativa C: incorreta, pois o aluno pode considerar seguir por uma
rota que não seja a mais vantajosa.
Alternativa D: incorreta, pois o aluno pode considerar seguir por uma
rota que não seja a mais vantajosa.
\item Alternativa A: correta, pois as coordenadas indicam essa localização.
Alternativa B: incorreta, pois o aluno pode se perder na condução do
mapa e chegar a essa conclusão erroneamente.
Alternativa C: incorreta, pois o aluno pode se perder na condução do
mapa e chegar a essa conclusão erroneamente.
Alternativa D: incorreta, pois o aluno pode se perder na condução do
mapa e chegar a essa conclusão erroneamente.
\item Alternativa A: correta, pois essa definição está correta.
Alternativa B: incorreta, pois as primeiras definições estão incorretas.
Alternativa C: incorreta, pois a definição de planta está incorreta.
Alternativa D: incorreta, pois há diferenças entre esses conceitos.
\end{enumerate}

\section*{Matemática — Módulo 10 — Treino}

\begin{enumerate}
\item Altern ativa A: correta, pois $25\,\%$ de $8$ é igual a $2$.
Alternativa B: incorreta, pois o aluno pode contar vagões a mais que não
estão coloridos e chegar a essa conclusão.
Alternativa C: incorreta, pois o aluno pode considerar realizar a conta
sobre o represente de vagões que não estão coloridos.
Alternativa D: incorreta, pois o aluno pode não compreender o conceito e
frações e porcentagem e chegar a esse valor erroneamente.
\item Alternativa A: incorreta, pois o aluno pode realizar o cálculo da
mediana ao invés da media aritmética e chegar a essa conclusão.
Alternativa b: incorreta, pois o aluno pode conspirar que o tempo esteja
sendo modificado por uma P.A, de constante de mesmo valor subindo e
descendo o tempo.
Alternativa C: correta, pois utilizando a média aritmética $18 + 16 + 14$
+ $20 = 68$; 68/4 = $17$.
Alternativa D: incorreta, pois o aluno pode somar todos os termos e
esquecer de dividir chegando a esse valor.
\item Alternativa A: incorreta, pois o aluno que não compreender corretamente
o enunciado pode calcular $2$ notas a menos e chegar a esse resultado.
Alternativa b: correta, pois o resultado final da média é $48$ alunos.
Alternativa c: incorreta, pois o aluno que não compreender corretamente
o enunciado pode calcular $2$ notas a mais e chegar a esse valor.
Alternativa d: incorreta, pois o aluno que não compreender corretamente
o enunciado pode calcular $4$ notas a mais e chegar a esse valor.
erroneamente
\end{enumerate}

\section*{Matemática — Módulo 11 — Treino}

\begin{enumerate}

\item Alternativa A: correta, pois realizamos o cálculo de $36.000$ : $120 = 300$
Alternativa B: incorreta, pois o aluno pode converter erroneamente
toneladas para\,kg e chegar a esse valor.
Alternativa C: incorreta, pois o aluno pode converter erroneamente
toneladas para\,kg e chegar a esse valor.
Alternativa D: incorreta, pois o aluno pode converter erroneamente
toneladas para\,kg e chegar a esse valor.
\item Alternativa A: incorreta, pois o aluno pode realizar a subtração entre
valores e chegar a esse valor.
Alternativa B: incorreta, pois o aluno pode simplesmente dividir o valor
325 por $2$ e chegar a essa conclusão.
Alternativa C: correta, pois a operação chega ao valor $35\,g$.
Alternativa D: incorreta, pois o aluno pode simplesmente ler o enunciado
e colocar esse valor como correto.
\item Alternativa A: correta, pois a operação matemática a partir da fórmula
da área chega ao valor $16cm^2$.
Alternativa B: incorreta, pois o aluno pode considerar que o perímetro e
a área sejam iguais e chegar a essa conclusão.
Alternativa C: incorreta, pois o aluno pode erroneamente dividir o
retângulo em $2$, descobrindo essa área incorretamente.
Alternativa D: incorreta, pois o aluno pode considerar a área final do
retângulo ao invés da área de cada quadrado.
\end{enumerate}

\section*{Matemática — Módulo 12 — Treino}

\begin{enumerate}
\item Alternativa A: correta, pois $4 x 5 = 20$ maneiras.
Alternativa B: incorreta pois, o aluno pode realizar a soma ao invés da
multiplicação.
Alternativa C: incorreta, pois o aluno pode realizar a potenciação ao
invés da multiplicação.
Alternativa D: incorreta, pois o aluno pode realizar a subtração ao
invés da multiplicação.
\item Alternativa A: incorreta, pois o aluno pode realizar a soma ao invés da
multiplicação e chegar a esse resultado.
Alternativa b: incorreta, pois o aluno pode realizar uma potenciação ao
invés da multiplicação. 
Alternativa C: correta, pois são possíveis $36$ combinações.
Alternativa D: incorreta, pois o aluno pode realizar uma soma ao invés
da multiplicação.
\item Alternativa A: incorreta, pois o aluno pode ter considerado apenas dois
números.
Alternativa B: correta, pois há cinco números pares entre um total de
10.
Alternativa C: incorreta, incorreta, pois o aluno pode ter considerado
seis números.
alternativa D: incorreta, pois incorreta, pois o aluno pode ter
considerado oito números.
\end{enumerate}

\section*{Matemática — Simulado 1}

\begin{enumerate}
\item Alternativa A: incorreta, pois, caso o aluno resolva transformar km em
m, esse seria o valor, mas não é isso que o enunciado pede.
Alternativa B: correta, pois O número $2.251$ possui $2$ classes e $4$ ordens.
Alternativa C: incorreta, pois o aluno pode considerar que o numeral
signifique o valor da classe, o que está incorreto.
Alternativa D: incorreta, pois o aluno pode confundir classes com
ordens.
\item Alternativa A: incorreta, pois o aluno pode realizar uma divisão ao
invés da multiplicação.
Alternativa B: incorreta, pois o aluno pode realizar uma soma ao invés
da multiplicação.
Alternativa C: incorreta, pois o aluno pode assinalar essa erroneamente
por conta da semelhança em relação ao resultado correto.
Alternativa D: correta, pois número de quadradinhos é $12$ ·12 = $144$.
\item Alternativa A: incorreta, pois o aluno pode visualizar erroneamente o
número de arvores na figura acima e chegar a essa conclusão
precipitadamente.
Alternativa B: correta, pois,como o tamanho total está dividido em $5$
partes iguais, a fração será $2/5$.
Alternativa C: incorreta, pois o aluno pode simplesmente considerar
todas as àrvores e chegar nesse valor.
Alternativa D: incorreta, pois, ao contar uma árvore a menos, o aluno
chegaria a esse valor incorretamente.
\item Alternativa A: incorreta, pois o aluno pode realizar uma soma ao invés
de calcular a porcentagem e chegar a esse valor.
Alternativa b: incorreta, pois o aluno pode realizar o cálculo de apenas
uma parte do enunciado esquecendo o restante.
Alternativa C: pois, o aluno pode considerar que o mesmo valor dado de
20\% para a primeira parte do enunciado seria o mesmo para a segunda, o
que está equivocado.
Alternativa D: correta, pois $1.000 x 1,2 x 1,2$ = $R\$1.440,00$.
\item Alternativa A: incorreta, pois o aluno pode realizar a subtração ao
invés da divisão.
Alternativa B: correta, pois 4,75· x = $19$; x = $4$ chocolates.
Alternativa C: incorreta, pois o aluno pode realizar a soma ao invés da
divisão.
Alternativa D: incorreta, pois o aluno pode realizar a multiplicação ao
invés da divisão.
\item Alternativa A: incorreta, pois esse seria o valor de apenas $1$ pizza.
Alternativa B: incorreta, pois esse seria o valor de apenas $3$ pizzas.
Alternativa C: incorreta, pois esse seria o valor de apenas $4$ pizzas.
Alternativa D: correta, pois Valor de cada pizza: $R\$135,00/3 = R\$
45,00$, Valor de $8$ pizzas: $8 x 45,00 = R\$380,00$.
\item Alternativa A: correta, pois, pela análise da tabela percebe-se que o
candidato A teve todas as notas acima de $30$ e é o que teve mais notas
iguais. Portanto, o candidato A deverá ser aprovado.
Alternativa B: incorreta, pois o aluno pode observar erroneamente os
dados da tabela e chegar a essa conclusão sem fundamentos.
Alternativa C: incorreta, pois o aluno pode observar erroneamente os
dados da tabela e chegar a essa conclusão sem fundamentos.
Alternativa d: incorreta, pois o aluno pode observar erroneamente os
dados da tabela e chegar a essa conclusão sem fundamentos.
\item Alternativa A: incorreta, pois o aluno pode considerar o valor da turma
a como resposta.
Alternativa B: correta, pois número de alunos do $4$º ano: $32 + 29 + 25 = 86$.
Alternativa C: incorreta, pois o aluno pode considerar o valor da turma
a como resposta.
Alternativa D: incorreta, pois o aluno pode considerar o valor da turma
a como resposta.
\item Alternativa A: incorreta, pois o aluno pode considerar um dia ao invés
das duas incógnitas.
Alternativa B: pois Dias da semana: $7$; Escolha: $2$; Probabilidade: $2/7$.
Alternativa C: incorreta, pois o aluno pode considerar um dia do meio da
semana encontrar erroneamente esse valor.
Alternativa D: incorreta, pois o aluno pode considerar dois dias do meio
da semana apenas e encontrar erroneamente esse valor.
\item Alternativa A: incorreta, poic o aluno pode realizar uma soma ao invés
de uma razão.
Alternativa B: incorreta, pois o aluno pode realizar uma divisão ao
invés de uma razão.
Alternativa C: correta, pois, como cada máquina faz uma peça a cada $6$
horas, ao triplicar a quantidade de peças iremos triplicar o tempo.
Alternativa D: incorreta, pois o aluno pode realizar uma potenciação ao
invés de uma razão.
\item Alternativa A: incorreta, pois o aluno pode realizar uma subtração ao
invés de uma razão.
Alternativa B: correta, pois pela razão entre a distância e tempo
teremos a velocidade.
Alternativa C: incorreta, pois o aluno pode realizar uma soma ao invés
de uma razão.
Alternativa D: incorreta, pois o aluno pode realizar uma potência ao
invés de uma razão.
\item Alternativa A: incorreta, pois o aluno pode compreender erroneamente a
numeração romana e seu sistema consequentemente não tendo um fundamento
básico qualquer valor descrito nas alternativas será um valor possível
para que o aluno seja induzido a qualquer resposta.
Alternativa B: incorreta, pois o aluno pode compreender erroneamente a
numeração romana e seu sistema consequentemente não tendo um fundamento
básico qualquer valor descrito nas alternativas será um valor possível
para que o aluno seja induzido a qualquer resposta.
Alternativa C: correta, pois M = $1000$; CM = $900$; LXXX = $80$; V = $5$.
Alternativa D: incorreta, pois o aluno pode compreender erroneamente a
numeração romana e seu sistema consequentemente não tendo um fundamento
básico qualquer valor descrito nas alternativas será um valor possível
para que o aluno seja induzido a qualquer resposta.
\item Alternativa A: incorreta, pois o aluno pode colocar erroneamente o valor
de ``zeros'' na expressão obtendo um resultado equivocado.
Alternativa b: incorreta, pois o aluno pode colocar erroneamente o valor
de ``zeros'' na expressão obtendo um resultado equivocado.
Alternativa C: correta, pois 1 cm = $10$ mm, então $25$ mm = $25 / 10 = 2,5\,cm$.
Alternativa d: incorreta, pois o aluno pode colocar erroneamente o valor
de ``zeros'' na expressão obtendo um resultado equivocado.
\item Alternativa A: incorreta, pois o aluno não compreendeu o conceito
correto de $1/2$ (meio ou metade), logo, qualquer alternativa passa a ser
válida como reposta.
Alternativa B: incorreta, pois o aluno não compreendeu o conceito
correto de $1/2$ (meio ou metade), logo, qualquer alternativa passa a ser
válida como reposta.
Alternativa C: incorreta, pois o aluno não compreendeu o conceito
correto de $1/2$ (meio ou metade), logo, qualquer alternativa passa a ser
válida como reposta.
Alternativa D: correta, pois a figura representa a fração $1/2$.
\item Alternativa A: incorreta, pois a falta de adaptação a planificações e
vistas aéreas pode levar ao aluno confusão de figuras chegando a
assinalar essa alternativa sem compreender realmente o que a figura
representa.
Alternativa B: incorreta, pois a falta de adaptação a planificações e
vistas aéreas pode levar ao aluno confusão de figuras chegando a
assinalar essa alternativa sem compreender realmente o que a figura
representa.
Alternativa C: correta, pois a vista $1$ está representada na alternativa
A, vista $2$ na alternativa B, vista $3$ na alternativa C e a alternativa D
não representa nenhuma vista, pois tem uma sequência de $4$ blocos.
Alternativa D: incorreta, pois a falta de adaptação a planificações e
vistas aéreas pode levar ao aluno confusão de figuras chegando a
assinalar essa alternativa sem compreender realmente o que a figura
representa.
\end{enumerate}

\section*{Matemática — Simulado 2}

\begin{enumerate}
\item Alternativa A: incorreta, pois a conversão entre os dois sistemas
numéricos não foi realizada corretamente
Alternativa B: incorreta, pois a conversão entre os dois sistemas
numéricos não foi realizada corretamente.
Alternativa C: correta, pois X = $10$; XXIII = $23$; VII = $7$; Idade de Marcos: $10 + 23 + 7 = 40$ anos.
Alternativa D: incorreta, pois a conversão entre os dois sistemas
numéricos não foi realizada corretamente.
\item Alternativa A: incorreta, pois, ao não ter um conhecimento prévio sobre
a realização do M.\,D.\,C. como forma de resposta, qualquer alternativa
passa a ser um resposta viável ao aluno.
Alternativa B: incorreta, pois, ao não ter um conhecimento prévio sobre
a realização do M.\,D.\,C. como forma de resposta, qualquer alternativa
passa a ser um resposta viável ao aluno.
Alternativa C: correta , pois MDC $(240; 576; 1 080) = 30$ famílias.
Alternativa D: incorreta, pois, ao não ter um conhecimento prévio sobre
a realização do M.\,D.\,C. como forma de resposta, qualquer alternativa
passa a ser um resposta viável ao aluno.
\item Alternativa A: correta, pois a dízima periódica em questão corresponde a
essa fração.
Alternativa B: incorreta, pois $6$/9: essa fração pode ser simplificada
dividindo o numerador e o denominador por $3$, resultando em $2/3$.
Portanto, essa é uma resposta equivocada.
Alternativa C: incorreta, pois $7/10$: essa fração não é equivalente a
0,666\ldots, que é maior do que $0,7$. Portanto, essa é uma resposta
incorreta.
Alternativa D: incorreta, pois $3/4$: essa fração não é equivalente a
0,666\ldots, que é menor do que $0,75$. Portanto, essa é uma resposta
incorreta.
\item Alternativa A: incorreta, pois $45$º: essa medida é obtida dividindo a
medida do arco por metade do diâmetro, ou seja, $4$/(8/2) = $1$. Logo, o
ângulo central correspondente seria de $1 x 180º = 180$º.
Alternativa B: correta, pois esse é o resultado da operação.
Alternativa C: incorreta, pois $90$º: essa medida é obtida dividindo a
medida do arco pelo raio, ou seja, $4/4 = 1$. Logo, o ângulo central
correspondente seria de $1 x 180º = 180$º.
Alternativa D: incorreta, pois $120$º: essa medida é obtida dividindo a
medida do arco por dois terços do diâmetro, ou seja, $4/((2/3) x $8$) =
1,5$. Logo, o ângulo central correspondente seria de $1,5 x 180º = 270$º.
\item Alternativa A: correta, pois X = quantidade do produto A em gramas y = quantidade do produto B em
gramas; A = Custo do produto A B = Custo do produto B; x + y = $100$\ldots{} (I) x·A + y·B = $7,20$\ldots{} (II); De (I), deduzimos: y = $100$ \textless{} x; Que aplicamos em (II): x·A + (100-x)·B = $7,20$; Substituindo A e B pelos seus custos em reais: x·0,06 +(100-x)·0,10 = 7,20; Multiplicando toda a equação acima por $100$, a fim de tornar inteiros seus coeficientes: $x·6 + (100-x)·10 = 720$; $6x + 1 000 - 10x = 720 -4x =$; $720 - 1000 -4x = -280 x = 70\,gramas$
Alternativa B: incorreta, pois esse valor não corresponde aos números
encontrados na solução da expressão.
Alternativa C: incorreta, pois esse valor não corresponde aos números
encontrados na solução da expressão.
Alternativa D: incorreta, pois esse valor não corresponde aos números
encontrados na solução da expressão.
\item Alternativa A: incorreta, pois ,ao realizar incorretamente a montagem da
regra de $3$, o aluno chegará a esse valor incorreto.
Alternativa B: incorreta, pois, ao realizar incorretamente a
multiplicação do resultado final por $2$, o aluno chegará a esse resultado
equivocadamente.
Alternativa C: correta.
Alternativa D: incorreta, pois, ao realizar incorretamente a montagem da
regra de $3$, o aluno chegará a esse valor incorreto.
\item Alternativa A: incorreta, pois, ao visualizar $2$ valores a menos do que
realmente está na tabela, o aluno chegará a essa conclusão erroneamente.
Alternativa B: correta, pois entre as notas fornecidas, temos $10$ notas
maiores ou iguais a $7,0$.
Alternativa C: incorreta, pois, ao visualizar $1$ valor a menos do que
realmente está na tabela, o aluno chegará a essa conclusão erroneamente.
Alternativa D: incorreta, pois, ao visualizar $1$ valor a mais do que
realmente está na tabela, o aluno chegará a essa conclusão erroneamente.
\item Alternativa A: incorreta, pois o aluno pode considerar $2$ meses e uma
razão e chegar nessa conclusão.
Alternativa B: incorreta, pois o aluno pode inverter as razões e chegar
nessa conclusão.
Alternativa C: correta, pois $205/210 = 51/52$.
Alternativa D: incorreta, pois o aluno pode dividir as razões e
aproximar os valores para $1$ como resposta.
\item Alternativa A: incorreta, pois a soma dos ângulos internos de um
triângulo é igual a $180$ graus.
Alternativa B: incorreta, pois, embora seja possível ter triângulos com
um ângulo interno igual a $90$ graus (triângulo retângulo), essa não é uma
condição necessária para a existência de um triângulo.
Alternativa C: correta, pois essa é uma regra aplicada aos triângulos.
Alternativa D: incorreta, pois essa alternativa apresenta uma informação
contraditória, pois a hipotenusa é o maior lado em um triângulo
retângulo, portanto não pode ser menor que um dos outros lados. Além
disso, não há relação definida entre a medida do menor lado e a da
hipotenusa.
\item Alternativa A: é incorreta, pois o polígono não possui lados e ângulos
congruentes. Alternativa B: correta, pois essa é uma das características
do polígono. Alternativa C: é incorreta, pois a medida dos lados não
influencia na classificação do polígono como regular ou não regular.
Alternativa D: é incorreta, pois mesmo que os ângulos internos do
polígono sejam congruentes, ainda assim é impossível que seus lados
sejam congruentes.
\item Alternativa A: incorreta, pois a alternativa afirma que existem apenas
dois pares de ângulos correspondentes, o que está errado. Alternativa B:
incorreta, pois a alternativa afirma que existem quatro pares de ângulos
correspondentes, o que corresponde ao número de pares de ângulos
correspondentes formados por duas retas paralelas cortadas por uma
transversal, não por três retas paralelas. Alternativa C: correta, pois,
quando uma transversal corta duas retas paralelas, cada par de ângulos
correspondentes possui medidas iguais. Nesse caso, existem quatro pares
de ângulos correspondentes (ângulos $1$ e $5$, $2$ e $6$, $3$ e $7$, $4$ e $8$). A
alternativa D: incorreta, pois a alternativa afirma que existem oito
pares de ângulos correspondentes, o que está errado.
\item Alternativa A: incorreta, o aluno pode chegar a essa conclusão
esquecendo que que no sistema romano são apenas em casos específicos de
impossibilidade de colocar mais de $3$ símbolos iguais para representar o
mesmo número para inserir um número antes de outro representando uma
subtração momentânea.
Alternativa B: incorreta, o aluno pode chegar a essa conclusão
esquecendo que que no sistema romano são apenas em casos específicos de
impossibilidade de colocar mais de $3$ símbolos iguais para representar o
mesmo número para inserir um número antes de outro representando uma
subtração momentânea.
Alternativa C: correta, pois 40 = XL; 9 = IX; XLIX = $40 + 9 = 49$
Alternativa D: incorreta, pois o aluno pode chegar a essa conclusão
esquecendo que que no sistema romano são apenas em casos específicos de
impossibilidade de colocar mais de $3$ símbolos iguais para representar o
mesmo número para inserir um número antes de outro representando uma
subtração momentânea.
\item Alternativa A: incorreta, pois esta alternativa corresponde ao cálculo
do volume de um cubo cujo lado mede $2,5\,cm$, que não é o caso do prisma
em questão.
Alternativa B: incorreta, esta alternativa corresponde ao cálculo do
volume de um prisma retangular com base de $5\,cm$ x $4\,cm$ e altura de $2\,cm$,
que não é o caso do prisma em questão.
Alternativa C: incorreta, pois esta alternativa corresponde ao cálculo
do volume de um prisma triangular com base de $10\,cm^2$ e altura de $10\,cm$,
que não é o caso do prisma em questão.
Alternativa D: correta, pois a fórmula para o cálculo do volume de um
prisma reto é dada por V = Ab x h, em que Ab é a área da base e h é a
altura. No caso deste prisma, a base é um quadrado, cuja área é dada por
Ab = lado². Portanto, Ab = $5^2 = 25\,cm^2$. Substituindo os valores na
fórmula, temos: V = $25\,cm^2$ x $8\,cm$ = $200\,cm$³.
\item Alternativa A: incorreta, pois a média aritmética simples não é a menor
altura nem a média entre a menor e a maior altura.
Alternativa B: incorreta, pois a soma das alturas é maior que $17,5$ (10 x
1,75), portanto a média aritmética simples não pode ser menor que $1,75$.
Alternativa C: correta, pois A média aritmética simples é obtida pela
soma de todos os valores e divisão pelo número total de elementos.
Portanto, somando as alturas dos $10$ estudantes, temos: $1,45 + 1,50$ +
1,60 + $1,70 + 1,75 + 1,80 + 1,85 + 1,90 + 1,95 + 2,00 = 18,50$ Dividindo
esse valor por $10$, temos a média aritmética simples: $18,50 / 10 = 1,85$.
Alternativa D: incorreta, pois A média aritmética simples é menor que a
maior altura (2,00 m) e maior que a menor altura (1,45 m), portanto a
alternativa d) é uma possibilidade, mas não é a resposta correta, pois a
média aritmética simples não é a média entre a menor e a maior altura.
\item Alternativa A: incorreta, pois provavelmente houve uma confusão com os
valores da base e da altura.
Alternativa B: incorreta, pois o valor encontrado é menor que o produto
dos valores da base e altura.
Alternativa C: correta, pois a área de um retângulo é dada pelo produto
da base pela altura, ou seja: Área = Base x Altura; Substituindo os valores dados na questão, temos: Área = $5\,cm$ x $8\,cm$ Área = $40cm^2$.
Alternativa D: incorreta, pois o valor encontrado é maior que o produto
dos valores da base e altura.
\end{enumerate}

\section*{Matemática — Simulado 3}

\begin{enumerate}
\item Alternativa A: incorreta, pois $0.5$ é um número racional, podendo ser
expresso como a fração $1/2$. Alternativa B: incorreta, pois $2/3$ é um
número racional, já que é uma fração. Alternativa C: incorreta, pois a
raiz quadrada de $25$ é um número inteiro. Alternativa D: correta, pois pi
(π) é um número irracional, já que não pode ser expresso como uma fração
finita ou decimal exata. Ele tem uma representação decimal infinita e
não periódica, o que significa que não há um padrão repetido na
sequência de seus dígitos.
\item Alternativa A: incorreta, pois, por questão de semelhança, o aluno pode
considerar que as $3$ pessoas citadas no enunciado abastecerão sempre no
mesmo dia $28$ de todo mês.
Alternativa B: incorreta, pois o aluno pode realizar uma soma dos dias
ao invés de realizar o M.\,M.\,C.
Alternativa C: correta, pois MMC(5;7,2) = $70$. 
Alternativa D: incorreta, pois o aluno pode confundir M.\,M.\,C. por M.\,D.\,C.
\item Alternativa A: incorreta, pois a resolução da expressão não corresponde
a esse resultado.
Alternativa B: incorreta, pois a resolução da expressão não corresponde
a esse resultado.
Alternativa C: incorreta, pois a resolução da expressão não corresponde
a esse resultado.
Alternativa D: correta pois $3^2$ significa $3$ elevado ao quadrado, ou seja,
3 x $3 = 9$. A raiz quadrada de $16$ é $4$. Assim, a expressão pode ser
reescrita como $9 + 4 = 13$.
\item Alternativa A: correta, pois m número primo é aquele que possui apenas
dois divisores distintos: $1$ e ele mesmo. O número $17$ atende a essa
definição, já que é divisível apenas por $1$ e por $17$.
Alternativa B: incorreta, pois um número composto é aquele que possui
mais de dois divisores distintos.
Alternativa C: incorreta, pois o número $17$ não é múltiplo de $3$, já que
não é divisível por $3$.
Alternativa D: incorreta, pois o número $17$ não é um divisor de $64$, já
que $64$ não é divisível por $17$.
\item Alternativa A, incorreta, pois, ao aplicar a fórmula do perímetro,
chegamos a um resultado diferente.
Alternativa B: incorreta, pois, ao aplicar a fórmula do perímetro,
chegamos a um resultado diferente.
Alternativa C: correta, pois, no caso da cerca retangular descrita na
questão, temos que ``a'' = $10$ metros e ``b'' = $6$ metros. Então, podemos
calcular o perímetro da seguinte maneira: P = $2$a + $2$b P = $2$(10) + $2$(6) P = $20 + 12$ P = $32$.
Alternativa D: incorreta, pois, ao aplicar a fórmula do perímetro,
chegamos a um resultado diferente.
\item Alternativa A: incorreta, pois o cálculo da média não chega a esse
valor.
Alternativa B: correta, pois a média aritmética simples é a soma de
todos os valores dividida pelo número total de valores. No caso das
notas dos alunos, temos:
Alternativa C: incorreta, pois o cálculo da média não chega a esse
valor.
Alternativa D: incorreta, pois o cálculo da média não chega a esse
valor.
\item Alternativa A: correta, pois essa é a ordem correta de uma pesquisa
estatística.
Alternativa B: incorreta, pois o primeiro passo está na posição $2$.
Alternativa C: incorreta, pois o primeiro passo está na posição $2$.
Alternativa D: incorreta, pois o primeiro passo está na posição $3$.
\item Alternativa A: correta, pois uma equação polinomial de 1º grau com duas
variáveis, na forma y = ax + b, representa uma reta no plano cartesiano.
O coeficiente a é a inclinação da reta e o coeficiente b é o intercepto
no eixo y. Assim, podemos associar a equação y = $2$x - $3$ com a reta que
passa pelo ponto (0, -3) e tem inclinação $2$. A inclinação positiva
indica que a reta sobe da esquerda para a direita no plano cartesiano. O
intercepto no eixo y indica que a reta cruza o eixo y no ponto (0, -3
Alternativa B: incorreta, pois alternativa apresenta uma equação com
inclinação negativa.
Alternativa C: incorreta, pois a alternativa apresenta uma equação com
inclinação diferente de $1$.
Alternativa D: incorreta, pois a alternativa apresenta uma equação com
inclinação diferente de $1$.
\item Alternativa A: incorreta, pois a solução da expressão traz outro
resultado.
Alternativa B: incorreta, pois a solução da expressão traz outro
resultado.
Alternativa C: incorreta, pois a solução da expressão traz outro
resultado.
Alternativa D: correta, pois $2$x + $5 - 5 = 11 - 5$ $2$x = $6$ $2$x/2 = $6/2$ x = $3$
\item Alternativa A: incorreta, pois o número de faces está errado.
Alternativa B: correta, pois, no caso de uma pirâmide com base
pentagonal, o polígono da base tem $5$ lados. Cada face da pirâmide é um
triângulo, portanto há $5$ faces triangulares. Há um vértice comum a todas
as faces, mais $5$ vértices correspondentes aos vértices do pentágono da
base, totalizando $6$ vértices. E como cada face tem $3$ arestas e a base
tem $5$, há um total de $10$ arestas na pirâmide.
Alternativa C: incorreta, pois os números de vértices e faces estão
errados.
Alternativa D: incorreta, pois os números de vértices e faces estão
errados.
\item Alternativa A: incorreta, pois a alternativa menciona secante e
tangente.
Alternativa B: correta, pois foram mencionados todos os elementos de uma
circunferência.
Alternativa C: incorreta, pois a alternativa menciona ângulo externo.
Alternativa D: incorreta, pois a alternativa menciona ângulo oposto ao
central.
\item Alternativa A: incorreta, pois o número de lados não é suficiente para
determinar se dois polígonos são semelhantes.
Alternativa B: correta, pois dois polígonos são semelhantes se, ao
sobrepor um sobre o outro, as medidas de seus lados correspondentes são
multiplicadas por uma mesma constante de proporcionalidade.
Alternativa C: incorreta, pois mesmo que dois polígonos tenham áreas
iguais, eles não necessariamente são semelhantes.
Alternativa D: incorreta, pois dois polígonos podem ter os mesmos
comprimentos de lado, mas não serem semelhantes.
\item Alternativa A: correta, pois a reflexão em relação ao eixo y faz com que
o ponto A(2,3) se transforme no ponto A'(-2,3). Em seguida, a translação
de $4$ unidades para a esquerda e $2$ unidades para baixo transforma o ponto
A'(-2,3) no ponto B(-6,1).
Alternativa B: incorreta, pois não leva em consideração a reflexão em
relação ao eixo y.
Alternativa C: incorreta, pois não leva em consideração a reflexão em
relação ao eixo y.
Alternativa D: incorreta, pois não leva em consideração a translação de
4 unidades para a esquerda e $2$ unidades para baixo.
\item Alternativa A: incorreta, pois é apresentado um triângulo com lados
diferentes.
Alternativa B: correta, pois Um triângulo equilátero possui os três
lados com a mesma medida. Logo, se um dos lados mede $6\,cm$, os outros dois
também medem $6\,cm$.
Alternativa C: incorreta, pois é apresentado um triângulo com lados
diferentes.
Alternativa D: incorreta, pois é apresentado um triângulo com lados
diferentes.
\item Alternativa A: incorreta, pois um triângulo escaleno não possui lados
congruentes.
Alternativa B: correta, pois essa é a definição de um triângulo
isósceles.
Alternativa C: incorreta, pois um triângulo equilátero possui todos os
lados congruentes.
Alternativa D: incorreta, pois triângulo retângulo possui um ângulo
interno reto.
\end{enumerate}

\section*{Matemática — Simulado 4}

\begin{enumerate}
\item Alternativa A: incorreta, pois duas retas são paralelas se possuem a
mesma inclinação e intercepto y iguais.
Alternativa B: incorreta, pois duas retas são concorrentes se possuem
inclinações diferentes, e duas retas são paralelas se possuem a mesma
inclinação.
Alternativa C: incorreta, pois duas retas podem ser paralelas sem ter
inclinação igual a zero.
Alternativa D: correta, duas retas são perpendiculares se a inclinação
de uma é a negativa inversa da outra (isto é, seus coeficientes
angulares multiplicados resultam em -1).
\item Alternativa incorreta, pois a escala indica outra relação entre as
unidades no mapa e a distância no mundo real.
Alternativa B: incorreta, pois a escala indica outra relação entre as
unidades no mapa e a distância no mundo real.
Alternativa C: incorreta, pois a escala indica outra relação entre as
unidades no mapa e a distância no mundo real.
Alternativa D: correta, pois a escala do mapa indica que $1\,cm$ no mapa
representa $5$ km na realidade. 
\item Alternativa A: incorreta, pois os gráficos de linhas são utilizados para
representar dados de variáveis contínuas e quantitativas.
Alternativa B: correta, pois o gráfico de setores (também conhecido como
gráfico de pizza) é utilizado para representar dados de uma variável
nominal, ou seja, dados que não têm ordem ou sequência lógica. Este tipo
de gráfico é circular e é dividido em fatias que representam as
diferentes categorias da variável.
Alternativa C: incorreta, pois os gráficos de barras são usados para
representar dados de variáveis discretas ou contínuas.
Alternativa D: incorreta, pois o histograma é utilizado para representar
dados de variáveis quantitativas contínuas, mas não é adequado para
variáveis nominais.
\item Alternativa A: incorreta, pois o texto da alternativa apresenta
definições erradas.
Alternativa B: correta, pois essa alternativa apresenta a definição
precisa dos conceitos.
Alternativa C: incorreta, pois o texto da alternativa apresenta
definições erradas.
Alternativa D: incorreta, pois o texto da alternativa apresenta
definições erradas.
\item Alternativa A: incorreta, pois os valores obtidos após as operações não
correspondem à alternativa.
Alternativa B: correta, pois o ponto médio de um segmento de reta $AB$,
cujas coordenadas dos pontos extremos são: 
$A(x1, y1)$ e $B(x2, y2)$, é dado pelas coordenadas do ponto $M(xm, ym)$, em que 
$xm = (x1 + x2)/2$ e $ym = (y1 + y2)/2$. Substituindo os valores dados na questão, temos: $xm = (1 + $5$)/2 = 3$ e $ym = (2 + $6$)/2 = 4$. Portanto, o ponto médio do segmento de reta AB é $M(3, 4)$. As demais alternativas não correspondem às coordenadas do ponto médio.
Alternativa C: incorreta, pois os valores obtidos após as operações não
correspondem à alternativa.
Alternativa D: incorreta, pois os valores obtidos após as operações não
correspondem à alternativa.
\item Alternativa A: incorreta, pois a definição dessa relação foi descrita
erroneamente.
Alternativa B: incorreta, pois a definição dessa relação foi descrita
erroneamente.
Alternativa C: correta, pois As retas paralelas cortadas por uma
transversal formam oito ângulos, sendo quatro pares de ângulos alternos
internos, quatro pares de ângulos alternos externos e dois pares de
ângulos correspondentes. Quando a transversal corta duas retas
paralelas, a distância da transversal em relação a cada uma das retas
paralelas é constante, portanto a relação entre as medidas dos ângulos
formados é inversamente proporcional à distância da transversal em
relação a cada uma das retas paralelas.
Alternativa D: incorreta, pois a definição dessa relação foi descrita
erroneamente.
\item Alternativa A: incorreta, pois essa é a metade do comprimento do
primeiro terreno, o que não condiz com a proporção estabelecida entre os
terrenos.
Alternativa B: incorreta, pois essa é uma opção que poderia ser
confundida com a resposta correta, já que é um múltiplo do comprimento
do primeiro terreno. No entanto, não corresponde à proporção
estabelecida entre os terrenos.
Alternativa C: correta, pois sabemos que os terrenos são semelhantes,
logo, os lados correspondentes são proporcionais. Como a largura do
segundo terreno é $1,5$ vezes maior do que a largura do primeiro (30 m/20
m), podemos afirmar que o comprimento do segundo terreno é $1,5$ vezes
maior do que o comprimento do primeiro terreno.
Alternativa D: incorreta, pois essa é uma opção que poderia ser
confundida com a resposta correta, já que é um múltiplo da largura do
segundo terreno. No entanto, não corresponde à proporção estabelecida
entre os terrenos.
\item Alternativa A: incorreta, pois $2\,cm$ é a medida da altura relativa ao
lado BC.
Alternativa B: incorreta, pois $3\,cm$ é a medida da altura relativa ao
lado AC.
Alternativa C: correta, pois a altura relativa ao lado AB divide o
triângulo ABC em dois triângulos retângulos, onde a altura é a
hipotenusa e os catetos são os segmentos AH e BH. Podemos utilizar o
teorema de Pitágoras para calcular a medida da altura e chegar ao
resultado.
Alternativa D: incorreta, pois $5\,cm$ é a medida da mediana relativa ao
lado AB.
\item Alternativa A: correta, pois pelo Teorema de Pitágoras, a soma dos
quadrados das medidas dos catetos é igual ao quadrado da medida da
hipotenusa. Assim, temos: $hipotenusa^2 = cateto1^2 + cateto2^2 hipotenusa^2$; $= 3^2 + 4^2 hipotenusa^2 = 9 + 16 hipotenusa^2 =$; $25 hipotenusa = 5$.
Alternativa B: incorreta, pois esta alternativa apresenta uma medida que
não é condizente com a aplicação do Teorema de Pitágoras.
Alternativa C: incorreta, pois esta alternativa apresenta uma medida que
é maior do que a hipotenusa de um triângulo retângulo formado por
catetos medindo $3\,cm$ e $4\,cm$.
Alternativa D: incorreta, pois esta alternativa apresenta uma medida que
é maior do que a hipotenusa de um triângulo retângulo formado por
catetos medindo $3\,cm$ e $4\,cm$.
\item Alternativa A: correta, pois essa é a representação correta do número.
Alternativa B: incorreta, pois esse é o número $24$.
Alternativa C: incorreta, pois esse é o número $49$.
Alternativa D: incorreta, pois esse é o número $38$.
\item 
Alternativa A: incorreta, pois, ao compreender erroneamente a quantidade
de zeros a ser colocada na expressão, o aluno chegará a esse resultado.
Alternativa B: incorreta, pois, ao compreender erroneamente a quantidade
de zeros a ser colocada na expressão, o aluno chegará a esse resultado.
Alternativa C: correta, pois cada metro tem $100\,cm$, desta forma $105$
metros representam $105$·100 cm = $10.500\,cm$.
Alternativa D: incorreta, pois, ao compreender erroneamente a quantidade
de zeros a ser colocada na expressão, o aluno chegará a esse resultado.
\item Alterntiva A: incorreta, pois pois é igual ao resultado da divisão dos
coeficientes, sem levar em conta a divisão dos expoentes.
Alternativa B: correta, pois, para resolver a operação, é necessário
multiplicar os números antes da divisão e depois dividir os resultados,
obtendo: $(4,5 x 10^2) ÷ (1,5 x 10^2) = (4,5 ÷ 1,5) x (10² ÷ 10^2) = 3 x 1 = 3$.
Alternativa C: incorreta, pois é igual à multiplicação dos coeficientes
e dos expoentes, sem levar em conta a divisão dos coeficientes.
Alternativa D: incorreta, pois é igual à multiplicação dos coeficientes
e dos expoentes, sem levar em conta a divisão dos expoentes.
\item Alternativa A: incorreta, pois não leva em conta a relação de
proporcionalidade direta entre a quantidade de bolachas produzidas e o
número de funcionários trabalhando na linha de produção.
Alternativa B: incorreta, pois não leva em conta a relação de
proporcionalidade direta entre a quantidade de bolachas produzidas e o
número de funcionários trabalhando na linha de produção.
Alternativa C: incorreta, pois não leva em conta a relação de
proporcionalidade direta entre a quantidade de bolachas produzidas e o
número de funcionários trabalhando na linha de produção.
Alternativa D: correta, pois para $8$ funcionários em $6$ horas, temos: $8$
funcionários x $6$ horas x $50$ bolachas por funcionário por hora = $2.400$
bolachas.
\item Alternativa A: incorreta, pois ao resolvermos o sistema, encontramos
resultados diferentes.
Alternativa B: incorreta, pois ao resolvermos o sistema, encontramos
resultados diferentes.
Alternativa C: correta, pois, para resolver esse problema, podemos
utilizar um sistema de equações de 1º grau com duas incógnitas. Seja x o
valor de um teclado e y o valor de um mouse, temos: $2x + 3y = 160$ (equação $1$) e $3x + 2y = 170$ (equação $2$). Subtraindo a equação $1$ da equação $2$, obtemos: $5x = 190$, $x = 38$. Substituindo o valor de x na equação $1$ ou na equação $2$, obtemos: $2.38 + 3y = 160$, $76 + 3y = 160$, $3y = 84, y = 28$. 
Alternativa D: incorreta, pois ao resolvermos o sistema, encontramos
resultados diferentes.
\item Alternativa A: incorreta, pois essa fração representa a probabilidade de
uma bola ser vermelha (3/12), não azul.
Alternativa B: correta, pois Para calcular a probabilidade de uma bola
ser azul, precisamos dividir o número de bolas azuis pelo número total
de bolas no saco: Probabilidade = número de bolas azuis / número total
de bolas Número total de bolas = $4 + 3 + 5 = 12$ Número de bolas azuis =
4 Probabilidade = $4/12 = 1/3$
Alternativa C: incorreta, pois essa fração representa o número de bolas
azuis, não a probabilidade.
Alternativa D: incorreta, pois essa fração não é encontrada a partir do
sistema.

\end{enumerate}