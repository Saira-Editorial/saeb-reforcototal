%!TEX root=./LIVRO.tex
\chapter{Argumentar e convencer}
\markboth{Módulo 1}{}

\section*{Habilidades do SAEB}

\begin{itemize}

  \item Identificar o uso de recursos persuasivos em textos verbais e não verbais.

  \item Identificar teses, opiniões, posicionamentos explícitos e argumentos em textos.

\end{itemize}

\subsection{Habilidades da BNCC} 

\begin{itemize}

  \item EF67LP05, EF67LP07.

\end{itemize}

\conteudo{Você já deve ter percebido que, o tempo todo, estamos expostos a
textos que pretendem nos convencer de alguma coisa --- da qualidade de um
produto, da verdade de uma informação ou da necessidade de uma conduta.

Para que seja convincente, o texto deve ter algumas características
específicas. Em primeiro lugar, a linguagem deve ser adequada ao público a que
se destina a informação. Não vale, por exemplo, escrever difícil, com
linguagem muito técnica ou vocabulário desconhecido, para crianças ou adultos
que não tiveram a chance de estudar. E o contrário também é verdadeiro: um
texto muito simples não é eficiente para convencer um professor universitário,
que é especializado em assuntos complexos.

Outra forma de convencer o leitor é usar recursos não verbais: as imagens
podem ajudar muito a ilustrar o que as palavras querem dizer. Existe até um
ditado que diz que ``uma imagem vale mais do que mil palavras''. E é verdade
mesmo: uma foto ou um gráfico podem ajudar muito a explicar e simplificar o
sentido de longas frases e parágrafos.

Detalhes como esses são importantes tanto na hora de escrever quanto na hora
de ler um texto argumentativo. É por meio de textos assim, repletos de dados e
estatísticas --- e principalmente de relações lógicas estabelecidas entre
essas informações --- que podemos formar nossas próprias opiniões e planejar
nossas ações no mundo.

Cada vez mais recebemos uma quantidade enorme de informações, de forma
imediata, por meio de aplicativos, da imprensa, de diferentes canais de
notícias e das redes sociais. Precisamos estar atentos a tais recursos e à
forma como somos influenciados por eles. É fundamental saber que \textbf{toda
comunicação parte de uma intenção}, mesmo que isso não apareça explicitamente
no texto. Com atenção, podemos reconhecer os recursos persuasivos em campanhas
publicitárias, debates políticos, discursos de vendas, em diversos tipos de
texto.}

%Paulo: este autor fez, ao final do conteúdo, extensas orientações ao professor. Coloquei-as todas em \coment{}
% \coment{Professor(a), relembre os principais gêneros textuais que utilizam
% recursos verbais e não verbais, questione gêneros conhecidos pelos alunos.
% Estimule os estudantes a pensar sobre os argumentos, mostre como podem servir
% para a formação de opinião no caso de textos jornalísticos, ou para
% convencimento, nos casos de textos publicitários. Relembre também a
% necessidade do rigor técnico no caso de textos de divulgação científica.

% Utilize exemplos distintos para explicar as funções dos recursos verbais e não
% verbais e estimule os estudantes na identificação das associações que podem
% ser feitas entre as imagens e as palavras. Chame a atenção para a citação dos
% órgãos oficiais e ressalte que esse tipo de referência pode tornar um texto
% mais ou menos confiável por conta da possibilidade de confrontar e questionar
% os dados apontados. Relembre os estudantes sobre o uso de recursos persuasivos
% em demais gêneros textuais, apontando também para suas especificidades quanto
% ao público alvo, função comunicativa e meios de veiculação.}

\section*{Atividades}

% \num{1} De acordo com seus conhecimentos, escreva dois tipos de recursos persuasivos presentes em campanhas publicitárias:

% \reduline{1. Linguagem objetiva, clara e direta; 2. recursos verbais e não verbais;
% 3. escolha de fontes adequadas; 4. imagens que chamem a atenção do público alvo, entre outras.\hfill}

% \num{2} Sobre os textos de divulgação científica assinale (V) para as afirmações verdadeiras e (F) para as afirmações falsas:

% \begin{table}[h!]
% \begin{tabular}{|c|l|}
% \hline
% \textbf{\begin{tabular}[c]{@{}c@{}}Verdadeiro (V)\\ ou \\ Falso (F)?\end{tabular}} & \multicolumn{1}{c|}{\textbf{Afirmação}} \\ \hline
% \rosa{F} & Apresentam linguagem de difícil compreensão \\ \hline
% \rosa{F} & Utilizam apenas recursos não verbais \\ \hline
% \rosa{V} & Pretendem convencer \\ \hline
% \rosa{V} & Apresentam linguagem objetiva \\ \hline
% \end{tabular}
% \end{table}

%\coment{F, F, V, V} Deixei o gabarito aqui caso minha tabela não funcione. (Rogério, 25/5/23, 15h15)

% Textos de divulgação científica apresentam linguagem clara e objetiva
% para divulgar conhecimentos científicos de forma acessível para o
% público em geral e neles podem-se utilizar recursos não verbais tais como gráficos
% e tabelas para apresentar os resultados descritos.

Leia a sentença a seguir e responda às questões 1 e 2. 

\begin{myquote}

\begin{minipage}{0.7\textwidth}
A dieta vegetariana é considerada uma opção muito mais saudável devido a uma
série de benefícios comprovados. Primeiramente, ao abandonar o consumo de
carne de animais, é possível reduzir significativamente o risco de colesterol
alto. Além disso, quando nos tornamos vegetarianos, conseguimos prevenir
certos tipos de câncer. Por exemplo, alimentos como brócolis, couve e
couve-flor contêm compostos fitoquímicos, como sulforafano, que demonstraram
ter propriedades anticancerígenas.
\end{minipage}
\hfill
\begin{minipage}{0.4\textwidth}
  \centering
  \includegraphics[width=\textwidth]{./imgSAEB_7_POR/media/image25.png}
  %\caption{Legenda da imagem}
  %\label{fig:exemplo}
\end{minipage}

% \begin{figure}[H]
% \centering
% \includegraphics[scale=0.35]{./imgSAEB_7_POR/media/image25.png}
% \end{figure}
%Fonte: freepik https://br.freepik.com/fotos-gratis/prato-de-tigela-de-buda-com-legumes-e-legumes-vista-do-topo_13807905.htm

\fonte{Texto formulado para este material.}

\end{myquote}

\num{1} Qual a afirmação principal do texto?

\reduline{A afirmação principal é a de que a dieta vegetariana é mais saudável.\hfill}

\num{2} Quais argumentos sustentam a afirmação principal?

\reduline{Os argumentos que baseiam a afirmação principal são as afirmações de que  
a dieta vegetariana reduz o risco de colesterol alto e previne contra o câncer.\hfill}
 
\pagebreak

Leia a notícia abaixo e responda às questões de 3 a 8.

\begin{myquote}

\textbf{No Brasil, mulher é vítima de violência a cada quatro horas} \\

Um estudo de dados de sete estados brasileiros revela que, no ano de
2022, uma mulher tornou-se vítima de violência a cada quatro horas. Trata-se de 
um total de 2.423 casos de violência, dos quais 495 resultaram em morte.

É no estado de Pernambuco que ocorre o maior número de transfeminicídios ---
posição que estava nas estatísticas do Ceará nos últimos dois anos. Segundo
as conclusões da pesquisadora da Rede em Pernambuco, Dália Celeste, essa 
situação está diretamente relacionada à negligência por parte do governo. Ela 
aponta que houve falta de ação pública, mesmo
após uma grande onda de ataques transfóbicos em 2021. Celeste
destaca que os corpos de pessoas trans e travestis são frequentemente
submetidos a um processo de desumanização, sendo vistos como corpos que não
deveriam existir, o que contribui para o aumento dos crimes de ódio.

Mas a situação não é menos preocupante em outros estados: A Bahia destaca-se 
pela maior taxa de crescimento de feminicídios em comparação com o último relatório, 
apresentando um aumento significativo de 58\%. Esse estado registra pelo 
menos um caso por dia e, lamentavelmente, lidera as ocorrências no Nordeste, 
com um total de 91 registros. O Rio de Janeiro, por sua vez, alcançou a deplorável marca de
ao menos um caso de violência contra a mulher a cada 17 horas, e dobrou 
o número de casos de violência sexual, de 39 para 75.

\fonte{Texto formulado para este material. Fonte de pesquisa: 
https://g1.globo.com/rj/rio-de-janeiro/noticia/2023/03/06/
estudo-em-sete-estados-aponta-que-uma-mulher-e-vitima-de-violencia-a-cada-quatro-horas.ghtml
Acesso: 25 set. 2023}

\end{myquote}

\num{3} Sintetize o conteúdo da notícia usando apenas uma palavra. 

\reduline{Com base no título e no texto, pode-se sintetizar o conteúdo da notícia com 
a palavra ``feminicídio'', isto é, o assassinato de mulheres envolvendo violência 
doméstica e familiar, menosprezo ou discriminação à condição de mulher da vítima.\hfill}

\pagebreak
\num{4} Podemos considerar que o título e o primeiro parágrafo contêm uma apresentação 
geral do texto. Explique essa afirmação com suas próprias palavras.

\reduline{Com base no título e no primeiro parágrafo, podemos afirmar que o conjunto do texto 
apresentará dados referentes ao feminicídio em todo o Brasil. É o que ocorre na informação
sintética do título, especificada no primeiro parágrafo, mas ainda referente ao país como
um todo. É somente no segundo e terceiro parágrafos que dados específicos de outros estados
serão apresentados. \hfill}

% REVER
% \begin{table}[h!]
% \begin{tabular}{ll} 
% (I) Ceará\\ 
% (II) Pernambuco & (\rosa{II}) O segundo colocado em número de casos de feminicídio.\\ (\rosa{II}) Estado onde ocorre pelo menos um caso a cada dois dias.\\ (\rosa{II}) O primeiro colocado em número de casos de transfeminícidio.\\ (\rosa{I}) O primeiro colocado em número de casos de transfeminicídio nos anos anteriores.
% \end{tabular}
% \end{table}

%\coment{(II) O segundo colocado em número casos de feminicídio. 
%(II) Estado onde ocorre pelo menos um caso a cada dois dias.
%(II) O primeiro colocado em número de casos de transfeminícidio.
%(I) O primeiro colocado em número de casos de transfeminicídio nos anos anteriores.}

\num{5} No segundo parágrafo, a notícia traz uma nova informação sobre violência contra mulher. 
Que informação é essa?

\reduline{A notícia traz novos dados sobre os casos de transfeminicídio --- isto é, 
os homicídios de mulheres trans. A pesquisa indica que o número dessas ocorrências em 
Pernambuco superou o do Ceará.\hfill}

\num{6} Segundo o texto da notícia, qual a principal motivação dos crimes de transfeminicídio
no estado de Pernambuco?

\reduline{Segundo a matéria, a negligência do governo em promover políticas públicas
contra os crimes de ódio é fator que influencia no aumento dos
casos no estado de Pernambuco.\hfill}

\num{7} O texto deixa claro que crimes de transfobia podem ser considerados um crime específico.
Que tipo de crime é esse? Segundo a pesquisadora citada no texto, o que incita esse tipo
de crime? 

\reduline{Segundo o texto, crimes de transfobia podem ser considerados crimes de ódio. Para
a pesquisadora Dália Celeste, a desumanização dos corpos trans e travestis contribui para que eles
sejam vistos como corpos que não deveriam existir, o que alimenta os crimes de ódio.\hfill} 

\enlargethispage{2\baselineskip}
\num{8} Identifique, no segundo e terceiro parágrafos, recursos e expressões que permitem 
concluir qual é o ponto de vista do autor e explicite qual é a opinião dele sobre os feminicídios 
no Brasil.

\reduline{O autor do texto pretende denunciar o aumento do feminicídio no Brasil. No segundo 
parágrafo, o autor dá voz a uma pesquisadora que se solidariza com as vítimas; no terceiro, frases e
expressões como ``a situação não é menos preocupante'', ``lamentavelmente'' e ``deplorável'' revelam
a oposição frontal do autor às ocorrências de feminicídio no Brasil.\hfill}

\section*{Treino}

\num{1} Leia o texto abaixo para responder à questão.

\begin{myquote}

\begin{minipage}{0.7\textwidth}
Nas Américas, a malária continua ameaçando a vida de cerca de 138 milhões de pessoas.
Para a erradicação dessa doença, é fundamental a cooperação entre os países atingidos e 
entre distintos setores da sociedade. É digno de nota que os males causados por essa doença
não se limitam apenas à saúde, mas também impactam a economia, as relações de trabalho e 
o meio-ambiente. Como os mosquitos transmissores da doença não respeitam fronteiras, a migração
de áreas endêmicas desempenha um papel significativo na propagação da malária.
\end{minipage}
\hfill
\begin{minipage}{0.2\textwidth}
  \centering
  \includegraphics[width=\textwidth]{./imgSAEB_7_POR/media/image26.png}
  %\caption{Legenda da imagem}
  \label{fig:exemplo}
\end{minipage}

% \begin{figure}[H]
% \centering
% \includegraphics[scale=0.35]{./imgSAEB_7_POR/media/image26.png}
% \end{figure}
%Fonte: freepik https://br.freepik.com/vetores-gratis/mosquito-em-estilo-simples-doodle-no-fundo-branco_24086540.htm 

\fonte{Texto formulado para este material. Fonte de pesquisa:
https://www.paho.org/pt/noticias/4-11-2022-intervencoes-locais-sao-cruciais-para-atingir-objetivo-eliminacao-da-malaria.
Acesso em: 25 set. 2023.}

\end{myquote}

Segundo o texto, a erradicação da malária se baseia na cooperação entre diferentes países 
da América porque:

\begin{escolha}

  \item a migração ajuda a espalhar a doença de um país para outro.

  \item os países da América não tem nenhuma meta para eliminar a doença.

  \item não há correlação entre casos de malária e migração. 

  \item o mundo da economia e do trabalho é pouco saudável.

\end{escolha}

\num{2} Leia o texto abaixo para responder à questão.
\enlargethispage{2\baselineskip}

\begin{myquote}
Depois da queda de casos de \textsc{covid}-19, escolas públicas e privadas 
do Estado de São Paulo começaram a receber os estudantes. No entanto, em vez 
de dar prioridade ao acolhimento e à criação coletiva de novas abordagens para
organizar tempos, espaços, grupos e relações, muitas delas optaram por retornar 
à estrutura tradicional e enfatizaram as avaliações externas para ``diagnosticar 
as deficiências na aprendizagem''. O resultado dessa abordagem foi o aumento 
significativo de incidentes de violência e problemas de saúde tanto entre os
estudantes quanto os professores.

\fonte{Texto formulado para este material. Fonte de pesquisa: https://www.uol.com.br/ecoa/colunas/opiniao/2023/03/28/ataque-em-escola-policial-no-colegio-nao-e-solucao-para-evitar-tragedia.htm
Acesso em: 25 set. 2023.}
\end{myquote}

Para o autor do texto acima, quando receberam alunos e professores depois da pandemia de 
\textsc{covid}-19, as escolas deveriam: 

\begin{escolha}

  \item ensinar formas de organizar tempos, espaços, grupos e relações.

  \item priorizar a saúde de estudantes e professores por meio de palestras.

  \item amparar estudantes e professores e discutir com eles modos de organização.

  \item valorizar as tradições escolares, sobretudo as avaliações de deficiências.  

\end{escolha}

\begin{figure}[H]
\centering
\includegraphics[width=\textwidth]{./imgSAEB_7_POR/media/image27.png}
\end{figure}
%Fonte: freepick https://br.freepik.com/fotos-gratis/retrato-de-uma-linda-professora-de-pre-escola-hispanica-amando-seu-trabalho-e-se-divertindo-com-seus-alunos_27998943.htm

\num{3} Leia o texto abaixo para responder à questão.
\enlargethispage{2\baselineskip}

\begin{myquote}
Uma alimentação saudável deve ser baseada em práticas alimentares que
assumam a significação social e cultural dos alimentos como fundamento
básico conceitual. Neste sentido é fundamental resgatar estas práticas
bem como estimular a produção e o consumo de alimentos saudáveis
regionais (como legumes, verduras e frutas), sempre levando em
consideração os aspectos comportamentais e afetivos relacionados às
práticas alimentares.

\fonte{Biblioteca Virtual em Saúde do Ministério da Saúde. Alimentação saudável. 
Disponível em: https://bvsms.saude.gov.br/alimentacao-saudavel/.
Acesso em: 19 mai. 2023.}
\end{myquote}

\pagebreak
Segundo as afirmações presentes no texto, é correto afirmar que a alimentação saudável
está diretamente relacionada

\begin{escolha}

  \item à cultura e às relações afetivas do consumidor.
  
  \item ao acesso a mercados e pontos de consumo.
  
  \item à difusão oficial de informações sobre alimentos.
  
  \item à produção e ao consumo de alimentos industriais.

\end{escolha}

\begin{figure}[H]
\centering
\includegraphics[width=\textwidth]{./imgSAEB_7_POR/media/image28.png}
\end{figure}
%Fonte: freepick https://br.freepik.com/fotos-gratis/mulher-de-tiro-medio-segurando-pimenta_14959919.htm

\chapter{Os domínios da comunicação}
\markboth{Módulo 2}{}

\section*{Habilidades do SAEB}

\begin{itemize}

  \item Identificar elementos constitutivos de textos pertencentes ao
domínio jornalístico/midiático.

  \item Identificar formas de organização de textos normativos, legais e/ou
reivindicatórios. Identificar elementos constitutivos de gêneros de
divulgação científica.

  \item Analisar a relação temática entre diferentes gêneros jornalísticos.

\end{itemize}

\subsection{Habilidades da BNCC}

\begin{itemize}
  
  \item EF69LP02, EF69LP20, EF69LP27, EF67LP16, EF67LP17.

\end{itemize}

\conteudo{Você já deve ter percebido que, no nosso dia a dia, temos de ler
diferentes tipos de texto, cada um deles com uma finalidade específica. Uma
reportagem em um site de notícias serve para a gente se informar. Uma receita
é um guia prático para fazer comidas saborosas. As placas de rua servem para
as pessoas se orientarem na cidade. E assim por diante.

Cada um desses textos pertence a diferentes \textbf{gêneros textuais}, que
devem ser adequados às muitas situações do cotidiano. Eles podem servir para
informar, formar opinião, persuadir, regulamentar, propor mudanças, fazer
reclamações, reivindicações ou solicitações. Em outras circunstâncias, servem
para registrar normas e leis, para garantir direitos a partir de mecanismos
legais e ainda podem ser usados para divulgação de pesquisas e resultados de
experimentos, por exemplo. O importante é saber que, para cada função
comunicativa, há um gênero adequado de texto.

Textos como editoriais, reportagens, notícias e crônicas fazem parte do
contexto jornalístico e têm características específicas: uma reportagem, por
exemplo, propõe-se a informar objetivamente --- em linguagem clara, destinada
ao maior número de pessoas --- o que aconteceu, com quem, quando, onde e por
quê. No universo legal ou jurídico, a linguagem dos textos deve ser técnica,
com vocabulário específico para garantia da clareza e precisão conceitual. Por
isso, textos normativos ou de âmbito legal são, em geral, bastante
padronizados. Eles pretendem propor ações jurídicas, argumentar e defender
pontos de vista e atitudes com base em disposições legais. Já os textos que
visam defender causas sociais, propor mudanças políticas e econômicas
--- tais como petições, manifestos e cartas abertas --- pertencem ao universo
da reivindicação e geralmente apresentam elementos persuasivos e de
convencimento sobre determinado ponto de vista. Por essa característica,
neste domínio encontramos também os discursos públicos e políticos.

Veja que curioso: textos científicos podem soar, muitas vezes,
incompreensíveis para o público em geral, devido ao rigor conceitual e ao
vocabulário muito específico. É interessante pensar que um pesquisador, ao
escrever um artigo científico, se dirige a um público especializado,
acostumado com aquela linguagem difícil para a maioria das pessoas. Mas esse
mesmo autor pode escrever um texto mais simples e publicá-lo no jornal,
destinado ao público em geral, com a finalidade de divulgar suas descobertas a
muito mais gente.

Concluindo: para cada situação de comunicação, existe um texto correspondente,
que é adequado ao contexto em que se insere e ao público a que se destina. O
bom autor e bom leitor são as pessoas que conseguem identificar os recursos
específicos do maior número de gêneros.}

% \coment{Professor, relembre com os estudantes as características dos gêneros
% citados:

% \textbf{Domínio jornalístico midiático}: Linguagem clara e direta. Este
% domínio tem peculiaridades como o uso de títulos, subtítulos, fotografias e
% vídeos. Para que um fato ou ocorrido seja bem noticiado, o autor deve oferecer
% aos leitores ou ouvintes os dados sobre o local, a data e um resumo direto do
% que está sendo comunicado. Textos deste domínio podem ainda conter ou não a
% opinião de quem escreve ou do veículo que disponibiliza o conteúdo.

% \textbf{Domínio legal ou jurídico}: Estes textos podem ser documentos,
% estatutos, códigos de legislação e servem para leis e demais documentos
% regulatórios. São divididos em parágrafos, seções e capítulos.

% \textbf{Textos de divulgação científica}: com a finalidade de promover a
% divulgação de pesquisas e resultados de trabalhos científicos baseados
% em estudos e práticas de observação.}

\section*{Atividades}

Analise a imagem a seguir e responda às questões 1 e 2. 

\begin{figure}[H]
\centering
\includegraphics[scale=0.35]{./imgSAEB_7_POR/media/image1.png}
\end{figure}

\fonte{Prefeitura de Capivari. Secretaria de Saúde anuncia Campanha de Doação de Sangue no dia 20 de agosto. Disponível em:
https://capivari.sp.gov.br/portal/secretaria-de-saude-anuncia-campanha-de-doacao-de-sangue-no-dia-20-de-agosto/
Acesso em: 19 mai. 2023.}

\num{1} A imagem contém uma campanha de conscientização. Sabendo que
o objetivo da campanha é sensibilizar o público, aponte
recursos não verbais contribuem para a persuasão.

\reduline{Os recursos não verbais que contribuem para a persuasão são:
1. predomínio da cor vermelha no conjunto; 2. associação da imagem da bolsa de
sangue equiparada a um telefone para ilustrar o sentido da palavra
``compartilhar''; 3. caixa de destaque com informações de data e
horário.\hfill}

\num{2} Encontre no texto o verbo que indica uma recomendação de mudança 
de atitude. Escreva-o abaixo

\reduline{Os verbos ``doar'' e ``compartilhar'', no infinitivo, sugerem mudança 
de atitude. O uso do verbo no infinitivo é uma característica do gênero. Para
incentivar uma atitude, o texto utiliza linguagem apelativa e multimodal.\hfill}

% \num{3} Assinale com (V) verdadeiro ou (F) falso as afirmações a seguir sobre
% \textbf{reportagens}, de gênero jornalístico e midiático:

% \begin{table}[h!]
% \begin{tabular}{|c|c|}
% \hline
% \textbf{\begin{tabular}[c]{@{}c@{}}Verdadeiro (V) \\ ou Falso (F)?\end{tabular}} & \textbf{Afirmação} \\ \hline
% \rosa{V} & Apresentam linguagem direta \\ \hline
% \rosa{F} & Contêm verbos no imperativo \\ \hline
% \rosa{F} & Servem para divulgar conhecimentos \\ \hline
% \rosa{V} & Têm como objetivo informar \\ \hline
% \end{tabular}
% \end{table}

%\coment{V, F, F, V}

Leia o seguinte trecho do Estatuto da Criança e do Adolescente -- ECA -- para 
responder à questão 3.

\begin{myquote}

\begin{wrapfigure}{r}{0.4\textwidth}  % 'l' para alinhar à esquerda, 'r' para alinhar à direita
  \centering
  \includegraphics[width=0.4\textwidth]{./imgSAEB_7_POR/media/image33.png}
  %\caption{\textit{Texto sobre a imagem}}
\end{wrapfigure}
%Fonte: Pixabay https://pixabay.com/pt/photos/garoto-filho-feliz-retrato-1854308/

\textsc{o presidente da república}: faço saber que o Congresso Nacional decreta e
eu sanciono a seguinte Lei: (\ldots{})

Título II: Dos Direitos Fundamentais (\ldots{})

Capítulo II: Do Direito à Liberdade, ao Respeito e à Dignidade (\ldots{})

Art. 16. O direito à liberdade compreende os seguintes aspectos:

I --- ir, vir e estar nos logradouros públicos e espaços comunitários,
ressalvadas as restrições legais;

II --- opinião e expressão;

III --- crença e culto religioso;

IV --- brincar, praticar esportes e divertir-se;

V --- participar da vida familiar e comunitária, sem discriminação;

VI --- participar da vida política, na forma da lei;

VII --- buscar refúgio, auxílio e orientação. \\

\fonte{Ministério dos Direitos Humanos e da Cidadania. O Estatuto da Criança e 
do Adolescente -- ECA.
Disponível em: http://www.gov.br/mdh/pt-br/navegue-por-temas/crianca-e-adolescente/publicacoes/o-estatuto-da-crianca-e-do-adolescente. Acesso em: 19 mai. 2023.}

\end{myquote}

\num{3} Identifique as características que indicam que o texto pertence à 
esfera jurídica.

\reduline{As características que indicam que o texto pertence à esfera jurídica 
são as seguinte: divisão em artigos e capítulos, uso de numerais romanos, uso de
linguagem formal, verbos no infinitivo.\hfill}

% \num{5} O uso do verbo no infinitivo em textos jurídicos e legais tem uma função. 
% Que função é essa?

% \reduline{Os verbos no infinitivo, no caso de textos de regulamentação, indicam
% orientação, têm uma função de persuadir, convencer ou convidar a uma
% ação ou atitude.\hfill}

Analise o texto a seguir para responder às questões 4 e 5. 

% Suprimi essa imagem porque ela não tem grande contribuição para o material. Rogério, 19/5/23, 16h21
%\begin{longtable}[]{@{}
%>{\raggedright\arraybackslash}p{(\columnwidth - 0\tabcolsep) * \real{0.9861}}@{}}
%\toprule
%\endhead
%\textbf{Alerta importante para você que é jovem e vai ler esta cartilha}\includegraphics[width=1.625in,height=3.36458in]{./imgSAEB_7_POR/media/image2.png}

\begin{myquote}

O Estatuto da Criança e do Adolescente -- \textsc{eca} -- proíbe a venda de
qualquer tipo de bebida alcoólica para menores de 18 anos.

Portanto, fique esperto!

Se alguém lhe oferecer, mesmo que gratuitamente, qualquer bebida
alcoólica, \textsc{não aceite}!

Essa pessoa estará cometendo um crime.

É bom lembrar que o uso do álcool pode levar ao alcoolismo, uma doença
grave que atinge 12,3\% da população brasileira com idade entre 12 e 65
anos.

Entre os jovens de 12 a 17 anos, a taxa de alcoolismo é de 7\%.

Considere que este nível representa 554.000 jovens com sérios problemas
sociais e de saúde.

\fonte{Portal do Professor do Ministério da Educação. Drogas: Cartilha Álcool e Jovens. 
Disponível em: http://portaldoprofessor.mec.gov.br/storage/materiais/0000011863.pdf.
Acesso em: 2 de abri de 2023.}

\end{myquote}

\num{4} A quem se destina este texto? Explique e copie do texto um trecho que 
justifique sua resposta.

\reduline{O texto se destina a jovens, como se pode verificar por meio do
trecho ``Alerta importante para \textbf{você que é jovem} e vai ler esta
cartilha''.\hfill}

\num{5} Releia a frase: ``Considere que este nível representa 554.000 jovens
com sérios problemas sociais e de saúde.'' Que efeito persuasivo o autor 
alcançou ao utilizar o número absoluto que representa 7\% da população jovem?

\reduline{Ao apresentar o número absoluto de jovens que sofrem com 
problemas de alcoolismo, o autor se dirige direta e explicitamente 
ao público-alvo do texto, com o propósito de sensibilizá-lo.\hfill}

\num{6} No que diz respeito à linguagem, cite duas características comuns aos
gêneros jornalístico e de divulgação científica.

\reduline{Tanto no texto jornalístico quanto no de divulgação científica a
linguagem empregada deve ser clara e objetiva e pode haver o uso de
recursos não verbais tais como imagens, gráficos e tabelas.\hfill} 

Leia o texto abaixo e responda às questões 7 e 8. 

\begin{myquote}

\begin{figure}[H]
\centering
\includegraphics[width=0.5\textwidth]{./imgSAEB_7_POR/media/image30.png}
\end{figure}
%Fonte: pixabay https://pixabay.com/pt/vectors/m%C3%A9dico-hospital-sa%C3%BAde-m%C3%A9dica-5187733/

\emph{Brasília, 23 de março de 2020}

Excelentíssimos(as) senhores(as), Presidente da República, Ministros de
Estado, Governadores(as), Prefeitos(as), Secretários(as) de Saúde e
gestores(as) do SUS,

O CNS, enquanto órgão responsável pelo controle social no SUS, orienta
que todos(as) os(as) referidos(as) nesta carta adotem medidas
emergenciais, em todas as unidades da federação, para os próximos dois
meses (abril e maio), visando conter a crise de saúde que vivemos hoje e
que pode se agravar nos próximos dias. O objetivo é zelar pela
integridade física e mental dos cidadãos e cidadãs brasileiros, buscando
também ações específicas e sensíveis à realidade de pessoas em regime
carcerário ou cumprindo medidas socioeducativas, dentre outras
populações vulneráveis. (\ldots{})

\fonte{Conselho Nacional de Saúde. Carta aberta do CNS às autoridades 
brasileiras no enfrentamento ao Novo Coronavírus. Disponível em: https://conselho.saude.gov.br/ultimas-noticias-cns/1074-carta-aberta-do-cns-as-autoridades-brasileiras-no-enfrentamento-ao-novo-coronavirus.
Acesso em: 19 mai. 2023.}

\end{myquote} 

\num{7} A quem se destina a carta?

\reduline{A carta se destina ao Presidente da República e aos Ministros 
de Estado, Governadores (as), Prefeitos(as), Secretários(as) de Saúde e gestores(as) do SUS.\hfill}

\num{8} Selecione do texto um trecho que expressa o motivo de reivindicação da carta.

\reduline{O trecho que expressa o motivo de reivindicação da carta é ``visando
conter a crise de saúde que vivemos hoje e que pode se agravar nos próximos
dias. Nesse sentido, é fundamental que sejam potencializadas ou desenvolvidas
as seguintes ações''.\hfill}

% \num{10} Relacione os textos e suas características.

% \begin{table}[h!]
% \begin{tabular}{c|c|c|c|}
% \cline{2-2} \cline{4-4}
%  & \textbf{Texto} &  & \textbf{Característica} \\ \hline
% \multicolumn{1}{|c|}{\textbf{1}} & \textbf{petição} & \rosa{1} & É um texto argumentativo de reivindicação \\ \hline
% \multicolumn{1}{|c|}{\textbf{2}} & \textbf{estatuto} & \rosa{4} & Tem como objetivo divulgar informações \\ \hline
% \multicolumn{1}{|c|}{\textbf{3}} & \textbf{folheto} & \rosa{3} & \begin{tabular}[c]{@{}c@{}}Utiliza recursos não verbais como \\ forma de persuasão\end{tabular} \\ \hline
% \multicolumn{1}{|c|}{\textbf{4}} & \textbf{notícia} & \rosa{2} & Tem por objetivo regulamentar e normatizar \\ \hline
% \end{tabular}
% \end{table}

%Gabarito(1) É um texto argumentativo de reivindicação. (4) Tem como objetivo divulgar informações. (3) utiliza recursos não verbais como forma de persuasão. (2) tem por objetivo regulamentar e normatizar

\pagebreak

\section*{Treino}

\num{1} Leia o texto abaixo para responder à questão.

\begin{myquote}

\begin{figure}[H]
\centering
\includegraphics[scale=0.3]{./imgSAEB_7_POR/media/image31.png}
\end{figure}
%Fonte: freepick https://br.freepik.com/fotos-gratis/foto-completa-casal-senior-caminhando-juntos_15725676.htm

De acordo com o professor José Carlos Farah, as quedas em idosos são comuns
porque o processo de envelhecimento traz algumas perdas importantes no nosso
corpo.

``A osteopenia, que é a perda de massa óssea, deixa o osso enfraquecido, e a
perda da massa muscular, como consequência, traz a falta do controle do
movimento e equilíbrio, a perda cognitiva, que diminui a nossa atenção e
percepção e a baixa aptidão física''. Quando se soma isso a um quadro de
doenças preexistentes, o resultado é o de um cenário propício às quedas. Ainda
segundo o colunista, o processo de perdas do envelhecimento é inevitável. Mas
alguns hábitos podem ajudar e ele cita entre estes a prática da atividade
física. ``Os benefícios da prática da atividade física se contrapõem ao
processo de envelhecimento. O exercício promove o aumento da massa muscular e
da coordenação motora, do equilíbrio e das funções cognitivas. Este hábito já
diminui a possibilidade de quedas, associado ao ambiente livre de possíveis
obstáculos que podem atrapalhar o dia a dia do idoso''.

\fonte{José Carlos Simon Farah. Jornal da USP. Exercícios físicos podem contribuir 
para a redução da queda de idosos. Disponível em: https://jornal.usp.br/radio-usp/exercicios-fisicos-podem-contribuir-para-a-reducao-da-queda-de-idosos/.
Acesso em: 25 set. 2023. com adaptações.}
\end{myquote}

Pode-se afirmar que o trecho acima pertence a um texto de divulgação científica, pois contém:

\begin{escolha}
  
  \item informações técnicas voltadas ao público especializado da área médica.
  
  \item explicações de especialista sobre os benefícios do exercício físico para idosos.
  
  \item citações literais de artigos científicos sobre benefícios do exercício físico.
  
  \item narrativas pessoais sobre osteopenia em idosos e benefícios do exercício físico.

\end{escolha}

\num{2} Leia os dois textos a seguir para responder à questão. 

\begin{myquote}

%\textbf{Texto 1: ES registra aumento de casos de dengue na 1ª semana de janeiro}
% Em todo o mês de janeiro de 2022, o Espírito Santo teve 951
% casos. Só nesta primeira semana já foram 1.453. Infectologista acredita
% que estado pode estar vivendo uma epidemia de casos da doença.

% Só na primeira semana do ano foram registrados 1.453 casos da dengue no
% Espírito Santo. Número bem maior do que os registros do mês de janeiro
% de 2022, quando no mês todo foram registrados 951 casos da doença.

% Segundo o infectologista Crispim Cerutti, o estado pode estar vivendo
% uma epidemia da doença.

% ``A gente tem epidemias que ocorrem a aproximadamente a cada três anos. A
% última que a gente teve foi em 2019/2020. Embora o intervalo seja um
% pouco curto, a gente pode dizer que sim, estamos em uma nova epidemia da
% doença. A frequência de casos acompanha o ciclo de vida dos mosquitos'',
% explicou o infectologista.

% \begin{figure}[H]
% \centering
% \includegraphics[scale=0.15]{./imgSAEB_7_POR/media/image32.png}
% \end{figure}

\textbf{Texto 1}

\begin{wrapfigure}{r}{0.4\textwidth}  % 'l' para alinhar à esquerda, 'r' para alinhar à direita
  \centering
  \includegraphics[width=0.35\textwidth]{./imgSAEB_7_POR/media/image32.png}
  %\caption{\textit{Texto sobre a imagem}}
\end{wrapfigure} %Fonte: pixabay https://pixabay.com/pt/vectors/mosquito-picada-lib%C3%A9lula-asa-voe-1743558/

No mês de janeiro de 2022, o estado do Espírito Santo registrou um
total de 951 casos de dengue. Surpreendentemente, apenas na primeira semana do
ano seguinte, já foram identificados 1.453 casos da doença. Esse aumento
significativo nas infecções levanta preocupações entre os especialistas em
doenças infecciosas, sugerindo a possibilidade de o estado estar enfrentando
uma epidemia da dengue.

Apenas nos primeiros sete dias do ano, o Espírito Santo já superou em muito o
número de casos relatados durante todo o mês de janeiro do ano anterior, que
totalizou 951 infecções. O doutor Crispim Cerutti, especialista em
infectologia, expressou sua preocupação com a situação e sugeriu que o estado
possa estar enfrentando uma epidemia.

O infectologista explicou que historicamente vivemos epidemias a
cada três anos, aproximadamente. A última aconteceu em 2019/2020.
Apesar de esse intervalo ser relativamente curto, Cerutti explica que 
é possível concluir que estamos enfrentando uma nova epidemia da doença. 
A frequência dos casos acompanha o ciclo de vida dos mosquitos transmissores.

\fonte{Texto formulado para este material. Fonte de pesquisa: https://g1.globo.com/es/espirito-santo/noticia/2023/01/18/es-registra-aumento-de-casos-de-dengue-na-1a-semana-de-janeiro.ghtml. Acesso em:
25 set. 2023.}

\textbf{Texto 2}

Para evitar a reprodução do Aedes aegypti, o Ministério da Saúde recomenda 

\begin{itemize}
  
  \item Utilizar telas de proteção nas janelas de casa;
  
  \item Fechar portas e janelas;
  
  \item Manter o terreno limpo e livre de materiais ou entulhos;
  
  \item Tampar os tonéis e caixas d'água;
  
  \item Limpar as calhas regularmente;
  
  \item Tampar garrafas ou deixá-las com a boca para baixo;
  
  \item Manter lixeiras tampadas;
  
  \item Manter ralos limpos e com tela;
  
  \item Completar pratos de vasos de plantas com areia;
  
  \item Limpar com escova ou bucha os potes de água para animais;
  
  \item Limpar acessórios de decoração que ficam fora de casa 

  \item Evitar acúmulo de água em pneus e calhas;
  
  \item Usar repelentes elétricos ou naturais próximos às janelas;
  
  \item Evitar produtos de higiene com perfume, pois podem atrair insetos;
  
  \item Retirar água acumulada.
 
\end{itemize}

\fonte{Helio Carvalho. G1. Sete cidades da região de Campinas vivem situação de alerta para dengue; entenda o que significa.
Disponível em: https://g1.globo.com/sp/campinas-regiao/noticia/2023/01/17/sete-cidades-da-regiao-de-campinas-vivem-situacao-de-alerta-para-dengue-entenda-o-que-significa.ghtml. 
Acesso em: 19 mai. 2023.}

\end{myquote}

Os dois textos tratam de questões relacionadas à dengue. Quanto às diferenças
e semelhanças entre eles, assinale a alternativa correta:

\begin{escolha}

    \item Os dois exemplos pretendem informar a população sobre como evitar os casos de dengue.

    \item O primeiro apresenta uma notícia, e o segundo traz indicações de ações de prevenção.

    \item O primeiro exemplo é um texto de opinião; o segundo, um folheto informativo.

    \item O primeiro texto traz dados científicos sobre a dengue, e o segundo é um texto argumentativo.

\end{escolha}

\num{3} Leia os trechos extraídos da Constituição Brasileira de 1988 e responda
ao que se pede:

\begin{myquote}

\textbf{Trecho 1}

Art. 1.º A República Federativa do Brasil, formada pela união
indissolúvel dos Estados e Municípios e do Distrito Federal,
constitui-se em Estado democrático de direito e tem como fundamentos: (\ldots{})

Art. 3.º Constituem objetivos fundamentais da República Federativa do
Brasil:

IV - promover o bem de todos, sem preconceitos de origem, raça, sexo,
cor, idade e quaisquer outras formas de discriminação.

Art. 4.º A República Federativa do Brasil rege-se nas suas relações
internacionais pelos seguintes princípios:

III - autodeterminação dos povos;

\textbf{Trecho 2}

Art. 215. O Estado garantirá a todos o pleno exercício dos direitos
culturais e acesso às fontes da cultura nacional, e apoiará e
incentivará a valorização e a difusão das manifestações culturais.

§ 1.º O Estado protegerá as manifestações das culturas populares,
indígenas e afro-brasileiras, e das de outros grupos participantes do
processo civilizatório nacional. 

Art. 216. Constituem patrimônio
cultural brasileiro os bens de natureza material e imaterial, tomados
individualmente ou em conjunto, portadores de referência à identidade, à
ação, à memória dos diferentes grupos formadores da sociedade
brasileira, nos quais se incluem:

I - as formas de expressão;

II - os modos de criar, fazer e viver;

III - as criações científicas, artísticas e tecnológicas;

IV - as obras, objetos, documentos, edificações e demais espaços
destinados às manifestações artístico-culturais;

V - os conjuntos urbanos e sítios de valor histórico, paisagístico,
artístico, arqueológico, paleontológico, ecológico e científico.

§ 1.º O poder público, com a colaboração da comunidade, promoverá e
protegerá o patrimônio cultural brasileiro, por meio de inventários,
registros, vigilância, tombamento e desapropriação, e de outras formas
de acautelamento e preservação. \\

\fonte{\textbf{Constituição da República Federativa do Brasil}.}

\end{myquote}

Sobre os trechos selecionados é correto afirmar que:

\begin{escolha}

  \item os dois trechos são contraditórios entre si.

  \item não há relação direta entre os dois trechos.

  \item os dois trechos versam sobre questões distintas.

  \item os dois trechos são complementares. 

\end{escolha}

\chapter{Nas tramas do texto literário}
\markboth{Módulo 3}{}

\section*{Habilidades do SAEB}

\begin{itemize}
  
  \item Analisar elementos constitutivos de textos pertencentes ao domínio literário.
  
  \item Analisar a intertextualidade entre textos literários ou entre estes e outros textos 
verbais ou não verbais.
  
  \item Inferir a presença de valores sociais, culturais e humanos em textos literários.

\end{itemize}

\subsection{Habilidades da BNCC}

\begin{itemize}

  \item EF69LP44, EF69LP47, EF67LP27

\end{itemize}

\conteudo{Você já deve ter notado que o prazer de ler um texto 
literário é a possibilidade de encontrar mundos, pessoas e costumes
desconhecidos. Esse é uma das mais poderosas ferramentas
da literatura: por meio dela, junto com a nossa 
imaginação, a gente pode ir muito longe, a lugares possíveis
e até impossíveis, dependendo do caso. 

Uma das características mais fascinantes da literatura é que, ao 
mesmo tempo, ela é expressão de crenças, valores e ideias de determinada
sociedade. Da mesma maneira que estimula nossa criatividade, ela também
nos mostra características do que está perto de nós, de nossa
cultura. Por meio de textos literários, podemos observar hábitos, 
acontecimentos, desafios e questões de determinada época. A análise 
da literatura parte do texto, mas encontra nele sentidos filosóficos,
sociológicos, históricos e antropológicos.

Ainda mais incrível é que alguns textos literários são tão ricos 
e tratam de temas tão essenciais e universais que acabam por sobreviver 
ao tempo. É fabuloso pensar que um texto escrito há mais de quinhentos
anos --- como as peças do inglês William Shakespeare, por exemplo ---
ainda pode ser interessante para um jovem leitor, como você. Não por 
acaso, marcas de \textbf{intertextualidade} permeiam a cultura
literária: trata-se de marcas indicadoras de que um autor está dialogando
com outros, do passado, seja para reafirmar o que fizeram, seja para 
superá-los. 

Por tudo isso, é fundamental o estudo de aspectos pertinentes à
linguagem, ao estilo, à estrutura e à temática das narrativas. 
A literatura tem o poder de despertar sensações,
questionamentos e proporcionar experiências estéticas que se refletem na
maneira como as pessoas escolhem ser e estar no mundo. O reconhecimento
do valor e da importância das obras do domínio literário pode
proporcionar uma formação ética e questionadora que participa da
formação cultural e intelectual da sociedade.}

% \coment{Professor(a), questione os estudantes sobre os gêneros textuais
% literários que conhecem, retome as principais características
% do conto, crônica, poesia, literatura de cordel, romance e demais
% gêneros que surgirem.

% Reforce a ideia de que tais gêneros retratam a sociedade com seus
% valores e desafios. Retome com os os estudantes os conhecimentos
% prévios sobre mitos, lendas e demais textos de origem popular e saliente
% a forma como os temas trabalhados remontam a hábitos, valores e crenças
% da cultura e da época em que foram escritos (lenda do milho, da
% mandioca, do arroz, sobre os rios, a paisagem, ritos de passagem,
% comemorações, etc). Chame a atenção dos estudantes para o
% fato de que os diversos tipos de artes podem se comunicar. Neste caso,
% pode ser citado como exemplo o texto dramático --- que
% pode ser lido, mas é escrito para ser encenado. Mostre como o teatro
% também promove integração de diversas expressões artísticas. Cite também
% outras expressões tais como a música e as artes plásticas e estimule os
% estudantes a pensarem como essas expressões se complementam.}

\section*{Atividades}

\num{1} O conto popular é um gênero literário proveniente da tradição oral e
em sua textualidade mantém algumas marcas de estilo quanto à linguagem. Sobre
este aspecto da linguagem dos contos populares, cite duas marcas fundamentais
deste gênero.

\reduline{O gênero conto popular é marcado pela linguagem simples, direta e com
fortes traços de oralidade tais como regionalismos, gírias e expressões
populares.\hfill}

\num{2} Textos narrativos curtos e objetivos, baseados em eventos do
cotidiano: este gênero textual pretende divertir, entreter, provocar
reflexão ou fazer uma crítica. Qual gênero literário possui essas
características?

\reduline{As características apresentadas se referem à crônica, que traz questões 
cotidianas e pode conter traços de humor.\hfill}

Leia o trecho a seguir e responda às questões 3 e 4.

\begin{myquote}

\begin{verse}

Disse Pedro isso é blasfêmia \\
É bastante astucioso \\
Pistoleiro e cangaceiro \\ 
Esse povo é impiedoso \\*
Não ganharão o perdão \\*
Do santo Pai Poderoso 

Inda mais tem sua má fama \\
Vez por outra comentada \\
Quando há um julgamento \\
Duma alma tão penada \\*
Porque fora violenta \\*
Em sua vida é baseada. 

\end{verse}

\fonte{Guaipuan Vieira. A chegada de lampião no céu. Disponível em: http://www.dominiopublico.gov.br/download/texto/rd000001.pdf.
Acesso em: 20 mai. 2023.}

\end{myquote}

\num{3} O texto acima pertence a qual gênero textual?

\reduline{O texto pertence à Literatura de Cordel.\hfill}

\num{4} Cite duas características do gênero presentes no trecho.

\reduline{Divisão em estrofes e versos, rima, métrica, temas da cultura
nordestina, no caso o personagem Lampião, marcas de oralidade.\hfill}

% \num{5} Use a legenda para indicar as características da crônica e do conto.
% Marque (1) para o crônica e (2) para o conto. Os dois podem ter
% características em comum:

% % Please add the following required packages to your document preamble:
% % \usepackage[table,xcdraw]{xcolor}
% % If you use beamer only pass "xcolor=table" option, i.e. \documentclass[xcolor=table]{beamer}
% \begin{table}[h]
% \begin{tabular}{cl}  %{| >{\columncolor[HTML]{DAE8FC}}c |l|}
% \hline
% \rosa{1, 2} & Poucos personagens \\ \hline
% \rosa{1, 2} & Curto e objetivo \\ \hline
% \rosa{1} & Histórias narram fatos do cotidiano e podem estimular a reflexão \\ \hline
% \rosa{2} & Podem surgir de narrativas populares transmitidos pela tradição oral \\ \hline
% \rosa{1} & É comum que seja publicado em jornais \\ \hline
% \end{tabular}
% \end{table}

Leia o trecho a seguir e responda às questões 5 e 6.

\begin{myquote}

% \begin{figure}[H]
% \centering
% \includegraphics[width=0.5\textwidth]{./imgSAEB_7_POR/media/image35.png}
% \end{figure}

\begin{minipage}{0.6\textwidth}
Houve noutro tempo um rei que tinha o hábito de jogar, e todos com
quem jogava perdiam. Uma vez convidou a um outro rei para jogar, e, no
dia marcado, este se apresentou; mas perdeu todas as mãos do jogo, até
que se desenganou e despediu-se para se ir embora.
\end{minipage}
\hfill
\begin{minipage}{0.35\textwidth}
  \centering
  \includegraphics[width=\textwidth]{./imgSAEB_7_POR/media/image35.png}
  %\caption{Legenda da imagem}
  %\label{fig:exemplo}
\end{minipage} %Fonte: pixabay https://pixabay.com/pt/vectors/coroa-rei-o-rei-de-monarca-j%C3%B3ia-5334406/

\fonte{Sílvio Romero. Contos Populares do Brasil. Coleção acervo brasileiro.
Volume 3, 2ª edição. Jundiaí: Cadernos do mundo inteiro, 2018 p. 128.}

\end{myquote}

\num{5} O que a expressão ``Houve noutro tempo'' quer dizer? Comumente outra
expressão é usada para iniciar os contos. Que expressão é essa?

\reduline{A expressão faz alusão a um tempo passado indeterminado. Nos contos é
comum que apareça a expressão ``Era uma vez'' com o mesmo significado de
tempo indeterminado.\hfill}

\num{6} Qual o tipo de narrador do trecho acima?

\reduline{O narrador do trecho é o narrador observador.\hfill}

\pagebreak

% \num{7} Sobre a intertextualidade, assinale verdadeiro (V) ou falso (F) para as seguintes
% afirmações.

% \begin{table}[h!]
% \begin{tabular}{|c|c|}
% \hline
% \textbf{\begin{tabular}[c]{@{}c@{}}Verdadeiro (V) \\ ou Falso (F)?\end{tabular}} & \textbf{Afirmação} \\ \hline
% \rosa{F} & É a prática de copiar trechos de outras obras \\ \hline
% \rosa{V} & \begin{tabular}[c]{@{}c@{}}É um recurso que pode trazer ainda mais \\ possibilidades de interpretação para um texto\end{tabular} \\ \hline
% \rosa{V} & \begin{tabular}[c]{@{}c@{}}Pode estar implícita ou explícita em textos \\ e demais obras artísticas\end{tabular} \\ \hline
% \rosa{V} & \begin{tabular}[c]{@{}c@{}}Pode estar presente em traduções, paródias \\ e releituras de obras\end{tabular} \\ \hline
% \rosa{F} & Pode ser caracterizada como plágio \\ \hline
% \end{tabular}
% \end{table}

%\coment{F, V, V, V, F}

\num{7} No trecho abaixo vemos um exemplo de discurso indireto:

\begin{myquote}

O pai explicou: 

--- Filha, você precisa fazer sua tarefa agora. Mais tarde
temos compromisso.

\end{myquote}

Passe a frase para o discurso indireto.

\reduline{O pai explicou para a filha que ela precisava fazer a tarefa naquele
momento, pois, mais tarde, eles tinham compromisso.\hfill}

\section*{Treino}

\num{1} Leia o texto abaixo para responder à questão.

\begin{myquote}

\textbf{Lenda da Mandioca}

\emph{Lenda Indígena}

Era uma vez uma índia chamada Atiolô. Quando o chão começou a ficar
coberto de frutinhas de murici, ela se casou com Zatiamarê. As frutinhas
desapareceram, as águas do rio subiram apodrecendo o chão. Depois, o sol
queimou a terra, um ventinho molhado começou a chegar do alto da serra.
Quando os muricis começaram outra vez a cair, numa chuvinha amarela,
Atiolô começou a rir sozinha. Estava esperando uma menininha.

\fonte{Ana Rosa Abreu e outros autores. Contos tradicionais, fábulas,
lendas e mitos. Disponível em http://www.dominiopublico.gov.br/download/texto/me001614.pdf. Acesso em: 20 mai. 2023}

\end{myquote}

Nesse trecho da lenda da mandioca, notam-se traços da cultura
indígena com relação ao modo de explicar a origem dos alimentos, a
origem da natureza, a preservação dos costumes e a contagem do tempo.
Pode-se notar que a fruta murici é indicadora da passagem de certo 
período de tempo. Assinale a alternativa que explica corretamente 
quanto tempo se passou.

\begin{escolha}

  \item É possível concluir que cerca de um ano se passou.
  
  \item Não se pode saber ao certo quanto tempo se passou.
  
  \item A mudança de estação indica a passagem de três meses.
  
  \item É legítimo pressupor que muitos anos se passaram. 

\end{escolha}

\num{2} Leia os poemas abaixo para responder à questão. 

\begin{myquote}

\textbf{Texto I: Canção do exílio} 

\emph{Gonçalves Dias}

% \begin{figure}[H]
% \centering
% \includegraphics[scale=0.25]{./imgSAEB_7_POR/media/image36.png}
% \end{figure}

\begin{verse}

\begin{minipage}{0.5\textwidth}
Minha terra tem palmeiras, \\
Onde canta o Sabiá; \\
As aves, que aqui gorjeiam, \\
Não gorjeiam como lá. \\

Nosso céu tem mais estrelas,\\
Nossas várzeas têm mais flores, \\
Nossos bosques têm mais vida, \\
Nossa vida mais amores. \\

Em cismar, sozinho, à noite, \\
Mais prazer eu encontro lá; \\
Minha terra tem palmeiras, \\
Onde canta o Sabiá. \\

Minha terra tem primores, \\
Que tais não encontro eu cá; \\
Em cismar -- sozinho, à noite -- \\
Mais prazer eu encontro lá; \\
Minha terra tem palmeiras, \\
Onde canta o Sabiá. \\

Não permita Deus que eu morra, \\
Sem que eu volte para lá; \\
Sem que desfrute os primores \\
Que não encontro por cá; \\
Sem qu'inda aviste as palmeiras, \\
Onde canta o Sabiá.
\end{minipage}
\hfill
\begin{minipage}{0.45\textwidth}
  \centering
  \includegraphics[width=\textwidth]{./imgSAEB_7_POR/media/image36.png}
  %Fonte: wikipedia https://pt.wikipedia.org/wiki/Gon%C3%A7alves_Dias#/media/Ficheiro:Gon%C3%A7alves_dias.jpg
  %\caption{Legenda da imagem}
  %\label{fig:exemplo}
\end{minipage}

\end{verse}

\fonte{Gonçalves Dias. Primeiros Cantos. Disponível em:
http://www.dominiopublico.gov.br/download/texto/bn000100.pdf.
Acesso em: 6 abr de 2023.}

\pagebreak 

\textbf{Texto II: Canção do exílio}

\emph{Casimiro de Abreu}

% \begin{figure}[H]
% \centering
% \includegraphics[scale=0.25]{./imgSAEB_7_POR/media/image37.png}
% \end{figure}
%Fonte: wikipedia https://pt.wikipedia.org/wiki/Casimiro_de_Abreu#/media/Ficheiro:Casimiro_de_Abreu_(Iconogr%C3%A1fico).jpg

\begin{verse}

\begin{minipage}{0.5\textwidth}
Eu nasci além dos mares: \\
\qquad Os meus lares, \\
Meus amores ficam lá! \\
– Onde canta nos retiros \\
\qquad Seus suspiros, \\
Suspiros o sabiá! \\

Oh que céu, que terra aquela, \\
\qquad Rica e bela \\
Como o céu de claro anil! \\
Que seiva, que luz, que galas, \\
\qquad Não exalas \\
Não exalas, meu Brasil! \\

Oh! que saudades tamanhas \\
\qquad Das montanhas, \\
Daqueles campos natais! \\
Daquele céu de safira \\
\qquad Que se mira, \\
Que se mira nos cristais! \\

Não amo a terra do exílio, \\
\qquad Sou bom filho, \\
Quero a pátria, o meu país, \\
Quero a terra das mangueiras \\
\qquad E as palmeiras, \\
E as palmeiras tão gentis! 
\end{minipage}
\hfill
\begin{minipage}{0.45\textwidth}
  \centering
  \includegraphics[width=\textwidth]{./imgSAEB_7_POR/media/image37.png}
  %Fonte: wikipedia https://pt.wikipedia.org/wiki/Casimiro_de_Abreu#/media/Ficheiro:Casimiro_de_Abreu_(Iconogr%C3%A1fico).jpg
  %\caption{Legenda da imagem}
  %\label{fig:exemplo}
\end{minipage}

\end{verse}

\fonte{Casimiro de Abreu. \textit{Primaveras}.}

\end{myquote}

A primeira versão da ``Canção do exílio'', escrita por Gonçalves Dias em 1847,
tornou-se importante obra do Romantismo brasileiro. Posteriormente, em 1859, o
poeta Casimiro de Abreu retomou a poesia de Gonçalves Dias. Sobre esta relação
de intertextualidade em obras literárias assinale a alternativa correta.

\begin{escolha}

  \item O poema de Casimiro de Abreu é um plágio do poema de Gonçalves Dias.

  \item O poema de Casimiro de Abreu não se refere ao poema de Gonçalves Dias.

  \item O poema de Casimiro de Abreu é uma homenagem ao poema de Gonçalves Dias.

  \item O poema de Gonçalves Dias é uma paródia do poema de Casimiro de Abreu.

\end{escolha}

\num{3} Leia o poema abaixo para responder à questão.

\begin{figure}
\centering
\includegraphics[width=4.11458in,height=2.80208in]{./imgSAEB_7_POR/media/image3.png}
\end{figure}

\fonte{OBRANOME. Catálogo. Caixa Econômica Federal e
Embaixada da Espanha, no Conjunto Cultural da Caixa, Brasília, 2003.}

O texto acima pode ser caracterizado como poema visual. Os poemas visuais
foram amplamente explorados pelos poetas do concretismo e contêm
características próprias. Sobre as características da poesia concreta,
assinale a alternativa correta.

\begin{escolha}

  \item A combinação de palavras e imagens é irrelevante para a criação dos sentidos do poema. 

  \item Regras rígidas para a estrutura em versos rimados e estrofes caracterizam a poesia visual.

  \item Imagens ilustrativas não têm relação com os sentidos propostos pelo autor.

  \item A disposição de letras e palavras na página é recurso de produção de sentidos do poema.

\end{escolha}

\chapter{Formas de composição do sentido}
\markboth{Módulo 4}{}

\section*{Habilidades do SAEB}

\begin{itemize}
  
  \item Analisar efeitos de sentido produzido pelo uso de formas de apropriação 
  textual (paráfrase, citação etc.).
  
  \item Analisar os efeitos de sentido decorrentes dos mecanismos de construção 
  de textos jornalísticos/midiáticos.

\end{itemize}


\subsection{Habilidades da BNCC}

\begin{itemize}

  \item EF69LP16, EF69LP43.

\end{itemize}

\conteudo{Textos argumentativos são aqueles em que a pessoa tenta
convencer alguém de algo ou expressar sua opinião de forma convincente. 
É o caso, por exemplo, de um editorial de jornal ou de uma postagem na
rede social, cujos autores querem defender seu ponto de vista. 
Esses textos têm organização, vocabulários e estilo específicos ---
e todas essas características servem para tentar persuadir o leitor.

Às vezes, os autores usam recursos específicos para convencer os leitores. Eles
podem citar especialistas, mostrar dados de pesquisas ou dar exemplos para
apoiar o que estão dizendo. É como quando você quer convencer seus amigos de
algo e apresenta provas para mostrar que está certo.

Não é só em textos que as pessoas usam esses truques. No dia a dia, em
conversas com amigos e familiares, ou até mesmo em negociações e reuniões de
trabalho, as pessoas tentam persuadir e convencer umas às outras usando essas
técnicas.

Portanto, saber reconhecer e usar esses recursos é importante para resolver
conflitos de forma eficaz. Cada tipo de texto tem suas próprias maneiras de
usar essas técnicas, como usar aspas para citar alguém, dar exemplos para
apoiar argumentos e escolher as palavras certas para quem vai ler ou ouvir. É
como um jogo de estratégia em que o autor tenta ganhar o leitor para o seu
lado, e entender esses truques pode ser muito útil.}

% Há muitos elementos utilizados para argumentação em textos.
% Por meio da estrutura do texto, da escolha de palavras e de recursos de estilo,
% é possível perceber objetivos e intenções do autor. Os textos argumentativos
% -- tais como artigos de opinião, editoriais ou discursos -- apresentam em sua
% construção elementos que visam convencer de alguma ideia ou expor determinado 
% ponto de vista. Para atingir esse objetivo, o autor se vale da coesão e da 
% coerência. Com argumentos consistentes, bem concatenados e ideias claras, 
% é possível encaminhar ao leitor as ideias almejadas. Existem recursos adequados
% para persuadir ou convencer o
% leitor: é o caso de argumentos de especialistas, dados de pesquisas ou exemplos,
% entre muitos outros. Toda pessoa que deseja comunicar algo faz uso desses
% recursos: no âmbito pessoal, em conversas informais entre colegas e familiares;
% no âmbito profissional, em negociações e reuniões; no âmbito político, em
% discursos; no âmbito estudantil e acadêmico, na elaboração de teses,
% dissertações e apresentações de trabalho. Portanto, saber reconhecer e
% utilizar recursos de persuasão e convencimento é fundamental para a 
% convivência em sociedade e para a resolução de conflitos nos quais há a
% necessidade discursos claros e orientados por argumentos. 

% Para cada gênero textual, existem recursos comuns que auxiliam na boa
% comunicação, de acordo com a finalidade e intenção do comunicador.
% Dentre eles, o uso de aspas para introduzir citações, o uso de exemplos
% para produzir argumentos, a escolha das palavras e de expressões de
% acordo com o receptor.}

% \coment{Professor(a), relembre os estudantes sobre a necessidade de persuasão e
% convencimento nos gêneros textuais já estudados. É comum que relacionem
% a persuasão aos textos publicitários, mas vá além e discuta recursos de
% convencimento também em textos de opinião e em situações cotidianas.}

\section*{Atividades}

Leia o texto abaixo para responder às questões de 1 a 9.

\begin{myquote}

\textbf{Especialistas indicam formas de combate a atos de intimidação}

\begin{figure}[H]
\centering
\includegraphics[width=0.95\textwidth]{./imgSAEB_7_POR/media/image38.png}
\end{figure}
%Fonte: pixabay https://pixabay.com/pt/vectors/ass%C3%A9dio-moral-anti-bullying-filho-7107525/

Um em cada dez estudantes brasileiros é vítima de \textit{bullying} -- anglicismo
que se refere a atos de intimidação e violência física ou psicológica,
geralmente em ambiente escolar. O dado foi divulgado esta semana pelo
Programa Internacional de Avaliação de Estudantes (Pisa) 2015.

Especialistas, como a professora de psicologia Ciomara Shcneider,
psicanalista de crianças e adolescentes, defendem que pais e escola
devem estar atentos ao comportamento dos jovens e manter sempre abertos
os canais de comunicação com eles. Para ela, o diálogo continua a ser a
melhor arma contra esse tipo de violência, que pode causar efeitos
devastadores em crianças e adolescentes.

A Lei nº 13.185, em vigor desde 2016, classifica o \textit{bullying} como
intimidação sistemática, quando há violência física ou psicológica em
atos de humilhação ou discriminação. A classificação também inclui
ataques físicos, insultos, ameaças, comentários e apelidos pejorativos,
entre outros.

``Os casos de \textit{bullying} começam muito mais silenciosos e, por isso, são
mais graves. Quem sofre a agressão não conta nem na escola nem na
família, mas começa a mudar o comportamento'', explica. De acordo com
ela, queda no rendimento escolar, faltas na escola e mudanças no
comportamento são os sinais mais frequentes apresentados por quem sofre
esse tipo de violência. Por isso, família e escola devem estar sempre
atentos para os sinais que são apresentados pelos jovens.

Os mesmos cuidados, alerta a psicóloga, valem para situações enfrentadas
fora da escola, seja no mundo virtual -- como em casos de \textit{cyberbullying}
--, na vizinhança onde moram ou nos locais que costumam frequentar.

\fonte{Ministério da Educação. Disponível em: http://portal.mec.gov.br/component/tags/tag/34487.
Acesso em: 27 set. 2023.}

\end{myquote}

\num{1} Qual o assunto central do texto?

\reduline{O assunto central do texto é o \textit{bullying}.\hfill}

\num{2} Cite um elemento do texto que traz maior confiabilidade às informações.

\reduline{Discursos de autoridade, exemplos, dados de pesquisas e institutos de
pesquisa conferem credibilidade ao conjunto do texto e às informações nele 
contidas.\hfill}

\num{3} Qual a função do travessão no primeiro parágrafo do texto?

\reduline{A função do travessão é explicar o termo \textit{bullying}.\hfill}

\num{4} Transcreva do texto o trecho em que a especialista oferece recursos
para lidar contra esse tipo de violência. 

\reduline{O trecho solicitado é ``Para ela, o diálogo continua a ser a melhor
arma contra esse tipo de violência, que pode causar efeitos devastadores em
crianças e adolescentes.''\hfill}
% \num{5} Utilize a legenda para classificar os tipos de argumento selecionados:
% % Please add the following required packages to your document preamble:
% % \usepackage[table,xcdraw]{xcolor}
% % If you use beamer only pass "xcolor=table" option, i.e. \documentclass[xcolor=table]{beamer}
% \begin{table}[h!]
% \begin{tabular}{lp{7cm}} %{| >{\columncolor[HTML]{DAE8FC}}l |l|}
% \hline
% \textbf{(I) Argumento por Provas Concretas} & \begin{tabular}[c]{@{}p{7cm}}(\ )\small Especialistas, como a professora de psicologia Ciomara Shcneider,\\ psicanalista de crianças e adolescentes, defendem que pais e escola\\ devem estar atentos ao comportamento dos jovens e manter sempre abertos\\ os canais de comunicação com eles.\end{tabular} \\ \hline
% \textbf{(II) Argumento de Autoridade} & \begin{tabular}[c]{@{}p{7cm}}(\ )\small Um em cada dez estudantes brasileiros é vítima de \textit{bullying} -\\ anglicismo que se refere a atos de intimidação e violência física ou\\ psicológica, geralmente em ambiente escolar. O dado foi divulgado esta\\ semana pelo Programa Internacional de Avaliação de Estudantes (Pisa)\\ 2015.\end{tabular} \\ \hline
% \textbf{(III) Argumento por Exemplificação} & \begin{tabular}[c]{@{}p{7cm}}(\ )\small A Lei nº 13.185, em vigor desde 2016, classifica o \textit{bullying} como \\ intimidação sistemática, quando há violência física ou psicológica em\\ atos de humilhação ou discriminação. A classificação também inclui\\ ataques físicos, insultos, ameaças, comentários e apelidos pejorativos,\\ entre outros.\end{tabular} \\ \hline
% \end{tabular}
% \end{table}
% Please add the following required packages to your document preamble:
% \usepackage[table,xcdraw]{xcolor}
% If you use beamer only pass "xcolor=table" option, i.e. \documentclass[xcolor=table]{beamer}
% \begin{table}[h!]
% \begin{tabular}{|l|l|} %{| >{\columncolor[HTML]{DAE8FC}}l |l|}
% \hline
% \textbf{(I) Argumento por Provas Concretas} & \begin{tabular}[c]{@{}l@{}}(II) Especialistas, como a professora de psicologia Ciomara Shcneider,\\ psicanalista de crianças e adolescentes, defendem que pais e escola\\ devem estar atentos ao comportamento dos jovens e manter sempre abertos\\ os canais de comunicação com eles.\end{tabular} \\ \hline
% \textbf{(II) Argumento de Autoridade} & \begin{tabular}[c]{@{}l@{}}(I) Um em cada dez estudantes brasileiros é vítima de \textit{bullying} -\\ anglicismo que se refere a atos de intimidação e violência física ou\\ psicológica, geralmente em ambiente escolar. O dado foi divulgado esta\\ semana pelo Programa Internacional de Avaliação de Estudantes (Pisa)\\ 2015.\end{tabular} \\ \hline
% \textbf{(III) Argumento por Exemplificação} & \begin{tabular}[c]{@{}l@{}}(III) A Lei nº 13.185, em vigor desde 2016, classifica o \textit{bullying} como \\ intimidação sistemática, quando há violência física ou psicológica em\\ atos de humilhação ou discriminação. A classificação também inclui\\ ataques físicos, insultos, ameaças, comentários e apelidos pejorativos,\\ entre outros.\end{tabular} \\ \hline
% \end{tabular}
% \end{table}

\num{5} Copie do texto o trecho em que a especialista explica outro tipo de
violência associada ao \textit{bullying} que ocorre fora da escola.

\reduline{O trecho solicitado é ``Os mesmos cuidados, alerta a psicóloga, valem
para situações enfrentadas fora da escola, seja no mundo virtual -- como em
casos de \textit{cyberbullying} --, na vizinhança onde moram ou nos locais que costumam frequentar.\hfill}

\num{6} Qual o sinal usado para marcar as falas da especialista? 

\reduline{O sinal usado para marcar as falas da especialista são as aspas.\hfill}

\num{7} No trecho ``De acordo com ela, queda no rendimento escolar, faltas na
escola e mudanças no comportamento são os sinais mais frequentes
apresentados por quem sofre esse tipo de violência.'' O pronome \textbf{ela} se
refere a quem?

\reduline{O pronome se refere à professora de psicologia Ciomara Shcneider.\hfill}

\num{8} A paráfrase é um recurso muito comum em textos de opinião e
argumentativos. Este recurso visa à reescrita de alguma fala ou
citação sem que seja necessária a referência direta. Retire do texto
um exemplo de paráfrase.

\reduline{O trecho a seguir contém paráfrase: ``Para ela, o diálogo continua a ser a
melhor arma contra esse tipo de violência, que pode causar efeitos devastadores
em crianças e adolescentes.''\hfill}

\num{9} Reescreva em forma de paráfrase a seguinte fala da especialista: ``Os 
casos de \textit{bullying} começam muito mais silenciosos e, por isso, são mais graves. 
Quem sofre a agressão não conta nem na escola nem na família, mas começa a mudar 
o comportamento.''

\reduline{A professora de psicologia explica que, por ocorrerem de maneira
silenciosa, os casos de \textit{bullying} podem ser mais graves. A vítima pode
apresentar mudanças de comportamento e pode não comunicar a escola e a
família sobre as agressões.\hfill}

\section*{Treino}

\num{1} Leia o texto a seguir para responder à questão.

\begin{myquote}

\textbf{Celulares: vantagens e desvantagens}

O telefone celular se tornou uma parte essencial de nossas vidas,
revolucionando a maneira como nos comunicamos e interagimos com o mundo ao
nosso redor. No entanto, como qualquer tecnologia, ele possui tanto vantagens
quanto desvantagens significativas.

No que diz respeito às vantagens, os telefones celulares proporcionam uma
comunicação instantânea e conveniente. Eles permitem que as pessoas permaneçam
conectadas em tempo real, independentemente da distância geográfica,
promovendo relacionamentos pessoais e profissionais sólidos. 

No entanto, o uso excessivo de telefones celulares também apresenta
desvantagens. A dependência dos dispositivos móveis pode levar à distração
constante, prejudicando a concentração em tarefas importantes. Além disso, o
uso prolongado de telefones celulares tem sido associado a problemas de saúde.

Portanto, os telefones celulares têm vantagens notáveis em termos de
comunicação e eficiência, mas também apresentam desvantagens, incluindo
distração e impacto na saúde. É essencial encontrar um equilíbrio saudável 
no uso desses dispositivos, garantindo que aproveitemos ao máximo seus 
benefícios sem comprometer nossa qualidade de vida.

\fonte{Texto formulado para este material.}

\end{myquote}

A leitura integral do texto acima permite afirmar que o autor

\begin{escolha}
  
  \item afirma que os telefones celulares devem ser abandonados pelos usuários.
  
  \item desconsidera problemas de saúde causados pelo uso excessivo de celulares.
  
  \item expõe vantagens e desvantagens de usar o celular, sem se posicionar.
  
  \item sugere que o uso dos celulares seja moderado, sem exagero nem descarte.

\end{escolha}

\num{2} Leia o texto abaixo para responder à questão.

\begin{myquote}

O Brasil subiu cinco posições no Índice Global de Inovação (IGI) na comparação
com o ranking de 2022 e agora ocupa o 49º lugar entre 132 países. Após 12 anos
fora do recorte das 50 economias mais bem classificadas no IGI, o Brasil
passou a liderar o ranking dos países da América Latina e Caribe,
ultrapassando pela primeira vez o Chile (52ª). Os dados foram divulgados pela 
Confederação Nacional da Indústria (CNI).

Apesar dos ganhos de posição, a colocação brasileira ainda é considerada aquém 
do potencial do país. Para o presidente da entidade, Robson Braga de Andrade, 
o Brasil tem condições de crescer a cada ano no ranking, por meio de investimentos 
e políticas direcionadas à ciência, tecnologia e inovação. ``Temos um potencial muito
inexplorado para melhorar o nosso ecossistema de inovação, atingir o objetivo de 
integrar os setores científico e empresarial e, consequentemente, promover maior 
inovação'', afirma.

\fonte{Agência Brasil. Brasil lidera ranking de inovação na América Latina. 
Disponível em: https://agenciabrasil.ebc.com.br/economia/noticia/2023-09/brasil-lidera-ranking-de-inovacao-na-america-latina. Acesso em: 27 set. 2023. 
com adaptações}

\end{myquote}

No final do segundo parágrafo, as aspas serviram para:

\begin{escolha}

  \item complementar afirmação anterior referente ao potencial de crescimento do Brasil.

  \item explicar por que o Brasil ficou tanto tempo fora das boas classificações.

  \item celebrar a subida de posição do Brasil no Índice Global de Inovação.

  \item contradizer a afirmação de que o Brasil ainda pode crescer no ranking.

\end{escolha}


\num{3} Leia o texto a seguir para responder à questão. 

\begin{myquote}

\begin{figure}[H]
\centering
\includegraphics[width=0.95\textwidth]{./imgSAEB_7_POR/media/image40.png}
\end{figure}
%Fonte pixabay https://pixabay.com/pt/photos/rafting-jangada-de-águas-brancas-293542/

Segundo o Instituto Brasileiro de Turismo (Embratur), algumas das exigências de
segurança do Turismo de Aventura elaboradas e implementadas no Brasil acabaram 
servindo de exemplo para regras de segurança depois adotadas em vários países.

``Isso significa que o Brasil está na vanguarda desta discussão há tempos. E
que, paralelamente ao trabalho de promoção que é feito no exterior, o país vem
executando uma série de ações importantes para atrair turistas de aventura e
esportivos'', ressalta Leonardo Persi, da Embratur, ao tratar dos planos para tentar
aumentar a captação de clientes estrangeiros, incluindo os turistas de
aventura e esportivos. Para tanto, a autarquia cogita inclusive se valer da
imagem dos atletas profissionais brasileiros de destaque no exterior e outros
formadores de opinião.

\fonte{Agência Brasil. Turismo de aventura: associação defende fiscalização de prática segura. 
Disponível em: https://agenciabrasil.ebc.com.br/geral/noticia/2023-09/turismo-de-aventura-associacao-defende-fiscalizacao-de-pratica-segura.
Acesso em: 27 set. 2023.}
\end{myquote}

Para o coordenador da Embratur, o Brasil está na vanguarda da discussão sobre 
segurança do Turismo de Aventura porque 

\begin{escolha}

  \item divulga no exterior as atrações nacionais dessa modalidade de turismo.

  \item tem captado com sucesso clientes estrangeiros que se interessam pelo Brasil.

  \item promove imagens de atletas nacionais que são conhecidos no exterior.

  \item formulou exigências de segurança que serviram de modelo para outros países.

\end{escolha}


\chapter{Informações implícitas no texto: fato ou opinião}
\markboth{Módulo 5}{}

\section*{Habilidades do SAEB}

\begin{itemize} 

  \item Inferir informações implícitas em distintos textos.

  \item Distinguir fatos de opiniões em textos.

\end{itemize}

\subsection{Habilidades da BNCC}

\begin{itemize} 

  \item EF67LP04.

\end{itemize} 

\conteudo{Com a rápida expansão da tecnologia digital nas últimas décadas, a
quantidade de informação disponível e acessível está cada vez maior. Em nenhum
outro momento da história houve tanto acesso a vídeos, imagens, propagandas,
opiniões, artigos acadêmicos, notícias e reportagens, tutoriais, dentre tantos
outros conteúdos.

É preciso ter em mente, contudo, que a crescente quantidade de informações não
garante a qualidade dos conteúdos gerados e distribuídos de maneira
vertiginosa pela internet. Mais do que nunca precisamos aprender a distinguir
fatos de opiniões, fontes confiáveis e fontes questionáveis, argumentos
sólidos de opiniões sem fundamento.

Portanto, saber identificar em um texto as marcas de subjetividade e
objetividade, a intertextualidade e credibilidade das informações apresentadas
é muito importante. Para isso, é preciso conhecer formas de verificação da
informação recebida, o suporte conceitual dado a elas, a existência ou não de
evidências, as fontes e os possíveis interesses por parte daqueles que
divulgam textos, com argumentos e fatos, ou com opiniões e sugestões. A
habilidade de questionar informações nunca foi tão necessária como é nos dias
de hoje.}

% \coment{Professor(a), chame a atenção dos estudantes para as diferenças entre
% opiniões e fatos. Mostre aos alunos como opiniões podem ser convincentes
% por apelarem para as percepções e dilemas pessoais. Explique que
% opiniões são importantes e tem sua validade, porém não podem ser
% transpostas para a construção de um argumento sólido. Mostre como os
% fatos são mais convincentes e aponte as formas de construção de
% argumentos baseados em fatos e faça as distinções quanto à construção de
% argumentos baseados em opiniões. Chame a atenção para a importância
% destas distinções para a vida prática.}

\section*{Atividades}

% \num{1} Leia as afirmações abaixo e assinale (F) para fatos e (O) para
%   opiniões na coluna esquerda da tabela. 

% \begin{table}[h!]
% \begin{tabular}{cc}
% \hline
% \textbf{\begin{tabular}[c]{@{}c@{}}Fato (F) \\ ou \\ Opinião (O)\end{tabular}} & \textbf{Afirmação} \\ \hline
% \rosa{O} & A melhor hora para dormir é o começo da tarde \\ \hline
% \rosa{F} & É saudável dormir cerca de 8 horas por dia \\ \hline
% \rosa{F} & Automedicar-se é prática arriscada  \\ \hline
% \rosa{O} & Os remédios são baratos no Brasil \\ \hline
% \end{tabular}
% \end{table}

%Gabarito caso seja necessário \coment{O, F, F, O}

\num{1} Descreva algumas diferenças entre fatos e opiniões.

\reduline{Fatos podem ser verificados de maneira objetiva; opiniões são
subjetivas. Fatos, geralmente, são sustentados por fontes seguras, por 
muitas pessoas que estudam o assunto; opiniões, por sua vez, se baseiam apenas nas percepções e crenças
do sujeito, e, mesmo que sejam compartilhadas por muitas pessoas, não
podem ser consideradas como conhecimento formal.\hfill}

Leia o texto abaixo para responder às questões de 2 a 9.

\begin{myquote}

\begin{figure}[H]
\centering
\includegraphics[width=0.95\textwidth]{./imgSAEB_7_POR/media/image41.png}
\end{figure}
%Fonte: pixabay https://pixabay.com/pt/photos/uma-conversa-por-telefone-conexão-5025656/

A franquia de filmes Barbie é uma série de produções cinematográficas
protagonizadas pela icônica boneca Barbie. Esses filmes, destinados
principalmente ao público infantil, têm sido lançados ao longo das últimas
décadas, com histórias diversas que exploram temas de amizade, aventura,
romance e superação de desafios.

Uma das características notáveis dos filmes Barbie é a sua abordagem
positiva e encorajadora. Eles costumam promover mensagens de empoderamento
feminino, resiliência e trabalho em equipe. Além disso, as animações e designs
de personagens são geralmente bem elaborados e coloridos, cativando o
público-alvo de crianças.

Alguns críticos apontam, contudo, que a narrativa muitas vezes é simples e
previsível, o que pode não atrair tanto um público mais maduro. Além disso, a
série de filmes Barbie já enfrentou algumas controvérsias relacionadas à
representação de um modelo tradicional de mulher, embora esforços tenham sido feitos
para modernizar a personagem ao longo do tempo, especialmente no filme 
lançado em 2023.

Em última análise, os filmes Barbie são uma escolha sólida para famílias com
crianças que procuram entretenimento alegre e mensagens positivas. Embora
possam não ser do gosto de todos, eles têm um lugar especial na cultura
popular, proporcionando diversão e lições valiosas para as gerações mais
jovens.

\fonte{Texto formulado para este material.}

\end{myquote}

\num{2} Qual o assunto central do texto?

\reduline{O assunto central do texto é a franquia de filmes da boneca Barbie.\hfill}

\num{3} O texto pode ser lido como pertencente a qual gênero textual?

\reduline{O texto apresenta características de resenha crítica.\hfill}

\num{4} No primeiro parágrafo predominam fatos ou opiniões? Justifique sua resposta.

\reduline{No primeiro parágrafo predominam fatos, como se verifica em trechos como 
``Esses filmes, destinados principalmente ao público infantil, têm sido lançados 
ao longo das últimas décadas''.\hfill}

\num{5} Identifique fatos e opiniões do segundo parágrafo.

\reduline{Na opinião do autor, os filmes Barbie são ``notáveis'' e sua abordagem é
``positiva e encorajadora'', e ``as animações e designs de personagens são geralmente 
bem elaborados''. A promoção de ``mensagens de empoderamento feminino, resiliência e 
trabalho em equipe'' e o sucesso entre crianças, por sua vez, são fatos.\hfill}

\num{6} No terceiro parágrafo, existe uma expressão que indica que o autor vai
relativizar as opiniões positivas do parágrafo anterior. Que expressão é essa? 

\reduline{Contudo.\hfill}

\num{7} Quais são as principais críticas aos filmes Barbie? 

\reduline{O autor explica que há críticos que afirmam que os filmes Barbie
são simples e previsíveis, além de insistirem na representação de um modelo 
tradicional de mulher.\hfill}

% \num{9} Sobre fatos e opiniões, assinale (V) para verdadeiro e (F) 
% para falso na coluna esquerda da tabela abaixo.

% \begin{table}[h!]
% \begin{tabular}{cc}
% \hline
% \textbf{\begin{tabular}[c]{@{}c@{}}Verdadeiro (V) \\ ou \\ Falso (F)\end{tabular}} & \textbf{Afirmação} \\ \hline
% \rosa{F} & Fatos são baseados em sentimentos e impressões \\ \hline
% \rosa{F} & Opiniões são suficientes para adquirir conhecimento \\ \hline
% \rosa{F} & Opiniões são baseadas em questões objetivas \\ \hline
% \rosa{V} & Opiniões são baseadas em sentimentos e impressões \\ \hline
% \rosa{V} & Fatos se apoiam em evidências \\ \hline
% \end{tabular}
% \end{table}

%Gabarito caso seja necessário \coment{F, F, F, V, V}

\num{8} No terceiro parágrafo, existe um trecho no qual o autor se mostra 
favorável aos filmes Barbie, apesar das críticas que eles sofrem. Que trecho é 
esse?

\reduline{No final do terceiro parágrafo, o autor afirma que esforços foram feitos
para modernizar a Barbie ao longo do tempo e cita o filme lançado em 2023 como
exemplo desse esforço.\hfill}

\num{9} De acordo com a leitura do último parágrafo, podemos afirmar que o autor
é favorável ou desfavorável aos filmes Barbie? Justifique sua resposta. 

\reduline{O autor é claramente favorável aos filmes Barbie. Para ele, trata-se de
``escolha sólida'' para famílias, com ``entretenimento alegre e mensagens positivas'' e
``diversão e lições valiosas para as gerações mais jovens''.\hfill}

\section*{Treino}

\num{1} Leia o texto abaixo para responder à questão.

\begin{myquote}

\begin{figure}[H]
\centering
\includegraphics[width=0.95\textwidth]{./imgSAEB_7_POR/media/image42.png}
\end{figure}
%Fonte pixabay https://pixabay.com/pt/photos/cinema-quarto-filme-2502213/

O retorno do público aos cinemas é essencial para a experiência cultural como
um todo. Os cinemas proporcionam um ambiente único, onde o público pode
mergulhar completamente em uma história, cercado por uma tela grande e som
imersivo. É uma experiência muito diferente de assistir a um filme 
no conforto de nossas próprias casas, em que desviamos nossa atenção da tela 
por muitos motivos. Além disso, os cinemas desempenham um papel importante 
na criação de comunidades -- especialmente depois do isolamento social imposto 
pela pandemia do coronavírus -- e no fortalecimento dos laços sociais, permitindo 
que as pessoas compartilhem emoções e experiências em conjunto. O retorno seguro 
do público aos cinemas não apenas enriquece nosso repertório como estimula o convívio
saudável com outras pessoas.

\fonte{Texto formulado para este material.}

\end{myquote}

De acordo com o texto, devemos retornar à prática de assistir aos filmes 
no cinema porque esses espaços proporcionam

\begin{escolha}

  \item mais conforto e convívio do que nossas próprias casas.

  \item experiência única e compartilhamento de dados pessoais. 

  \item concentração no filme e convívio com outras pessoas.

  \item qualidade de som e imagem e afastamento das pessoas. 

\end{escolha}

\num{2} Leia o texto abaixo para responder à questão. 

\begin{myquote}

\begin{figure}[H]
\centering
\includegraphics[scale=0.25]{./imgSAEB_7_POR/media/image39.png}
\end{figure} 
%Fonte Pixabay https://pixabay.com/pt/photos/smartphone-celular-1894723/

Para muitos especialistas, os pais devem controlar o uso excessivo de celulares 
pelos filhos. Em primeiro lugar, o uso desenfreado de dispositivos
móveis pode prejudicar o desenvolvimento saudável das crianças, limitando o
tempo que elas passam em atividades físicas, interações sociais presenciais e
estudos. Além disso, a exposição constante a conteúdos inadequados na internet
pode ter um impacto negativo no desenvolvimento cognitivo e emocional das
crianças. Em segundo lugar, o controle dos pais ajuda a proteger as crianças
de perigos online, como \textit{cyberbullying}, predadores virtuais e acesso a
conteúdos impróprios. Por último, mas não menos importante, o uso excessivo de
celulares pode prejudicar o sono das crianças, afetando negativamente sua
saúde física e mental. É fundamental que os pais monitorem e
estabeleçam limites claros para o uso de celulares, garantindo um ambiente
saudável e seguro para o desenvolvimento de seus filhos.

\fonte{Texto formulado para este material.}

\end{myquote}

Segundo o texto acima, os pais devem

\begin{escolha}

  \item proibir que seus filhos usem celulares, para protegê-los.
  
  \item limitar o tempo dos filhos em atividades presenciais.
  
  \item impedir o exagero dos filhos no uso de celulares.
  
  \item proteger os filhos com a suspensão do uso de celulares.

\end{escolha}

\num{3} Leia o texto abaixo para responder à questão.

\begin{myquote}

\begin{wrapfigure}{r}{0.4\textwidth}  % 'l' para alinhar à esquerda, 'r' para alinhar à direita
  \centering
  \includegraphics[width=0.35\textwidth]{./imgSAEB_7_POR/media/image43.png}
  \caption{\textit{O escritor Lima Barreto}}
\end{wrapfigure}

A vida e a obra de Lima Barreto (1881-1922) são marcadas por sua contribuição
literária e seu compromisso com a denúncia das injustiças sociais e do racismo
no Brasil. Ele foi um importante escritor da literatura brasileira, autor de
obras como ``Triste Fim de Policarpo Quaresma'' e ``Memórias do Escrivão
Isaías Caminha''. Sua escrita foi influenciada pela sua experiência pessoal
como um negro de classe trabalhadora, e ele abordou questões sociais e raciais
com um olhar crítico e perspicaz.

A atualidade de sua obra reside na persistência dos temas que ele abordou. 
A discussão sobre desigualdades sociais, racismo e a luta por justiça
continua relevante na sociedade brasileira contemporânea. A obra de Lima
Barreto ainda é importante para entendermos questões
sociais e culturais do Brasil e promove reflexões sobre a necessidade de
mudanças e de uma sociedade mais justa e igualitária. 

% \end{minipage}
% \hfill
% \begin{minipage}{0.5\textwidth}
%   \centering
%   \includegraphics[width=\textwidth]{./imgSAEB_7_POR/media/image43.png}
  %\caption{Lima Barreto}
%\end{minipage}

% \begin{figure}[H]
% \centering
% \includegraphics[scale=0.25]{./imgSAEB_7_POR/media/image43.png}
% \end{figure} 

%Fonte wikipedia https://pt.wikipedia.org/wiki/Lima_Barreto_%28escritor%29#/media/Ficheiro:Lima_Barreto_-_detalhe_da_ficha_da_primeira_interna%C3%A7%C3%A3o_manicomial.png

%\fonte{Texto formulado para este material.}

\end{myquote}

De acordo com o autor do texto acima, a obra de Lima Barreto segue atual 
porque 

\begin{escolha}

  \item a contribuição literária desse autor é marcada pela sua vida pessoal.

  \item ele foi um importante escritor da literatura brasileira de seu tempo.
  
  \item os temas nela abordados continuam presentes na sociedade brasileira.
  
  \item contribuiu para o Brasil se tornar uma sociedade mais justa e igualitária.

\end{escolha}


\chapter{Humor e as ferramentas da crítica}
\markboth{Módulo 6}{}

\section*{Habilidades do SAEB}

\begin{itemize}

  \item Inferir, em textos multissemiótico, efeitos de humor, ironia e/ou
  crítica.

\end{itemize}

\subsection{Habilidades da BNCC}

\begin{itemize}

  \item EF69LP03, EF69LP05.

\end{itemize}

\conteudo{O que faz a gente rir? Você já reparou que tem algumas piadas ou
textos que nos fazem gargalhar? Existem muitos recursos e maneiras de
relacionar ideias para provocar humor. Muitas vezes, apenas a inversão do
sentido de uma palavra, um trocadilho, uma paródia ou até mesmo um desenho
podem provocar o riso e a reflexão.

Atualmente, os memes representam muito bem a forma como imagens e poucas
palavras podem garantir efeitos de humor, críticas e ironias. Antes dos memes,
associados diretamente ao surgimento da internet, charges e tirinhas de jornal
já causavam esses efeitos.

Para que se possa perceber ironia ou crítica em determinado texto, é
necessário algum conhecimento prévio, ou seja, muitas vezes uma tirinha pode
se referir a um problema social, a um acontecimento recente ou a alguma
atitude do senso comum. Por isso, memes tirinhas e
charges são recursos que dialogam com amplos contextos, embora sejam muito
simples, diretos e curtos. Também por esse motivo é comum que charges e
tirinhas publicadas em jornais e revistas digitais podem perder a graça
se envelhecerem.}

% \coment{Professor(a), estimule os estudantes a pensarem em exemplos. O uso de
% memes pode ser bastante motivador pois traz a definição dos termos
% estudados para um recurso conhecido pelos estudantes. Estimule-os a
% pensar quais são os recursos utilizados e como eles se articulam para
% produzir humor ou ironia.}

\section*{Atividades}

Leia a tirinha abaixo e responda às questões de 1 a 6.

\begin{figure}[h!]
\centering\includegraphics[width=\textwidth]{./imgSAEB_7_POR/media/image4.png}
\end{figure}

\num{1} No meme acima pode-se perceber que o humor está presente. Qual palavra 
foi usada para produzir esse efeito?

\reduline{A forma verbal ``vendo'' é utilizada para obter o efeito do 
humor.\hfill} 

\num{2} Por que o uso dessa palavra produz efeito de humor?

\reduline{No contexto, a palavra ``vendo'' pode ter assumir
sentidos diferentes, que causam o efeito de humor.\hfill}

\num{3} Quais as possíveis interpretações da palavra usada para produzir
efeito de humor no meme?

\reduline{A forma verbal ``vendo'' é entendida pela pessoa que faz a pergunta
como primeira pessoa do singular do presente do indicativo do verbo
\textit{vender}, isto é: para quem faz a pergunta, a criança \textit{está
vendendo} o pôr do sol. Ela, por sua vez, quer dizer que está
\textit{observando} o pôr do sol; neste sentido, ``vendo'' é gerúndio do verbo
\textit{ver}.\hfill}

% \num{4} Em qual sentido a forma verbal ``vendo'' está sendo usada pela
% criança?

% \reduline{A criança quer dizer que está 
% \textit{observando} o pôr do sol; neste sentido, ``vendo'' é gerúndio do
% verbo \textit{ver}.\hfill} 

% \num{5} Em qual sentido está sendo compreendida a forma verbal ``vendo''
% pela pessoa que faz a pergunta?

% \reduline{A forma verbal ``vendo'' é entendida pela pessoa que faz a 
% pergunta como primeira pessoa do singular do presente do indicativo 
% do verbo \textit{vender}, isto é: para quem faz a pergunta, a criança 
% \textit{está vendendo} o pôr do sol.\hfill}

\num{4} Em qual balão a questão das diferentes interpretações da 
forma verbal se esclarece?

\reduline{As diferentes interpretações da forma verbal se esclarecem 
no último balão.\hfill}

Analise a imagem abaixo e responda às questões de 5 a 8. 

\begin{figure}[h!]
\centering\includegraphics[width=5in]{./imgSAEB_7_POR/media/image5.png}
\end{figure}

\num{5} Considerando a imagem, é possível
inferir que a fala do morador em situação de rua alude a um importante
documento da democracia brasileira. Que documento é
esse?

\reduline{É possível inferir que a fala do morador em situação de rua
alude à Constituição Federal de 1988.\hfill}

\num{6} Qual a função deste documento e o que ele pretende garantir?

\reduline{A Constituição Federal trata das questões mais importantes 
para a manutenção da democracia e pretende esclarecer os direitos 
e deveres de todos os âmbitos da sociedade.\hfill}

\num{7} Qual o efeito de sentido obtido por meio do meme?

\reduline{O efeito de sentido obtido por meio do meme é a ironia.\hfill}

\num{8} De que forma a imagem se articula com o texto para produzir
o efeito de sentido?

\reduline{No meme, uma pessoa em situação de rua reflete sobre um dos direitos 
fundamentais garantidos pela Constituição Federal de 1988: o direito à
moradia. A \textit{declaração} desse direito é rigorosamente oposta à 
\textit{situação concreta} em que se encontra o morador. Essa oposição
compõe a ironia, que consiste em dizer o inverso do que se quer afirmar.
Evidentemente, o autor do meme pretende evidenciar o contraste entre discurso e
prática, isto é, entre a afirmação dos direitos na Constituição de 1988 e a 
desigualdade social.\hfill} 

\section*{Treino}

\num{1} Leia o trecho abaixo para responder à questão. 

\begin{myquote}

% \begin{figure}[H]
% \centering
% \includegraphics[scale=0.25]{./imgSAEB_7_POR/media/image44.png}
% \end{figure}
\begin{minipage}{0.5\textwidth}
Há meio século, os escravos fugiam com frequência. Eram muitos, e nem todos
gostavam da escravidão. Sucedia ocasionalmente apanharem pancada, e nem todos
gostavam de apanhar pancada. Grande parte era apenas repreendida; havia alguém de
casa que servia de padrinho, e o mesmo dono não era mau; além disso, o sentimento da
propriedade moderava a ação, porque dinheiro também dói. A fuga repetia-se,
entretanto. Casos houve, ainda que raros, em que o escravo de contrabando, apenas
comprado no Valongo, deitava a correr, sem conhecer as ruas da cidade.
\end{minipage}
\hfill
\begin{minipage}{0.6\textwidth}
  \centering
  \includegraphics[width=\textwidth]{./imgSAEB_7_POR/media/image44.png}
  %\caption{Legenda da imagem}
  %\label{fig:exemplo}
  %Fonte wikipedia https://pt.wikipedia.org/wiki/M%C3%A1scara_de_Flandres#/media/Ficheiro:Jacques_Etienne_Arago_-_Castigo_de_Escravos,_1839.jpg
\end{minipage}


\fonte{Machado de Assis. Pai contra Mãe. Disponível em: 
http://www.dominiopublico.gov.br/download/texto/bv000245.pdf.
Acesso em: 22 mai. 2023.}

\end{myquote}

No fragmento do conto ``Pai contra Mãe'', o narrador de Machado de Assis faz a crítica
à escravidão brasileira por meio de:

\begin{escolha}

  \item repetições que resultam em ironia.
  
  \item relato objetivo da realidade.
  
  \item observações nostálgicas. 
  
  \item valorização dos escravizados. 

\end{escolha}

\num{2} Observe a imagem abaixo para responder à questão.


\includegraphics[width=5.90551in,height=4.43056in]{./imgSAEB_7_POR/media/image6.png}

\fonte{Prefeitura de Santa Quitéria. Dicas de como evitar a proliferação 
do foco do mosquito Aedes Aegypti. Disponível em: 
https://www.santaquiteria.ce.gov.br/informa.php?id=1018.
Acesso em: 22 mai. 2023.}

Na imagem acima observa-se uma campanha para evitar a proliferação do
mosquito da dengue. A escolha das palavras e imagens tem como objetivo
sensibilizar a população. Assinale a alternativa que contém as palavras
usadas para a produção de efeito de sentido na campanha. 

\begin{escolha}
  
  \item Proliferação e foco.
  
  \item Veja e evitar.
  
  \item Fuja e elimine.
  
  \item Alvo e foco. 

\end{escolha}

\num{3} Leia o texto abaixo para responder à questão.

\includegraphics[width=\textwidth]{./imgSAEB_7_POR/media/image7.png}

A respeito do meme acima, pode-se afirmar que

\begin{escolha}
    
    \item ironiza a educação por meio do duplo sentido da palavra ``casa''.
    
    \item critica a falta de moradia com o uso da palavra ``casa''.
    
    \item contém propaganda subliminar em benefício da construção de apartamentos.
    
    \item questiona a qualidade de vida dos moradores de apartamentos.

\end{escolha}

\chapter{Parcialidade nos textos jornalísticos}
\markboth{Módulo 7}{}

\section*{Habilidades do SAEB}

\begin{itemize}

  \item Analisar marcas de parcialidade em textos jornalísticos.

  \item Avaliar diferentes graus de parcialidade em textos jornalísticos.

  \item Avaliar a fidedignidade de informações sobre um mesmo fato divulgado 
  em diferentes veículos e mídias.

\end{itemize}

\subsection{Habilidades da BNCC}

\begin{itemize}

  \item EF07LP02, EF67LP03, EF67LP04.

\end{itemize}

\conteudo{A função ideal do jornalismo é informar de forma imparcial e objetiva
fatos e notícias para oferecer informações e ferramentas para a
formação de opinião. No entanto, nem sempre é possível separar os fatos
das opiniões, pois todo texto, em alguma medida, é produzido a partir
das perspectivas pessoais do autor ou do veículo de comunicação que o
divulga. Por isso, é importante que os leitores aprendam a analisar
marcas de parcialidade em textos jornalísticos, a fim de avaliar o grau
de confiabilidade das informações que estão recebendo.

Uma das formas de identificar as marcas de parcialidade em textos
jornalísticos é perceber valores expressos pelo uso de
adjetivos, advérbios e na forma como são construídos os argumentos nos
textos de divulgação de notícias e acontecimentos. Também a escolha de
fontes e a seleção editorial dos assuntos e temas a serem tratados podem
indicar interesses e, portanto, certo grau de parcialidade. A escolha
dos pontos de vista expressos em uma notícia ou reportagem revela muito
sobre as intenções do autor ou do veículo que divulga o texto. Dessa maneira, 
comparar fontes e analisar de forma atenta os contextos em que as
informações são divulgadas em cada veículo de informação pode ser uma
forma eficaz de avaliar a confiabilidade e fidedignidade de determinada
notícia ou reportagem, porque a linha editorial e a relação dos
veículos de comunicação com empresas e grupos políticos ou econômicos
podem revelar possíveis interesses e pontos de vista. 

Portanto, aprender
a avaliar tais marcas de parcialidade e comparar diferentes formas de
divulgação de notícias em diversos veículos e texto é uma habilidade
importante para que o leitor possa formar uma opinião de forma reflexiva
e autônoma.}

% \coment{Professor(a), faça um exercício de reflexão com os estudantes,
% questionando como percebem os traços e interesses presentes em algumas
% manchetes, e chamadas. Chame a atenção para veículos sensacionalistas,
% para determinados temas e assuntos abordados e veiculados com a intenção
% de gerar polêmicas, discussões ou busca de soluções. Converse com os
% estudantes sobre os diversos tipos de textos jornalísticos e estimule-os
% a refletir sobre como percebem as marcas de parcialidade. Instrua-os a
% pesquisar e questionar os veículos de comunicação trazendo para a
% discussão quais podem ser os interesses por trás dos recursos e
% elementos multissemióticos presentes nas informações que consomem. Cite
% a monetização de determinados conteúdos, explique como é importante
% analisar quem são os financiadores e quais as propagandas e empresas
% relacionadas aos veículos de informação que conhecem.}

\section*{Atividades}

% Conversei com Felipe Augusto em 28/9/23, às 10h58; decidimos jogar tudo que está abaixo fora
% Analise as duas notícias abaixo e responda às questões.

% \begin{myquote}

% \textbf{Texto I}: Cinco desafios para a economia mundial em 2023

% Se 2022 foi um ano difícil para a economia global, 2023 promete ser
% ainda pior, com uma recessão a caminho.

% Espera-se que 2023 seja o terceiro ano com o pior crescimento econômico
% global neste século, atrás de 2009, quando a crise financeira global
% causou a Grande Recessão, e 2020, quando os lockdowns da covid-19
% virtualmente paralisaram a economia global.

% Analistas esperam que as principais economias do mundo, incluindo os
% Estados Unidos e o Reino Unido, assim como a zona do euro, entrem em
% recessão este ano, já que os bancos centrais continuam aumentando as
% taxas de juros para moderar a demanda por bens de consumo e serviços, em
% um esforço para conter a inflação. 

% \fonte{Ashutosh Pandey. DW. Cinco desafios para a economia mundial em 2023.
% Disponível em: https://www.dw.com/pt-br/cinco-desafios-para-a-economia-mundial-em-2023/a-64264182.
% Acesso em: 22 mai. 2023. com adaptações.}

% \end{myquote}

% \begin{myquote}

% \textbf{Texto II}:Por que a inflação mundial deve cair em 2023 (e por que a
% notícia não é tão boa)

% Provavelmente o pior em termos de inflação já passou.

% Pelo menos este é o consenso entre os economistas e as principais
% organizações econômicas como o Fundo Monetário Internacional (FMI) ou o
% Banco Mundial depois que a maioria dos países do mundo experimentou, em
% 2022, aumentos de preços não vistos em quatro décadas.

% Não há dúvida de que a inflação continuará a pesar no bolso de milhões
% de cidadãos em 2023, mas deve registrar uma queda lenta nos próximos 12
% meses.

% Quando esse período terminar, o FMI espera que a inflação mundial caia
% para 4,7\%, pouco menos da metade do nível atual. 

% \end{myquote}

% \fonte{Cristina J. Orgaz. BBC News Brasil. Por que a inflação mundial deve cair em 2023 (e por que a notícia não é tão boa). 
% Disponível em: https://www.bbc.com/portuguese/internacional-64145595.
% Acesso em: 22 mai. 2023.}

% \num{1} Qual é o fato central relatado nas notícias?

% \reduline{O fato central das duas notícias é a crise econômica de 2023.\hfill}

% \num{2} Quais são as semelhanças entre os fatos noticiados sobre o tema?

% \reduline{As duas notícias tratam da recessão e da influência da queda da
% inflação na crise econômica mundial.\hfill}

% \num{3} Quais são as diferenças entre os fatos noticiados sobre o tema?

% \reduline{No primeiro texto, o autor prevê, em 2023, a continuidade da crise
% econômica. O autor segundo sinaliza, por sua vez, uma queda na inflação.\hfill} 

% \num{4} Em qual das duas notícias a crise mundial é tratada com mais otimismo?
% Copie do texto um trecho que justifique sua resposta.

% \reduline{O Texto II é mais otimista, como se verifica no trecho 
% ``Provavelmente o pior em termos de inflação já passou''.\hfill}

% \num{5} Na tabela abaixo, relacione os títulos numerados à esquerda
% com as linhas finas à direita.

% \begin{table}[h!]
% \begin{tabular}{|c|c|c|c|}
% \hline
% \textbf{} & \textbf{Título} & \textbf{} & \textbf{Linha fina} \\ \hline
% \textbf{1} & \begin{tabular}[c]{@{}c@{}}Estudiosos alertam para os riscos de doenças \\ cardiovasculares e o sedentarismo\end{tabular} & \rosa{1} & \begin{tabular}[c]{@{}c@{}}Análise de hábitos de pacientes com problemas cardiovasculares \\ traz novas informações sobre como prevenir doenças crônicas\end{tabular} \\ \hline
% \textbf{2} & Casos de dengue preocupam secretarias de saúde & \rosa{3} & \begin{tabular}[c]{@{}c@{}}Associação brasileira de pediatria publica estudo sobre os impactos\\  dos jogos e redes sociais na vida de adolescentes\end{tabular} \\ \hline
% \textbf{3} & \begin{tabular}[c]{@{}c@{}}O perigo está em casa: Pediatras alertam sobre o uso \\ excessivo de telas por crianças e adolescentes\end{tabular} & \rosa{4} & \begin{tabular}[c]{@{}c@{}}Ministério da saúde lança cartilha para informar a população \\ sobre as formas de contágio da doença\end{tabular} \\ \hline
% \textbf{4} & Covid-19: Informar para proteger & \rosa{2} & \begin{tabular}[c]{@{}c@{}}Alerta foi dado às secretarias de saúde de todo país, especialmente \\ nas regiões em que se inicia o período de chuvas\end{tabular} \\ \hline
% \end{tabular}
% \end{table}
% %Gabarito \coment{1, 3, 4, 2}

% \num{6} Sobre a parcialidade dos textos jornalísticos, assinale V para as
% afirmações verdadeiras e F para as afirmações falsas na tabela abaixo.

% \begin{table}[h!]
% \begin{tabular}{|c|c|}
% \hline
% \textbf{\begin{tabular}[c]{@{}c@{}}Verdeiro (V) \\ ou Falso (F)?\end{tabular}} & \textbf{Afirmações} \\ \hline
% \rosa{V} & \begin{tabular}[c]{@{}c@{}}A imparcialidade no jornalismo é importante pois promove \\ a formação de opinião a partir dos fatos apresentados na reportagem\end{tabular} \\ \hline
% \rosa{F} & \begin{tabular}[c]{@{}c@{}}O princípio da imparcialidade no jornalismo significa que o jornalista \\ não deve apresentar os diferentes pontos de vista e deixar que o público \\ forme sua própria opinião\end{tabular} \\ \hline
% \rosa{V} & \begin{tabular}[c]{@{}c@{}}A imparcialidade no jornalismo pode levar a erros de interpretação e \\ compreensão dos fatos por parte do público, prejudicando a credibilidade da reportagem do jornalista\end{tabular} \\ \hline
% \rosa{V} & \begin{tabular}[c]{@{}c@{}}A falta de apresentação de fontes e dados confiáveis pode ser uma \\ característica de parcialidade por parte do jornalista ou do veículo de \\ comunicação colocando em dúvida a qualidade das notícias veiculadas\end{tabular} \\ \hline
% \end{tabular}
% \end{table}

% %\coment{V, F, V, V}

% \num{7} Leia os exemplos abaixo e sublinhe as marcas de parcialidade:

% \begin{escolha}

%   \item Polícia age com violência contra manifestantes pacíficos.

%   \item Deputado faz discurso inflamado em defesa dos direitos dos
% trabalhadores.

%   \item Falas preconceituosas de senadores exibem a impunidade da classe
% política no Brasil.

% \end{escolha}

% % \coment{a) Polícia age com \uline{violência} contra manifestantes
% % \uline{pacíficos}.

% % b) Deputado faz discurso \uline{inflamado} em defesa dos direitos dos
% % trabalhadores.

% % c) Falas \uline{preconceituosas} de senadores exibem a
% % \uline{impunidade} da classe política no Brasil.}

% \num{8} Quais aspectos devem ser levados em conta para questionar a
% parcialidade ou imparcialidade de uma notícia

% \reduline{Os aspectos que devem ser levados em conta são: a exposição de
% mais de um ponto de vista sobre o mesmo tema, a citação de dados e 
% pesquisas de fontes confiáveis e o uso ou não de adjetivos que qualifiquem
% os fatos apresentados. É importante também a checagem da veracidade dos 
% dados apresentados e o questionamento sobre os contextos em que ocorrem os
% fatos e se há interesses políticos ou econômicos por parte do veículo de
% comunicação que divulga a notícia.\hfill}

% \num{9} Analise as duas afirmações a seguir e responda qual das duas apresenta
% maior grau de parcialidade. Justifique sua resposta

% \begin{itemize}
 
%  \item Policiamento nas cidades não gera mais segurança, é o que afirmam
% especialistas

%   \item Policiamento nas cidades é a principal medida de segurança anunciada pela prefeitura.

% \end{itemize}

% \reduline{A primeira afirmação apresenta maior grau de parcialidade, pois
% apresenta uma afirmação categórica amparada em afirmação de especialistas. 
% A segunda é uma sentença menos parcial, embora contenha o adjetivo ``principal'' 
% para referir-se ao fato noticiado.\hfill}

% \num{10} De que forma a apresentação de imagens, gráficos e tabelas auxilia 
% na formação de opinião de leitores?

% \reduline{A apresentação de imagens, gráficos e tabelas confere 
% clareza e credibilidade à notícia, pois esses recursos podem ser
% interpretados, questionados e verificados.}

Leia o texto abaixo para responder às questões de 1 a 5.

\begin{myquote}

\textbf{Maioria dos futuros professores não conclui estágio em escolas}

\begin{figure}[H]
\centering
\includegraphics[width=0.95\textwidth]{./imgSAEB_7_POR/media/image45.png}
\end{figure}
%Fonte Pixabay https://pixabay.com/pt/photos/alunos-escola-primaria-cidade-laos-1177711/

A maior parte dos formandos em licenciaturas no Brasil não cumpre a carga
horária mínima exigida no estágio obrigatório. Além disso, cerca de um a cada
dez futuros professores sequer fez o estágio. Os dados são do último Exame
Nacional de Desempenho dos Estudantes (Enade), de 2021, e foram compilados
pelo Todos pela Educação, com exclusividade para a Agência Brasil.

O estágio obrigatório é um período em que os estudantes de licenciaturas
acompanham a rotina escolar, sempre supervisionados por professores. A
intenção é que eles tenham contato com as escolas e se preparem para o
trabalho como professores. De acordo com resolução do Conselho Nacional de
Educação (CNE), esse estágio deve ter a duração de pelo menos 400 horas.

Os dados do Enade, no entanto, mostram que a regra, na prática, não está sendo
cumprida. O Enade é um exame realizado por estudantes que estão concluindo os
cursos de graduação. A cada ano, o exame avalia um conjunto diferente de
cursos. Em 2021, foi a vez das licenciaturas. Além de realizar as provas, os
alunos respondem a um questionário sobre a formação. As perguntas sobre o
estágio fazem parte deste questionário.

Cerca de 55\% dos concluintes em licenciaturas, o equivalente a cerca de 165
mil estudantes, disseram que cumpriram menos de 300 horas de estágio. Outros
11,82\%, o equivalente a 35,5 mil alunos, disseram que sequer fizeram o
estágio. Os dados mostram ainda que 19,49\%, ou 58,5 mil, cumpriram entre 301
e 400 horas e apenas 13,71\%, ou 41,2 mil, fizeram estágios de mais de 400
horas.

``O estágio permite essa conexão da teoria com a prática. Tudo que se aprende
na teoria, se vê aplicações práticas na escola'', diz o gerente de Políticas
Educacionais do Todos Pela Educação, Ivan Gontijo.

``É importante que os estudantes conheçam a dinâmica da escola, os papéis e as
responsabilidades de cada um dos atores da equipe escolar. Nesse período, vão
entender como é a organização do espaço e como é o trabalho ali. O estágio tem
caráter de observação e, progressivamente vai permitindo participar mais,
acompanhar professores nas avaliações e atividades. Por isso a carga horária é
grande''.

\fonte{Agência Brasil. Disponível em: https://agenciabrasil.ebc.com.br/educacao/noticia/2023-09/maioria-dos-futuros-professores-nao-conclui-estagio-em-escolas. Acesso em: 28 set. 2023.}

\end{myquote}

\num{1} O primeiro parágrafo causa forte impacto no leitor. Como o autor conseguiu 
produzir esse efeito? 

\reduline{No primeiro parágrafo, o autor apresenta duas informações de impacto, 
que surpreendem o leitor.\hfill}

\num{2} Ainda no primeiro parágrafo, depois de causar o impacto que você analisou
na questão anterior, o autor uma afirmação para obter a confiança do leitor. 
Explique como ele fez isso. 

\reduline{No primeiro parágrafo, depois de apresentar duas informações de impacto, 
que surpreendem o leitor, o autor explica onde as encontrou: em um no Enade de 2021, 
fonte confiável que ele usou para escrever a notícia.\hfill}

\num{3} No terceiro parágrafo, por que o autor usou a palavra ``no entanto''? 
Explique detalhadamente. 

\reduline{No terceiro parágrafo, o autor o autor usou a palavra ``no entanto'' para
explicitar a diferença entre o que ele explicou no segundo parágrafo e a realidade
que ele vai explicar no terceiro. De fato, existe uma diferença grande entre o estágio
que os futuros professores deveriam fazer e o que eles cumprem concretamente.\hfill}

\num{4} Levando em consideração o texto como um todo, para que servem as 
informações do quarto parágrafo?

\reduline{As informações do quarto parágrafo servem para exemplificar as 
afirmações dos parágrafos anteriores, de maneira mais detalhada.\hfill}

\num{5} No quinto e sexto parágrafos, o autor do texto transcreveu a fala de uma pessoa. 
Como reconhecemos o trecho em que o autor ``deu voz'' a alguém que ele entrevistou? 
Por que é importante que uma outra pessoa seja ouvida pelo autor? 

\reduline{Reconhecemos que o autor registrou a fala de um entrevistado com o 
uso de aspas, que indicam o começo e o fim dessa fala. Em uma reportagem, 
é importante ouvir especialistas sobre o assunto analisado, de modo a dar
mais credibilidade para a notícia.\hfill}

\num{6} Mesmo que o autor não escreva explicitamente, podemos concluir qual 
deve ser a opinião dele sobre o assunto, levando em consideração o conjunto
do texto. Qual é a opinião dele? Explique detalhadamente. 

\reduline{A leitura integral do texto permite concluir que, para o autor, 
o não cumprimento do estágio obrigatório é um problema. Em primeiro lugar,
ele escolheu essa informação, entre muitas outras, para noticiar --- o que 
já indica a importância do assunto. Além disso, a oposição entre o que 
deveria acontecer e o que de fato acontece também sugere que a notícia
tem algum tom de alerta para o problema. Finalmente, as declarações do
especialista aprofundam aquela oposição, indicando que o autor do texto, 
embora não declare explicitamente, está apontando uma falha na educação
brasileira.\hfill}

Leia o texto abaixo para responder às questões de 7 e 8.

\begin{myquote}

\textbf{Voltaire e a história da criança brasileira}

\begin{figure}[H]
\centering
\includegraphics[width=0.95\textwidth]{./imgSAEB_7_POR/media/image46.png}
\end{figure}
%Fonte Pixabay https://pixabay.com/pt/photos/r%C3%BAssia-s%C3%A3o-petersburgo-eremit%C3%A9rio-1381260/

Nossa situação, sem dúvidas, melhorou. Mas estamos longe do ideal. Segundo
dados do relatório do Unicef intitulado \textit{As múltiplas dimensões da pobreza na
infância e na adolescência no Brasil}, temos 32 milhões de meninos e meninas
vivendo na pobreza. Isso representa 63\% da população com idade até 17 anos.

Considerando que os primeiros anos de vida são fundamentais para o
desenvolvimento cognitivo, estamos comprometendo o futuro de uma geração
inteira. E não restam dúvidas de que essa realidade é sempre pior para as
classes mais desfavorecidas.

E qual é a solução para tamanho desafio? Recorremos à obra de
Voltaire, \textit{Cândido, ou o Otimismo}: ``É preciso cultivar o nosso jardim.
Trabalhemos sem tanta argumentação. É o único meio de tornar a vida
suportável. Isto afasta o tédio, o vício e a necessidade''.

E qual é o jardim que precisa ser cultivado? 

Não hesito em responder: atender bem as crianças e, consequentemente, ensinar
uma Pediatria de excelência para todas as áreas da saúde.

\fonte{Giuseppe Pastura. Conexão UFRJ. Voltaire e a história da criança brasileira.
Disponível em: https://conexao.ufrj.br/2023/06/voltaire-e-a-historia-da-crianca-brasileira/.
Com adaptações. Acesso em: 28 set. 2023.}

\end{myquote}

\num{7} Na primeira frase do texto, existe uma marca evidente de parcialidade.
Que marca é essa?

\reduline{Na primeira frase do texto, o uso da primeira pessoa do singular é
uma marca evidente de parcialidade.\hfill} 

\num{8} Levando em consideração os dois primeiros parágrafos, responda: a marca de 
parcialidade que você identificou no exercício anterior faz o texto perder
força de argumentação? Explique.

\reduline{Lendo o primeiro parágrafo, podemos afirmar que a primeira pessoa
do singular não faz o texto perder força argumentativa. Em primeiro lugar,
porque o autor articula essa marca de parcialidade com dados de uma pesquisa
feita por uma fonte confiável. Em segundo lugar, porque ele também usa a primeira 
pessoa do plural (por exemplo, na frase ``estamos comprometendo o futuro de uma 
geração inteira'') para persuadir o leitor. Dizendo de forma simples, o uso 
explícito do ``eu'' é uma forma de criar uma relação de cumplicidade com o leitor,
que é bastante persuasiva.\hfill} 

Leia o texto abaixo para responder às questões 9 e 10. 

\begin{myquote}

\begin{figure}[H]
\centering
\includegraphics[width=0.95\textwidth]{./imgSAEB_7_POR/media/image47.png}
\end{figure}
%Fonte Pixabay https://pixabay.com/pt/photos/arvores-avenida-carros-veículos-6040099/

Existe uma solução ou ao menos uma ação paliativa que ajude a população a
lidar com as altas temperaturas? Pesquisas mostram uma relação direta entre a
arborização nas áreas urbanas e a diminuição da sensação térmica. O professor
do Departamento de Meteorologia e coordenador do Laboratório de Aplicações de
Satélites Ambientais (Lasa/UFRJ), Leonardo de Faria Peres, conta que a
vegetação, em geral, reduz as temperaturas dos ambientes urbanos diretamente
pelo fornecimento de sombras e indiretamente por meio de sua transpiração.

Segundo ele, ``áreas vegetadas presentes em regiões urbanas comportam-se como
focos de frescor, mitigando o efeito das ilhas de calor. Portanto, esses
espaços verdes devem ser preservados''. Além disso, o professor também lembrou
que as árvores trazem outros benefícios, como a ``diminuição da necessidade de
uso de ar-condicionado, melhora da qualidade do ar e da água, proporcionam
valor estético, reduzem o ruído e servem de habitat para diferentes espécies''.

\fonte{Maria Clara Paiva. Conexão UFRJ. Árvores: uma maneira eficaz de conter 
o calorão nas grandes cidades. Disponível em: https://conexao.ufrj.br/2023/09/arvores-uma-maneira-eficaz-de-conter-o-calorao-nas-grandes-cidades/. Acesso em:
28 set. 2023.}

\end{myquote}

\num{9} Logo no primeiro parágrafo, a autora usa um recurso para chamar e prender
a atenção do leitor. Que recurso é esse? Explique. 

\reduline{A primeira frase do texto é uma pergunta. Quando perguntas aparecem
no texto, o leitor se sente impelido a avançar no texto, de modo a encontrar a
resposta à pergunta que foi feita.\hfill}

\num{10} Depois de usar o recurso que você reconheceu na questão anterior, a 
autora apresenta elementos que trazem credibilidade para o texto. Que recursos
são esses? 

\reduline{Para conferir credibilidade à resposta que deu à pergunta que abriu 
o parágrafo, a autora recorreu a uma referência genérica a ``pesquisas'', depois
apresentou as opiniões de um especialista.\hfill}

\section*{Treino}

\num{1} Leia o texto abaixo para responder à questão.

\begin{myquote}
O plantio de árvores nas cidades é um dos melhores investimentos que qualquer 
município pode fazer. Não há como negar os inúmeros benefícios que árvores urbanas 
proporcionam. Elas não apenas embelezam nossas ruas, praças e parques, como também purificam
o ar, oferecem sombra em dias escaldantes, reduzem a poluição sonora, fornecem
abrigo para a fauna local e melhoram significativamente a qualidade de vida
dos habitantes. Além disso, a presença de árvores está intrinsecamente ligada
ao combate às mudanças climáticas, pois absorvem CO2 e ajudam a mitigar o
aquecimento global. É inegável, dessa forma, que devemos dar prioridade ao plantio
de árvores em nossas cidades, em prol de um ambiente mais saudável e
sustentável para as gerações presentes e futuras.

\fonte{Texto formulado para este material.}

\end{myquote}

Assinale a alternativa que contém uma explícita marca de parcialidade do texto acima.

\begin{escolha}

  \item ``reduzem a poluição sonora''.
  \item ``fornecem abrigo para a fauna local''.
  \item ``mitigar o aquecimento global''.
  \item ``É inegável, dessa forma, que devemos dar prioridade''. 

\end{escolha}

\num{2} Leia o texto abaixo para responder à questão.

\begin{myquote}
Comida africana na cidade do Rio de Janeiro não é algo trivial. Por isso, os
restaurantes africanos que surgiram nos últimos dez anos são novidades nos
circuitos gastronômicos da cidade. Hoje temos o \textit{Zungu -- Guia de Gastronomia
Preta}, que mapeia esses empreendimentos e oferece 20 opções para quem quer
comer fora de casa. Mais do que uma opção ocasional para o almoço de domingo,
esses locais são pontos de articulação de pessoas negras e hubs de ideias
afrodiaspóricas no Rio. Na minha pesquisa de doutorado, venho estudando como a
gastronomia carioca, com esses restaurantes, vem se (re)africanizando.

\fonte{Rodolfo Teixeira Alves. Conexão UFRJ. Você conhece a gastronomia africana 
do Rio de Janeiro? Disponível em: https://conexao.ufrj.br/2022/04/voce-conhece-a-gastronomia-africana-do-rio-de-janeiro/. 
Acesso em: 28 set. 2023.}
\end{myquote}

Assinale a alternativa que contém uma explícita marca de parcialidade do texto acima.

\begin{escolha}

  \item ``Comida africana na cidade do Rio de Janeiro não é algo trivial''.
  
  \item ``mapeia esses empreendimentos''.
  
  \item ``oferece 20 opções para quem quer comer fora de casa''.
  
  \item ``opção ocasional para o almoço de domingo''.

\end{escolha}

\num{3} O texto abaixo foi escrito para este material e é baseado em opiniões
de internautas. Leia-o atentamente para responder à questão.

\begin{myquote}

\begin{figure}[H]
\centering
\includegraphics[width=0.95\textwidth]{./imgSAEB_7_POR/media/image48.png}
\end{figure}
%fonte pixabay https://pixabay.com/pt/photos/hist%C3%B3rias-em-quadrinhos-novelas-1239698/

A Marvel Studios é conhecida por sua vasta galeria de filmes de super-heróis,
mas nem todos são unanimidades de sucesso. Um exemplo é o filme \textit{Thor: O Mundo
Sombrio}. Neste longa-metragem, Thor, o Deus do Trovão, enfrenta uma ameaça
ancestral que ameaça mergulhar o universo em trevas eternas. A trama tenta
misturar elementos de fantasia épica com super-heróis, mas acaba se perdendo
no processo. Os personagens, embora interpretados por atores talentosos,
carecem de profundidade e desenvolvimento. O vilão Malekith, interpretado por
Christopher Eccleston, é particularmente unidimensional e esquecível. Além
disso, o filme falha em equilibrar humor e drama, resultando em piadas que
parecem forçadas e um tom geral inconsistente. Enquanto as cenas de ação são
visualmente impressionantes, não são suficientes para sustentar o interesse ao
longo do filme. \textit{Thor: O Mundo Sombrio} é um exemplo de como, às vezes, a
Marvel pode tropeçar ao tentar expandir seu universo cinematográfico,
oferecendo uma experiência que não está à altura de seus melhores trabalhos.

\end{myquote}

Assinale a alternativa que contém uma explícita marca de parcialidade do texto acima.

\begin{escolha}
  
  \item ``A Marvel Studios é conhecida por sua vasta galeria de filmes''.
  
  \item ``Thor, o Deus do Trovão, enfrenta uma ameaça ancestral''.
  
  \item ``A trama tenta misturar elementos de fantasia épica com super-heróis''.
  
  \item ``não são suficientes para sustentar o interesse''.

\end{escolha}

\chapter{Recursos de modalização e argumentação}
\markboth{Módulo 8}{}

\section*{Habilidades do SAEB}

\begin{itemize}

  \item Identificar os recursos de modalização em textos diversos.

  \item Analisar os efeitos de sentido dos tempos, modos e/ou vozes 
verbais com base no gênero textual e na intenção comunicativa.

  \item Analisar os efeitos de sentido produzidos pelo uso de modalizadores em textos diversos.

\end{itemize}

\subsection{Habilidades da BNCC}

\begin{itemize}

  \item EF69LP04, EF69LP28, EF07LP14.

\end{itemize}

\conteudo{A capacidade de identificar e compreender os recursos de modalização
presentes em diferentes tipos de texto é fundamental para uma
comunicação eficiente e coerente. A modalização envolve o uso de
recursos linguísticos para expressar atitudes e opiniões do emissor em
relação ao conteúdo abordado. Esses recursos podem estar
presentes nas formas de uso de modos e vozes verbais, além de adjetivos e advérbios.

A análise dos efeitos de sentido produzidos pelos recursos de
modalização em diferentes gêneros textuais e intenções comunicativas é
fundamental para identificar as influências destes recursos na percepção
e compreensão do conteúdo pelo receptor.

Os recursos de modalização permitem que o emissor expresse suas
opiniões, transmita suas expectativas, faça indicações, influenciando
assim na percepção dos leitores. Dentre os principais recursos
de modalização, encontram-se os tempos verbais, os modos verbais, as
vozes verbais e os modalizadores, que podem indicar graus de
probabilidade, certeza, necessidade e urgência.

Portanto, compreender e utilizar os recursos de modalização de forma
adequada é essencial para produzir textos claros, coerentes e
persuasivos, adaptando a linguagem ao tipo de público alvo e ao contexto
em que são emitidos. Além disso, a análise crítica desses recursos
permite que o leitor compreenda as intenções do emissor e as influências
que podem estar presentes no conteúdo do texto.}

% \coment{Professor(a), estimule os estudantes a lembrarem do uso destes recursos
% em exemplos conhecidos e práticos tais como campanhas de conscientização
% ou publicitárias. Retome os recursos de modalização presentes em
% diversos gêneros textuais já estudados demonstrando as diferenças de
% interpretação e sentido possíveis para os usos de tempos e modos
% verbais, tais como infinitivo e imperativo, das vozes do texto, dos
% discursos diretos e indiretos e do uso de advérbios que qualificam as
% ações com advérbios que indicam urgência, condições, sugestões,
% obrigatoriedade, etc.}

\section*{Atividades}

\includegraphics[width=2.85536in,height=2.85536in]{./imgSAEB_7_POR/media/image9.png}

% \fonte{https://itajuba.mg.gov.br/secretariaspmi/semdes/nao_desvie_o_olhar/}{\uline{https://itajuba.mg.gov.br/secretariaspmi/semdes/nao\_desvie\_o\_olhar/}}.
% Acesso em: 18 de abr de 2023

\fonte{itajuba.mg.gov.br}

\num{1} Retire da imagem acima os elementos modalizadores que produzem o sentido de instrução.

\reduline{As formas verbais no imperativo produzem o sentido de instrução no texto.\hfill}

\num{2} De que forma o uso desses recursos sensibiliza o leitor?

\reduline{As formas verbais no imperativo apelam para a responsabilidade individual do leitor quanto
à violência contra crianças e adolescentes.\hfill}

\num{3} Observe as informações contidas no cartaz e explique qual é a ação
recomendada ao tomar conhecimento de casos de violência contra crianças e adolescentes.

\reduline{Ligar para o telefone indicado e denunciar a violência contra crianças e adolescentes é 
a ação recomendada.\hfill}

%\num{4} Relacione as colunas de expressões com seus respectivos efeitos de sentido:

% Please add the following required packages to your document preamble:
% \usepackage[table,xcdraw]{xcolor}
% If you use beamer only pass "xcolor=table" option, i.e. \documentclass[xcolor=table]{beamer}

% REVER
% \begin{table}[h!]
% \begin{tabular}{|cc|cc|}
% \hline
% \rowcolor[HTML]{FD6864} 
% \multicolumn{2}{|c|}{\cellcolor[HTML]{FD6864}\textbf{Efeito de sentido}} & \multicolumn{2}{c|}{\cellcolor[HTML]{FD6864}\textbf{Expressão}} \\ \hline
% \multicolumn{1}{|c|}{\textbf{I}} & Obrigação & \multicolumn{1}{c|}{} & felizmente \\ \hline
% \rowcolor[HTML]{FFCCC9} 
% \multicolumn{1}{|c|}{\cellcolor[HTML]{FFCCC9}\textbf{II}} & Apreciação & \multicolumn{1}{c|}{\cellcolor[HTML]{FFCCC9}} & é dever de todos \\ \hline
% \multicolumn{1}{|c|}{\textbf{III}} & Possibilidade & \multicolumn{1}{c|}{} & é impossível \\ \hline
% \rowcolor[HTML]{FFCCC9} 
% \multicolumn{1}{|c|}{\cellcolor[HTML]{FFCCC9}\textbf{IV}} & Certeza & \multicolumn{1}{c|}{\cellcolor[HTML]{FFCCC9}} & provavelmente \\ \hline
% \end{tabular}
% \end{table}

% \coment{II, I, IV, III}

% \begin{myquote}

% \textbf{Certamente}, com a implementação dos projetos de mobilidade
% urbana, as pessoas passarão a ter mais qualidade de vida e mais
% liberdade para explorar e ocupar a cidade.

% \end{myquote}

% \num{5} Qual sentido a palavra destacada pretende expressar?

% \reduline{A palavra destacada pretende expressar certeza sobre as consequências da melhoria da
% mobilidade urbana.\hfill}

Observe a campanha a seguir e responda à questão 4.

\includegraphics[width=5.90551in,height=1.47222in]{./imgSAEB_7_POR/media/image10.png}

\fonte{Agência de Transporte do Estado de São Paulo.}

% \fonte{Agência de Transporte do Estado de São Paulo. #FocaNaVida. Disponível em:
% http://www.artesp.sp.gov.br/Style\%20Library/extranet/campanha-interna.aspx?id=1
% Acesso em: 22 mai. 2023.}

\num{4} Qual a ideia transmitida pela conjunção ``se'' na frase, 
``Nesse carnaval se beber, não dirija''? Qual a relação dessa conjunção 
com o sentido da frase ``não dirija''?

\reduline{A conjunção ``se'', na frase analisada, é condicional, ou seja, é por meio dela que 
se propõe a condição expressa na frase: caso o leitor tenha bebido, não deve dirigir.\hfill} 

% \num{7} Ainda sobre a campanha acima, assinale verdadeiro ou falso na tabela para as afirmações a seguir.

% \begin{table}[h!]
% \begin{tabular}{|c|c|}
% \hline
% \textbf{\begin{tabular}[c]{@{}c@{}}Verdadeiro (V)\\ ou Falso (F)?\end{tabular}} & \textbf{Afirmações} \\ \hline
% \rosa{V} & a campanha sugere que não se deve dirigir após beber \\ \hline
% \rosa{F} & a campanha sugere que as pessoas não devem beber durante o carnaval \\ \hline
% \rosa{F} & a campanha orienta os foliões a não dirigir no carnaval \\ \hline
% \rosa{F} & a campanha pretende alertar sobre a violência no trânsito \\ \hline
% \end{tabular}
% \end{table}

% %Gabarito \coment {V, F, F, F}

% \num{8} Leia os enunciados e enumere as proposições na tabela.

% \begin{table}[h!]
% \begin{tabular}{|c|l|c|l|}
% \hline
% \textbf{} & \multicolumn{1}{c|}{\textbf{Enunciado}} & \textbf{} & \multicolumn{1}{c|}{\textbf{Efeito}} \\ \hline
% \textbf{1} & É necessária a presença de um acompanhante & \rosa{2} & Expressa proibição \\ \hline
% \textbf{2} & É proibida a entrada de acompanhante & \rosa{1} & Expressa obrigatoriedade \\ \hline
% \textbf{3} & É permitida a entrada de um acompanhante & \rosa{3} & Expressa permissão \\ \hline
% \end{tabular}
% \end{table}

%Gabarito: \coment{2, 1, 3}

%Leia tirinha abaixo para responder à questão.

%\includegraphics[width=5.90551in,height=1.72222in]{./imgSAEB_7_POR/media/image11.png}

%\fonte{https://tirasarmandinho.tumblr.com/}{\uline{https://tirasarmandinho.tumblr.com/}}
%Acesso em 19 de Abr de 2023

%\num{9}
%Qual é a confusão comunicativa presente na tirinha?
%
%
%A confusão se dá pois os personagens entendem o verbo falar em sentidos diferentes

%\num{10} Qual o sentido do verbo falar no segundo quadrinho? Escreva um sinônimo.

%O verbo falar no segundo quadrinho se refere a opinião, e poderia ser
%substituído por dizer ou pensar. Pensaria, diria.

Leia o texto abaixo para responder às questões de 5 a 8.

\begin{myquote}


\begin{wrapfigure}{r}{0.5\textwidth}  % 'l' para alinhar à esquerda, 'r' para alinhar à direita
  \centering
  \includegraphics[width=0.55\textwidth]{./imgSAEB_7_POR/media/image49.png}
  %\caption{\textit{Texto sobre a imagem}}
\end{wrapfigure}
Indiscutivelmente, o reflorestamento de áreas degradadas é uma ação essencial
para a preservação do meio ambiente. A recuperação de
ecossistemas devastados contribui de maneira significativa para a mitigação das
mudanças climáticas, assegurando, assim, um futuro mais sustentável para as
próximas gerações. Nesse sentido, é imperativo que sejam implementadas medidas
concretas e imediatas, sobretudo considerando a urgência com que o
desmatamento avança em muitas regiões do mundo. É inegável que a restauração
de áreas degradadas é uma necessidade premente para restaurar o equilíbrio
ecológico e garantir a disponibilidade de recursos naturais tão cruciais para
a vida na Terra. Portanto, é indubitável que o reflorestamento deve ser
encarado como uma tarefa inadiável, capaz de assegurar um futuro mais seguro e
saudável para o planeta.

% \begin{figure}[H]
% \centering
% \includegraphics[scale=0.25]{./imgSAEB_7_POR/media/image49.png}
% \end{figure}

\fonte{Texto formulado para este material.}

\end{myquote}

\num{5} Explique o sentido do termo destacado na frase ``\textsc{\textbf{Indiscutivelmente}}, o
reflorestamento de áreas degradadas é uma ação essencial para a preservação do
meio ambiente''.

\reduline{Para o autor do texto, não é necessário discutir a importância do
reflorestamento de áreas degradadas.\hfill}

\num{6} Explique o sentido do termo destacado na frase ``A recuperação de
ecossistemas devastados contribui \textsc{\textbf{de maneira significativa}} para a mitigação das
mudanças climáticas''.

\reduline{O termo destacado serve para indicar o modo significativo como a recuperação dos ecossistemas 
contribui para reduzir mudanças climáticas.\hfill}

\num{7} Explique o sentido do trecho destacado na frase ``\textsc{\textbf{é imperativo que}} sejam 
implementadas medidas concretas e imediatas, sobretudo considerando a urgência com que o
desmatamento avança em muitas regiões do mundo''.

\reduline{O termo destacado serve para indicar uma atitude que, na opinião do autor, deve ser
tomada com urgência.\hfill}

\num{8} Explique o sentido do trecho destacado na frase ``\textsc{\textbf{É
inegável que}} a restauração de áreas degradadas é uma necessidade premente
para restaurar o equilíbrio ecológico''.

\reduline{O termo destacado serve para evidenciar uma opinião do autor. Para ele,
não se pode negar que é preciso restaurar áreas degradadas.\hfill}

Leia o texto abaixo para responder às questões 9 e 10.

\begin{myquote}

\begin{figure}[H]
\centering
\includegraphics[width=0.95\textwidth]{./imgSAEB_7_POR/media/image50.png}
\end{figure}
%Fonte Imagem de <a href="https://br.freepik.com/fotos-gratis/carros-borrados-abstratos-veiculos-na-rua-da-cidade_4605202.htm#query=leis%20de%20tr%C3%A2nsito&position=23&from_view=search&track=ais">Freepik</a>

Eu nunca entendi como funciona a cabeça de uma pessoa que desrespeita as leis
de trânsito de propósito, até que conheci um funcionário da CET, a Companhia
de Engenharia de Tráfego, aqui de São Paulo. Ele era casado com minha prima e
passamos a tarde conversando, numa festa de família. 

Sem falar de um caso específico, para evitar problemas, ele me contou que, se
eu conhecesse algumas pessoas que infringem as regra de trânsito,
elas me diriam que as regras são excessivas e desnecessárias, um
verdadeiro desperdício de tempo. Além disso, elas argumentariam que muitas das
leis de trânsito são aplicadas de forma errada pelos fiscais e que, no final
das contas, todos fazem suas próprias regras, o que, segundo essa perspectiva,
torna a obediência às leis de trânsito opcional.

Para nós, no entanto, essa visão negligencia os propósitos fundamentais das leis 
de trânsito. Elas garantem a segurança de todos os usuários das vias públicas e 
promovem a ordem e a fluidez do tráfego.

\fonte{Texto formulado para este material.}

\end{myquote}

\num{9} Explique por que o primo do autor usou as expressões sublinhadas em  
`` \textsc{\textbf{se}} eu \textsc{\textbf{conhecesse}} algumas pessoas que 
infringem as regra de trânsito, elas me \textsc{\textbf{diriam}} que \ldots{}''

\reduline{As expressões sublinhadas no trecho servem para o marido da prima
do autor criar uma situação hipotética, sem dar nomes às pessoas que pensam
dessa forma. Ao fazer isso, ele quer evitar problemas no trabalho. \hfill}

\num{10} As frases ``as regras são excessivas e desnecessárias'' e 
``Elas garantem a segurança de todos os usuários das vias públicas'' expressam
opiniões opostas sobre as leis de trânsito. Segundo o texto, quem pensa
da primeira maneira? Quem pensa da segunda?

\reduline{Segundo o texto, pessoas que não respeitam as leis de trânsito 
acreditam que ``as regras são excessivas e desnecessárias''; o narrador e 
o marido de sua prima, por sua vez, são da opinião de que ``Elas garantem a
segurança de todos os usuários das vias públicas''.\hfill}

\section*{Treino}

\num{1} Leia o texto abaixo para responder à questão.

\begin{myquote}

Antes da pandemia causada pelo novo coronavírus, era quase impensável
ver grande parte da população usando máscaras de proteção na rua.
Contudo, a situação mudou, principalmente após o governo do estado tornar
seu uso obrigatório. Mesmo assim, ainda há dúvidas de parte da população
quanto à necessidade e ao benefício do seu uso.

\fonte{Secretaria de Estado de Saúde de Minas Gerais. Notas de Recomendação: Covid 19.
Disponível em: https://coronavirus.saude.mg.gov.br/blog/101-mascaras-e-covid-19.
Acesso em: 22 mai. 2023.}

\end{myquote}

No contexto em que se insere, a expressão ``mesmo assim'' expressa oposição entre:

\begin{escolha}

  \item o período anterior e o posterior à pandemia do coronavírus.

  \item a alta adesão da população ao uso de máscaras e a obrigatoriedade de usá-las.

  \item a obrigatoriedade do uso de máscaras e a atitude da população quanto a essa medida.

  \item grande parte da população utilizando máscaras de proteção e a minoria em dúvida.

\end{escolha}

\num{2} Leia o texto abaixo para responder à questão. 

\begin{myquote}

Carlos observava toda aquela pompa ao seu redor. Ele mesmo estava de passagem,
era um turista simples, brasileiro comum, passeando a custo do dinheiro suado, 
guardado todo mês, para conhecer aquela terra estrangeira e faustosa, cheia de 
gente milionária e bem vestida, cheirosa e fútil. Como deve ser triste depender
do luxo para ser feliz! --- pensava ele.

\fonte{Rogério Duarte. Viagens pela terra dos outros. Saíra Editorial, no prelo.}

\end{myquote}

Na frase ``Como \textbf{deve} ser triste depender do luxo para ser feliz!'', 
a forma verbal destacada expressa:

\begin{escolha}
  
  \item obrigação.
  
  \item conselho.
  
  \item causa.
  
  \item possibilidade.

\end{escolha}

\num{3} Analise o cartaz abaixo para responder à questão.


\includegraphics[width=5.90551in,height=4.29167in]{./imgSAEB_7_POR/media/image13.png}

\fonte{Ministério da Integração e do Desenvolvimento Regional. Consumo consciente da água 
é base para um futuro sustentável. Disponível em: https://www.gov.br/dnocs/pt-br/assuntos/noticias/consumo-consciente-da-agua-e-base-para-um-futuro-sustentavel.
Acesso em: 22 mai. 2023.}

O uso do gerúndio na campanha acima expressa:

\begin{escolha}

  \item ação contínua.
  \item ordem a ser acatada.
  \item sugestão a ser considerada.
  \item condição.

\end{escolha}


\chapter{Figuras de linguagem como estratégia argumentativa}
\markboth{Módulo 9}{}

\section*{Habilidades do SAEB}

\begin{itemize}
  
  \item Analisar o uso de figuras de linguagem como estratégia argumentativa.

  \item Avaliar a eficácia das estratégias argumentativas em textos de diferentes gêneros.

\end{itemize}

\subsection{Habilidades da BNCC}

\begin{itemize}

  \item EF69LP17, EF67LP38.

\end{itemize}

\conteudo{A linguagem tem como principal objetivo a comunicação, portanto, é
impossível distinguir a linguagem da interação que ela provoca. Quanto
maior o domínio das ferramentas de linguagem por parte do autor, maior o
nível de interação e conexão será atingido com o leitor. Do outro lado,
quanto mais o leitor for capaz de reconhecer as ferramentas e recursos
de expressividade, melhor será sua experiência de leitura. 

Existem diversas 
maneiras de promover maior expressividade e, assim, facilitar e potencializar a interação por meio
da comunicação. Dentre esses recursos estão as chamadas \textbf{figuras de
linguagem}.

Cada figura de linguagem tem características próprias e pode ser
utilizada de diversas maneiras, de acordo com o objetivo do falante ou
do escritor, com a situação comunicativa e com o interlocutor. As
figuras de linguagem podem fazer alusão ao sentido figurado das palavras
e expressões, tratar com exagero para exacerbar características e
impressões ou contrapor palavras com sentidos antagônicos demonstrando
valores de forma implícita. Por essas características, as figuras de
linguagem são recursos comunicativos de grande utilidade, especialmente
para escrita de textos do campo artístico literário como letras de
música e poemas. Também podem apresentar-se como recurso de persuasão e
comunicação importante nos textos do campo jornalístico
midiático.

A \textbf{metáfora} é uma figura de linguagem utilizada para
estabelecer uma relação de semelhança e comparação entre dois termos
distintos. Um exemplo do uso da metáfora é a expressão ``sua vida era um
mar de rosas''. Nesse caso, a expressão ``mar de rosas'' não tem sentido
literal, mas figurado, pois se refere a algo agradável e
prazeroso e não ao mar em sentido literal.

A \textbf{metonímia}, por sua vez, é a figura de linguagem utilizada quando se
lança mão um termo para se referir a outro, com o qual ele mantém uma
relação de proximidade, contiguidade e pertencimento. Ou seja, dizer que
``a cidade se beneficiou com as obras de infraestrutura'' é utilizar a metonímia,
por meio da qual o termo ``a cidade'' se refere à população da cidade. A metonímia
tem o efeito de tornar a linguagem mais concisa e precisa, evitando a repetição de
termos e textos muito extensos. É uma forma de dinamizar a comunicação e
potencializar a capacidade comunicativa.

Por meio da \textbf{personificação} atribuem-se características humanas a animais
e objetos. A personificação tem o efeito de tornar a linguagem mais expressiva e 
poética, pois humaniza seres e fenômenos, produzindo alto grau de identificação do
leitor com a experiência, sensação ou impressão que se pretende comunicar.

Por fim, a \textbf{hipérbole} é a figura de linguagem que aumenta ou exagera
determinada característica como forma de expressão.}

% \coment{Professor(a) faça um levantamento dos conhecimentos prévios que os 
% estudantes já têm. Deixe que tragam
% exemplos e forneça mais informações sobre os conceitos, chamando a
% atenção para as possibilidades expressivas das figuras de linguagem.
% Mostre como o uso das figuras de linguagem podem ampliar os efeitos de
% sentido, em quaisquer gêneros textuais.}

\section*{Atividades}

Leia os primeiros versos do Hino Nacional Brasileiro para responder às perguntas:

\begin{myquote} 
% \begin{wrapfigure}{r}{0.4\textwidth}  % 'l' para alinhar à esquerda, 'r' para alinhar à direita
%   \centering
%   \includegraphics[width=0.35\textwidth]{./imgSAEB_7_POR/media/image51.png}
%   %Fonte Wikipedia https://pt.wikipedia.org/wiki/Osório_Duque-Estrada#/media/Ficheiro:Joaquim_Osório_Duque-Estrada_(poeta_brasileiro).jpg
%   %\caption{\textit{Texto sobre a imagem}}
% \end{wrapfigure}
\begin{verse}
Ouviram do Ipiranga as margens plácidas \\
De um povo heroico o brado retumbante.
\end{verse}
\fonte{Joaquim Osório Duque Estrada. Hino Nacional Brasileiro.}
\end{myquote}

% \fonte{Joaquim Osório Duque Estrada. Hino Nacional Brasileiro. 
% Disponível em: https://www.planalto.gov.br/ccivil_03/constituicao/hino.htm.
% Acesso em: 22 mai. 2023.}

\num{1} Coloque os versos e termos da oração na ordem direta para compreender 
melhor o sentido da frase. 

\reduline{As margens plácidas do Ipiranga ouviram o brado retumbante de um povo heroico.\hfill}

\num{2} Quais são as figuras de linguagem presentes nos dois primeiros versos do 
Hino Nacional Brasileiro? Justifique sua resposta.

\reduline{As figuras de linguagem são o hipérbato -- isto é, a inversão dos termos -- e a
personificação, atribuição de ações humanas (ouvir) a seres inanimados (as margens do 
Rio Ipiranga).\hfill}

\num{3} Qual é a figura de linguagem presente na frase ``Você tem um coração de pedra!''? 
Explique.

\reduline{Na frase ``Você tem um coração de pedra!'' existe metáfora. A metáfora é uma comparação
cujos termos são omitidos. Nesse caso, a comparação se dá por meio do elemento comum à pedra
e ao coração: sua \textit{dureza}.\hfill}

\num{4} Qual é a figura de linguagem presente na frase ``Estou morrendo de medo''? Explique.

\reduline{Na frase  ``Estou morrendo de medo'' observa-se hipérbole, devido ao exagero.\hfill}

\num{5} Qual é a figura de linguagem presente na frase ``As estrelas são os olhos dos deuses''?
Explique.

\reduline{Na frase ``As estrelas são os olhos dos deuses'' existe metáfora. A metáfora é uma comparação
cujos termos são omitidos. Nesse caso, a comparação se dá por meio do elemento comum às estrelas e 
e aos olhos: seu \textit{brilho e/ou sua posição privilegiada em relação aos homens}.\hfill}

% \num{6} Sobre a hipérbole, assinale V para as afirmações verdadeiras e F para as falsas:

% \begin{table}[h!]
% \begin{tabular}{|c|c|}
% \hline
% \textbf{\begin{tabular}[c]{@{}c@{}}Verdadeiro (V) \\ ou\\ Falso (F)?\end{tabular}} &  \\ \hline
%  & comparação entre palavras distintas, sem mostrar os termos dessa comparação \\ \hline
%  & exagero do sentido de uma palavra ou expressão \\ \hline
%  & substituição de uma palavra por outra \\ \hline
%  & atribuição de características humanas a seres inanimados \\ \hline
% \end{tabular}
% \end{table}

% %\coment{F, V, V, F}

% Leia o texto abaixo para responder à questão 6.

% \begin{myquote}

% Na cordilheira que fica em cima do vale de Yyucay, em Cusco,
% pode-se ouvir todos os sons. O vento sopra com sua bocarra;
% a manhã, obrigada a se levantar sempre antes dos outros,
% boceja morta de sono; os pássaros, seus eternos namorados,
% acordam cantando ao ouvi-la se espreguiçar.

% \fonte{Ana Rosa Abreu e outros autores. Alfabetização: livro do aluno. Vol.2: 
% contos tradicionais, fábulas, lendas e mitos. Disponível em: 
% http://www.dominiopublico.gov.br/download/texto/me001614.pdf.
% Acesso em: 22 mai. 2023.}

% \end{myquote}

Leia o texto abaixo para responder às questões de 6 a 9.

\begin{myquote}

Naquela manhã, a escola amanheceu confusa. Ninguém sabia quem tinha pregado no
mural uma redação misteriosa, meio poética, meio engraçada, que dizia assim:
``Naquela noite, o céu era um oceano escuro e piscante, enquanto os carros, 
ferozes bestas de metal, rugiam pelas ruas famintos por asfalto. No entanto, 
em meio a esse frenesi, o silêncio ecoava na casa do professor Carlos como 
um grito ensurdecedor. Era um silêncio sepulcral, como se a própria noite 
tivesse engolido todas as palavras do mestre e deixado apenas os murmúrios 
das sombras como testemunhas da solidão. Estava elegante o professor -- 
dormindo de meias velhas, camiseta furada e calça arrebentada, preparando aula.''

\fonte{Texto formulado para este material.}

\end{myquote}

\num{6} Explique a metáfora do trecho ``o céu era um oceano escuro e piscante''.

\reduline{Nessa metáfora, o autor compara o céu com o oceano. Esse céu é escuro
porque é noite; é piscante por causa do brilho das estrelas.\hfill}

\num{7} Copie do texto um exemplo de personificação e explique por que você
escolheu esse trecho.

\reduline{Existe personificação em``os carros ferozes bestas de metal, rugiam 
pelas ruas famintos por asfalto.'' Nessa passagem, atribuem-se aos carros as ações
de ``rugir'' e ter fome, típicas de seres animados, como animais ou pessoas.\hfill}

\num{8} Explique a figura de linguagem estabelecida pelos termos destacados no trecho 
``o \textsc{\textbf{silêncio}} ecoava na casa do professor Carlos como um 
\textsc{\textbf{grito ensurdecedor}}''.

\reduline{Os termos destacados compõem uma antítese, em que o ``silêncio'' se opõe
ao ``grito ensurdecedor''.\hfill}

\num{9} Explique a ironia do trecho final do texto.

\reduline{No último período do texto, o autor afirma que o professor dormia 
\textit{elegante}, mas a descrição contraria esse adjetivo. Dessa maneira,
o autor declara o inverso do que quer dizer, isto é, que, na verdade, o 
professor se vestia de maneira desleixada.\hfill}

\section*{Treino}

\num{1} Qual é a figura de linguagem presente na frase ``Nunca li Machado de Assis''?

\begin{escolha}

  \item Personificação.
  
  \item Comparação.
  
  \item Hipérbole.
  
  \item Metonímia. 

\end{escolha}

\num{2} Leia um famoso trecho do romance \textit{Iracema}, de José de Alencar.

\begin{myquote}

% \begin{figure}[H]
% \centering
% \includegraphics[scale=0.25]{./imgSAEB_7_POR/media/image52.png}
% \end{figure}

\begin{minipage}{0.6\textwidth}
Além, muito além daquela serra, que ainda azula no
horizonte, nasceu Iracema.

Iracema, a virgem dos lábios de mel, que tinha os
cabelos mais negros que a asa da graúna e mais longos
que seu talhe de palmeira.

O favo da jati não era doce como seu sorriso; nem
a baunilha recendia no bosque como seu hálito perfumado.
\end{minipage}
\hfill
\begin{minipage}{0.4\textwidth}
  \centering
  \includegraphics[width=\textwidth]{./imgSAEB_7_POR/media/image52.png}
  %Fonte Wikipedia https://upload.wikimedia.org/wikipedia/commons/e/ec/Jose_de_Alencar.png
  %\caption{Legenda da imagem}
  %\label{fig:exemplo}
\end{minipage}

\fonte{José de Alencar. Iracema.}

\end{myquote}

% \fonte{José de Alencar. Iracema. 
% Disponível em: http://objdigital.bn.br/Acervo_Digital/Livros_eletronicos/iracema.pdf.
% Acesso em: 22 mai. 2023.}

Para descrever a personagem Iracema, o autor se valeu sobretudo de

\begin{escolha}

  \item repetições de palavras para enfatizar a beleza. 

  \item hipérboles para enaltecer a natureza.

  \item comparações entre a personagem e a natureza.

  \item depreciações da natureza brasileira.

\end{escolha}

\num{3} Leia o trecho do romance \textit{O cortiço}, de Aluísio Azevedo:

%\includegraphics[width=3.51042in,height=3.02083in]{./imgSAEB_7_POR/media/image14.png}

%\fonte{https://tirasarmandinho.tumblr.com/}{\uline{https://tirasarmandinho.tumblr.com/}}.
%Acesso em 20 de Abr de 2023.

\begin{myquote}
% \begin{figure}[H]
% \centering
% \includegraphics[scale=0.25]{./imgSAEB_7_POR/media/image53.png}
% \end{figure}
\begin{minipage}{0.5\textwidth}
Desde que a febre de possuir se apoderou de João Romão totalmente, todos os seus atos, todos, 
fosse o mais simples, visavam um interesse pecuniário. Só tinha uma preocupação: aumentar 
os bens. Das suas hortas recolhia para si e para a companheira os piores legumes, aqueles que,
por maus, ninguém compraria; as suas galinhas produziam muito e ele não comia um ovo, do que, 
no entanto, gostava imenso; vendia-os todos e contentava-se com os restos da comida dos 
trabalhadores. Aquilo já não era ambição, era uma moléstia nervosa, uma loucura, um desespero
de acumular; de reduzir tudo a moeda.
\end{minipage}
\hfill
\begin{minipage}{0.5\textwidth}
  \centering
  \includegraphics[width=\textwidth]{./imgSAEB_7_POR/media/image53.png}
  %Fonte Wikipedia https://upload.wikimedia.org/wikipedia/commons/4/4e/Aluisio_Azevedo.jpg
  %\caption{Legenda da imagem}
  %\label{fig:exemplo}
\end{minipage}

\fonte{Aluísio Azevedo. \textit{O cortiço}.}

\end{myquote}

No trecho acima, o efeito obtido por meio da personificação da ``febre de possuir'' é descrever

\begin{escolha}
  
  \item a simplicidade do trabalhador João Romão. 
  
  \item a miséria de João Romão e sua companheira.
  
  \item a resignação humilde de João Romão. 
  
  \item a ambição João Romão de acumular bens. 

\end{escolha}


\chapter{Tecendo com as palavras: recursos de coesão e progressão textual}
\markboth{Módulo 10}{}

\section*{Habilidades do SAEB}

\begin{itemize}

  \item Analisar os mecanismos que contribuem para a progressão textual.

  \item Analisar os processos de referenciação lexical e pronominal.

\end{itemize}


\subsection{Habilidades da BNCC}

\begin{itemize}

  \item EF07LP12, EF07LP13.

\end{itemize}

\conteudo{A coesão textual é um importante recurso da comunicação escrita.
Para que um texto seja bem escrito e bem compreendido, é necessário que todas
as partes e ideias sejam ordenadas de forma organizada, a fim de produzir um
sentido de conjunto. Para integrar ideias e argumentos que compõem um texto,
alguns recursos são importantes: conjunções e advérbios, por exemplo, servem
para estabelecer relações entre ideias, parágrafos, orações e frases. Em
alguns casos, a fluência da leitura pode exigir o uso de sinônimos. Pronomes
servem para evitar a repetição de palavras, evitando a monotonia e trazendo
maior objetividade ao texto.

Portanto, a coesão textual é essencial para a produção de um texto claro e de
fácil entendimento. Utilizando os recursos adequados e organizando ideias de
maneira linear, é possível criar textos coesos e bem estruturados, que
transmitem as informações de forma eficaz de acordo com o contexto, o público,
a função comunicativa e o gênero textual que se pretende produzir.}

% \coment{Professor(a), converse com os estudantes sobre a coesão textual que
% ocorre de maneira imediata na linguagem oral e coloquial. Chame a
% atenção para o uso de expressões repetidas tais como ``daí'', ``então'' e ``né'' 
% na linguagem coloquial e questione o abuso destes termos e o prejuízo
% para a qualidade da comunicação, especialmente a escrita.

% Chame a atenção para os diversos tipos, funções e classes gramaticais de
% palavras que podem ser usadas a fim de contribuir para estabelecer um
% discurso coeso e bem elaborado.}

\section*{Atividades}

Leia o texto a seguir para responder às questões de 1 a 3.

\begin{myquote}

\begin{figure}[H]
\centering
\includegraphics[width=0.95\textwidth]{./imgSAEB_7_POR/media/image54.png}
\caption{Detalhe da Praça da Sé da cidade de São Paulo que fica localizado na região central da cidade, em frente à Catedral da Sé. Foto: Marcos Santos/USP Imagens}
\end{figure}
%Fonte USP Imagens https://imagens.usp.br/editorias/arquitetura-categorias/centro-de-sao-paulo-capital/attachment/7698_28072011centrosp014/
%Mensagem na página: A reprodução de fotografias é livre mediante a citação do USP Imagens e do nome do fotógrafo

A Praça da Sé é o centro geográfico da capital paulista. No local está o Marco Zero 
e é partir dele que são medidas as distâncias das rodovias e fronteiras estaduais, 
assim como a numeração das vias públicas da cidade. Junto à praça está situada a 
Catedral Metropolitana da Sé. Em estilo gótico modificado, sua construção iniciou-se em 1913.
É a maior igreja de São Paulo, com capacidade para oito mil pessoas em seus 110m de comprimento, 
46m de largura, além de torres com 92m e cúpula com 30m de altura. Em sua cripta encontram-se
obras do escultor Francisco Leopoldo.

\fonte{Governo do Estado de São Paulo. Praça da Sé. 
https://www.saopaulo.sp.gov.br/conhecasp/pontos-turisticos/praca-da-se/.
Acesso em: 1 out. 2023.}

\end{myquote}

\num{1} Na frase ``No \textbf{local} está o Marco Zero e é partir \textbf{dele} que são medidas
as distâncias das rodovias'', a quem se referem os termos destacados?

\reduline{Os termos sublinhados se referem, respectivamente, à Praça da Sé e ao Marco Zero.\hfill}

\num{2} O nome da cidade a que o autor se refere no trecho ``assim como a numeração das vias 
públicas da \textbf{cidade}'' só aparece depois dessa passagem. Explique que cidade é essa e
como foi possível identificá-la no contexto.

\reduline{A cidade a que se refere o autor é a cidade de São Paulo, cujo nome só 
aparece depois do trecho destacado. Ele já havia se referido a essa cidade com a expressão ``capital
paulista''. Além disso,  o contexto em que o texto se insere (o site do governo estadual) também
permite inferir que o autor se refere a São Paulo.\hfill}

\num{3} Quais foram as expressões usadas pelo autor para referir-se à Catedral da Sé?

\reduline{Para referir-se à Catedral da Sé, o autor usou o substantivo ``igreja'' e os pronomes 
possessivos em ``sua construção'', ``seus 110m de comprimento'' e ``sua cripta''.\hfill}

Leia o texto a seguir para responder às questões 4 e 5.

\begin{myquote}

Profissionais da área da educação, da saúde e da assistência social têm
definido ações de cuidado para as comunidades escolares que vivem
situações de violência. Nada fácil, pois a precarização desses setores
tem gerado acúmulo de trabalho e esgotamento.

\fonte{Adriana Marcondes Machado. Jornal da USP. Violência às escolas: reflexões. 
Disponível em: https://jornal.usp.br/artigos/violencia-as-escolas-reflexoes/.
Acesso em: 22 mai. 2023.}

\end{myquote}

\num{4} A expressão ``desses setores'' se refere a que outro termos?

\reduline{A expressão ``desses setores'' se refere às áreas da educação, da saúde e 
da assistência social.\hfill}

\num{5} A expressão ``nada fácil'' se refere a que afirmação?

\reduline{A expressão ``nada fácil'' se refere às ``ações de cuidado para as comunidades escolares que vivem
situações de violência''.\hfill}

Leia o texto abaixo para responder às questões de 6 a 8.

\begin{myquote}

A professora decidiu dividir os alunos do sétimo ano em dois grupos. De um
lado, aqueles que já haviam realizado a prova; de outro, os estudantes que
haviam faltado e que ainda precisavam realizar o exame. Estes deveriam
ser encaminhados para a sala ao lado. O local estava pronto para receber
os estudantes.

\end{myquote}

\num{6} Quais são os termos utilizados como recursos de coesão para
evitar repetição dos termos? 

\reduline{Os termos usados para evitar repetição são ``aqueles'', ``os
jovens'', ``o exame'', ``Estes'', ``O local'' e ``os estudantes''.\hfill}

\num{7} O termo \textbf{os jovens} se refere a um termo anterior, restringindo-o. 
Explique essa afirmação.

\reduline{O termo ``os jovens'' se refere ao antecedente ``os alunos'', mas não a todos eles:
apenas aos que haviam faltado.\hfill}

\num{8} O pronome \textbf{estes} se refere a qual grupo de estudantes?

\reduline{O pronome \textbf{estes} se refere aos estudantes que não haviam realizado a prova.\hfill}

\num{9}Complete as lacunas com as palavras do quadro.

\begin{myquote}

obra garota cujo ela

\end{myquote}

O livro \_\_\_\_\_\_\_\_ autor Maria conhecia, foi premiado em diversos
concursos literários. A \_\_\_\_\_ tratava das questões que interessavam
a \_\_\_\_\_. A \_\_\_\_\_\_\_\_\_ era uma leitora incansável.

% \coment{O livro \textbf{cujo} autor Maria conhecia, foi premiado em diversos
% concursos literários. A \textbf{obra} tratava das questões que
% interessavam a \textbf{ela}. A \textbf{garota} era uma leitora incansável.}

\num{10} Nas frases abaixo substitua o primeiro ponto-final por uma conjunção,
ligando as duas frases de forma coerente. 

\begin{escolha}

  \item O clima da cidade é muito árido. Quando vem o período das chuvas a
  vegetação revigora.

\item\reduline{O clima da cidade é muito árido, mas, quando vem o período das chuvas, 
a vegetação revigora.\hfill}
  
  \item O engarrafamento no centro da cidade se prolongou. Houve inundações em
  vários pontos.

\item\reduline{O engarrafamento no centro da cidade se prolongou, pois houve
inundações em vários pontos.\hfill}
  
  \item Adoraria viajar nas minhas férias. Não tenho dinheiro para isso.

\item\reduline{Adoraria viajar nas minhas férias, mas não tenho dinheiro para isso.\hfill}

\end{escolha}

\section*{Treino}

\num{1} Leia o texto abaixo para responder à pergunta.

\begin{myquote}

% \begin{figure}[H]
% \centering
% \includegraphics[scale=0.25]{./imgSAEB_7_POR/media/image55.png}
% \end{figure}

\begin{minipage}{0.6\textwidth}
Tenho ali na parede o retrato dela, ao lado do marido, tais quais na outra
casa. A pintura escureceu muito, mas ainda dá ideia de ambos. Não me lembra
nada dele, a não ser vagamente que era alto e usava cabeleira grande; o
retrato mostra uns olhos redondos, que me acompanham para todos os lados,
efeito da pintura que me assombrava em pequeno. O pescoço sai de uma gravata
preta de muitas voltas, a cara é toda rapada, salvo um trechozinho pegado às
orelhas. O de minha mãe mostra que era linda. Contava então vinte anos, e
tinha uma flor entre os dedos. No painel parece oferecer a flor ao marido.
\end{minipage}
\hfill
\begin{minipage}{0.35\textwidth}
  \centering
  \includegraphics[width=\textwidth]{./imgSAEB_7_POR/media/image55.png}
  %Fonte Wikipedia https://pt.wikipedia.org/wiki/Dom_Casmurro#/media/Ficheiro:DomCasmurroMachadodeAssis.jpg
  %\caption{Legenda da imagem}
  %\label{fig:exemplo}
\end{minipage}

\fonte{Machado de Assis. \textit{Dom Casmurro}.}

\end{myquote}

Na frase ``Tenho ali na parede o retrato \textsc{\textbf{dela}}'', expressão destacada
se refere

\begin{escolha}

  \item à mãe do narrador.
  
  \item à outra casa. 
  
  \item à pintura que assombrava.
  
  \item à flor entre os dedos.

\end{escolha}


\num{2} Leia o texto abaixo para responder à pergunta.

\begin{myquote}

% \begin{figure}[H]
% \centering
% \includegraphics[scale=0.25]{./imgSAEB_7_POR/media/image56.png}
% \end{figure}

\begin{minipage}{0.6\textwidth}
A Sra. D. Ana, este o nome da avó de Filipe, é uma senhora de espírito e
alguma instrução. Em consideração a seus sessenta anos, ela dispensa tudo
quanto se poderia dizer sobre seu físico. Em suma, cheia de bondade e de
agrado, ela recebe a todos com o sorriso nos lábios; seu coração se pode
talvez dizer o templo da amizade cujo mais nobre altar é exclusivamente
consagrado à querida neta, irmã de Filipe; e ainda mais: seu afeto para com
essa menina não se limita à doçura da amizade, vai ao ardor da paixão.
Perdendo seus pais, quando apenas contava oito anos, a inocente criança tinha,
assim como Filipe, achado no seio da melhor das avós toda a ternura de sua
extremosa mãe.
\end{minipage}
\hfill
\begin{minipage}{0.5\textwidth}
  \centering
  \includegraphics[width=\textwidth]{./imgSAEB_7_POR/media/image56.png}
  %Fonte Wikipedia https://pt.wikipedia.org/wiki/Joaquim_Manuel_de_Macedo#/media/Ficheiro:Joaquim_Manuel_de_Macedo_1866.png
  %\caption{Legenda da imagem}
  %\label{fig:exemplo}
\end{minipage}

\fonte{Joaquim Manuel de Macedo. \textit{A Moreninha}.}

\end{myquote}

No trecho acima, a expressão ``a inocente criança'' se refere

\begin{escolha}
  
  \item à Sra. D. Ana.
  
  \item a Felipe.
  
  \item à irmã de Felipe. 
  
  \item à melhor das avós.

\end{escolha}

\num{3} Leia o texto abaixo para responder à pergunta.

\begin{myquote}

Logo que pôde andar e falar, o garoto tornou-se um flagelo; quebrava e rasgava tudo
que lhe vinha à mão. Tinha uma paixão decidida pelo chapéu do pai;
se este \textsc{\textbf{o}} deixava por esquecimento em algum lugar ao seu alcance, tomava-o
imediatamente, esganava com \textsc{\textbf{ele}} todos os móveis, punha-lhe dentro tudo que
encontrava, esfregava-o em uma parede, e acabava por varrer com \textsc{\textbf{ele}} a casa.

\fonte{Manuel Antônio de Almeida. \textit{Memórias de um Sargento de Milícias}
com adaptações.}

\end{myquote}

% \fonte{Marta Rebós. El País Brasil. O lobo devorou, sim, a Chapeuzinho. 
% https://brasil.elpais.com/brasil/2018/09/18/eps/1537265048_460929.html
% Acesso em: 22 mai. 2023.}

No texto acima, todos os pronomes destacados se referem 

\begin{escolha}

  \item ao garoto.

  \item ao pai. 

  \item ao termo ``tudo que lhe vinha à mão''. 

  \item ao chapéu do pai.

\end{escolha}


\chapter{Variedades linguísticas}
\markboth{Módulo 11}{}

\section*{Habilidades do SAEB}

\begin{itemize}

  \item Analisar as variedades linguísticas em textos.

  \item Avaliar a adequação das variedades linguísticas em contextos de uso.

\end{itemize}

\subsection{Habilidades da BNCC}

\begin{itemize}

  \item EF69LP55, EF69LP56.

\end{itemize}

% conteudo não funciona aqui
\conteudo{A língua é um fenômeno social complexo que se manifesta de diversas
maneiras, de acordo com o contexto. As várias
formas de uso da língua são chamadas de \textbf{variedades linguísticas}, e estão
presentes nas diferenças de pronúncia, vocabulário e situação
comunicativa.

Existem muitas variações da língua falada, pois a linguagem é
influenciada por fatores como a região onde são usadas usadas
expressões, o nível de escolaridade, a idade, a classe social e a etnia
dos falantes. As variações linguísticas podem ser referentes
ao uso coloquial ou formal, ao uso profissional, ou depender de aspectos
geográficos e históricos. Por esses motivos, as variações
se diferenciam da chamada norma-padrão.

Contudo, mesmo que a diversidade seja uma característica
inerente à linguagem, as variações linguísticas podem sofrer preconceitos
e discriminação. Por outro lado, conhecê-las permite a criação de um
discurso mais acessível e atento ao público a que se dirige. No entanto,
considerar determinadas formas de uso da língua inferiores ou
equivocadas pode ser considerado preconceito linguístico.

O preconceito linguístico pode ter diversas consequências negativas,
como a exclusão social de falantes de determinadas variedades
linguísticas, pois não são levados em conta diversos fatores como a
dificuldade de acesso e de oportunidades educacionais e profissionais.

Portanto, é importante valorizar e respeitar a diversidade linguística
como um patrimônio cultural e expressivo que considera todos os falantes
da língua, combatendo os preconceitos e promovendo a inclusão social.}

\section*{Atividades}

% \num{1} Quais são os fatores que devem ser considerados para compreender as
% variações linguísticas existentes entre os falantes de uma mesma

% \reduline{Para compreender as variações linguísticas, é preciso levar em 
% consideração fatores sociais, nível de escolaridade, poder aquisitivo, 
% peculiaridades culturais, étnicas e regionais.\hfill} 

% \num{2} Por que a desvalorização de determinados usos da linguagem pode ser
% considerada preconceito?

% \reduline{A desconsideração das questões que promovem as variações linguísticas
% induz ao preconceito, pois não se pode desconsiderar o contexto do
% emissor e nem a situação comunicativa. Em muitos casos, a norma-padrão
% não se apresenta como a forma mais eficaz de elaborar um texto ou
% discurso. Além disso, a norma-padrão pode servir à manutenção da desigualdade
% social, especialmente em países como o Brasil.\hfill} 

Leia um trecho do romance \textit{Dom Casmurro}, de Machado de Assis:

\begin{myquote}
% \begin{figure}[H]
% \centering
% \includegraphics[scale=0.25]{./imgSAEB_7_POR/media/image57.png}
% \end{figure}

\begin{wrapfigure}{r}{0.55\textwidth}  % 'l' para alinhar à esquerda, 'r' para alinhar à direita
  \centering
  \includegraphics[width=0.55\textwidth]{./imgSAEB_7_POR/media/image57.png}
  % Fonte: Wikipedia https://commons.wikimedia.org/wiki/File:Machado_de_Assis_1904.jpg
  %\caption{\textit{Texto sobre a imagem}}
\end{wrapfigure}

Capitu deixou-se ir, rindo; depois a conversa entrou a cochilar e
dormir. Tínhamos chegado à janela; um preto, que, desde algum
tempo, vinha apregoando cocadas, parou em frente e perguntou:

--- Sinhazinha, qué cocada hoje?

--- Não, respondeu Capitu.

--- Cocadinha tá boa.

--- Vá-se embora, replicou ela sem rispidez.

--- Dê cá! disse eu descendo o braço para receber duas.

Comprei-as, mas tive de as comer sozinho; Capitu recusou. Vi
que, em meio da crise, eu conservava um canto para as cocadas, o
que tanto pode ser perfeição como imperfeição, mas o momento
não é para definições tais; fiquemos em que a minha amiga,
apesar de equilibrada e lúcida, não quis saber de doce, e gostava
muito de doce. Ao contrário, o pregão que o preto foi cantando, o
pregão das velhas tardes, tão sabido do bairro e da nossa infância:

\begin{verse}
Chora, menina, chora, \\
Chora, porque não tem \\
Vintém
\end{verse}

\fonte{Machado de Assis. Dom Casmurro. 
Disponível em: https://www.ic.unicamp.br/~stolfi/misc/2012-02-13-domine-casmurrus.pdf.
Acesso em: 22 mai. 2023.}

\end{myquote}

\num{1} No diálogo entre o vendedor de cocadas, de um lado, e Capitu e Bentinho, 
o narrador do romance, de outro, é possível perceber variações linguísticas. 
Explique essas variações, considerando que, no período em que se passa o romance, o 
vendedor é uma pessoa escravizada no Brasil de meados do século XIX, ao contrário do 
narrador e Capitu, que são brancos com acesso à educação formal. 

\reduline{Fica evidente no diálogo entre o vendedor de cocadas e Capitu e Bentinho que
o abismo social entre essas personagens se manifesta nas variações linguísticas. Enquanto
as falas de Capitu e Bentinho tendem à norma-padrão (por exemplo, no uso do pronome em 
``vá-se embora''), as do vendedor se aproximam da informalidade oral, como em 
``qué cocada hoje?'' e ``Cocadinha tá boa''.\hfill}

\num{2} Nos parágrafos do narrador, qual é a variedade linguística predominante? 
Copie um exemplo de uso de pronomes que confirma essa explicação. 

\reduline{O uso dos pronomes pessoais oblíquos átonos em ``Comprei-as, mas tive de 
as comer sozinho'' corresponde à norma-padrão. Em variedade linguística menos rigorosa,
e mais natural para muitos falantes de língua portuguesa do Brasil, seria usado o pronome
``elas''.\hfill} 

\num{3} O que são os \textbf{pregões} citados pelo narrador? Ainda existem pregões no
seu cotidiano? Se sim, explique; se não, levante hipóteses sobre o desaparecimento dessa 
prática.

\reduline{Pregões são divulgações em voz alta de produtos à venda, por vendedores ambulantes. 
Feirantes de rua ainda costumam usar pregões, bem como vendedores ambulantes nas praias de 
todo o Brasil. Contudo, o ambiente do comércio em shopping centers, mais formal, não contempla
essa prática.\hfill} 

%Analise a tirinha e responda:

%\includegraphics[width=5.90551in,height=1.70833in]{./imgSAEB_7_POR/media/image15.png}

%\num{6}
%  A tirinha apresenta um tipo de variação linguística? Justifique sua resposta
%\end{enumerate}

%Sim. A tirinha apresenta uma variação linguística pois apresenta vários
%nomes de um mesmo vegetal, cada forma é utilizada em uma região
%diferente do país

%\num{7}
%  Alguma das formas mandioca, aipim ou macaxeira pode ser considerada
%  mais correta do que as outras? Por que?
%\end{enumerate}

%Não, pois cada região possui suas formas de falar, suas gírias e
%expressões e muitas vezes, podem haver vários sinônimos em uma mesma
%língua usado em regiões e culturas diferentes.

% \num{6} Na tabela a seguir, assinale (V) para as afirmações verdadeiras e (F) para as falsas no que
% se refere às variedades linguísticas. 

% \begin{table}[h!]
% \begin{tabular}{|c|c|}
% \hline
% \textbf{\begin{tabular}[c]{@{}c@{}}Verdadeiro (V)\\ ou Falso (F)?\end{tabular}} & \textbf{Afirmação} \\ \hline
% \rosa{F} & toda situação comunicativa exige o uso da norma-padrão \\ \hline
% \rosa{V} & \begin{tabular}[c]{@{}c@{}}o uso da norma-padrão é mais adequado para a redação de\\ documentos oficiais, textos de lei e petições\end{tabular} \\ \hline
% \rosa{F} & textos literários devem sempre fazer uso da norma-padrão \\ \hline
% \rosa{V} & em algumas situações, é mais adequado não utilizar a norma-padrão \\ \hline
% \rosa{V} & \begin{tabular}[c]{@{}c@{}}a escolha pelo uso da norma-padrão ou de variações que se distanciam dela\\  deve orientar-se pelo contexto da comunicação\end{tabular} \\ \hline
% \end{tabular}
% \end{table}

% % Gabarito: \coment{F, V, F, V, V}

% \num{7} Na tabela abaixo, assinale com um X os gêneros textuais em que o uso da norma-padrão 
% tende a ser mais adequado.

% \begin{table}[h!]
% \begin{tabular}{l|l|}
% \cline{2-2}
%  & \textbf{Gêneros textuais} \\ \hline
% \multicolumn{1}{|l|}{} & cartas pessoais \\ \hline
% \multicolumn{1}{|l|}{\rosa{x}} & cartas abertas à sociedade \\ \hline
% \multicolumn{1}{|l|}{\rosa{x}} & petições \\ \hline
% \multicolumn{1}{|l|}{\rosa{x}} & textos de divulgação científica \\ \hline
% \multicolumn{1}{|l|}{\rosa{x}} & leis e estatutos \\ \hline
% \multicolumn{1}{|l|}{} & poemas \\ \hline
% \end{tabular}
% \end{table}

% \num{8} Na tabela abaixo, assinale com um X as situações comunicativas em que tende a ser mais adequado
% o uso da linguagem coloquial.

% \begin{table}[h!]
% \begin{tabular}{l|l|}
% \cline{2-2}
%  & \textbf{Situações comunicativas} \\ \hline
% \multicolumn{1}{|l|}{\rosa{x}} & conversa com amigos \\ \hline
% \multicolumn{1}{|l|}{} & matérias de jornal \\ \hline
% \multicolumn{1}{|l|}{} & apresentação de trabalho escolar \\ \hline
% \multicolumn{1}{|l|}{\rosa{x}} & mensagens de redes sociais \\ \hline
% \end{tabular}
% \end{table}

Leia o texto abaixo para responder às questões de 4 a 10.

\begin{myquote}

\textbf{Brasília, 60 anos: as gírias da capital federal}

\begin{figure}[H]
\centering
\includegraphics[width=0.95\textwidth]{./imgSAEB_7_POR/media/image58.png}
\end{figure}
%Fonte Freepick https://br.freepik.com/fotos-gratis/felizes-alegres-amigos-conversando-e-rindo_5546747.htm#query=jovens%20rindo&position=2&from_view=search&track=ais

Como a própria língua portuguesa, as gírias também são vivas e mudam com o
passar do tempo; e em 60 anos, as gerações de Brasília já usaram muitas delas,
mas quais são genuínas da capital federal? Segundo o professor e
sociolinguista Newton Lima Neto, mais importante que saber se nasceram aqui,
ou vieram de outro lugar, é entender se há consistência no uso delas.

``A gente dificilmente chegará a uma resposta se `véi' é uma gíria tipicamente
brasiliense. Você pode ir no interior de São Paulo e encontrar também essa
gíria, mas como que ela é falada lá? Em Brasília, você altera a entonação e
você pode mudar completamente o sentido da frase''”, explica.

As gírias são palavras que ganham novos sentidos entre um grupo de pessoas.
Segundo Newton, elas fazem parte do dialeto, que inclui sotaque, ritmo e
palavras. Muitas gírias são lançadas entre grupos de pessoas que vivem em uma cidade,
mas nem todas vão realmente compor o repertório de expressões daquele lugar.
Newton lembra do exemplo da gíria ``camelo'' empregada para identificar a
bicicleta, que ganhou fama em música da banda Legião Urbana. ``Talvez aquele
pequeno grupo de jovens usava camelo, mas um deles ficou famoso. Então, a
gente não pode atribuir necessariamente camelo como uma expressão tipicamente
brasiliense, porque se você escuta a fala dos jovens hoje, quem é que está
usando isso?'', diz.

\fonte{EBC. Brasília, 60 anos: as gírias da capital federal. Disponível em:
https://memoria.ebc.com.br/cultura/2020/04/brasilia-60-anos-girias-da-capital-federal.
Acesso em: 2 out.2023. com adaptações}

\end{myquote}

\num{4} De acordo com as afirmações do primeiro parágrafo, as gírias podem
mudar ao longo do tempo. Pergunte às pessoas mais velhas quais são as 
gírias que elas usavam na juventude e que não são mais usadas. Faça uma
lista das gírias que você conhece e que os adultos não sabem usar. 

\reduline{Resposta pessoal do aluno\hfill}
\linhas{10}

\num{5} No segundo parágrafo, o professor afirma que a mesma gíria pode
mudar de significado de acordo com o contexto. Faça uma lista de gírias
que surgiram com o uso da internet e das redes sociais.

\reduline{Resposta pessoal do aluno. Alguns exemplos de gírias que
pode ser usadas nesta resposta são LOL, shippar, stalkear e trollar.\hfill}
\linhas{10}

\num{6} No terceiro parágrafo, o mesmo professor afirma que um pequeno
grupo de pessoas pode criar as suas próprias gírias. Você e seus amigos 
criaram alguma palavra ou expressão que apenas vocês entendam? E sua 
família? 

\reduline{Resposta pessoal do aluno\hfill}
\linhas{10}

\section*{Treino}

\num{1} Leia os versos do coco abaixo para responder à questão.

\begin{myquote}
% \begin{figure}[H]
% \centering
% \includegraphics[scale=0.25]{./imgSAEB_7_POR/media/image59.png}
% \caption{Domínio público / Acervo Arquivo Nacional}
% \end{figure}
%Fonte: Wikipedia https://upload.wikimedia.org/wikipedia/commons/1/13/Mario_de_andrade_1928b.png
\begin{verse}
\begin{minipage}{0.45\textwidth}
Eu vi uma lagartixa \\
Ai, tava numa jinela, \\
Ai, dizeno que era honrada, \\
Que era moça donzela, \\
Vi quat' calango verde, \\
Tudo era fio dela.
\end{minipage}
\hfill
\begin{minipage}{0.4\textwidth}
  \centering
  \includegraphics[width=\textwidth]{./imgSAEB_7_POR/media/image59.png}
  %Fonte: Wikipedia https://upload.wikimedia.org/wikipedia/commons/1/13/Mario_de_andrade_1928b.png
  %\caption{Legenda da imagem}
  %\label{fig:exemplo}
\end{minipage} 
\end{verse}
\fonte{Mário de Andrade. Os cocos. Belo Horizonte: Itatiaia, 2002. p. 157.}

\end{myquote}

O coco é uma dança popular de roda, que se executa ao som de canto, batida de palmas e 
toque de percussão. Nos versos acima, registrados pelo escritor e pesquisador Mário de 
Andrade, observa-se a transcrição da fala coloquial em

\begin{escolha}
  
  \item ``eu vi''.
  
  \item ``era honrada''.
  
  \item ``donzela''.
  
  \item ``fio dela''

\end{escolha}

\num{2} Leia o texto abaixo para responder à questão.

\begin{myquote}
Nota-se também que as diferentes regiões brasileiras não apresentam um
cenário socioeconômico igual, o que afeta a frequência do câncer e de
outras doenças. ``Pensar em regionalização é essencial. Nas regiões Norte
e Nordeste, por exemplo, esse tipo de câncer não é tão frequente como em
outros espaços do País. Quando você pensa em um projeto de prevenção, é
necessário pensar em regionalização'', discorre Hoff.''

\fonte{Jornal da USP. Câncer de intestino já é mais comum em grupos de pessoas mais jovens.
Disponível em: https://jornal.usp.br/radio-usp/cancer-de-intestino-ja-e-mais-comum-em-grupos-de-pessoas-mais-jovens/.
Acesso em: 22 mai. 2023.}
\end{myquote}

No exemplo acima, o autor usou da norma-padrão para adequar-se à linguagem apropriada ao texto

\begin{escolha}
  
  \item literário.
  
  \item humorístico.
  
  \item jornalístico.
  
  \item jurídico.

\end{escolha}

\num{3} Leia o trecho do romance \textit{A Família Medeiros}, de Júlia Lopes de Almeida, para 
responder à questão. 

\begin{myquote}

% \begin{figure}[H]
% \centering
% \includegraphics[scale=0.25]{./imgSAEB_7_POR/media/image59.png}
% \caption{Domínio público / Acervo Arquivo Nacional}
% \end{figure}

\begin{minipage}{0.55\textwidth}
Passada a curva do jequitibá grande, topou com um negro
que vencia o morro a largas passadas e que o saudou com um
soturno:
--- Sum Cristo!
--- Sabe-me dizer se há algum atalho novo para a Fazenda de
Santa Genoveva?
--- Prá o sito do coroné Medêro? Não há não, sinhô. Eu
também tô indo prá lá.
E atentando em Otávio:
--- A modo qui tô conhecendo mecê \ldots{}
--- Está, sim. E você, como se chama?
--- Me chamo Antonho; fio de Luzia pernambucana, sim sinhô.
--- Foi a algum recado à cidade?
--- Fui na vila buscá rémedio pró fio do feitô, qui foi moldido
de cobra, sim sinhô \ldots{}
--- Ah, então apresse-se --- disse Otávio, para dizer alguma
coisa, e tocou o animal para diante.
\end{minipage}
\hfill
\begin{minipage}{0.4\textwidth}
  \centering
  \includegraphics[width=\textwidth]{./imgSAEB_7_POR/media/image60.png}
  %Fonte: Wikipedia https://commons.wikimedia.org/wiki/File:Júlia_Lopes_de_Almeida,_sem_data.tif
  %\caption{Legenda da imagem}
  %\label{fig:exemplo}
\end{minipage} 

\fonte{Júlia Lopes de Almeida. \textit{A Família Medeiros}. São Paulo: Editora Hedra. no prelo.}

\end{myquote}

No diálogo acima, observa-se a transcrição da fala coloquial em:

\begin{escolha}
  
  \item ``vencia o morro a largas passadas''.
  
  \item ``Sabe-me dizer se há algum atalho novo''. 
  
  \item ``A modo qui tô conhecendo mecê''.
  
  \item ``então apresse-se''. 

\end{escolha}
