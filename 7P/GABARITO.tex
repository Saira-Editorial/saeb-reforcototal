%!TEX root=./LIVRO.tex
\chapter{Respostas}
\pagestyle{plain}
\footnotesize

\pagecolor{gray!40}

\section*{Módulo 1 – Treino}

\begin{enumerate}
\item
SAEB: Identificar teses, opiniões, posicionamentos explícitos e
argumentos em textos.

BNCC: EF67LP05 -- Identificar e avaliar teses/opiniões/posicionamentos explícitos 
e argumentos em textos argumentativos (carta de leitor, comentário, artigo de opinião,
resenha crítica etc.), manifestando concordância ou discordância.

a) Correta. Segundo o texto, ``a migração de áreas endêmicas desempenha um papel 
significativo na propagação da malária'', de modo que apenas contribuições internacionais
podem erradicar a doença.
b) Incorreta. Não há, no texto, referência à ausência de metas nacionais para eliminação 
da malária.
c) Incorreta. O conjunto do texto permite afirmar que há correlação entre casos de malária 
e migração. 
d) Incorreta. Não se pode afirmar, de acordo com o texto, que o mundo da economia e do trabalho
é pouco saudável.

\item
SAEB: Identificar o uso de recursos persuasivos em textos verbais e não
verbais.

BNCC: EF67LP07 -- Identificar o uso de recursos persuasivos em
textos argumentativos diversos (como a elaboração do título, escolhas
lexicais, construções metafóricas, a explicitação ou a ocultação de
fontes de informação) e perceber seus efeitos de sentido.

a) Incorreta. Para o autor, a organização de tempos, espaços, grupos e 
relações não deveria ser \textit{ensinada}, mas \textit{discutida}
por alunos, professores e escola.
b) Incorreta. De acordo com a opinião expressa no texto, era preciso acolher 
professores e alunos e debater com eles a organização de tempos, espaços, grupos e 
relações. Não há referências a palestras.
c) Correta. Acolher estudantes e professores e discutir com eles modos de organização
corresponde são condutas esperadas pelo autor do texto.
d) Incorreta. Para o autor, as tradições escolares, sobretudo as avaliações de deficiências,
são fatores que levaram à violência e à falta de saúde nas escolas.

\item
SAEB: Identificar teses, opiniões, posicionamentos explícitos e
argumentos em textos.

BNCC: EF67LP07 -- Identificar o uso de recursos persuasivos em textos
argumentativos diversos (como a elaboração do título, escolhas lexicais,
construções metafóricas, a explicitação ou a ocultação de fontes de
informação) e perceber seus efeitos de sentido.

a) Correta. Segundo o texto, os aspectos culturais e afetivos devem ser
levados em conta para que a produção e consumo de alimentos saudáveis
seja estimulada.
b) Incorreta. Não há alusão, no texto, ao acesso a mercados e pontos de consumo.
c) Incorreta. O texto não contém referência a essa afirmação.
d) Incorreta. O texto não contém referência a essa afirmação.

\end{enumerate}

\section*{Módulo 2 – Treino}

\begin{enumerate}

\item
SAEB: Identificar elementos constitutivos de gêneros de divulgação
científica.

BNCC: EF69LP02 -- Analisar e comparar peças publicitárias variadas
(cartazes, folhetos, outdoor, anúncios e propagandas em diferentes
mídias, spots, jingle, vídeos etc.), de forma a perceber a articulação
entre elas em campanhas, as especificidades das várias semioses e
mídias, a adequação dessas peças ao público-alvo, aos objetivos do
anunciante e/ou da campanha e à construção composicional e estilo dos
gêneros em questão, como forma de ampliar suas possibilidades de
compreensão (e produção) de textos pertencentes a esses gêneros.

a) Incorreta. Embora o texto contenha alguns vocábulos ligados à área médica, não se pode 
afirmar que as informações veiculadas sejam voltadas apenas ao público especializado.
b) Correta. O texto contém explicações sobre os benefícios do exercício físico para idosos.
c) Incorreta. O texto não contém citações de artigos científicos. Há apenas falas curtas de um
especialista no assunto.
d) Incorreta. O texto não contém narrativas pessoais sobre os processos inerentes ao envelhecimento.

\item
SAEB:Analisar a relação temática entre diferentes gêneros jornalísticos.

a) Incorreta. O primeiro exemplo traz uma notícia sobre o número de casos
e segundo exemplo traz indicações de ações para prevenir a
proliferação do mosquito.
b) Correta. O primeiro exemplo é uma
notícia sobre o número de casos de dengue em 2023; o segundo contém
indicações de ações de prevenção e combate à proliferação do
mosquito que transmite a doença.
c) Incorreta. Embora o segundo exemplo possa ser considerado um texto
informativo, o primeiro exemplo não é um texto de opinião.
d) Incorreta. O primeiro exemplo não traz dados científicos e nem possui
linguagem científica e técnica sobre o assunto; o segundo texto contém
indicações de ações de prevenção e combate à proliferação do
mosquito que transmite a doença, e não é
argumentativo.

\item
SAEB: Identificar formas de organização de textos normativos, legais
e/ou reivindicatórios.

BNCC: EF69LP27 -- Analisar a forma composicional de textos
pertencentes a gêneros normativos/ jurídicos e a gêneros da esfera
política, tais como propostas, programas políticos (posicionamento
quanto a diferentes ações a serem propostas, objetivos, ações previstas
etc.), propaganda política (propostas e sua sustentação, posicionamento
quanto a temas em discussão) e textos reivindicatórios: cartas de
reclamação, petição (proposta, suas justificativas e ações a serem
adotadas) e suas marcas linguísticas, de forma a incrementar a
compreensão de textos pertencentes a esses gêneros e a possibilitar a
produção de textos mais adequados e/ou fundamentados quando isso for
requerido.

a) Incorreta. Os trechos não são contraditórios. O segundo trecho
apenas trata de uma parcela da população em particular enquanto o
primeiro trata do conjunto da população brasileira.
b) Incorreta. Os trechos apresentam relação direta já que a
população indígena faz parte da população brasileira e como tal também
tem direitos civis garantidos pela lei.
c) Incorreta. Os dois trechos versam sobre direitos essenciais, portanto
não tratam de questões distintas.
d)Correta. Os dois trechos apresentam relação de complementaridade, pois
as disposições presentes no segundo trecho apenas especificam direitos
dos povos indígenas e deveres do Estado brasileiro para com essa
população a fim de garantir as obrigações do Estado brasileiro dispostas
no Artigo 3º.

\end{enumerate}

\section*{Módulo 3 – Treino}

\begin{enumerate}

\item
SAEB: Inferir a presença de valores sociais, culturais e humanos em
textos literários.

BNCC: EF69LP44 -- Inferir a presença de valores sociais,
culturais e humanos e de diferentes visões de mundo, em textos
literários, reconhecendo nesses textos formas de estabelecer múltiplos
olhares sobre as identidades, sociedades e culturas e considerando a
autoria e o contexto social e histórico de sua produção.

a) Correta. O trecho faz alusão à passagem das quatro estações do ano
   exemplificadas pelo ciclo natural da fruta murici, além de referir-se a
   períodos de seca e cheia, que permitem inferir a passagem de cerca de um
   ano.
b) Incorreta. Por meio das características citadas, tais como ocorrência de
   chuvas e seca, pode-se inferir que um ano se passou.
c) Incorreta. O trecho faz alusão a diversas características das estações do
   ano, portanto mais de uma estação se passou, indicando a passagem de um
   ano.
d) Incorreta. O trecho faz alusão à passagem das quatro estações do ano
   exemplificadas pelo ciclo natural da fruta murici, além de referir-se a
   períodos de seca e cheia, que permitem inferir a passagem de cerca de um
   ano.

\item
SAEB: Analisar a intertextualidade entre textos literários ou entre
estes e outros textos verbais ou não verbais.

BNCC: EF67LP27 -- Analisar, entre os textos literários e entreestes e outras manifestações artísticas (como cinema, teatro, música,
artes visuais e midiáticas), referências explícitas ou implícitas a
outros textos, quanto aos temas, personagens e recursos literários e
semióticos

a) Incorreta. Traços de intertextualidade não se configuram como plágio.
b) Incorreta. Além do título, existem diversas alusões de Murilo Mendes ao 
poema de Gonçalves Dias.
c) Correta. O tipo de intertextualidade entre os dois poemas é chamado de
paródia, que pode satirizar, de criticar ou de homenagear o texto de referência.
Neste caso, Murilo Mendes critica o nacionalismo no Romantismo
brasileiro e junto ao movimento modernista propõe uma nova forma de
nacionalismo que valoriza a cultura brasileira.
d)  O poema de Gonçalves Dias foi escrito cerca de 100 anos antes,
portanto não pode ser baseado no poema de Murilo Mendes.

\item
SAEB: Analisar elementos constitutivos de textos pertencentes ao domínio
literário.

a) Incorreta. Na poesia concreta, pode haver combinação de palavras e imagens 
para a criação dos sentidos do poema.
b) Incorreta. Não se verificam versos rimados e estrofes tradicionais na poesia
visual. 
c) Incorreta. As imagens podem fazer parte da constituição do poema e contribuir
para a produção de sentido.
d) Correta. A disposição de letras e palavras na página é, de fato, recurso de
produção de sentidos do poema.

\end{enumerate}

\section*{Módulo 4 – Treino}

\begin{enumerate}

\item
SAEB: Analisar efeitos de sentido produzido pelo uso de formas de
apropriação textual (paráfrase, citação etc.).

BNCC: EF69LP43 -- Identificar e utilizar os modos de introdução de
outras vozes no texto -- citação literal e sua formatação e paráfrase
--, as pistas linguísticas responsáveis por introduzir no texto a
posição do autor e dos outros autores citados (``Segundo X; De acordo
com Y; De minha/nossa parte, penso/amos que''...) e os elementos de
normatização (tais como as regras de inclusão e formatação de citações e
paráfrases, de organização de referências bibliográficas) em textos
científicos, desenvolvendo reflexão sobre o modo como a
intertextualidade e a retextualização ocorrem nesses textos.

a) Incorreta. O autor não defende o abandono dos telefones celulares pelos usuários; 
ele sugere uso moderado desses aparelhos no último parágrafo.
b) Incorreta. O autor alude a problemas de saúde causados pelo uso execssivo de celulares.
c) Incorreta. De fato, o autor expõe vantagens e desvantagens de usar o celular ao longo
do texto, mas se posiciona claramente no último parágrafo, ao sugerir uso moderado desses aparelhos.
d) Correta. No último parágrafo, o autor sugere que o uso dos celulares seja moderado, sem exagero 
nem descarte.

\item
SAEB: Analisar efeitos de sentido produzido pelo uso de formas de
apropriação textual (paráfrase, citação etc.).

BNCC: EF69LP43 -- Identificar e utilizar os modos de introdução de
outras vozes no texto -- citação literal e sua formatação e paráfrase
--, as pistas linguísticas responsáveis por introduzir no texto a
posição do autor e dos outros autores citados (``Segundo X; De acordo
com Y; De minha/nossa parte, penso/amos que''...) e os elementos de
normatização (tais como as regras de inclusão e formatação de citações e
paráfrases, de organização de referências bibliográficas) em textos
científicos, desenvolvendo reflexão sobre o modo como a
intertextualidade e a retextualização ocorrem nesses textos.

a) Correta. No segundo parágrafo, o autor do texto afirma que ``o Brasil tem
   condições de crescer a cada ano no ranking, por meio de investimentos e
   políticas direcionadas à ciência, tecnologia e inovação complementar'';
   essa afirmação é complementada pela citação entre aspas.
b) Incorreta. A citação entre aspas não contém explicação sobre as
   classificações insatisfatórias do Brasil no passado.
c) Incorreta. A subida de posição do Brasil no Índice Global de Inovação é
   celebrada, de forma implícita, no primeiro parágrafo.
d) Incorreta. A citação entre aspas não contradiz a afirmação de que o Brasil
   ainda pode crescer no ranking ao contrário: reafirma essa possibilidade.

\item
SAEB: Analisar os efeitos de sentido decorrentes dos mecanismos de construção
de textos jornalísticos/midiáticos.

a) Incorreta. Segundo as afirmações do primeiro parágrafo, o Brasil está na vanguarda da discussão sobre 
segurança do Turismo de Aventura porque implementou medidas de segurança nessa categoria que servira de
referência para países estrangeiros.
b) Incorreta. Segundo as afirmações do primeiro parágrafo, o Brasil está na vanguarda da discussão sobre 
segurança do Turismo de Aventura porque implementou medidas de segurança nessa categoria que servira de
referência para países estrangeiros.
c) Incorreta. Segundo as afirmações do primeiro parágrafo, o Brasil está na vanguarda da discussão sobre 
segurança do Turismo de Aventura porque implementou medidas de segurança nessa categoria que servira de
referência para países estrangeiros.
d) Correta. Segundo as afirmações do primeiro parágrafo, o Brasil está na vanguarda da discussão sobre 
segurança do Turismo de Aventura porque implementou medidas de segurança nessa categoria que servira de
referência para países estrangeiros.

\end{enumerate}

\section*{Módulo 5 – Treino}

\begin{enumerate}

\item
SAEB: Distinguir fatos de opiniões em textos.

a) Incorreta. O autor não afirma que os cinemais são mais confortáveis do que nossas próprias casas.
b) Incorreta. De fato, o autor afirma que a experiência de assistir a um filme no cinema é única,
mas não se refere ao ``compartilhamento de dados pessoais''. 
c) Correta. Para o autor, o ambiente do cinema favorece a concentração no filme, sem desvios de atenção,
e o convívio com outras pessoas.
d) Incorreta. Segundo o autor, o cinema favoree o convívio das pessoas -- não o afastamento. 

\item
SAEB:Inferir informações implícitas em distintos textos.
BNCC: EF67LP04 -- Distinguir, em segmentos descontínuos de
textos, fato da opinião enunciada em relação a esse mesmo fato.

a) Incorreta. Em nenhum trecho do texto o autor sugere proibição do uso de celulares.
b) Incorreta. Em nenhum trecho do texto o autor sugere limitação do tempo gasto em atividades presenciais.
c) Correta. O autor sugere, no último período do texto, monitoramento e
limites claros para impedir o exagero dos filhos no uso de celulares.
a) Incorreta. Em nenhum trecho do texto o autor sugere suspensão do uso de celulares.

\item
SAEB: Distinguir fatos de opiniões em textos.
BNCC: EF67LP04 -- Distinguir, em segmentos descontínuos de textos,
fato da opinião enunciada em relação a esse mesmo fato.

a)Incorreta. Segundo as afirmações do texto, a obra de Lima Barreto segue atual porque os temas
que encontramos em seus livros ainda estão presentes na sociedade brasileira.     
a)Incorreta. Segundo as afirmações do texto, a obra de Lima Barreto segue atual porque os temas
que encontramos em seus livros ainda estão presentes na sociedade brasileira. 
a)Correta. Segundo as afirmações do texto, a obra de Lima Barreto segue atual porque os temas
que encontramos em seus livros ainda estão presentes na sociedade brasileira. 
a)Incorreta. Não se afirma, no texto, que a obra de Lima Barreto contribuiu para o Brasil se tornar uma sociedade mais justa e igualitária.

\end{enumerate}

\section*{Módulo 6 – Treino}

\begin{enumerate}

\item
SAEB: Inferir, em textos multissemiótico, efeitos de humor, ironia e/ou
crítica.

a) Correta. Nos primeiros períodos do parágrafo, a repetição de 
``nem todos gostavam'', seguida de expressões que se referem à brutalidade
com que os escravizados eram tratados, dirige o olhar do leitor à violência
por eles vivida, naturalizando a linguagem da brutalidade e resultando na 
ironia, por meio da qual Machado de Assis critica a escravidão.
b) Incorreta. Embora haja passagens descritivas da realidade no fragmento, 
a crítica de Machado de Assis só se constitui por meio das ironias no conjunto. 
c) Incorreta. A nostalgia observável em alguns trechos não é suficiente para
estabelecer a crítica, que só terá lugar por meio da naturalização da linguagem
violenta da escravidão. 
d) Incorreta. Não se observa, no texto, a valorização dos escravizados. Evidentemente
Machado de Assis reproduz, no conjunto, a linguagem que os depreciava, evidenciando
a naturalização da brutalidade contra eles.

\item
SAEB: Inferir, em textos multissemiótico, efeitos de humor, ironia e/ou
crítica

BNCC: EF69LP05 -- Inferir e justificar, em textos multissemióticos --
tirinhas, charges, memes, gifs etc. --, o efeito de humor, ironia e/ou
crítica pelo uso ambíguo de palavras, expressões ou imagens ambíguas, de
clichês, de recursos iconográficos, de pontuação etc.

a) Incorreta. A força da campanha é dada pelo jogo entre as palavras
``alvo'' e ``foco''.
b)Incorreta. A força da campanha é dada pelo jogo entre as palavras
``alvo'' e ``foco''.
c)Incorreta. A força da campanha é dada pelo jogo entre as palavras
``alvo'' e ``foco''.
d) Correta. A força da campanha é dada pelo jogo entre as palavras
``alvo'' e ``foco''. Em primeiro lugar, podemos perceber que elas duas
se referem à visão. Além disso, a palavra ``foco'' também pertence ao 
vocabulário típico da doença da dengue: o foco é o lugar de onde se propaga a doença. 
Finalmente: deixaremos de estar ``na mira'' do mosquito (não seremos ``alvo'')
se eliminarmos o ``foco'' (não no sentido de perdermos nosso ponto principal, mas no de
destruírmos os lugares em que o mosquito se reproduz). 

\item
SAEB: Inferir, em textos multissemiótico, efeitos de humor, ironia e/ou
crítica

BNCC: EF69LP05 -- Inferir e justificar, em textos multissemióticos --
tirinhas, charges, memes, gifs etc. --, o efeito de humor, ironia e/ou
crítica pelo uso ambíguo de palavras, expressões ou imagens ambíguas, de
clichês, de recursos iconográficos, de pontuação etc.

a) Correta. No meme, a palavra ``casa'' pode assumir dois sentidos. O primeiro
é figurado: ``casa'' equivale a ``família''; o segundo é literal: a educação
viria da \textit{casa} propriamente dita, que se opõe ao apartamento. A criança 
estaria, assim, ironizando e relativizando, de forma sagaz, a máxima de que 
``a educação vem de casa'' (expressão na qual ``casa'' assume o primeiro dos 
sentidos). Morando em apartamento, o garoto considera o segundo sentido da palavra
``casa'', dispensa a si mesmo da educação e faz a bagunça retratada na imagem.    
b) Incorreta. O meme não contém crítica à falta de moradia.
c) Incorreta. Não há elementos que permitam inferir que o meme contém propaganda 
subliminar. 
d)Incorreta. Não há elementos que permitam inferir que o meme questiona a qualidade 
de vida dos moradores de apartamentos.
\end{enumerate}

\section*{Módulo 7 – Treino}

\begin{enumerate}

\item
SAEB: Analisar marcas de parcialidade em textos jornalísticos.

a) Incorreta. A afirmação contida nesta alternativa contém um fato apresentado no texto, 
sem marca explícita de parcialidade. 
b) Incorreta. A afirmação contida nesta alternativa contém um fato apresentado no texto, 
sem marca explícita de parcialidade. 
c) Incorreta. A afirmação contida nesta alternativa contém um fato apresentado no texto, 
sem marca explícita de parcialidade.
d) Correta. Na frase  ``É inegável, dessa forma, que devemos dar prioridade'' há duas marcas
explícita de parcialidade: as locuções ``é inegável'' e ``devemos dar'', que indicam de maneira
flagrante o ponto de vista pessoal do autor. 

\item
SAEB:Avaliar diferentes graus de parcialidade em textos jornalísticos.

BNCC: EF67LP04 -- Distinguir, em segmentos descontínuos de textos,
fato da opinião enunciada em relação a esse mesmo fato.

a) Na frase  ``Comida africana na cidade do Rio de Janeiro não é algo trivial'', observa-se marca
explícita de parcialidade na expressão ``não é algo trivial''. 
b) Incorreta. A afirmação contida nesta alternativa contém um fato apresentado no texto, 
sem marca explícita de parcialidade. 
c) Incorreta. A afirmação contida nesta alternativa contém um fato apresentado no texto, 
sem marca explícita de parcialidade. 
d) Incorreta. A afirmação contida nesta alternativa contém um fato apresentado no texto, 
sem marca explícita de parcialidade. 

\item
SAEB:Analisar marcas de parcialidade em textos jornalísticos.

BNCC: EF67LP04 -- Distinguir, em segmentos descontínuos de textos,
fato da opinião enunciada em relação a esse mesmo fato.

a) Incorreta. A afirmação contida nesta alternativa contém um fato apresentado no texto, 
sem marca explícita de parcialidade. 
b) Incorreta. A afirmação contida nesta alternativa contém um fato apresentado no texto, 
sem marca explícita de parcialidade. 
c) Incorreta. A afirmação contida nesta alternativa contém um fato apresentado no texto, 
sem marca explícita de parcialidade.
d) Correta. Na frase  ``não são suficientes para sustentar o interesse'' observa-se marca
explícita de parcialidade na expressão ``não são suficientes''. 

\end{enumerate}

\section*{Módulo 8 – Treino}

\begin{enumerate}

\item
SAEB: Analisar os efeitos de sentido produzidos pelo uso de modalizadores em textos diversos.

BNCC: EF07LP14 -- Identificar, em textos, os efeitos de sentido do
uso de estratégias de modalização e argumentatividade.

a) Incorreta. No trecho em que se insere, a expressão indicada opõe a afirmação anterior, sobre
a obrigatoriedade do uso de máscaras decretada pelo governo, e a posterior, ``ainda há dúvidas 
de parte da população quanto à necessidade e ao benefício do seu uso''. 
b) Incorreta. No trecho em que se insere, a expressão indicada opõe a afirmação anterior, sobre
a obrigatoriedade do uso de máscaras decretada pelo governo, e a posterior, ``ainda há dúvidas 
de parte da população quanto à necessidade e ao benefício do seu uso''.
c) Correta. No trecho em que se insere, a expressão indicada opõe a afirmação anterior, sobre
a obrigatoriedade do uso de máscaras decretada pelo governo, e a posterior, ``ainda há dúvidas 
de parte da população quanto à necessidade e ao benefício do seu uso''.
c)Incorreta. No trecho em que se insere, a expressão indicada opõe a afirmação anterior, sobre
a obrigatoriedade do uso de máscaras decretada pelo governo, e a posterior, ``ainda há dúvidas 
de parte da população quanto à necessidade e ao benefício do seu uso''.

\item
SAEB:Analisar os efeitos de sentido dos tempos, modos e/ou vozes verbais
com base no gênero textual e na intenção comunicativa.

a) Incorreta. A forma verbal ``deve'', no contexto em que se insere, expressa possibilidade.
b) Incorreta. A forma verbal ``deve'', no contexto em que se insere, expressa possibilidade.
c) Incorreta. A forma verbal ``deve'', no contexto em que se insere, expressa possibilidade.
d) Correta. A forma verbal ``deve'', no contexto em que se insere, expressa possibilidade.

\item
SAEB: Analisar os efeitos de sentido dos tempos, modos e/ou vozes
verbais com base no gênero textual e na intenção comunicativa.

BNCC: EF69LP04 -- Identificar e analisar os efeitos de sentido
que fortalecem a persuasão nos textos publicitários, relacionando as
estratégias de persuasão e apelo ao consumo com os recursos
linguístico-discursivos utilizados, como imagens, tempo verbal, jogos de
palavras, figuras de linguagem etc., com vistas a fomentar práticas de
consumo conscientes.

a) Incorreta. No contexto em que se insere, a forma verbal no gerúndio tem valor condicional: 
``Água: se soubermos usar, não vai faltar''. 
b) Incorreta. No contexto em que se insere, a forma verbal no gerúndio tem valor condicional: 
``Água: se soubermos usar, não vai faltar''. 
c) Incorreta. No contexto em que se insere, a forma verbal no gerúndio tem valor condicional: 
``Água: se soubermos usar, não vai faltar''.
d) Correta. No contexto em que se insere, a forma verbal no gerúndio tem valor condicional: 
``Água: se soubermos usar, não vai faltar''.

\end{enumerate}

\section*{Módulo 9 – Treino}

\begin{enumerate}

\item
SAEB:Avaliar a eficácia das estratégias argumentativas em textos de
diferentes gêneros.

a) Incorreta. Na frase em destaque ocorre metonímia.

b) Incorreta. Na frase em destaque ocorre metonímia.

c) Incorreta. Na frase em destaque ocorre metonímia.

d) Correta. Na frase em destaque ocorre metonímia, isto é, o uso de nome do autor 
no lugar de sua obra. 

\item
SAEB: Avaliar a eficácia das estratégias argumentativas em textos de
diferentes gêneros.

a) Incorreta. A palavra que mais se repete no texto é a conjunção ``como'', que contribui
para as comparações entre a personagem e a natureza.
b) Incorreta. As hipérboles do texto contribuem
para as comparações entre a personagem e a natureza.
c) Correta. As descrições de Iracema são todas compostas por meio de comparações:
seus cabelos são ``mais negros que a asa da graúna e mais longos
que seu talhe de palmeira''. Seu sorriso é mais doce que o favo da jati,
e seu hálito perfumado cheira mais do que a baunilha. 
d) Incorreta. No texto, a natureza brasileira não é depreciada.

\item
SAEB: Analisar o uso de figuras de linguagem como estratégia argumentativa.

a) Incorreta. O narrador insiste nas ambições de João Romão, não em sua simplicidade.
b) Incorreta. As galinhas de João Romão e sua companheira davam muitos ovos, o que significa 
que ele e a companheira não viviam em miséria.
c) Incorreta. João Romão não se resigna: ele é tomado pela ``a febre de possuir'' para ``aumentar 
os bens''.
d) Correta. O conjunto do trecho insiste em que ``a febre de possuir'' é a personificação da 
ambiçao de João Romão, cujos atos ``visavam um interesse pecuniário'', que só queria ``aumentar 
os bens'' e ``reduzir tudo a moeda''.

\end{enumerate}

\section*{Módulo 10 – Treino}

\begin{enumerate}

\item
SAEB: Analisar os processos de referenciação lexical e pronominal.

BNCC: EF07LP12 -- Reconhecer recursos de coesão referencial:
substituições lexicais (de substantivos por sinônimos) ou pronominais
(uso de pronomes anafóricos -- pessoais, possessivos, demonstrativos.

a) Correta. O termo se refere à mãe do narrador, que está representada em um retrato com o pai dele.
b) Incorreta. O termo se refere à mãe do narrador, que está representada em um retrato com o pai dele.
c) Incorreta. O termo se refere à mãe do narrador, que está representada em um retrato com o pai dele.
d) Incorreta. O termo se refere à mãe do narrador, que está representada em um retrato com o pai dele.

\item
SAEB:Analisar os mecanismos que contribuem para a progressão textual.

BNCC: EF07LP12 -- Reconhecer recursos de coesão referencial:
substituições lexicais (de substantivos por sinônimos) ou pronominais
(uso de pronomes anafóricos -- pessoais, possessivos, demonstrativos).

a) Incorreta. A expressão se refere à irmã de Felipe.
b) Incorreta. A expressão se refere à irmã de Felipe.
c) Correta. A expressão se refere à irmã de Felipe. Ela é que perdeu os pais aos oito anos e encontrou na avó a ternura de uma mãe.
d) Incorreta. A expressão se refere à irmã de Felipe.

\item
SAEB: Analisar os processos de referenciação lexical e pronominal.

BNCC: EF07LP12 -- Reconhecer recursos de coesão referencial:
substituições lexicais (de substantivos por sinônimos) ou pronominais
(uso de pronomes anafóricos -- pessoais, possessivos, demonstrativos.

a) Incorreta. Todas as ocorrências destacadas no texto se referem ao chapéu do pai do garoto. 
b) Incorreta. Todas as ocorrências destacadas no texto se referem ao chapéu do pai do garoto. 
c) Incorreta. Todas as ocorrências destacadas no texto se referem ao chapéu do pai do garoto. 
d) Correta. Todas as ocorrências destacadas no texto se referem ao chapéu do pai do garoto. 

\end{enumerate}

\section*{Módulo 11 – Treino}

\begin{enumerate}

\item
SAEB: Avaliar a adequação das variedades linguísticas em contextos de uso.

a) Incorreta. A transcrição da fala popular se observa em ``fio dela'', equivalente a ``filho dela''.
b) Incorreta. A transcrição da fala popular se observa em ``fio dela'', equivalente a ``filho dela''.
c) Incorreta. A transcrição da fala popular se observa em ``fio dela'', equivalente a ``filho dela''.
d) Correta. A transcrição da fala popular se observa em ``fio dela'', equivalente a ``filho dela''.

\item
Saeb:Avaliar a adequação das variedades linguísticas em contextos de
uso.

a) Incorreta. O texto da questão é jornalístico. 
b) Incorreta. O texto da questão é jornalístico. 
c) Correta. O texto da questão é jornalístico, por isso o autor optou pelo uso da norma-padrão. 
d) Incorreta. O texto da questão é jornalístico.

\item
SAEB: Analisar as variedades linguísticas em textos.

a) Incorreta. A transcrição da fala popular se observa em ``A modo qui tô conhecendo mecê''.
b) Incorreta. A transcrição da fala popular se observa em ``A modo qui tô conhecendo mecê''. 
c) Incorreta. A transcrição da fala popular se observa em ``A modo qui tô conhecendo mecê''. 
d) Incorreta. A transcrição da fala popular se observa em ``A modo qui tô conhecendo mecê''.

\end{enumerate}

\section*{Simulado 1}

\begin{enumerate}

\item
SAEB: Identificar o uso de recursos persuasivos em textos verbais e
não verbais.

BNCC: EF67LP07 -- Identificar o uso de recursos persuasivos em textos
argumentativos diversos (como a elaboração do título, escolhas lexicais,
construções metafóricas, a explicitação ou a ocultação de fontes de
informação) e perceber seus efeitos de sentido.

a) Correta. A parte de palavra ``resta'' dialoga com a frase ``se você fizer diferente, a floresta faz o
restante'': \textit{resta} e \textit{restante} dizem respeito ao papel da \textit{floresta}.     
b)Incorreta. A parte destacada da palavra ``floresta'' não remete aos malefícios das queimadas e desmatamento, 
mas ao contrário: \textit{resta} e \textit{restante} dizem respeito ao papel da \textit{floresta}. 
c)Incorreta. O destaque da palavra ``floresta'' dialoga com ``diferença'' e ``diferente'': se as pessoas fizerem
diferente, a floresta faz o \textit{restante}, palavra que remete à parte da palavra ``floresta'', destacada 
à esquerda. 
d)Incorreta. O uso da palavra ``floresta'' certamente dialoga com as imagens à direita e com o texto central do cartaz da campanha.

\item
SAEB: Identificar elementos constitutivos de textos pertencentes ao
domínio jornalístico/midiático.

a) Incorreta. O trecho corresponde ao título da notícia.
b) Incorreta. O trecho pertence ao corpo da notícia.
c) Incorreta. O trecho pertence ao corpo da notícia.
d) Correta. A linha fina aparece como uma introdução ao assunto da notícia, logo depois do título.

\item 
SAEB: Identificar elementos constitutivos de gêneros de divulgação científica

a) Incorreta. O texto não apresenta linguagem técnica. 
b) Incorreta. O texto não é destinado apenas a cientistas.
c) Incorreta. O trecho não aborda fatos cotidianos. 
d) Correta. O texto faz uso de linguagem clara e acessível para divulgar estudos científicos
para o público em geral.

\item
SAEB: Identificar formas de organização de textos normativos, legais e/ou
reivindicatórios.

BNCC: EF67LP17 -- Analisar, a partir do contexto de produção, a forma de
organização das cartas de solicitação e de reclamação (datação, forma de
início, apresentação contextualizada do pedido ou da reclamação, em
geral, acompanhada de explicações, argumentos e/ou relatos do problema,
fórmula de finalização mais ou menos cordata, dependendo do tipo de
carta e subscrição) e algumas das marcas linguísticas relacionadas à
argumentação, explicação ou relato de fatos, como forma de possibilitar
a escrita fundamentada de cartas como essas ou de postagens em canais
próprios de reclamações e solicitações em situações que envolvam
questões relativas à escola, à comunidade ou a algum dos seus membros.


a) Incorreta. O texto não tem a pretensão de convencer o leitor.
b) Correta. O texto contém solicitação de transferência de instituição.
c) Incorreta. Não há referência sobre expectativa de mudança de atitude do leitor. 
d) Incorreta. O objetivo da carta não é informar, mas solicitar transferência de instituição.

\item
SAEB: Analisar elementos constitutivos de textos pertencentes ao domínio literário.

a) Correta. O texto do exercício é parte de uma peça de teatro, com discurso direto, indicação
de personagens e rubricas.
b) Incorreta. O texto do exercício é parte de uma peça de teatro, com discurso direto, indicação
de personagens e rubricas.
c) Incorreta. O texto do exercício é parte de uma peça de teatro, com discurso direto, indicação
de personagens e rubricas.
d) Incorreta. O texto do exercício é parte de uma peça de teatro, com discurso direto, indicação
de personagens e rubricas.

\item
Saeb: Distinguir fatos de opiniões em textos.
 
a) Correta. Este trecho apresenta um fato.
b) Incorreta. Este trecho contém opinião do autor.
c) Incorreta. Este trecho contém opinião do autor.
d) Incorreta. Este trecho contém opinião do autor.

\item
SAEB: Inferir, em textos multissemióticos, efeitos de humor, ironia e/ou
crítica.

BNCC: EF69LP03 -- Inferir e justificar, em textos multissemióticos --
tirinhas, charges, memes, gifs etc. --, o efeito de humor, ironia e/ou
crítica pelo uso ambíguo de palavras, expressões ou imagens ambíguas, de
clichês, de recursos iconográficos, de pontuação etc.

a) Incorreta. No trecho ``\textbf{aí} nada se encontra de novo'', o termo destacado se refere a 
``o quarto de um estudante''.
b) Incorreta. No trecho ``onde escreve'': o sentido da frase não equivale a ``onde eles escrevem''.
Pela coesão e coerência do texto, nota-se que a frase equivale a ``onde ele (o estudante) escreve''.
c) Correta. Pela coesão e coerência do texto, verifica-se que, no trecho ``o quarto de Augusto'', 
Augusto é o estudante cujo quarto é citado no início do parágrafo.  
d) Incorreta. No trecho ``Agora \textbf{ele} está só'': o termo destacado refere-se a Augusto.

\item
SAEB: Analisar a intertextualidade entre textos literários ou entre
estes e outros textos verbais ou não verbais.

Bncc: EF67LP27 -- Analisar, entre os textos literários e entre estes e
outras manifestações artísticas (como cinema, teatro, música, artes
visuais e midiáticas), referências explícitas ou implícitas a outros
textos, quanto aos temas, personagens e recursos literários e semióticos

a) Incorreta. No trecho ``\textbf{Você} sabe; já \textbf{lhe} disse'', os pronomes se referem a Camilo.
b) Incorreta. No trecho ``Camilo pegou-\textbf{lhe} nas mãos'', o pronome se refere às mãos de Rita.
c) Incorreta. No trecho ``Jurou que \textbf{lhe} queria muito'', o pronome se refere a Rita.
d) Correta. No trecho ``disse-\textbf{lhe} que era imprudente'', o pronome se refere a Rita.

\item
SAEB: Inferir a presença de valores sociais, culturais e humanos em
textos literários.

BNCC:EF69LP44 -- Inferir a presença de valores sociais, culturais e
humanos e de diferentes visões de mundo, em textos literários,
reconhecendo nesses textos formas de estabelecer múltiplos olhares sobre
as identidades, sociedades e culturas e considerando a autoria e o
contexto social e histórico de sua produção.
 
a) Incorreta. O trecho não contém narrativas tradicionais.
b) Correta. No trecho, o autor explica que, para os indígenas, a montanha é uma pessoa, ponto de vista bastante diferente dos não indígenas, que a veem como recurso natural a explorar. 
c) Incorreta. O autor não apresentou a língua falada por seu povo no trecho. 
d) Incorreta. O trecho não contém uma conciliação de cosmovisões, mas a
oposição entre elas: indígenas e não indígenas veem a montanha de maneiras 
bastante distintas.

\item
SAEB: Analisar os mecanismos que contribuem para a progressão textual.
a) Incorreta. O termo se refere ao agravamento dos problemas de saúde de 
Glória Pordeus. 
b) Correta. O termo se refere ao agravamento dos problemas de saúde de 
Glória Pordeus. 
c) Incorreta. O termo se refere ao agravamento dos problemas de saúde de 
Glória Pordeus. 
d) Incorreta. O termo se refere ao agravamento dos problemas de saúde de 
Glória Pordeus.

\item
SAEB:Analisar o uso de figuras de linguagem como estratégia
argumentativa.

a) Incorreta. No trecho destacado, não houve uso de figura de linguagem.
b) Incorreta. No trecho destacado, não houve uso de figura de linguagem.
c) Correta. A comparação entre o vício em smartphones e uma âncora é uma metáfora. 
d) Incorreta. No trecho destacado, não houve uso de figura de linguagem.

\item
SAEB: Analisar os processos de referenciação lexical e pronominal.

a) Correta. O texto é iniciado com a afirmação sobre ``possibilidade
de ocorrer um grande terremoto nos próximos anos''; é a esse possível terremoto que se refere a expressão ``um novo evento''. 
b) Incorreta. O terremoto de 1755 ocorreu no passado; evidentemente a expressão ``um novo evento'' não pode se referir a ele.
c) Incorreta. A expressão ``risco de atividade sísmica'' não se refere ao possível terremoto que pode ocorrer em Lisboa nos próximos anos.
d) Incorreta. A expressão ``potenciais danos'' não se refere ao possível terremoto que pode ocorrer em Lisboa nos próximos anos.

\item
SAEB: Avaliar a eficácia das estratégias argumentativas em textos de
diferentes gêneros.

opiniões de especialistas, alunos e professores.

    \item exemplos de sucesso em escolas estrangeiras. 

    \item dados estatísticos de fontes confiáveis.

    \item afirmações genéricas que não podem ser refutadas. 

a) Incorreta. O autor do texto não usou opiniões de especialistas, alunos e professores.
b) Incorreta. O autor do texto não usou exemplos de sucesso em escolas brasileiras.
c) Correta. O trecho contém dados estatísticos de fontes confiáveis: a pesquisa do Instituto Pró-Livro e os dados do PISA.
d) Incorreta. O autor do texto não usou afirmações genéricas que não podem ser refutadas.

\item
SAEB: Analisar os efeitos de sentido dos tempos, modos e/ou
vozes verbais com base no gênero textual e na intenção comunicativa.
 
a) Correta. O modo imperativo tem, de forma geral, a função de expressar
ordem, sugestão ou instrução. No caso específico do cartaz, sugere mudanças
de atitude por parte do leitor.
b) Incorreta. As formas verbais do cartaz não estão no infinitivo.
c) Incorreta. As formas verbais do cartaz não estão flexionadas no modo subjuntivo.
d) Incorreta. As formas verbais do cartaz não estão no infinitivo.

\item
Avaliar a adequação das variedades linguísticas em contextos de uso.

a) Incorreta. A campanha é voltada ao público jovem: na imagem, é retratada 
uma adolescente, com a qual esse público se identifica; o texto de 
chamada contém uma hashtag, o que sugere uso das redes sociais, que é mais
recorrente entre os jovens.  
b) Incorreta. A campanha é voltada ao público jovem: na imagem, é retratada 
uma adolescente, com a qual esse público se identifica; o texto de 
chamada contém uma hashtag, o que sugere uso das redes sociais, que é mais
recorrente entre os jovens.  
c) Correta. A campanha é voltada ao público jovem: na imagem, é retratada 
uma adolescente, com a qual esse público se identifica; o texto de 
chamada contém uma hashtag, o que sugere uso das redes sociais, que é mais
recorrente entre os jovens.  
d) Incorreta. A campanha é voltada ao público jovem: na imagem, é retratada 
uma adolescente, com a qual esse público se identifica; o texto de 
chamada contém uma hashtag, o que sugere uso das redes sociais, que é mais
recorrente entre os jovens.

\end{enumerate}

\section*{Simulado 2}

\begin{enumerate}
	
	\item
SAEB: Identificar teses, opiniões, posicionamentos explícitos e
argumentos em textos.  

a) Incorreta. Não há alusão à intensidade dos ventos como fator de risco.
b) Correta. Segundo a especialista, o período de estiagem e a baixa
umidade do ar são fatores de alerta para risco de queimadas.
c) Incorreta. Segundo a especialista, a seca traz duas preocupações para 
os produtores rurais: a necessidade de garantir alimento aos animais e a 
atenção ao risco de queimadas. A alimentação dos animais não é, portanto,
fator de risco de queimadas: é \textit{outra preocupação} dos produtores. 
d)Incorreta. No texto, não há alusão ao descaso com questões ambientais.

	\item
SAEB: Identificar formas de organização de textos normativos, legais e/ou reivindicatórios.

BNCC: EF69LP27 -- Analisar a forma composicional de textos pertencentes a
gêneros normativos/ jurídicos e a gêneros da esfera política, tais como
propostas, programas políticos (posicionamento quanto a diferentes ações
a serem propostas, objetivos, ações previstas etc.), propaganda política
(propostas e sua sustentação, posicionamento quanto a temas em
discussão) e textos reivindicatórios: cartas de reclamação, petição
(proposta, suas justificativas e ações a serem adotadas) e suas marcas
linguísticas, de forma a incrementar a compreensão de textos
pertencentes a esses gêneros e a possibilitar a produção de textos mais
adequados e/ou fundamentados quando isso for requerido

a) Correta. A divisão em artigos e parágrafos é uma característica
composicional de textos do gênero normativo/jurídico.
b) Incorreta. A divisão e forma composicional do trecho não correspondem
a cartas de solicitação.
c) Incorreta. A divisão e forma composicional do trecho não correspondem
a uma reivindicação.
d) Incorreta. A divisão e forma composicional do trecho não correspondem
a uma petição.	

	\item
SAEB: Analisar elementos constitutivos de textos pertencentes ao domínio
literário.

BNCC: EF69LP47 Analisar, em textos narrativos ficcionais, as diferentes
formas de composição próprias de cada gênero, os recursos coesivos que
constroem a passagem do tempo e articulam suas partes, a escolha lexical
típica de cada gênero para a caracterização dos cenários e dos
personagens e os efeitos de sentido decorrentes dos tempos verbais, dos
tipos de discurso, dos verbos de enunciação e das variedades
linguísticas (no discurso direto, se houver) empregados, identificando o
enredo e o foco narrativo e percebendo como se estrutura a narrativa nos
diferentes gêneros e os efeitos de sentido decorrentes do foco narrativo
típico de cada gênero, da caracterização dos espaços físico e
psicológico e dos tempos cronológico e psicológico, das diferentes vozes
no texto (do narrador, de personagens em discurso direto e indireto), do
uso de pontuação expressiva, palavras e expressões conotativas e
processos figurativos e do uso de recursos linguístico-gramaticais
próprios a cada gênero narrativo.

a) Incorreta. Os romances, embora apresentem algumas das
características elencadas no enunciado, em geral, são textos longos, 
com vários personagens que desenvolvem suas ações em vários cenários 
e conflitos.
b) Correta. As características elencadas no enunciado descrevem os 
contos.
c) Incorreta. Cartas pessoais possuem outras formas composicionais, tais
como saudação, corpo do texto e despedida.
d) Incorreta. As crônicas em geral possuem outras características tais
como temas do cotidiano e efeitos de humor.

	\item
SAEB: Identificar elementos constitutivos de gêneros de divulgação
científica.

a) Incorreta. A linguagem objetiva e a apresentação de dados de pesquisa são 
traços do gênero de divulgação científica.
b) Correta.  Incorreta. A linguagem objetiva e a apresentação de dados de pesquisa são 
traços do gênero de divulgação científica.
c) Incorreta. A linguagem objetiva e a apresentação de dados de pesquisa são 
traços do gênero de divulgação científica.
d) Incorreta. A linguagem objetiva e a apresentação de dados de pesquisa são 
traços do gênero de divulgação científica.

	\item
SAEB: Analisar elementos constitutivos de textos pertencentes ao 
domínioliterário.

a) Incorreta. O texto analisado apresenta características e explicações de outra obra.
b) Correta. O texto analisado apresenta características e explicações de outra obra.
c) Incorreta. O texto analisado apresenta características e explicações de outra obra.
d) Incorreta. O texto analisado apresenta características e explicações de outra obra

	\item
SAEB: Inferir informações implícitas em distintos textos.

a) Correta. O texto contém referências a apenas duas doenças relacionadas 
ao consumo de açúcar: diabetes e obesidade.
b) Incorreta. O texto contém referências a apenas duas doenças relacionadas 
ao consumo de açúcar: diabetes e obesidade.
c) Incorreta. O texto contém referências a apenas duas doenças relacionadas 
ao consumo de açúcar: diabetes e obesidade.
d) Incorreta. O texto contém referências a apenas duas doenças relacionadas 
ao consumo de açúcar: diabetes e obesidade.

	\item
SAEB: Inferir, em textos multissemiótico, efeitos de humor, ironia e/ou
crítica.

Bncc: EF69LP05 -- Inferir e justificar, em textos multissemióticos -- tirinhas, charges,
memes, gifs etc. --, o efeito de humor, ironia e/ou crítica pelo uso
ambíguo de palavras, expressões ou imagens ambíguas, de clichês, de
recursos iconográficos, de pontuação etc.

a) Correta. A crítica ao desmatamento se dá por meio da oposição entre o
desolamento causado pela destruição da natureza e o tom entusiástico da
propaganda do empreendimento imobiliário. 
b) Incorreta. O meme não contém divulgação de informações. 
c) Incorreta. O meme não contém frases injuntivas, isto é, imperativas.
d) Incorreta. O meme não contém frases motivacionais.

	\item
SAEB: Analisar efeitos de sentido produzido pelo uso de formas de
apropriação textual (paráfrase, citação etc.)

a) Incorreta. Os trechos entre aspas indicam citações das palavras da teórica Kirin Narayan.
b) Incorreta. Os trechos entre aspas indicam citações das palavras da teórica Kirin Narayan.
c) Incorreta. Os trechos entre aspas indicam citações das palavras da teórica Kirin Narayan.
d) Correta. Os trechos entre aspas indicam citações das palavras da teórica Kirin Narayan.

	\item
SAEB: Analisar o uso de figuras de linguagem como estratégia
argumentativa.

a) Incorreta. No contexto em que se insere, a expressão significa que a vida 
do garoto mudou radicalmente. Não há elementos que indiquem que ele levava
uma vida miserável. 
b) Incorreta. No contexto em que se insere, a expressão significa que a vida 
do garoto mudou radicalmente.
c) Incorreta. No contexto em que se insere, a expressão significa que a vida 
do garoto mudou radicalmente.
d) Correta. No contexto em que se insere, a expressão significa que a vida 
do garoto mudou radicalmente.

	\item
SAEB: Analisar os efeitos de sentido decorrentes dos mecanismos de construção
de textos jornalísticos/midiáticos.

a) Incorreta. Na manchete e no corpo do texto as aspas foram usadas para citar o nome da peça.
b) Correta. Na manchete e no corpo do texto as aspas foram usadas para citar o nome da peça.
c) Incorreta. Na manchete e no corpo do texto as aspas foram usadas para citar o nome da peça.
d) Incorreta. Na manchete e no corpo do texto as aspas foram usadas para citar o nome da peça.

	\item
SAEB: Avaliar a fidedignidade de informações sobre um mesmo fato divulgado em
diferentes veículos e mídias.

a) Incorreta. A informação de que o acidente deixou 33 operários presos é a única comum aos dois textos.
b) Correta. A informação de que o acidente deixou 33 operários presos é a única comum aos dois textos.
c) Incorreta. A informação de que o acidente deixou 33 operários presos é a única comum aos dois textos.
d) Incorreta. A informação de que o acidente deixou 33 operários presos é a única comum aos dois textos.

	\item
SAEB: Identificar os recursos de modalização em textos diversos.

a) Incorreta. De acordo com o texto, o Nordeste crescerá mais do que o Brasil se combinar aumento
da produtividade com redução da desigualdade. 
b) Correta. De acordo com o texto, o Nordeste crescerá mais do que o Brasil se combinar aumento
da produtividade com redução da desigualdade. 
c) Incorreta. De acordo com o texto, o Nordeste crescerá mais do que o Brasil se combinar aumento
da produtividade com redução da desigualdade. 
d) Incorreta. De acordo com o texto, o Nordeste crescerá mais do que o Brasil se combinar aumento
da produtividade com redução da desigualdade.

	\item
SAEB: Analisar os processos de referenciação lexical e pronominal.

a) Incorreta.   Em ``\textbf{Sua história} tem pouca coisa de notável'', o termo destacado 
se refere à história de Leonardo.
a) Correta.   Em ``assentou-\textbf{lhe} uma valente pisadela no pé direito'', o termo destacado 
se refere ao pé de Maria da Hortaliça.
a) Incorreta.   Em ``deu-\textbf{lhe} também em ar de disfarce um tremendo beliscão'', o termo  
destacado se refere às costas da mão de Leonardo.
a) Incorreta.   A expressão ``um pouco mais fortes'' refere-se às ações de Leonardo e  
Maria da Hortaliça.	

	\item
SAEB: Avaliar a eficácia das estratégias argumentativas em textos de
diferentes gêneros.

a) Incorreta. O autor não apela diretamente para as emoções do leitor,
mas para argumentos lógicos e fatos concretos.
b) Incorreta. O autor apresenta exemplos concretos de doenças
erradicadas e reduzidas graças à vacinação, sem generalizações.
c) Correta. O autor usa uma abordagem lógica para argumentar a favor da
vacinação, apresentando exemplos de doenças erradicadas e reduzidas
graças à vacinação, argumentando que a vacinação é estratégia mais
eficaz para prevenir doenças e salvar vidas.
d) Incorreta. O autor não se apoia em afirmações de especialistas para
argumentar a favor da vacinação.

	\item
SAEB: Avaliar a adequação das variedades linguísticas em contextos de uso

a) Incorreta. O uso de recursos utilizados em redes sociais, como hashtags, 
não é o mais indicado ao público idoso.
b) Incorreta. O uso de gírias, símbolos e siglas de internet não é eficaz 
para o público infantil.
c) Incorreta. Não há nada na imagem que remeta ao público feminino.
d) Correta. O uso de gírias e hashtags utilizados no meio
digital é recurso eficaz para atingir o público jovem.
\end{enumerate}

\section*{Simulado 3}

\begin{enumerate}

	\item
SAEB: Identificar teses, opiniões, posicionamentos explícitos e
argumentos em textos.

a) Incorreta. A reportagem não cita os ovos.
b)Correta. Estes estão entre os alimentos mais saudáveis e de menor
impacto junto com azeite e frutas secas.
c) Incorreta. Segundo o texto, o consumo de frango tem impacto ambiental
maior.
d) Incorreta. As carnes, segundo a reportagem, deveriam ser retiradas
da dieta.

	\item
SAEB: Identificar elementos constitutivos de gêneros de divulgação científica.

a) Incorreta. Essa afirmação não corresponde às informações do texto. 
b) Incorreta. Essa afirmação não corresponde às informações do texto. 
c) Incorreta. Segundo as afirmações do segundo parágrafo, ``A falta de acesso a esses serviços básicos expõe, principalmente, a população
mais vulnerável a diversas doenças como cólera e diarreia, além de impactar o
desempenho escolar, profissional e até mesmo os momentos de lazer''.  
d) Incorreta. Essa afirmação não corresponde às informações do texto. 

	\item
SAEB:  Identificar formas de organização de textos normativos, legais
e/ou reivindicatórios.

BNCC: EF69LP20 -- Identificar, tendo em vista o contexto de produção, a
forma de organização dos textos normativos e legais, a lógica de
hierarquização de seus itens e subitens e suas partes: parte inicial
(título -- nome e data -- e ementa), blocos de artigos (parte, livro,
capítulo, seção, subseção), artigos (caput e parágrafos e incisos) e
parte final (disposições pertinentes à sua implementação) e analisar
efeitos de sentido causados pelo uso de vocabulário técnico, pelo uso do
imperativo, de palavras e expressões que indicam circunstâncias, como
advérbios e locuções adverbiais, de palavras que indicam generalidade,
como alguns pronomes indefinidos, de forma a poder compreender o caráter
imperativo, coercitivo e generalista das leis e de outras formas de
regulamentação.

a) Correta. O texto se caracteriza pela linguagem impessoal e organização em 
títulos, capítulos e seções.
b) Incorreta. O texto se caracteriza pela linguagem impessoal e organização em 
títulos, capítulos e seções. 
c) Incorreta. O texto se caracteriza pela linguagem impessoal e organização em 
títulos, capítulos e seções. 
d) Incorreta. O texto se caracteriza pela linguagem impessoal e organização em 
títulos, capítulos e seções.

	\item
Identificar elementos constitutivos de textos pertencentes ao domínio
jornalístico/midiático.

a) Incorreta. A informação sobre São Paulo está localizada na manchete.
b) Incorreta. A informação sobre São Paulo está localizada na manchete.
c) Incorreta. A informação sobre São Paulo está localizada na manchete.
d) Correta. A informação sobre São Paulo está localizada na manchete.

	\item
SAEB: Inferir informações implícitas em distintos textos.

a) Incorreta. Para o eu lírico, sua tristeza é sossego, sem causa confusão.
b) Incorreta. Para o eu lírico, sua tristeza é natural e justa. 
c) Incorreta. Para o eu lírico, sua tristeza está na alma.
d) Correta. Para o eu lírico, sua tristeza é sossego (portanto, quietude),
por ser natural e justa.

	\item
Saeb: Analisar elementos constitutivos de textos pertencentes ao domínio literário.

a) Correta. Os dois últimos períodos do parágrafo pretendem justificar a baixa
adesão, de apenas onze amigos, ao enterro do narrador. Note-se: ao iniciar o 
penúltimo período com ``verdade é que'' ele pretende explicar por que tão 
poucas pessoas compareceram: porque não houve anúncio de sua morte e porque chovia.
b) Incorreta. Os dois últimos períodos do parágrafo pretendem justificar a baixa
adesão, de apenas onze amigos, ao enterro do narrador. Note-se: ao iniciar o 
penúltimo período com ``verdade é que'' ele pretende explicar por que tão 
poucas pessoas compareceram: porque não houve anúncio de sua morte e porque chovia.
c) Incorreta. Os dois últimos períodos do parágrafo pretendem justificar a baixa
adesão, de apenas onze amigos, ao enterro do narrador. Note-se: ao iniciar o 
penúltimo período com ``verdade é que'' ele pretende explicar por que tão 
poucas pessoas compareceram: porque não houve anúncio de sua morte e porque chovia.
d) Correta. Os dois últimos períodos do parágrafo pretendem justificar a baixa
adesão, de apenas onze amigos, ao enterro do narrador. Note-se: ao iniciar o 
penúltimo período com ``verdade é que'' ele pretende explicar por que tão 
poucas pessoas compareceram: porque não houve anúncio de sua morte e porque chovia.
	
	\item
SAEB: Avaliar a fidedignidade de informações sobre um mesmo fato
divulgado em diferentes veículos e mídias.

a) Incorreta. O texto não contém gírias. 
b) Correta. O texto é repleto de dados estatísticos de fonte governamental, que lhe conferem maior confiabilidade.
c) Incorreta. O texto não contém citação de estudos e pesquisadores internacionais.
d) Incorreta. O texto não contém trechos de falas de especialistas no assunto.
	
	\item
SAEB: Analisar os efeitos de sentido produzidos pelo uso de modalizadores em
textos diversos.

a) Correta. Segundo o texto, o momento em que mais desejamos 
o sono é aquele em que não sentimos sono. 
b) Incorreta. A afirmação desta alternativa não corresponde à do texto, segundo a qual o momento em que mais desejamos 
o sono é aquele em que não sentimos sono.
c) Incorreta. A afirmação desta alternativa não corresponde à do texto, segundo a qual o momento em que mais desejamos 
o sono é aquele em que não sentimos sono.
d) Incorreta. A afirmação desta alternativa não corresponde à do texto, segundo a qual o momento em que mais desejamos 
o sono é aquele em que não sentimos sono.

	\item
SAEB: Avaliar a eficácia das estratégias argumentativas em textos de
diferentes gêneros.

a) Incorreta. De acordo com o poema, só os portugueses conseguem sentir saudades. Essa palavra pode ser conhecida por falantes de outras nacionalidades, mas eles não poderão sentir saudades como os portugueses sentem.  
b) Correta. De acordo com o texto, somente os portugueses conseguem sentir bem as saudades. 
c) Incorreta. A palavra ``saudades'' pertence aos portugueses, os únicos que conseguem senti-las adequadamente. 
d) Incorreta. Os portugueses criaram a palavra ``saudades'' não por capricho, mas para dizer que sentem saudades -- sentimento somente eles conseguem experimentar adequadamente. 

	\item
SAEB: Analisar os processos de referenciação lexical e pronominal.

a) Incorreta. As amigas de Clara eram evidentemente mais invejosas do que amigas, 
como se verifica na expressão ``menos por amizade que por inveja''.
b) Incorreta. Os sentimentos do noivo não são questionados. As amigas de Clara, 
invejosas, dizem-lhe que ele, embora apaixonado e virtuoso, era festeiro demais.
c) Incorreta. Embora invejosas, as amigas de Clara tinham-lhe alguma amizade.
d) Correta. A chave para escolher esta alternativa é compreender que, por meio da  
expressão ``menos por amizade que por inveja'' o narrador evidencia que, por mais 
houvesse amizade entre Clara e as amigas, estas tentaram convencê-la a desistir 
do casamento por inveja.
	
	\item
SAEB: Analisar os mecanismos que contribuem para a progressão textual.

a) Correta. Na primeira afirmação, a causa da decisão de não sair é o vento soprar forte;
na segunda, estar doente é a causa da impossibilidade de ir ao escritório; na terceira, 
as necessidades dos jovens só não serão ouvidas em uma condição: se eles se afastarem dos
responsáveis; na última, apesar da tristeza, não houve lágrimas. 
b) Incorreta. Na primeira afirmação, a causa da decisão de não sair é o vento soprar forte;
na segunda, estar doente é a causa da impossibilidade de ir ao escritório; na terceira, 
as necessidades dos jovens só não serão ouvidas em uma condição: se eles se afastarem dos
responsáveis; na última, apesar da tristeza, não houve lágrimas.  
c) Incorreta. Na primeira afirmação, a causa da decisão de não sair é o vento soprar forte;
na segunda, estar doente é a causa da impossibilidade de ir ao escritório; na terceira, 
as necessidades dos jovens só não serão ouvidas em uma condição: se eles se afastarem dos
responsáveis; na última, apesar da tristeza, não houve lágrimas. 
d) Incorreta. Na primeira afirmação, a causa da decisão de não sair é o vento soprar forte;
na segunda, estar doente é a causa da impossibilidade de ir ao escritório; na terceira, 
as necessidades dos jovens só não serão ouvidas em uma condição: se eles se afastarem dos
responsáveis; na última, apesar da tristeza, não houve lágrimas.
	
	\item
SAEB: Analisar o uso de figuras de linguagem como estratégia argumentativa.

a) Incorreta. Incorreta. Não ocorre hipérbole no verso ``Lá fora a chuva canta e chora''.
b) Incorreta. Não ocorre ironia no verso ``Lá fora a chuva canta e chora''.
c) Incorreta. Não ocorre ambiguidade no verso ``Lá fora a chuva canta e chora''.
d) Correta. No verso ``Lá fora a chuva canta e chora'', atribuem-se as ações de cantar e chorar a um ser inanimado, a chuva.

	\item
SAEB: Analisar os efeitos de sentido dos tempos, modos e/ou vozes verbais com
base no gênero textual e na intenção comunicativa.

a) Incorreta. O narrador não deixa claro que considerava os versos ruins, como se
pode verificar pela escolha das formas verbais destacadas no texto.
b) Incorreta. O narrador não deixa claro que considera bons os versos do rapaz,como se
pode verificar pela escolha das formas verbais destacadas no texto.
c) Incorreta. A escolha das formas verbais destacadas no texto não permite fazer uma
afirmação tão categórica quanto a desta alternativa.
d) Correta. A escolha das formas verbais destacadas no texto sugere que os versos do
rapaz, embora não pudessem ser chamados de completamente ruins, também não poderiam
ser chamados de bons.
	
	\item
SAEB: Analisar as variedades linguísticas em textos.

a) Incorreta. Esta alternativa não contém expressão informal.
b) Incorreta. Esta alternativa não contém expressão informal.
c) Incorreta. Esta alternativa não contém expressão informal.
d) Correta. O uso do verbo ``curtir'' com sentido de ``usufruir'' pode ser considerado informal. 
	
	\item
SAEB: Avaliar a adequação das variedades linguísticas em contextos de uso.

a) Incorreta. O uso do termo ``causos'' alude ao vocabulário da cultura popular, 
não ao uso da linguagem formal.
b) Incorreta. O uso do termo ``causos'' alude ao vocabulário da cultura popular, 
não à objetividade breve dos cronistas.
c) Incorreta. O uso do termo ``causos'' alude ao vocabulário da cultura popular, 
não às gírias da juventude.
d) Correta. O uso do termo ``causos'' alude ao vocabulário da cultura popular.
\end{enumerate}

\section*{Simulado 4}

\begin{enumerate}

	\item
SAEB: Identificar teses, opiniões, posicionamentos explícitos e argumentos em
textos.

a) Incorreta. O texto não contém referências à atitude dos consumidores.
b) Incorreta. A interrupção do programa governamental não é fator que 
tenha interferido no aumento.
c) Correta. Segundo o texto, estes são os principais fatores que levaram
ao aumento dos preços da energia na Europa. 
d) Incorreta. O texto se refere ao poder de compra dos franceses como uma
preocupação, mas não trata a questão como fator para o aumento da energia.

	\item
SAEB: Identificar o uso de recursos persuasivos em textos verbais e não
verbais

a) Incorreta. A imagem que ilustra o cartaz não é inesperada.
b) Incorreta. O contraste entre as cores de fundo não alcança
grande força expressiva.
c) Incorreta. A lista de locais de vacinação não está em destaque.
d) Correta. O jogo com palavras de sons semelhantes, mas de sentidos
distintos, confere força expressiva ao cartaz.

	\item
SAEB: Identificar formas de organização de textos normativos, 
legais e/ou reivindicatórios.

a) Correta. Logo no primeiro artigo aparecem as disposições gerais e os
temas a serem normalizados pela lei.
b) Incorreta. O artigo quinto da Constituição aparece como reforço e
justificativa da importância da Lei.
c) Incorreta. O capítulo primeiro compreende demais partes e
peculiaridades da Lei.
d) Incorreta. O parágrafo único apresenta as justificativas sobre
pertinência da regulamentação do Estatuto da Pessoa com Deficiência.

	\item
SAEB: Identificar elementos constitutivos de gêneros de divulgação científica

a) Incorreta. A expressão destacada expressa a circunstância de finalidade.
b) Incorreta. A expressão destacada expressa a circunstância de adição.
c) Incorreta. A expressão destacada expressa a circunstância de adição.
d) Correta. A expressão destacada expressa a circunstância de causa.

	\item
SAEB: Analisar a relação temática entre diferentes gêneros jornalísticos.

a) Incorreta. O Texto 1 é um artigo de opinião; o Texto 2, uma notícia.
b) Correta. O Texto 1 é artigo de opinião, pois apresenta
argumentação acerca do tema; o Texto 2 apresenta claramente estrutura
de notícia, 
restringindo-se a noticiar o resgate dos animais, sem expressão de 
opinião. 
c) Incorreta. O Texto 1 é um artigo de opinião; o Texto 2, uma notícia.
d) Incorreta. O Texto 1 é um artigo de opinião; o Texto 2, uma notícia.

	\item
SAEB: Analisar elementos constitutivos de textos pertencentes ao domínio
literário.

a) Incorreta. O texto é um poema.
b) Incorreta. O texto é um poema.
c) Correta. O texto é um soneto, forma poemática de quatro estrofes, 
dois quartetos e dois tercetos, cujos versos rimados e metrificados 
imprimem sonoridade regular ao conjunto. 
d) Incorreta. O texto é um poema.

	\item
SAEB: Analisar efeitos de sentido produzido pelo uso de formas
de apropriação textual (paráfrase, citação etc.).

a) Correta. O texto versa sobre \textit{problemas de mobilidade}, que são chamados ironicamente de \textit{corrida}.
b) Incorreta. O uso de aspas neste caso não indica a fala de
autoridades.
c) Incorreta. O uso das aspas neste caso não introduz fala
de especialista.
d) Incorreta. Embora, ao ser isolado por aspas, o termo ganhe ênfase, o motivo de usá-las é a ironia.

	\item
SAEB: Analisar os efeitos de sentido decorrentes dos mecanismos de 
construção de textos jornalísticos/midiáticos.

a) Incorreta. As expressões destacadas não revelam explicitamente a opinião do autor.
b) Incorreta. As expressões destacadas não revelam explicitamente a opinião do autor.
c) Correta. A opinião do autor se revela explicitamente nas duas expressões: ``armadilha insidiosa'' expressa o repúdio ao sedentarismo; ``investimento valioso'' manifesta o elogio da vida ativa.
d) Incorreta. As expressões destacadas não revelam explicitamente a opinião do autor.

	\item
SAEB: Inferir informações implícitas em distintos textos

a) Correta. O trecho final, em destaque, chama-se \textit{moral da história}
e serve para sintetizar, de maneira expressiva (observe-se a rima entre
\textit{mato} e \textit{gato}) o sentido da fábula. 
b) Incorreta. A moral da história resume de forma expressiva o sentido da 
fábula.
c) Incorreta. A moral da história resume de forma expressiva o sentido da 
fábula.
d) Incorreta. A moral da história resume de forma expressiva o sentido da 
fábula.

	\item
SAEB: Distinguir fatos de opiniões em textos.

a) Incorreta. O turismo e o comércio não se referem à importância 
\textit{biológica} dos recifes de coral: essas atividades dizem respeito a
seu valor socioeconômico.
b) Incorreta. A alternativa contém a descrição objetiva da estrutura dos
corais, não sua importância biológica.
c) Correta. Segundo o texto, os recifes de corais são importantes
biologicamente por abrigarem grande biodiversidade.
d) Incorreta. A produção de pescado não se refere à importância 
\textit{biológica} dos recifes de coral: essa atividade diz respeito
a seu valor socioeconômico.

	\item
SAEB: Analisar os processos de referenciação lexical e pronominal.

a) Incorreta. No último parágrafo, a autora afirma que, apesar das restrições
e atitudes de algumas escolas, há pais que não contribuem para oferecer uma alimentação saudável para os filhos.
b) Incorreta. No último parágrafo, a autora afirma que, apesar das restrições
e atitudes de algumas escolas, há pais que não contribuem para oferecer uma alimentação saudável para os filhos.
c) Incorreta. No último parágrafo, a autora afirma que, apesar das restrições
e atitudes de algumas escolas, há pais que não contribuem para oferecer uma alimentação saudável para os filhos.
d) Correta. No último parágrafo, a autora afirma que, apesar das restrições
e atitudes de algumas escolas, há pais que não contribuem para oferecer uma alimentação saudável para os filhos.
	
	\item
SAEB: Avaliar diferentes graus de parcialidade em textos jornalísticos.

a) Incorreta. O trecho não apresenta parcialidade.
b) Incorreta. O trecho não apresenta parcialidade.
c) Correta. A opinião do autor está expressa no uso do advérbio ``infelizmente''.
d) Incorreta. O trecho não apresenta parcialidade.

	\item
SAEB: Identificar os recursos de modalização em textos diversos.

a) Incorreta. Na frase ``o Brasil deve
investir em infraestrutura, desenvolvimento de talentos e
sustentabilidade ambiental nas praias e oceanos'', a locução verbal ``deve investir'' expressa a obrigação que o Brasil tem, na opinião do autor, para tornar-se a ``nação do surfe''. 
b) Incorreta. Na frase ``o Brasil deve
investir em infraestrutura, desenvolvimento de talentos e
sustentabilidade ambiental nas praias e oceanos'', a locução verbal ``deve investir'' expressa a obrigação que o Brasil tem, na opinião do autor, para tornar-se a ``nação do surfe''.
c) Correta. Na frase ``o Brasil deve
investir em infraestrutura, desenvolvimento de talentos e
sustentabilidade ambiental nas praias e oceanos'', a locução verbal ``deve investir'' expressa a obrigação que o Brasil tem, na opinião do autor, para tornar-se a ``nação do surfe''.
d) Incorreta. Na frase ``o Brasil deve
investir em infraestrutura, desenvolvimento de talentos e
sustentabilidade ambiental nas praias e oceanos'', a locução verbal ``deve investir'' expressa a obrigação que o Brasil tem, na opinião do autor, para tornar-se a ``nação do surfe''.

	\item
SAEB: Avaliar a eficácia das estratégias argumentativas em textos de
diferentes gêneros.

a) Incorreta. O texto não contém falas de especialistas nem opinião de pais.
b) Incorreta. Não há citação de discursos de autoridade no texto.
c) Incorreta. Não há exemplos históricos no texto.
d) Correta. O texto se apoia em dados de pesquisas.

	\item
SAEB: Avaliar a adequação das variedades linguísticas em contextos de uso.

a) Incorreta. A expressão ``engenheiro inglês'' é usada pelo narrador para
referir-se ao estrangeiro, não pela população local.
b) Correta. A expressão \textit{``seu'' mister} é usada pela população local
para referir-se ao estrangeiro. O uso de ``seu'' já manifesta o respeito 
a ele, que era considerado um sábio. 
c) Incorreta. A expressão ``o sábio inglês'' é usada pelo narrador para
referir-se ao estrangeiro, não pela população local.
d) Incorreta. A expressão ``um inglês que levava a catar pedras'', meramente
descritiva, é usada pelo narrador para referir-se ao estrangeiro e não expressa
o respeito da população a ele.

\end{enumerate}
