
\section{SIMULADO 4}\label{simulado-4}

5. Leia o texto.

Joana era nutricionista de uma empresa, mas perdeu o emprego durante a
pandemia e já estava há mais de dois anos sem trabalhar. Fazia um
servicinho aqui, outro ali, para conseguir levar algum dinheiro para
casa e ajudar nas despesas.

Como ela sempre foi boa funcionária, um dia, ela recebeu o telefonema de
um antigo chefe perguntando se ela teria interesse em atuar como
nutricionista de uma escola. Ela aceitou a oferta e agarrou a
oportunidade com unhas e dentes, objetivando guardar dinheiro e
contribuir com a família para comprar uma casa.

\fonte{Texto escrito para este material.}

A expressão ``agarrar com unhas e dentes'' significa

\begin{escolha}
\item esforçar-se ao máximo para não perder o novo emprego.

\item descansar para recomeçar a procura mais tarde.

\item partir para a briga para conseguir o que se deseja.

\item seguir em frente, apesar dos momentos difíceis.
\end{escolha}

\includegraphics[width=2.43333in,height=3.24514in]{media/image7.png}6.
Analise cuidadosamente o infográfico a seguir.

\textit{https://www.gov.br/pt-br/noticias/transito-e-transportes/2021/04/programa-ja-instalou-mais-de-13-2-mil-pontos-de-internet-no-pais}

Esse texto foi elaborado pelo Ministério das Comunicações utilizando
vários elementos com a intenção de transmitir a ideia central com
clareza: onde estão os pontos instalados do programa Wi-Fi Brasil. Por
meio desse material, entendemos que o estado e o local mais beneficiados
foram

\begin{escolha}
\item o Maranhão e as unidades de saúde.

\item o Ceará e os telecentros.

\item a Bahia e as escolas.

\item o Amazonas e as comunidades indígenas.
\end{escolha}

7. Leia o texto.

\textbf{``Lá vem história'': por que é saudável ler para crianças?}

\emph{Além de ser um exercício que desperta o encantamento, estudos
apontam que ler em voz alta para crianças é fundamental para o
desenvolvimento da linguagem}

{[}...{]}

O artigo ``Aprender a partir da leitura em voz alta dos adultos'',
publicado pelas pesquisadoras Ana Teberosky e Angelica Sepúlveda, da
Universidade de Barcelona, na Espanha, confirma a riqueza de
aprendizados gerados a partir da leitura de histórias {[}...{]}.

{[}...{]}

As autoras defendem que essa escuta leva {[}à{]} construção de um
vocabulário mais rico e ao desenvolvimento de uma linguagem mais
complexa e elaborada pelos pequenos.

Lunetas. ``Lá vem história'': por que é saudável ler para crianças?
\fonte{Disponível em: \textit{https://lunetas.com.br/ler-para-criancas/} . Acesso
em: 24 abr. 2023.

O artigo publicado pelas pesquisadoras da Universidade de Barcelona

\begin{escolha}
\item combate a ideia de que é bom ler em voz alta para crianças.

\item nega que as crianças desenvolvam uma linguagem mais complexa.

\item deixa dúvidas se faz bem ler em voz alta para crianças.

\item reforça a importância de ler em voz alta para crianças.
\end{escolha}


\num{8} Observe os dois textos reproduzidos a seguir.

\textbf{Recomece}

{[}...{]}

Quando tudo for escuro

e nada iluminar,

quando tudo for incerto

e você só duvidar...

É hora do recomeço.

Recomece a ACREDITAR.

{[}...{]}

Bráulio Bessa. Recomece. \fonte{Disponível em
\textit{https://www.brauliobessa.com/post/recomece}. Acesso em 24 abr. 2023

\textbf{O guardador de rebanhos}

Eu nunca guardei rebanhos,

Mas é como se os guardasse.

Minha alma é como um pastor,

Conhece o vento e o sol

E anda pela mão das Estações

A seguir e a olhar.

Toda a paz da Natureza sem gente

Vem sentar-se a meu lado.

Mas eu fico triste como um pôr de sol

Para a nossa imaginação,

Quando esfria no fundo da planície

E se sente a noite entrada

Como uma borboleta pela janela.

{[}...{]}

Alberto Caeiro (heterônimo de Fernando Pessoa). O guardador de rebanhos.
\fonte{Disponível em:
\textit{http://www.dominiopublico.gov.br/download/texto/pe000001.pdf}.
Acesso em: 25 abr. 2023.

Os dois textos são trechos de poemas, respectivamente, com 6 e 13
versos. Outra diferença é que o

\begin{escolha}
\item primeiro tem rimas e o segundo não.

\item primeiro tem estrofe e o segundo não.

\item segundo tem rimas e o primeiro não.

\item segundo tem estrofe e o primeiro não.
\end{escolha}


\num{9} Leia o parágrafo.

Os microplásticos podem ter diferentes origens. Uma delas é a degradação
de plásticos maiores que, com o tempo, fragmentam-se em pedaços cada vez
menores por conta da ação de raios ultravioleta, ondas e o atrito com
areias e pedras.

Julia di Spagna. Guia do Estudante. Microplásticos: entenda os impactos
no meio ambiente e na saúde humana. \fonte{Disponível em:
\textit{https://guiadoestudante.abril.com.br/atualidades/o-que-sao-microplasticos-e-o-impacto-no-meio-ambiente-e-na-saude-humana/}.
Acesso em: 25 abr. 2023.

Uma palavra do texto formada por um prefixo que indica diminutivo é

\begin{escolha}
\item ``ultravioleta''.

\item ``degradação''.

\item ``microplásticos''.

\item ``maiores''.
\end{escolha}


\num{10} Leia o texto.

25/04/2023

Hoje assisti a um filme muito legal que se passava na Grécia, um país
muito, mas muito antigo mesmo, que fica na Europa. Lá tem praias lindas,
muito azuis, e diversos museus --- eu adoro museus! E tem
construções muito antigas... pena que algumas delas estão meio
destruídas.

O filme mostra que os gregos são muito alegres, estão sempre cantando e
dançando. E têm um costume diferente: em festas de casamentos, eles
jogam pratos no chão para quebrá-los... dizem que é para dar sorte aos
noivos.

E as comidas também parecem muito gostosas, como o mussaká, que é tipo
uma lasanha de berinjela.

Ah, e como a Grécia é formada por várias ilhas, dá para fazer muitos
passeios de barco, o que deve ser bem divertido.

Quando eu crescer e trabalhar, vou economizar para visitar a Grécia!

\fonte{Texto escrito para este material.}

Pode-se associar esse texto ao gênero

\begin{escolha}
\item notícia.

\item reportagem.

\item diário.

\item carta.
\end{escolha}

\num{11} Leia o texto.

\section{Caso Patolino: Polícia analisa imagens de câmeras e ouve
pessoas que trabalhavam em condomínio onde pato
desapareceu}\label{caso-patolino-poluxedcia-analisa-imagens-de-cuxe2meras-e-ouve-pessoas-que-trabalhavam-em-condomuxednio-onde-pato-desapareceu}

{[}...{]}

A Secretaria de Segurança Pública (SSP) informou que investigadores da
Delegacia de Investigações sobre Infrações Contra o Meio Ambiente
(Dicma) de
Mogi
das Cruzes, na Grande São Paulo, estão analisando as imagens das
câmeras de segurança do condomínio onde
o
pato de estimação da influenciadora digital Julia Olympio desapareceu.
O caso ocorreu no início do mês.

Além disso, a SSP informou ainda que as pessoas que estavam realizando
algum tipo de serviço no local no dia do sumiço do pato estão prestando
depoimento à polícia.

{[}...{]}

Eduarda Hutter. G1. Caso Patolino: Polícia analisa imagens de câmeras e
ouve pessoas que trabalhavam em condomínio onde pato desapareceu.
\fonte{Disponível em:
\textit{https://g1.globo.com/sp/mogi-das-cruzes-suzano/noticia/2023/04/25/caso-patolino-policia-analisa-imagens-de-cameras-e-ouve-pessoas-que-estavam-no-condominio-no-dia-do-desaparecimento-do-pato.ghtml}.
Acesso em: 25 abr. 2023.

Um elemento presente nesse texto é

a) narração de um fato.

b) construção de um cenário.

c) interação entre personagens.

d) mudança de tempo da narrativa.



\num{12} Leia a receita a seguir.

\textbf{Brigadeirão de micro-ondas}

Ingredientes:

\begin{itemize}
\item
  1 lata de leite condensado
\item
  1 lata de creme de leite
\item
  1 xícara (chá) de chocolate em pó
\item
  1/4 xícara (chá) de açúcar
\item
  1 colher (sopa) de margarina
\item
  3 ovos
\end{itemize}

Modo de preparo:

\begin{enumerate}
\def\labelenumi{\arabic{enumi}.}
\item
  Unte uma forma com cone no meio que possa ir ao micro-ondas.
\item
  Bata todos os ingredientes no liquidificador até que tudo fique muito
  bem misturado.
\item
  Despeje a mistura na forma untada.
\item
  Leve ao micro-ondas por aproximadamente 7 minutos em potência alta. O
  tempo pode variar de acordo com o micro-ondas; então fique atento para
  não tirar o brigadeirão ainda mole.
\item
  Desenforme morno.
\item
  Decore com chocolate granulado.
\item
  Leve à geladeira até a hora de servir.
\end{enumerate}

\fonte{Texto escrito para este material.}

Os textos instrucionais têm características muito definidas. Entre elas,
uma presente nesse texto

\begin{escolha}
\item é a utilização de metáforas.

\item são as descrições detalhadas.

\item é a lista de passos a seguir.

\item são as informações implícitas.
\end{escolha}
13. Preste atenção no diálogo a seguir, que ocorre ao telefone.

Sônia: Alô.

Cris: Sônia, é você?

Sônia: É sim... e aí: É a Cris?

Cris: Eu mesma (risos). Tudo bem com você?

Sônia: Tudo. E com você?

Cris: Por aqui também tudo bem. Queria saber se você me empresta um
livro que preciso ler para a escola.

Sônia: Claro que te empresto. Qual livro?

Cris: É um que fala que está tudo bem ser diferente... Vamos fazer uma
roda de conversa na minha turma sobre as diferenças entre as pessoas.

Sônia: Que legal! Sei que livro é. Claro que te empresto.

Cris: Obrigada! Posso pegar com você amanhã à tarde?

Sônia: Pode, sim. Minha mãe vai fazer bolo de cenoura, aí você come um
pedaço.

Cris: Eba! Adoro bolo de cenoura.

Sônia: Então até amanhã.

Cris: Até amanhã. Um beijo!

Sônia: Outro!

\fonte{Texto escrito para este material.}

Os sinais de pontuação são utilizados para ajudar o leitor a compreender
a intenção --- demonstrar surpresa, alegria; afirmar algo; fazer
pausas maiores ou menores; perguntar algo, por exemplo --- de cada
frase e, assim, entender o texto todo. Nesse diálogo, é possível saber
quando Cris ou Sônia estão fazendo perguntas por meio

\begin{escolha}
\item da vírgula.

\item do ponto de interrogação.

\item dos dois-pontos.

\item do ponto de exclamação.
\end{escolha}


\num{14} Leia os parágrafos escritos a seguir.

1) Em uma rua tranquila, com árvores floridas, existe um casarão antigo
que tem um belo jardim. Na semana passada, um jovem casal mudou-se para
lá com um cachorrinho lindo e branquinho.

2) Em uma rua com árvores, existe um casarão que tem um jardim. Na
semana passada, um casal mudou-se para lá com um cachorrinho.

Ao se comparar os dois parágrafos, é possível perceber que eles tratam
do mesmo casarão, do mesmo casal e do mesmo cachorrinho. No primeiro, o
autor dá informações exatas sobre cada elemento. Já no segundo, não há
tantos detalhes sobre os elementos. A diferença entre os dois textos é a
presença de

\begin{escolha}
\item adjetivos no primeiro texto.

\item conjunções no primeiro texto.

\item advérbios no segundo texto.

\item interjeições no segundo texto.
\end{escolha}


15. Leia o diálogo reproduzido a seguir.

--- Filho, está chegando seu aniversário --- disse o pai a Davi.

--- É, pai. Já vou fazer 9 anos! --- Exclamou, orgulhoso, o
garoto.

--- Então, filho, estava pensando em te dar aquele livro de
presente. Você gostaria? --- Perguntou o pai.

--- Eu adoraria, papai! --- Gritou Davi, emocionado, atirando-se
ao pescoço do pai.

\fonte{Texto escrito para este material.}

Nesse texto, os verbos de enunciação auxiliam a compreender:

\begin{escolha}
\item o cenário em que acontece o diálogo entre os personagens.

\item o momento da ação de falar dos personagens.

\item o assunto da conversa entre os personagens.

\item a emoção dos personagens na ação de conversar.
\end{escolha}
