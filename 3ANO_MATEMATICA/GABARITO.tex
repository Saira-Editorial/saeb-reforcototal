\chapter{Respostas}
\pagestyle{plain}
\footnotesize

\pagecolor{gray!40}

\colorsec{Matemática -- Módulo 1 -- Treino}

\begin{enumerate}
\item 
a) Correta. A decomposição 700 + 30 + 4 equivale a 734.
b) Incorreta. 70 + 3 + 4 = 707.
c) Incorreta. 700 + 40 + 3 = 743.
d) Incorreta. 70 + 300 + 4 = 374.

\item
a) Incorreta. Neste ponto estará o número 550.
b) Incorreta. Neste ponto estará o número 560.
c) Correta. O 570 estará no ponto S, pois, como no ponto P está o 540 e cada
repartição é de 10 unidades, ele deve estar no terceiro ponto após o P,
sem contar o ponto S.
d) Incorreta. Neste ponto estará o número 580.

\item
a) Incorreta. Repetem-se os algarismos 5 e 9, mas o número formado está errado.
b) Incorreta. Não se trata do número formado.
c) Correta. 5 x 100 + 1 x 9 = 509.
d) Incorreta. 1.509 tem 1000 unidades a mais do que 509.
\end{enumerate}

\colorsec{Matemática -- Módulo 2 -- Treino}

\begin{enumerate}
\item
a) Incorreta. Trata-se somente do número de botões.
b) Incorreta. Trata-se, simplesmente, do número de camisas.
c) Incorreta. Trata-se da soma de 6 com 9.
d) Correta. A conta é a seguinte: 6 x 9 = 54 botões.

\item
a) Incorreta. Trata-se do número de latas em um único fardo.
b) Incorreta. Trata-se da quantidade de fardos.
c) Correta. Os cálculos são os seguintes: 6 x 12 = 72 garrafas.
d) Incorreta. Trata-se do dobro da quantidade real de latas de suco.

\item
a) Incorreta. Trata-se do número de times multiplicado pelo número de reservas.
b) Incorreta. Nesse caso, não se consideraram os reservas.
c) Incorreta. Trata-se do número de times multiplicado por ele mesmo.
d) Correta. O cálculo é o seguinte: (5 + 3) x 6 = 48 alunos.
\end{enumerate}

\colorsec{Matemática -- Módulo 3 -- Treino}

\begin{enumerate}
\item
a) Incorreta. O número 25 será o primeiro.
b) Incorreta. O número 48 ficará antes de 54.
c) Correta. Colocando os números em ordem crescente, teremos: 25; 48; 54; 63; 68; 76; 95. Sendo assim, o número que aparece imediatamente antes do 63 é o 54.
d) Incorreta. O número 68 aparecerá antes de 76.

\item
a) Incorreta. O número não faz parte da sequência.
b) Incorreta. Trata-se do número de bolinhas da figura 5.
c) Incorreta. O número não faz parte da sequência.
d) Correta. (2; 6; 12; 20; 30; 42). Em cada imagem, aumenta uma bolinha em cada coluna e aumenta em 1 o número de colunas. Assim, teremos: 2; 3 x 2; 4 x 3; 5 x 4; 6 x 5; 7 x 6.

\item
a) Incorreta. Nesse caso, a sequência aumenta de 5 em 5. Para haver antecessor e sucessor, esse aumento precisa ser de 1 em 1.
b) Incorreta. Nesse caso, a sequência aumenta de 100 em 100. Para haver antecessor e sucessor, esse aumento precisa ser de 1 em 1.
c) Incorreta. Nesse caso, a sequência aumenta de 10 em 10. Para haver antecessor e sucessor, esse aumento precisa ser de 1 em 1.
d) Correta. Antecessor de um número: Número imediatamente anterior a um número na sequência dos números naturais. Sucessor de um número: Número imediatamente a frente, ou após, de um número na sequência dos números naturais.
\end{enumerate}


\colorsec{Matemática -- Módulo 4 -- Treino}

\begin{enumerate}
\item
a) Incorreta. Nesse caso, ele cumpriria apenas 3h.
b) Incorreta. Nesse caso, ele cumpriria apenas 3h30min.
c) Incorreta. Nesse caso, ele cumpriria apenas 4h.
d) Correta. Como pela manhã ele entra às 8:00 e deve cumprir nesse período 4 horas e
meia de trabalho antes de sair para o almoço, conclui-se que ele sairá para o almoço às 12:30.

\item
a) Incorreta. 1 garrafinha daria para 3 dias apenas.
b) Incorreta. 2 garrafinhas dariam para 6 dias apenas.
c) Incorreta. 3 garrafinhas dariam para 9 dias apenas.
d) Correta. 4 x 40 x 10- = 1600 mL. Como cada garrafinha contém 500 mL, ela terá que
preparar 4 garrafinhas e haverá uma sobra de chá.

\item
a) Correta. Saída: 8 horas e 15 minutos. Chegada: 11 horas e 30 minutos. Tempo de voo: 3 horas e 15 minutos.
b) Incorreta. Se o voo tivesse tido 2h de duração, ela chegaria às 10h15min.
c) Incorreta. Se o voo tivesse tido 2h45min de duração, ela chegaria às 11h.
d) Incorreta. Se o voo tivesse tido 3h50min de duração, ela chegaria às 12h05min.
\end{enumerate}

\colorsec{Matemática -- Módulo 5 -- Treino}

\begin{enumerate}
\item
a) Incorreta. Trata-se de apenas um terço do caminho.
b) Incorreta. Faltariam, ainda, 5 m.
c) Incorreta. Trata-se de quase a totalidade, mas ainda faltariam 3m.
d) Correta. Ele deverá andar 5 lados de quadrado. Como cada lado de quadrado possui
medida igual a 3 m, ele deverá andar 15 metros.

\item
a) Incorreta. O L visivelmente ocupa mais do que 3 quadradinhos.
b) Incorreta. 14 quadradinhos representakm a área aproximada de pouco mais da metade do L.
c) Correta. O L não ocupa espaços exatos, mas, de forma aproximada, ocupa 26 quadradinhos. É importante imaginá-lo mais ajustado à malha.
d) Incorreta. O L visivelmente ocupa muito menos do que 120 quadradinhos.

\item
Resposta: D
A nova pista de caminhada tem o quíntuplo da extensão da anterior.
\end{enumerate}

\colorsec{Matemática -- Módulo 6 -- Treino}

\begin{enumerate}
\item
a) Incorreta. Nesse caso, não se considerou o desconto.
b) Incorreta. Nesse caso, simplesmente se repetiu o valor do desconto.
c) Correta.Resposta: R\$ 985,00 -- R\$ 75,00 = R\$ 910,00.
d) Incorreta. Nesse caso, somou-se o valor do desconto, em vez de se descontar.

\item
a) Incorreta. Há apenas 6 moedas de 1 real representadas.
b) Incorreta. Há apenas 8 moedas de 1 real representadas.
c) Correta. Ela tem R\$ 12,00; então, receberá 12 moedas de 1 real.
d) Incorreta. Há 20 moedas de 1 real representadas.

\item
a) Incorreta. Trata-se de valor inferior ao real.
b) Incorreta. Trata-se de 13 reais a menos do que o esperado.
c) Incorreta. Trata-se de 11 reais a menos do que o esperado.
d) Correta.
1 x 5,00 + 2 x 6,00 + 2 x 12,00 = 5,00 + 12,00 + 24,00 =R\$ 41,00
\end{enumerate}

\colorsec{Matemática -- Módulo 7 -- Treino}

\begin{enumerate}
\item
a) Incorreta. Essa seria a probabilidade caso o alvo fosse qualquer caminhão.
b) Correta. Se metade dos 10 caminhões da coleção tem 6 rodas, então eles são 5 no total.
c) Incorreta. Há mais do que um caminhão de 6 rodas no total.
d) Incorreta. Nesse caso, estariam sendo considerados todos os carrinhos da coleção como alvo.

\item
a) Correta. Somando-se as fantasias de maçã, uva e kiwi, são 30 no total (e o total são 100).
b) Incorreta. Há menos do que 50 pessoas fantasiadas de frutas.
c) Incorreta. Se há pessoas fantasiadas de frutas, então essa chance existe.
d) Incorreta. Como há outras fantasias, que não são de frutas, existe a chance de a pessoa escolhida não usar uma fantasia de fruta.

\item
a) Incorreta. Nesse caso, consideraram-se os três candidatos ou as três iniciais.
b) Correta. São 7 alunos no total com iniciais A, C e T. Porém, os candidatos não votam. Restam, então, 4 alunos possíveis (Abigail, Aparecida, Carla e Coralina.)
c) Incorreta. Nesse caso, consideraram-se os candidatos como alunos votantes.
d) Incorreta. Se há alunos com as iniciais indicadas como alvos, então existe chance.
\end{enumerate}

\colorsec{Matemática -- Módulo 8 -- Treino}

\begin{enumerate}
\item
a) Correta. Pela análise da tabela, percebe-se que o candidato A teve todas as notas
acima de 30 e é o que teve mais notas iguais. Portanto, o candidato A
deverá ser aprovado.
b) Incorreta. O candidato B teve apenas duas notas 32 repetidas.
c) Incorreta. O candidato C, além de ter uma nota abaixo de 30, não tem notas repetidas.
d) Incorreta. O candidato D, além de ter uma nota abaixo de 30, não tem notas repetidas.

\item
a) Correta.  A soma 440 + 320 + 270 é igual a 1.030 alunos.
b) Incorreta. Trata-se da soma 440 + 320.
c) Incorreta. Trata-se da soma 440 + 270.
d) Incorreta. Trata-se da soma 320 + 270.

\item
a) Incorreta. Há mais do que 4 alunos com notas maiores ou iguais a 7,0.
b) Correta. Entre as notas fornecidas temos 10 notas maiores ou iguais a 7,0.
c) Incorreta. Há mais do que 6 alunos com notas maiores ou iguais a 7,0.
d) Incorreta. Há menos do que 16 alunos com notas maiores ou iguais a 7,0.
\end{enumerate}

\colorsec{Simulado 1}

\begin{enumerate}
\item
a) Incorreta. 623 equivaleria a 6 x 100 + 2 x 10 + 3 x 1 = 623.
b) Incorreta. 423 equivaleria a 4 x 100 + 2 x 10 + 3 x 1 = 423.
c) Incorreta. 503 equivaleria a 5 x 100 + 3 x 1 = 503.
d) Correta. 5 x 100 + 2 x 10 + 3 x 1 = 523.

\item
a) Incorreta. Trata-se de um número sem o último zero à direita; logo, dez vezes menor.
b) Incorreta. A posição dos algarismos está invertida.
c) Incorreta. A posição dos algarismos está invertida.
d) Correta. 2 x 1.000 + 3 x 100 + 1 x 10 = 2.310.

\item
a) Incorreta. De fato está como 4, mas deveria ser 400, não 40.
b) Incorreta. De fato deveria ser 400, mas está como 4, não como 40.
c) Correta. Como o número apresentado no enunciado está com o primeiro e o último
algarismos trocados, conclui-se que o número correto seria 483. Na placa,
o último algarismo é o 4, que tem valor relativo de 4 unidades, mas no
número correto ele estaria na centena comum, apresentando, então, um valor
relativo de 400 (quatrocentos).
d) Incorreta. Ambas as análises estão erradas.

\item
a) Incorreta. A soma não totaliza 513.
b) Incorreta. Não se trata do total de árvores.
c) Correta. 359 + 246 = 605.
d) Incorreta. Pode ter havido um erro na soma, para 100 unidades a mais.

\item
a) Incorreta. Trata-se do dobro da resposta correta.
b) Correta. 4 054 -- 2 843 = 1 211. Como o aumento foi o mesmo nnas duas escolas, não
precisamos fazer a soma do aumento aos números antigos, já que a diferença entre eles irá se manter.
c) Incorreta. Trata-se de um número menor que a diferença real.
d) Incorreta. Não é esse o número que quantifica a diferença.

\item
a) Incorreta. A sala 301 é a que fica na lateral mais perto da árvore.
b) Incorreta. A sala 302 está com a janela entreaberta.
c) Correta. A sala será a 303 seguindo as instruções do enunciado.
d) Incorreta. Seguindo a lógica, nota-se que não há sala 304.

\item
a) Correta. 42 -- 14 + 5 = 33 pontos.
b) Incorreta. Nesse caso, somaram-se 4 pontos a mais.
c) Incorreta. Nesse caso, foi considerada uma pontuação quase dobrada.
d) Incorreta. Nesse caso, foram considerados apenas os pontos da terceira rodada.

\item
a) Incorreta. Consideraram-se, nesse caso, 4 quadradinhos a menos.
b) Incorreta. Consideraram-se, nesse caso, 3 quadradinhos a menos.
c) Correta. Observando a figura e realizando a contagem do número de quadradinhos
pintados, temos que esse número é igual a 29.
Incorreta. Consideraram-se, nesse caso, 5 quadradinhos a mais.

\item
a) Incorreta. Trata-se do valor apenas de Camila.
b) Incorreta. Trata-se do valor apenas de Ana Beatriz.
c) Correta. Ana Beatriz possui R\$ 37,70; Camila possui R\$ 36,40. Então, elas possuem juntas: 37,70 + 36,40 = R\$ 74,10.
d) Incorreta. Nesse caso, pode ter havido um erro de soma.

\item
a) Incorreta. A ocupa 3 quadradinhos a mais que C.
b) Incorreta. D ocupa 1 quadradinho a menos que E.
c) Incorreta. D ocupa 2 quadradinhos a mais que C.
d) Correta. Letra A: 14 quadradinhos; letra C: 11 quadradinhos; letra D: 13 quadradinhos; letra E: 14 quadradinhos. Portanto, as duas letras que ocupam as superfícies de mesmo tamanho são A e E.

\item
b) Incorreta. Não se trata da quantidade correta.
b) Correta. Número de alunos do 4º ano: 32 + 29 + 25 = 86 
c) Incorreta. Há 5 alunos a mais nessa soma.
d) Incorreta. Não se trata da quantidade correta.

\item
a) Incorreta. A sequência é crescente.
b) Incorreta. Não se trata de sequência ordenada.
c) Incorreta. Não se trata de sequência ordenada.
d) Correta. (400; 326; 280; 153; 120; 88; 25)

\item
a) Correta. 273 x 3 = 819
b) Incorreta. Não seria o cálculo de multiplicação correto.
c) Incorreta. Nesse caso, a multiplicação seria por 2.
d) Incorreta. Trata-se, simplesmente, da repetição de um dos números que aparece na lousa.

\item
a) Correta. Observando a posição dos ponteiros, conclui-se que a hora marcada pelo
relógio é 8 horas e 30 minutos.
b) Incorreta. Nesse caso, o ponteiro menos deveria apontar para perto do número 7 e o maior deveria apontar para o número 3.
c) Incorreta. Nesse caso, o ponteiro menos deveria apontar para perto do número 9 e o maior deveria apontar para o número 10.
d) Incorreta. Nesse caso, o ponteiro menos deveria apontar para perto do número 8 e o maior deveria apontar para o número 10.

\item
a) Incorreta. O lápis menor mede menos da metade do maior.
b) Correta. Pela análise da figura, percebe-se que o lápis maior mede quatro vezes a medida do lápis menor.
c) Incorreta. Isso significaria uma diferença de comprimeiro bem maior entre os lápis.
d) Incorreta. Isso significaria que os dois lápis seriam do mesmo tamanho.
\end{enumerate}

\colorsec{Simulado 2}

\begin{enumerate}
\item

\item

\item

\item

\item

\item

\item

\item

\item

\item

\item

\item

\item

\item

\item
\end{enumerate}

\colorsec{Simulado 3}

\begin{enumerate}
\item

\item

\item

\item

\item

\item

\item

\item

\item

\item

\item

\item

\item

\item

\item
\end{enumerate}

\colorsec{Simulado 4}

\begin{enumerate}
\item

\item

\item

\item

\item

\item

\item

\item

\item

\item

\item

\item

\item

\item

\item
\end{enumerate}