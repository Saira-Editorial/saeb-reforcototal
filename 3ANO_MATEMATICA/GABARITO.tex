\chapter{Respostas}
\pagestyle{plain}
\footnotesize

\pagecolor{gray!40}

\colorsec{Matemática -- Módulo 1 -- Treino}

\begin{enumerate}
\item 
a) Correta. A decomposição 700 + 30 + 4 equivale a 734.
b) Incorreta. 70 + 3 + 4 = 707.
c) Incorreta. 700 + 40 + 3 = 743.
d) Incorreta. 70 + 300 + 4 = 374.

\item
a) Incorreta. Neste ponto estará o número 550.
b) Incorreta. Neste ponto estará o número 560.
c) Correta. O 570 estará no ponto S, pois, como no ponto P está o 540 e cada
repartição é de 10 unidades, ele deve estar no terceiro ponto após o P,
sem contar o ponto S.
d) Incorreta. Neste ponto estará o número 580.

\item
a) Incorreta. Repetem-se os algarismos 5 e 9, mas o número formado está errado.
b) Incorreta. Não se trata do número formado.
c) Correta. 5 x 100 + 1 x 9 = 509.
d) Incorreta. 1.509 tem 1000 unidades a mais do que 509.
\end{enumerate}

\colorsec{Matemática -- Módulo 2 -- Treino}

\begin{enumerate}
\item
a) Incorreta. Trata-se somente do número de botões.
b) Incorreta. Trata-se, simplesmente, do número de camisas.
c) Incorreta. Trata-se da soma de 6 com 9.
d) Correta. A conta é a seguinte: 6 x 9 = 54 botões.

\item
a) Incorreta. Trata-se do número de latas em um único fardo.
b) Incorreta. Trata-se da quantidade de fardos.
c) Correta. Os cálculos são os seguintes: 6 x 12 = 72 garrafas.
d) Incorreta. Trata-se do dobro da quantidade real de latas de suco.

\item
a) Incorreta. Trata-se do número de times multiplicado pelo número de reservas.
b) Incorreta. Nesse caso, não se consideraram os reservas.
c) Incorreta. Trata-se do número de times multiplicado por ele mesmo.
d) Correta. O cálculo é o seguinte: (5 + 3) x 6 = 48 alunos.
\end{enumerate}

\colorsec{Matemática -- Módulo 3 -- Treino}

\begin{enumerate}
\item

\item

\item
\end{enumerate}


\colorsec{Matemática -- Módulo 4 -- Treino}

\begin{enumerate}
\item

\item

\item
\end{enumerate}

\colorsec{Matemática -- Módulo 5 -- Treino}

\begin{enumerate}
\item

\item

\item
\end{enumerate}

\colorsec{Matemática -- Módulo 6 -- Treino}

\begin{enumerate}
\item

\item

\item
\end{enumerate}

\colorsec{Matemática -- Módulo 7 -- Treino}

\begin{enumerate}
\item

\item

\item
\end{enumerate}

\colorsec{Matemática -- Módulo 8 -- Treino}

\begin{enumerate}
\item

\item

\item
\end{enumerate}

\colorsec{Simulado 1}

\begin{enumerate}
\item

\item

\item

\item

\item

\item

\item

\item

\item

\item

\item

\item

\item

\item

\item

\item

\item

\item

\item

\item
\end{enumerate}

\colorsec{Simulado 2}

\begin{enumerate}
\item

\item

\item

\item

\item

\item

\item

\item

\item

\item

\item

\item

\item

\item

\item

\item

\item

\item

\item

\item
\end{enumerate}

\colorsec{Simulado 3}

\begin{enumerate}
\item

\item

\item

\item

\item

\item

\item

\item

\item

\item

\item

\item

\item

\item

\item

\item

\item

\item

\item

\item
\end{enumerate}

\colorsec{Simulado 4}

\begin{enumerate}
\item

\item

\item

\item

\item

\item

\item

\item

\item

\item

\item

\item

\item

\item

\item

\item

\item

\item

\item

\item
\end{enumerate}