\chapter{Respostas}
\pagestyle{plain}
\footnotesize

\pagecolor{gray!40}

\colorsec{Matemática -- Módulo 1 -- Treino}

\begin{enumerate}
\item 
a) Correta. A decomposição 700 + 30 + 4 equivale a 734.
b) Incorreta. 70 + 3 + 4 = 707.
c) Incorreta. 700 + 40 + 3 = 743.
d) Incorreta. 70 + 300 + 4 = 374.

\item
a) Incorreta. Neste ponto estará o número 550.
b) Incorreta. Neste ponto estará o número 560.
c) Correta. O 570 estará no ponto S, pois, como no ponto P está o 540 e cada
repartição é de 10 unidades, ele deve estar no terceiro ponto após o P,
sem contar o ponto S.
d) Incorreta. Neste ponto estará o número 580.

\item
a) Incorreta. Repetem-se os algarismos 5 e 9, mas o número formado está errado.
b) Incorreta. Não se trata do número formado.
c) Correta. 5 x 100 + 1 x 9 = 509.
d) Incorreta. 1.509 tem 1000 unidades a mais do que 509.
\end{enumerate}

\colorsec{Matemática -- Módulo 2 -- Treino}

\begin{enumerate}
\item
a) Incorreta. Trata-se somente do número de botões.
b) Incorreta. Trata-se, simplesmente, do número de camisas.
c) Incorreta. Trata-se da soma de 6 com 9.
d) Correta. A conta é a seguinte: 6 x 9 = 54 botões.

\item
a) Incorreta. Trata-se do número de latas em um único fardo.
b) Incorreta. Trata-se da quantidade de fardos.
c) Correta. Os cálculos são os seguintes: 6 x 12 = 72 garrafas.
d) Incorreta. Trata-se do dobro da quantidade real de latas de suco.

\item
a) Incorreta. Trata-se do número de times multiplicado pelo número de reservas.
b) Incorreta. Nesse caso, não se consideraram os reservas.
c) Incorreta. Trata-se do número de times multiplicado por ele mesmo.
d) Correta. O cálculo é o seguinte: (5 + 3) x 6 = 48 alunos.
\end{enumerate}

\colorsec{Matemática -- Módulo 3 -- Treino}

\begin{enumerate}
\item
a) Incorreta. O número 25 será o primeiro.
b) Incorreta. O número 48 ficará antes de 54.
c) Correta. Colocando os números em ordem crescente, teremos: 25; 48; 54; 63; 68; 76; 95. Sendo assim, o número que aparece imediatamente antes do 63 é o 54.
d) Incorreta. O número 68 aparecerá antes de 76.

\item
a) Incorreta. O número não faz parte da sequência.
b) Incorreta. Trata-se do número de bolinhas da figura 5.
c) Incorreta. O número não faz parte da sequência.
d) Correta. (2; 6; 12; 20; 30; 42). Em cada imagem, aumenta uma bolinha em cada coluna e aumenta em 1 o número de colunas. Assim, teremos: 2; 3 x 2; 4 x 3; 5 x 4; 6 x 5; 7 x 6.

\item
a) Incorreta. Nesse caso, a sequência aumenta de 5 em 5. Para haver antecessor e sucessor, esse aumento precisa ser de 1 em 1.
b) Incorreta. Nesse caso, a sequência aumenta de 100 em 100. Para haver antecessor e sucessor, esse aumento precisa ser de 1 em 1.
c) Incorreta. Nesse caso, a sequência aumenta de 10 em 10. Para haver antecessor e sucessor, esse aumento precisa ser de 1 em 1.
d) Correta. Antecessor de um número: Número imediatamente anterior a um número na sequência dos números naturais. Sucessor de um número: Número imediatamente a frente, ou após, de um número na sequência dos números naturais.
\end{enumerate}


\colorsec{Matemática -- Módulo 4 -- Treino}

\begin{enumerate}
\item
a) Incorreta. Nesse caso, ele cumpriria apenas 3h.
b) Incorreta. Nesse caso, ele cumpriria apenas 3h30min.
c) Incorreta. Nesse caso, ele cumpriria apenas 4h.
d) Correta. Como pela manhã ele entra às 8:00 e deve cumprir nesse período 4 horas e
meia de trabalho antes de sair para o almoço, conclui-se que ele sairá para o almoço às 12:30.

\item
a) Incorreta. 1 garrafinha daria para 3 dias apenas.
b) Incorreta. 2 garrafinhas dariam para 6 dias apenas.
c) Incorreta. 3 garrafinhas dariam para 9 dias apenas.
d) Correta. 4 x 40 x 10- = 1600 mL. Como cada garrafinha contém 500 mL, ela terá que
preparar 4 garrafinhas e haverá uma sobra de chá.

\item
a) Correta. Saída: 8 horas e 15 minutos. Chegada: 11 horas e 30 minutos. Tempo de voo: 3 horas e 15 minutos.
b) Incorreta. Se o voo tivesse tido 2h de duração, ela chegaria às 10h15min.
c) Incorreta. Se o voo tivesse tido 2h45min de duração, ela chegaria às 11h.
d) Incorreta. Se o voo tivesse tido 3h50min de duração, ela chegaria às 12h05min.
\end{enumerate}

\colorsec{Matemática -- Módulo 5 -- Treino}

\begin{enumerate}
\item
a) Incorreta. Trata-se de apenas um terço do caminho.
b) Incorreta. Faltariam, ainda, 5 m.
c) Incorreta. Trata-se de quase a totalidade, mas ainda faltariam 3m.
d) Correta. Ele deverá andar 5 lados de quadrado. Como cada lado de quadrado possui
medida igual a 3 m, ele deverá andar 15 metros.

\item
a) Incorreta. O L visivelmente ocupa mais do que 3 quadradinhos.
b) Incorreta. 14 quadradinhos representakm a área aproximada de pouco mais da metade do L.
c) Correta. O L não ocupa espaços exatos, mas, de forma aproximada, ocupa 26 quadradinhos. É importante imaginá-lo mais ajustado à malha.
d) Incorreta. O L visivelmente ocupa muito menos do que 120 quadradinhos.

\item
Resposta: D
A nova pista de caminhada tem o quíntuplo da extensão da anterior.
\end{enumerate}

\colorsec{Matemática -- Módulo 6 -- Treino}

\begin{enumerate}
\item

\item

\item
\end{enumerate}

\colorsec{Matemática -- Módulo 7 -- Treino}

\begin{enumerate}
\item

\item

\item
\end{enumerate}

\colorsec{Matemática -- Módulo 8 -- Treino}

\begin{enumerate}
\item

\item

\item
\end{enumerate}

\colorsec{Simulado 1}

\begin{enumerate}
\item

\item

\item

\item

\item

\item

\item

\item

\item

\item

\item

\item

\item

\item

\item

\item

\item

\item

\item

\item
\end{enumerate}

\colorsec{Simulado 2}

\begin{enumerate}
\item

\item

\item

\item

\item

\item

\item

\item

\item

\item

\item

\item

\item

\item

\item

\item

\item

\item

\item

\item
\end{enumerate}

\colorsec{Simulado 3}

\begin{enumerate}
\item

\item

\item

\item

\item

\item

\item

\item

\item

\item

\item

\item

\item

\item

\item

\item

\item

\item

\item

\item
\end{enumerate}

\colorsec{Simulado 4}

\begin{enumerate}
\item

\item

\item

\item

\item

\item

\item

\item

\item

\item

\item

\item

\item

\item

\item

\item

\item

\item

\item

\item
\end{enumerate}