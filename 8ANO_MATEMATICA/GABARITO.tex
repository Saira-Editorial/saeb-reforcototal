\chapter{Respostas}
\pagestyle{plain}
\footnotesize

\pagecolor{gray!40}

\section*{Matemática – Módulo 1 – Treino}

\begin{enumerate}
\item

Habilidades Saeb

Compor ou decompor números racionais positivos (representação decimal
  finita) na forma aditiva, ou em suas ordens, ou em adições e
  multiplicações.

A: Incorreta ao não computar 1 elemento 2 e um elemento 5 na fatoração
chegará a esse resultado erroneamente.

B: Incorreta ao computar 1 elemento 2 a mais e um elemento 5 a menos na
fatoração chegará a esse resultado erroneamente.

C: Correta, Ao decompor o número 1 000 em fatores primos obtemos 2³ x 5³

D: Incorreta ao não computar 1 elemento 5 na fatoração chegará a esse
resultado erroneamente.

\item

Habilidades Saeb

  Identificar um número natural como primo, composto, ``múltiplo/fator
  de'' ou ``divisor de'' ou identificar a decomposição de um número
  natural em fatores primos ou relacionar as propriedades aritméticas
  (primo, composto, ``múltiplo/fator de'' ou ``divisor de'') de um
  número natural à sua decomposição em fatores primos.

A: incorreta, o aluno pode realizar a divisão incorretamente do valor
123 456 789 e chegar a essa conclusão ao errar nos cálculos básicos.

B: incorreta, o aluno pode realizar a divisão incorretamente do valor
123 456 789 e chegar a essa conclusão ao errar nos cálculos básicos.

C: Correta, Realizando a soma de $1+2+3+4+5+6+7+8+9=45$ logo $45$ é
múltiplo de 3, então 123 456 789 também será.

D: incorreta, o aluno pode realizar a divisão incorretamente do valor
123 456 789 e chegar a essa conclusão ao errar nos cálculos básicos.

\item

Habilidades Saeb

  Compor ou decompor números racionais positivos (representação decimal
  finita) na forma aditiva, ou em suas ordens, ou em adições e
  multiplicações.

A: incorreta: ao contar as casas erroneamente e considerar o número uma
casa a direita o aluno pode considerar esse resultado erroneamente.

Alternativa: B incorreta ao contar as casas erroneamente e considerar o
número duas casas a direita o aluno pode considerar esse resultado
erroneamente.

C: incorreta, ao contar as casas erroneamente e considerar o número uma
casa a esquerda o aluno pode considerar esse resultado erroneamente.

D, Correta o número 3 está situado na casa das centenas de milhar.

\end{enumerate}


\section*{Matemática – Módulo 2 – Treino}

\begin{enumerate}
\item


BNCC: EF09MA01

Habilidade Saeb:

Resolver problemas de adição, subtração, multiplicação, divisão,
potenciação ou radiciação envolvendo número reais, inclusive notação
científica

A, incorreta, ao considerar o número de ``zeros'' incorretamente ao
invés de 4 zeros consideráveis para 2, ocorreria esse erro.

B, incorreta, ao considerar o número de ``zeros'' incorretamente ao
invés de 4 zeros consideráveis para 3, ocorreria esse erro.

C, Correta pois:

10000 =
10\textsuperscript{.}10\textsuperscript{.}10\textsuperscript{.}10 ou
seja 10\textsuperscript{4}

D, incorreta, ao considerar o número de ``zeros'' incorretamente ao
invés de 4 zeros consideráveis para 5, ocorreria esse erro.


\item


BNCC: EF09MA01

Habilidade Saeb:

Resolver problemas de adição, subtração, multiplicação, divisão,
potenciação ou radiciação envolvendo número reais, inclusive notação
científica

A: incorreta, ao confundir expoente com base chegaria a esse resultado
erroneamente.

B: incorreta: o aluno chegaria a esse conclusão se somasse todos os
expoentes incorretamente.

C: incorreta, pois ao dividir o expoente do numerado por 2 ao invés de
3, o aluno chegaria a esse resultado.

D: Correta pois, utilizando a propriedade de multiplicação e divisão de
potencias de mesma base temos que

No numerador 2\textsuperscript{25+35+30}=2\textsuperscript{90}

No denominador 2\textsuperscript{2+1}=2³

Realizando a divisão temos que
2\textsuperscript{90:3}=2\textsuperscript{30}

Considerando que a idade de Thiago é apenas o expoente temos que Thiago
tem 30 anos.

\item

BNCC: EF09MA01

Habilidade Saeb:

Resolver problemas de adição, subtração, multiplicação, divisão,
potenciação ou radiciação envolvendo número reais, inclusive notação
científica.

A:Correta pois:

2\textsuperscript{10} = 1024

1024^{.}32= 32768 megabytes.

B: incorreta, o aluno chegaria a esse resultado erroneamente ao
considerar que 2\textsuperscript{10} seja 1 000 ao invés de 1 024.

C: incorreta, ao considerar um ``dois'' a menos na expressão, chegaria a
esse resultado incorreto.

D: incorreta, ao realizar apenas a multiplicação ao invés de realizar o
cálculo da potência o aluno chegaria a esse resultado erroneamente.


\end{enumerate}

\section*{Matemática – Módulo 3 – Treino}

\begin{numerate}
\item
BNCC: EF08MA05

Habilidade Saeb:

Determinar uma fração geratriz para uma dízima periódica.

A: incorreta, o aluno pode ter uma confusão conceitual passando a
acreditar que o número (\pi) seja de fato uma dizima periódica simples
pelo fato de ter números pares na sua composição.

B: incorreta, o aluno pode ter uma confusão conceitual ao não conseguir
identificar uma a diferença entre uma dizima periódica simples e um
irracional.

C, Correta. Questão conceitual.

D, incorreta, o aluno pode ter uma confusão conceitual ao não conseguir
identificar uma a diferença entre uma dizima periódica simples e um
irracional.

\item
BNCC: EF08MA05

Habilidade Saeb: Representar frações menores ou maiores que a unidade
por meio de representações pictóricas ou associar frações a
representações pictóricas.

A: incorreta, o aluno chegaria a esse resultado caso realizasse a soma
de ambos os termos do enunciado.

B: incorreta, o aluno chegaria a esse valor caso multiplicasse os
valores do enunciado.

C:incorreta, o aluno chegaria a esse valor caso dividisse os valores do
enunciado.

D:Correta, pois

Considerando que (\frac{1}{3}) não gostou ou 0,333333333... Retirando
(\frac{1}{6}) obtemos 0,0555555.\ldots{}

Logo

55,55555555... -\/-\/-\/-\/-\/-\/-\/-\/-\/-\/-\/-\/-\/-\/- 1000 x

5,555555555.\ldots{} -\/-\/-\/-\/-\/-\/-\/-\/-\/-\/-\/-\/-\/-\/- 100x

55,555555.\ldots. -- 5,5555555.\ldots{} = 50

1000x -- 100x = 900

Logo temos (\frac{50}{900}) dividindo numerador e denominador por 50
temos (\frac{1}{18})


\item

BNCC: EF08MA05

Habilidade Saeb: Determinar uma fração geratriz para uma dízima
periódica.

A: incorreta, o aluno pode chegar a esse valor ao calcular erroneamente
a expressão e ao invés de calcular 1 800 : 99 , calcular 1 800 : 9

B: incorreta, o aluno pode chegar a esse valor ao calcular erroneamente
a expressão e ao invés de calcular 1 800 : 99 , calcular 1 000 : 9

C: Correta, pois: 400 g : 22 pedaços = 18,1818181818.\ldots{}

18,181818181818...
-\/-\/-\/-\/-\/-\/-\/-\/-\/-\/-\/-\/-\/-\/-\/-\/-\/-\/-\/- x

1818,18181818...
-\/-\/-\/-\/-\/-\/-\/-\/-\/-\/-\/-\/-\/-\/-\/-\/-\/-\/-\/-\/-\/-\/-100x

1818,181818.\ldots{} -- 18,18181818... = 1800

100x -- x = 99x

Logo temos (\frac{1\ 800}{99}) dividindo ambos por 9 temos que
(\frac{200}{9})

D : incorreta, o aluno pode chegar a esse valor ao calcular erroneamente
a expressão e ao invés de calcular 1 800 : 99 , calcular 1 800 : 101

\end{numerate}


\section*{Matemática – Módulo 4 – Treino}
\begin{numerate}
\item


