\chapter{Respostas}
\pagestyle{plain}
\footnotesize

\pagecolor{gray!40}

\section*{Matemática – Módulo 1 – Treino}

\begin{enumerate}
\item SAEB: Compor ou decompor números racionais positivos (representação
decimal finita) na forma aditiva, ou em suas ordens, ou em adições e
multiplicações.

a) Incorreta. O aluno não computou um elemento 2 e um elemento 5
na fatoração.
b) Incorreta. O aluno computou um elemento a mais e um elemento 5 a
menos na fatoração.
c) Correta. Ao decompor o número 1.000 em fatores primos, obtemos
$$2^3 \times 5^3$$.
d) Incorreta. O aluno não computou um elemento 5 na fatoração.

\item SAEB: Identificar um número natural como primo, composto,
``múltiplo/fator de'' ou ``divisor de'' ou identificar a decomposição de
um número natural em fatores primos ou relacionar as propriedades
aritméticas (primo, composto, ``múltiplo/fator de'' ou ``divisor de'')
de um número natural à sua decomposição em fatores primos.

a) Incorreta. O aluno pode realizar a divisão incorretamente do
valor 123.456.789 e chegar a essa conclusão.
b) Incorreta. O aluno pode realizar a divisão incorretamente do
valor 123.456.789 e chegar a essa conclusão.
c) Correta. Temos: 1 + 2 + 3 + 4 + 5 + 6 + 7 + 8 + 9 = 45, que é
múltiplo de 3; então 123.456.789 também será.
d) Incorreta. O aluno pode realizar a divisão incorretamente do
valor 123.456.789 e chegar a essa conclusão.

\item SAEB: Compor ou decompor números racionais positivos (representação
decimal finita) na forma aditiva, ou em suas ordens, ou em adições e
multiplicações.

a) Incorreta. Ao contar as casas erroneamente e considerar o
número uma casa à direita o aluno pode considerar esse.
b) Incorreta. Ao contar as casas erroneamente e considerar o
número duas casas à direita, o aluno pode considerar esse resultado.
c) Incorreta. Ao contar as casas erroneamente e considerar o
número uma casa à esquerda, o aluno pode considerar esse resultado.
d) Correta. O número 3 está situado na casa das centenas de milhar.

\end{enumerate}


\section*{Matemática – Módulo 2 – Treino}

\begin{enumerate}
\item SAEB: Resolver problemas de adição, subtração, multiplicação, divisão,
potenciação ou radiciação envolvendo número reais, inclusive notação
científica.
BNCC: EF08MA02 -- Resolver e elaborar problemas usando a relação entre
potenciação e radiciação, para representar uma raiz como potência de
expoente fracionário.

a) Incorreta. O número de zeros estava errado.
b) Incorreta. O número de zeros estava errado.
c) Correta. 10.000 = 10 . 10 . 10 . 10.
d) Incorreta. O número de zeros estava errado.

\item SAEB: Resolver problemas de adição, subtração, multiplicação, divisão,
potenciação ou radiciação envolvendo número reais, inclusive notação
científica

a) Incorreta. Ao confundir expoente com base, o aluno chegaria a
esse resultado.
b) Incorreta. O aluno chegaria a esse conclusão se somasse todos os
expoentes incorretamente.
c) Incorreta. Ao dividir o expoente do numerador por 2 ao invés de
3, o aluno chegaria a esse resultado.
d) Correta. Utilizando a propriedade de multiplicação e divisão de
potências de mesma base, temos: $$(2^{25+35+30}=2^{90})$$ no numerador
e $$(2^{2+1}) = 2^3$$ no denominador.
Realizando a divisão, temos que $$(2^{90}:3) = (2^{30})$$.
Considerando que a idade de Thiago é apenas o expoente, sabemos que
Thiago tem 30 anos.

\item SAEB: Resolver problemas de adição, subtração, multiplicação, divisão,
potenciação ou radiciação envolvendo número reais, inclusive notação
científica.
BNCC: EF08MA01 -- Efetuar cálculos com potências de expoentes inteiros e
aplicar esse conhecimento na representação de números em notação
científica.

a) Correta. $$(2^10) = 1024 \times 32 = 32.768$$ megabytes.
b) Incorreta. O aluno chegaria a esse resultado ao considerar que
$$(2^{10})$$ seja 1.000 ao invés de 1.024.
c) Incorreta. Ao considerar um 2 a menos na expressão, o aluno
chegaria a esse resultado.
d) Incorreta. Ao realizar apenas a multiplicação ao invés de
realizar o cálculo da potência, o aluno chegaria a esse resultado.


\end{enumerate}

\section*{Matemática – Módulo 3 – Treino}

\begin{numerate}
\item SAEB: Determinar uma fração geratriz para uma dízima periódica.
BNCC: EF08MA05 -- Reconhecer e utilizar procedimentos para a obtenção de
uma fração geratriz para uma dízima periódica.

a) Incorreta. O aluno pode considerar que o número (\pi) seja de
fato uma dízima periódica simples pelo fato de ter números pares na sua
composição.
b) Incorreta. O aluno pode não conhecer a diferença entre uma
dízima periódica simples e uma irracional.
c) Correta. O conceito foi descrito corretamente.
d) Incorreta. O aluno pode não ser capaz de identificar a diferença
entre uma dizima periódica simples e uma irracional.

\item SAEB: Representar frações menores ou maiores que a unidade por meio de
representações pictóricas ou associar frações a representações
pictóricas.
BNCC: EF08MA05 -- Reconhecer e utilizar procedimentos para a obtenção de
uma fração geratriz para uma dízima periódica.

a) Incorreta. Esse valor corresponde a $$(\frac{2}{3})$$ dos
participantes.
b) Incorreta. Esse valor corresponde à metade dos participantes.
c) Incorreta. Esse valor corresponde ao dobro dos partcipantes.
d) Correta. Um terço de 300 é igual a 100.


\item SAEB: Determinar uma fração geratriz para uma dízima periódica.
BNCC: EF08MA05 -- Reconhecer e utilizar procedimentos para a obtenção de
uma fração geratriz para uma dízima periódica.

a) Incorreta. O aluno pode chegar a esse valor ao calcular
erroneamente a expressão.
b) Incorreta. O aluno pode chegar a esse valor ao calcular
erroneamente a expressão.
c) Correta. 400 g : 22 pedaços = 18,1818181818\ldots{}.
d) Incorreta. O aluno pode chegar a esse valor ao calcular
erroneamente a expressão.

\end{numerate}


\section*{Matemática – Módulo 4 – Treino}
\begin{numerate}
\item SAEB: Resolver problemas que envolvam porcentagens, incluindo os que
lidam com acréscimos e decréscimos simples, aplicação de percentuais
sucessivos e determinação de taxas percentuais.
BNCC: EF08MA04 -- Resolver e elaborar problemas, envolvendo cálculo de
porcentagens, incluindo o uso de tecnologias digitais.

a) Correta. Somando os dois líquidos, temos que 120 + 80 = 200.
Utilizando a regra de 3 simples temos que:
$$(\frac{200}{120} = \frac{100}{x})
Logo 200 . x = 100 . 120
200 x = 12 000
x = 60\%.$$
b) Incorreta. O aluno chegaria a esse resultado caso inserisse um
zero a menos na expressão.
c) Incorreta. O aluno chegaria a esse resultado caso inserisse dois
zeros a menos na expressão.
d) incorreta. O aluno chegaria a esse resultado caso inserisse três
zeros a menos na expressão.

\item SAEB: Resolver problemas que envolvam porcentagens, incluindo os que
lidam com acréscimos e decréscimos simples, aplicação de percentuais
sucessivos e determinação de taxas percentuais.
BNCC: EF08MA04 -- Resolver e elaborar problemas, envolvendo cálculo de
porcentagens, incluindo o uso de tecnologias digitais.

a) Incorreta. O aluno chegaria a essa conclusão realizando a
multiplicação reta na regra de três ao invés de realizar a multiplicação
cruzada.
b) Correta. $$(\frac{x}{22} = \frac{100}{55})
55 . x = 22 . 100
55x = 2200
x = 44$$ pessoas votaram nessa eleição.
c) Incorreta. O aluno poderia chegar a essa conclusão confundindo o
valor de votos ao total real de pessoas que votaram.
d) Incorreta. O aluno poderia chegar a essa conclusão confundindo o
valor da porcentagem com o valor total de pessoas na votação.

\item SAEB: Resolver problemas que envolvam porcentagens, incluindo os que
lidam com acréscimos e decréscimos simples, aplicação de percentuais
sucessivos e determinação de taxas percentuais.

BNCC: EF08MA04 -- Resolver e elaborar problemas, envolvendo cálculo de
porcentagens, incluindo o uso de tecnologias digitais.

A: Correta, pois, aplicando juros simples, temos que

J = C × i × t

6 000 = 120 000 . 0,01 . x

6 000 = 1200 x

x = 5 meses

B: Incorreta, pois o aluno chegaria a essa conclusão caso não realizasse
a operação de 1\% : 100.

C: Incorreta, pois o aluno chegaria a essa conclusão caso multiplicasse
120.000 por 0,01.

D: Incorreta, pois o aluno chegaria a essa conclusão caso considerasse
que o valor dos juros a 1\% ao mês.
\end{enumerate}


\section*{Matemática – Módulo 5 – Treino}
\begin{numerate}
\item SAEB: Resolver problemas que possam ser representados por sistema de
equações de 1º grau com duas incógnitas.

BNCC: EF08MA08 -- Resolver e elaborar problemas relacionados ao seu
contexto próximo, que possam ser representados por sistemas de equações
de 1º grau com duas incógnitas e interpretá-los, utilizando, inclusive,
o plano cartesiano como recurso.

A: Incorreta, epis sse seria o resultado em litros do primeiro tanque.

B: Correta, pois, definindo como x e y os tanques, temos que:

x + y = 900 x - 100 = y + 100 y + 100 = 2. (x - 100)

Isolando o termo y temos que: y + 100 = 2x- 200 y = 2x - 200 - 100 = 2x
- 300

Substituindo y na primeira equação x + y = 900 x + 2x - 300 = 900 3x =
900 + 300 = 1.200 x = 1.200 : 3 x = 400

Analogamente, temos que x + y = 900 400 + y = 900 y = 900 - 400 y = 500
L

C: Incorreta, pois o aluno chegará a essa conclusão ao errar o jogo de
sinal na equação final.

D: Incorreta, pois o aluno chegaria a essa conclusão ao somar os valores
do enunciado sem ao mesmo tentar realizar as operações.

\item SAEB: Resolver problemas que possam ser representados por sistema de
equações de 1º grau com duas incógnitas.

BNCC: EF08MA08 -- Resolver e elaborar problemas relacionados ao seu
contexto próximo, que possam ser representados por sistemas de equações
de 1º grau com duas incógnitas e interpretá-los, utilizando, inclusive,
o plano cartesiano como recurso.

A:Incorreta, pois o aluno pode confundir as operações e realizar a
operação de multiplicação .

B: Incorreta, pois o aluno pode esquecer de realizar o m.m.c. e chegar a
essa conclusão.

C: Correta, pois, extraindo as informações do enunciado, temos que:

X + (\frac{1}{3}) = 11

x = 11 - (\frac{1}{3})

x = (\frac{32}{3})

D: Incorreta, pois o aluno pode confundir as operações e realizar a
operação de divisão .

\item SAEB: Resolver uma equação polinomial de 1º grau.

BNCC: EF08MA07 -- Associar uma equação linear de 1º grau com duas
incógnitas a uma reta no plano cartesiano.

A: Incorreta, pois, durante a resolução da equação, caso o aluno erre o
sinal de 2,5 e passe o negativo, o valor final será esse.

B: Correta, pois, realizando as operações algébricas, temos que:

2x - 7 = -2,5x + 2

4,5x = 9

x = 2

C: Incorreta, pois, durante a resolução da equação, caso o aluno erre o
sinal de -7 e passe o negativo, o valor final será esse.

D: Incorreta, pois, caso o aluno, no final da expressão, passe o valor
4,5 multiplicando ao invés de dividir, chegará a esse resultado.
\end{enumerate}


\section*{Matemática – Módulo 6 – Treino}
\begin{numerate}
\item SAEB: Resolver problemas que envolvam cálculo do valor numérico de
expressões algébricas.

BNCC: EF08MA10 -- Identificar a regularidade de uma sequência numérica
ou figural não recursiva e construir um algoritmo por meio de um
fluxograma que permita indicar os números ou as figuras seguintes.

A: Correta, pois, considerando a medida dos lados, temos que:

3x + 2 . x + 2=

(3x + 2) . (x + 2)=

Utilizando a distributiva:

3x^2 + 6x + 2x + 4

3x^2 + 8x + 4

B: Incorreta, pois o aluno poderia confundir o enunciado e colocar o
resultado do perímetro.

C: Incorreta, pois o aluno poderia realizar uma soma ao invés de uma
multiplicação.

D: Incorreta, pois o aluno poderia chegar a esse valor realizando uma
divisão ao invés de uma multiplicação.
\item SAEB: Resolver problemas que envolvam cálculo do valor numérico de
expressões algébricas.

BNCC: EF08MA10 -- Identificar a regularidade de uma sequência numérica
ou figural não recursiva e construir um algoritmo por meio de um
fluxograma que permita indicar os números ou as figuras seguintes.

A: Incorreta, pois o aluno chegaria a essa conclusão apenas dividindo um
termo pelo outro.

B: Incorreta, pois o aluno chegaria a essa conclusão realizando apenas a
subtração de um termo pelo outro.

C: Correta, pois

(\frac{(a + b)^2}{8}\ ) = (\frac{a^{2} + 2ab + b^2}{8})

Fazendo a substituição de a^2 + b^2 = 34 e ab = 15, temos que:

(\frac{34 + (2.15)}{8} =) (\frac{34 + 30}{8}) = (\frac{64}{8}) = 8

D: incorreta, pois foi realizada a soma ao invés da multiplicação no
último termo.
\item SAEB: Resolver problemas que envolvam cálculo do valor numérico de
expressões algébricas.

BNCC: EF08MA10 -- Identificar a regularidade de uma sequência numérica
ou figural não recursiva e construir um algoritmo por meio de um
fluxograma que permita indicar os números ou as figuras seguintes.

A: Incorreta, pois o aluno, ao errar o jogo de sinal no cálculo da área,
encontrará esse valor.

B: Incorreta, pois o aluno, ao errar o jogo de sinal no cálculo do
perímetro, encontrará esse valor.

C: Incorreta, pois o aluno, ao errar o jogo de sinal no cálculo da área,
encontrará esse valor.

D: Correta, pois, para o cálculo da área, temos:

(A = \pi r^{2}).

Logo,

A = 3 . (x^2-3) ^2

A = 3 . ((x^4) - 6x^2 + 9)

A = (3x^4) - 18x^2 + 27

Perímetro

P= 2(\text{\ π\ .\ r})

P = 2 . 3. (x^2-3)

P = 2 . (3x^2 - 9)

P = 6x^2 - 18
\end{enumerate}


\section*{Matemática – Módulo 7 – Treino}
\begin{numerate}
\item SAEB: Resolver problemas que possam ser representados por equações
polinomiais de 2º grau.

BNCC: EF08MA09 -- Resolver e elaborar, com e sem uso de tecnologias,
problemas que possam ser representados por equações polinomiais de 2º
grau do tipo ax2 = b.

A: Incorreta, pois o aluno pode chegar a essa conclusão considerando que
o enunciado pede apenas 1 valor das raízes da equação.


A: Incorreta, pois o aluno pode chegar a essa conclusão considerando que


%% Encontro Felipe Jorge

A: Incorreta, pois o aluno pode chegar a essa conclusão considerando que
o enunciado pede apenas 1 valor das raízes da equação.

B: Correta, pois, utilizando Bhaskhara, temos:

x^2 - 7x = 0

A = 1

B = - 7

C = 0

(\frac{- ( - 7) \pm \sqrt{{( - 7)}^{2} - 4.1.0}}{2.1})

(\frac{7 \pm \sqrt{49}}{2})

(\frac{7 \pm 7}{2})

X1 = (\frac{7 + 7}{2})\$ = (\frac{14}{2}) = 7

X2 = (\frac{7 - 7}{2}) = 0

Logo, realizando a soma de 7 + 0 = 7, obtemos que Clebinho chutou 7
vezes ao gol.

C: Incorreta, pois o aluno pode chegar a essa conclusão considerando que
o enunciado pede apenas o valor antes de extrairmos as raízes da
equação.

D: Incorreta, pois o aluno pode chegar a essa conclusão esquecendo de
dividir o valor de uma das raízes por 2.
\item SAEB: Resolver problemas que possam ser representados por equações
polinomiais de 2º grau.

BNCC: EF08MA09 -- Resolver e elaborar, com e sem uso de tecnologias,
problemas que possam ser representados por equações polinomiais de 2º
grau do tipo ax2 = b.

A: Incorreta, pois o aluno pode chegar a essa conclusão considerando que
o enunciado pede apenas 1 valor das raízes da equação descrita.

B: Incorreta, pois o aluno pode chegar a essa conclusão considerando que
o enunciado pede apenas o valor antes de extrairmos as raízes da equação
descrita.

C: Incorreta, pois o aluno pode chegar a essa conclusão esquecendo de
dividir o valor de uma das raízes por 4.

D: Correta, pois, utilizando Bhaskhara, temos:

4x^2 + 9x = 0

A = 4

B = 9

C = 0

(\frac{- 9 \pm \sqrt{9^{2} - 4.4.0}}{2.4})

(\frac{- 9 \pm \sqrt{81}}{8})

(\frac{- 9 \pm 9}{8})

X1 = (\frac{- 9 + 9}{8}) = 0

X2= (\frac{- 9 - 9}{8}) = (\frac{- 18}{8}) = - (\frac{9}{4}) =
-2,25 segundos
\item SAEB: Resolver problemas que possam ser representados por equações
polinomiais de 2º grau.

BNCC: EF08MA09 -- Resolver e elaborar, com e sem uso de tecnologias,
problemas que possam ser representados por equações polinomiais de 2º
grau do tipo ax2 = b.

A: Incorreta, pois o aluno pode chegar a esse valor ao não realizar a
radiciação necessária.

B: Incorreta, pois o aluno pode chegar a essa conclusão considerando que
o enunciado pede apenas 1 valor das raízes da equação descrita.

C: Correta, pois, utilizando Bhaskhara, temos:

6x^2 - 5x = 0

A = 6

B = -5

C = 0

(\frac{- ( - 5) \pm \sqrt{{( - 5)}^{2} - 4.6.0}}{2( - 5)})

(\frac{5 \pm \sqrt{25}}{12})

(\frac{5 \pm 5}{12})

X1 = (\frac{5 + 5}{12}) = (\frac{10}{12}) = (\frac{5}{6})

X2 = (\frac{5 - 5}{12}) = 0

Logo, temos que a parte preenchida do tanque foi de (\frac{5}{6}).

D: Incorreta, pois o aluno chegaria a esse valor apenas somando os
termos da equação e não realizando a operação por completo.
\end{enumerate}


\section*{Matemática – Módulo 8 – Treino}
\begin{numerate}
\item SAEB: Resolver problemas que envolvam variação de proporcionalidade
direta ou inversa entre duas ou mais grandezas, inclusive escalas,
divisões proporcionais e taxa de variação.

BNCC: EF08MA13 -- Resolver e elaborar problemas que envolvam grandezas
diretamente ou inversamente proporcionais, por meio de estratégias
variadas.

A: Correta, pois, utilizando a regra de 3 simples, temos que:

(\frac{3,5}{4,2} = \frac{135}{x})

3,5 . x = 567

x = 162 minutos ou 2 horas e 42 minutos.

B: Incorreta, o aluno poderia chegar a esse valor realizando a
multiplicação reta na regra de 3 e não a multiplicação cruzada.

C: Incorreta, pois esse seria o valor caso o aluno não convertesse horas
em minutos como forma de solução.

D: Incorreta, pois o aluno chegaria a esse resultado caso não realizasse
a última operação necessária, que é a divisão.
\item SAEB: Resolver problemas que envolvam variação de proporcionalidade
direta ou inversa entre duas ou mais grandezas, inclusive escalas,
divisões proporcionais e taxa de variação.

BNCC: EF08MA13 -- Resolver e elaborar problemas que envolvam grandezas
diretamente ou inversamente proporcionais, por meio de estratégias
variadas.

A: Incorreta, pois o aluno chegaria nesse resultado multiplicando reto a
regra de três ao invés de multiplicar cruzado.

B: Correta, pois, utilizando a Regra de 3 simples, temos que:

(\frac{5}{3} = \frac{64}{x})

5x = 64 . 3

5x = 192

X = 38,4

C: Incorreta, pois o aluno chegaria a esse valor caso, no final da
expressão, ao invés de realizar uma divisão, realizasse uma
multiplicação.

D: Incorreta, pois o aluno chegaria a essa conclusão considerando cada
rosquinha com 64 calorias.
\item SAEB: Resolver problemas que envolvam variação de proporcionalidade
direta ou inversa entre duas ou mais grandezas, inclusive escalas,
divisões proporcionais e taxa de variação.

BNCC: EF08MA13 -- Resolver e elaborar problemas que envolvam grandezas
diretamente ou inversamente proporcionais, por meio de estratégias
variadas.

A: Incorreta, pois o aluno chegaria a esse resultado realizando uma
multiplicação reta ao invés de uma multiplicação cruzada.

B: Correta, pois, utilizando a regra de 3 simples, temos que:

(\frac{10}{27} = \frac{9}{x})

10.x = 9 . 27

10x = 243

X = 24,3 ml

C: Incorreta, pois o aluno chegaria a essa conclusão se, ao final da
expressão, realizasse uma multiplicação.

D: Incorreta, pois o aluno chegaria a essa conclusão deslocando a
virgula uma casa para esquerda durante o cálculo final.
\end{enumerate}


\section*{Matemática – Módulo 9 – Treino}
\begin{numerate}
\item SAEB: Construir/desenhar figuras geométricas planas ou espaciais que
satisfaçam condições dadas.

BNCC: EF08MA18 -- Reconhecer e construir figuras obtidas por composições
de transformações geométricas (translação, reflexão e rotação), com o
uso de instrumentos de desenho ou de softwares de geometria dinâmica.

A: Correta, pois 117 : 9 = 13 cm de lado tem essa figura.

B: Incorreta, pois o aluno poderia chegar a essa conclusão caso
considerasse que o eneágono regular possui 10 lados e não 9.

c: Incorreta, pois o aluno poderia chegar a essa conclusão caso
considerasse que o eneágono regular possui 7 lados e não 9.

D: Incorreta, pois o aluno poderia chegar a essa conclusão caso
considerasse que o eneágono regular possui 5 lados e não 9.
\item SAEB: Construir/desenhar figuras geométricas planas ou espaciais que
satisfaçam condições dadas.

BNCC: EF08MA18 -- Reconhecer e construir figuras obtidas por composições
de transformações geométricas (translação, reflexão e rotação), com o
uso de instrumentos de desenho ou de softwares de geometria dinâmica.

A: Incorreta, poiso aluno chegaria a esse valor caso confundisse a
fórmula da área do círculo com a fórmula do perímetro do círculo.

B: Incorreta, pois o aluno chegaria a essa conclusão caso se esquecesse
do termo quadrático da expressão.

C: Correta, pois

(A = \pi r^{2})

A = 3 . 9,15^2

A= 3. 83,7225

A= 251 m^2

D: Incorreta, pois o aluno chegaria a esse resultado caso dividisse a
expressão no final da fórmula ao invés de multiplicar.
\item SAEB: Construir/desenhar figuras geométricas planas ou espaciais que
satisfaçam condições dadas.

BNCC: EF08MA18 -- Reconhecer e construir figuras obtidas por composições
de transformações geométricas (translação, reflexão e rotação), com o
uso de instrumentos de desenho ou de softwares de geometria dinâmica.

A: Correta, pois:

Hexágono regular = 6 lados

30 cm por porta joias

20 . 30 cm = 600 cm de fita ou 6 metros de fita

B: Incorreta, pois o aluno pode chegar a esse valor confundindo cm com
metros.

C: Incorreta, pois o aluno pode chegar a esse valor considerando que um
hexágono regular contenha 5 lados.

D: Incorreta, pois o aluno pode chegar a esse valor considerando que um
hexágono regular contenha 4 lados.
\end{enumerate}


\section*{Matemática – Módulo 10 – Treino}
\begin{numerate}
\item SAEB: Resolver problemas que envolvam relações entre ângulos formados
por retas paralelas cortadas por uma transversal, ângulos internos ou
externos de polígonos ou cevianas (altura, bissetriz, mediana,
mediatriz) de polígonos.

BNCC: EF08MA14 -- Demonstrar propriedades de quadriláteros por meio da
identificação da congruência de triângulos.

A: Incorreta, pois essa definição de incentro está errada.

B: Incorreta, pois essa definição de incentro está errada.

C: Incorreta, pois essa definição de incentro está errada.

D: Correta, pois as bissetrizes do triângulo correspondem ao incentro.
\item SAEB: Resolver problemas que envolvam relações entre ângulos formados
por retas paralelas cortadas por uma transversal, ângulos internos ou
externos de polígonos ou cevianas (altura, bissetriz, mediana,
mediatriz) de polígonos.

BNCC: EF08MA14 -- Demonstrar propriedades de quadriláteros por meio da
identificação da congruência de triângulos.

A: Incorreta, pois esse seria o valor relativo, e não a medida final.

B: Incorreta, pois houve um erro na multiplicação.

C: Correta, pois, se a altura relativa à hipotenusa BC mede 9,3~cm, a
medida da hipotenusa será 18,6.

D: Incorreta, pois o aluno chegará a essa conclusão ao errar o cálculo
de multiplicação dos termos destacados no enunciado.
\item SAEB: Identificar propriedades e relações existentes entre os elementos
de um triângulo (condição de existência, relações de ordem entre as
medidas dos lados e as medidas dos ângulos internos, soma dos ângulos
internos, determinação da medida de um ângulo interno ou externo)

BNCC: EF08MA14 -- Demonstrar propriedades de quadriláteros por meio da
identificação da congruência de triângulos.

A: Incorreta, pois o aluno chegará a essa conclusão se, durante o
cálculo da soma dos lados do triangulo, encontrar 2 números a menos.

B: Incorreta, pois o aluno chegará a essa conclusão se, durante o
cálculo da soma dos lados do triangulo, encontrar 1 número a menos.

C: Correta, pois:

Perímetro = soma dos lados, logo 6 + 7 + 8 = 21 cm

D: Incorreta, pois o aluno chegará a essa conclusão se, durante o
cálculo da soma dos lados do triangulo, encontrar 1 número a mais.
\end{enumerate}


\section*{Matemática – Módulo 11 – Treino}
\begin{numerate}
\item Habilidade SAEB: Descrever ou esboçar deslocamento de pessoas e/ou de
objetos em representações bidimensionais (mapas, croquis etc.), plantas
de ambientes ou vistas, de acordo com condições dadas.

A: Incorreta, pois, ao considerar que o enunciado pede o valor de km por
partida, o aluno pode chegar a essa conclusão.

B: Incorreta, pois ao considerar a quantidade de rodadas ao invés da
quantidade de km percorridos, o aluno pode chegar a esse valor.

C: Incorreta, pois, ao deslocar erroneamente uma vírgula para a
esquerda, o aluno pode chegar a esse resultado.

D: Correta, pois

8 km x 38 partidas = 304 km por campeonato.
\item SAEB: Descrever ou esboçar deslocamento de pessoas e/ou de objetos em
representações bidimensionais (mapas, croquis etc.), plantas de
ambientes ou vistas, de acordo com condições dadas.

A: Incorreta, pois esse seria o valor caso o aluno realizasse
incorretamente a multiplicação.

B: Incorreta, pois esse seria o valor caso o aluno realizasse
incorretamente a multiplicação, adicionando um ``zero'' na expressão.

C: Correta pois:

(\frac{50}{x} = \frac{1}{8})

50 . 8 = x

400 km.

D: Incorreta, pois esse seria o valor caso o aluno realizasse
incorretamente a multiplicação, adicionando dois ``zeros'' na expressão.
\item SAEB: Descrever ou esboçar deslocamento de pessoas e/ou de objetos em
representações bidimensionais (mapas, croquis etc.), plantas de
ambientes ou vistas, de acordo com condições dadas.

A: Incorreta, pois o aluno pode se confundir em relação aos pontos
cardeais.

B: Incorreta, pois o aluno pode se confundir em relação aos pontos
cardeais.

C: Incorreta, pois o aluno pode se confundir em relação aos pontos
cardeais.

D: Correta, pois Charles deve seguir rumo ao Oeste.
\end{enumerate}


\section*{Matemática – Módulo 12 – Treino}
\begin{numerate}
\item SAEB: Calcular os valores de medidas de tendência central de uma
pesquisa estatística (média aritmética simples, moda ou mediana).

BNCC: EF08MA25 -- Obter os valores de medidas de tendência central de
uma pesquisa estatística (média, moda e mediana) com a compreensão de
seus significados e relacioná-los com a dispersão de dados, indicada
pela amplitude.

A: Incorreta, pois o aluno chegaria a esse resultado a partir da maior
altura da equipe.

B: Incorreta, pois o aluno chegaria a esse resultado a partir da menor
altura da equipe.

C: Correta, pois:

Somando a altura dos atletas temos:

2,01 + 1,99 + 2,00 + 2,02 + 1,98 =10

Como são 5 atletas

10 : 5 = 2 metros de altura é a média

D: Incorreta, pois o aluno chegaria a esse resultado a partir da soma
das alturas da equipe.
\item SAEB: Calcular os valores de medidas de tendência central de uma
pesquisa estatística (média aritmética simples, moda ou mediana).

BNCC: EF08MA25 -- Obter os valores de medidas de tendência central de
uma pesquisa estatística (média, moda e mediana) com a compreensão de
seus significados e relacioná-los com a dispersão de dados, indicada
pela amplitude.

A: Incorreta, pois o aluno chegaria a esse resultado calculando a moda,
chegando a uma conclusão equivocada.

B: Correta, pois:

Considerando que os dois pesos centrais são 46 e 45 kg,

46 + 45 = 91

91 : 2 = 45,5

C: Incorreta, pois o aluno chegaria a esse resultado calculando a média
aritmética, e não a mediana.

D: Incorreta, pois o aluno chegará a esse resultado não levando em
consideração que, em casos de conteúdos pares, a mediana deve ser a
média entre os dois valores centrais.
\item SAEB: Calcular os valores de medidas de tendência central de uma
pesquisa estatística (média aritmética simples, moda ou mediana).

BNCC: EF08MA25 -- Obter os valores de medidas de tendência central de
uma pesquisa estatística (média, moda e mediana) com a compreensão de
seus significados e relacioná-los com a dispersão de dados, indicada
pela amplitude.

A: Incorreta, pois o aluno chegaria a esse resultado considerando
somente o menor tempo gasto.

B: Incorreta, pois o aluno chegaria a esse resultado considerando
somente o maior tempo gasto.

C: Correta, pois:

Somando o tempo que Maria gastou na semana, temos que:

170 minutos / 5 = média de 34 minutos por dia.

D: Incorreta, pois o aluno chegaria a esse resultado considerando
somente a soma dos tempos.
\end{enumerate}


\section*{Matemática – Módulo 13 – Treino}
\begin{numerate}
\item SAEB: Resolver problemas que envolvam volume de prismas retos ou
cilindros retos.

BNCC: EF08MA20 -- Reconhecer a relação entre um litro e um decímetro
cúbico e a relação entre litro e metro cúbico, para resolver problemas
de cálculo de capacidade de recipientes.

A: Correta, pois:

V = (\Pi) . R^2 .h

V = 3 . 6^2 . 70

V = 3 . 36 . 70

V = 7.560 cm^3

B: Incorreta, pois o aluno chegaria a esse valor utilizando a formula da
área, e não a formula do volume como o enunciado pede.

C: Incorreta, pois o aluno chegaria a esse valor calculando o perímetro
da circunferência do cano, e não o volume como pede o enunciado.

D: Incorreta, pois o aluno chegaria a esse valor ao esquecer o termo
quadrático.
\item EB: Resolver problemas que envolvam volume de prismas retos ou
cilindros retos.

BNCC: EF08MA19 -- Resolver e elaborar problemas que envolvam medidas de
área de figuras geométricas, utilizando expressões de cálculo de área
(quadriláteros, triângulos e círculos), em situações como determinar
medida de terrenos.

A: Correta, pois:

Utilizando a fórmula da área do losango, temos que:

A = (\frac{\text{D\ .\ d}}{2})=

A = (\frac{70\ .\ 50}{2})=

A = (\frac{3500}{2})

A = 1.750 cm^2

Calculando a área do retângulo, temos que:

A = 100 . 70

A = 7.000 cm^2

Subtraindo

7.000 - 1.750 = 5.250 cm^2

B: Incorreta, pois este valor é referente apenas ao valor da área do
retângulo do quadro.

C: Incorreta, pois este valor é referente apenas à área que já foi
pintada.

D: Incorreta, pois o aluno chegaria nesse valor ao não converter o valor
em metros para centímetros.
\item SAEB: Resolver problemas que envolvam volume de prismas retos ou
cilindros retos.

BNCC: EF08MA20 -- Reconhecer a relação entre um litro e um decímetro
cúbico e a relação entre litro e metro cúbico, para resolver problemas
de cálculo de capacidade de recipientes.

A: Incorreta, pois o aluno poderia chegar a esse valor utilizando
erroneamente a formula da área da base.

B: Incorreta, pois o aluno poderia chegar a esse valor calculando o
perímetro da base.

C: Incorreta, pois o aluno poderia chegar a essa conclusão ao não
observar o termo quadrático da fórmula.

D: Correta, pois

V= \Pi). R^2 .h

V= 3,1 . 4^2 . 5

V= 248 m^3
\end{enumerate}


\section*{Matemática – Módulo 14 – Treino}
\begin{numerate}
\item SAEB: Resolver problemas que envolvam a probabilidade de ocorrência de
um resultado em eventos aleatórios equiprováveis independentes ou
dependentes

BNCC: EF08MA22 -- Calcular a probabilidade de eventos, com base na
construção do espaço amostral, utilizando o princípio multiplicativo, e
reconhecer que a soma das probabilidades de todos os elementos do espaço
amostral é igual a 1.

A: Incorreta, pois, ao converter o valor em porcentagem erroneamente, o
aluno chegaria a esse valor.

B: Incorreta, pois esse valor seria o número de calçados a serem
selecionados, e não a porcentagem final.

C: Correta, pois:

(P(E)\frac{n(4)}{n(100)}) = 0,04 ou 4\%

D: Incorreta, pois o aluno poderia chegar a essa conclusão ao apenas
dividir o número de calçados totais pelo número de pares.
\item SAEB: Resolver problemas que envolvam a probabilidade de ocorrência de
um resultado em eventos aleatórios equiprováveis independentes ou
dependentes

BNCC: EF08MA22 -- Calcular a probabilidade de eventos, com base na
construção do espaço amostral, utilizando o princípio multiplicativo, e
reconhecer que a soma das probabilidades de todos os elementos do espaço
amostral é igual a 1.

A: Incorreta, pois o aluno pode realizar uma soma ao invés de uma
multiplicação.

b: Incorreta, pois o aluno chegará a esse valor caso esqueça do elemento
``massas''.

c: Incorreta, pois o aluno chegará a esse valor caso esqueça do elemento
``saladas''.

D: Correta, pois:

Relendo o enunciado, temos que:

4 . 6 . 5 = 300
\item SAEB: Resolver problemas que envolvam a probabilidade de ocorrência de
um resultado em eventos aleatórios equiprováveis independentes ou
dependentes

BNCC: EF08MA22 -- Calcular a probabilidade de eventos, com base na
construção do espaço amostral, utilizando o princípio multiplicativo, e
reconhecer que a soma das probabilidades de todos os elementos do espaço
amostral é igual a 1.

A: Correta, pois:

(P(E)\frac{n(1)}{n(26)}) = 0,03 ou aproximadamente 3\%

B: Incorreta, pois o aluno chegaria a essa conclusão ao confundir a
quantidade de letras do alfabeto com a probabilidade do fato acontecer.

C: Incorreta, pois o aluno chegaria a essa conclusão ao confundir a
quantidade de iniciais com a probabilidade do fato acontecer.

D: Incorreta, pois, ao deslocar a virgula uma casa para a direita, o
aluno chegaria a essa resposta.
\end{enumerate}


\section*{Matemática – Simulado 1}
\begin{numerate}
\item SAEB: Identificar números racionais ou irracionais. BNCC: F09MA02 --
Reconhecer um número irracional como um número real cuja representação
decimal é infinita e não periódica, e estimar a localização de alguns
deles na reta numérica.

A: Correta, pois se trata de um número irracional.

B: Incorreta, pois não se trata de um número racional.

C: Incorreta, pois não se trata de um número inteiro.

D: Incorreta, pois não se trata de um número natural.
\item SAEB: Resolver problemas de adição, subtração, multiplicação, divisão,
potenciação ou radiciação envolvendo número reais, inclusive notação
científica.

BNCC: EF08MA01 -- Efetuar cálculos com potências de expoentes inteiros e
aplicar esse conhecimento na representação de números em notação
científica.

A: Incorreta, pois, ao contar um ``zero'' a mais, o aluno chegaria a
esse resultado.

B: Correta, pois, utilizando a notação científica, temos que:

0,000000000000000000000000000911,

30 casas após a vírgula; logo, é necessário o deslocamento do primeiro
número após o 0:

(0,000000000000000000000000000911 = 9,11 . 10^{-28})

C: Incorreta, pois, ao contar um ``zero'' a menos, o aluno chegaria a
esse resultado.

D: Incorreta, pois, ao contar dois ``zeros'' a menos, o aluno chegaria a
esse resultado.
\item SAEB: Representar frações menores ou maiores que a unidade por meio de
representações pictóricas ou associar frações a representações
pictóricas.

A: Incorreta, pois o aluno pode chegar à conclusão de que 100 - 99 = 1.

B: Correta, pois, realizando o Cálculo, temos que
( \frac{50.000} {100} = 500)

500 . 99 = 49.500.

C: Incorreta, pois o aluno pode considerar retirar 99 do valor de 50.000
e chegar ao resultado da alternativa descrita.

D: Incorreta, pois o aluno pode chegar a esse resultado realizando a
multiplicação por 0,099.
\item SAEB: Resolver problemas que envolvam porcentagens, incluindo os que
lidam com acréscimos e decréscimos simples, aplicação de percentuais
sucessivos e determinação de taxas percentuais.

BNCC: EF08MA04 -- Resolver e elaborar problemas, envolvendo cálculo de
porcentagens, incluindo o uso de tecnologias digitais.

A: Incorreta, pois esse seria o acréscimo do valor da passagem, e não o
valor final.

B: Incorreta, pois esse seria o valor da passagem caso ocorresse 12\% de
desconto.

C:Incorreta, pois o aluno pchegaria a esse valor caso somasse os dois
números.

D: Correta, pois

(\frac{4,25}{x} \times \frac{100}{112})

4,25 . 112 = x . 100

476= 100x

X = 4,76
\item AEB: Resolver problemas que possam ser representados por sistema de
equações de 1º grau com duas incógnitas.

A: Incorreta, pois o aluno pode considerar o valor parcial como o valor
final do número de moedas.

B: pois o aluno pode considerar o valor parcial como o valor final do
número de moedas.

C: Incorreta, pois o aluno pode realizar a multipicação ao invés da
soma.

D: Correta, pois, lendo o enunciado, obtemos o seguinte sistema de
equações:

0,25x + 10y = 15,60

X = 2y

Logo, já conseguimos substituir a 2ª. equação na primeira

0,25 . 2y + 0,10y = 15,60

0,50y + 0,10y = 15,60

0,60y = 15,60

Y = 26

Como x é o dobro de y, temos: x = 2y = 2 . 26 = 52
\item SAEB: Resolver problemas que envolvam volume de prismas retos ou
cilindros retos.

BNCC: EF08MA21 -- Resolver e elaborar problemas que envolvam o cálculo
do volume de recipiente cujo formato é o de um bloco retangular.

A: incorreta o aluno poderia chegar a essa conclusão chegando apenas ao
valor do volume apenas da primeira caixa.

B: incorreta o aluno poderia chegar a essa conclusão chegando apenas ao
valor do volume apenas da segunda caixa.

C: o aluno chegaria a esse valor calculando incorretamente o ultimo
termo da equação esquecendo de realizar a última parte valor numérico

D: Correta, pois, considerando a fórmula do cálculo do volume, temos
que:

1ª. caixa

(x+2) . (x+2) . (x+2) =

(X^2 + 2x + 2x + 4) . (x + 2) =

(x^2 + 4x + 4) . (x + 2) =

X^3 + 2x^2 + 4x^2 + 8x + 4x + 8 =

X^3 + 6x^2 + 12x + 8

2ª. caixa

(x + 2) . (3x + 2) . (7x + 2)=

(3x^2 +2x + 6x + 4) . (7x + 2)=

(3x^2 + 8x + 4 ) . (7x + 2)=

21x^3 + 6x^2 + 56x^2 + 16x + 28x + 8=

21x^3 + 62x^2 + 44x + 8

Somando o valor das 2 caixas, temos que

X^3 + 6x^2 + 12x +8 + 21x^3 + 62x^2 + 44x + 8 =

22x^3 + 68x^2 + 56x + 16
\item SAEB: Resolver problemas que possam ser representados por equações
polinomiais de 2º grau.

BNCC: EF08MA09 -- Resolver e elaborar, com e sem uso de tecnologias,
problemas que possam ser representados por equações polinomiais de 2º
grau do tipo ax2 = b.

a: Incorreta, pois esse valor representa apenas um dos termos da
equação.

b: Correta, pois, realizando a operação, obtemos:

t^2 - 36 = 0

t^2 = (\sqrt{36})

t = ± 6

C: Incorreta, pois o aluno pode chegar a esse valor somando todos os
termos da equação.

D: Incorreta, pois o aluno pode apenas retirar o termo quadrático e
cogitar que essa possa ser a alternativa correta.
\item AEB: Resolver problemas que envolvam variação de proporcionalidade
direta ou inversa entre duas ou mais grandezas, inclusive escalas,
divisões proporcionais e taxa de variação.

BNCC: EF08MA12 -- Identificar a natureza da variação de duas grandezas,
diretamente, inversamente proporcionais ou não proporcionais,
expressando a relação existente por meio de sentença algébrica e
representá-la no plano cartesiano.

A: Correta, pois, utilizando a razão (\frac{distância}{\text{tempo}}),
temos que (\frac{455}{7}) = 65km/h

A velocidade média desse carro foi de 65 km/h.

B: Incorreta, pois o aluno pode chegar a essa conclusão caso divida a
distância pelo valor de 60 minutos ao invés de 7 horas.

C: Incorreta, pois o aluno chegará a essa conclusão confundindo km/h por
km/m.

D: Incorreta, pois o aluno chegará a essa conclusão confundindo km/h por
m/s.
\item SAEB: Relacionar o número de vértices, faces ou arestas de prismas ou
pirâmides, em função do seu polígono da base.

BNCC: EF08MA18 -- Reconhecer e construir figuras obtidas por composições
de transformações geométricas (translação, reflexão e rotação), com o
uso de instrumentos de desenho ou de softwares de geometria dinâmica.

A: Incorreta, pois o aluno chegaria a essa conclusão realizando apenas a
primeira parte do cálculo.

B: Correta, pois, utilizando a fórmula para obter o valor da figura,
temos que:

(\frac{Ai = \left( 8 - 2 \right)\ \ .\ \ 180}{8}) =

(\frac{Ai = \ 6\ \ .\ \ 180}{8\ }) =

(\frac{Ai = 1080}{8}) = 135°

Calculando os ângulos do triângulo:

(\frac{Ai = \left( 3 - 2 \right)\ \ .\ \ 180}{3}) =

(\frac{Ai = \ \ \ 180}{3}) = 60°

C: Incorreta, pois o aluno pode considerar o valor do ângulo interno do
triângulo como resposta, o que é incorreto.

D: Incorreta, pois o aluno chegaria a esse valor caso não subtraísse
também o valor do ângulo do triângulo.
\item SAEB: Resolver problemas que envolvam relações entre ângulos formados
por retas paralelas cortadas por uma transversal, ângulos internos ou
externos de polígonos ou cevianas (altura, bissetriz, mediana,
mediatriz) de polígonos.

BNCC: EF08MA14 -- Demonstrar propriedades de quadriláteros por meio da
identificação da congruência de triângulos.

A: Incorreta, pois esse valor é referente ao menor ângulo do triângulo.

B: Incorreta, pois esse valor é referente ao segundo maior ângulo do
triângulo.

C: Correta, pois

3x + 4x + 15 + 6x - 30 = 180

13x - 15 = 180

13x = 195

X = 15

Fazendo a substituição, temos que os ângulos são 45º, 75º e 60º.

D: Incorreta, pois esse valor é referente ao total da soma dos ângulos
do triângulo.
\item SAEB: Descrever ou esboçar deslocamento de pessoas e/ou de objetos em
representações bidimensionais (mapas, croquis etc.), plantas de
ambientes ou vistas, de acordo com condições dadas.

A: Correta, pois

3,337 x 78 voltas = 260,286 km

B: Incorreta, pois o aluno realizou a soma dos valores ao invés de
multiplicá-los.

C: Incorreta, pois o aluno realizou a subtração dos valores ao invés de
multiplicá-los.

D: Incorreta, pois pois o aluno realizou a divisão dos valores ao invés
de multiplicá-los.
\item SAEB: Calcular os valores de medidas de tendência central de uma
pesquisa estatística (média aritmética simples, moda ou mediana).

A: Incorreta, pois esse valor é referente apenas ao 1° dia de Geraldo.

B: Correta, pois, somando as palavras digitadas durante os dias, temos:

125.000 + 112.000 + 175.000 + 140.000 + 101.000 =

653.000 : 6 = 130.600 palavras em média são digitadas por dia.

C: Incorreta, pois esse valor é referente apenas ao 2° dia de Geraldo.

D: Incorreta, pois esse valor é referente apenas ao 3° dia de Geraldo.
\item SAEB: Resolver problemas que envolvam medidas de grandezas (comprimento,
massa, tempo, temperatura, capacidade ou volume) em que haja conversões
entre unidades mais usuais.

BNCC: EF08MA19 -- Resolver e elaborar problemas que envolvam medidas de
área de figuras geométricas, utilizando expressões de cálculo de área
(quadriláteros, triângulos e círculos), em situações como determinar
medida de terrenos.

A: Incorreta, pois esse valor seria correspondente ao valor do volume e
não à quantidade de tempo.

B: Incorreta, pois, ao converter erroneamente o valor de minutos para
horas, o aluno chegaria a essa conclusão.

C: Correta, pois, para calcular o volume de um bloco retangular, temos
que V = l.l.l

Logo, substituindo :

V= 2.2.1

V=4 m^3

4m^3 = 4.000 litros

20 litros -\/-\/-\/-\/-\/-\/-\/-\/-\/-\/-\/-\/- 1 minuto

4000 litros -\/-\/-\/-\/-\/-\/-\/-\/-\/-\/-\/-\/- x minutos

(\frac{20 \; litros}{4000 \;} = \frac{1 \; minuto}{x \; minutos})

20 . x = 4000

X = 200 minutos

200 minutos = 3 horas e 20 minutos.

D: Incorreta, pois o aluno chegaria a essa conclusão caso inserisse um
``zero'' a menos na expressão.
\item SAEB: Resolver problemas que envolvam a probabilidade de ocorrência de
um resultado em eventos aleatórios equiprováveis independentes ou
dependentes.

BNCC: EF08MA22 -- Calcular a probabilidade de eventos, com base na
construção do espaço amostral, utilizando o princípio multiplicativo, e
reconhecer que a soma das probabilidades de todos os elementos do espaço
amostral é igual a 1.

A: Correta, pois:

(P(E)\frac{n(1)}{n(40)})= 2,5\%

B: Incorreta, pois o aluno poderia considerar o número de tentativas ao
invés da probabilidade.

C: Incorreta, pois o aluno chegaria a esse valor deslocando
incorretamente a virgula para a esquerda.

D: Incorreta, pois o aluno chegaria a essa conclusão ao considerar o
número de funcionários como a quantidade de probabilidades.
\item SAEB: Resolver uma equação polinomial de 2º grau.

BNCC: EF08MA09 -- Resolver e elaborar, com e sem uso de tecnologias,
problemas que possam ser representados por equações polinomiais de 2º
grau do tipo ax2 = b.

A: Correta, pois, para resolver uma equação polinomial de 2º grau,
podemos utilizar a fórmula quadrática, que é dada por
(x = \frac{(-b ± √(b^2 - 4ac)} {(2a)}), onde a, b e c são os
coeficientes da equação ax^2 + bx + c = 0.

No caso da equação x^2 - 4x + 3 = 0, temos a = 1, b = -4 e c = 3.

B: Incorreta, pois essas soluções não estão de acordo com os cálculos
realizados.

C: Correta, pois essas soluções não estão de acordo com os cálculos
realizados.

D: Incorreta, pois essas soluções não estão de acordo com os cálculos
realizados.
\item SAEB: Identificar uma representação algébrica para o padrão ou a
regularidade de uma sequência de números racionais ou representar
algebricamente o padrão ou a regularidade de uma sequência de números
racionais.

A: Incorreta, pois usou a porcentagem na forma percentual.

B: Incorreta, pois apesar de transformar 3\% para (\frac{3}{100}),
colocou o 100 como denominador da parte fixa do salário, transformando
ela para 25 ao invés de 2500.

C: Incorreta, pois multiplicou a comissão pelo fixo ao invés de somar.

D: Correta, pois

(\text{Fixo}\ = \ 2.500 \; {Vari}á\text{vel}\  = \ 3\%\ \text{de}\ x \; {Sal}á\text{rio} = 2.500 + 0,03x).
\end{enumerate}


\section*{Matemática – Simulado 2}
\begin{numerate}
\item SAEB: Resolver problemas de contagem cuja resolução envolva a aplicação
do princípio multiplicativo.

BNCC: EF08MA03 -- Resolver e elaborar problemas de contagem cuja
resolução envolva a aplicação do princípio multiplicativo.

A: Incorreta, pois o aluno pode considerar que a palavra quadrado remete
ao número dois, por semelhança.

B: Incorreta, pois o aluno pode chegar à conclusão de que 5^3 equivale a
125, formando um quadrado perfeito.

C: Incorreta, pois o aluno pode se confundir a partir do número de lados
do quadrado.

D: Correta, pois 5 x 125 = 625 e (\sqrt{625}) = 25, logo um quadrado
perfeito.
\item SAEB: Resolver problemas de adição, subtração, multiplicação, divisão,
potenciação ou radiciação envolvendo número reais, inclusive notação
científica.

BNCC: EF08MA01 -- Efetuar cálculos com potências de expoentes inteiros e
aplicar esse conhecimento na representação de números em notação
científica.

A: Correta, pois, considerando a notação científica, temos que

(2,0 . 10^{-5} . 2,0 . 10^{-5} =)

Separando os termos

2,0 . 2,0 = 4,0

(10^{-5} . 10^{-5} = 10^{-10})

Logo temos que a área do Nano chip em cm^2 é (4,0 . 10^{-10})

B: Incorreta, pois, ao errar o jogo de sinal na expressão e não realizar
a multiplicação de 2,0 por 2,0, o aluno chegaria a esse resultado.

C: Incorreta, pois, ao errar o jogo de sinal na expressão, o aluno
chegaria a esse resultado.

D: Incorreta, pois, ao errar o jogo de sinal na expressão, o aluno
chegaria a esse resultado.
\item SAEB: Determinar uma fração geratriz para uma dízima periódica.

A: Incorreta, pois este problema tem 2 soluções possíveis. A primeira.
realizando um divisão entre os dois termos, e a segunda encontrando a
fração geratriz do problema.

B: Correta, pois temos (\frac{84}{90}). Simplificando por 6, temos que
(\frac{14}{15}). A idade de Larissa e Leila são respectivamente 14 e
15 anos.

C: Incorreta, pois este problema tem 2 soluções possíveis. A primeira.
realizando um divisão entre os dois termos, e a segunda encontrando a
fração geratriz do problema.

D: Incorreta, pois este problema tem 2 soluções possíveis. A primeira.
realizando um divisão entre os dois termos, e a segunda encontrando a
fração geratriz do problema.
\item SAEB: Resolver problemas que envolvam porcentagens, incluindo os que
lidam com acréscimos e decréscimos simples, aplicação de percentuais
sucessivos e determinação de taxas percentuais.

BNCC: EF08MA04 -- Resolver e elaborar problemas, envolvendo cálculo de
porcentagens, incluindo o uso de tecnologias digitais.

A: Incorreta, pois o aluno chegará a essa conclusão caso não realize os
cálculos necessários.

B: Correta, pois

Valor inicial = x. Ao darmos o primeiro desconto temos que x .
(\frac{1}{10}) = (\frac{x}{10})

Ao acrescermos 10\% ao valor dado, temos que (\frac{x}{10\ }) .
(\frac{9}{10}) = (\frac{9x}{100}) ou 0,09 ou 9\%, logo, o valor
final diminuiu 1\%.

C: Incorreta, pois o aluno pode chegar a essa conclusão devido a
semelhança entre os termos da resposta correta.

D: Incorreta, pois o aluno chegaria a essa conclusão caso confundisse o
valor final do produto com a porcentagem correspondente.
\item SAEB: Resolver uma equação polinomial de 1º grau.

A: Correta, pois

Determinando com x = bola e y = boneca, temos que

10x + 15y = 800

X + y = 60

Isolando o x na segunda equação, temos que:

X = 60 - y

Substituindo a segunda equação na primeira, temos:

10(60 - y) + 15y = 800

600 - 10y + 15y = 800

5y = 200

y = 40

Logo, se o preço de uma boneca é igual a 40 reais e o preço de uma
boneca e uma bola custam juntos 60 reais, 60 - 40 = 20. O preço da bola
é 20 reais.

B: Incorreta, pois o aluno chegaria a essa conclusão caso errasse a
troca de sinal da equação resultante do sistema.

C: Incorreta, pois o aluno pode confundir os resultados entre o preço da
bola e o da boneca.

D: Incorreta, pois o aluno pode não conseguir deduzir o preço de um
brinquedo por meio do valor do outro, chegando a essa conclusão
precipitada.
\item SAEB: Identificar representações algébricas equivalentes.

A: Incorreta, pois o aluno poderia, ao invés de realizar o cálculo da
área, realizar o cálculo do perímetro.

B: Incorreta, pois o aluno errou a multiplicação no termo ``x''.

C: Correta, pois, considerando que cada lado dos quadrados tem (x+6), e
que a área do quadrado é calculada pela fórmula l^2, temos que:

(x+6) . (x+6) =

X^2 + 12x + 36

Como temos 4 quadrados na figura:

4 . (X^2 + 12x+ 36) =

4x^2 + 48x + 144.

D: Incorreta, pois, caso o aluno considere calcular apenas 1 quadrado ao
invés da figura toda, chegará a esse resultado erroneamente.
\item SAEB: Resolver uma equação polinomial de 2º grau.

A: Incorreta, pois o aluno chegaria a essa conclusão ao considerar
apenas uma parte da equação correspondente.

B: Incorreta, pois o aluno chegaria a essa conclusão ao somar uma parte
da equação correspondente, e não o resultado final.

C: Correta, pois:

Realizando as operações, obtemos:

X^2 - 2 809 = 0

X^2 = 2 809

X= (\sqrt{2\ 809})

X = 53

D: Incorreta, pois o aluno chegaria a essa conclusão ao somar os
elementos da equação correspondente, e não o resultado final.
\item SAEB: Calcular o resultado de adições, subtrações, multiplicações ou
divisões envolvendo número reais.

BNCC: EF08MA03 -- Resolver e elaborar problemas de contagem cuja
resolução envolva a aplicação do princípio multiplicativo.

A: Correta, pois, utilizando a razão da densidade demográfica -

(densidade\ demografica = \frac{número\ de\ pessoas\ }{dimensão\ do\ espaço\ em\ km^2})
-, temos que:

(densidade\ demográfica = \frac{7\ 264\ 200}{35\ 000})

(densidade\ demográfica = 207,5\ habitantes)/km^2

B: Incorreta, pois o aluno, ao invés de realizar a divisão entre número
de pessoas e dimensão do espaço em km^2, realizou a divisão da dimensão
pelo número de pessoas.

C: Incorreta, pois o aluno pode chegar a esse resultado cortando um
``zero'' a mais da expressão .

D: Incorreta, pois o aluno pode chegar a esse resultado cortando um
``zero'' a menos da expressão.
\item SAEB: Construir/desenhar figuras geométricas planas ou espaciais que
satisfaçam condições dadas.

BNCC: EF08MA18 -- Reconhecer e construir figuras obtidas por composições
de transformações geométricas (translação, reflexão e rotação), com o
uso de instrumentos de desenho ou de softwares de geometria dinâmica.

A: Correta, pois

Calculando a parede da casa obtemos que a mesma possuirá 28m^2.

Cada tijolo possui 0,11 . 0,24 = 0,0264m^2 de área.

Dividindo (28 \div 0,0264 = 1.060\; tijolos).

Logo, 1500 - 1060 = 440 tijolos sobraram.

B: Incorreta, pois o aluno chegaria a esse valor caso confundisse área
com perímetro.

C: Incorreta, pois o aluno, por meio de indução, pode ser levado a
assinalá-la.

D: Incorreta, pois o aluno pode assinalar a alternativa que pareça mais
plausível para o enunciado.
\item SAEB: Resolver problemas que envolvam relações entre ângulos formados
por retas paralelas cortadas por uma transversal, ângulos internos ou
externos de polígonos ou cevianas (altura, bissetriz, mediana,
mediatriz) de polígonos.

A: Correta, pois, utilizando as informações da imagem, temos que

68 + 90 + ângulo BHA = 180

158 + ângulo BHA = 180

Ângulo BHA = 22.

Logo, como há uma bissetriz, temos que 22 + 22= 44°

180°- 44°= 136°.

B: Incorreta, pois o aluno chegaria a essa conclusão calculando apenas o
valor do ângulo BHA.

C: Incorreta, pois o aluno chegaria a esse valor calculando apenas o
valor da bissetriz do ângulo.

D: Incorreta, pois o aluno chegaria a essa conclusão considerando o
valor de outro ângulo diferente ao enunciado.
\item SAEB: Descrever ou esboçar deslocamento de pessoas e/ou de objetos em
representações bidimensionais (mapas, croquis etc.), plantas de
ambientes ou vistas, de acordo com condições dadas.

A: pois, o aluno, ao invés de realizar uma divisão dos termos, pode
realizar uma porcentagem, chegando a esse resultado erroneamente.

B: Incorreta, pois o aluno pode chegar a essa conclusão confundindo o
número de dias com a quantidade de kms a serem percorridos.

C: Incorreta, pois o aluno chegaria a essa conclusão desconsiderando as
casas decimais.

D: Correta, pois

1.235km : 30 dias = 41,16 km diários; logo, deveremos completar, no
mínimo, 42 km por dia.
\item SAEB: Calcular os valores de medidas de tendência central de uma
pesquisa estatística (média aritmética simples, moda ou mediana).

BNCC: EF08MA25 -- Obter os valores de medidas de tendência central de
uma pesquisa estatística (média, moda e mediana) com a compreensão de
seus significados e relacioná-los com a dispersão de dados, indicada
pela amplitude.

A: Correta, pois

Somando os meses, obtemos: 73,05

73,05 : 6 = 12,175.

B: Incorreta, pois o aluno chegaria a esse resultado selecionando menor
preço.

C: Incorreta, pois o aluno chegaria a esse resultado selecionando o
maior valor.

D: Incorreta, pois o aluno chegaria a esse resultado calculando a
mediana dos preços.
\item SAEB: Resolver problemas que envolvam medidas de grandezas (comprimento,
massa, tempo, temperatura, capacidade ou volume) em que haja conversões
entre unidades mais usuais.

BNCC: EF08MA19 -- Resolver e elaborar problemas que envolvam medidas de
área de figuras geométricas, utilizando expressões de cálculo de área
(quadriláteros, triângulos e círculos), em situações como determinar
medida de terrenos.

A: Incorreta, pois o aluno chegará a esse valor caso considere
multiplicar as duas diagonais em busca do valor da área.

B: Correta, pois, utilizando a formula da área do losango, temos que:

A=(\frac{\text{D\ .\ d}}{2})=

A= (\frac{7,2\ .\ 4,4}{2})

A= 15,84 m^2

C: Incorreta, pois o aluno, ao invés de utilizar a fórmula do losango
para definir a área, pode somar os valores.

D: Incorreta, pois o aluno, ao invés de utilizar a fórmula do losango
para definir a área, pode dividir os valores.
\item SAEB: Resolver problemas que envolvam a probabilidade de ocorrência de
um resultado em eventos aleatórios equiprováveis independentes ou
dependentes.

A: Correta, pois, temos que o número de vogais do nosso alfabeto é 5,
logo:

5 . 5 . 10 . 10 . 10 . 10 = 250.000 possibilidades de senhas diferentes.

B: Incorreta, pois, ao não considerar ``um elemento multiplicativo 10'',
o aluno chegaria a essa conclusão erroneamente.

C: Incorreta, pois, ao não considerar ``dois elementos multiplicativos
10'', o aluno chegaria a essa conclusão erroneamente.

D: Incorreta, pois, ao não considerar ``três elementos multiplicativos
10'', o aluno chegaria a essa conclusão erroneamente.
\item SAEB: Identificar, no plano cartesiano, figuras obtidas por uma ou mais
transformações geométricas (reflexão, translação, rotação).

BNCC: EF08MA18 -- Reconhecer e construir figuras obtidas por composições
de transformações geométricas (translação, reflexão e rotação), com o
uso de instrumentos de desenho ou de softwares de geometria dinâmica.

A: Incorreta, pois os valores dos vértices não condizem com o processo
de reflexão.

B: Correta, pois, ao realizar uma reflexão em relação ao eixo x, os
pontos mantêm a mesma coordenada x, mas têm sua coordenada y negativa.
No triângulo ABC original, o ponto A(2, 4) terá a mesma coordenada x,
mas sua coordenada y será negativa, resultando em A'(-2, -4). Da mesma
forma, os pontos B(5, 6) e C(7, 2) terão suas coordenadas y negativas
após a reflexão, resultando em B'(5, -6) e C'(7, -2), respectivamente.

C: Incorreta, pois os valores dos vértices não condizem com o processo
de reflexão.

D: Incorreta, pois os valores dos vértices não condizem com o processo
de reflexão.
\item SAEB: Resolver problemas que envolvam dados estatísticos apresentados em
tabelas (simples ou de dupla entrada) ou gráficos (barras simples ou
agrupadas, colunas simples ou agrupadas, pictóricos, de linhas, de
setores ou em histograma).

BNCC: EF08MA25 -- Obter os valores de medidas de tendência central de
uma pesquisa estatística (média, moda e mediana) com a compreensão de
seus significados e relacioná-los com a dispersão de dados, indicada
pela amplitude.

A: Incorreta, pois a Livraria C vendeu 150 livros, enquanto a Livraria A
e a Livraria B juntas venderam 120 + 80 = 200 livros. Portanto, as
livrarias A e B venderam mais livros do que a Livraria C.

B: Incorreta, pois a Livraria A vendeu 120 livros, enquanto a Livraria B
vendeu 80 livros. Portanto, a Livraria B vendeu menos livros do que a
Livraria A.

C: Incorreta, pois a diferença entre o número de livros vendidos pela
Livraria C (150) e a Livraria A (120) é de 30 livros, e não 70 livros.

D: Correta, pois a Livraria B vendeu 80 livros, enquanto a Livraria C
vendeu 150 livros, e 150 é o dobro de 80.
\end{enumerate}


\section*{Matemática – Simulado 3}
\begin{numerate}
\item SAEB: Escrever números racionais (representação fracionária ou decimal
finita) em sua representação por algarismos ou em língua materna ou
associar o registro numérico ao registro em língua materna.

A: Incorreta, pois 35 segundos não é representado por números decimais.

B: Correta, pois, quando há três casas decimais, temos milésimos de
segundos.

C: Incorreta, pois 35 centésimos de segundos apareceria com duas casas
decimais após a vírgula.

D: Incorreta, pois 3 minutos é uma quantidade representada por um número
inteiro
\item SAEB: Identificar os indivíduos (universo ou população-alvo da
pesquisa), as variáveis e os tipos de variáveis (quantitativas ou
categóricas) em um conjunto de dados.

BNCC: EF08MA25 -- Obter os valores de medidas de tendência central de
uma pesquisa estatística (média, moda e mediana) com a compreensão de
seus significados e relacioná-los com a dispersão de dados, indicada
pela amplitude.

A: Correta, pois somente profissão não pode ser quantificada.

B: Incorreta, pois batimentos cardíacos constituem uma variável
quantitativa.

C: Incorreta, pois profissão não é uma variável quantitativa e
batimentos cardíacos não são qualitativos.

D: Incorreta, pois profissão não é uma variável quantitativa.
\item SAEB: Representar ou associar os dados de uma pesquisa estatística ou de
um levantamento em listas, tabelas (simples ou de dupla entrada) ou
gráficos (barras simples ou agrupadas, colunas simples ou agrupadas,
pictóricos, de linhas, de setores, ou em histograma).

BNCC: EF08MA25 -- Obter os valores de medidas de tendência central de
uma pesquisa estatística (média, moda e mediana) com a compreensão de
seus significados e relacioná-los com a dispersão de dados, indicada
pela amplitude.

A: Incorreta, pois o histograma é feito por linhas e barras.

B: Incorreta, pois o gráfico de barras é formado por barras retangulares
e com base maior na horizontal.

C: Correta, pois esse tipo de gráico apresenta setores de uma figura
geométrica, geralmente, um círculo.

D: Incorreta, pois o gráfico de linhas é representado por pontos unidos
por linhas.
\item SAEB: Resolver problemas que envolvam relações métricas do triângulo
retângulo, incluindo o teorema de Pitágoras.

A: Correta, pois a^2 + b^2 = c^2 representa corretamente o Teorema de
Pitágoras, que estabelece que, em um triângulo retângulo, o quadrado da
hipotenusa (c) é igual à soma dos quadrados dos catetos (a e b).

B: Incorreta, pois, na verdade, a soma dos quadrados dos catetos a e b é
igual ao quadrado da hipotenusa c, como afirma o Teorema de Pitágoras.

C: Incorreta, pois, em um triângulo retângulo, os lados a e b são os
catetos, e eles podem ter medidas diferentes.

D: Incorreta, pois não há uma relação específica entre as medidas dos
lados a, b e c de um triângulo retângulo. As medidas podem variar
dependendo do triângulo em questão.
\item SAEB: Converter uma representação de um número racional positivo para
outra representação.

A: Incorreta, pois todo número decimal finito pode ser representado por
fração e o número 2 é o único par que é primo.

B: Correta, pois essas são as afirmações certas.

C: Incorreta, pois o número 2 é o único número par que é primo, então é
uma verdade

D: Incorreta, pois a afirmação III é falsa. O número 21 não é primo, ele
possui 4 divisores: 1, 3,7 e 21.
\item SAEB: Inferir a finalidade da realização de uma pesquisa estatística ou
de um levantamento, dada uma tabela (simples ou de dupla entrada) ou
gráfico, (barras simples ou agrupadas, colunas simples ou agrupadas,
pictóricos, de linhas, de setores ou em histograma) com os dados dessa
pesquisa.

A: Incorreta, pois o gráfico não faz essa relação.

B: Correta, pois o gráfica não apresenta faixas etárias.

C: Incorreta, pois o gráfico não apresenta tais fatores.

D: Correta, pois a tabela de dupla entrada apresenta a relação entre
duas variáveis: o nível de escolaridade e a renda média mensal. A
finalidade dessa pesquisa é analisar e inferir como o nível de
escolaridade influencia a renda dos indivíduos. Ao cruzar os dados da
tabela, é possível observar se há uma relação entre a escolaridade e a
renda e, assim, avaliar a influência dessa variável na determinação da
renda média mensal.
\item SAEB: Interpretar o significado das medidas de tendência central (média,
aritmética simples, moda e mediana) ou da amplitude.

A: Incorreta, pois a I não está correta. mMédia, é a soma dos valores do
conjunto de dados,dividido pela quantidade dos dados.

B: Incorreta, pois somente a I está incorreta.

C: Correta, pois a I é falsa. Moda é o valor que mais se repete.

D: Incorreta, pois, a definição de moda está incorreta na I.
\item SAEB: Relacionar o número de vértices, faces ou arestas de prismas ou
pirâmides, em função do seu polígono da base.

BNCC: EF08MA18 -- Reconhecer e construir figuras obtidas por composições
de transformações geométricas (translação, reflexão e rotação), com o
uso de instrumentos de desenho ou de softwares de geometria dinâmica.

A: Incorreta, pois, se um prisma retangular tivesse 8 arestas, teria
apenas duas arestas por face, o que não seria suficiente para formar as
arestas laterais.

B: Incorreta, pois, se um prisma retangular tivesse 10 arestas, teria
três arestas por face, o que também não seria suficiente para formar as
arestas laterais.

C: Incorreta, pois, se um prisma retangular tivesse 12 arestas, teria
quatro arestas por face, o que ainda não seria suficiente para formar as
arestas laterais.

D: Correta, pois um prisma retangular possui 12 arestas na base (4
arestas do retângulo superior + 4 arestas do retângulo inferior + 4
arestas verticais que conectam as bases). Além disso, existem duas
arestas laterais que se estendem verticalmente e conectam os vértices
das bases, totalizando 14 arestas.
\item SAEB: Reconhecer circunferência/círculo como lugares geométricos, seus
elementos (centro, raio, diâmetro, corda, arco, ângulo central, ângulo
inscrito).

BNCC: EF08MA18 -- Reconhecer e construir figuras obtidas por composições
de transformações geométricas (translação, reflexão e rotação), com o
uso de instrumentos de desenho ou de softwares de geometria dinâmica.

A: Incorreta, pois uma corda não é apenas um segmento de reta que liga
dois pontos da circunferência, já que não necessariamente passa pelo
centro da circunferência.

B: Incorreta, pois essa opção descreve o raio da circunferência, não a
corda. O raio liga o centro da circunferência a um ponto específico na
circunferência, enquanto a corda liga dois pontos quaisquer da
circunferência.

C: Incorreta, pois um arco da circunferência não pode ser considerado
uma corda. A corda é um segmento de reta, enquanto o arco é uma parte da
circunferência.

D: Correta, pois a definição correta de uma corda é um segmento de reta
que liga o centro da circunferência a um ponto médio de um arco da
circunferência. Isso significa que a corda passa pelo centro da
circunferência e divide o arco em duas partes iguais.
\item SAEB: Determinar o ponto médio de um segmento de reta ou a distância
entre dois pontos quaisquer, dadas as coordenadas desses pontos no plano
cartesiano.

BNCC: EF08MA14 -- Demonstrar propriedades de quadriláteros por meio da
identificação da congruência de triângulos.

A: Incorreta, pois os valores obtidos após as operações não correspondem
à alternativa.

B: Correta, pois o ponto médio de um segmento de reta AB, cujas
coordenadas dos pontos extremos são A(x1, y1) e B(x2, y2), é dado pelas
coordenadas do ponto M(xm, ym), em que xm = (x1 + x2)/2 e ym = (y1 +
y2)/2. Substituindo os valores dados na questão, temos: xm = (1 + 5)/2 =
3 e ym = (2 + 6)/2 = 4. Portanto, o ponto médio do segmento de reta AB é
M(3, 4). As demais alternativas não correspondem às coordenadas do ponto
médio.

C: Incorreta, pois os valores obtidos após as operações não correspondem
à alternativa.

D: Incorreta, pois os valores obtidos após as operações não correspondem
à alternativa.
\item SAEB: Identificar retas ou segmentos de retas concorrentes, paralelos ou
perpendiculares.

BNCC: EF08MA14 -- Demonstrar propriedades de quadriláteros por meio da
identificação da congruência de triângulos.

A: Incorreta, pois duas retas são paralelas se possuem a mesma
inclinação e intercepto y iguais.

B: Incorreta, pois duas retas são concorrentes se possuem inclinações
diferentes, e duas retas são paralelas se possuem a mesma inclinação.

C: Incorreta, pois duas retas podem ser paralelas sem ter inclinação
igual a zero.

D: Correta, duas retas são perpendiculares se a inclinação de uma é a
negativa inversa da outra (isto é, seus coeficientes angulares
multiplicados resultam em -1).
\item SAEB: Identificar propriedades e relações existentes entre os elementos
de um triângulo (condição de existência, relações de ordem entre as
medidas dos lados e as medidas dos ângulos internos, soma dos ângulos
internos, determinação da medida de um ângulo interno ou externo).

BNCC: EF08MA14 -- Demonstrar propriedades de quadriláteros por meio da
identificação da congruência de triângulos.

A: Incorreta, pois a soma dos ângulos internos de um triângulo é igual a
180 graus.

B: Incorreta, pois, embora seja possível ter triângulos com um ângulo
interno igual a 90 graus (triângulo retângulo), essa não é uma condição
necessária para a existência de um triângulo.

C: Correta, pois essa é uma regra aplicada aos triângulos.

D: Incorreta, pois essa alternativa apresenta uma informação
contraditória, pois a hipotenusa é o maior lado em um triângulo
retângulo, portanto não pode ser menor que um dos outros lados. Além
disso, não há relação definida entre a medida do menor lado e a da
hipotenusa.
\item SAEB: Classificar triângulos ou quadriláteros em relação aos lados ou
aos ângulos internos.

BNCC: EF08MA14 -- Demonstrar propriedades de quadriláteros por meio da
identificação da congruência de triângulos.

A: Incorreta, pois um triângulo escaleno não possui lados congruentes.

B: Correta, pois essa é a definição de um triângulo isósceles.

C: Incorreta, pois um triângulo equilátero possui todos os lados
congruentes.

D: Incorreta, pois triângulo retângulo possui um ângulo interno reto.
\item SAEB: Identificar relações entre ângulos formados por retas paralelas
cortadas por uma transversal.

BNCC: EF08MA14 -- Demonstrar propriedades de quadriláteros por meio da
identificação da congruência de triângulos.

A: Incorreta, pois o aluno errou na soma dos ângulos.

B: Incorreta, pois o aluno não soube aplicar a soma dos ângulos
internos.

C: Incorreta, pois o aluno não somou 80º para encontrar o resultado
correto.

D: Correta, pois a soma dos ângulos internos de um triângulo é sempre
igual a 180 graus. Se um dos ângulos mede 90 graus, a soma dos outros
dois ângulos deve ser igual a 180 - 90 = 90 graus.
\item SAEB: Resolver problemas que envolvam polígonos semelhantes.

BNCC: EF08MA14 -- Demonstrar propriedades de quadriláteros por meio da
identificação da congruência de triângulos.

A: Incorreta, pois essa é a metade do comprimento do primeiro terreno, o
que não condiz com a proporção estabelecida entre os terrenos.

B: Incorreta, pois essa é uma opção que poderia ser confundida com a
resposta correta, já que é um múltiplo do comprimento do primeiro
terreno. No entanto, não corresponde à proporção estabelecida entre os
terrenos.

C: correta, pois sabemos que os terrenos são semelhantes, logo, os lados
correspondentes são proporcionais. Como a largura do segundo terreno é
1,5 vezes maior do que a largura do primeiro (30 m/20 m), podemos
afirmar que o comprimento do segundo terreno é 1,5 vezes maior do que o
comprimento do primeiro terreno.

D: Incorreta, pois essa é uma opção que poderia ser confundida com a
resposta correta, já que é um múltiplo da largura do segundo terreno. No
entanto, não corresponde à proporção estabelecida entre os terrenos.
\item SAEB: Inferir uma equação polinomial de 2º grau que modela um problema.

BNCC: EF08MA09 -- Resolver e elaborar, com e sem uso de tecnologias,
problemas que possam ser representados por equações polinomiais de 2º
grau do tipo ax^2 = b.

A: Correta, pois a combinação dos termos -9,8t^2 e 30t representa
corretamente a queda da altura do objeto devido à gravidade e o aumento
inicial da altura com a velocidade inicial de lançamento.

B: Incorreta, pois o sinal negativo em ambos os termos não representa
corretamente o comportamento do objeto em relação à altura.

C: Incorreta, pois o sinal positivo em ambos os termos não representa
corretamente o comportamento do objeto em relação à altura.

D: Incorreta, pois o sinal positivo em ambos os termos não representa
corretamente o comportamento do objeto em relação à altura.
\end{enumerate}


\section*{Matemática – Simulado 4}
\begin{numerate}
\item SAEB: Resolver problemas que envolvam perímetro de figuras planas.

A: Incorreta, pois, ao aplicar a fórmula do perímetro, chegamos a um
resultado diferente.

B: Incorreta, pois, ao aplicar a fórmula do perímetro, chegamos a um
resultado diferente.

C: Correta, pois, no caso da cerca retangular descrita na questão, temos
que ``a'' = 10 metros e ``b'' = 6 metros. Então, podemos calcular o
perímetro da seguinte maneira:

P = 2a + 2b P = 2(10) + 2(6) P = 20 + 12 P = 32

D: Incorreta, pois, ao aplicar a fórmula do perímetro, chegamos a um
resultado diferente.
\item SAEB: Explicar/descrever os passos para a realização de uma pesquisa
estatística ou de um levantamento.

BNCC: EF08MA25 -- Obter os valores de medidas de tendência central de
uma pesquisa estatística (média, moda e mediana) com a compreensão de
seus significados e relacioná-los com a dispersão de dados, indicada
pela amplitude.

A: Correta, pois essa é a ordem correta de uma pesquisa estatística.

B: Incorreta, pois o primeiro passo está na posição 2.

C: Incorreta, pois o primeiro passo está na posição 2.

D: Incorreta, pois o primeiro passo está na posição 3.
\item SAEB: Argumentar ou analisar argumentações/conclusões com base nos dados
apresentados em tabelas (simples ou de dupla entrada) ou gráficos
(barras simples ou agrupadas, colunas simples ou agrupadas, pictóricos,
de linhas, de setores ou em histograma).

BNCC: EF08MA25 -- Obter os valores de medidas de tendência central de
uma pesquisa estatística (média, moda e mediana) com a compreensão de
seus significados e relacioná-los com a dispersão de dados, indicada
pela amplitude.

A: Incorreta, pois, ao visualizar 2 valores a menos do que realmente
está na tabela, o aluno chegará a essa conclusão erroneamente.

B: Correta, pois entre as notas fornecidas, temos 10 notas maiores ou
iguais a 7,0.

C: Incorreta, pois, ao visualizar 1 valor a menos do que realmente está
na tabela, o aluno chegará a essa conclusão erroneamente.

D: Incorreta, pois, ao visualizar 1 valor a mais do que realmente está
na tabela, o aluno chegará a essa conclusão erroneamente.
\item SAEB: Resolver problemas que envolvam relações entre os elementos de uma
circunferência/círculo (raio, diâmetro, corda, arco, ângulo central,
ângulo inscrito).

BNCC: EF08MA18 -- Reconhecer e construir figuras obtidas por composições
de transformações geométricas (translação, reflexão e rotação), com o
uso de instrumentos de desenho ou de softwares de geometria dinâmica.

A: Incorreta, pois o aluno pode esquecer que o valor do raio é a metade
do diâmetro, chegando nesse valor.

B: Incorreta, pois ao confundir a fórmula do perímetro da circunferência
com a fórmula da área da circunferência chegará a esse valor.

C: Incorreta, pois, ao realizar uma soma ao invés de uma multiplicação
na fórmula, obterá esse valor.

D: Correta, pois ao considerar pi = 3, temos que 2.3.6 = 36 cm
\item SAEB: Reconhecer polígonos semelhantes ou as relações existentes entre
ângulos e lados correspondentes nesses tipos de polígonos.

BNCC: EF08MA18 -- Reconhecer e construir figuras obtidas por composições
de transformações geométricas (translação, reflexão e rotação), com o
uso de instrumentos de desenho ou de softwares de geometria dinâmica.

A: Incorreta, pois o número de lados não é suficiente para determinar se
dois polígonos são semelhantes.

B: Correta, pois dois polígonos são semelhantes se, ao sobrepor um sobre
o outro, as medidas de seus lados correspondentes são multiplicadas por
uma mesma constante de proporcionalidade.

C: Incorreta, pois mesmo que dois polígonos tenham áreas iguais, eles
não necessariamente são semelhantes.

D: Incorreta, pois dois polígonos podem ter os mesmos comprimentos de
lado, mas não serem semelhantes.
\item SAEB: Classificar polígonos em regulares e não regulares.

BNCC: EF08MA18 -- Reconhecer e construir figuras obtidas por composições
de transformações geométricas (translação, reflexão e rotação), com o
uso de instrumentos de desenho ou de softwares de geometria dinâmica.

A: é Incorreta, pois o polígono não possui lados e ângulos congruentes.

B: Correta, pois essa é uma das características do polígono.

C: é Incorreta, pois a medida dos lados não influencia na classificação
do polígono como regular ou não regular.

D: é Incorreta, pois mesmo que os ângulos internos do polígono sejam
congruentes, ainda assim é impossível que seus lados sejam congruentes.
\item SAEB: Associar uma equação polinomial de 1º grau com duas variáveis a
uma reta no plano cartesiano.

BNCC: EF08MA07 -- Associar uma equação linear de 1º grau com duas
incógnitas a uma reta no plano cartesiano.

A: Correta, pois uma equação polinomial de 1º grau com duas variáveis,
na forma y = ax + b, representa uma reta no plano cartesiano. O
coeficiente a é a inclinação da reta e o coeficiente b é o intercepto no
eixo y. Assim, podemos associar a equação y = 2x - 3 com a reta que
passa pelo ponto (0, -3) e tem inclinação 2. A inclinação positiva
indica que a reta sobe da esquerda para a direita no plano cartesiano. O
intercepto no eixo y indica que a reta cruza o eixo y no ponto (0, -3

B: Incorreta, pois alternativa apresenta uma equação com inclinação
negativa.

C: Incorreta, pois a alternativa apresenta uma equação com inclinação
diferente de 1.

D: Incorreta, pois a alternativa apresenta uma equação com inclinação
diferente de 1.
\item SAEB: Inferir uma equação, inequação polinomial de 1º grau ou um sistema
de equações de 1º grau com duas incógnitas que modela um problema.

BNCC: EF08MA07 -- Associar uma equação linear de 1º grau com duas
incógnitas a uma reta no plano cartesiano.

Alternativa A: Incorreta, pois ,ao realizar incorretamente a montagem da
regra de 3, o aluno chegará a esse valor incorreto.

Alternativa B: Incorreta, pois, ao realizar incorretamente a
multiplicação do resultado final por 2, o aluno chegará a esse resultado
equivocadamente.

Alternativa C: Correta, pois, ao realizarmos a regra de três
(\frac{14}{7} . \frac{28}{x}), levando em conta os outros 3.360 m^2,
chegamos a 14 horas.

Alternativa D: Incorreta, pois, ao realizar incorretamente a montagem da
regra de 3, o aluno chegará a esse valor incorreto.
\item SAEB: Representar frações menores ou maiores que a unidade por meio de
representações pictóricas ou associar frações a representações
pictóricas.

BNCC: EF08MA05 -- Reconhecer e utilizar procedimentos para a obtenção de
uma fração geratriz para uma dízima periódica.

A: Incorreta, pois o aluno provavelmente efetuou a operação de maneira
incorreta.

B: Correta, pois
(\frac{2}{4} = \frac{(2 . 2)} {(4 . 2)} = \frac{4}{8} > \frac{3}{8})
logo, Maria comeu mais torta.

C: Incorreta, pois a operação demonstra que Maria e João comeram porções
diferentes.

D: Incorreta, pois o aluno deve saber comparar frações diferentes.
\item SAEB: Resolver problemas que envolvam as ideias de múltiplo, divisor,
máximo divisor comum ou mínimo múltiplo comum.

BNCC: EF08MA02 -- Resolver e elaborar problemas usando a relação entre
potenciação e radiciação, para representar uma raiz como potência de
expoente fracionário.

A: Incorreta, pois considerou que o tempo de encontro seria pelo M.D.C
ao invés do M.M.C. .

B: Incorreta, pois apenas foi feita a soma do tempo de cada corredor.

C: Incorreta, pois foi associado um número errado de minutos.

D: Correta, pois devemos usar a ideia de mínimo múltiplo comum. Assim,
calculando o M.M.C dos tempos, temos:
\item SAEB: Argumentar ou analisar argumentações/conclusões com base nos dados
apresentados em tabelas (simples ou de dupla entrada) ou gráficos
(barras simples ou agrupadas, colunas simples ou agrupadas, pictóricos,
de linhas, de setores ou em histograma).

BNCC: EF08MA25 -- Obter os valores de medidas de tendência central de
uma pesquisa estatística (média, moda e mediana) com a compreensão de
seus significados e relacioná-los com a dispersão de dados, indicada
pela amplitude.

A: Incorreta, pois, ao visualizar 2 valores a menos do que realmente
está na tabela, o aluno chegará a essa conclusão erroneamente.

B: Correta, pois entre as notas fornecidas, temos 10 notas maiores ou
iguais a 7,0.

C: Incorreta, pois, ao visualizar 1 valor a menos do que realmente está
na tabela, o aluno chegará a essa conclusão erroneamente.

D: Incorreta, pois, ao visualizar 1 valor a mais do que realmente está
na tabela, o aluno chegará a essa conclusão erroneamente.
\item SAEB: Calcular o resultado de potenciação ou radiciação envolvendo
números reais.

BNCC: EF08MA02 -- Resolver e elaborar problemas usando a relação entre
potenciação e radiciação, para representar uma raiz como potência de
expoente fracionário.

Alternativa A: Incorreta, pois a resolução da expressão não corresponde
a esse resultado.

Alternativa B: Incorreta, pois a resolução da expressão não corresponde
a esse resultado.

Alternativa C: Incorreta, pois a resolução da expressão não corresponde
a esse resultado.

Alternativa D: correta pois 3^2 significa 3 elevado ao quadrado, ou seja,
3 x 3 = 9. (\sqrt 16) é 4. Assim, a expressão pode ser reescrita como
9 + 4 = 13.
\item SAEB: Compor ou decompor números racionais positivos (representação
decimal finita) na forma aditiva, ou em suas ordens, ou em adições e
multiplicações.

A: Incorreta, pois, caso o aluno resolva transformar km em m, esse seria
o valor, mas não é isso que o enunciado pede.

B: Correta, pois O número 2.251 possui 2 classes e 4 ordens.

C: Incorreta, pois o aluno pode considerar que o numeral signifique o
valor da classe, o que está incorreto.

D: Incorreta, pois o aluno pode confundir classes com ordens.
\item SAEB: Comparar ou ordenar números reais, com ou sem suporte da reta
numérica, ou aproximar número reais para múltiplos de potência de 10
mais próxima.

A: Incorreta, pois o aluno pode chegar a essa conclusão esquecendo que
que no sistema romano são apenas em casos específicos de impossibilidade
de colocar mais de 3 símbolos iguais para representar o mesmo número
para inserir um número antes de outro representando uma subtração
momentânea.

B: Incorreta, pois o aluno pode chegar a essa conclusão esquecendo que
que no sistema romano são apenas em casos específicos de impossibilidade
de colocar mais de 3 símbolos iguais para representar o mesmo número
para inserir um número antes de outro representando uma subtração
momentânea.

C: Correta, pois

40 = XL

9 = IX

XLIX = 40 + 9 = 49

D: Incorreta, pois o aluno pode chegar a essa conclusão esquecendo que
que no sistema romano são apenas em casos específicos de impossibilidade
de colocar mais de 3 símbolos iguais para representar o mesmo número
para inserir um número antes de outro representando uma subtração
momentânea.
\item  SAEB: Identificar um número natural como primo, composto,
``múltiplo/fator de'' ou ``divisor de'' ou identificar a decomposição de
um número natural em fatores primos ou relacionar as propriedades
aritméticas (primo, composto, ``múltiplo/fator de'' ou ``divisor de'')
de um número natural à sua decomposição em fatores primos.

A: Incorreta, pois, por questão de semelhança, o aluno pode considerar
que as 3 pessoas citadas no enunciado abastecerão sempre no mesmo dia 28
de todo mês.

B: Incorreta, pois o aluno pode realizar uma soma dos dias ao invés de
realizar o m.m.c.

C: Correta, pois

MMC(5;7,2) = 70

D: Incorreta, pois o aluno pode confundir m.m.c. por m.d.c.
\end{enumerate}