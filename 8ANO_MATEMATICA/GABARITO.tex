\chapter{Respostas}
\pagestyle{plain}
\footnotesize

\pagecolor{gray!40}

\section*{Matemática – Módulo 1 – Treino}

\begin{enumerate}
\item SAEB: Compor ou decompor números racionais positivos (representação
decimal finita) na forma aditiva, ou em suas ordens, ou em adições e
multiplicações.

a) Incorreta. O aluno não computou um elemento 2 e um elemento 5
na fatoração.
b) Incorreta. O aluno computou um elemento a mais e um elemento 5 a
menos na fatoração.
c) Correta. Ao decompor o número 1.000 em fatores primos, obtemos
$$2^3 \times 5^3$$.
d) Incorreta. O aluno não computou um elemento 5 na fatoração.

\item SAEB: Identificar um número natural como primo, composto,
``múltiplo/fator de'' ou ``divisor de'' ou identificar a decomposição de
um número natural em fatores primos ou relacionar as propriedades
aritméticas (primo, composto, ``múltiplo/fator de'' ou ``divisor de'')
de um número natural à sua decomposição em fatores primos.

a) Incorreta. O aluno pode realizar a divisão incorretamente do
valor 123.456.789 e chegar a essa conclusão.
b) Incorreta. O aluno pode realizar a divisão incorretamente do
valor 123.456.789 e chegar a essa conclusão.
c) Correta. Temos: 1 + 2 + 3 + 4 + 5 + 6 + 7 + 8 + 9 = 45, que é
múltiplo de 3; então 123.456.789 também será.
d) Incorreta. O aluno pode realizar a divisão incorretamente do
valor 123.456.789 e chegar a essa conclusão.

\item SAEB: Compor ou decompor números racionais positivos (representação
decimal finita) na forma aditiva, ou em suas ordens, ou em adições e
multiplicações.

a) Incorreta. Ao contar as casas erroneamente e considerar o
número uma casa à direita o aluno pode considerar esse.
b) Incorreta. Ao contar as casas erroneamente e considerar o
número duas casas à direita, o aluno pode considerar esse resultado.
c) Incorreta. Ao contar as casas erroneamente e considerar o
número uma casa à esquerda, o aluno pode considerar esse resultado.
d) Correta. O número 3 está situado na casa das centenas de milhar.

\end{enumerate}


\section*{Matemática – Módulo 2 – Treino}

\begin{enumerate}
\item SAEB: Resolver problemas de adição, subtração, multiplicação, divisão,
potenciação ou radiciação envolvendo número reais, inclusive notação
científica.
BNCC: EF08MA02 -- Resolver e elaborar problemas usando a relação entre
potenciação e radiciação, para representar uma raiz como potência de
expoente fracionário.

a) Incorreta. O número de zeros estava errado.
b) Incorreta. O número de zeros estava errado.
c) Correta. 10.000 = 10 . 10 . 10 . 10.
d) Incorreta. O número de zeros estava errado.

\item SAEB: Resolver problemas de adição, subtração, multiplicação, divisão,
potenciação ou radiciação envolvendo número reais, inclusive notação
científica

a) Incorreta. Ao confundir expoente com base, o aluno chegaria a
esse resultado.
b) Incorreta. O aluno chegaria a esse conclusão se somasse todos os
expoentes incorretamente.
c) Incorreta. Ao dividir o expoente do numerador por 2 ao invés de
3, o aluno chegaria a esse resultado.
d) Correta. Utilizando a propriedade de multiplicação e divisão de
potências de mesma base, temos: $$(2^{25+35+30}=2^{90})$$ no numerador
e $$(2^{2+1}) = 2^3$$ no denominador.
Realizando a divisão, temos que $$(2^{90}:3) = (2^{30})$$.
Considerando que a idade de Thiago é apenas o expoente, sabemos que
Thiago tem 30 anos.

\item SAEB: Resolver problemas de adição, subtração, multiplicação, divisão,
potenciação ou radiciação envolvendo número reais, inclusive notação
científica.
BNCC: EF08MA01 -- Efetuar cálculos com potências de expoentes inteiros e
aplicar esse conhecimento na representação de números em notação
científica.

a) Correta. $$(2^10) = 1024 \times 32 = 32.768$$ megabytes.
b) Incorreta. O aluno chegaria a esse resultado ao considerar que
$$(2^{10})$$ seja 1.000 ao invés de 1.024.
c) Incorreta. Ao considerar um 2 a menos na expressão, o aluno
chegaria a esse resultado.
d) Incorreta. Ao realizar apenas a multiplicação ao invés de
realizar o cálculo da potência, o aluno chegaria a esse resultado.


\end{enumerate}

\section*{Matemática – Módulo 3 – Treino}

\begin{numerate}
\item SAEB: Determinar uma fração geratriz para uma dízima periódica.
BNCC: EF08MA05 -- Reconhecer e utilizar procedimentos para a obtenção de
uma fração geratriz para uma dízima periódica.

a) Incorreta. O aluno pode considerar que o número (\pi) seja de
fato uma dízima periódica simples pelo fato de ter números pares na sua
composição.
b) Incorreta. O aluno pode não conhecer a diferença entre uma
dízima periódica simples e uma irracional.
c) Correta. O conceito foi descrito corretamente.
d) Incorreta. O aluno pode não ser capaz de identificar a diferença
entre uma dizima periódica simples e uma irracional.

\item SAEB: Representar frações menores ou maiores que a unidade por meio de
representações pictóricas ou associar frações a representações
pictóricas.
BNCC: EF08MA05 -- Reconhecer e utilizar procedimentos para a obtenção de
uma fração geratriz para uma dízima periódica.

a) Incorreta. Esse valor corresponde a $$(\frac{2}{3})$$ dos
participantes.
b) Incorreta. Esse valor corresponde à metade dos participantes.
c) Incorreta. Esse valor corresponde ao dobro dos partcipantes.
d) Correta. Um terço de 300 é igual a 100.


\item SAEB: Determinar uma fração geratriz para uma dízima periódica.
BNCC: EF08MA05 -- Reconhecer e utilizar procedimentos para a obtenção de
uma fração geratriz para uma dízima periódica.

a) Incorreta. O aluno pode chegar a esse valor ao calcular
erroneamente a expressão.
b) Incorreta. O aluno pode chegar a esse valor ao calcular
erroneamente a expressão.
c) Correta. 400 g : 22 pedaços = 18,1818181818\ldots{}.
d) Incorreta. O aluno pode chegar a esse valor ao calcular
erroneamente a expressão.

\end{numerate}


\section*{Matemática – Módulo 4 – Treino}
\begin{numerate}
\item SAEB: Resolver problemas que envolvam porcentagens, incluindo os que
lidam com acréscimos e decréscimos simples, aplicação de percentuais
sucessivos e determinação de taxas percentuais.
BNCC: EF08MA04 -- Resolver e elaborar problemas, envolvendo cálculo de
porcentagens, incluindo o uso de tecnologias digitais.

a) Correta. Somando os dois líquidos, temos que 120 + 80 = 200.
Utilizando a regra de 3 simples temos que:
$$(\frac{200}{120} = \frac{100}{x})
Logo 200 . x = 100 . 120
200 x = 12 000
x = 60\%.$$
b) Incorreta. O aluno chegaria a esse resultado caso inserisse um
zero a menos na expressão.
c) Incorreta. O aluno chegaria a esse resultado caso inserisse dois
zeros a menos na expressão.
d) incorreta. O aluno chegaria a esse resultado caso inserisse três
zeros a menos na expressão.

\item SAEB: Resolver problemas que envolvam porcentagens, incluindo os que
lidam com acréscimos e decréscimos simples, aplicação de percentuais
sucessivos e determinação de taxas percentuais.
BNCC: EF08MA04 -- Resolver e elaborar problemas, envolvendo cálculo de
porcentagens, incluindo o uso de tecnologias digitais.

a) Incorreta. O aluno chegaria a essa conclusão realizando a
multiplicação reta na regra de três ao invés de realizar a multiplicação
cruzada.
b) Correta. $$(\frac{x}{22} = \frac{100}{55})
55 . x = 22 . 100
55x = 2200
x = 44$$. 44 pessoas votaram nessa eleição.
c) Incorreta. O aluno poderia chegar a essa conclusão confundindo o
valor de votos ao total real de pessoas que votaram.
d) Incorreta. O aluno poderia chegar a essa conclusão confundindo o
valor da porcentagem com o valor total de pessoas na votação.

\item SAEB: Resolver problemas que envolvam porcentagens, incluindo os que
lidam com acréscimos e decréscimos simples, aplicação de percentuais
sucessivos e determinação de taxas percentuais.
BNCC: EF08MA04 -- Resolver e elaborar problemas, envolvendo cálculo de
porcentagens, incluindo o uso de tecnologias digitais.

a) Correta. Aplicando juros simples, temos:

$$J = C × i × t

6 000 = 120 000 . 0,01 . x

6 000 = 1200 x

x = 5 meses$$
b) Incorreta. O aluno chegaria a essa conclusão caso não realizasse
a operação de 1\% : 100.
c) Incorreta. O aluno chegaria a essa conclusão caso multiplicasse
120.000 por 0,01.
d) Incorreta. O aluno chegaria a essa conclusão caso considerasse
que o valor dos juros a 1\% ao mês.
\end{enumerate}


\section*{Matemática – Módulo 5 – Treino}
\begin{numerate}
\item SAEB: Resolver problemas que possam ser representados por sistema de
equações de 1º grau com duas incógnitas.
BNCC: EF08MA08 -- Resolver e elaborar problemas relacionados ao seu
contexto próximo, que possam ser representados por sistemas de equações
de 1º grau com duas incógnitas e interpretá-los, utilizando, inclusive,
o plano cartesiano como recurso.

a) Incorreta. Esse seria o resultado em litros do primeiro tanque.
b) Correta. Definindo como x e y os tanques, temos que:
$x + y = 900 x - 100 = y + 100 y + 100 = 2. (x - 100)$
Isolando o termo y, temos que:
$y + 100 = 2x- 200 y = 2x - 200 - 100 = 2x - 300$
Substituindo y na primeira equação, temos que:
$x + y = 900 x + 2x - 300 = 900 3x = 900 + 300 = 1.200 x = 1.200 : 3 x = 400$
Analogamente, temos que:
$x + y = 900 400 + y = 900 y = 900 - 400 y = 500 L$
c) Incorreta. O aluno chegará a essa conclusão ao errar o jogo de
sinal na equação final.
d) Incorreta. O aluno chegaria a essa conclusão ao somar os valores
do enunciado sem ao mesmo tentar realizar as operações.

\item SAEB: Resolver problemas que possam ser representados por sistema de
equações de 1º grau com duas incógnitas.
BNCC: EF08MA08 -- Resolver e elaborar problemas relacionados ao seu
contexto próximo, que possam ser representados por sistemas de equações
de 1º grau com duas incógnitas e interpretá-los, utilizando, inclusive,
o plano cartesiano como recurso.

a) Incorreta. O aluno pode confundir as operações e realizar a
operação de multiplicação .
b) Incorreta. O aluno pode esquecer de realizar o m.m.c. e chegar a
essa conclusão.
c) Correta.
$x + (\frac{1}{3}) = 11
x = 11 - (\frac{1}{3})
x = (\frac{32}{3})$
d) Incorreta. O aluno pode confundir as operações e realizar a
operação de divisão .

\item SAEB: Resolver uma equação polinomial de 1º grau.
BNCC: EF08MA07 -- Associar uma equação linear de 1º grau com duas
incógnitas a uma reta no plano cartesiano.

a) Incorreta. Durante a resolução da equação, caso o aluno erre o
sinal de 2,5 e passe o negativo, o valor final será esse.
b) Correta. Temos que:
$2x - 7 = -2,5x + 2
4,5x = 9
x = 2$
c) Incorreta. Durante a resolução da equação, caso o aluno erre o
sinal de -7 e passe o negativo, o valor final será esse.
d) Incorreta. Caso o aluno, no final da expressão, passe o valor
4,5 multiplicando ao invés de dividir, chegará a esse resultado.
\end{enumerate}


\section*{Matemática – Módulo 6 – Treino}
\begin{numerate}
\item SAEB: Resolver problemas que envolvam cálculo do valor numérico de
expressões algébricas.
BNCC: EF08MA10 -- Identificar a regularidade de uma sequência numérica
ou figural não recursiva e construir um algoritmo por meio de um
fluxograma que permita indicar os números ou as figuras seguintes.

a) Correta. Temos que:
$(3x + 2) . (x + 2)
3x^2 + 6x + 2x + 4
3x^2 + 8x + 4$
b) Incorreta. O aluno poderia confundir o enunciado e colocar o
resultado do perímetro.
c) Incorreta. O aluno poderia realizar uma soma ao invés de uma
multiplicação.
d) Incorreta. O aluno poderia chegar a esse valor realizando uma
divisão ao invés de uma multiplicação.

\item SAEB: Resolver problemas que envolvam cálculo do valor numérico de
expressões algébricas.
BNCC: EF08MA10 -- Identificar a regularidade de uma sequência numérica
ou figural não recursiva e construir um algoritmo por meio de um
fluxograma que permita indicar os números ou as figuras seguintes.

a) Incorreta. O aluno chegaria a essa conclusão apenas dividindo um
termo pelo outro.
b) Incorreta. O aluno chegaria a essa conclusão realizando apenas a
subtração de um termo pelo outro.
c) Correta. Temos que:
$(\frac{(a + b)^2}{8}\ ) = (\frac{a^{2} + 2ab + b^2}{8})$
Fazendo a substituição de $a^2 + b^2 = 34$ e $ab = 15$, temos que:
$(\frac{34 + (2.15)}{8} =) (\frac{34 + 30}{8}) = (\frac{64}{8}) = 8$
d) Incorreta. Foi realizada a soma ao invés da multiplicação no
último termo.

\item SAEB: Resolver problemas que envolvam cálculo do valor numérico de
expressões algébricas.
BNCC: EF08MA10 -- Identificar a regularidade de uma sequência numérica
ou figural não recursiva e construir um algoritmo por meio de um
fluxograma que permita indicar os números ou as figuras seguintes.

a) Incorreta. O aluno, ao errar o jogo de sinal no cálculo da área,
encontrará esse valor.
b) Incorreta. O aluno, ao errar o jogo de sinal no cálculo do
perímetro, encontrará esse valor.
c) Incorreta. O aluno, ao errar o jogo de sinal no cálculo da área,
encontrará esse valor.
d) Correta. Temos que:
$A = \pi r^{2}
A = 3 . (x^2-3) ^2
A = 3 . ((x^4) - 6x^2 + 9)
A = (3x^4) - 18x^2 + 27$

$P= 2\pi . r
P = 2 . 3. (x^2-3)
P = 2 . (3x^2 - 9)
P = 6x^2 - 18$
\end{enumerate}


\section*{Matemática – Módulo 7 – Treino}

\begin{numerate}
\item SAEB: Resolver problemas que possam ser representados por equações
polinomiais de 2º grau.
BNCC: EF08MA09 -- Resolver e elaborar, com e sem uso de tecnologias,
problemas que possam ser representados por equações polinomiais de 2º
grau do tipo ax2 = b.

a) Incorreta. O aluno pode chegar a essa conclusão considerando que
o enunciado pede apenas um valor das raízes da equação.
b) Correta. Temos que:
$x^2 - 7x = 0
(\frac{- ( - 7) \pm \sqrt{{( - 7)}^{2} - 4.1.0}}{2.1})
(\frac{7 \pm \sqrt{49}}{2})
(\frac{7 \pm 7}{2})
X1 = (\frac{7 + 7}{2})\$ = (\frac{14}{2}) = 7
X2 = (\frac{7 - 7}{2}) = 0$
c) Incorreta. O aluno pode chegar a essa conclusão considerando que
o enunciado pede apenas o valor antes de extrairmos as raízes da
equação.
d) Incorreta. O aluno pode chegar a essa conclusão esquecendo de
dividir o valor de uma das raízes por 2.

\item SAEB: Resolver problemas que possam ser representados por equações
polinomiais de 2º grau.
BNCC: EF08MA09 -- Resolver e elaborar, com e sem uso de tecnologias,
problemas que possam ser representados por equações polinomiais de 2º
grau do tipo ax2 = b.

a) Incorreta. O aluno pode chegar a essa conclusão considerando que
o enunciado pede apenas 1 valor das raízes da equação descrita.
b) Incorreta. O aluno pode chegar a essa conclusão considerando que
o enunciado pede apenas o valor antes de extrairmos as raízes da equação
descrita.
c) Incorreta. O aluno pode chegar a essa conclusão esquecendo de
dividir o valor de uma das raízes por 4.
d) Correta. Temos que:
$4x^2 + 9x = 0
(\frac{- 9 \pm \sqrt{9^{2} - 4.4.0}}{2.4})
(\frac{- 9 \pm \sqrt{81}}{8})
(\frac{- 9 \pm 9}{8})
X1 = (\frac{- 9 + 9}{8}) = 0
X2= (\frac{- 9 - 9}{8}) = (\frac{- 18}{8}) = - (\frac{9}{4}) = -2,25$

\item SAEB: Resolver problemas que possam ser representados por equações
polinomiais de 2º grau.
BNCC: EF08MA09 -- Resolver e elaborar, com e sem uso de tecnologias,
problemas que possam ser representados por equações polinomiais de 2º
grau do tipo ax2 = b.

a) Incorreta. O aluno pode chegar a esse valor ao não realizar a
radiciação necessária.

b) Incorreta. O aluno pode chegar a essa conclusão considerando que
o enunciado pede apenas um valor das raízes da equação descrita.

c) Correta. Temos que:
$6x^2 - 5x = 0
(\frac{- ( - 5) \pm \sqrt{{( - 5)}^{2} - 4.6.0}}{2( - 5)})
(\frac{5 \pm \sqrt{25}}{12})
(\frac{5 \pm 5}{12})
X1 = (\frac{5 + 5}{12}) = (\frac{10}{12}) = (\frac{5}{6})
X2 = (\frac{5 - 5}{12}) = 0$
Logo, temos que a parte preenchida do tanque foi de 5/6.

d) Incorreta. O aluno chegaria a esse valor apenas somando os
termos da equação e não realizando a operação por completo.
\end{enumerate}


\section*{Matemática – Módulo 8 – Treino}
\begin{numerate}
\item SAEB: Resolver problemas que envolvam variação de proporcionalidade
direta ou inversa entre duas ou mais grandezas, inclusive escalas,
divisões proporcionais e taxa de variação.

BNCC: EF08MA13 -- Resolver e elaborar problemas que envolvam grandezas
diretamente ou inversamente proporcionais, por meio de estratégias
variadas.

a) Correta. Por a regra de 3 simples, temos que:
3,5 . x = 567 -> x = 162 minutos = 2 horas e 42 minutos.

b) Incorreta. O aluno poderia chegar a esse valor realizando a
multiplicação reta na regra de 3 e não a multiplicação cruzada.

c) Incorreta. Esse seria o valor caso o aluno não convertesse horas
em minutos como forma de solução.

d) Incorreta. O aluno chegaria a esse resultado caso não realizasse
a última operação necessária, que é a divisão.

\item SAEB: Resolver problemas que envolvam variação de proporcionalidade
direta ou inversa entre duas ou mais grandezas, inclusive escalas,
divisões proporcionais e taxa de variação.

BNCC: EF08MA13 -- Resolver e elaborar problemas que envolvam grandezas
diretamente ou inversamente proporcionais, por meio de estratégias
variadas.

a) Incorreta. O aluno chegaria nesse resultado multiplicando reto a
regra de três ao invés de multiplicar cruzado.

b) Correta. Por regra de 3 simples, temos que:
5x = 64 . 3 -> 5x = 192 -> x = 38,4.

c) Incorreta. O aluno chegaria a esse valor caso, no final da
expressão, ao invés de realizar uma divisão, realizasse uma
multiplicação.

d) Incorreta. O aluno chegaria a essa conclusão considerando cada
rosquinha com 64 calorias.
\item SAEB: Resolver problemas que envolvam variação de proporcionalidade
direta ou inversa entre duas ou mais grandezas, inclusive escalas,
divisões proporcionais e taxa de variação.

BNCC: EF08MA13 -- Resolver e elaborar problemas que envolvam grandezas
diretamente ou inversamente proporcionais, por meio de estratégias
variadas.

a) Incorreta. O aluno chegaria a esse resultado realizando uma
multiplicação reta ao invés de uma multiplicação cruzada.

b) Correta. Por regra de 3 simples, temos que:
10x = 9 . 27 -> 10x = 243 -> x = 24,3 mL.

c) Incorreta. O aluno chegaria a essa conclusão se, ao final da
expressão, realizasse uma multiplicação.

d) Incorreta. O aluno chegaria a essa conclusão deslocando a
vírgula uma casa para esquerda durante o cálculo final.
\end{enumerate}


\section*{Matemática – Módulo 9 – Treino}
\begin{numerate}
\item SAEB: Construir/desenhar figuras geométricas planas ou espaciais que
satisfaçam condições dadas.

BNCC: EF08MA18 -- Reconhecer e construir figuras obtidas por composições
de transformações geométricas (translação, reflexão e rotação), com o
uso de instrumentos de desenho ou de softwares de geometria dinâmica.

a) Correta. Lado da figura = 117 : 9 = 13 cm.

b) Incorreta. O aluno poderia chegar a essa conclusão caso
considerasse que o eneágono regular possui 10 lados e não 9.

c: Incorreta. O aluno poderia chegar a essa conclusão caso
considerasse que o eneágono regular possui 7 lados e não 9.

d) Incorreta. O aluno poderia chegar a essa conclusão caso
considerasse que o eneágono regular possui 5 lados e não 9.
\item SAEB: Construir/desenhar figuras geométricas planas ou espaciais que
satisfaçam condições dadas.

BNCC: EF08MA18 -- Reconhecer e construir figuras obtidas por composições
de transformações geométricas (translação, reflexão e rotação), com o
uso de instrumentos de desenho ou de softwares de geometria dinâmica.

a) Incorreta. O aluno chegaria a esse valor caso confundisse a
fórmula da área do círculo com a fórmula do perímetro do círculo.

b) Incorreta. O aluno chegaria a essa conclusão caso se esquecesse
do termo quadrático da expressão.

c) Correta. $A = 3 . 9,15^2 = 251 m^2$.

d) Incorreta. O aluno chegaria a esse resultado caso dividisse a
expressão no final da fórmula ao invés de multiplicar.
\item SAEB: Construir/desenhar figuras geométricas planas ou espaciais que
satisfaçam condições dadas.

BNCC: EF08MA18 -- Reconhecer e construir figuras obtidas por composições
de transformações geométricas (translação, reflexão e rotação), com o
uso de instrumentos de desenho ou de softwares de geometria dinâmica.

a) Correta. O hexágono regular tem seis lados, e são 30 cm por porta-joia.
20 vezes 30 = 600 cm = 6 m de fita.

b) Incorreta. O aluno pode chegar a esse valor confundindo cm com
metros.

c) Incorreta. O aluno pode chegar a esse valor considerando que um
hexágono regular contenha 5 lados.

d) Incorreta. O aluno pode chegar a esse valor considerando que um
hexágono regular contenha 4 lados.
\end{enumerate}


\section*{Matemática – Módulo 10 – Treino}
\begin{numerate}
\item SAEB: Resolver problemas que envolvam relações entre ângulos formados
por retas paralelas cortadas por uma transversal, ângulos internos ou
externos de polígonos ou cevianas (altura, bissetriz, mediana,
mediatriz) de polígonos.

BNCC: EF08MA14 -- Demonstrar propriedades de quadriláteros por meio da
identificação da congruência de triângulos.

a) Incorreta. A definição de incentro está errada.

b) Incorreta. A definição de incentro está errada.

c) Incorreta. A definição de incentro está errada.

d) Correta. As bissetrizes do triângulo correspondem ao incentro.
\item SAEB: Resolver problemas que envolvam relações entre ângulos formados
por retas paralelas cortadas por uma transversal, ângulos internos ou
externos de polígonos ou cevianas (altura, bissetriz, mediana,
mediatriz) de polígonos.

BNCC: EF08MA14 -- Demonstrar propriedades de quadriláteros por meio da
identificação da congruência de triângulos.

a) Incorreta. Esse seria o valor relativo, e não a medida final.

b) Incorreta. Houve um erro na multiplicação.

c) Correta. Se a altura relativa à hipotenusa BC mede 9,3~cm, a
medida da hipotenusa será 18,6 cm.

d) Incorreta. O aluno chegará a essa conclusão ao errar o cálculo
de multiplicação dos termos destacados no enunciado.
\item SAEB: Identificar propriedades e relações existentes entre os elementos
de um triângulo (condição de existência, relações de ordem entre as
medidas dos lados e as medidas dos ângulos internos, soma dos ângulos
internos, determinação da medida de um ângulo interno ou externo)

BNCC: EF08MA14 -- Demonstrar propriedades de quadriláteros por meio da
identificação da congruência de triângulos.

a) Incorreta. O aluno chegará a essa conclusão se, durante o
cálculo da soma dos lados do triangulo, encontrar dois números a menos.

b) Incorreta. O aluno chegará a essa conclusão se, durante o
cálculo da soma dos lados do triangulo, encontrar um número a menos.

c) Correta. Perímetro = soma dos lados = 6 + 7 + 8 = 21 cm.

d) Incorreta. O aluno chegará a essa conclusão se, durante o
cálculo da soma dos lados do triangulo, encontrar um número a mais.
\end{enumerate}


\section*{Matemática – Módulo 11 – Treino}
\begin{numerate}
\item Habilidade SAEB: Descrever ou esboçar deslocamento de pessoas e/ou de
objetos em representações bidimensionais (mapas, croquis etc.), plantas
de ambientes ou vistas, de acordo com condições dadas.

a) Incorreta. Ao considerar que o enunciado pede o valor de km por
partida, o aluno pode chegar a essa conclusão.

b) Incorreta. Ao considerar a quantidade de rodadas ao invés da
quantidade de km percorridos, o aluno pode chegar a esse valor.

c) Incorreta. Ao deslocar erroneamente uma vírgula para a
esquerda, o aluno pode chegar a esse resultado.

d) Correta. Temos que: 8 km x 38 partidas = 304 km por campeonato.
\item SAEB: Descrever ou esboçar deslocamento de pessoas e/ou de objetos em
representações bidimensionais (mapas, croquis etc.), plantas de
ambientes ou vistas, de acordo com condições dadas.

a) Incorreta. Esse seria o valor caso o aluno realizasse
incorretamente a multiplicação.

b) Incorreta. Esse seria o valor caso o aluno realizasse
incorretamente a multiplicação, adicionando um zero na expressão.

c) Correta. Temos que: $(\frac{50}{x} = \frac{1}{8})$ -> x = 400 km.

d) Incorreta. Esse seria o valor caso o aluno realizasse
incorretamente a multiplicação, adicionando dois zeros na expressão.
\item SAEB: Descrever ou esboçar deslocamento de pessoas e/ou de objetos em
representações bidimensionais (mapas, croquis etc.), plantas de
ambientes ou vistas, de acordo com condições dadas.

a) Incorreta. O aluno pode se confundir em relação aos pontos
cardeais.

b) Incorreta. O aluno pode se confundir em relação aos pontos
cardeais.

c) Incorreta. O aluno pode se confundir em relação aos pontos
cardeais.

d) Correta. Charles deve seguir rumo ao Oeste.
\end{enumerate}


\section*{Matemática – Módulo 12 – Treino}
\begin{numerate}
\item SAEB: Calcular os valores de medidas de tendência central de uma
pesquisa estatística (média aritmética simples, moda ou mediana).

BNCC: EF08MA25 -- Obter os valores de medidas de tendência central de
uma pesquisa estatística (média, moda e mediana) com a compreensão de
seus significados e relacioná-los com a dispersão de dados, indicada
pela amplitude.

a) Incorreta. O aluno chegaria a esse resultado a partir da maior
altura da equipe.

b) Incorreta. O aluno chegaria a esse resultado a partir da menor
altura da equipe.

c) Correta. Somando a altura dos atletas, temos:
$2,01 + 1,99 + 2,00 + 2,02 + 1,98 =10$
Como são 5 atletas, temos:
$10 : 5 = 2$ metros de altura é a média.

d) Incorreta. O aluno chegaria a esse resultado a partir da soma
das alturas da equipe.
\item SAEB: Calcular os valores de medidas de tendência central de uma
pesquisa estatística (média aritmética simples, moda ou mediana).

BNCC: EF08MA25 -- Obter os valores de medidas de tendência central de
uma pesquisa estatística (média, moda e mediana) com a compreensão de
seus significados e relacioná-los com a dispersão de dados, indicada
pela amplitude.

a) Incorreta. O aluno chegaria a esse resultado calculando a moda,
chegando a uma conclusão equivocada.

b) Correta. Considerando que os dois pesos centrais são 46 e 45 kg,
temos que: 46 + 45 = 91 -> 91 : 2 = 45,5.

c) Incorreta. O aluno chegaria a esse resultado calculando a média
aritmética, e não a mediana.

d) Incorreta. O aluno chegará a esse resultado não levando em
consideração que, em casos de conteúdos pares, a mediana deve ser a
média entre os dois valores centrais.
\item SAEB: Calcular os valores de medidas de tendência central de uma
pesquisa estatística (média aritmética simples, moda ou mediana).

BNCC: EF08MA25 -- Obter os valores de medidas de tendência central de
uma pesquisa estatística (média, moda e mediana) com a compreensão de
seus significados e relacioná-los com a dispersão de dados, indicada
pela amplitude.

a) Incorreta. O aluno chegaria a esse resultado considerando
somente o menor tempo gasto.

b) Incorreta. O aluno chegaria a esse resultado considerando
somente o maior tempo gasto.

c) Correta. Somando o tempo que Maria gastou na semana, temos que:
170 minutos / 5 dias = média de 34 minutos por dia.

d) Incorreta. O aluno chegaria a esse resultado considerando
somente a soma dos tempos.
\end{enumerate}


\section*{Matemática – Módulo 13 – Treino}
\begin{numerate}
\item SAEB: Resolver problemas que envolvam volume de prismas retos ou
cilindros retos.

BNCC: EF08MA20 -- Reconhecer a relação entre um litro e um decímetro
cúbico e a relação entre litro e metro cúbico, para resolver problemas
de cálculo de capacidade de recipientes.

a) Correta.
$V = (\Pi) . R^2 .h
V = 3 . 6^2 . 70
V = 3 . 36 . 70
V = 7.560 cm^3$

b) Incorreta. O aluno chegaria a esse valor utilizando a fórmula da
área, e não a fórmula do volume, como se pede no enunciado.

c) Incorreta. O aluno chegaria a esse valor calculando o perímetro
da circunferência do cano, e não o volume, como se pede no enunciado.

d) Incorreta. O aluno chegaria a esse valor ao esquecer o termo
quadrático.
\item SAEB: Resolver problemas que envolvam volume de prismas retos ou
cilindros retos.

BNCC: EF08MA19 -- Resolver e elaborar problemas que envolvam medidas de
área de figuras geométricas, utilizando expressões de cálculo de área
(quadriláteros, triângulos e círculos), em situações como determinar
medida de terrenos.

a) Correta. Para a área do losango, temos que:
$A = (\frac{\text{D\ .\ d}}{2})=
A = (\frac{70\ .\ 50}{2})=
A = (\frac{3500}{2})
A = 1.750 cm^2$

Para a área do retângulo, temos que:
$A = 100 . 70
A = 7.000 cm^2$

$7.000 - 1.750 = 5.250 cm^2$

b) Incorreta. Esse valor é referente apenas ao valor da área do
retângulo do quadro.

c) Incorreta. pois este valor é referente apenas à área que já foi
pintada.

d) Incorreta. O aluno chegaria nesse valor ao não converter o valor
em metros para centímetros.
\item SAEB: Resolver problemas que envolvam volume de prismas retos ou
cilindros retos.

BNCC: EF08MA20 -- Reconhecer a relação entre um litro e um decímetro
cúbico e a relação entre litro e metro cúbico, para resolver problemas
de cálculo de capacidade de recipientes.

a) Incorreta. O aluno poderia chegar a esse valor utilizando
erroneamente a formula da área da base.

b) Incorreta. O aluno poderia chegar a esse valor calculando o
perímetro da base.

c) Incorreta. O aluno poderia chegar a essa conclusão ao não
observar o termo quadrático da fórmula.

d) Correta. pois

V= \Pi). R^2 .h

V= 3,1 . 4^2 . 5

V= 248 m^3
\end{enumerate}


\section*{Matemática – Módulo 14 – Treino}
\begin{numerate}
\item SAEB: Resolver problemas que envolvam a probabilidade de ocorrência de
um resultado em eventos aleatórios equiprováveis independentes ou
dependentes

BNCC: EF08MA22 -- Calcular a probabilidade de eventos, com base na
construção do espaço amostral, utilizando o princípio multiplicativo, e
reconhecer que a soma das probabilidades de todos os elementos do espaço
amostral é igual a 1.

a) Incorreta. pois, ao converter o valor em porcentagem erroneamente, o
aluno chegaria a esse valor.

b) Incorreta. pois esse valor seria o número de calçados a serem
selecionados, e não a porcentagem final.

c) Correta. pois:

(P(E)\frac{n(4)}{n(100)}) = 0,04 ou 4\%

d) Incorreta. O aluno poderia chegar a essa conclusão ao apenas
dividir o número de calçados totais pelo número de pares.
\item SAEB: Resolver problemas que envolvam a probabilidade de ocorrência de
um resultado em eventos aleatórios equiprováveis independentes ou
dependentes

BNCC: EF08MA22 -- Calcular a probabilidade de eventos, com base na
construção do espaço amostral, utilizando o princípio multiplicativo, e
reconhecer que a soma das probabilidades de todos os elementos do espaço
amostral é igual a 1.

a) Incorreta. O aluno pode realizar uma soma ao invés de uma
multiplicação.

b: Incorreta. O aluno chegará a esse valor caso esqueça do elemento
``massas''.

c: Incorreta. O aluno chegará a esse valor caso esqueça do elemento
``saladas''.

d) Correta. pois:

Relendo o enunciado, temos que:

4 . 6 . 5 = 300
\item SAEB: Resolver problemas que envolvam a probabilidade de ocorrência de
um resultado em eventos aleatórios equiprováveis independentes ou
dependentes

BNCC: EF08MA22 -- Calcular a probabilidade de eventos, com base na
construção do espaço amostral, utilizando o princípio multiplicativo, e
reconhecer que a soma das probabilidades de todos os elementos do espaço
amostral é igual a 1.

a) Correta. pois:

(P(E)\frac{n(1)}{n(26)}) = 0,03 ou aproximadamente 3\%

b) Incorreta. O aluno chegaria a essa conclusão ao confundir a
quantidade de letras do alfabeto com a probabilidade do fato acontecer.

c) Incorreta. O aluno chegaria a essa conclusão ao confundir a
quantidade de iniciais com a probabilidade do fato acontecer.

d) Incorreta. pois, ao deslocar a virgula uma casa para a direita, o
aluno chegaria a essa resposta.
\end{enumerate}


\section*{Matemática – Simulado 1}
\begin{numerate}
\item SAEB: Identificar números racionais ou irracionais. BNCC: F09MA02 --
Reconhecer um número irracional como um número real cuja representação
decimal é infinita e não periódica, e estimar a localização de alguns
deles na reta numérica.

a) Correta. pois se trata de um número irracional.

b) Incorreta. pois não se trata de um número racional.

c) Incorreta. pois não se trata de um número inteiro.

d) Incorreta. pois não se trata de um número natural.
\item SAEB: Resolver problemas de adição, subtração, multiplicação, divisão,
potenciação ou radiciação envolvendo número reais, inclusive notação
científica.

BNCC: EF08MA01 -- Efetuar cálculos com potências de expoentes inteiros e
aplicar esse conhecimento na representação de números em notação
científica.

a) Incorreta. pois, ao contar um ``zero'' a mais, o aluno chegaria a
esse resultado.

b) Correta. pois, utilizando a notação científica, temos que:

0,000000000000000000000000000911,

30 casas após a vírgula; logo, é necessário o deslocamento do primeiro
número após o 0:

(0,000000000000000000000000000911 = 9,11 . 10^{-28})

c) Incorreta. pois, ao contar um ``zero'' a menos, o aluno chegaria a
esse resultado.

d) Incorreta. pois, ao contar dois ``zeros'' a menos, o aluno chegaria a
esse resultado.
\item SAEB: Representar frações menores ou maiores que a unidade por meio de
representações pictóricas ou associar frações a representações
pictóricas.

a) Incorreta. O aluno pode chegar à conclusão de que 100 - 99 = 1.

b) Correta. pois, realizando o Cálculo, temos que
( \frac{50.000} {100} = 500)

500 . 99 = 49.500.

c) Incorreta. O aluno pode considerar retirar 99 do valor de 50.000
e chegar ao resultado da alternativa descrita.

d) Incorreta. O aluno pode chegar a esse resultado realizando a
multiplicação por 0,099.

