\documentclass[border=0.2cm]{standalone}
\usepackage{tikz}
\usetikzlibrary{angles,quotes}
\usetikzlibrary{calc}


\begin{document}
\thispagestyle{empty}
\begin{tikzpicture}

% Definindo os vértices do triângulo
    \coordinate[label=above:$A$] (A) at (1,2);
    \coordinate[label=below:$B$] (B) at (0,0);
    \coordinate[label=below:$C$] (C) at (6,0);
    \coordinate[label=below:$D$] (D) at (2,0);
    
    % Desenhando o triângulo
    \draw (A) -- (B) -- (C) -- cycle;
    
    % Desenhando a linha saindo de A até a base (prolongamento de AB)
    \draw (A) -- (D) node[midway, left];
    
    % Marcando os ângulos internos
    \draw pic[draw,angle radius=0.5cm,angle eccentricity=1.5,"$52^\circ$"] {angle=C--B--A};
    \draw pic[draw,angle radius=0.5cm,angle eccentricity=1.5,"$48^\circ$"] {angle=A--C--B};
    \draw pic[draw,angle radius=0.5cm,angle eccentricity=1.5,"$x$"] {angle=C--D--A};

\end{tikzpicture}

\end{document}
