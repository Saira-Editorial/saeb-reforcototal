\documentclass{article}
\usepackage{comment}

\begin{document}

\textit{Razão:} A razão é uma relação de divisão entre duas grandezas de algum
 sistema de medidas. Ela é representada por 

 $$r = \frac{A}{B}$$, onde $A$ e $B$ são
 
 os valores das grandezas.


\textit{Exemplo:} Se temos a razão de 3 para 5, escrevemos $\frac{3}{5} =
 0,6$. 

Além da matemática, a razão também é usada em outras áreas, como velocidade
média, escala, densidade demográfica e densidade de um corpo.


%$$\text{Velocidade média} = \frac{\text{distancia}}{\text{tempo}}$$
%$$\text{Escala} = \frac{\text{desenho}}{\text{real}}$$
%$$\text{Densidade demográfica} = \frac{\text{nº de habitantes}}{\text{área da região}}$$
%$$\text{Densidade  corpo} = \frac{\text{massa}}{\text{volume}}$$

 \medskip \noindent  \textit{Proporção:} A proporção é a igualdade entre duas
  razões. Quando temos os números reais não nulos $A, B, C, D,$ eles formam
  uma proporção se $\frac{A}{B} = \frac{C}{D}$, onde $K$ é a constante de
  proporcionalidade.

\medskip \noindent \textit{Propriedade fundamental das proporções:} Em uma proporção $\frac{A}
 {B} = \frac{C}{D}$, a propriedade fundamental é que $A \times D = B \times
 C$.

\textit{Observação:} Outra propriedade útil para encontrar o valor da
 constante de proporcionalidade é $\frac{A}{B} = \frac{C}{D} = \frac{A \pm C}
 {B \pm D}$.

\textit{Exemplo:} Se temos a proporção $\frac{x}{3} = \frac{6}{15}$, podemos
 encontrar o valor de $x$ fazendo $15x = 18$, o que resulta em $x = 1,2$.

\textit{Exemplo:} Se temos a proporção $\frac{x}{4} = \frac{y}{6} = \frac
 {x + y}{10} = \frac{25}{10} = 2,5$, podemos encontrar os valores de $x$ e $y$.
 Como sabemos que $x + y = 25$, podemos concluir que $x = 10$ e $y = 15$.

\textbf{Grandezas Diretamente Proporcionais (G.D.P):} Duas grandezas são
 diretamente proporcionais quando aumentam ou diminuem na mesma proporção.
 Isso significa que $\frac{A}{B} = K$, onde $K$ é a constante de
 proporcionalidade.

\textbf{Grandezas Inversamente Proporcionais (G.I.P):} Duas grandezas são
 inversamente proporcionais quando uma aumenta e a outra diminui na mesma
 proporção. Isso significa que $A \times B = K$, onde $K$ é a constante de
 proporcionalidade.

\textit{Observação:} Para comparar duas grandezas e classificá-las como G.D.P
 ou G.I.P, precisamos ter uma constante de referência para a comparação.

 \medskip\noindent {Exemplo:} Comprando 15 maçãs, paga-se $R\$7,50$; comprando $8$
maçãs, quanto será pago?

%$$\text{Maçã} - \text{Preço} \rightarrow \text{quanto mais maçãs, maior o preço, logo são G.D.P}$$

A constante é o preço por maçã.

\medskip\noindent {Exemplo:} 

Com $5$ funcionários, uma reforma é feita em $80$ dias.

Caso sejam contratados mais $3$ funcionários, em quantos dias a reforma
será concluída?


$$\text{Funcionários} - \text{Dias}...$$

$\rightarrow$ quanto mais pessoas trabalhando, menos dias serão gastos, logo são G.I.P.

A constante é a reforma.

\medskip \noindent  \textit{Regra de três simples direta e inversa:} A regra
 de três simples direta e inversa é utilizada para encontrar uma grandeza
 desconhecida em uma proporção. A regra de três direta é usada quando as
 grandezas são diretamente proporcionais, enquanto a regra de três inversa é
 usada quando elas são inversamente proporcionais.

Por exemplo, se comprarmos 15 maçãs e pagarmos R\$ 7,50, e quisermos descobrir
quanto pagaremos ao comprar 8 maçãs, podemos utilizar a regra de três.
Montamos a proporção: $\frac{15}{7,5} = \frac{8}{x}$. Multiplicando em cruz,
temos $15x = 60$, e dividindo por 15, encontramos que $x = 4$ reais.


Outro exemplo é quando temos uma reforma que é feita em 80 dias com 5
funcionários, e queremos saber em quantos dias a reforma será feita se
contratarmos mais 3 funcionários. Montamos a proporção: $5 \times 80 =
8 \times y$. Multiplicando em cruz, temos $400 = 8y$, e dividindo por 8,
encontramos que $y = 50$ dias.


\end{document}