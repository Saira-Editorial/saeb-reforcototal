\chapter{Práticas corporais}
\markboth{Módulo 1}{}

\colorsec{Habilidades do SAEB}

\begin{itemize}
\item
  Identificar elementos constitutivos dos esportes, da ginástica e das
  lutas.
\item
  Identificar a importância do respeito ao oponente e às normas de
  segurança na vivência das práticas corporais (jogos, lutas,
  ginásticas, esportes e dança).
\item
  Analisar os esportes e as lutas nas suas manifestações profissional e
  de lazer.
\item
  Avaliar situações de preconceito no contexto das práticas corporais.
\item
  Avaliar meios para superar situações de preconceito no contexto das
  práticas corporais.
\end{itemize}

%\textless{}Arte: a ideia é fazer um mapa mental com os textos a seguir\textgreater{}

\conteudo{As práticas corporais são todas as ações que fazem com que nosso
corpo se movimente - seja para fazer uma simples tarefa do dia, realizar algum
exercício físico ou praticar alguma modalidade esportiva.

\textbf{Esporte}: atividade voltada para competições que têm regras
fixas e que não podem ser alteradas. Outra característica é que, para os esportes, existem confederações que os
fiscalizam, assim como suas competições.
\marginnote{Este módulo tem o objetivo de o aluno identificar as principais
características das práticas corporais reconhecendo as modalidades que
são definidas como jogos, esportes, lutas, ginásticas e danças.\\
Habilidades da BNCC: EF35EF03, EF35EF15.}
\textbf{Dança}: atividade em que o praticante usa o corpo para se
expressar por meio de passos, necessariamente com a presença de música.
Também é usada para diversão e socialização, podendo ser realizada
individualmente, em duplas ou em grupos.

\textbf{Luta}: Prática que foi criada para a defesa pessoal e apresenta modalidades com
diferentes golpes (chutes, socos, técnicas de queda etc.). São práticas
que também transmitem, a seus praticantes, ambasamentos filosóficos que, geralmente, pregam justamente a não violência e a convivência pacífica.

\textbf{Ginástica}: prática que tem o objetivo de fortalecer o corpo por
meio de movimentos acrobáticos ou de exercícios físicos, ajudando na saúde
e na qualidade de vida. Pode-se usar algum material (bolas, bastões, halteres,
máquinas etc.) ou apenas o próprio corpo.

\textbf{Jogos e brincadeiras}: atividades que têm semelhanças com alguns
esportes e são voltadas para o lazer e a diversão. Sua principal
característica é que, no jogo, as regras podem ser modificadas. Podem
existir brincadeiras de diferentes culturas, cada uma carregando traços que são transmitidos de pessoaa pessoa dentro de uma comunidade.}

\coment{É importante que os estudantes saibam
diferenciar e identificar cada tipo de prática corporal. É possível que
a turma entenda que tudo está englobado no conceito de “esporte”, mas muitas práticas corporais não
foram criadas para o aspecto de competição, e sim para diversão ou
saúde.}

\colorsec{Atividades}

\num{1} Escreva exemplos de práticas corporais.

\begin{enumerate}
\item
  Esporte: \reduline{futebol, handebol, vôlei.\hfill}
\item
  Lutas: \reduline{judô, karatê, boxe.\hfill}
\item
  Danças: \reduline{samba, tango, valsa, hip-hop.\hfill}
\item
  Ginásticas: \reduline{pilates, yoga, exercícios de musculação.\hfill}
\item
  Jogos e brincadeiras: \reduline{pega-pega, esconde-esconde.\hfill}
\end{enumerate}

\coment{As respostas são pessoais para cada
estudante, então podem aparecer outras respostas. Esta atividade tem como
objetivo analisar o conhecimento prévio deles e identificar se
sabem diferenciar cada prática corporal.}

\num{2} Qual a principal diferença entre um esporte, como o futebol, e um
  jogo, como a brincadeira do “bobinho”?

\reduline{Os esportes têm regras fixas e são voltados para competições. Já os
jogos (brincadeiras) podem ter suas regras alteradas para promover a
diversão entre os praticantes.\hfill}
\linhas{2}

\num{3} Leia as afirmativas a seguir e marque V para as verdadeiras e F para
  as falsas.

\begin{boxlist}
\boxitem{F} As lutas ajudam as pessoas a brigarem.

\boxitem{V} As danças podem ajudar na socialização, mas também podem ser realizadas individualmente.

\boxitem{F} As ginásticas dificilmente melhoram a qualidade de vida do praticante.

\boxitem{V} As brincadeiras podem ser adaptadas para cada grupo de pessoas.

\boxitem{F} Os esportes têm como função incentivar discussões entre as pessoas.
\end{boxlist}

\coment{Esta atividade deve ser usada para os
estudantes identificarem cada atividade em sua prática. Caso seja necessário,
retome as principais características e os principais exemplos das práticas corporais.}

\num{4} Leia o texto a seguir e complete as lacunas com as palavras que seguem:

\begin{mdframed}[linewidth=2pt,linecolor=salmao,backgroundcolor=salmao!20]
\textbf{Segurança }\hfill
\textbf{Cuidados }\hfill
\textbf{Respeitar }\hfill
\textbf{Regras }\hfill
\textbf{Prática}\hfill
\end{mdframed}

\begin{quote}
Não importa a \reduline{prática} corporal, sempre devemos \reduline{respeitar} as pessoas que
estão participando da atividade, respeitando também as \reduline{regras}. Além
disso, também devemos realizar os devidos \reduline{cuidados} para que a atividade
praticada seja realizada com \reduline{segurança} para ninguém se machucar.
\end{quote}

\coment{Esta atividade vai ajudar o estudante a
entender que, em qualquer prática corporal, devem-se seguir as normas, ou
seja, as regras para promover a segurança e o respeito entre os
participantes.}


\num{5} Imagine a seguinte situação: um amigo da sua turma começou a praticar
  o hip-hop, mas alguns colegas falam que ele deve parar de dançar. O
  motivo é que falam que as danças só podem ser realizadas pelas
  meninas.

Com base nessa situação, podemos afirmar que existe preconceito?
Justifique sua resposta.

\reduline{Sim, pois o preconceito está no pensamento de achar que
algumas modalidades são exclusivas para meninas ou para meninos.\hfill}

\reduline{Esta atividade vai ajudar o estudante a
entender situações preconceituosas que podem surgir nas práticas
corporais.\hfill}

\num{6} Ligue as definições do lazer e da profissão com os exemplos das
  práticas corporais de lutas e de esportes.

\begin{multicols}{2}
Seguir as regras de uma competição

oficial de judô.\medskip

Brincar de esgrima com espadas de papel.\medskip
 
Treinar de 6 a 8 horas por dia para uma

partida importante. \medskip

Jogar bola no parque com os amigos. 

\columnbreak

Profissão no esporte \medskip

Lazer no esporte \medskip

Profissão nas lutas \medskip

Lazer nas lutas
\end{multicols}

%Como vai aparecer o gabarito desta atividade?

\coment{Esta atividade vai ajudar o estudante a
entender que tanto o esporte quanto as lutas podem ser realizados no
âmbito profissional (realizar várias horas de treino, treinar para uma
competição e seguir regras etc.) ou para o lazer (diversão com amigos ou
família, com adaptação de regras e materiais).}

\num{7} Imagine que na sua turma entrou uma colega nova que era de outra escola, mas
  alguns colegas a excluíram durante o recreio e em algumas
  atividades em sala de aula. Sendo assim, escreva uma solução para fazer com
  que, por meio das práticas corporais, essa colega seja incorporada, da forma mais natural possível, à turma.

\reduline{Uma solução é realizar jogos ou brincadeiras com a turma toda
para ajudar na socialização entre os colegas.
Outras respostas podem surgir e devem ser acolhidas.
Esta atividade tem como objetivo o aluno
analisar uma situação preconceituosa (excluir um colega na escola) e
encontrar uma solução para acabar com esse preconceito usando as
práticas corporais.\hfill}

\num{8} Explique como as lutas podem ser compreendidas como riquezas culturais de um povo ou de um grupo.

\reduline{As lutas, ao serem transmitidas, não só representam a transmissão da cultura e da maneira de pensar e agir de um povo como também ajudam na promoção de saberes e de filosofias daquele povo.\hfill}

\reduline{O objetivo desta atividade é levar os alunos a refletirem sobre a importância das lutas, com sua enorme variedade, para as culturas daqueles que as praticam.\hfill}

\num{9} Por que algumas atividades realizadas na escola, além das aulas de educação física, podem ser consideradas práricas corporais?

\reduline{Atividades do recreio e dos intervalos, como corrida, brincadeiras de roda e pega-pega, por exemplo, trabalham o corpo sem que haja, necessariamente, relação com a aula de Educção Física.\hfill}

\reduline{Trata-se de uma boa oportunidade para abrir uma discussão com os alunos sobre como é importante colocar o corpo em movimento, mesmo além das aulas de educação física.\hfill}
\linhas{2}

\num{10} Você pratica regularmente algum esporte, alguma luta, alguma dança ou algum tipo de ginástica? Se sim, conte sobre sua experiência. Se não, cite o que gostaria de praticar.

\reduline{Resposta pessoal. É importante estimular, nas discussões, o hábito de realizar essas atividades, reafirmando sua importância.\hfill}
\linhas{4}


\colorsec{Treino}

\num{1} Leia o texto a seguir.

\begin{quote}
{[}\ldots{}{]} Conhecido entre os Xavantes como “tobdaé”, essa é a brincadeira com a
peteca, palavra de origem Tupi que significa “golpear com as mãos”.
Feita com areia, penas, couro ou palha de milho, na brincadeira o
desafio é tocar na peteca sem deixá-la cair no chão. Uma variação dessa brincadeira é tentar acertar a peteca em outro jogador, que deve deixar a
partida se for acertado.{[}\ldots{}{]}

\fonte{Da Redação. Centro de Referências em Educação Integral. 6 brincadeiras indígenas para divertir crianças e aproximar culturas. Disponível em: \emph{https://educacaointegral.org.br/reportagens/6-brincadeiras-indigenas-para-divertir-criancas-e-aproximar-culturas/}.
Acesso em: 13 fev. 2023.}
\end{quote}

\noindent{}O texto mostra uma atividade para o lazer, pois a prática corporal indígena tem

\begin{minipage}{.5\textwidth}
\begin{escolha}
\item variações de como pode ser realizada.

\item movimentos com golpes de lutas.

\item regras que não podem ser alteradas.

\item materiais oficiais para a prática.
\end{escolha}
\end{minipage}
\sidetext{a) Correta. O texto fala de uma prática corporal voltada ao lazer, ou
seja, uma brincadeira que tem regras a adaptações para a diversão.
b) Incorreta. Por mais que o nome da brincadeira seja “golpear com as
mãos”, tobdaé é uma brincadeira para o lazer e não uma luta.
c) Incorreta. O texto mostra duas variações (regras) que podem ser
modificadas para brincar.
d) Incorreta. O texto mostra como a peteca é feita, mas não são
materiais oficiais semelhantes aos dos esportes e, sim, adaptações.}

\coment{SAEB: Analisar os esportes e as lutas nas suas manifestações
profissionais e de lazer.

BNCC: EF35EF03 - Descrever, por meio de múltiplas linguagens (corporal,
oral, escrita, audiovisual), as brincadeiras e os jogos populares do
Brasil e de matriz indígena e africana, explicando suas características
e a importância desse patrimônio histórico cultural na preservação das
diferentes culturas.}


\num{2} Leia o texto a seguir, que fala sobre a luta indígena.

\begin{quote}

{[}\ldots{}{]} Frente a frente, e abaixados para protegerem as pernas, os oponentes
giram em forma circular e se enfrentam primeiro pelo olhar.
Posteriormente, agarram-se para ver quem consegue levantar o adversário
e levá-lo ao chão, encostando as costas no solo {[}\ldots{}{]}

Como não há um juiz, são os próprios atletas que decidem pela vitória,
derrota ou empate: caso em que se soltam um do outro e nenhum dos dois é
derrubado. A vitória é recompensada pelo reconhecimento e respeito das
comunidades indígenas ao vencedor. {[}\ldots{}{]}

\fonte{Brasil. Ministério dos Povos Indígenas. Huka Huka, a luta corporal do Xingu, contribui para manter viva a
cultura indígena no Mato Grosso.
Disponível em: \emph{https://www.gov.br/funai/pt-br/assuntos/noticias/2022-02/huka-huka-a-luta-corporal-do-xingu-contribui-para-manter-viva-a-cultura-indigena-no-mato-grosso}.
Acesso em: 13 fev. 2023.}
\end{quote}

\noindent{}Com base no texto, podemos afirmar que a huka-huka é uma luta, pois

\begin{escolha}
\item tem regras que podem ser alteradas durante a prática.

\item prioriza o ganhador da luta com prêmios.

\item incentiva as brigas entres os indígenas.

\item promove o respeito entre os lutadores.
\end{escolha}

\coment{SAEB: Identificar a importância do respeito ao oponente e às normas de
segurança na vivência das práticas corporais (jogos, lutas, ginásticas,
esportes e dança).

BNCC: EF35EF15 - Identificar as características das lutas do contexto
comunitário e regional e lutas de matriz indígena e africana,
reconhecendo as diferenças entre lutas e brigas e entre lutas e as
demais práticas corporais.

a) Incorreta. A luta huka-huka tem regras e objetivos que não podem ser
modificados, ou seja, as regras se mantêm as mesmas.
b) Incorreta. O texto mostra que os lutadores ganham respeito e
reconhecimento, não prêmios.
c) Incorreta. A luta huka-hula é uma manifestação corporal que não
promove a briga entres os praticantes, e sim a cultura indígena e o
respeito.
d) Correta. Por meio do texto é possível analisar que a luta não tem
juiz e os próprios lutadores reconhecem a vitória do outro. Portanto, é
um sinal de demostrar respeito com outro.}

\num{3} Leia o texto a seguir.

\begin{quote}
\textbf{Jogos dos Povos Indígenas}

O critério para a participação desses jogos é a força cultural das
etnias, considerando tradições, como a língua, a dança, os rituais, os
cantos, as pinturas corporais, o artesanato e os esportes tradicionais.
{[}\ldots{}{]}

As lutas corporais são realizadas por homens e mulheres e o esporte está
inserido na cultura tradicional dos povos que o praticam: os povos
indígenas Xinguanos, Bakairis os Huka Hukas e os Xavantes, de Mato
Grosso. {[}\ldots{}{]}

{[}\ldots{}{]} Os lutadores se ajoelham girando em círculo anti-horário
frente ao oponente, até se entreolharem e se agarrarem, tentando
levantar o adversário e derrubá-lo ao chão. Os Karajá do Tocantins já
possuem outro estilo, pois os atletas iniciam a luta em pé, se agarrando
pela cintura, até que um consiga derrubar o outro ao chão. {[}\ldots{}{]}

\fonte{Secretaria da Educação. Jogos dos Povos Indígenas. Disponível em: \emph{http://www.educacaofisica.seed.pr.gov.br/modules/conteudo/conteudo.php?conteudo=218}.
Acesso em: 13 fev. 2023.}
\end{quote}

\noindent{}Após ler o texto, podemos concluir que as luta indígenas se tornaram um
esporte, pois

\begin{escolha}
\item são realizadas por diferentes etnias indígenas

\item estão presentes em uma competição oficial.

\item apresentam variações para iniciar a luta.

\item têm a presença de pinturas corporais.
\end{escolha}

\coment{SAEB: Analisar os esportes e as lutas nas suas manifestações
profissional e de lazer.

BNCC: EF35EF15 - Identificar as características das lutas do contexto
comunitário e regional e lutas de matriz indígena e africana,
reconhecendo as diferenças entre lutas e brigas e entre lutas e as
demais práticas corporais.

a) Incorreta. O fato de diferentes etnias indígenas as realizarem não faz com que as
lutas se tornem um esporte, apenas mostra como essas práticas corporais
são importantes para essa cultura.
b) Correta. Uma das principais característica de uma prática corporal
ser um esporte é que ela deve estar presente em uma competição esportiva
oficial, como os Jogos dos Povos Indígenas.
c) Incorreta. O fato de a luta ter diferentes formas de começar (em pé ou
ajoelhado) não é uma definição do esporte, só mostra algumas versões dela.
d) Incorreta. A presença de pinturas corporais é uma característica da própria
cultura indígena, não dos esportes.}

\chapter{Jogos e brincadeiras}
\markboth{Módulo 2}{}

\coment{Este módulo tem o objetivo de levar o aluno a reconhecer a importância dos jogos
e brincadeiras para o desenvolvimento social e físico da pessoa. Também
vai fazer com que ele conheça novas culturas, como a indígena e a
africana. O aluno também vai perceber que muitos jogos pré-depsortivos
servem como uma iniciação esportiva para alguns esportes.\\
Habilidades da BNCC: EF35EF01, EF35EF06.}

\colorsec{Habilidades do SAEB}

\begin{itemize}
\item
  Identificar as brincadeiras e os jogos populares como patrimônio
  histórico-cultural.
\item
  Valorizar o patrimônio histórico representado pelas brincadeiras e
  jogos, com ênfase naqueles de origem indígena e africana.
\item
  Analisar o protagonismo do trabalho coletivo na vivência dos jogos
  populares e dos esportes.
\end{itemize}

\conteudo{Os jogos e brincadeiras que realizamos em casa ou na escola trazem
vários benefícios para as pessoas. Tais atividades podem, por exemplo,
ser usadas para promover a participação de todos, ou seja, evitar que
algum colega seja excluído. Outro benefício é ajudar as pessoas a
melhorar suas habilidades motoras, como saltar, correr, rolar, arremessar.
Afinal, trata-se de habilidades essenciais para que as pessoas possam realizar as tarefas do dia a dia.

Além disso, muitas brincadeiras são utilizadas como uma iniciação
esportiva. Um exemplo é a brincadeira de “bobinho” do futebol, uma
atividade em que as pessoas treinam os passes do futebol.

Por fim, muitas brincadeiras são consideradas como um patrimônio
cultural do país. O motivo é que muitas dessas brincadeiras, assim como jogos ou brinquedos,
foram criados na cultura africana ou na cultura indígena, ambas fortemente
presentes no nosso país. Portanto, conhecer uma brincadeira faz com que a
gente conheça um pouco de uma cultura diferente.}

%https://br.freepik.com/vetores-gratis/criancas-felizes-brincando-de-amarelinha-no-playground\_24558957.htm\#page=2\&query=brincadeira\&position=10\&from\_view=search\&track=sph
\begin{figure}[htpb!]
\includegraphics[width=\textwidth]{./imgs/img7.jpg}
\end{figure}

\colorsec{Atividades}

\num{1} A seguir, aparecem algumas brincadeiras populares do Brasil. Escreva a cultura representada por cada brincadeira. Se necessário, peça ajuda ao seu professor.

\begin{enumerate}
\item
  Cabo de guerra: \reduline{Cultura indígena.\hfill}
\item
  Peteca: \reduline{Cultura indígena.\hfill}
\item
  Mancala: \reduline{Cultura africana.\hfill}
\item
  Terra-Mar: \reduline{Cultura africana.\hfill}
\item
  Arco e flecha: \reduline{Cultura indígena.\hfill}
\item
  Jogo da onça: \reduline{Cultura indígena.\hfill}
\item
  Mamba: \reduline{Cultura africana.\hfill}
\end{enumerate}

\coment{Nesta atividade é muito provável que os alunos precisem de atenção e de direcionamento. Ajude-os, inclusive, propondo, se possível, pesquisa em sala de aula sobre as brincadeiras mencionadas. Caso seja necessário, relembre os estudantes
de como as brincadeiras apresentadas são realizadas. Esta atividade tem
como objetivo levar o estudante a identificar a origem de algumas brincadeiras.}

\num{2} Você costuma participar de alguma dessas brincadeiras? Se sim, relate a experiência. Se não, escolha uma que lhe pareça mais divertida e justifique sua escolha.

\reduline{Resposta pessoal\hfill}
\linhas{3}

\num{3} \begingroup
%https://br.freepik.com/vetores-gratis/um-menino-jovem-jogando-voleibol\_5284614.htm\#query=v\%C3\%B4lei\%20kids\&position=10\&from\_view=search\&track=ais
\begin{wrapfigure}{r}{.13\textwidth}
\includegraphics[width=.12\textwidth]{./imgs/img8.jpg}
\end{wrapfigure}

Um jogo popular em algumas escolas é o “3 cortes”. Nesse jogo os
  participantes devem ficar passando a bola entre eles usando as mãos e,
  no terceiro passe (toque), qualquer um pode dar uma cortada para tentar
  acertar alguém.

Depois de ler o texto, reflita: a brincadeira apresenta características de qual
esporte? Justifique sua resposta.

\endgroup\bigskip

\reduline{A brincadeira “3 cortes” é um jogo pré-depsortivo do
voleibol, pelo fato de que, nela, os participantes realizam o
toque e a cortada, dois fundamentos desse esporte.\hfill}

\reduline{Os alunos podem escrever outros esportes
que usam a mão para lançar uma bola, como o basquete ou o handebol, mas a
brincadeira apresenta fundamentos do vôlei, que são a cortada e o toque.\hfill}

\num{4} Quais são as brincadeiras mais populares na sua escola?

\reduline{Resposta circunstancial. Ajude os alunos a pensarem em brincadeiras que envolvam mais claramente práticas corporais.\hfill}
\linhas{2}

\num{5}
\begin{wrapfigure}{r}{.2\textwidth}
\includegraphics[width=.2\textwidth]{./imgs/img9.jpg}
\end{wrapfigure}
Uma brincadeira comum é a peteca, na qual o objetivo é acetar esse objeto, lançando-o para cima,
  para que não encoste no chão. Existem vários tipos de petecas, mas uma certeza: a de que o objeto foi
  criado pelos povos indígenas. Sendo assim, circule os materiais que
  esses povos podem usar para confeccionar a peteca.

\begin{multicols}{2}
\red{Borracha}

\red{Palha} 

\red{Folhas}

\blue{Penas de animais}

\blue{Plástico}

\blue{Jornal}
\end{multicols}

%marcar como gabarito para o professor: palha, folhas, penas de animais.

\coment{Por meio dessa atividade o aluno vai
entender que muitos objetos usados na atualidade, como a peteca, podem
ser criados com elementos encontrados na natureza. }

\num{6} Você brinca de peteca? Se sim, usa as mesmas regras mencionadas na atividade anterior? Descreva as regras detalhadamente.

\reduline{Resposta circunstancial.\hfill}
\linhas{4}

\num{7} Leia as afirmativas a seguir e marque V para as verdadeiras e F para
  as falas.

\begin{boxlist}
\boxitem{V} As brincadeiras podem fazer com que as pessoas sejam incluídas.

\boxitem{F} Uma vantagem das brincadeiras é fazer com que as pessoas trabalhem sozinhas.

\boxitem{V} Muitos jogos podem ser usados para aprender um novo esporte.

\boxitem{V} O trabalho em equipe pode ser usado em qualquer brincadeira.

\boxitem{V} O pega-pega é uma brincadeira que ajuda a nossa habilidade motora de correr.
\end{boxlist}

\coment{Esta atividade tem o objetivo de o aluno
identificar outras vantagens de praticar uma brincadeira ou jogo
pré-depsortivo para promover a socialização e o trabalho em equipe.}

\num{8} A seguir, é apresentada uma ilustração de bolinhas de gude. Trata-se de uma
  brincadeira popular no Brasil, na qual, dependendo da região do país, a
  maneira de brincar pode mudar. Até mesmo o nome da brincadeira pode variar. Por exemplo, no Paraná o brinquedo é chamado de “bola de búrica”, enquanto em Alagoas
  é conhecido como “ximbra”.

%https://br.freepik.com/vetores-gratis/marbles\_797023.htm\#query=marbles\%20ball\&position=2\&from\_view=search\&track=ais
\begin{wrapfigure}{r}{.2\textwidth}
\includegraphics[width=.2\textwidth]{./imgs/img10.jpg}
\end{wrapfigure}

\noindent{}Depois da leitura do texto, punte a afirmação correta sobre a
brincadeira apresentada.

\begin{escolha}
\item
  As características das bolinhas de gude podem variar para cada região
  do país.
\item
  Somente no Sul do país as bolinhas de gude são conhecidas.
\item
  As regras da brincadeira com bolinha de gude não podem mudar.
\item
  Dependendo do lugar as brincadeiras com bolinhas de gude apenas mudam
  de nome.
\end{escolha}

\coment{A resposta a ser pintada é a primeira. Por
meio dessa atividade o aluno vai analisar criticamente que muitas
brincadeiras podem mudar dependendo da região do país e, por isso, muitas
atividades são populares no Brasil por serem consideradas um
patrimônio cultural do país, estando presentes em muitas regiões.}

\num{9} Você acha possível fazer brinquedos com materiais recicláveis? Qual é a importância disso para o meio ambiente?

\reduline{É, sim, possível fazer brinquedos com materiais recicláveis. A importância disso para o meio ambiente é a diminuição de resíduos e a proteção do planeta.\hfill}
\linhas{3}

\num{10} Faça um desenho que represente a importância dos jogos e das brincadeiras.

\begin{mdframed}[linewidth=2pt,linecolor=salmao]
\coment{Resposta pessoal.}
\vspace{10cm}
\end{mdframed}

\colorsec{Treino}

\num{1} Leia sobre o tag rugby.
\begin{quote}
  No tag rugby, os jogadores usam um cinto com
  velcro em que está presa uma bandeirola de tecido. Quando essa bandeirinha é retirada
  por um adversário, este é obrigado passar a bola, evitando-se lances violentos.

\fonte{Fonte de pesquisa: Juliana Ribeiro. UmComo. Tag rugby: o que é e regras. Disponível em: \emph{
https://esportes.umcomo.com.br/artigo/tag-rugby-o-que-e-e-regras-30566.html}.
Acesso em: 24 mar. 2023.}
\end{quote}

\noindent{}Com base no texto, pode-se afirmar que a atividade praticada citada tem o objetivo de

\begin{escolha}
\item desenvolver novos equipamentos esportivos.

\item diminuir brigas e conflitos nos esportes.

\item criar um novo esporte.

\item incentivar a prática de um espore.
\end{escolha}

\coment{SAEB: Analisar o protagonismo do trabalho coletivo na vivência dos jogos
populares e dos esportes.

BNCC: EF35EF06 - Diferenciar os conceitos de jogo e esporte, identificando as características que os constituem na contemporaneidade
e suas manifestações (profissional e comunitária/lazer).

a) Incorreta. O texto mostra que o praticante deve usar um cinto com
velcro com bandeirinha, mas o tag-rugby serve para popularizar o rugby e
não criar novos equipamentos.
b) Incorreta. Por mais que o tag-rugby evite o contato físico, esse jogo
pré-desportivo tem o propósito de incentivar a pratica do rugby e não
acabar com os conflitos nos outros esportes.
c) Incorreta. O tag-rugby não é um esporte e sim uma brincadeira do
rugby.
d) Correta. O texto mostra algumas variações do rugby para torná-lo mais
lúdico para as pessoas, com o propósito de popularizar esse esporte.}

\num{2} Leia o texto.

\begin{quote}
O mancala é um jogo de tabuleiro {[}\ldots{}{]} mais antigo do mundo. É um
recurso lúdico utilizado pela Educação do Acre, em atividade
de contraturno. {[}\ldots{}{]}

O ato de semear, germinação das sementes na terra, desenvolvimento e
colheita são etapas no tabuleiro. Atualmente, é jogado em diversas
partes do mundo e possui mais de 200 variações. “Mancala” significa
mover.

\fonte{Da Redação. Governo do Acre. Mancala: Cultura africana apresentada de forma lúdica.
Disponível em: \emph{
https://agencia.ac.gov.br/mancala-cultura-africana-apresentada-de-forma-ludica/}.
Acesso em: 14 fev. 2023.}
\end{quote}

\noindent{}Por meio da brincadeira apresentada, podemos

\begin{escolha}
\item adaptar a cultura para a nossa realidade.

\item aprender novas línguas.

\item conhecer tradições de diferentes locais do mundo.

\item estudar uma característica da cultura local.
\end{escolha}

\coment{SAEB: Valorizar o patrimônio histórico representado pelas brincadeiras e
jogos, com ênfase naqueles de origem indígena e africana.

BNCC: EF35EF01 - Experimentar e fruir brincadeiras e jogos populares do
Brasil e do mundo, incluindo aqueles de matriz indígena e africana, e
recriá-los, valorizando a importância desse patrimônio histórico-cultural.

a) Incorreta. São as regras do jogo mancala que podem ser
alteradas, não a cultura africana.
b) Incorreta. Por mais que o texto mostre o significado da palavra
“mancala”, o jogo de tabuleiro não ensina novas palavras, e sim
apresenta um costume da cultura africana.
c) Correta. Por meio do texto podemos perceber que o mancala é um jogo
que representa a força da cultura africana; ou seja, por meio do jogo podemos conhecer diferentes tradições de outras culturas.
d) Incorreta. O texto mostra que o mancala é usado na escola, mas não
para que os alunos estudem um conteúdo relacionado à cultura local
(Acre), e sim sobre a cultura africana.}

\num{3} Leia um trecho de notícia.

\begin{quote}
Soltar pipa, jogar bola, pular amarelinha e brincar de pique-esconde
foram algumas das brincadeiras que se tornaram Patrimônio Cultural do
Povo Carioca {[}\ldots{}{]}

{[}\ldots{}{]} o “Poder Executivo, através de seus órgãos competentes, apoiará as iniciativas que visem à valorização e divulgação desta cultura, bem
como oferecerá áreas específicas para que a prática dessas brincadeiras
possa continuar ocorrendo na Cidade” {[}\ldots{}{]}

\fonte{G1. Brincadeiras tradicionais viram Patrimônio Cultural do Povo Carioca;
veja a lista. Disponível em: \emph{
https://g1.globo.com/rj/rio-de-janeiro/noticia/2021/11/05/brincadeiras-tradicionais-viram-patrimonio-cultural-do-povo-carioca-veja-a-lista.ghtml}.
Acesso em: 14 fev. 2023.}
\end{quote}

\noindent{}Depois da leitura do texto, as brincadeiras citadas se tornaram um
patrimônio para que elas possam

\begin{escolha}
\item ser realizas em alguns lugares do país.

\item ter novas regras e variações.

\item incentivar a venda de materiais para brincar.

\item evitar que as pessoas esqueçam essas atividades.
\end{escolha}

\coment{SAEB: Identificar as brincadeiras e os jogos populares como patrimônio
histórico-cultural.

BNCC: EF35EF01 - Experimentar e fruir brincadeiras e jogos populares do
Brasil e do mundo, incluindo aqueles de matriz indígena e africana, e
recriá-los, valorizando a importância desse patrimônio histórico
cultural.

a) Incorreta. O texto fala que os locais específicos para brincar são para
incentivar a pratica de algumas brincadeiras, não para restringir as
brincadeiras tradicionais.
b) Incorreta. O objetivo é preservar a cultura local por meio das
brincadeiras, não de criar novas regras.
c) Incorreta. O texto não cita que os locais voltados para as
brincadeiras vão incentivar o comércio, e sim incentivar as pessoas a
realizarem algumas brincadeiras tradicionais.
d) Correta. O texto mostra que algumas brincadeiras se tornarem um
patrimônio cultural e vai haver locais para brincar com o objetivo de
as pessoas continuarem praticando essas brincadeiras e preservando a cultura
local.}

\chapter{Danças indígenas e africanas}
\markboth{Módulo 3}{}

\coment{Este módulo tem o objetivo de o estudante relembrar as principias
características das danças, especialmente as de origens africanas e
indígenas, além de identificar os elementos constitutivos da dança (ritmo,
espaço, gesto).\\

Habilidades da BNCC: EF35EF09, EF35EF10, EF35EF11.}

\colorsec{Habilidades do SAEB}

\begin{itemize}
\item
  Valorizar o patrimônio histórico representado pelas danças populares,
  com ênfase naquelas de matriz indígena e africana.
\item
  Comparar os elementos constitutivos de danças populares do Brasil e do
  mundo com aqueles de danças de matrizes indígena e africana.
\end{itemize}


\conteudo{As danças são práticas corporais que utilizam os movimentos do corpo para
as pessoas se expressarem e se comunicarem. Mesmo existindo diferentes tipos de
dança, todas elas têm três elementos comuns, que são:

\begin{itemize}
\item
  Ritmo: são as batidas fortes da música para que o dançarino possa
  realizar os movimentos de maneira coordenada e harmoniosa.
\item
  Espaço: é o trajeto que o corpo realiza ao dançar, dando a
  liberdade de a pessoa se movimentar para onde ela quiser. É próprio para
  cada um.
\item
  Gesto: são os passos de dança, que podem conter saltos, giros,
  movimentos acrobáticos. Os gestos podem ser padronizados, criados pelo
  próprio dançarino, e podem ser realizados em grupos ou
  individualmente.
\end{itemize}

É comum as pessoas conhecerem danças de outros lugares, como o tango, a
valsa etc., mas no Brasil existem muitas danças que surgiram aqui mesmo,
como o samba, o forró, entre outras. Uma curiosidade é que, no país,
existem muitas danças de origem africana e indígena.

Também devemos saber que as danças trazem vários benefícios, como cuidar
da saúde e interagir com outras pessoas.}

%https://br.freepik.com/vetores-gratis/pacote-de-silhuetas-de-danca-de-silhuetas-de-dancarinas_23885929.htm\#query=dan\%C3\%A7a\&position=1\&from_view=search\&track=sph
\begin{figure}[htpb!]
\includegraphics[width=\textwidth]{./imgs/img11.png}
\end{figure}

\colorsec{Atividades}

\num{1} Relacione as danças da primeira coluna com suas respectivas culturas de
  origem, que estão na segunda coluna.

\begin{multicols}{2}
(\rosa{2}) Toré
    
(\rosa{2}) Kuarup
    
(\rosa{1}) Samba

(\rosa{1}) Cateretê

(\rosa{2}) Maracatu

(\rosa{1}) Maculelê

\columnbreak

(1) Cultura africana\medskip

(2) Cultura indígena
\end{multicols}

\coment{Esta atividade tem a finalidade de levar o aluno a
identificar e relembrar as danças que são das matrizes indígena ou
africana.}

\num{2} Você tem conhecimentos de alguma das danças mencionadas na atividade anterior? Se sim, como é sua relação com essa dança? Se não, qual delas gostaria de conhecer melhor?

\reduline{Resposta pessoal.\hfill}
\linhas{3}

\num{3} Na sua escola as pessoas costumam dançar na hora do intervalo ou em festas específicas? Relate esse costume.

\reduline{Resposta pessoal.\hfill}
\linhas{6}

\num{4} Complete o texto a seguir com as palavras que estão faltando.

\begin{quote}
Não importa o tipo de dança, se é indígena, europeia ou africana, todas
elas têm algumas semelhanças!

Sabe aquelas batidas fortes que escutamos em uma música? Isso é o \reduline{ritmo}.
Ele é algo muito importante para que o dançarino consiga realizar o \reduline{gesto} de
uma determinada dança. Por fim, o praticante também deve prestar atenção ao
\reduline{espaço} para que ele possa se movimentar na melhor maneira possível.
\end{quote}

\coment{A atividade serve como uma fixação para que
o estudante consiga identificar e diferenciar os três elementos
constitutivos da dança.}

\num{5} Você já foi a um musical? Nesse tipo de espetáculo, acontecem encenações, como em um teatro, juntamente com canto e dança. Imagine um cenário para um espetáculo musical com alguma das danças mencionadas anteriormente. Desenhe um cenário que combine com esse espetáculo e, em seguida, explique a associação entre o tema proposto e seu desenho.

\begin{mdframed}[linewidth=2pt,linecolor=salmao]
\coment{A associação deve ser coerente, mas não se deve exigir muito conhecimento dos alunos em relação à dança escolhida.}
\vspace{10cm}
\end{mdframed}

\num{6} Observe a imagem e assinale V para as afirmativas verdadeiras e F para as falsas.

\begin{figure}[htpb!]
\includegraphics[width=\textwidth]{./imgs/img12.jpg}
\end{figure}

\begin{boxlist}
\boxitem{V} A ilustração mostra o maculelê.

\boxitem{F} As pessoas da imagem estão realizando uma luta.

\boxitem{V} Uma característica é o uso de bastões de madeira.

\boxitem{F} A prática corporal apresentada é de origem indígena.
\end{boxlist}

\coment{Essa atividade vai ajudar o estudante a
reconhecer as principais características de uma dança de origem
africana.}

\begingroup
%https://br.freepik.com/vetores-gratis/plano-de-fundo-do-carnaval-brasileiro\_3872754.htm\#query=samba\&position=8\&from\_view=search\&track=sph
\begin{wrapfigure}{r}{.25\textwidth}
\includegraphics[width=.25\textwidth]{./imgs/img13.jpg}
\end{wrapfigure}

\num{7} Sabe qual é a dança mais popular no Brasil? Se pensou no samba,\\ você
  acertou! É reconhecida internacionalmente como uma\\ dança
  afro-brasileira e está presente em algumas festas\\ populares no país.
  Além disso, as pessoas se organizam em um\\ grande círculo para dançar e
  tocar alguns instrumentos\\ musicais, como o cavaquinho e o pandeiro.


\noindent{}Com base no texto, responda às questões a seguir.

\endgroup

\begin{escolha}
\item Por que o samba é conhecido como dança afro-brasileira?

\reduline{O samba foi criado pelos negros escravizados que chegaram ao Brasil e
que tinha, desde sempre, influências da cultura brasileira.\hfill}

\item Em qual festa o samba é realizado de forma mais frequente?

\reduline{No carnaval.\hfill}

\item Qual o nome da dança de samba em que as pessoas formam um círculo?

\reduline{Samba de roda ou roda de samba.\hfill}
\end{escolha}

\coment{Por meio desta atividade é possível o
estudante perceber como o samba está presente na cultura brasileira e
relembrar algumas características dessa dança.}

\num{8} Por que uma dança é considerada uma prática corporal?

\reduline{Uma dança é considerada uma prática corporal poque, para que ela seja praticada, trabalha-se com o corpo.\hfill}
\linhas{3}

\num{9} Use o espaço a seguir para representar uma dança indígena ou uma dança
  africana por meio de uma colagem. Procure representar como são realizados
  os passos de dança, as vestimentas usadas e escreva o nome da dança
  que foi representada.

\begin{mdframed}[linewidth=2pt,linecolor=salmao]
\coment{As representações elaborados pelos alunos vão
ajudá-los a fixar algumas características da dança escolhida, bem como refletirem sobre sua origem.}
\vspace{6cm}
\end{mdframed}

\num{10} Você acha que danças de tradição africana devem ser praticadas apenas por descendentes de pessoas de países africanos? Por quê?

\reduline{Todas as respostas devem ser acolhidas, mas é importante mostrar aos alunos que, quanto mais pessoas praticarem uma dança, melhor para a divulgação dessa cultura e dessa dança.\hfill}
\linhas{3}

\colorsec{Treino}

\num{1} Leia o texto.
\begin{quote}
  {[}\ldots{}{]} A dança do \emph{toré} apresenta variações de ritmos e toadas
  dependendo de cada povo. O \emph{maracá} -- chocalho indígena feito de
  uma cabaça seca, sem miolo, na qual se colocam pedras ou sementes --
  marca o tom das pisadas e os índios dançam, em geral, ao ar livre e em
  círculos. O ritual do \emph{toré} é considerado o símbolo maior de
  resistência e união entre os índios
  do Nordeste brasileiro. {[}\ldots{}{]}

\fonte{Lúcia Gaspar. Fundação Joaquim Nabuco. Danças indígenas do Brasil. Disponível em: \emph{
http://basilio.fundaj.gov.br/pesquisaescolar/index.php?option=com\_content\&view=article\&id=839:dancas-indigenas-do-brasil\&catid=39:letra-d}.
Acesso em: 14 fev. 2023.}
\end{quote}

\noindent{}Assim como qualquer dança, a prática corporal indígena citada pode ser assim classificada por

\begin{escolha}
\item variar para cada povo indígena que existe no Brasil.

\item ter instrumentos musicais para ditar o ritmo da dança.

\item ser realizada em locais abertos.

\item promover a interação entres os indígenas.
\end{escolha}

\coment{SAEB: Comparar os elementos constitutivos de danças populares do Brasil
e do mundo com aqueles de danças de matrizes indígena e africana.

BNCC: EF35EF10 - Comparar e identificar os elementos constitutivos
comuns e diferentes (ritmo, espaço, gestos) em danças populares do
Brasil e do mundo e danças de matriz indígena e africana.

a) Incorreta. O fato de o toré ser realizado por diferentes povos não quer
dizer que é uma dança, e sim uma prática difundida na cultura indígena.
b) Correta. O instrumento musical (maracá) serve para marcar o ritmo na música e na dança; ou seja, é um elemento constitutivo da dança.
c) Incorreta. Porque a dança ser realizado ao ar livre não é uma
característica própria das danças.
d) Incorreta. Qualquer atividade promove a interação entre as
pessoas e não somente as danças.}

\num{2} Leia com atenção infomações sobre o jongo.

\begin{quote}
Considerado, desde 2005, patrimônio cultural do Brasil pelo Iphan, o
  jongo conta agora com um centro cultural de 2 mil metros quadrados aos
  pés do Morro da Serrinha, em Madureira, zona norte do Rio. {[}\ldots{}{]}

O jongo chegou ao Brasil com os escravizados africanos de origem bantu,
vindos do Congo e de Angola, permanecendo presente entre aqueles que
trabalhavam nas lavouras de café e cana-de-açúcar no vale do Rio
Paraíba, entre São Paulo e Minas Gerais. Os proprietários das fazendas
permitiam que seus escravos dançassem jongo nos dias dos santos {[}\ldots{}{]}

\fonte{Larissa Altoé. MultiRio. Jongo, expressão da cultura afro-brasileira. Disponível em: \emph{
https://www.multirio.rj.gov.br/index.php/reportagens/8637-jongo-expressao-da-cultura-afro-brasileira}.
Acesso em: 14 fev. 2023.}
\end{quote}

\noindent{}Com base no texto, a dança apresentada é de origem africana, pois ela

\begin{escolha}
\item foi criada pelos negros escravizados.

\item surgiu no Sudeste do Brasil.

\item era praticada pelos fazendeiros de café.

\item estava relacionada com eventos religiosos.
\end{escolha}

\coment{SAEB: Valorizar o patrimônio histórico representado pelas danças
populares, com ênfase naquelas de matriz indígena e africana.

BNCC: EF35EF09 - Experimentar, recriar e fruir danças populares
do Brasil e do mundo e danças de matriz indígena e africana, valorizando
e respeitando os diferentes sentidos e significados dessas danças em
suas culturas de origem.

a) Correta. O texto mostra que os negros escravizados, do Congo e
de Angola, desenvolveram o jongo e, por isso, ele tem elementos
culturais africanos.
b) Incorreta. O jongo tem influência da cultura africana do Congo
e de Angola, não da cultura brasileira.
c) Incorreta. Eram os negros escravizados que trabalhavam nas
fazendas que realizavam a dança do jongo.
d) Incorreta. Por mais que a dança fosse praticada em eventos religiosos,
isso foi criado no Brasil e não nos países africanos. Além disso, o
evento religioso não definia que a dança é de origem africana.}
 
\num{3} Leia sobre o samba de roda.
\begin{quote}
O Samba de Roda no Recôncavo Baiano foi reconhecido como Patrimônio Cutural Imaterial da Humanidade.

Essa prática, hoje em dia, envolve e reúne tradições que se transmitem de geração a geração, desde os africanos escravizados até os dias de hoje. Algumas das tradiçõs envolvidas nesse samba de roda são: culto aos orixás e caboclos, jogo da capoeira e “comida de azeite”. 

O que também surpreende nesse samba é como ele, vindo da África, se mesclou com a cultura trazida pelos portugueses (como o uso da viola e do pandeiro), além, claro, do uso da língua portuguesa com toda a sua potência.


\fonte{Brasil. Ministério da cultura. Samba de Roda do Recôncavo Baiano completa 18 anos como Patrimônio Cultural do Brasil. Disponível em: \emph{
https://www.gov.br/iphan/pt-br/assuntos/noticias/samba-de-roda-no-reconcavo-baiano-completa-18-anos-como-patrimonio-cultural-do-brasil}. Acesso em: 25 mar. 2023.}
\end{quote}

\noindent{}Depois da leitura do texto, é possível entender que a pessoa que pratica o samba vai

\begin{escolha}
\item realizar uma luta africana.

\item reconhecer costumes alimentares de origem europeia.

\item entender como uma atividade se torna um patrimônio.

\item conhecer diferentes culturais por meio da dança.
\end{escolha}

\coment{SAEB: Valorizar o patrimônio histórico representado pelas danças
populares, com ênfase naquelas de matriz indígena e africana.

BNCC: EF35EF11 - Formular e utilizar estratégias para a execução de
elementos constitutivos das danças populares do Brasil e do mundo, e das
danças de matriz indígena e africana.

a) Incorreta. O samba é voltado para a dança e não para praticar a capoeira (luta africana).
b) Incorreta. O samba é de origem africana e não europeia. Apenas
alguns instrumentos portugueses são usados, mas isso não faz com que o
praticante conheça a cultura alimentar europeia.
c) Incorreta. O samba não tem o objetivo de o praticante entender
como a Unesco reconhece uma atividade como patrimônio cultural.
d) Correta. O texto mostra alguns elementos culturais presentes
no samba e, por conta disso, o praticante dessa dança vai poder conhecer
alguns costumes e tradições da cultura africana.}

%Os simulados não devem aparecer aqui. Eles aparecem ao final do volume, na ordem em que os componentes foram apresentados nele.


\chapter{Simulado 1}

\num{1} Observe a imagem.
  \begin{figure}[htpb!]
\includegraphics[width=\textwidth]{./imgs/img14.jpg}
\end{figure}
%Disponível em: https://br.freepik.com/fotos-gratis/lutadores-de-karate-no-campeonato-de-luta-de-tatami\_30182444.htm\#query=jud\%C3\%B4\&position=42\&from\_view=search\&track=sph. Acesso em: 14 fev. 2023.

\noindent{}Observando os dois competidores, podemos perceber que eles estão
demostrando respeito um ao outro. O motivo é que eles estão

\begin{escolha}
\item realizando uma saudação antes de lutar.

\item usando um kimono branco.

\item praticando a luta em uma competição.

\item evitando usar golpes específicos das lutas.
\end{escolha}

\coment{SAEB: Identificar elementos constitutivos dos esportes, da ginástica e
das lutas.

BNCC: EF35EF06 - Diferenciar os conceitos de jogo e esporte,
identificando as características que os constituem na contemporaneidade
e suas manifestações (profissional e comunitária/lazer).

a) Correta. A saudação nas artes marciais ocidentais consiste em
inclinar o tronco para a frente e mostrar o devido respeito ao adversário
antes de lutar.
b) Incorreta. A vestimenta utilizada não é voltada para o
respeito e para simbolizar a paz.
c) Incorreta. A competição não é um evento que promove o
respeito, e sim a competição.
d) Incorreta. Em competições, os atletas devem usar técnicas da
luta para competir.}

\num{2} Leia sobre os jogos de oposição.
\begin{quote}
  Os Jogos de Oposição {[}\ldots{}{]} têm como característica o ato de
  confrontação que acontece entre duplas, trios ou até mesmo em grupos.
  Seus objetivos são vencer o adversário, impor-se fisicamente ao outro,
  respeitar as regras e acima de tudo assegurar a segurança do colega
  durante as atividades.

Durante a aplicação dos Jogos de Oposição, precisamos levar em
consideração alguns critérios de segurança para que não ocorram
acidentes. {[}\ldots{}{]}

\fonte{Paraná. Secretaria da Educação. Jogos de Oposição. Disponível em: \emph{
http://www.educacaofisica.seed.pr.gov.br/modules/conteudo/conteudo.php?conteudo=413}.
Acesso em: 14 fev. 2023.}
\end{quote}

\noindent{}Com base no texto, entende-se que as atividades práticas citadas são voltadas para
lutas, considerando que os participantes devem

\begin{escolha}
\item tentar para ganhar de qualquer maneira.

\item tomar os devidos cuidados para ninguém se machucar.

\item realizar a atividade individualmente.

\item modificar as regras do jogo.
\end{escolha}

\coment{SAEB: Identificar a importância do respeito ao oponente e às normas de
segurança na vivência das práticas corporais (jogos, lutas, ginásticas,
esportes e dança).

BNCC: EF35EF01 - Experimentar e fruir brincadeiras e jogos populares do
Brasil e do mundo, incluindo aqueles de matriz indígena e africana, e
recriá-los, valorizando a importância desse patrimônio histórico
cultural.

a) Incorreta. O participante pode tentar a vitória, desde que
respeite as regras e as normas de segurança.
b) Correta. O próprio texto cita que os participantes devem
seguir as regras e normas de segurança para preservar a integridade
física do outro.
c) Incorreta. O texto mostra que os jogos de oposição são
realizados em duplas ou em grupos.
d) Incorreta. Os participantes devem respeitar as regras, não modificá-las.}

\num{3} Leia uma notícia sobre um projeto de lei.
\begin{quote}
  A Câmara analisa o Projeto de Lei 6933/10, {[}\ldots{}{]} que regulamenta a
  profissão de instrutor de artes marciais. A proposta inclui na
  categoria os profissionais {[}\ldots{}{]} que possuírem certificado de
  instrutor, monitor, professor ou 1° dan (graduação de arte marcial)
  emitido por uma federação ou associação registrada.

O certificado será concedido a quem comprovar a prática do esporte por
pelo menos dois anos e meio. Segundo o projeto, as federações e
associações criarão o código de ética dos profissionais e fiscalizarão o
cumprimento do período mínimo para obtenção do certificado. {[}\ldots{}{]}

\fonte{Câmara dos deputados. Educação, cultura e esportes. Proposta regulamenta profissão de instrutor de artes marciais. Disponível em: \emph{
https://www.camara.leg.br/noticias/143647-PROPOSTA-REGULAMENTA-PROFISSAO-DE-INSTRUTOR-DE-ARTES-MARCIAIS}.
Acesso em: 14 fev. 2023.}
\end{quote}

\noindent{}Após a leitura do texto, fica claro que o projeto de lei tem o objetivo de

\begin{escolha}
\item formar novos instrutores de lutas.

\item criar novas federações esportivas de lutas.

\item regulamentar a profissão de professores de lutas.

\item incentivar a prática de lutas.
\end{escolha}

\coment{SAEB: Analisar os esportes e as lutas nas suas manifestações
profissional e de lazer.

BNCC: EF35EF06 - Diferenciar os conceitos de jogo e esporte,
identificando as características que os constituem na contemporaneidade
e suas manifestações (profissional e comunitária/lazer).

a) Incorreta. O projeto de lei é para regulamentar a profissão dos
instrutores, não ter novos profissionais.
b) Incorreta. O objetivo é regulamentar os professores de lutas, não criar novas entidades esportivas.
c) Correta. No trecho “\ldots{} regulamenta a profissão de instrutor
de artes marciais\ldots{}”, é possível analisar que o projeto de lei é
profissionalizar e regulamentar os instrutores de lutas.
d) Incorreta. O objetivo é regulamentar os instrutores, não
fazer com que mais pessoas pratiquem lutas.}

\chapter{Simulado 2}

\num{1} Leia um trecho de artigo científico.
\begin{quote}
  {[}\ldots{}{]} no desenvolvimento da dança são encontrados vários
  descaminhos, entre eles estão os fatores que apontam para a exclusão
  da dança nos planejamentos de educação física {[}\ldots{}{]}

{[}\ldots{}{]} perguntamos se acham que exista algum preconceito dos alunos a
respeito do conteúdo dança e {[}\ldots{}{]} pedimos para dizer quais os
preconceitos encontrados, e 100\% deles responderam que o maior
preconceito está ligado ao gênero por parte dos meninos.

\fonte{Vinicius Giacomini de Castro, Diogo Santos Silva e Marli das Graças Júlio. EFDEPORTES. O preconceito da dança nas escolas.
Revista Digital. Buenos Aires, Año 15, Nº 150, Noviembre de 2010.
Disponível em: \emph{
https://www.efdeportes.com/efd150/o-preconceito-da-danca-nas-escolas.htm}.
Acesso em: 14 fev. 2023.}
\end{quote}

\noindent{}Com base no texto, o pensamento estereotipado na dança é achar que é um(a)

\begin{escolha}
\item modalidade desconhecida por parte dos alunos.

\item esporte evitado na escola.

\item atividade de que os homens não podem participar.

\item prática corporal voltada para mulheres.
\end{escolha}

\coment{SAEB: Avaliar situações de preconceito no contexto das práticas
corporais.

BNCC: EF35EF09 - Experimentar, recriar e fruir danças populares do
Brasil e do mundo e danças de matriz indígena e africana, valorizando e
respeitando os diferentes sentidos e significados dessas danças em suas
culturas de origem.

a) Incorreta. Os alunos conhecem algumas danças, mas alguns preferem não praticar essa modalidade.
b) Incorreta. A dança pode ser, sim, ensinada na escola, mas é uma
prática que tem alguns preconceitos.
c) Incorreta. Os homens podem participar, mas existem alguns
pensamentos equivocados de que a dança é exclusiva para as mulheres.
d) Correta. Com no trecho “\ldots{} 100\% deles responderam que o maior
preconceito está ligado ao gênero por parte dos meninos\ldots{}”, é possível
analisar que os meninos acreditam que as danças são para apenas um
gênero, ou seja, para o gênero feminino.}

\num{2} Leia o texto.
\begin{quote}
  Acontece na próxima sexta-feira, {[}\ldots{}{]} no Ginásio Municipal de
  Esportes Domingos Angelino Régis, no Centro de Navegantes, um evento
  direcionado aos alunos das 8ª Séries da Rede Municipal de Ensino, que
  tem por objetivo despertar nos estudantes a importância do esporte
  como mecanismo de motivação, superação e combate ao preconceito. O
  evento também vai contar com a participação de atletas do paradesporto
  e da Apae de Navegantes.

{[}\ldots{}{]} no local haverá uma apresentação das equipes de Basquete e
Handebol do Clube Roda Solta {[}\ldots{}{]}.

\fonte{Prefeitura de Navegantes. Alunos participam de evento sobre motivação e superação através do esporte. Disponível em: \emph{
https://www.navegantes.sc.gov.br/noticia/9274/alunos-participam-de-evento-sobre-motivacao-e-superacao-atraves-do-esporte}.
Acesso em: 15 fev. 2023.}
\end{quote}

\noindent{}O evento citado, para os alunos, serviu para que eles

\begin{escolha}
\item praticassem novos esportes.

\item promovessem a conscientização dos paratletas.

\item entendessem os benefícios dos esportes.

\item ajudassem na organização do evento.
\end{escolha}

\coment{SAEB: Avaliar meios para superar situações de preconceito no contexto
das práticas corporais.

BNCC: EF35EF06 - Diferenciar os conceitos de jogo e esporte,
identificando as características que os constituem na contemporaneidade
e suas manifestações (profissional e comunitária/lazer).

a) Incorreta. O evento serviu para combater o preconceito, não
para apresentar novos esportes.
b) Incorreta. Foram os paratletas que deram palestras no vento
para os alunos falando sobre a inclusão no esporte.
c) Correta. No trecho “\ldots{} esporte como mecanismo de motivação,
superação e combate ao preconceito\ldots{}”, é possível analisar as vantagens
e benefícios que os esportes podem proporcionar.
d) Incorreta. Não foram os alunos que organizaram o evento e sim
os paratletas e a prefeitura local.}

\num{3} Leia sobre as cantigas de roda.
\begin{quote}
  {[}\ldots{}{]} caso das cantigas de roda que, historicamente fazem parte
  das tradicionais brincadeiras infantis {[}\ldots{}{]}

{[}\ldots{}{]} Ficou claro que a cantiga de roda é inserida em sala de aula
para promover o lúdico para a criança. {[}\ldots{}{]} Nem todas as crianças
sabem cantar muitas músicas que são tidas como tradicionais. Isso porque
o envolvimento das mesmas com tecnologias pode as estar afastando de
tradições ricas e importantes como são as cantigas de roda. {[}\ldots{}{]}

\fonte{UFG. Patrimônio, direitos culturais e cidadania. As cantigas de roda como manifestações do patrimônio cultural: o papel
da escola na perpetuação dessa cultura. Disponível em: \emph{
https://publica.ciar.ufg.br/ebooks/eipdcc-propostas-pratica-acoesdialogicas/artigos/artigo34.html}.
Acesso em: 15 fev. 2023.}
\end{quote}

\noindent{}Com base no texto, podemos perceber que a brincadeira tradicional citada

\begin{escolha}
\item está sendo esquecida por parte dos alunos.

\item vem ganhando popularidade por causa da tecnologia.

\item apresenta algumas desvantagens para os estudantes.

\item aparece como uma atividade pouco usada na escola.
\end{escolha}

\coment{SAEB: Identificar as brincadeiras e os jogos populares como patrimônio
histórico-cultural.

BNCC: EF35EF01 - Experimentar e fruir brincadeiras e jogos populares do
Brasil e do mundo, incluindo aqueles de matriz indígena e africana, e
recriá-los, valorizando a importância desse patrimônio histórico
cultural.

a) Correta. O texto mostra que algumas crianças não sabiam cantar
algumas cantigas tradicionais.
b) Incorreta. É a tecnologia que está afastando as crianças das
cantigas populares.
c) Incorreta. As cantigas trazem muitas vantagens e benefícios
aos alunos por serem algo lúdico.
d) Incorreta. As cantigas são atividades sempre usadas no ambiente escolar.}

\chapter{Simulado 3}

\num{1} 
\begin{quote}
  {[}\ldots{}{]} Existem muitos jeitos de brincar, mas o objetivo é sempre desfrutar o
  momento e a companhia dos amigos. Além disso, os jogos ajudam a
  desenvolver habilidades que serão importantes ao longo da vida.
  Brincar é também uma maneira de aprender!

Os índios possuem muitos jogos e brincadeiras. Alguns são bastante
conhecidos por vários povos indígenas, {[}\ldots{}{]} como a peteca e a perna
de pau. {[}\ldots{}{]}

\fonte{Mirim Povos Indígenas Brasil. Brincadeiras. Disponível em: \emph{
https://mirim.org/pt-br/como-vivem/brincadeiras}. Acesso em: 16 fev.
2023.}
\end{quote}

\noindent{}Segundo o texto, algumas brincadeiras indígenas

\begin{escolha}
\item são parecidas com algumas brincadeiras tradicionais também não indígenas.

\item são realizadas exclusivamente pelos indígenas, sem influências ou compartilhamentos.

\item são padronizadas para os povos indígenas.

\item são praticadas de maneira individual.
\end{escolha}

\coment{SAEB: Valorizar o patrimônio histórico representado pelas brincadeiras e
jogos, com ênfase naqueles de origem indígena e africana.

BNCC: EF35EF01 - Experimentar e fruir brincadeiras e jogos
populares do Brasil e do mundo, incluindo aqueles de matriz indígena e
africana, e recriá-los, valorizando a importância desse patrimônio
histórico cultural.

a) Correta. As brincadeiras citadas (peteca e perna de pau) são tradicionais de origem indígenas que muitas pessoas
conhecem.
b) Incorreta. As brincadeiras de origem indígena também são
realizadas por outros povos e culturas.
c) Incorreta. No trecho “\ldots{}Existem muitos jeitos de
brincar\ldots{}”, é possível analisar que existem variações nas brincadeiras.
d) Incorreta. No trecho “\ldots{}o objetivo é sempre desfrutar o
momento e a companhia dos amigos\ldots{}”, podemos compreender que as
brincadeiras são realizadas em grupo para promover a socialização.}

\num{2} Leia um trecho de notícia.
\begin{quote}
  {[}\ldots{}{]} as crianças aprendem a respeitar o próximo, a ceder, a
  ganhar e a perder e constroem o senso de coletividade. Isso vai
  refletir no convívio com a família, na escola e, futuramente, até no
  trabalho.

\fonte{G1. Bem estar. Esporte coletivo promove o respeito ao próximo e o trabalho em equipe.
Disponível em: \emph{
https://g1.globo.com/bemestar/noticia/2016/08/esporte-coletivo-promove-o-respeito-ao-proximo-e-o-senso-de-coletividade.html}.
Acesso em: 16 fev. 2023.}
\end{quote}

\noindent{}Depois da leitura, é possível perceber que o texto fala sobre

\begin{escolha}
\item os jogos pré-depsortivos.

\item os esportes competitivos.

\item as modalidades olímpicas.

\item as atividades escolares.
\end{escolha}

\coment{SAEB: Analisar o protagonismo do trabalho coletivo na vivência dos jogos
populares e dos esportes.

BNCC: EF35EF06 - Diferenciar os conceitos de jogo e esporte,
identificando as características que os constituem na contemporaneidade
e suas manifestações (profissional e comunitária/lazer).

a) Correta. Os jogos são atividades voltadas para a diversão e a socialização.
b) Incorreta. Os esportes competitivos visam apenas a competições e vitórias.
c) Incorreta. Assim como os esportes, as modalidades olímpicas
visam ao alto rendimento e às competições.
d) Incorreta. O texto fala sobre jogos pré-depsortivos e não
sobre atividades escolares.}

\num{3} Observe a imagem.
  \begin{figure}[htpb!]
\includegraphics[width=\textwidth]{./imgs/img15.jpg}
\end{figure}
%Disponível em: https://br.freepik.com/fotos-gratis/dancarinas-nigerianas-de-tiro-medio\_16130625.htm\#\&position=34\&from\_view=collections. Acesso em: 16 fev. 2023.

\noindent{}Após a análise, pode-se perceber que é uma dança, pois

\begin{escolha}
\item as pessoas estão dançando ao ar livre.

\item as pessoas estão realizando uma pratica corporal coletiva.

\item as pessoas estão com vestimentas e pinturas corporais da dança.

\item as pessoas estão se movimentando no ritmo do batuque do instrumento
musical.
\end{escolha}

\coment{SAEB: Comparar os elementos constitutivos de danças populares do Brasil
e do mundo com aqueles de danças de matrizes indígena e africana.

BNCC: EF35EF10 - Comparar e identificar os elementos constitutivos
comuns e diferentes (ritmo, espaço, gestos) em danças populares do
Brasil e do mundo e danças de matriz indígena e africana.

a) Incorreta. As danças podem ser realizadas em espaço aberto ou
fechado e isso não define se uma atividade é uma dança oficial ou não.
b) Incorreta. A dança pode ser realizada individualmente ou em
duplas. O fato de a dança ser em grupo não define que uma prática seja
considerada uma dança.
c) Incorreta. As vestimentas e pinturas não são próprias da
dança, já que em esportes, ginásticas e lutas podem aparecer esses elementos.
d) Correta. A pessoa se movimentado na batida da música
(instrumento musical) vai estar realizando o elemento constitutivo do
ritmo e do gesto da dança.}

\chapter{Simulado 4}

\num{1} Leia o texto.
\begin{quote}
  {[}\ldots{}{]} Semba: é uma dança de salão angolana urbana. Dançada em pares, com
  passadas distintas dos cavalheiros, seguidas pelas damas em passos
  totalmente largos, onde o malabarismo dos cavalheiros conta muito para
  o nível de improvisação. O Semba caracteriza-se como uma dança de
  passadas. Não é ritual nem guerreira, mas de divertimento,
  principalmente em festas. {[}\ldots{}{]}

\fonte{Secretaria da Educação do Estado do Paraná. Danças Africanas. Disponível
em: \emph{
http://www.educacaofisica.seed.pr.gov.br/modules/conteudo/conteudo.php?conteudo=62}.
Acesso em: 16 fev. 2023.}
\end{quote}

\noindent{}Depois de ler o texto, nota-se que se trata de uma prática corporal que tem semelhança
com o samba. O motivo é que as duas danças são

\begin{escolha}
\item realizadas no carnaval.

\item originárias da cultura africana.

\item atividades competitivas.

\item praticadas em eventos religiosos
\end{escolha}

\coment{SAEB: Valorizar o patrimônio histórico representado pelas danças
populares, com ênfase naquelas de matriz indígena e africana

BNCC: EF35EF11 - Formular e utilizar estratégias para a execução de
elementos constitutivos das danças populares do Brasil e do mundo, e das
danças de matriz indígena e africana.

a) Incorreta. Apenas o samba, entre ambas, é realizado no carnaval.
b) Correta. O samba se originou do semba, ou seja, as duas
surgiram com base nas influências culturais da África.
c) Incorreta. As duas danças não são voltadas para as
competições.
d) Incorreta. O samba e o semba não são danças religiosas.}

\num{2} A seguir, aparece um trecho de notícia. Leia-a.
\begin{quote}
  O Governo do Paraná vai levar as artes marciais para dentro das
  escolas estaduais, oferecendo treinamentos no contraturno às aulas
  convencionais {[}\ldots{}{]}

A ideia, explicou o governador, é começar o projeto-piloto no segundo semestre {[}\ldots{}{]} “Gosto muito do esporte, sou
um praticante. As artes marciais ensinam a filosofia do respeito, a
obedecer a hierarquia, a ser uma pessoa do bem” {[}\ldots{}{]}

Além da introdução de artes marciais nas escolas estaduais, há outra
iniciativa que diz respeito ao Japs Combat, espécie de Jogos Abertos do
Paraná, voltado apenas para as artes marciais. {[}\ldots{}{]}

\fonte{Paraná vai levar as artes marciais para dentro das escolas. Agência
Estadual de Notícias. Disponível em: \emph{
https://www.aen.pr.gov.br/Noticia/Parana-vai-levar-artes-marciais-para-dentro-das-escolas}.
Acesso em: 16 fev. 2023.}
\end{quote}

\noindent{}Depois de ler a notícia, nota-se que o objetivo das artes marciais é

\begin{escolha}
\item ensinar valores éticos aos alunos.

\item incentivar competições.

\item formar novos atletas.

\item aumentar os conflitos entre os alunos.
\end{escolha}

\coment{SAEB: Identificar a importância
do respeito ao oponente e às normas de segurança na vivência das
práticas corporais (jogos, lutas, ginásticas, esportes e dança).

BNCC: EF35EF15 - Identificar as características das lutas do contexto
comunitário e regional e lutas de matriz indígena e africana,
reconhecendo as diferenças entre lutas e brigas e entre lutas e as
demais práticas corporais.

a) Correta. O trecho “\ldots{} As artes marciais ensinam a filosofia
do respeito\ldots{}” mostra que as lutas ensinam o valor ético de respeitar
o outro.
b) Incorreta. O objetivo das lutas é ensinar valores éticos e
morais, não criar novas competições.
c) Incorreta. O projeto apresentado serve para transformar os alunos
em cidadãos do bem.
d) Incorreta. É justamente o oposto: as lutas
evitam e amenizam as brigas entre as pessoas.}

\num{3}
\begin{quote}
  {[}\ldots{}{]} Por volta de 1880, jogadores de um clube inglês improvisaram
  um novo jogo por causa do mau tempo. Sobre uma mesa de sinuca, com
  livros como raquetes, um barbante como rede e uma bola de tênis
  normal, surgiram as primeiras raquetadas do tênis de mesa. {[}\ldots{}{]}


\fonte{Prefeitura de Lençóis Paulista. Tênis de mesa. Disponível em: \emph{
https://apl2.lencoispaulista.sp.gov.br/esp-jomi-2022/Modalidade/Detalhe?id=21}.
Acesso em: 16 fev. 2023.}
\end{quote}

\noindent{}Com base no texto, percebe-se que o tênis de mesa, antes de ser um esporte olímpico, era

\begin{escolha}
\item uma atividade adaptada do tênis de campo.

\item uma modalidade esportiva.

\item um treinamento para usar as raquetes.

\item uma prática corporal para competição.
\end{escolha}

\coment{SAEB: Analisar os esportes e as
lutas nas suas manifestações profissional e de lazer.

BNCC: EF35EF06 - Diferenciar os conceitos de jogo e esporte,
identificando as características que os constituem na contemporaneidade
e suas manifestações (profissional e comunitária/lazer).

a) Correta. No começo as pessoas adaptaram alguns materiais para
simular as rebatidas na bola realizadas no tênis de campo.
b) Incorreta. O texto cita que antes o tênis de mesa era uma
brincadeira.
c) Incorreta. Não era um treinamento, e sim uma brincadeira.
d) Incorreta. O tênis de mesa antigamente era voltado para o
lazer.}

