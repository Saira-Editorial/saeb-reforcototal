\chapter{Simulado 3}

\num{1} Durante a aula de matemática a professora colocou na lousa a
seguinte decomposição de um número:

4 x 1.000 + 3 x 100 + 3 x 10 + 5 x 1

Muito rapidamente, Artur levantou a mão e disse que sabia qual era o
número. Qual o número representado por essa decomposição?

\begin{minipage}{.5\textwidth}
\begin{escolha}
\item
  4.035
\item
  4.335
\item
  5.034
\item
  5.304
\end{escolha}
\end{minipage}
\sidetext{SAEB: Compor ou decompor números naturais de até 6 ordens na
forma aditiva, ou em suas ordens, ou em adições e multiplicações.

BNCC: EF05MA01 - Ler, escrever e ordenar números naturais até a ordem das 
centenas de milhar com compreensão das principais características do 
sistema de numeração decimal. 
EF05MA10 - Concluir, por meio de investigações, que a relação de 
igualdade existente entre dois membros permanece ao adicionar, subtrair, 
multiplicar ou dividir cada um desses membros por um mesmo número, para 
construir a noção de equivalência.
EF05MA11 - Resolver e elaborar problemas cuja conversão em sentença 
matemática seja uma igualdade com uma operação em que um dos termos é 
desconhecido.

a) Incorreta. O total é de 4.335. 
b) Correta. 4 x 1.000 + 3 x 100 + 3 x 10 + 5 x 1 = 
4.000 + 300 + 30 + 5 = 4.335.
c) Incorreta. O total é de 4.335. 
d) Incorreta. O total é de 4.335. 
BNCC: EF05MA01, EF05MA10, EF05MA11}

\num{2} Ricardo deseja escrever o maior número possível
utilizando os algarismos 1, 2, 4, 5 e 7, sem repeti-los nenhuma vez. Qual
o maior número que ele irá escrever?

\begin{minipage}{.5\textwidth}
\begin{escolha}
\item
  Setecentos e cinquenta mil e quatrocentos e vinte um
\item
  Setenta e cinco mil e quatrocentos e vinte um
\item
  Quarenta e cinco mil e duzentos e cinquenta e sete
\item
  Dezessete mil e quinhetos e quarenta e cinco
\end{escolha}
\end{minipage}
\sidetext{SAEB: Escrever números racionais (naturais de até 6 ordens,
representação fracionária ou decimal finita até a ordem dos milésimos)
em sua representação por algarismos ou em língua materna ou associar o
registro numérico ao registro em língua materna.

BNCC: EF05MA01 - Ler, escrever e ordenar números naturais até a ordem das 
centenas de milhar com compreensão das principais características do 
sistema de numeração decimal. 
EF05MA10 - Concluir, por meio de investigações, que a relação de 
igualdade existente entre dois membros permanece ao adicionar, subtrair, 
multiplicar ou dividir cada um desses membros por um mesmo número, para 
construir a noção de equivalência.
EF05MA11 - Resolver e elaborar problemas cuja conversão em sentença 
matemática seja uma igualdade com uma operação em que um dos termos é 
desconhecido.


a) Incorreta. O número correto é 75.4271.
b) Correta. Basta colocar os algarismos dados em ordem decrescente para
formar o número desejado.
c) Incorreta. O número correto é 75.4271.
d) Incorreta. O número correto é 75.4271.}

\num{3} A seguinte conta foi colocada no quadro durante uma aula de
matemática.

\begin{figure}[htpb!]
\includegraphics[width=\textwidth]{../ilustracoes/MAT5/SAEB_5ANO_MAT_figura121.png}
\end{figure}
%ATENÇÃO: essa imagem NÃO É CLARA. A impressão do leitor é que a primeira conta é 396 X 54. Mas essa não é a conta proposta. 396 X 4 = 1.584 (número da terceira linha) e 396 X 5 = 1.980 (número da quarta linha). A visualização proposta não é clara para qualquer leitor; será muito mais confusa para um aluno em formação. Fiz um gabarito simples, porque acredito que é preciso reformular essa imagem.   

Qual o número devemos colocar no lugar dos quadradinhos para que a conta
fique correta?

\begin{minipage}{.5\textwidth}
\begin{escolha}
\item
  2
\item
  6
\item
  7
\item
  8
\end{escolha}
\end{minipage}
\sidetext{SAEB: Calcular o resultado de multiplicações ou divisões
envolvendo números naturais de até 6 ordens.

a) Incorreta. O número que completa adequadamente os quadradinhos é 8.
b) Incorreta. O número que completa adequadamente os quadradinhos é 8.
c) Incorreta. O número que completa adequadamente os quadradinhos é 8.
d) Correta. Basta fazer as contas para identificar que o número que
completa adequadamente os quadradinhos é o 8.}

\num{4} Alex estava observando a sequência numérica (3; 9; 27; 81; 243;
729). Pode-se dizer que para encontrarmos um elemento qualquer da
sequência devemos a um termo anterior:

\begin{minipage}{.5\textwidth}
\begin{escolha}
\item
  Somar 6
\item
  Dividir por 3
\item
  Multiplicar por 3
\item
  Somar 9
\end{escolha}
\end{minipage}
\sidetext{SAEB: Inferir o padrão ou a regularidade de uma sequência de
números naturais ordenados, objetos ou figuras.

a) Incorreta. Observando a sequência apresentava, verifica-se que,
para descobrir um termo, basta multiplicar por 3 seu antecessor. 
b) Incorreta. Observando a sequência apresentava, verifica-se que,
para descobrir um termo, basta multiplicar por 3 seu antecessor. 
c) Correta. 3 X 3 = 9; 9 X 3 = 27; 27 X 3 = 81; 81 X 3 = 243; 
243 X 3 = 729. 
d) Incorreta. Observando a sequência apresentava, verifica-se que,
para descobrir um termo, basta multiplicar por 3 seu antecessor.}

\num{5} O zoológico da cidade em que Fabiana mora abre às 9 horas da manhã
e fica aberto apenas 8 horas e meia por dia. Em que horário as atividades 
do zoológico são encerradas, sabendo-se que ele não fecha no horário do
almoço?

\begin{minipage}{.5\textwidth}
\begin{escolha}
\item
  16h30
\item
  17h30
\item
  17h45
\item
  18h30
\end{escolha}
\end{minipage}
\sidetext{SAEB: Determinar o horário de início, o horário de término ou
a duração de um acontecimento.

a) Incorreta. O horário de encerramento das atividades é 17h30.
b) Correta. 9 + 8,5 = 17,5 = 17 horas e 30 minutos.
c) Incorreta. O horário de encerramento das atividades é 17h30.
d) Incorreta. O horário de encerramento das atividades é 17h30.}

\num{6} O programa preferido de Marquinhos na internet começa pontualmente
às 14h55 e termina exatamente às 15h34.
Qual a duração do programa favorito de Marquinhos?

\begin{minipage}{.5\textwidth}
\begin{escolha}
\item
  39 minutos
\item
  45 minutos
\item
  50 minutos
\item
  1 hora e 20 minutos
\end{escolha}
\end{minipage}
\sidetext{SAEB: Determinar o horário de início, o horário de término ou
a duração de um acontecimento.

a) Correta. A diferença entre 15h34 e 14h55 é de 39 minutos.
b) Incorreta. O programa tem duração de 39 minutos.
c) Incorreta. O programa tem duração de 39 minutos.
d) Incorreta. O programa tem duração de 39 minutos.}

\num{7} Marina quer colocar um carpete de madeira no quarto de sua única
filha. Para isso representou o quarto da menina na malha quadriculada
abaixo, na qual a parte escura corresponde ao carpete de madeira que será
colocado.

\begin{figure}[htpb!]
\includegraphics[width=\textwidth]{../ilustracoes/MAT5/SAEB_5ANO_MAT_figura122.png}
\end{figure}

Como cada quadradinho possui 1 metro quadrado de área, qual a área total
de carpete de madeira que ela terá que encomendar para colocar no quarto
da filha sem que falte nenhum pedaço e também não sobre material?

%PAULO: Como usar a notação de metros quadrados aqui? 

\begin{minipage}{.5\textwidth}
\begin{escolha}
\item
  12 m²
\item
  17 m²
\item
  18 m²
\item
  20 m²
\end{escolha}
\end{minipage}
\sidetext{ 

a) Incorreta. O total é de 18 metros quadrados.
b) Incorreta. O total é de 18 metros quadrados.
c) Correta. Contando o número de quadradinhos que representa o tapete
chega-se a 18 (17 quadradinhos completos somados a duas metades), que
correspondem a 18 metros quadrados de carpete.
d) Incorreta. O total é de 18 metros quadrados.

Habilidade do SAEB: Resolver problemas que envolvam área de figuras planas.}

\num{8} Um cartão é retirado de forma aleatória de um conjunto de 50
cartões numerados de 1 a 50. Qual a probabilidade de que, no cartão
retirado, esteja escrito um número entre 20 e 40?

\begin{minipage}{.5\textwidth}
\begin{escolha}
\item
  15\%
\item
  38\%
\item
  56\%
\item
  74\%
\end{escolha}
\end{minipage}
\sidetext{SAEB: Determinar a probabilidade de ocorrência de um
resultado em eventos aleatórios, quando todos os resultados possíveis
têm a mesma chance de ocorrer (equiprováveis).

a) Incorreta. A probabilidade é de 38\%.
b) Correta. O total de números é 50. Os números de interesse são 19, 
pois o número 20 e o 40 não entram na contagem. Dessa forma, calcula-se
a probabilidade da seguinte maneira: 19/50 = 38/100 = 38\%.
c) Incorreta. A probabilidade é de 38\%.
d) Incorreta. A probabilidade é de 38\%.
BNCC: EF05MA22, EF05MA23}

\num{9} Um universítário recebeu seu extrato de notas:

\begin{longtable}[]{@{}ll@{}}
\toprule
Disciplinas & Notas\tabularnewline
\midrule
\endhead
II & 8,0\tabularnewline
III & 6,0\tabularnewline
IV & 5,0\tabularnewline
V & 7,5\tabularnewline
\bottomrule
\end{longtable}

Sabendo-se que a média para passar em cada disciplina é 6,00, a
disciplina em que ele foi reprovado é a:

\begin{minipage}{.5\textwidth}
\begin{escolha}
\item
  II
\item
  III
\item
  IV
\item
  V
\end{escolha}
\end{minipage}
\sidetext{SAEB: Argumentar ou analisar argumentações/conclusões com
base em dados apresentados em tabelas (simples ou de dupla entrada) ou
gráficos (barras simples ou agrupadas, colunas simples ou agrupadas,
pictóricos ou de linhas).

BNCC: EF05MA24 - Interpretar dados estatísticos apresentados em textos, 
tabelas e gráficos (colunas ou linhas), referentes a outras áreas do 
conhecimento ou a outros contextos, como saúde e trânsito, e produzir 
textos com o objetivo de sintetizar conclusões.

a) Incorreta. Na disciplina II, o universitário obteve média 8,0,
suficiente para aprovação. 
b) Incorreta. Na disciplina III, o universitário obteve média 6,0,
suficiente para aprovação. 
c) Correta. A única disciplina em que o universitário apresentou 
nota inferior à média foi a IV, com nota 5,00.
d) Incorreta. Na disciplina V, o universitário obteve média 7,5,
suficiente para aprovação.}

\num{10} Em uma seletiva para a fase final da prova de 100 metros livres
de natação, os 8 atletas que disputaram obtiveram os seguintes tempos:

\begin{tabular}{l|c|c|c|c|c|c|c|c}
\hline
\textbf{Raia} & 1 & 2 & 3 & 4 & 5 & 6 & 7 & 8 \\ \hline
\textbf{\begin{tabular}[c]{@{}l@{}}Tempo\\ (segundo)\end{tabular}} & 20,90 & 20,90 & 20,50 & 20,80 & 20,60 & 20,60 & 20,90 & 20,96 \\ \hline
\end{tabular}

Sabendo-se que apenas os três mais velozes passam para a próxima fase,
podemos afirmar que os atletas classificados foram os das raias:

\begin{minipage}{.5\textwidth}
\begin{escolha}
\item
  1, 3 e 8
\item
  3, 5 e 6
\item
  1, 7 e 8
\item
  5, 6 e 7
\end{escolha}
\end{minipage}
\sidetext{SAEB: Argumentar ou analisar
argumentações/conclusões com base em dados apresentados em tabelas
(simples ou de dupla entrada) ou gráficos (barras simples ou agrupadas,
colunas simples ou agrupadas, pictóricos ou de linhas). 

BNCC: EF05MA24 - Interpretar dados estatísticos apresentados em textos, 
tabelas e gráficos (colunas ou linhas), referentes a outras áreas do 
conhecimento ou a outros contextos, como saúde e trânsito, e produzir 
textos com o objetivo de sintetizar conclusões.

a) Incorreta. Os nadadores das raias 1 e 8 foram mais lentos do que
os das raias 3, 5 e 6. 
b) Correta. Os nadadores mais velozes são os que fizeram a prova em
menos tempo, isto é, os que nadaram nas raias 3 (20,50), 5 (20,60) 
e 6 (20,60).
c) Incorreta. Os nadadores das raias 1, 7 e 8 foram mais lentos do que
os das raias 3, 5 e 6.
d) Incorreta. Os nadadores das raias 5 e 7 foram mais lentos do que
os das raias 3, 5 e 6.}

\num{11} Um prêmio de R\$ 600,00 será dividido da seguinte forma
entre os três primeiros colocados:

\begin{itemize}
\item
  O primeiro receberá ½ do valor;
\item
  O segundo receberá 1/3 do prêmio;
\item
  O terceiro receberá o restante do prêmio.
\end{itemize}

Sendo assim, pode-se afirmar que o terceiro colocado receberá:

%Alterei o enunciado para a formulação acima ser usada de maneira integral. 

\begin{minipage}{.5\textwidth}
\begin{escolha}
\item
  R\$ 300,00
\item
  R\$ 200,00
\item
  R\$ 100,00
\item
  R\$ 50,00
\end{escolha}
\end{minipage}
\sidetext{SAEB: Resolver problemas que envolvam fração como resultado
de uma divisão (quociente).

BNCC: EF05MA03 - Identificar e representar frações (menores e maiores que 
a unidade), associando-as ao resultado de uma divisão ou à ideia de parte 
de um todo, utilizando a reta numérica como recurso.
EF05MA04 - Identificar frações equivalentes.
EF05MA06 - Associar as representações 10\%, 25\%, 50\%, 75\% e 100\% 
respectivamente à décima parte, quarta parte, metade, três quartos e um 
inteiro, para calcular porcentagens, utilizando estratégias pessoais, 
cálculo mental e calculadora, em contextos de educação financeira, entre 
outros.

a) Incorreta. R\$ 300,00 corresponde a metade do valor do prêmio. Esse é
o valor ganho pelo primeiro colocado.
b) Incorreta. R\$ 200,00 corresponde a um terço do valor do prêmio. Esse é
o valor ganho pelo segundo colocado.
c) Correta. O primeiro colocado receberá metade de R\$ 600,00, 
o que corresponde a R\$ 300,00. O segundo ganhará um terço de
R\$ 600,00, isto é, R\$ 200,00. O restante, que será destinado ao
terceiro, é igual a R\$ 600,00 - (R\$ 300,00 + R\$ 200,00) = R\$ 100,00.
d) Incorreta. Nenhum dos três primeiros colocados ganhou apenas 
R\$ 50,00.}

\num{12} Durante um treino de futebol Camilo acertou 8 pênaltis dos 14
que bateu. Pode-se afirmar que a razão do número de pênaltis que ele
errou em relação ao total de pênaltis que ele bateu é:

\begin{minipage}{.5\textwidth}
\begin{escolha}
\item
  4/7
\item
  3/7
\item
  3/4
\item
  4/3
\end{escolha}
\end{minipage}
\sidetext{SAEB: Identificar frações equivalentes.

BNCC: EF05MA03 - Identificar e representar frações (menores e maiores que 
a unidade), associando-as ao resultado de uma divisão ou à ideia de parte 
de um todo, utilizando a reta numérica como recurso.
EF05MA04 - Identificar frações equivalentes.
EF05MA06 - Associar as representações 10\%, 25\%, 50\%, 75\% e 100\% 
respectivamente à décima parte, quarta parte, metade, três quartos e um 
inteiro, para calcular porcentagens, utilizando estratégias pessoais, 
cálculo mental e calculadora, em contextos de educação financeira, entre 
outros.

a) Incorreta. Camilo perdeu 6 pênaltis, logo a razão é de 6/14, que 
é igual a 3/7.
b) Correta. Se ele acertou 8 de 14, então errou 6 pênaltis.
Portanto a razão será: 6/14 = 3/7.
c) Incorreta. Camilo perdeu 6 pênaltis, logo a razão é de 6/14, que 
é igual a 3/7.
d) Incorreta. Camilo perdeu 6 pênaltis, logo a razão é de 6/14, que 
é igual a 3/7.}

\num{13} Em uma cadeira reclinável o assento possui 3 opções de posições
diferentes e o encosto possui 5 opções de posições diferentes. Quantas
possibilidades de posições combinando uma posição para o assento e uma
posição para o encosto podemos formar:

\begin{minipage}{.5\textwidth}
\begin{escolha}
\item
  9
\item
  15
\item
  25
\item
  40
\end{escolha}
\end{minipage}
\sidetext{SAEB: Resolver problemas simples de contagem (combinatória).

EF05MA09 - Resolver e elaborar problemas simples de contagem envolvendo o 
princípio multiplicativo, como a determinação do número de agrupamentos 
possíveis ao se combinar cada elemento de uma coleção com todos os 
elementos de outra coleção, por meio de diagramas de árvore ou por 
tabelas.

a) Incorreta. Existem 15 maneiras diferentes de posicionar essa cadeira. 
b) Correta. Se temos 3 opções para o assento e 5 opções para o encosto, 
multiplicamos uma pela outra para obter o total de maneiras diferentes 
de se posicionar essa cadeira. 3 x 5 = 15.
c) Incorreta. Existem 15 maneiras diferentes de posicionar essa cadeira.
d) Incorreta. Existem 15 maneiras diferentes de posicionar essa cadeira.}

\num{14} Lucas, com o auxílio de seu professor está montando no
laboratório de robótica um super sistema de transmissão de dados.
Para isso, ele precisa de 7 metros de fio de cobre, cortados em 
pedaços menores de 0,14 metros de comprimento.

Ela já possui 8 pedaços no tamanho desejado. Quantos pedaços ainda
faltam para ele continuar a montar seu sistema?

\begin{minipage}{.5\textwidth}
\begin{escolha}
\item
  8
\item
  26
\item
  42
\item
  50
\end{escolha}
\end{minipage}
\sidetext{SAEB: Resolver problemas de multiplicação ou de divisão,
envolvendo números racionais apenas na representação decimal finita até
a ordem dos milésimos, com os significados de formação de grupos iguais
(incluindo repartição equitativa de medida), proporcionalidade ou
disposição retangular.

BNCC: EF05MA07 - Resolver e elaborar problemas de adição e subtração com 
números naturais e com números racionais, cuja representação decimal seja 
finita, utilizando estratégias diversas, como cálculo por estimativa, 
cálculo mental e algoritmos.
EF05MA08 - Resolver e elaborar problemas de multiplicação e divisão com 
números naturais e com números racionais cuja representação decimal é 
finita (com multiplicador natural e divisor natural e diferente de zero), 
utilizando estratégias diversas, como cálculo por estimativa, cálculo 
mental e algoritmos.

c) Correta. Dividindo o fio de cobre em pedaços de tamanho desejado, 
obtemos o total de 50 pedaços, pois 7 metros /0,14 metros = 50.
Como ele já possui 8 pedaços, ele precisará de mais 42, pois 50 - 8 = 42.}

\num{15}
\begin{quote}
  {[}\ldots{}{]} Existem muitos jeitos de brincar, mas o objetivo é sempre desfrutar o
  momento e a companhia dos amigos. Além disso, os jogos ajudam a
  desenvolver habilidades que serão importantes ao longo da vida.
  Brincar é também uma maneira de aprender!

Os índios possuem muitos jogos e brincadeiras. Alguns são bastante
conhecidos por vários povos indígenas, {[}\ldots{}{]} como a peteca e a perna
de pau. {[}\ldots{}{]}

\fonte{Mirim Povos Indígenas Brasil. Brincadeiras. Disponível em: \emph{
https://mirim.org/pt-br/como-vivem/brincadeiras}. Acesso em: 16 fev.
2023.}
\end{quote}

\noindent{}Segundo o texto, algumas brincadeiras indígenas

\begin{escolha}
\item são parecidas com algumas brincadeiras tradicionais também não indígenas.

\item são realizadas exclusivamente pelos indígenas, sem influências ou compartilhamentos.

\item são padronizadas para os povos indígenas.

\item são praticadas de maneira individual.
\end{escolha}

\coment{SAEB: Valorizar o patrimônio histórico representado pelas brincadeiras e
jogos, com ênfase naqueles de origem indígena e africana.

BNCC: EF35EF01 -- Experimentar e fruir brincadeiras e jogos
populares do Brasil e do mundo, incluindo aqueles de matriz indígena e
africana, e recriá-los, valorizando a importância desse patrimônio
histórico cultural.}


\num{16} Leia um trecho de notícia.
\begin{quote}
  {[}\ldots{}{]} as crianças aprendem a respeitar o próximo, a ceder, a
  ganhar e a perder e constroem o senso de coletividade. Isso vai
  refletir no convívio com a família, na escola e, futuramente, até no
  trabalho.

\fonte{G1. Bem estar. Esporte coletivo promove o respeito ao próximo e o trabalho em equipe.
Disponível em: \emph{
https://g1.globo.com/bemestar/noticia/2016/08/esporte-coletivo-promove-o-respeito-ao-proximo-e-o-senso-de-coletividade.html}.
Acesso em: 16 fev. 2023.}
\end{quote}

\noindent{}Depois da leitura, é possível perceber que o texto fala sobre

\begin{escolha}
\item os jogos pré-depsortivos.

\item os esportes competitivos.

\item as modalidades olímpicas.

\item as atividades escolares.
\end{escolha}

\coment{SAEB: Analisar o protagonismo do trabalho coletivo na vivência dos jogos
populares e dos esportes.

BNCC: EF35EF06 -- Diferenciar os conceitos de jogo e esporte,
identificando as características que os constituem na contemporaneidade
e suas manifestações (profissional e comunitária/lazer).}

\num{17} Observe a imagem.
  \begin{figure}[htpb!]
\includegraphics[width=\textwidth]{./imgs/img15.jpg}
\end{figure}
%Disponível em: https://br.freepik.com/fotos-gratis/dancarinas-nigerianas-de-tiro-medio\_16130625.htm\#\&position=34\&from\_view=collections. Acesso em: 16 fev. 2023.

\noindent{}Após a análise, pode-se perceber que é uma dança, pois

\begin{escolha}
\item as pessoas estão dançando ao ar livre.

\item as pessoas estão realizando uma pratica corporal coletiva.

\item as pessoas estão com vestimentas e pinturas corporais da dança.

\item as pessoas estão se movimentando no ritmo do batuque do instrumento
musical.
\end{escolha}

\coment{SAEB: Comparar os elementos constitutivos de danças populares do Brasil
e do mundo com aqueles de danças de matrizes indígena e africana.

BNCC: EF35EF10 -- Comparar e identificar os elementos constitutivos
comuns e diferentes (ritmo, espaço, gestos) em danças populares do
Brasil e do mundo e danças de matriz indígena e africana.}

\colorsec{Respostas}

\begin{enumerate}

\item
a) Correta. As brincadeiras citadas (peteca e perna de pau) são tradicionais de origem indígenas que muitas pessoas
conhecem.
b) Incorreta. As brincadeiras de origem indígena também são
realizadas por outros povos e culturas.
c) Incorreta. No trecho “\ldots{}Existem muitos jeitos de
brincar\ldots{}”, é possível analisar que existem variações nas brincadeiras.
d) Incorreta. No trecho “\ldots{}o objetivo é sempre desfrutar o
momento e a companhia dos amigos\ldots{}”, podemos compreender que as
brincadeiras são realizadas em grupo para promover a socialização.

\item
a) Correta. Os jogos são atividades voltadas para a diversão e a socialização.
b) Incorreta. Os esportes competitivos visam apenas a competições e vitórias.
c) Incorreta. Assim como os esportes, as modalidades olímpicas
visam ao alto rendimento e às competições.
d) Incorreta. O texto fala sobre jogos pré-depsortivos e não
sobre atividades escolares.

\item
a) Incorreta. As danças podem ser realizadas em espaço aberto ou
fechado e isso não define se uma atividade é uma dança oficial ou não.
b) Incorreta. A dança pode ser realizada individualmente ou em
duplas. O fato de a dança ser em grupo não define que uma prática seja
considerada uma dança.
c) Incorreta. As vestimentas e pinturas não são próprias da
dança, já que em esportes, ginásticas e lutas podem aparecer esses elementos.
d) Correta. A pessoa se movimentado na batida da música
(instrumento musical) vai estar realizando o elemento constitutivo do
ritmo e do gesto da dança.
\end{enumerate}

\num{18} No estômago, o alimento é envolvido pelo suco gástrico.
Forma-se, assim, uma massa de bolo alimentar chamada quimo. Quando chega
ao intestino delgado, o quimo passa por ações de várias substâncias, que
aproveitam os nutrientes necessários, como o amido e as proteínas, e,
então, passa a ser chamado de quilo.

O texto descreve etapas do processo de

\begin{minipage}{.5\textwidth}
\begin{escolha}
\item digestão.

\item alimentação.

\item hidratação.

\item evacuação.
\end{escolha}
\end{minipage}
\sidetext{BNCC: EF05CI06 - Selecionar argumentos que justifiquem por
que os sistemas digestório e respiratório são considerados
corresponsáveis pelo processo de nutrição do organismo, com base na
identificação das funções desses sistemas.}


\num{19} A densidade nutricional de um alimento é uma classificação
que separa as calorias consumidas em cheias ou vazias. As calorias
cheias são as que fornecem quantidades de vitaminas, fibras, gorduras e
proteínas em equilíbrio, enquanto as vazias apresentam um número elevado
de açúcares e gorduras.

Alimentos com menor densidade nutricional apresentam calorias vazias,
pois

\begin{minipage}{.5\textwidth}
\begin{escolha}
\item são alimentos com menor teor calórico.

\item são essenciais para qualquer dieta.

\item são uma fonte pobre de nutrientes.

\item são menos ofensivos ao organismo.
\end{escolha}
\end{minipage}
\sidetext{BNCC: EF05CI08 - Organizar um cardápio equilibrado com base
nas características dos grupos alimentares (nutrientes e calorias) e nas
necessidades individuais (atividades realizadas, idade, sexo etc.) para
a manutenção da saúde do organismo.}

\num{20} Ao estudar as fases da lua, um pesquisador observou o céu
todas as noites, a fim de anotar informações sobre a visibilidade do
satélite natural da Terra. Ele registrou as anotações numa tabela:

\begin{longtable}[]{@{}ll@{}}
\toprule
\textbf{Semana} & \textbf{Fase da lua}\tabularnewline
1 & Minguante\tabularnewline
2 & Nova\tabularnewline
3 & Crescente\tabularnewline
4 & Cheia\tabularnewline
\bottomrule
\end{longtable}

Com base nessas informações, observou que se tratava de um ciclo lunar
sinódico, no qual cada fase dura entre 7 e 8 dias.

No meio da semana 5, qual será a fase da lua?

\begin{minipage}{.5\textwidth}
\begin{escolha}
\item Cheia.

\item Nova.

\item Crescente.

\item Minguante.
\end{escolha}
\end{minipage}
\sidetext{BNCC: EF05CI12 - Concluir sobre a periodicidade das fases da
Lua, com base na observação e no registro das formas aparentes da Lua no
céu ao longo de, pelo menos, dois meses.}

\colorsec{Respostas}

\begin{enumerate}

\item
a) Correta. A produção do quimo e do quilo
é parte do processo de digestão.
b) Incorreta. A alimentação vem antes da digestão e é, costumeiramente,
uma ação mecânica que acontece a partir da boca. É dela que advêm os
nutrientes a serem digeridos e processados no organismo. 
c) Incorreta. O processo de hidratação do organismo ocorre com a
ingestão de água; não é, portanto, descrito nas etapas de produção de
quimo e quilo.
d) Incorreta. A evacuação é a etapa de eliminação das fezes do
organismo, que são o produto final restante de todas as etapas da
digestão. Trata-se, portanto, de apenas mais uma etapa do processo
de digestão.

\item
a) Incorreta. Alimentos ricos em gorduras e açúcares representam,
geralmente, maior teor calórico que outros, embora somente essa
informação não seja usada para classificação em calorias vazias ou
cheias.
b) Incorreta. A alimentação com fontes pobres de calorias não é
essencial para as dietas dos seres humanos, sejam elas quais forem, e
deve ser realizada de maneira consciente e sem excessos.
c) Correta. Como são compostas majoritariamente por gorduras e açúcares,
as calorias vazias são pobres em demais nutrientes como vitaminas,
fibras e proteínas.
d) Incorreta. Calorias vazias podem, quando ingeridas em excesso, causar
doenças como o câncer e a diabetes. Não são, portanto, menos ofensivas
ao organismo que as calorias cheias.

\item
a) Incorreta. A lua cheia acontece na semana 4; portanto, num ciclo
lunar sinódico, em meados da semana 5, ocorrerá a fase seguinte, que é a
lua minguante.
b) Incorreta. Em um ciclo lunar sinódico, a lua nova não pode ocorrer
logo após a lua cheia sem a passagem pela lua minguante.
c) Incorreta. A lua cheia não pode dar lugar à lua crescente, que é a
fase da lua que a antecede. Isso só poderia ocorrer caso a fase anterior
fosse a lua nova.
d) Correta. Como se trata de um ciclo lunar sinódico, ou seja, completo,
a fase seguinte que se espera observar no meio da semana é a lua
minguante.
\end{enumerate}