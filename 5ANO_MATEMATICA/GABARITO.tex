\chapter{Respostas}
\pagestyle{plain}
\footnotesize

\pagecolor{gray!40}

\colorsec{Matemática -- Módulo 1 -- Treino}

\begin{enumerate}
\item 
a) Correta. 600 000 + 50 000 + 700 + 30 + 4.
b) Incorreta. Não se trata de 5.000, mas de 50.000; não se trata de 70, mas de 700; não se trata de 3, mas de 30.
c) Incorreta. Não se trata de 500, mas de 50.000.
d) Incorreta. Não se trata de 60.000, nem de 70, nem de 300.

\item
a) Incorreta. 570 não pode estar no ponto Q, onde estaria o número 550.
b) Incorreta. No ponto R estaria o número 560, 20 unidades além de 540.
c) Correta. O 570 estará no ponto S, pois, como no ponto P está o 540 e cada
repartição é de 10 unidades, ele deve estar no terceiro ponto após o P
sem contar o ponto S.
d) Incorreta. No ponto T estaria o número 580, 30 unidades além de 540.

\item
a) Incorreta. Na posição correta seria a representação de 500, não de 50.
b) Incorreta. No número errado da casa de José, o 5 está na casa das unidades.
c) Correta. Como o número apresentado no enunciado está com o primeiro e o último
algarismos trocados, conclui-se que o número correto seria 521. Na placa,
o último algarismo é o 5 e tem valor relativo de 5 unidades, mas no
número correto ele estaria na centena comum, apresentando, então, um valor
relativo de 500 (quinhentos), ou cinco centenas.
d) Incorreta. No número errado se trata de 5 e, no número correto, de 500.

\end{enumerate}

\colorsec{Matemática -- Módulo 2 -- Treino}

\begin{enumerate}
\item
a) Incorreta. Se  o número escondido fosse 128, o resultado seria 289.
b) Correta. 417 -- 105 = 312.
c) Incorreta. Se  o número escondido fosse 158, o resultado seria 259.
d) Incorreta. Se  o número escondido fosse 256, o resultado seria 161.

\item
a) Incorreta. Para restarem 72 peças, teriam de ter sido vendidas 128.
b) Incorreta. Para restarem 94 peças, teriam de ter sido vendidas 106.
c) Incorreta. Para restarem 126 peças, teriam de ter sido vendidas 74.
d) 200 -- 2 x (1 x 5 + 2 x 7) = 200 -- 38 = 162 peças.

\item
a) Incorreta. Para restarem 268 pessoas na fila, deveriam estar aguardando 928 pessoas.
b) Correta.
1.200 -- 540 = 660.
932 -- 660 = 272 pessoas não conseguirão assistir a essa sessão.
c) Incorreta. Para restarem 294 pessoas na fila, deveriam estar aguardando 954 pessoas.
d) Incorreta. Para restarem 1.440 pessoas na fila, deveriam estar aguardando 2.100 pessoas.
\end{enumerate}

\colorsec{Matemática -- Módulo 3 -- Treino}

\begin{enumerate}
\item
a) Incorreta. 25 não é o resultado de 6, multiplicado por ele mesmo, mais ele mesmo.
b) Incorreta. 30 será o número de elementos da figura 5.
c) Incorreta. 35 não faz parte da sequência, em nenhuma posição.
d) Correta. (2; 6; 12; 20; 30; 42).

Professor, a sequência é dada por n² + n =
6² + 6 = 42 (sendo n o número da figura. 1² + 1 = 2; 2² + 2 = 6; 3² + 2 = 12; 4² + 4 = 20...)

\item
a) Incorreta. A sequência é decrescente e finita, ao contrário do que se afirma.
b) Incorreta. A sequência é realmente finita, mas decrescente.
c) Correta. Pela análise da sequência dada, percebemos que ela é finita e
decrescente.
d) Incorreta. A sequência é realmente decrescente, mas finita.

\item
a) Incorreta. 6.079 é o antecessor do antecessor de 6.081.
b) Incorreta. 6.080 é o antecessor de 6.081.
c) Incorreta. 6.082 é o sucessor de 6.081.
d) Correta. Sucessor do sucessor de 6.081 = 6.081 +1 +1 = 6.083.
\end{enumerate}


\colorsec{Matemática -- Módulo 4 -- Treino}

\begin{enumerate}
\item
a) Incorreta. Saindo a esse horário ele completará 3h de trabalho.
b) Incorreta. Saindo a esse horário ele completará 3h30min de trabalho.
c) Incorreta. Saindo a esse horário ele completará 4h de trabalho.
d) Correta. Como pela manhã ele entra às 8:00 e deve cumprir nesse período 4 horas e
meia de trabalho antes de sair para o almoço, conclui-se que ele sairá
para o almoço às 12:30.

\item
a) Incorreta. Um frasco contém 100 ml; menos do que a quantidade de que ela precisa.
b) Incorreta. Com a compra de 2 frascos faltará uma dose; ou seja, 10 ml.
c) Correta. 
3 x 10 x 7 = 210 ml. Como cada frasco contém 100 ml, ela terá que
comprar 3 frascos e haverá uma sobra de xarope.
d) Incorreta. Na compra de 4 frascos, um restará fechado e inutilizado; não há essa necessidade.

\item
a) Correta. 
Saída: 10 horas e 42 minutos;
Chegada: 14 horas e 8 minutos;
Tempo de voo: 3 horas e 26 minutos = 206 minutos = 12.300 segundos.
b) Incorreta. Trata-se de mais do que esse tempo.
c) Incorreta. Trata-se de mais do que o dobro desse tempo.
d) Incorreta. Trata-se de algo próximo de cinco vezes esse tempo mencionado.
\end{enumerate}

\colorsec{Matemática -- Módulo 5 -- Treino}

\begin{enumerate}
\item
a) Incorreta. Nesse caso, seriam 4 lados de quadrado.
b) Correta. Ele deverá andar 5 lados de quadrado. Como cada lado de quadrado possui
medida igual a 2 m, ele deverá andar 10 metros.
c) Incorreta. Nesse caso, seriam 6 lados de quadrado.
d) Incorreta. Nesse caso, seriam 7 lados de quadrado.

\item
a) Correta. Como cada lado será ampliado em duas vezes, as medidas dos lados do novo
triângulo deverão ser dobradas, ou seja, multiplicadas por 2.
b) Incorreta. Nesse caso, o triângulo seria reduzido.
c) Incorreta. Nese caso, o triângulo sofreria diminuição.
d) Incorreta. Nesse caso, a redução no tamanho do triângulo seria muito significativa.

\item
a) Incorreta. Nesse caso, ela teria gastado apenas 15 minutos para se arrumar.
b) Incorreta. Nesse caso, ela teria gastado 25 minutos para se arrumar.
c) Incorreta. Nesse caso, Maria teria se arrumado em 30 minutos.
d) Correta. O relógio está marcando 11 horas e 35 minutos; se acrescentarmos a esse
horário 35 minutos, teremos no relógio 12 horas e 10 minutos.
\end{enumerate}

\colorsec{Matemática -- Módulo 6 -- Treino}

\begin{enumerate}
\item
a) Correta.
12 x 0,50 + 8 x 0,25 = 6 + 2 = R\$ 8,00. Portanto, 4 notas de 2 reais.
b) Incorreta. 6 notas de 2 reais totalizariam R\$ 12,00.
c) Incorreta. 8 notas de 2 reais totalizariam R\$ 16,00.
d) Incorreta. 20 notas de 2 reais totalizariam R\$ 40,00.

\item
a) Incorreta. R\$ 9,00 é o valor somente das células.
b) Incorreta. R\$ 9,90 seria o valor das células mais R\$ 0,90 em moedas.
c) Incorreta. R\$ 10,10 seria um valor R\$ 0,05 menos que o total encontrado.
d) Correta. 
R\$ 9,00 em cédulas e R\$ 1,15 em moedas. Portanto, no total, ela
encontrou em sua bolsa R\$ 10,15.

\item
a) Incorreta. R\$ 18,00 seriam R\$ 10,00 a menos do que o valor pago.
b) Incorreta. R\$ 22,00 seriam R\$ 6,00 a menos do que o valor pago.
c) Correta. 2 x 5,00 + 1 x 6,00 + 1 x 12,00 = 10,00 + 6,00 + 12,00 = R\$ 28,00.
d) Incorreta. R\$ 30,00 seriam R\$ 2,00 a mais do que o valor pago.

Professor, você pode utilizar este exercício e estimular os alunos a
realizarem outras combinações conforme a preferência de cada um e
encontrarem qual o valor que eles pagariam nessa lanchonete pelo respectivo pedido.
\end{enumerate}

\colorsec{Matemática -- Módulo 7 -- Treino}

\begin{enumerate}
\item
a) Incorreta. Nesse caso, trata-se da chance de escolha de apenas um dia da semana.
b) Correta. 
Dias da semana: 7.
Escolhas determinadas: 2.
Probabilidade: 2/7.
c) Incorreta. Essa seria a chance de uma escolha dentre 5 possibilidades.
d) Incorreta. Nesse caso, as chances são maiores do que as reais.

\item
a) Correta. 
Total de pessoas: 28 + 7 = 35.
Número de garçons: 7.
Probabilidade: 7/35 = 1/5 = 0,2 = 20\%.
b) Incorreta. Trata-de de mais que o dobro das chances reais.
c) Incorreta. Trata-se de 7 chances em 10.
d) Incorreta. Trata-se de totais chances de ocorrência, como se todas as pessoas presentes no restaurantes fossem garçons.

\item
a) Incorreta. Trata-se de apenas um quarto das chances reais.
b) Incorreta. Trata-se de apenas metade das chances reais.
c) Incorreta. Trata-se de aproximadamente uma chance em três possibilidades.
d) Correta. Como Carol participará apenas do sorteio final, ela pode ser sorteada ou
não e isso no leva a concluir que ela terá 50\% de chance de iniciar a
disputa.
\end{enumerate}

\colorsec{Matemática -- Módulo 8 -- Treino}

\begin{enumerate}
\item
a) Incorreta. No dia 05/03 foram vendidos apenas 18 sanduíches.
b) Incorreta. No dia 06/03 foram vendidos apenas 16 sanduíches.
c) Correta. O dia com a maior quantidade vendas foi o dia 07/03, com 25 produtos
vendidos.
d) Incorreta. No dia 08/03 foram vendidos apenas 14 sanduíches, tendo sido o dia com as menores vendas.

\item
a) Incorreta. Trata-se do mês com as menores vendas.
b) Incorreta. 45 não é o triplo das vendas de nenhum dos outros meses.
c) Incorreta. 62 não é o triplo das vendas de nenhum dos outros meses.
d) Correta. Em julho, com uma venda de 72 bolas, que é o triplo das 24 unidades vendidas em abril.

\item
a) Incorreta. Trata-se do número de crianças de 4 a 6 anos.
b) Incorreta. Trata-se do número de crianças de 7 a 9 anos, apenas.
c) Incorreta. Trata-se da soma entre as quantidades de crianças de 4 a 6 anos e de 10 a 12 anos.
d) Correta. Segundo o gráfico apresentado, 12 + 9 = 21 crianças de 7 a 12 anos
visitaram a loja.
\end{enumerate}

\colorsec{Matemática -- Módulo 9 -- Treino}

\begin{enumerate}
\item
a) Incorreta. 3/3 representa todos os bombons.
b) Incorreta. 2/5 é menos do que a metade.
c) Correta. Chocolate branco/ chocolate ao leite = 4/8 = ½.
d) Incorreta. 4/6 é mais do que a metade.

Professor, reforce com os alunos a diferença entre a razão de componentes e a razão com relação ao total.

\item
a) Incorreta. As partes não são iguais.
b) Incorreta. As partes não são iguais entre si.
c) Correta. A única que contém divisões iguais e que condizem com a divisão é a figura que aparece nesta alternativa.
d) Incorreta. Há, de fato, divisão em três partes, mas elas não são iguais.

Professor, reforce bastante com os alunos que as partes devem ser de
mesmo tamanho para representarem uma parte do todo.

\item
a) Incorreta. 8 não representam 25\% dos alunos da turma.
b) Correta. 25\% de 36 = ¼ x36 = 9 alunos por sessão.
c) Incorreta. 10 representam mais do que 25\% dos alunos.
d) Incorreta. 11 representam mais do que 25\% dos alunos.
\end{enumerate}

\colorsec{Matemática -- Módulo 10 -- Treino}

\begin{enumerate}
\item
a) Incorreta. Esseé o valor de uma única pizza.
b) Incorreta. Trata-se do valor, já mencionado, de duas pizzas.
c) Incorreta. Esse é o valor aproximado de 3 pizzas.
d) Correta. Valor de cada pizza: R\$ 81,60/2 = R\$ 40,80. Valor de 6 pizzas: 6 x 40,80 = R\$ 244,80.

\item
a) Incorreta. 9 colheres fazem 3 receitas, ou seja, 24 cafezinhos.
b) Correta. Para 48 cafezinhos, ela terá que fazer 6 receitas. Sendo assim, basta multiplicar a quantidade de colheres de pó de café para 8 cafezinhos também por
6. 3 x 6 = 18 colheres de sopa de pó de café.
c) 24 colheres fariam 64 cafés.
d) 48 colheres fariam 128 cafés.

\item
a) Incorreta. Em 1 minuto imprimem-se 100 folhas.
b) Incorreta. Em 15 minutos imprimem-se 1500 folhas.
c) Correta. Quantidade de folhas para 700 jornais: 5 x 700 = 3.500 folhas. Tempo gasto para a produção de 3.500 folhas = 3.500/100 = 35 minutos.
d) Incorreta. Em 55 minutos imprimem-se 5500 folhas.
\end{enumerate}

\colorsec{Matemática -- Módulo 11 -- Treino}

\begin{enumerate}
\item
a) Incorreta. Isso aconteceria com uso de apenas uma cor, com 8 opções de cores.
b) Incorreta. Trata-se de um número muito menor de opções do que as possibilidades reais.
c) Correta. 8 x 7 = 56 combinações diferentes de cores para a bandeira.
d) Incorreta. Não chega a haver tantas combinações assim.

\item
a) Correta. 6 x 18 = 108 possibilidades.
b) Incorreta. Trata-se de um número muito menor de opções do que as possibilidades reais.
c) Incorreta. Trata-se de quase um décimos das possibilidades reais.
d) Incorreta. Não chega a haver tantas combinações assim.

\item
a) Incorreta. Trata-se de um número muito ptóximo daquele das possibilidades reais, mas ainda um pouco menor.
b) Correta. 
Sair de X e passar por S antes de chegar a Z: 3 x 2 = 6;
Sair de X passar por S e Y antes de chegar a Z: 3 x 2 x 2 = 12;
Sair de X passar por Y antes de chegar a Z: 1 x 2 = 2;
Sair de X passar por R antes de chegar a Z: 3 x 1 = 3;
Sair de X passar por R e Y antes chegar a Z: 3 x 3 x 2 = 18;
Total: 6 + 12 + 2 + 3 + 18 = 41 caminhos diferentes.
c) Incorreta. É preciso analizar o diagrama para se chegar à conclusão.
d) Incorreta. Não chega a haver tantas combinações assim.
\end{enumerate}

\colorsec{Matemática -- Módulo 12 -- Treino}

\begin{enumerate}
\item
a) Correta. 100 -- 38,25 -- 21,55 = R\$ 40,20.
b) Incorreta. Para sobrar esse valor de troco, a compra deveria totalizar R\$ 47,60.
c) Incorreta. Para sobrar esse valor de troco, a compra deveria totalizar R\$ 41,40.
d) Correta. Seria impossível sobrar um troco maior do que o valor pago, de cem reais.

\item
a) Incorreta. Nesse caso o aluno terá multiplicado o valor de R\$ 3,75 por 5, sem considerar o valor diferenciado da primeira hora.
b) Correta. 1 x 8 + (5 -- 1) x 3,75 = R\$ 23,00.
c) Incorreta. Nesse caso o aluno terá considerado o valor da primeira hora, mais 5 horas com o valor de R\$ 3,75.
d) Incorreta. Nesse caso o aluno terá multiplicado o valor da primeira hora pelas 5 horas.

\item
a) Incorreta. Nesse caso o aluno terá considerado a quantidade de um quarto da produção.
b) Correta. Metade do suco produzido: 248,40/2 = 124,20 = 124 200 ml.
c) Incorreta. Nesse caso o aluno terá multiplicado a produção por 2, em vez de dividir.
d) Incorreta. Nesse caso o aluno terá multiplicado a produção por 4, em vez de dividir por 2.
\end{enumerate}

\colorsec{Matemática -- Módulo 13 -- Treino}

\begin{enumerate}
\item
a) Incorreta: é necessário multiplicar o número de fileiras
pelo número de cadeiras, isto é: 25 X 8 e 6 X 6. 
b) Incorreta: é necessário multiplicar o número de fileiras
pelo número de cadeiras, isto é: 25 X 8 e 6 X 6.  
c) Correta: o número de fileiras foi adequadamente multiplicado 
pelo número de cadeiras, isto é: 25 X 8 e 6 X 6. 
d) Incorreta: é necessário multiplicar o número de fileiras
pelo número de cadeiras, isto é: 25 X 8 e 6 X 6.  
25 x 8 + 6 x 6.

\item
Para alcançar o valor total arrecadado com a 
venda de ingressos, é preciso somar a arrecadação de entradas inteiras
(87 X R\$ 26,00 = R\$ 2.262,00) com a de meias-entradas 
(65 X R\$ 13,00 = R\$ 845,00). O total é de R\$ 3.107,00. 
a) Incorreta: R\$ 2.262,00 corresponde apenas à arrecadação de entradas 
inteiras.  
b) Incorreta: R\$ 845,00 corresponde apenas à arrecadação de 
meias-entradas. 
c) Incorreta: R\$ 1.417,00 não corresponde à soma da arrecadação de 
entradas inteiras e meias-entradas.
d) Correta: R\$ 3.107,00 corresponde à soma da arrecadação de 
entradas inteiras e meias-entradas.

\item
a) Correta: (37 -- 5)/4 + [(2 x 5) + 6]/2 = (32)/4 + 16/2 = 8 + 8 = 16.
b) Incorreta: o valor pensado por Marcel é 16.
c) Incorreta: o valor pensado por Marcel é 16.
d) Incorreta: o valor pensado por Marcel é 16.
\end{enumerate}

\colorsec{Matemática -- Módulo 14 -- Treino}

\begin{enumerate}
\item
No enunciado, afirma-se que a empresa tem 350 funcionários; esse 
é, portanto, o número que representa a população. A amostra, por sua vez, 
se refere apenas àqueles que participaram da pesquisa, isto é, 75
funcionários.
a) Incorreta. A população é de 350 funcionários (não 10); a amostra é de
75 (não de 350).
b) Incorreta. A amostra correta é de 75 funcionários, não de 10.
c) Incorreta. Nessa alternativa, os números de população e de amostra
estão invertidos.
d) Correta: A população é de 350 funcionários; a amostra corresponde ao
número de candidatos que participaram da pesquisa: 75.

\item
a) Incorreta. As variáveis qualitativas não são representadas por 
meio de quantidades. O número do irmãos não pode, assim, ser uma variável 
qualitativa.
b) Correta. As variáveis qualitativas discretas só podem ser medidas
com números inteiros, sem frações. É o caso do número de irmãos. 
c) Incorreta. As variáveis qualitativas contínuas são fracionáveis,
divisíveis. Não é, claramente, o caso do número de irmãos. 
d) Incorretas. As definições de variáveis quantitativas e qualitativas
excluem a possibilidade de haver quaisquer variáveis que participem das
duas categorias ao mesmo tempo.

\item
a) Incorreta. Existem 10 notas maiores ou iguais a 7 (sete) e 10 notas menores do que 7 (sete).
b) Correta. Entre as notas apresentadas, temos 10 notas maiores ou iguais
a 7 (sete).
c) Incorreta. Incorreta. Existem 10 notas maiores ou iguais a 7 (sete) e 10 notas menores do que 7 (sete).
d) Incorreta. Incorreta. Existem 10 notas maiores ou iguais a 7 (sete) e 10 notas menores do que 7 (sete).
\end{enumerate}

\colorsec{Matemática -- Módulo 15 -- Treino}

\begin{enumerate}
\item
a) Correta. O candidato A cumpre os dois critérios exigidos pelas
regras do processo seletivo: suas notas são todas superiores a 30, e três
delas são iguais (as de Português, Matemática e Direito).
b) Incorreta. Embora as notas do candidato B sejam superiores a 30, 
cumprindo o primeiro dos critérios do processo seletivo, apenas duas 
delas são idênticas (as de Português e Direito), de modo que esse 
B perde para A, que teve três notas iguais. 
c) Incorreta. O candidato C não pode ser aprovado, porque teve uma nota
inferior a 30.
d) Incorreta. O candidato D não pode ser aprovado, porque teve duas notas
inferiores a 30.

\item
a) Incorreta. O número de alunos do 4º ano é 86, não 60.
b) Correta. O número de alunos do 4º ano é igual à soma do número de
alunos das turmas A, B e C: 32 + 29 + 25 = 86.
c) Incorreta. O número de alunos do 4º ano é 86, não 91.
d) Incorreta. O número de alunos do 4º ano é 86, não 150.

\item
A razão do número do livros retirados em abril (205) e junho 
(210) é expressa da seguinte maneira: 205/210.  
a) Incorreta. 
b) Incorreta.
c) Correta. 205/210 = 51/52.
b) Incorreta.
\end{enumerate}


\colorsec{Educação Física -- Módulo 1 -- Treino}

\begin{enumerate}
\item
a) Correta. O texto fala de uma prática corporal voltada ao lazer, ou
seja, uma brincadeira que tem regras a adaptações para a diversão.
b) Incorreta. Por mais que o nome da brincadeira seja “golpear com as
mãos”, tobdaé é uma brincadeira para o lazer e não uma luta.
c) Incorreta. O texto mostra duas variações (regras) que podem ser
modificadas para brincar.
d) Incorreta. O texto mostra como a peteca é feita, mas não são
materiais oficiais semelhantes aos dos esportes e, sim, adaptações.

\item
a) Incorreta. A luta huka-huka tem regras e objetivos que não podem ser
modificados, ou seja, as regras se mantêm as mesmas.
b) Incorreta. O texto mostra que os lutadores ganham respeito e
reconhecimento, não prêmios.
c) Incorreta. A luta huka-hula é uma manifestação corporal que não
promove a briga entres os praticantes, e sim a cultura indígena e o
respeito.
d) Correta. Por meio do texto é possível analisar que a luta não tem
juiz e os próprios lutadores reconhecem a vitória do outro. Portanto, é
um sinal de demostrar respeito com outro.

\item
a) Incorreta. O fato de diferentes etnias indígenas as realizarem não faz com que as
lutas se tornem um esporte, apenas mostra como essas práticas corporais
são importantes para essa cultura.
b) Correta. Uma das principais característica de uma prática corporal
ser um esporte é que ela deve estar presente em uma competição esportiva
oficial, como os Jogos dos Povos Indígenas.
c) Incorreta. O fato de a luta ter diferentes formas de começar (em pé ou
ajoelhado) não é uma definição do esporte, só mostra algumas versões dela.
d) Incorreta. A presença de pinturas corporais é uma característica da própria
cultura indígena, não dos esportes.
\end{enumerate}

\colorsec{Educação Física -- Módulo 2 -- Treino}

\begin{enumerate}
\item
a) Incorreta. O texto mostra que o praticante deve usar um cinto com
velcro com bandeirinha, mas o tag-rugby serve para popularizar o rugby e
não criar novos equipamentos.
b) Incorreta. Por mais que o tag-rugby evite o contato físico, esse jogo
pré-desportivo tem o propósito de incentivar a pratica do rugby e não
acabar com os conflitos nos outros esportes.
c) Incorreta. O tag-rugby não é um esporte e sim uma brincadeira do
rugby.
d) Correta. O texto mostra algumas variações do rugby para torná-lo mais
lúdico para as pessoas, com o propósito de popularizar esse esporte.

\item
a) Incorreta. São as regras do jogo mancala que podem ser
alteradas, não a cultura africana.
b) Incorreta. Por mais que o texto mostre o significado da palavra
“mancala”, o jogo de tabuleiro não ensina novas palavras, e sim
apresenta um costume da cultura africana.
c) Correta. Por meio do texto podemos perceber que o mancala é um jogo
que representa a força da cultura africana; ou seja, por meio do jogo podemos conhecer diferentes tradições de outras culturas.
d) Incorreta. O texto mostra que o mancala é usado na escola, mas não
para que os alunos estudem um conteúdo relacionado à cultura local
(Acre), e sim sobre a cultura africana.

\item
a) Incorreta. O texto fala que os locais específicos para brincar são para
incentivar a pratica de algumas brincadeiras, não para restringir as
brincadeiras tradicionais.
b) Incorreta. O objetivo é preservar a cultura local por meio das
brincadeiras, não de criar novas regras.
c) Incorreta. O texto não cita que os locais voltados para as
brincadeiras vão incentivar o comércio, e sim incentivar as pessoas a
realizarem algumas brincadeiras tradicionais.
d) Correta. O texto mostra que algumas brincadeiras se tornarem um
patrimônio cultural e vai haver locais para brincar com o objetivo de
as pessoas continuarem praticando essas brincadeiras e preservando a cultura
local.
\end{enumerate}

\colorsec{Educação Física -- Módulo 3 -- Treino}

\begin{enumerate}
\item
a) Incorreta. O fato de o toré ser realizado por diferentes povos não quer
dizer que é uma dança, e sim uma prática difundida na cultura indígena.
b) Correta. O instrumento musical (maracá) serve para marcar o ritmo na música e na dança; ou seja, é um elemento constitutivo da dança.
c) Incorreta. Porque a dança ser realizado ao ar livre não é uma
característica própria das danças.
d) Incorreta. Qualquer atividade promove a interação entre as
pessoas e não somente as danças.

\item
a) Correta. O texto mostra que os negros escravizados, do Congo e
de Angola, desenvolveram o jongo e, por isso, ele tem elementos
culturais africanos.
b) Incorreta. O jongo tem influência da cultura africana do Congo
e de Angola, não da cultura brasileira.
c) Incorreta. Eram os negros escravizados que trabalhavam nas
fazendas que realizavam a dança do jongo.
d) Incorreta. Por mais que a dança fosse praticada em eventos religiosos,
isso foi criado no Brasil e não nos países africanos. Além disso, o
evento religioso não definia que a dança é de origem africana.

\item
a) Incorreta. O samba é voltado para a dança e não para praticar a capoeira (luta africana).
b) Incorreta. O samba é de origem africana e não europeia. Apenas
alguns instrumentos portugueses são usados, mas isso não faz com que o
praticante conheça a cultura alimentar europeia.
c) Incorreta. O samba não tem o objetivo de o praticante entender
como a Unesco reconhece uma atividade como patrimônio cultural.
d) Correta. O texto mostra alguns elementos culturais presentes
no samba e, por conta disso, o praticante dessa dança vai poder conhecer
alguns costumes e tradições da cultura africana.
\end{enumerate}

\colorsec{Ciências da Natureza -- Módulo 1 -- Treino}

\begin{enumerate}
\item
a) Incorreta. A ebulição é a passagem rápida de uma substância do estado
líquido para o estado gasoso em determinada temperatura. No caso
mencionado, há um fenômeno específico realizado em condições próprias de
solo e vegetação, como na floresta amazônica, chamado de
evapotranspiração. 
b) Correta. A evapotranspiração é um fenômeno que combina a evaporação
de líquidos com a transpiração de folhas. No caso da floresta amazônica,
a umidade se eleva, pois as árvores funcionam como “bombas” de água,
participando também da regulação do regime de chuvas de toda a região.
c) Incorreta. A condensação ocorre quando há agregação de substâncias
gasosas, de modo que as partículas se unam e formem um líquido. Esse
fenômeno pode acontecer nas nuvens, numa das etapas do ciclo hidrológico
da água, mas não é descrito no texto.
d) Incorreta. A precipitação ocorre quando há quantidade suficiente de
água no estado líquido nas nuvens. Apesar de a chuva ser
mencionada no texto, não há descrição desse fenômeno, e sim da
evapotranspiração, processo em que a combinação de evaporação da água de
solos e transpiração das folhas acontece.

\item
a) Incorreta. A perda de vegetação acentua a desregulação do regime de
chuvas, pois desequilibra o ciclo hidrológico da água.
b) Incorreta. O problema mencionado é a diminuição da cobertura vegetal,
logo não há relação com a eliminação de poluentes das florestas, que
pode se dar a partir de iniciativas para localizar os focos de poluição
e realizar forças-tarefa para preservar o ambiente.
c) Incorreta. A redução da cobertura vegetal pode acarretar o
desequilíbrio do ciclo hidrológico da água, e favorece o acontecimento
de fenômenos como o assoreamento, que altera os cursos d'água e eleva o
leito de rios e lagos.
d) Correta. A perda de vegetação pode acarretar um extenso processo de
erosão dos solos que tem como uma das principais consequências a
inundação de rios, causada por fenômenos como o assoreamento, quando a
falta de vegetação, aliada com a perda da qualidade do solo, faz com que
detritos sólidos sejam arrastados para o fundo dos rios, elevando o
leito.

\item
a) Incorreta. A água residual, desde que não tenha componentes
corrosivos, não deve ser contaminante das superfícies metálicas a serem
lavadas. Em muitas situações, ela pode ser reaproveitada em atividades
como lavagens, irrigação de plantas e descarga de bacias sanitárias.
b) Incorreta. A água residual não tratada pode contaminar os mananciais
de água limpa, sem passar pelo tratamento adequado. Caso se deseje
devolver a água para os mananciais, deve-se adotar um regime de
tratamento avançado para eliminar microrganismos e componentes tóxicos
da água residual.
c) Correta. Caso se deseje reutilizar água residual para fins potáveis,
deve-se adotar um esquema avançado de tratamento a fim de filtrar-lhe as
impurezas e eliminar microrganismos, além de propor uma
avaliação da qualidade para o consumo humano.
d) Incorreta. O reaproveitamento da água traz grandes benefícios em
termos de economia de recursos e do uso sustentável de um bem natural.
Ele pode ser realizado em escala doméstica, reaproveitando água para
procedimentos como lavagens de calçadas e carros, ou em escala
industrial, com a reutilização de água residual em processos de lavagem
e produção de energia.
\end{enumerate}

\colorsec{Ciências da Natureza -- Módulo 2 -- Treino}

\begin{enumerate}
\item
a) Correta. No sistema circulatório, há o transporte de oxigênio e dos
nutrientes para todas as extremidades do corpo.
b) Incorreta. O sistema digestório é responsável por digerir os
alimentos, separar os nutrientes para o aproveitamento em diferentes
órgãos e descartar as fezes.
c) Incorreta. No sistema respiratório, há a troca gasosa e o uso do
oxigênio para alguns processos no organismo humano, como a divisão da
glicose em porções menores para o transporte. Esse sistema, em si, não
consegue irrigar todas as extremidades do corpo humano.
d) Incorreta. O sistema reprodutivo é responsável por processos relacionados
à formação e ao desenvolvimento da vida, que não dizem respeito, por sua 
vez, ao transporte de nutrientes para as extremidades do corpo.

\item
a) Incorreta. O corpo humano pode produzir células de defesa sozinho, mas
precisa de imunização prévia para determinadas doenças, que são muito agressivas.
A vacinação se faz, portanto, essencial.
b) Incorreta. O conteúdo da vacina não participa da regulação dos sistemas do
organismo, e sim da estimulação da produção de células de defesa pelo sistema
imunológico.
c) Correta. Sem a vacinação, as pessoas ficam vulneráveis ao retorno de
doenças por conta da fragilidade da defesa do organismo, que necessita
de anticorpos de memória para combater algumas doenças, e, assim, prevenir
casos graves e mortes.
d) Incorreta. O sangue não leva o conteúdo da vacina para combater
infecções. As vacinas atuam simulando uma infecção mais fraca, para
fazer com que o organismo produza defesas que ficarão na memória
imunológica. Assim, quando a infecção real acontecer, o organismo terá
estruturas para combatê-la.

\item
a) Incorreta. As informações dos rótulos nutricionais estimam a
quantidade de nutrientes do alimento com base no consumo calórico
recomendado diariamente, além de também considerar o consumo médio de
calorias necessárias para o corpo humano, sendo, assim, estimativas
confiáveis.
b) Incorreta. Apesar das necessidades calóricas variarem de corpo a
corpo, o assunto em questão é a contagem de calorias, que não deve ser o
único quesito de avaliação para uma alimentação saudável, visto que dois
alimentos podem apresentar a mesma quantidade calórica numa porção, mas
quantidades diferentes de nutrientes como carboidratos, gorduras e
proteínas.
c) Correta. Demais nutrientes, como carboidratos, gorduras e proteínas
são importantes numa alimentação balanceada. Deve-e evitar o exagero no
consumo de alimentos ricos em gorduras, como os ultraprocessados, pois
eles representam uma fonte pobre de calorias.
d) Incorreta. Os alimentos ultraprocessados representam calorias vazias,
ricas em gorduras e com pouco aproveitamento para o organismo além do
acúmulo de gordura. O consumo em excesso pode, inclusive, ser maléfico à
saúde, causando doenças.
\end{enumerate}

\colorsec{Ciências da Natureza -- Módulo 3 -- Treino}

\begin{enumerate}
\item
a) Correta. Como a irradiação de luz solar não é uniforme em todas as
regiões da Terra, devido aos movimentos de rotação e translação,
observam-se nos exemplos diferentes fusos horários e distintas estações do
ano. Enquanto o Brasil tem um dia quente e ensolarado, é madrugada 
no Japão, o dia começa na Austrália e faz frio nos Estados Unidos.
b) Incorreta. A existência do dia e da noite em dois polos distintos da
Terra ao mesmo tempo não é explicada pela ocorrência ou falta de camadas
de proteção solar no planeta.
c) Incorreta. Embora seja preocupante, a elevação dos níveis d'água nos
hemisférios não explica ocorrências milenares como a existência do dia
e da noite em pontos distintos do planeta ao mesmo tempo.
d) Incorreta. As mudanças climáticas não têm relação direta com o
fenômeno descrito, no qual, em pontos distintos do planeta Terra, ao mesmo
tempo, observa-se um dia quente, uma madrugada, um alvorecer e um dia
frio. Essas ocorrências são explicadas pela irradiação solar no planeta,
que muda conforme o movimento de Rotação.

\item
a) Incorreta. A Terra não realiza Revolução, e o texto descreve, além
dos movimentos citados, a Translação terrestre, movimento de giro da
Terra em torno do sol durante 365 dias, e a Revolução lunar, movimento
de giro da Lua em torno da Terra durante aproximadamente 28 dias.
b) Incorreta. A Terra não realiza Revolução, e os movimentos de Rotação
e Translação terrestre descritos no texto não foram citados.
c) Correta. Os movimentos realizados, na ordem de descrição do texto,
são Rotação terrestre, Translação terrestre, Rotação lunar, Translação
lunar e Revolução lunar.
d) Incorreta. Nenhum movimento do Sol foi mencionado no texto, e, além
disso, não se cita o movimento da Terra em torno do seu próprio eixo, a
Rotação terrestre, e nem o movimento de giro da Lua em torno do sol, a
Translação lunar.

\item
a) Incorreta. O equinócio de primavera representa o período de fim do
inverno e começo da primavera, ou seja, nessas datas, a incidência da
luz solar é maior na região equatorial da Terra, fazendo com que os dias
e as noites tenham durações iguais.
b) Correta. No solstício de verão, há maior incidência da luz solar em
um dos hemisférios da Terra --- norte, em junho, e sul, em dezembro.
Nesse período, inicia-se o verão, em que os dias duram mais do que as
noites.
c) Incorreta. No solstício de inverno, há menor incidência da luz solar
em um dos hemisférios da terra --- norte, em dezembro, e sul, em junho.
Nesse período, inicia-se o inverno, quando as noites duram mais do que
os dias.
d) Incorreta. O equinócio de outono representa o período de fim do verão
e começo do outono, ou seja, nessas datas, a incidência da luz solar é
maior na região equatorial da Terra, fazendo com que os dias e as noites
tenham durações iguais.
\end{enumerate}