\chapter{Respostas}
\pagestyle{plain}
\footnotesize

\pagecolor{gray!40}

\colorsec{Matemática -- Módulo 1 -- Treino}

\begin{enumerate}
\item 
a) Correta. 600 000 + 50 000 + 700 + 30 + 4.
b) Incorreta. Não se trata de 5.000, mas de 50.000; não se trata de 70, mas de 700; não se trata de 3, mas de 30.
c) Incorreta. Não se trata de 500, mas de 50.000.
d) Incorreta. Não se trata de 60.000, nem de 70, nem de 300.

\item
a) Incorreta. 570 não pode estar no ponto Q, onde estaria o número 550.
b) Incorreta. No ponto R estaria o número 560, 20 unidades além de 540.
c) Correta. O 570 estará no ponto S, pois, como no ponto P está o 540 e cada
repartição é de 10 unidades, ele deve estar no terceiro ponto após o P
sem contar o ponto S.
d) Incorreta. No ponto T estaria o número 580, 30 unidades além de 540.

\item
a) Incorreta. Na posição correta seria a representação de 500, não de 50.
b) Incorreta. No número errado da casa de José, o 5 está na casa das unidades.
c) Correta. Como o número apresentado no enunciado está com o primeiro e o último
algarismos trocados, conclui-se que o número correto seria 521. Na placa,
o último algarismo é o 5 e tem valor relativo de 5 unidades, mas no
número correto ele estaria na centena comum, apresentando, então, um valor
relativo de 500 (quinhentos), ou cinco centenas.
d) Incorreta. No número errado se trata de 5 e, no número correto, de 500.

\end{enumerate}

\colorsec{Matemática -- Módulo 2 -- Treino}

\begin{enumerate}
\item
a) Incorreta. Se  o número escondido fosse 128, o resultado seria 289.
b) Correta. 417 -- 105 = 312.
c) Incorreta. Se  o número escondido fosse 158, o resultado seria 259.
d) Incorreta. Se  o número escondido fosse 256, o resultado seria 161.

\item
a) Incorreta. Para restarem 72 peças, teriam de ter sido vendidas 128.
b) Incorreta. Para restarem 94 peças, teriam de ter sido vendidas 106.
c) Incorreta. Para restarem 126 peças, teriam de ter sido vendidas 74.
d) 200 -- 2 x (1 x 5 + 2 x 7) = 200 -- 38 = 162 peças.

\item
a) Incorreta. Para restarem 268 pessoas na fila, deveriam estar aguardando 928 pessoas.
b) Correta.
1.200 -- 540 = 660.
932 -- 660 = 272 pessoas não conseguirão assistir a essa sessão.
c) Incorreta. Para restarem 294 pessoas na fila, deveriam estar aguardando 954 pessoas.
d) Incorreta. Para restarem 1.440 pessoas na fila, deveriam estar aguardando 2.100 pessoas.
\end{enumerate}

\colorsec{Matemática -- Módulo 3 -- Treino}

\begin{enumerate}
\item
a) Incorreta. 25 não é o resultado de 6, multiplicado por ele mesmo, mais ele mesmo.
b) Incorreta. 30 será o número de elementos da figura 5.
c) Incorreta. 35 não faz parte da sequência, em nenhuma posição.
d) Correta. (2; 6; 12; 20; 30; 42).

Professor, a sequência é dada por n² + n =
6² + 6 = 42 (sendo n o número da figura. 1² + 1 = 2; 2² + 2 = 6; 3² + 2 = 12; 4² + 4 = 20...)

\item
a) Incorreta. A sequência é decrescente e finita, ao contrário do que se afirma.
b) Incorreta. A sequência é realmente finita, mas decrescente.
c) Correta. Pela análise da sequência dada, percebemos que ela é finita e
decrescente.
d) Incorreta. A sequência é realmente decrescente, mas finita.

\item
a) Incorreta. 6.079 é o antecessor do antecessor de 6.081.
b) Incorreta. 6.080 é o antecessor de 6.081.
c) Incorreta. 6.082 é o sucessor de 6.081.
d) Correta. Sucessor do sucessor de 6.081 = 6.081 +1 +1 = 6.083.
\end{enumerate}


\colorsec{Matemática -- Módulo 4 -- Treino}

\begin{enumerate}
\item
a) Incorreta. Saindo a esse horário ele completará 3h de trabalho.
b) Incorreta. Saindo a esse horário ele completará 3h30min de trabalho.
c) Incorreta. Saindo a esse horário ele completará 4h de trabalho.
d) Correta. Como pela manhã ele entra às 8:00 e deve cumprir nesse período 4 horas e
meia de trabalho antes de sair para o almoço, conclui-se que ele sairá
para o almoço às 12:30.

\item
a) Incorreta. Um frasco contém 100 ml; menos do que a quantidade de que ela precisa.
b) Incorreta. Com a compra de 2 frascos faltará uma dose; ou seja, 10 ml.
c) Correta. 
3 x 10 x 7 = 210 ml. Como cada frasco contém 100 ml, ela terá que
comprar 3 frascos e haverá uma sobra de xarope.
d) Incorreta. Na compra de 4 frascos, um restará fechado e inutilizado; não há essa necessidade.

\item
a) Correta. 
Saída: 10 horas e 42 minutos;
Chegada: 14 horas e 8 minutos;
Tempo de voo: 3 horas e 26 minutos = 206 minutos = 12.300 segundos.
b) Incorreta. Trata-se de mais do que esse tempo.
c) Incorreta. Trata-se de mais do que o dobro desse tempo.
d) Incorreta. Trata-se de algo próximo de cinco vezes esse tempo mencionado.
\end{enumerate}

\colorsec{Matemática -- Módulo 5 -- Treino}

\begin{enumerate}
\item
a) Incorreta. Nesse caso, seriam 4 lados de quadrado.
b) Correta. Ele deverá andar 5 lados de quadrado. Como cada lado de quadrado possui
medida igual a 2 m, ele deverá andar 10 metros.
c) Incorreta. Nesse caso, seriam 6 lados de quadrado.
d) Incorreta. Nesse caso, seriam 7 lados de quadrado.

\item
a) Correta. Como cada lado será ampliado em duas vezes, as medidas dos lados do novo
triângulo deverão ser dobradas, ou seja, multiplicadas por 2.
b) Incorreta. Nesse caso, o triângulo seria reduzido.
c) Incorreta. Nese caso, o triângulo sofreria diminuição.
d) Incorreta. Nesse caso, a redução no tamanho do triângulo seria muito significativa.

\item
a) Incorreta. Nesse caso, ela teria gastado apenas 15 minutos para se arrumar.
b) Incorreta. Nesse caso, ela teria gastado 25 minutos para se arrumar.
c) Incorreta. Nesse caso, Maria teria se arrumado em 30 minutos.
d) Correta. O relógio está marcando 11 horas e 35 minutos; se acrescentarmos a esse
horário 35 minutos, teremos no relógio 12 horas e 10 minutos.
\end{enumerate}

\colorsec{Matemática -- Módulo 6 -- Treino}

\begin{enumerate}
\item
a) Correta.
12 x 0,50 + 8 x 0,25 = 6 + 2 = R\$ 8,00. Portanto, 4 notas de 2 reais.
b) Incorreta. 6 notas de 2 reais totalizariam R\$ 12,00.
c) Incorreta. 8 notas de 2 reais totalizariam R\$ 16,00.
d) Incorreta. 20 notas de 2 reais totalizariam R\$ 40,00.

\item
a) Incorreta. R\$ 9,00 é o valor somente das células.
b) Incorreta. R\$ 9,90 seria o valor das células mais R\$ 0,90 em moedas.
c) Incorreta. R\$ 10,10 seria um valor R\$ 0,05 menos que o total encontrado.
d) Correta. 
R\$ 9,00 em cédulas e R\$ 1,15 em moedas. Portanto, no total, ela
encontrou em sua bolsa R\$ 10,15.

\item
a) Incorreta. R\$ 18,00 seriam R\$ 10,00 a menos do que o valor pago.
b) Incorreta. R\$ 22,00 seriam R\$ 6,00 a menos do que o valor pago.
c) Correta. 2 x 5,00 + 1 x 6,00 + 1 x 12,00 = 10,00 + 6,00 + 12,00 = R\$ 28,00.
d) Incorreta. R\$ 30,00 seriam R\$ 2,00 a mais do que o valor pago.

Professor, você pode utilizar este exercício e estimular os alunos a
realizarem outras combinações conforme a preferência de cada um e
encontrarem qual o valor que eles pagariam nessa lanchonete pelo respectivo pedido.
\end{enumerate}

\colorsec{Matemática -- Módulo 7 -- Treino}

\begin{enumerate}
\item
a) Incorreta. Nesse caso, trata-se da chance de escolha de apenas um dia da semana.
b) Correta. 
Dias da semana: 7.
Escolhas determinadas: 2.
Probabilidade: 2/7.
c) Incorreta. Essa seria a chance de uma escolha dentre 5 possibilidades.
d) Incorreta. Nesse caso, as chances são maiores do que as reais.

\item
a) Correta. 
Total de pessoas: 28 + 7 = 35.
Número de garçons: 7.
Probabilidade: 7/35 = 1/5 = 0,2 = 20\%.
b) Incorreta. Trata-de de mais que o dobro das chances reais.
c) Incorreta. Trata-se de 7 chances em 10.
d) Incorreta. Trata-se de totais chances de ocorrência, como se todas as pessoas presentes no restaurantes fossem garçons.

\item
a) Incorreta. Trata-se de apenas um quarto das chances reais.
b) Incorreta. Trata-se de apenas metade das chances reais.
c) Incorreta. Trata-se de aproximadamente uma chance em três possibilidades.
d) Correta. Como Carol participará apenas do sorteio final, ela pode ser sorteada ou
não e isso no leva a concluir que ela terá 50\% de chance de iniciar a
disputa.
\end{enumerate}

\colorsec{Matemática -- Módulo 8 -- Treino}

\begin{enumerate}
\item
a) Incorreta. No dia 05/03 foram vendidos apenas 18 sanduíches.
b) Incorreta. No dia 06/03 foram vendidos apenas 16 sanduíches.
c) Correta. O dia com a maior quantidade vendas foi o dia 07/03, com 25 produtos
vendidos.
d) Incorreta. No dia 08/03 foram vendidos apenas 14 sanduíches, tendo sido o dia com as menores vendas.

\item
a) Incorreta. Trata-se do mês com as menores vendas.
b) Incorreta. 45 não é o triplo das vendas de nenhum dos outros meses.
c) Incorreta. 62 não é o triplo das vendas de nenhum dos outros meses.
d) Correta. Em julho, com uma venda de 72 bolas, que é o triplo das 24 unidades vendidas em abril.

\item
a) Incorreta. Trata-se do número de crianças de 4 a 6 anos.
b) Incorreta. Trata-se do número de crianças de 7 a 9 anos, apenas.
c) Incorreta. Trata-se da soma entre as quantidades de crianças de 4 a 6 anos e de 10 a 12 anos.
d) Correta. Segundo o gráfico apresentado, 12 + 9 = 21 crianças de 7 a 12 anos
visitaram a loja.
\end{enumerate}

\colorsec{Matemática -- Módulo 9 -- Treino}

\begin{enumerate}
\item
a) Incorreta. 3/3 representa todos os bombons.
b) Incorreta. 2/5 é menos do que a metade.
c) Correta. Chocolate branco/ chocolate ao leite = 4/8 = ½.
d) Incorreta. 4/6 é mais do que a metade.

Professor, reforce com os alunos a diferença entre a razão de componentes e a razão com relação ao total.

\item
a) Incorreta. As partes não são iguais.
b) Incorreta. As partes não são iguais entre si.
c) Correta. A única que contém divisões iguais e que condizem com a divisão é a figura que aparece nesta alternativa.
d) Incorreta. Há, de fato, divisão em três partes, mas elas não são iguais.

Professor, reforce bastante com os alunos que as partes devem ser de
mesmo tamanho para representarem uma parte do todo.

\item
a) Incorreta. 8 não representam 25\% dos alunos da turma.
b) Correta. 25\% de 36 = ¼ x36 = 9 alunos por sessão.
c) Incorreta. 10 representam mais do que 25\% dos alunos.
d) Incorreta. 11 representam mais do que 25\% dos alunos.
\end{enumerate}

\colorsec{Matemática -- Módulo 10 -- Treino}

\begin{enumerate}
\item
a) Incorreta. Esseé o valor de uma única pizza.
b) Incorreta. Trata-se do valor, já mencionado, de duas pizzas.
c) Incorreta. Esse é o valor aproximado de 3 pizzas.
d) Correta. Valor de cada pizza: R\$ 81,60/2 = R\$ 40,80. Valor de 6 pizzas: 6 x 40,80 = R\$ 244,80.

\item
a) Incorreta. 9 colheres fazem 3 receitas, ou seja, 24 cafezinhos.
b) Correta. Para 48 cafezinhos, ela terá que fazer 6 receitas. Sendo assim, basta multiplicar a quantidade de colheres de pó de café para 8 cafezinhos também por
6. 3 x 6 = 18 colheres de sopa de pó de café.
c) 24 colheres fariam 64 cafés.
d) 48 colheres fariam 128 cafés.

\item
a) Incorreta. Em 1 minuto imprimem-se 100 folhas.
b) Incorreta. Em 15 minutos imprimem-se 1500 folhas.
c) Correta. Quantidade de folhas para 700 jornais: 5 x 700 = 3.500 folhas. Tempo gasto para a produção de 3.500 folhas = 3.500/100 = 35 minutos.
d) Incorreta. Em 55 minutos imprimem-se 5500 folhas.
\end{enumerate}

\colorsec{Matemática -- Módulo 11 -- Treino}

\begin{enumerate}
\item
a) Incorreta. Isso aconteceria com uso de apenas uma cor, com 8 opções de cores.
b) Incorreta. Trata-se de um número muito menor de opções do que as possibilidades reais.
c) Correta. 8 x 7 = 56 combinações diferentes de cores para a bandeira.
d) Incorreta. Não chega a haver tantas combinações assim.

\item
a) Correta. 6 x 18 = 108 possibilidades.
b) Incorreta. Trata-se de um número muito menor de opções do que as possibilidades reais.
c) Incorreta. Trata-se de quase um décimos das possibilidades reais.
d) Incorreta. Não chega a haver tantas combinações assim.

\item
a) Incorreta. Trata-se de um número muito ptóximo daquele das possibilidades reais, mas ainda um pouco menor.
b) Correta. 
Sair de X e passar por S antes de chegar a Z: 3 x 2 = 6;
Sair de X passar por S e Y antes de chegar a Z: 3 x 2 x 2 = 12;
Sair de X passar por Y antes de chegar a Z: 1 x 2 = 2;
Sair de X passar por R antes de chegar a Z: 3 x 1 = 3;
Sair de X passar por R e Y antes chegar a Z: 3 x 3 x 2 = 18;
Total: 6 + 12 + 2 + 3 + 18 = 41 caminhos diferentes.
c) Incorreta. É preciso analizar o diagrama para se chegar à conclusão.
d) Incorreta. Não chega a haver tantas combinações assim.
\end{enumerate}

\colorsec{Matemática -- Módulo 12 -- Treino}

\begin{enumerate}
\item
a) Correta. 100 -- 38,25 -- 21,55 = R\$ 40,20.
b) Incorreta. Para sobrar esse valor de troco, a compra deveria totalizar R\$ 47,60.
c) Incorreta. Para sobrar esse valor de troco, a compra deveria totalizar R\$ 41,40.
d) Correta. Seria impossível sobrar um troco maior do que o valor pago, de cem reais.

\item
a) Incorreta. Nesse caso o aluno terá multiplicado o valor de R\$ 3,75 por 5, sem considerar o valor diferenciado da primeira hora.
b) Correta. 1 x 8 + (5 -- 1) x 3,75 = R\$ 23,00.
c) Incorreta. Nesse caso o aluno terá considerado o valor da primeira hora, mais 5 horas com o valor de R\$ 3,75.
d) Incorreta. Nesse caso o aluno terá multiplicado o valor da primeira hora pelas 5 horas.

\item
a) Incorreta. Nesse caso o aluno terá considerado a quantidade de um quarto da produção.
b) Correta. Metade do suco produzido: 248,40/2 = 124,20 = 124 200 ml.
c) Incorreta. Nesse caso o aluno terá multiplicado a produção por 2, em vez de dividir.
d) Incorreta. Nesse caso o aluno terá multiplicado a produção por 4, em vez de dividir por 2.
\end{enumerate}

\colorsec{Matemática -- Módulo 13 -- Treino}

\begin{enumerate}
\item
a) Incorreta: é necessário multiplicar o número de fileiras
pelo número de cadeiras, isto é: 25 X 8 e 6 X 6. 
b) Incorreta: é necessário multiplicar o número de fileiras
pelo número de cadeiras, isto é: 25 X 8 e 6 X 6.  
c) Correta: o número de fileiras foi adequadamente multiplicado 
pelo número de cadeiras, isto é: 25 X 8 e 6 X 6. 
d) Incorreta: é necessário multiplicar o número de fileiras
pelo número de cadeiras, isto é: 25 X 8 e 6 X 6.  
25 x 8 + 6 x 6.

\item
Para alcançar o valor total arrecadado com a 
venda de ingressos, é preciso somar a arrecadação de entradas inteiras
(87 X R\$ 26,00 = R\$ 2.262,00) com a de meias-entradas 
(65 X R\$ 13,00 = R\$ 845,00). O total é de R\$ 3.107,00. 
a) Incorreta: R\$ 2.262,00 corresponde apenas à arrecadação de entradas 
inteiras.  
b) Incorreta: R\$ 845,00 corresponde apenas à arrecadação de 
meias-entradas. 
c) Incorreta: R\$ 1.417,00 não corresponde à soma da arrecadação de 
entradas inteiras e meias-entradas.
d) Correta: R\$ 3.107,00 corresponde à soma da arrecadação de 
entradas inteiras e meias-entradas.

\item
a) Correta: (37 -- 5)/4 + [(2 x 5) + 6]/2 = (32)/4 + 16/2 = 8 + 8 = 16.
b) Incorreta: o valor pensado por Marcel é 16.
c) Incorreta: o valor pensado por Marcel é 16.
d) Incorreta: o valor pensado por Marcel é 16.
\end{enumerate}

\colorsec{Matemática -- Módulo 14 -- Treino}

\begin{enumerate}
\item
No enunciado, afirma-se que a empresa tem 350 funcionários; esse 
é, portanto, o número que representa a população. A amostra, por sua vez, 
se refere apenas àqueles que participaram da pesquisa, isto é, 75
funcionários.
a) Incorreta. A população é de 350 funcionários (não 10); a amostra é de
75 (não de 350).
b) Incorreta. A amostra correta é de 75 funcionários, não de 10.
c) Incorreta. Nessa alternativa, os números de população e de amostra
estão invertidos.
d) Correta: A população é de 350 funcionários; a amostra corresponde ao
número de candidatos que participaram da pesquisa: 75.

\item
a) Incorreta. As variáveis qualitativas não são representadas por 
meio de quantidades. O número do irmãos não pode, assim, ser uma variável 
qualitativa.
b) Correta. As variáveis qualitativas discretas só podem ser medidas
com números inteiros, sem frações. É o caso do número de irmãos. 
c) Incorreta. As variáveis qualitativas contínuas são fracionáveis,
divisíveis. Não é, claramente, o caso do número de irmãos. 
d) Incorretas. As definições de variáveis quantitativas e qualitativas
excluem a possibilidade de haver quaisquer variáveis que participem das
duas categorias ao mesmo tempo.

\item
a) Incorreta. Existem 10 notas maiores ou iguais a 7 (sete) e 10 notas menores do que 7 (sete).
b) Correta. Entre as notas apresentadas, temos 10 notas maiores ou iguais
a 7 (sete).
c) Incorreta. Incorreta. Existem 10 notas maiores ou iguais a 7 (sete) e 10 notas menores do que 7 (sete).
d) Incorreta. Incorreta. Existem 10 notas maiores ou iguais a 7 (sete) e 10 notas menores do que 7 (sete).
\end{enumerate}

\colorsec{Matemática -- Módulo 15 -- Treino}

\begin{enumerate}
\item
a) Correta. O candidato A cumpre os dois critérios exigidos pelas
regras do processo seletivo: suas notas são todas superiores a 30, e três
delas são iguais (as de Português, Matemática e Direito).
b) Incorreta. Embora as notas do candidato B sejam superiores a 30, 
cumprindo o primeiro dos critérios do processo seletivo, apenas duas 
delas são idênticas (as de Português e Direito), de modo que esse 
B perde para A, que teve três notas iguais. 
c) Incorreta. O candidato C não pode ser aprovado, porque teve uma nota
inferior a 30.
d) Incorreta. O candidato D não pode ser aprovado, porque teve duas notas
inferiores a 30.

\item
a) Incorreta. O número de alunos do 4º ano é 86, não 60.
b) Correta. O número de alunos do 4º ano é igual à soma do número de
alunos das turmas A, B e C: 32 + 29 + 25 = 86.
c) Incorreta. O número de alunos do 4º ano é 86, não 91.
d) Incorreta. O número de alunos do 4º ano é 86, não 150.

\item
A razão do número do livros retirados em abril (205) e junho 
(210) é expressa da seguinte maneira: 205/210.  
a) Incorreta. 
b) Incorreta.
c) Correta. 205/210 = 51/52.
b) Incorreta.
\end{enumerate}