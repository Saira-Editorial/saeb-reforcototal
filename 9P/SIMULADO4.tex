\chapter*{Simulado}

Leia o texto para responder às questões de 1 a 3. Trata-se de uma
dissertação sobre o tema ``Por que a produção cultural não é valorizada
como a produção industrial?''.

\begin{quote}
\textbf{Cultura x indústria}

%Imagem ilustrativa que pode ser cortada: \url{https://www.freepik.com/free-photo/greek-statue-double-color-exposure-remixed-media_17227603.htm\#query=culture\&position=15\&from_view=search\&track=sph}

A valorização da produção industrial em nossa sociedade é evidente, uma
vez que a indústria é frequentemente considerada o pilar do
desenvolvimento econômico. No entanto, a produção cultural muitas vezes
é relegada a um papel secundário, não recebendo a devida valorização e
reconhecimento. Isso se dá porque a indústria produz bens materiais e a
cultura, imateriais.

A produção industrial, caracterizada pela fabricação em massa de bens
tangíveis, é valorizada devido à sua capacidade de gerar empregos,
impulsionar o crescimento econômico e satisfazer as necessidades
materiais da sociedade. Por outro lado, a produção cultural, que abrange
áreas como arte, música, literatura e cinema, é infelizmente vista como
uma atividade subjetiva e não prioritária.

Uma das razões para essa disparidade reside na forma como a sociedade
enxerga o valor econômico. A produção industrial é diretamente
mensurável em termos de números, como o PIB, taxas de emprego e lucro
das empresas. Por outro lado, o valor da produção cultural é mais
difícil de quantificar, uma vez que muitas vezes é baseado em aspectos
subjetivos, como a apreciação estética e a expressão criativa.

Além disso, a produção cultural muitas vezes enfrenta desafios em termos
de distribuição e acesso. A produção industrial tem uma presença física
e pode ser facilmente comercializada e distribuída em larga escala. Por
outro lado, a produção cultural, especialmente em formatos como arte e
música, pode enfrentar barreiras geográficas, dificuldades na
comercialização e falta de acesso para determinados públicos.

Outro fator relevante é a falta de compreensão e educação sobre a
importância da produção cultural. A educação tradicional muitas vezes dá
mais ênfase às disciplinas científicas e técnicas, negligenciando a
importância das artes e humanidades. Essa falta de compreensão contribui
para a desvalorização da produção cultural, pois não se reconhece
plenamente o impacto que ela tem no enriquecimento da sociedade e no
desenvolvimento humano.

É essencial que a sociedade reconheça a importância da produção cultural
e a valorize de forma equilibrada em relação à produção industrial. A
cultura é um pilar fundamental da identidade de um povo, promove a
diversidade, estimula a criatividade e é essencial para a expressão e
reflexão sobre os valores humanos. Para alcançar essa valorização, é
necessário promover uma mudança de mentalidade, fomentando a educação e
a compreensão sobre a produção cultural. É fundamental apoiar e investir
em projetos culturais, oferecer espaços de exposição e garantir o acesso
igualitário à cultura.
\end{quote}

\num{1} Quais são os parágrafos, no texto, dedicados à argumentação?

\begin{escolha}
\item
  3 e 4.
\item
  4 e 5.
\item
  De 2 a 5.
\item
  De 1 a 4.
\end{escolha}

\pagebreak
\num{2} A tese defendida no texto é a de que

\begin{escolha}
\item
  a diferença de valorização está no fato de que a indústria produz bens
  palpáveis e a cultura, impalpáveis.
\item
  a distância entre indústria e cultura está na rapidez da produção,
  sendo a indústria bem mais ágil.
\item
  a divergência entre valorização da cultura e da indústria está no tipo
  de mão de obra desses dois segmentos.
\item
  a diferença de valorização não está nas produções desses dois
  segmentos, mas no público-alvo.
\end{escolha}

\num{3} No segundo parágrafo, aparece uma palavra que funciona como
modalizador. Ela antecipa o fato de que, na concepção do enunciador, a
cultura é injustamente desvalorizada. Trata-se da palavra

\begin{escolha}
\item
  massa
\item
  impulsionar
\item
  sociedade
\item
  infelizmente
\end{escolha}

\num{4} Leia um trecho de notícia.

%Imagem ilustrativa que pode ser cortada: \url{https://www.freepik.com/free-photo/side-view-woman-singing-microphone_9521259.htm\#query=female\%20singer\&position=2\&from_view=search\&track=ais}

\begin{quote}
Em um espetáculo musical surpreendente, a renomada cantora Maria Silva
encantou o público presente com sua voz única e emocionante. A plateia,
extasiada, pôde apreciar a performance espetacular da artista, que
conquistou a todos com sua técnica impecável e presença de palco
cativante. Essa noite inesquecível reforçou o talento excepcional de
Maria Silva.
\end{quote}

Há, no trecho, quanto modalizadores apreciativos?

\begin{escolha}
\item
  Nenhum.
\item
  Um.
\item
  Menos de três.
\item
  Mais de quatro.
\end{escolha}

\pagebreak
Leia o texto para responder às questões de 5 a 7.

\begin{quote}
\textbf{Ancestralidade}

%Imagem ilustrativa que pode ser cortada: \url{https://www.freepik.com/free-photo/front-view-brown-box-light-floor_8853437.htm\#query=ba\%C3\%BA\%20antigo\&position=8\&from_view=search\&track=ais}

Havia uma pequena cidade cercada por montanhas majestosas, onde vivia
uma jovem chamada Isabela. Ela sempre se sentiu intrigada com a história
de seus antepassados, mas nunca teve a oportunidade de mergulhar mais
fundo em suas raízes familiares.

Um dia, Isabela encontrou um antigo baú no sótão de sua casa. Dentro
dele, havia fotografias amareladas, cartas empoeiradas e um diário
escrito por sua bisavó. Curiosa, ela começou a ler as palavras
cuidadosamente escritas naquelas páginas.

À medida que mergulhava nas histórias contadas por sua bisavó, ela
sentia uma conexão profunda com aqueles que vieram antes dela. As
dificuldades, os triunfos, os amores e as perdas de seus antepassados
ganhavam vida nas palavras do diário. Lá pela página dez, a menina
começou a perceber como essas histórias moldaram sua própria identidade.
Ela descobriu que tinha um talento natural para a música, assim como seu
avô, que era um exímio violinista. Percebeu, também, que sua paixão pela
natureza e pela preservação do meio ambiente vinha de suas raízes
indígenas, passadas de geração em geração.

Conforme avançava na leitura, ela também percebia a importância de
honrar e preservar a herança deixada por seus antepassados. Bela
aprendeu a valorizar as tradições culturais, as histórias transmitidas
oralmente e a sabedoria acumulada ao longo dos tempos. Com o tempo,
Isabela se tornou uma contadora de histórias, levando adiante as
narrativas de sua família e da comunidade em que vivia. Ela
compartilhava as lições aprendidas com os jovens, para que também
pudessem reconhecer a importância de suas raízes e se sentir conectados
com a história de sua própria gente.

A mulher Isabela notou, então, que suas raízes eram um elo entre
passado, presente e futuro. Ela compreendeu que as histórias e tradições
transmitidas de geração em geração eram um tesouro valioso que poderia
inspirar e guiar as pessoas ao longo de suas vidas. Assim, descobriu que
a ancestralidade não era apenas uma questão de conhecimento histórico,
mas uma fonte de força, identidade e inspiração. Ela aprendeu que
conhecer e honrar suas raízes era essencial para sua própria formação
como pessoa, permitindo-lhe compreender sua história pessoal e encontrar
seu lugar no mundo.

Com um coração cheio de gratidão e sabedoria, Isabela continuou sua
jornada, compartilhando as histórias de seus antepassados e lembrando a
todos que a importância da ancestralidade reside no poder de nutrir e
enriquecer a alma, conectando-se com as raízes que nos tornam quem
somos.

\fonte{Diva Lopes}
\end{quote}

\num{5} O texto, que é uma narrativa, enquadra-se no gênero

\begin{escolha}
\item
  crônica.
\item
  conto.
\item
  novela.
\item
  romance.
\end{escolha}

\pagebreak
\num{6} Qual é o recorte de tempo que se passa no conto?

\begin{escolha}
\item
  Alguns dias, apenas.
\item
  Meses, nos quais Isabela não chega a completar um ano de vida.
\item
  Anos, nos quais Isabela amadurece e descobre seu talento.
\item
  Passa-se toda a vida de Isabela.
\end{escolha}

\num{7} Quanto mais Isabela mergulha nas histórias, mais ela descobre que

\begin{escolha}
\item
  não há memórias de seus antepassados.
\item
  sua bisavó era excelente inventora de histórias.
\item
  sua história é independente da de seus antepassados.
\item
  a história de seus ancestrais influencia em sua formação como mulher.
\end{escolha}

\num{8} Leia a definição.

\begin{quote}
Local onde se encontram vestígios das pessoas que viveram no
passado, os quais revelam as atividades e a cultura de homens e
mulheres, identificadas nos restos de construções, alimentação,
instrumentos de trabalho, armas, enfeites e pinturas.
\end{quote}

A definição exposta é a de patrimônio

\begin{escolha}
\item
  arqueológico.
\item
  paisagístico.
\item
  etnográfico.
\item
  das artes aplicadas.
\end{escolha}


\num{9} Leia o texto.

\begin{quote}
\textbf{O artists Piet Mondrian}

Piet Mondrian foi um dos principais artistas de determinado movimento artístico. Nascido em 1872, na 
Holanda, Mondrian foi um pioneiro na busca por uma linguagem visual livre das restrições da representação 
figurativa.

Esse movimento surgiu no início do século XX como uma reação à arte tradicional e à representação realista. 
Os artistas buscavam explorar a arte além da mera imitação da natureza, buscando formas de expressão mais 
puras e universais.

Mondrian foi um dos principais expoentes do movimento, desenvolvendo um estilo único e inovador. Sua obra é 
caracterizada por linhas retas, formas geométricas e o uso de cores primárias (vermelho, azul e amarelo) 
juntamente com o branco e o preto. Ele acreditava que essa redução da forma e do uso limitado de cores 
poderia levar a uma arte mais pura e espiritual.

Ao longo de sua carreira, Mondrian foi refinando sua estética, buscando uma expressão cada vez mais 
simplificada. Suas pinturas evoluíram para composições abstratas compostas de grades retangulares e 
quadrados coloridos. Ele acreditava que, ao simplificar as formas e limitar as cores, poderia alcançar uma 
harmonia universal.

Além de suas pinturas, Mondrian também desenvolveu teorias sobre esse tipo de arte, escrevendo sobre seus 
princípios estéticos e sua filosofia. Ele acreditava que essa arte tinha o potencial de transcender as 
limitações da realidade física e se conectar com uma ordem espiritual mais elevada.

A contribuição de Mondrian para o movimento foi imensa, influenciando artistas posteriores e deixando um legado duradouro na história da arte. Sua busca pela pureza estética e sua abordagem inovadora continuam a inspirar e intrigar admiradores até os dias de hoje.

\fonte{Texto escrito para este material.}
\end{quote}

O movimento a que se faz referência no texto é o

\begin{escolha}
\item
  Cubismo.
  \item
  Impressionismo.
\item
  Expressionismo.
\item
  Abstracionismo.
\end{escolha}

\num{10} Leia o texto.

\begin{quote}
\textbf{Explorando novos horizontes criativos}

A integração da arte com a tecnologia abre caminho para a criação e a divulgação de formatos artísticos 
novos e fascinantes. O avanço tecnológico, especialmente no campo dos computadores, tem possibilitado o 
surgimento de expressões artísticas que anteriormente eram inimagináveis. As instalações artísticas, que 
combinam imagens e sons, são apenas um exemplo de como a arte tem se libertado das formas convencionais, 
como pinturas e esculturas, e tem se manifestado de maneiras inovadoras e surpreendentes. Ao aproveitar as 
novas ferramentas tecnológicas, a arte também tem encontrado seu espaço fora dos museus, ganhando vida nas 
ruas. Dispositivos cada vez menores e mais acessíveis permitem que os artistas exibam suas criações em 
locais abertos e inusitados. A fachada de um prédio, por exemplo, pode se transformar em uma tela 
multimídia para a expressão criativa.

\fonte{Fonte de pesquisa: Artout. Arte e tecnologia: como elas se unem e seus benefícios.
Disponível em: \emph{https://artout.com.br/arte-e-tecnologia}. Acesso em: 16 mar. 2023.}
\end{quote}

Assinale a alternativa que contém uma característica específica das
instalações sonoras.

\begin{escolha}
\item
  Obras tridimensionais inseridas em espaços específicos.
\item
  Obras que podem explorar recursos digitais.
\item
  Obras que podem ser instaladas fora das galerias e dos museus.
\item
  Obras que exploram silêncio, ruído e música.
\end{escolha}

\num{11} Leia o texto.

\begin{quote}
I had never heard Joe read aloud to any greater extent than this
monosyllable, and I had observed at church last Sunday, when I
accidentally held our Prayer-Book upside down, that it seemed to suit
his convenience quite as well as if it had been all right. Wishing to
embrace the present occasion of finding out whether in teaching Joe, I
should have to begin quite at the beginning, I said, ``Ah! But read the
rest, Jo.''

``The rest, eh, Pip?'' said Joe, looking at it with a slow, searching
eye, ``One, two, three. Why, here's three Js, and three Os, and three
J-O, Joes in it, Pip!''

\fonte{Charles Dickens. Great Expectations. Disponível em:
https://www.gutenberg.org/cache/epub/1400/pg1400-images.html. Acesso
em: 02 mar. 2023.}
\end{quote}

As duas personagens envolvidas no diálogo reproduzido acima são:

\begin{escolha}
\item Os e Js.

\item J-O e Sunday.

\item Pip e Prayer-Book.

\item Joe e Pip.
\end{escolha}

\num{12} Leia o texto.

\begin{quote}
\textbf{Astronomers Discover Massive Early Galaxies Dating Back to 600
Million Years After Big Bang}

Scientists studying the universe have recently detected what appears to
be colossal galaxies that existed approximately 600 million years after
the Big Bang. These findings suggest that the early universe may have
undergone a rapid development that led to the formation of these
enormous galaxies. While the James Webb Space Telescope has identified
galaxies that are even older, dating back to a mere 300 million years
after the universe's beginning, it is the vast size and advanced age of
these six massive galaxies that have astonished astronomers. These
discoveries were reported on Wednesday.

\fonte{Fonte de pesquisa: The Associate Press. NPR. The
Webb telescope finds surprisingly massive galaxies from the universe's
youth. Disponível em:
\emph{https://www.npr.org/2023/02/22/1158793897/webb-telescope-huge-early-galaxies-big-bang}.
Acesso em: 02 mar. 2023.}
\end{quote}

O texto trata

\begin{escolha}
\item da natureza da galáxia que habitamos.

\item da descoberta de galáxias pelo telescópio James Webb.

\item do funcionamento do telescópio James Webb.

\item da idade de nosso universo.
\end{escolha}

\pagebreak
\num{13} Leia o texto.

\begin{quote}
\textbf{Preserved Remains of Dinosaur's Last Meal Found in Fossilized
Microraptor}

Around 120 million years ago, during the Cretaceous Period, a dinosaur
devoured its final meal: a tiny mammal measuring the size of a mouse.
Remarkably, the remains of the meal still exist to this day. An
observant researcher noticed the preserved foot of the mammal inside the
abdomen of a fossilized Microraptor zhaoianus, a feathered therapod that
measured less than one meter (or three feet) in length.

\fonte{Fonte de pesquisa: Katie Hunt. CNN. Rare evidence that dinosaurs feasted on mammals uncovered. Disponível em: *https://edition.cnn.com/2022/12/26/world/dinosaur-mammal-last-meal-scn/index.html*. Acesso em: 02 mar. 2023.}
\end{quote}

De acordo com o texto, os pesquisadores encontraram

\begin{escolha}
\item evidências de que dinossauros consumiam mamíferos.

\item uma nova espécie de dinossauros.

\item novas espécies de mamíferos.

\item espécies de dinossauros minúsculas.
\end{escolha}

\num{14} Leia o texto.

\begin{quote}
Por volta do século XIX, a Grã-Bretanha havia se tornado o maior produtor mundial de tecidos de algodão, superando a Índia. Isso não ocorreu por acaso, mas sim devido a uma mudança do centro de beneficiamento do algodão da Índia para a Inglaterra. Esse declínio na produção indiana não se deveu à situação política do país, mas sim às conquistas europeias, que trouxeram novas formas de organização do trabalho e tecnologias avançadas.

\fonte{Texto escrito para este material.}
\end{quote}

O texto permite inferir que, em meados do Século XIX, a

\begin{escolha}
\item  Índia manteve o mercado consumidor local dos seus tecidos.

\item  produção inglesa era insuficiente para o mercado consumidor indiano.

\item  produção indiana perdeu em competitividade frente à produção inglesa.

\item  Inglaterra não acessou o mercado consumidor de tecidos da região do Oceano Índico.
\end{escolha}


\num{15} Leia o texto.

\begin{quote}
\textbf{Mecanização no campo muda as relações de trabalho}

A introdução de máquinas no campo está transformando as dinâmicas de trabalho no setor agrícola brasileiro. O trabalhador rural que antes era contratado para realizar o plantio e a colheita manual de culturas como cana-de-açúcar, café e algodão, agora está operando máquinas. Consequentemente, o antigo boia-fria migra para setores urbanos, como a construção civil. De acordo com especialistas, o crescimento econômico que acompanha o aumento da produção tem compensado os efeitos da tecnologia no emprego, pois uma única máquina pode substituir mais de 100 trabalhadores.

\fonte{Fonte de pesquisa: Marinella Castro. Estado de Minas. Mecanização no campo muda as relações de trabalho. Disponível em: https://www.em.com.br/app/noticia/economia/2013/01/14/internas\_economia,343131/mecanizacao-no-campo-muda-as-relacoes-de-trabalho.shtml. Acesso em: 05 maio 2023.}
\end{quote}

No caso apontado, a mecanização do campo foi responsável pela

\begin{escolha}
\item  realocação dos trabalhadores.

\item  diminuição dos postos de trabalho.

\item  manutenção de atividades manuais.

\item  flexibilização das atividades laborais.
\end{escolha}

\num{16} Leia o texto.

\begin{quote}
Em diversas cidades em que a governabilidade é questionável, o Estado frequentemente não é capaz de assegurar a manutenção da lei e da ordem, bem como suprir as necessidades básicas de segurança. Em decorrência disso, diversas alternativas ilegais de sistemas de segurança acabam sendo estabelecidas.

\fonte{Texto escrito para este material.}
\end{quote}

A situação descrita no texto exemplifica

\begin{escolha}
\item  o vácuo de poder deixado pelo Estado.

\item  o gerenciamento popular de demandas comuns.

\item  a concessão de legitimidade ao crime organizado.

\item  a opção deliberada no oferecimento de segurança pública.
\end{escolha}

\num{17} Leia o texto, sobre o Movimento dos Trabalhadores Rurais Sem Terra (MST).

\begin{quote}
Organizado nacionalmente, ele se constitui no principal
movimento social no campo e busca, através das ocupações de terras,
criar fatos políticos que mobilizem e sensibilizem os governantes para a
necessidade da reforma agrária. Esse movimento utiliza-se também das
caminhadas pelas estradas até as capitais, onde se realizam
manifestações e ocupações de repartições públicas para pressionar os
governos.

\fonte{Ariovaldo Umbelino de Oliveira. Os movimentos sociais no campo e a reforma agrária no Brasil. In: \textit{Geografia do Brasil}. São Paulo: EDUSP, 2019.}
\end{quote}

Depreende-se que a finalidade do movimento social citado compreende a

\begin{escolha}
\item  correção de uma assimetria criada pelo Estado brasileiro ao longo do tempo.

\item  promoção de insegurança aos produtores rurais do agronegócio nacional.

\item  ampliação das relações entre pequenos e grandes produtores agrícolas.

\item  centralização e o direcionamento da produção agrícola nacional.
\end{escolha}

\num{18} Leia o texto.

\begin{quote}
\textbf{Ditadura militar contribuiu para genocídio dos povos indígenas}

{[}...{]}

Em depoimento à Comissão Nacional da Verdade, Davi Kopenawa, líder
yanomami, relembrou o descaso do governo durante a realização de grandes
obras. Segundo a liderança, as estradas abriram caminho para os
invasores garimpeiros e fazendeiros.

``Eu não sabia que o governo vinha deixar estrada na terra yanomami. A
autoridade não avisou antes de destruir o nosso meio ambiente, antes
de matar o nosso povo yanomami. A estrada é o caminho de invasores
garimpeiros, fazendeiros, pescadores e caçadores''.

A tomada das terras indígenas para ampliação da fronteira agrícola e
para exploração mineral e de energia foi um dos eixos do Plano de
Integração Nacional dos militares. {[}...{]}

Já na década de 1980, a situação se agravou com a invasão de cerca de
40 mil garimpeiros na região. Uma campanha internacional exigiu que a
ditadura fosse responsabilizada pelo genocídio yanomami. {[}...{]}

\fonte{Gésio Passos. Agência Brasil. Ditadura militar contribuiu para genocídio dos povos indígenas. Disponível em: https://agenciabrasil.ebc.com.br/direitos-humanos/noticia/2023-03/ditadura-militar-contribuiu-para-genocidio-dos-povos-indigenas. Acesso em: 05 maio 2023.}
\end{quote}

Para evitar os problemas relatados na reportagem, a solução possível
seria estabelecer

\begin{escolha}
\item  a urbanização do território indígena.

\item  o controle territorial por parte dos povos indígenas.

\item  a consolidação de uma economia de mercado na região.

\item  a distribuição de terras agricultáveis a camponeses da região.
\end{escolha}

