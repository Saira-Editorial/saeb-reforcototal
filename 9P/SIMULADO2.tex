\chapter[Simulado 2]{Simulado}

Leia o trecho de uma entrevista com uma escritora. Depois, responda às
questões de 1 a 3.

\begin{quote}
\textbf{Margaret Atwood, autora de O conto de aia: ``Para mim é natural
ser uma `raposa velha' malvada''}

{[}\ldots{}{]} Mais de três décadas depois de sua publicação, \emph{O
conto da aia} retornou às listas de livros mais vendidos {[}\ldots{}{]}.
A distopia de Margaret Atwood (Ottawa, 1939) voltou com força, e a
prolífica autora -- que publicou seu primeiro livro de poemas em 1969
{[}\ldots{}{]} -- se decidiu a escrever a sequência.

{[}\ldots{}{]}

P. A senhora escreveu que existem romances que enfeitiçam os leitores, e
outros a seus escritores, e que com \emph{O conto da aia} as duas coisas
aconteceram. Esse livro a perseguiu?

R. Sim, com certeza. Há obras que resistem a ser somente livros e foi
assim com esse romance. Também há personagens que escapam das páginas
como Dom Quixote, que passou a ser uma metáfora e está aí dando voltas
pelo mundo.

Andrea Aguilar. Margaret Atwood, autora de O conto de aia: ``Para mim é
natural ser uma `raposa velha' malvada''. Disponível em:
https://brasil.elpais.com/brasil/2019/09/06/cultura/1567786560\_937893.html.
Acesso em: 7 abr. 2023.
\end{quote}

\num{1} Compreende-se que o primeiro parágrafo

\begin{escolha}
\item mostra como serão as perguntas e as respostas da dinâmica de
conversa entre os participantes, iniciando o assunto de maneira
objetiva.

\item mostra uma avaliação dos argumentos da entrevistada, evidenciando a
opinião da entrevistadora sobre o assunto tratado no texto.

\item contextualiza para os leitores quem é a entrevistada, deixando mais
evidente qual é a importância de entrevistá-la para o veículo
jornalístico.

\item justifica a elaboração das perguntas que aparecem posteriormente,
pois a entrevistadora mostra quais objetivos tinha com a entrevista.
\end{escolha}

\num{2} Como se organizam, por escrito, os turnos de fala da entrevista?

\begin{escolha}
\item perguntas e respostas aparecem separadas - primeiro uma bateria de
perguntas, depois suas respostas.

\item perguntas e respostas aparecem em cores diferentes para que se
esclareçam suas funções.

\item perguntas e respostas se alternam, com indicações simples e diretas
para se diferenciarem entre si.

\item aparecem apenas perguntas, sendo que as respostas ficam a cargo do
leitor.
\end{escolha}

\pagebreak
\num{3} A citação de D. Quixote, pela entrevistada, serve como
argumentação por meio de

%Imagem ilustrativa que pode ser cortada: \url{https://www.freepik.com/free-vector/watercolor-don-quixote-illustration_34658458.htm\#query=don\%20quixote\&position=5\&from_view=search\&track=ais}

\begin{escolha}
\item fato concreto.

\item exemplificação.

\item citação de autoridade - D. Quixote, nesse caso.

\item contra-argumentação.

\end{escolha}

Leia um trecho do artigo de divulgação científica para responder às
questões 4 e 5.

\begin{quote}
\textbf{Inteligência artificial a favor do coração}

%Imagem ilustrativa que pode ser cortada: \url{https://www.freepik.com/free-photo/person-holding-anatomic-heart-model-educational-purpose_34136869.htm\#query=human\%20heart\&position=3\&from_view=search\&track=ais}

{[}...{]} projeto da Universidade Estadual de Campinas (Unicamp)
{[}\ldots{}{]} pode reverter em economia de R\$ 50 milhões por ano para
o Sistema Único demSaúde (SUS). Conduzido pelo Laboratório Aterolab da
Faculdade de Ciências Médicas (FCM), o projeto visa identificar os
pacientes de doenças coronarianas crônicas com maior risco de sofrerem
eventos clínicos adversos no intervalo de um ano. A pesquisa, orientada
pelo cardiologista Andrei Sposito, coordenador do Aterolab, foi premiada
pela Sociedade Brasileira de Cardiologia e pelo Congresso Europeu de
Inovação.

``No primeiro ano após um infarto, um a cada cinco pacientes pode sofrer
novo infarto ou, mesmo, morte súbita. Muitas tentativas já foram feitas
para identificar quem é esse paciente que corre mais risco'', diz o
cardiologista. {[}...{]}

\fonte{Suzel Tunes. Inteligência artificial a favor do coração. Disponível em:
https://revistapesquisa.fapesp.br/inteligencia-artificial-a-favor-do-coracao/.
Acesso em: 8 abr. 2023.}
\end{quote}

\num{4} Percebe-se que a fala do cardiologista é uma citação direta
porque

\begin{escolha}
\item apresenta um verbo de elocução após a fala.

\item está entre aspas com indicação de quem deu a declaração.

\item é composta pelas palavras do autor do texto.

\item mostra a ideia do especialista sem utilizar a fala literal dele.

\end{escolha}

\num{5} Para falar a favor da importância do estudo citado

\begin{escolha}
\item utiliza-se o nome do especialista que o comanda, sem que haja
citação de benefícios financeiros.

\item foca-se na ideia de que os pacientes podem sobreviver a infartos,
mesmo que eles ocorram.

\item apresentam-se dados numéricos dos resultados do estudo, organizados
em gráficos.

\item mostra-se que, além dos benefícios à saúde da população, há também
ganhos econômicos.
\end{escolha}


Leia um trecho de uma notícia sobre as epidemias. Depois, responda às
questões 6 e 7.

\begin{quote}
\textbf{Como acabam as epidemias}

{[}...{]} durante a gripe de 1918 um pastor de cabras {[}...{]} levou
seus cinco filhos para um monte onde deveriam permanecer escondidos de
um mal que estava dizimando a cidadezinha. {[}...{]}

Na época, como em grande parte das epidemias da história da humanidade,
o distanciamento social foi um modo de reduzir os contágios. Quando um
determinado número de pessoas já superou a doença e está imune a ela, o
contágio fica mais difícil, e a enfermidade míngua. Historicamente, esse
é o final das epidemias. ``Às vezes acontece isso'', explica José
Prieto, catedrático de microbiologia da Universidade Complutense de
Madri {[}...{]}

\fonte{Daniel Mediavilla. Como acabam as epidemias. El País. Disponível em:
https://brasil.elpais.com/ciencia/2020-03-26/como-acabam-as-epidemias.html.
Acesso em: 9 abr. 2023.}
\end{quote}

\num{6} Um dos modos de a reportagem mostrar credibilidade aos fatos
expostos é

\begin{escolha}
\item expor as ideias do especialista, de forma indireta, sobre as
maneiras de dar um fim ao contágio por vírus, o que retoma ações do
passado.

\item mostrar uma relação histórica entre os acontecimentos do passado e
do presente, como as formas adequadas de isolamento em 1918 e 2020.

\item transcrever a fala direta de um especialista no assunto, de modo a
embasar o que está sendo informado, como as palavras do microbiologista.

\item utilizar dos conhecimentos do especialista e dos dados históricos
para fundamentar as opiniões do autor da reportagem sobre as epidemias.
\end{escolha}

\num{7} Para estruturar o texto, o autor recorreu a

\begin{escolha}
\item expor nomes de personagens comuns que viveram as duas pandemias
citadas.

\item construir um paralelo entre dois eventos semelhantes, em momentos
distintos na história.

\item reproduzir depoimentos de cidadãos que viveram a pandemia de
COVID-19.

\item selecionar palavras raras do nosso léxico, como ``míngua''.

\end{escolha}

\num{8} Leia o texto.

\begin{quote}
\textbf{A Árvore do Teatro do Oprimido}

A Árvore do Teatro do Oprimido é um conceito central na metodologia desenvolvida pelo renomado dramaturgo e 
diretor brasileiro Augusto Boal. O Teatro do Oprimido é uma abordagem teatral que busca promover a 
conscientização social, a transformação e o empoderamento por meio da participação ativa dos espectadores.

A árvore é uma metáfora utilizada por Boal para ilustrar os princípios e os processos envolvidos nessa 
forma de teatro. Segundo o dramaturgo, a árvore tem raízes, tronco, galhos e folhas, cada uma representando diferentes aspectos da prática teatral.

\begin{itemize}
  \item As \textbf{raízes} da árvore representam o conhecimento das experiências e das realidades vividas 
  pelos participantes, suas histórias e suas identidades. Essas raízes são fundamentais para a construção 
  de uma base sólida para o trabalho teatral, conectando-se com as questões e os conflitos sociais que 
  afetam as comunidades.
  \item O \textbf{tronco} da árvore simboliza a técnica teatral, os métodos e as estratégias utilizadas 
  para expressar as experiências e os desafios vividos pelos oprimidos. Boal desenvolveu uma série de 
  exercícios e jogos teatrais que permitem que o público participe ativamente, explore diferentes 
  perspectivas e experimente soluções para os problemas apresentados.
  \item Os \textbf{galhos} da árvore representam as diferentes formas de atuação e intervenção social. Boal 
  defendia que o teatro não se limitasse ao palco, mas fosse levado para ruas, comunidades e espaços 
  públicos, permitindo que as vozes dos oprimidos fossem ouvidas e que novas possibilidades de ação e 
  mudança social fossem exploradas.
  \item As \textbf{folhas} da árvore simbolizam os resultados e as transformações que podem ocorrer a 
  partir da prática do Teatro do Oprimido. Essas transformações podem incluir uma maior conscientização, 
  empatia, diálogo e a busca por soluções coletivas para as injustiças e opressões existentes na sociedade.
\end{itemize}

A Árvore do Teatro do Oprimido é uma representação visual e conceitual poderosa que encapsula a visão de 
Boal sobre o papel do teatro como uma ferramenta de transformação social. Ela incentiva a participação 
ativa, a reflexão crítica e a busca por alternativas e soluções para desigualdades e injustiças presentes 
no mundo.

\fonte{Texto escrito para este material.}
\end{quote}

Nesse representação, no solo, encontra-se a representação

\begin{escolha}
\item
  das técnicas desenvolvidas pelo Teatro do Oprimido.
\item
  das técnicas consideradas como base do método.
\item
  das ações necessárias para a transformação social.
\item
  do conjunto de saberes acumulados pela humanidade.
\end{escolha}


\num{9} Leia o texto.

\begin{quote}
\textbf{Notação musical}

A notação musical convencional é um sistema de escrita utilizado para representar e comunicar informações 
musicais. É uma linguagem padronizada que permite aos músicos ler e interpretar uma composição musical.
A notação musical convencional utiliza uma combinação de símbolos, notas e marcas para representar os 
elementos musicais, como a altura do som, a duração das notas, o ritmo, a dinâmica e outros aspectos 
importantes da música.

Os elementos básicos da notação musical convencional incluem:

\begin{itemize}
  \item \textbf{Claves.} São símbolos colocados no início do pentagrama (a pauta musical) para indicar a 
  posição das notas na linha musical. Os dois tipos mais comuns são a clave de sol (usada para notas mais 
  agudas) e a clave de fá (usada para notas mais graves).
  \item \textbf{Pauta Musical.} É um conjunto de linhas horizontais paralelas, geralmente cinco, em que as 
  notas são escritas. Cada linha e espaço representa uma nota musical específica.
  \item \textbf{Notas.} São símbolos ovais ou cabeças de notas que são posicionados nas linhas e nos 
  espaços da pauta musical para indicar a altura do som. As notas podem ter diferentes formatos (como as 
  redondas, as quadradas e as pretas, por exemplo) e podem ter hastes (linhas verticais adicionais) para indicar sua duração.
  \item \textbf{Claves de ritmo.} São símbolos que indicam o ritmo e a divisão do tempo. Entre os mais 
  comuns, estão a semibreve (nota de maior duração), a mínima, a semínima, a colcheia e a semicolcheia.
  \item \textbf{Sinais de dinâmica.} São marcas escritas acima ou abaixo das notas que indicam a 
  intensidade ou o volume da música, como ``piano'' (suave) ou ``forte'' (alto). Outros sinais de dinâmica 
  incluem ``crescendo'' (aumentar gradualmente o volume) e ``diminuendo'' (diminuir gradualmente o volume).
\end{itemize}

Além desses elementos, a notação musical convencional também inclui símbolos para indicar articulações, expressões de estilo, mudanças de tempo, repetições e outros aspectos que ajudam na interpretação precisa da música.

\fonte{Texto escrito para este material.}
\end{quote}

Corresponde à descrição do texto a

\begin{escolha}
\item partitura.
\item notação textual.
\item tablatura numérica.
\item tablatura para violão.
\end{escolha}


\num{10} Leia o texto.

\begin{quote}
\textbf{Chiclete com banana}

Formada em 1980, a banda Chiclete com banana é muito famosa no Brasil. Com seu estilo contagiante e músicas 
animadas, eles se tornaram ícones do carnaval baiano. A banda teve uma carreira de sucesso, lançando 
diversos \textit{hits} e se destacando pela energia de suas performances ao vivo. Apesar das mudanças ao 
longo dos anos, seu legado na música brasileira permanece forte.

\fonte{Texto escrito para este material.}
\end{quote}

Que gênero musical brasileiro consagrou essa banda?

\begin{escolha}
\item
  Samba.
\item
  Sertanejo.
\item
  Funk.
\item
  Axé.
\end{escolha}

\pagebreak
\num{11} Leia o texto.

\begin{quote}
\textbf{Latin American Digital Media Outlets Turn to English to Reach New Audiences}

Despite Spanish being spoken by more than 500 million people worldwide,
English has become the dominant global language for communication in
entertainment, information, and business. In recent years, several
digital media outlets in Latin America, spanning from Mexico to Chile,
have opted to translate and produce content in English to broaden their
audience reach and ultimately boost profits. However, this task can
prove challenging at times, and success is not always guaranteed.

\fonte{Fonte de pesquisa: Katherine Pennacchio. MediaTalks. Latin American media seek to influence public debate and engage audience by translating its journalism into English. Disponível em: *https://mediatalks.uol.com.br/en/2021/10/15/latin-american-media-seek-to-influence-public-debate-and-engage-audience-in-u-s-by-translating-their-journalism-to-english/*. Acesso em: 01 mar. 2023.}
\end{quote}

A partir do texto, conclui-se que

\begin{escolha}
\item a língua inglesa exerce influência na mídia europeia.

\item o inglês é irrelevante no mundo contemporâneo.

\item a demanda para o ensino de inglês no Brasil não existe.

\item o inglês é considerado a língua internacional da área midiática.
\end{escolha}

\num{12} Leia o texto.

\begin{quote}
\textbf{World Leaders Gather at COP27 to Address Climate Change and Build on Paris Agreement}

The COP27 summit in Egypt will host representatives from various
nations, who will gather to negotiate extensive measures aimed at
combating climate change. The pressure is on for them to surpass the
significant pledges made in Paris seven years ago.

The Paris Agreement was a momentous agreement, in which almost all of
the world's nations endorsed a unified approach to reducing greenhouse
gas emissions, which are responsible for global warming. Despite former
US President Donald Trump's decision to pull out of the agreement, his
successor Joe Biden reinstated it on his first day in office in January
2021.

\fonte{Fonte de pesquisa: Helen Briggs e Esme Stallard. BBC. COP27: Why is the Paris climate agreement still important? Disponível em: *https://www.bbc.com/news/science-environment-35073297*. Acesso em: 01 mar. 2023.}
\end{quote}

O que se compreende do texto como um todo?

\begin{escolha}
\item Todos os líderes mundiais concordam a respeito do acordo de Paris.

\item O evento COP27 não é muito relevante.

\item Diversos líderes mundiais não compareceram ao encontro COP27.

\item Os últimos presidentes dos EUA discordam a respeito do acordo de Paris.
\end{escolha}

\pagebreak
\num{13} Leia o texto.

\begin{quote}
\textbf{West Ham United Stadium to be Wrapped in Solar Membrane in
Effort to Reduce Carbon Emissions}

The West Ham United stadium, which was originally built for the 2012
Olympics, is set to undergo a green transformation with the installation
of a solar membrane aimed at reducing carbon emissions.

The project is expected to cost around £4 million during the first two
years of implementation, but the investment is expected to be recouped
within five years. Planning documents suggest that work on the site
located in east London could commence later this year.

The stadium's owner, the London Legacy Development Corporation (LLDC),
established to oversee the area's development around the Queen Elizabeth
Olympic Park in Stratford after the 2012 Games, states that the building
could start producing energy by the end of 2024.

\fonte{Fonte de pesquisa: Noah Vickers. BBC. London
Stadium to be covered in solar panels to generate power. Disponível em:
\emph{https://www.bbc.com/news/uk-england-london-64758344}. Acesso em 01
mar. 2023.}
\end{quote}

De acordo com o texto, os painéis solares serão instalados no estádio
para

\begin{escolha}
\item auxiliar os treinos do time.

\item reduzir as emissões de carbono.

\item evitar apagões.

\item estimular o crescimento da cidade de Londres.
\end{escolha}

\num{14} Leia o texto.

\begin{quote}
\textbf{O arado}

Por volta do ano 5000 a.C., o ser humano
teria deixado de vagar atrás de terras
para cultivo e já começava a domesticar
animais. Foi nesse contexto que surgiu um
instrumento feito com galhos bifurcados e
uma pedra afiada na ponta, que servia
para arar a terra. Segundo especialistas,
a invenção do arado é um marco da Revolução Agrícola, que permitiu o ser humano fixar-se em aldeias, aumentar sua produtividade e dar início a atividades comerciais.

\fonte{Fonte de pesquisa: Superinteressante. Arado. Disponível em: \emph{https://super.abril.com.br/comportamento/arado/}. Acesso em: 04 abr. 2023.}
\end{quote}

No contexto apresentado, o arado cumpria a função de

\begin{escolha}
\item
  otimizar o trabalho humano.
\item
  domesticar os animais.
\item
  criar redes de água.
\item
  limpar terrenos.
\end{escolha}

\pagebreak
\num{15} Leia os textos.

\textbf{TEXTO 1}

\begin{quote}
\textbf{Processo de desindustrialização no Brasil se acentua}

A indústria brasileira dá sinais de que algo de errado acontece no
setor. Do início do ano [de 2021] até agora [ou seja, março de 2021], três gigantes multinacionais
anunciaram que vão abandonar o Brasil. A norte-americana Ford deixa o
mercado de fabricação de veículos nacional depois de mais de 100 anos. A
alemã Mercedes-Benz fecha a única fábrica no Brasil de carros de luxo. A
japonesa Sony fecha a fábrica em Manaus (AM) e abandona o mercado de
televisores, câmeras e aparelhos de áudio. Esse movimento mostra que o
País passa por um processo de desindustrialização, e não é de hoje, como
sugerem alguns números e apontam especialistas.

{[}...{]}

\fonte{Ferraz Jr. Jornal da USP. Processo de desindustrialização no Brasil se
acentua. Disponível em:
\emph{https://jornal.usp.br/atualidades/processo-de-desindustrializacao-no-brasil-se-acentua/}.
Acesso em: 22 fev. 2023.}
\end{quote}

\textbf{TEXTO 2}

\begin{quote}
\textbf{Exportações do agronegócio batem recorde em dezembro e no ano de 2021}

As exportações do agronegócio alcançaram valores recordes para o
mês de dezembro passado e também para o ano de 2021. Foram US\$ 9,88
bilhões, valor recorde para os meses de dezembro: 36,5\% superior aos
US\$ 7,24 bilhões de 2020. Em 2021, o total exportado com o agronegócio
resultou em US\$ 120,59 bilhões, alta de 19,7\%, em relação ao ano
anterior, conforme dados divulgados nesta quinta-feira [13/01/2022] pela
Secretaria de Comércio e Relações Internacionais (SCRI) do Ministério da
Agricultura, Pecuária e Abastecimento (Mapa).

{[}...{]}

\fonte{Ministério da Agricultura e Pecuária. Exportações do agronegócio batem recorde em dezembro e no ano de 2021. Disponível em: \emph{www.gov.br/agricultura/pt-br/assuntos/noticias/exportacoes-do-agronegocio-batem-recorde-em-dezembro-e-no-ano-de-2021}. Acesso em: 22 fev. 2023.}
\end{quote}

Associando-se os textos, é possível diagnosticar que, no contexto
globalizado atual, o Brasil tem se constituído como

\begin{escolha}
\item
  exportador de serviços e consumidor de \textit{commodities}.
\item
  investidor tecnológico e consumidor de industrializados.
\item
  exportador de produtos básicos e consumidor de tecnologia.
\item
  centralizador da indústria global e exportador de produtos básicos.
\end{escolha}

\num{16} 

\begin{quote}
\textbf{Sem fé, lei ou rei}

No século XVI, Pero de Magalhães (um cronista português) acreditava ter encontrado a chamada \textit{mácula original do silvícola brasileiro}. O cronista escreveu que os indígenas estavam fadados a não se destacar, o que se devia ao fato de que, na língua deles, não havia F, L ou R, de modo que não tinham nem fé, nem lei, nem rei.

\fonte{Terras indígenas no Brasil. Sem fé, lei ou rei. Disponível em:
\emph{https://terrasindigenas.org.br/noticia/30695}. Acesso em: 22 fev. 2023.}
\end{quote}

A fala de Pero Magalhães é um demonstrativo de que a sociedade tida como
correta, em sua visão, teria como base

\begin{escolha}
\item
  a ausência da figura do rei.
\item
  a adoção da língua portuguesa.
\item
  a ideia de proteção ambiental dos indígenas.
\item
  as características políticas e sociais da Europa.
\end{escolha}

\num{17} Analise o mapa.

\begin{quote}
\textbf{Ilú Obá De Min}

A associação Ilú Obá De Min é uma entidade sem fins lucrativos de São Paulo que tem como foco o trabalho com culturas de matriz africana, afro-brasileira e a valorização da mulher. Fundada em 2004 pelas percussionistas Beth Beli, Adriana Aragão e Girlei Miranda, a associação se tornou uma pessoa jurídica em 2006. Seu principal objetivo é preservar e divulgar a cultura negra no Brasil e fortalecer a participação e representatividade das mulheres negras. O projeto mais reconhecido da entidade é o Bloco Afro Ilú Oba De Min, cuja bateria é formada exclusivamente por mulheres. Desde 2005, o bloco realiza cortejos pelas ruas de São Paulo, honrando e celebrando a cultura afro-brasileira, além de destacar a participação e protagonismo feminino. Os cortejos do Bloco são uma grande intervenção cultural que promove a cultura negra, a cultura popular e a participação ativa das mulheres na sociedade por meio da arte. A iniciativa também traz para as áreas urbanas diversas manifestações da cultura negra, como o maracatu, batuque, coco, jongo, entre outras.

\fonte{Fonte de pesquisa: Ilú Obá De Min. Quem somos. Disponível em: \emph{https://iluobademin.com.br/institucional/quem-somos/}. Acesso em: 06 mar. 2023.}
\end{quote}

As ações executadas pela associação Ilú Obá De Min objetivam

\begin{escolha}
\item  naturalizar as expressões culturais negras como identidade
    sociocultural.
\item  dramatizar as expressões culturais negras como meras atitudes
    lúdicas.
\item  diferenciar as expressões culturais negras como práticas
    estrangeiras.
\item  promover atos de protesto pontuais contra o racismo persistente.
\end{escolha}

\pagebreak
\num{18}

\begin{quote}
\textbf{Brasil é líder em mortes de ambientalistas na última década}

Nos últimos 10 anos, o Brasil liderou o ranking mundial de assassinatos de defensores e defensoras do meio ambiente. De acordo com registros globais de 2012 a 2021, das 1.733 mortes, 342 ocorreram no país, representando quase 20\% do total. Entre as vítimas estão Maria José Rodrigues, de 78 anos, e seu filho José do Carmo Correa Junior, que foram esmagados por uma palmeira derrubada por um trator enquanto coletavam coco de babaçu em Penalva, Maranhão, em novembro de 2021. O tratorista estava desmatando uma área que já havia sido assegurada para uma comunidade tradicional, mas que estava sendo invadida a mando de um fazendeiro que, segundo denúncias dos moradores, pretendia plantar capim no terreno.

\fonte{Fonte de pesquisa: Nádia Pontes. G1. Brasil é líder em mortes de ambientalistas na última década. Disponível em: \emph{https://g1.globo.com/meio-ambiente/noticia/2022/09/29/brasil-e-lider-em-mortes-de-ambientalistas-na-ultima-decada.ghtml}. Acesso em: 04 maio 2023.}
\end{quote}

Qual das alternativas sintetiza as causas de violência contra
ambientalistas como as relatadas?

\begin{escolha}
\item  Invasão de terras públicas.

\item  Disputa por recursos naturais.

\item  Roubo de produção agropecuária.

\item  Desregulamentação do extrativismo.
\end{escolha}
