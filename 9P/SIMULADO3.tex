\chapter[Simulado 3]{Simulado}

Leia um trecho de um manifesto para responder às questões 1 e 2.

\begin{quote}
\textbf{Em defesa do meio ambiente brasileiro e da produção de alimentos
saudáveis: não ao substitutivo do código florestal!}

%Imagem ilustrativa que pode ser cortada: \url{https://www.freepik.com/free-photo/full-shot-forest-warden-looking-tree_29803333.htm\#query=forest\%20protection\&position=5\&from_view=search\&track=ais}

O Código Florestal (Lei nº. 4.771, de 15 de setembro de 1965) está
baseado em uma série de princípios que respondem às principais
preocupações no que tange ao uso sustentável do meio ambiente. Apesar
disso, entidades populares, agrárias, sindicais e ambientalistas,
admitem a concreta necessidade de aperfeiçoamento do Código criando
regulamentações que possibilitem atender às especificidades da
agricultura familiar e camponesa, reconhecidamente provedoras da maior
parte dos alimentos produzidos no país.

\fonte{EM DEFESA do meio ambiente brasileiro e da produção de alimentos
saudáveis: não ao substitutivo do código florestal! Disponível em:
www.mma.gov.br/estruturas/182/\_arquivos/manifestoagriculturafamiliar\_182.pdf.
Acesso em: 10 abr. 2023.}
\end{quote}

\num{1} Identifica-se que a finalidade do texto é

\begin{escolha}
\item advertir as entidades populares sobre o Código Florestal.

\item pedir por mudanças e melhorias para o meio ambiente.

\item exigir melhorias no Código Florestal vigente desde 1965.

\item mudar a forma de prover alimentos produzidos no país.
\end{escolha}

\num{2} A expressão que aparece entre parênteses -- Lei nº. 4.771, de 15
de setembro de 1965 -- tem o objetivo de

\begin{escolha}
\item indicar o número da Lei em que see stabeleceu o referido código,
além da data de sua publicação.

\item indicar o período em que se deu a redação da notícia, já que ela é
anterior à data de sua publicação.

\item mostrar que as melhorias do código florestal são recentes e que ele
já está atualizado o bastante.

\item indicar o código numérico complexo que delimita e identifica a lei
deste ano -- por isso chamada de código.
\end{escolha}


Leia um trecho do texto divulgado pela Reitoria da Universidade Estadual
Paulista (Unesp) para responder às questões 3 e 4.

\begin{quote}
\textbf{Da prova de títulos}

%Imagem ilustrativa que pode ser cortada: \url{https://www.freepik.com/free-photo/hand-with-pen-writing-paper_978444.htm\#query=concurso\%20p\%C3\%BAblico\&position=3\&from_view=search\&track=ais}

35. A prova de títulos será aplicada no mesmo dia e local das provas
objetiva e dissertativa, será aplicada logo após a prova dissertativa,
no período da tarde, devendo o candidato observar, total e atentamente o
disposto nos itens 1. ao 12., e suas alíneas, deste Capítulo, não
podendo ser alegada qualquer espécie de desconhecimento.


36. Após o fechamento dos portões do local da prova de títulos, não será
permitida a saída do candidato, do prédio, para providenciar/buscar
títulos, nem a entrega desses por terceiros no portão do prédio de
aplicação.

\fonte{REITORIA, Universidade Estadual Paulista ``Júlio De Mesquita Filho''.
Edital de abertura de inscrições (Concurso Público Nº 26/2020).
Disponível em:
https://documento.vunesp.com.br/documento/stream/MTU4NTQyMQ\%3d\%3d.
Acesso em: 9 abr. 2023.}
\end{quote}

\num{3} O texto em questão é um edital em razão da

\begin{escolha}
\item reflexão que o texto faz sobre o modo de se aplicar provas, propondo
uma ordem específica.

\item linguagem pessoal que o texto traz, fazendo menção ao ``candidato''
e a ``terceiros''.

\item estrutura típica de um texto normativo, como os itens e capítulos
mencionados no texto.

\item maneira informal com que o texto se dirige ao candidato, tentando
atingir diversas pessoas.
\end{escolha}

\num{4} Basicamente, o texto veta

\begin{escolha}
\item a alegação de não legibilidade e a presença de familiares no espaço
determinado.

\item a alegação de desconhecimento das cláusulas e a entrega de
documentos por outras pessoas, no prédio.

\item o atraso dos candidatos no dia da prova de títulos e que o mesmo
candidato faça prova de títulos e prova dissertativa.

\item a presença dos candidatos no lugar onde ocorrem as provas e a
comunicação entre candidatos.
\end{escolha}

Leia o poema para responder às questões de 5 a 7.

\begin{quote}
\textbf{No casarão}

%Imagem ilustrativa que pode ser cortada: \url{https://www.freepik.com/free-photo/spooky-scene-with-old-house_32879726.htm\#query=terror\%20mansion\&position=1\&from_view=search\&track=ais}

No casarão antigo e sombrio, Onde o tempo parece ter perdido o fio, Um
mistério aguarda noite e dia, Segredos ocultos em cada fresta vazia.

Eles, amigos corajosos e audazes, Decidiram explorar aqueles lugares
desvairados, Há tempos companheiros, curiosos e aventureiros, Ansiando
desvendar enigmas verdadeiros.

Na primeira noite, sob a luz prateada, Adentraram o casarão sem receio
ou parada. Passos cautelosos, corações acelerados, Cada passo, um
suspense desenhado.

Cortinas empoeiradas dançavam no ar, A velha escada rangia ao pisar,
Quadros sinistros encaravam em silêncio, Um eco misterioso percorria o
palácio imenso.

Entre corredores sombrios, encontraram um baú, Chaves enferrujadas,
mistério a perscrutar, Ao abri-lo, revelou-se um mapa enigmático, Que
levava a um tesouro, tesouro fantástico.

Desvendando pistas, em noites de escuridão, Seguiram trilhas, em busca
de eoção. Passaram por salões cheios de poeira, O medo e a emoção em sua
pele inteira.

Mas, ao chegarem ao aposento final, O tesouro esperado não estava no
local. No seu lugar, um bilhete escrito à mão, ``A verdadeira riqueza
está na jornada, não no chão''.

Embora desapontados, entenderam a lição, Que a aventura vivida foi a
verdadeira atração, Desvendaram segredos, fortaleceram a amizade,
naquela noite, ganharam mais que a vaidade.

E, assim, o casarão abandonado Permaneceu com seu mistério preservado.
Uma história contada para outro desvendar, E despertar em seu coração a
vontade de explorar.

\fonte{Diva Lopes}
\end{quote}

\num{5} O poema, organizado em estrofes de 4 versos, é um texto de
natureza

\begin{escolha}
\item
  informativa.
\item
  injuntiva.
\item
  narrativa.
\item
  dissertativa.
\end{escolha}


\num{6} Ao final, a lição que se constrói no texto é a de que nem tudo
se resume ao resultado daquilo que buscamos, mas está na busca em si e
nos aprendizados que chegam com ela. Isso está mais concentrado em
sentido no verso

\begin{escolha}
\item
  ``Onde o tempo parece ter perdido o fio''
\item
  ``Há tempos companheiros, curiosos e aventureiros''
\item
  ``A verdadeira riqueza está na jornada, não no chão''
\item
  ``E despertar em seu coração a vontade de explorar.''
\end{escolha}

\num{7} Pode-se dizer que o cenário, o ambiente, o espaço, representam
elementos enriquecedores dos textos. Nesse caso, o ambiente delineado
pela descrição é, majoritariamente,

\begin{escolha}
\item
  solar e energético.
\item
  dinâmico e acelerado.
\item
  leve e despreocupado.
\item
  sombrio e obscuro.
\end{escolha}


\pagebreak
\num{8} Leia o texto.

\begin{quote}
{[}...{]}

Constituída por vários rituais religiosos e expressões culturais, {[}...{]} é uma celebração profundamente 
enraizada no cotidiano dos moradores de Pirenópolis e determinante dos padrões de sociabilidade local. A 
esta estrutura básica, os agentes [da festa] vêm incorporando outros ritos e representações, como as 
encenações de mascarados e cavalhadas, responsáveis pela grande notoriedade da festa, que se realiza nesta 
cidade a cada ano, desde 1819, durante cerca de 60 dias, com clímax no Domingo de Pentecostes, cinquenta 
dias após a Páscoa.

\fonte{Instituto do Patrimônio Histórico e Artístico Nacional. Disponível em:
\emph{https://www.gov.br/iphan/pt-br/patrimonio-cultural/patrimonio-imaterial/reconhecimento-de-bens-culturais/livros-de-registro/celebracoes/festa-do-divino-espirito-santo-de-pirenopolis}. Acesso em: 15 maio 2023.}
\end{quote}

A descrição do patrimônio cultural dada pelo texto refere-se

\begin{escolha}
\item
  ao Círio de Nossa Senhora de Nazaré.
\item
  à Festa do Divino Espírito Santo.
\item
  à Festa do Bumba Meu Boi.
\item
  à Festa do Senhor Bom Jesus do Bonfim.
\end{escolha}

\num{9} Leia o texto.

\begin{quote}
\textbf{A luta contra o preconceito no balé}

A prática do balé traz inúmeros benefícios físicos e emocionais para as crianças, independentemente de 
outros fatores. Além de melhorar a postura e desenvolver os músculos e a agilidade, o balé também estimula 
a flexibilidade, a coordenação motora e promove disciplina e persistência. Estudos mostram que dançar em 
conjunto com colegas ajuda a construir a confiança, reduz preconceitos em relação à expressão corporal, tem 
efeitos positivos na interação social, no espírito de grupo, na empatia e diminui o medo de se apresentar.

No entanto, mesmo com tantos benefícios, ainda existem muitos meninos que não têm oportunidades de 
experimentar essa forma de arte. Isso se deve aos estereótipos e aos preconceitos que partem principalmente 
dos adultos. Mãe de um pequeno bailarino compartilhou suas experiências em uma entrevista, relatando as 
barreiras que enfrentou para permitir que seu filho seguisse sua paixão pelo balé clássico.

Apesar das críticas e dos redulineários maldosos, a mãe não se deixou abater e continuou apoiando seu filho. 
Ela sempre defendeu a prática da dança para meninos. Criou uma página no Instagram para compartilhar a rotina de exercícios de seu filho e encontrou apoio e reconhecimento nessa plataforma.

A pandemia trouxe desafios, mas ela adaptou um espaço em casa para que seu filho pudesse continuar 
treinando e fazendo aulas \textit{on-line}. Atualmente, o menino voltou a ter aulas presenciais em turmas 
reduzidas. A determinação e o esforço do filho em perseguir sua paixão são admiráveis, e a mãe está grata pelos professores, que ajudaram seu filho nesse processo de evolução e construção.

\fonte{Fonte de pesquisa: Juliana Malacarne. Crescer. Meninos também dançam! Mães de bailarinos falam sobre 
preconceito. Disponível em:
\emph{https://revistacrescer.globo.com/Educacao-Comportamento/noticia/2021/04/meninos-tambem-dancam-maes-de-bailarinos-falam-sobre-preconceito.html}.
Acesso: 15 mar. 2023.}
\end{quote}

O texto trada do estereótipo

\begin{escolha}
\item
  cultural.
\item
  de gênero.
\item
  de beleza.
\item
  socioeconômico.
\end{escolha}

\num{10} Leia o texto.

\begin{quote}
Segundo a classificação da UNESCO, o patrimônio cultural abrange monumentos, grupos de edifícios ou sítios 
que possuem um valor universal excepcional em termos históricos, estéticos, arqueológicos, científicos, 
etnológicos ou antropológicos. Isso inclui obras de arquitetura, escultura e pintura monumentais, além de 
artefatos arqueológicos. Por outro lado, o \textbf{patrimônio natural} engloba formações físicas, 
biológicas e geológicas notáveis, habitáts de espécies ameaçadas e áreas de valor científico, 
conservacionista ou estético excepcional e universal.

\fonte{Texto escrito para este material.}
\end{quote}

É um patrimônio natural

\begin{escolha}
  \item a cidade de Ouro Preto, em Minas Gerais.
  \item o Pantanal, entre o Mato Grosso e o Mato Grosso do Sul.
  \item a cidade de Paraty, no Rio de Janeiro.
  \item o monumento do Cristo Redentor, no Rio de Janeiro.
\end{escolha}

\num{11} Leia o texto.

\begin{quote}
\textbf{Coronavirus effect}

COVID is a respiratory tract infection, although it can sometimes affect
other parts of the body. The coronavirus primarily enters the body
through the nostrils and mouth, where it then replicates in the
respiratory tract. The only way to prevent contracting the infection is
by limiting the virus's entry through the nose and mouth. Properly
wearing masks that cover the nose and mouth can help, as well as
regularly sanitizing hands, particularly after touching a contaminated
surface, to reduce the likelihood of the virus entering the body.

\fonte{Fonte de pesquisa: Times of India. Coronavirus Effect: Here's Why You Should Wear Masks In Public Places Even When There Is No Mandate. Disponível em: *https://timesofindia.indiatimes.com/life-style/health-fitness/health-news/coronavirus-effect-heres-why-you-should-wear-masks-in-public-places-even-when-there-is-no-mandate/photostory/92539871.cms?picid=92540018*. Acesso em: 02 mar. 2023.}
\end{quote}

\pagebreak
De acordo com o texto, um dos argumentos para o uso de máscaras contra a
covid-19 refere-se ao fato de que

\begin{escolha}
\item nem todos os órgãos do corpo podem ser afetados.

\item a medida impede a contaminação de superfícies.

\item a doença afeta o trato respiratório primeiro.

\item a higienização de mãos não é importante.
\end{escolha}

\num{12} Leia o texto.

\begin{quote}
\textbf{Tips for Navigating Breaking News in the Age of Social Media}

Breaking news can be accessed by anyone with an internet connection,
albeit sometimes in a modified version. Social media posts tend to
circulate at a rate that surpasses the capacity of most moderators and
fact-checkers, and their content can be a mixed bag of factual, false,
out-of-context, and even propagandistic information. How can you
differentiate between trustworthy and untrustworthy sources, and what
criteria should you employ when deciding what to share and report to
tech companies? Here are some fundamental guidelines that everyone
should utilize when consuming breaking news online: analyze who is
disseminating the information. If it's your acquaintances or relatives,
be cautious and do not accept their posts as accurate, unless they have
direct experience or are acknowledged experts on the subject. If it's a
stranger or organization, keep in mind that having a verified check mark
or being popular does not equate to reliability.

\fonte{Fonte de pesquisa: Heather Kelly. The Washington Post. How to avoid falling for and spreading misinformation online. Disponível em: *https://www.washingtonpost.com/technology/2022/02/24/tips-avoid-misinformation-ukraine-2/*. Acesso em: 02 mar. 2023.}
\end{quote}

Segundo o texto, como podemos identificar um argumento confiável?

\begin{escolha}
\item Devemos confiar somente em membros de nossa família.

\item Devemos checar as fontes, dando prioridade a especialistas.

\item Devemos confiar somente nas informações encontradas na internet.

\item Devemos utilizar somente redes sociais confiáveis.
\end{escolha}

\num{13} Leia o texto.

\begin{quote}
\textbf{Discovery of a Potential New Hummingbird Species in Peru's
Cordillera Azul National Park}

When researchers spotted a hummingbird with shiny gold feathers on its
throat in Peru's Cordillera Azul National Park, they thought it might be
a newly discovered species. The park, located on a remote \textbf{spot}
of the Andes Mountains' eastern slopes, is an ideal place to spot
genetically distinct species.

\fonte{Fonte de pesquisa:  Ashley Strickland. CNN. Newly discovered hummingbird looks like it's wearing a golden collar. Disponível em: *https://edition.cnn.com/2023/03/02/world/gold-throated-hummingbird-hybrid-scn/index.html*. Acesso em: 02 mar. 2023.}
\end{quote}

\pagebreak
No contexto apresentado no texto, a palavra ``spot'' significa

\begin{escolha}
\item ``mácula''.

\item ``mancha''.

\item ``local''.

\item ``pinta''.
\end{escolha}

\num{14} Leia o texto.

\begin{quote}
\textbf{O caso de injustiça envolvendo a modelo Babiy Querino e a falha na identificação por foto}

Em 2017, a vida de Barbara Querino, então com 22 anos, foi drasticamente alterada por um reconhecimento 
fotográfico irregular. A modelo e dançarina, conhecida como Babiy, foi fotografada por policiais militares 
no dia em que seu irmão e primo foram presos, apesar de não ter qualquer envolvimento com o crime. A foto 
circulou em grupos de WhatsApp e páginas do Facebook, que a apresentaram falsamente como membro de uma 
quadrilha de assaltantes de carros atuando na zona sul de São Paulo. Em janeiro de 2018, Babiy foi presa 
sob acusação de ter participado de dois roubos em setembro de 2017, e permaneceu na prisão por 1 ano e 8 
meses, apesar de apresentar provas da sua inocência. Em 2020, a dançarina finalmente foi absolvida de todas 
as acusações.

\fonte{Caê Vasconcelos. Ponte. Por que tantos negros são alvo de prisão injusta com base em reconhecimentos. Disponível em: \emph{https://ponte.org/por-que-tantos-negros-sao-alvo-de-prisao-injusta-com-base-em-reconhecimentos/}. Acesso em: 10 mar. 2023.}
\end{quote}

Acontecimentos como o relatado são um claro desrespeito à "Declaração
universal dos direitos humanos" porque contrariam o princípio de que ninguém será

\begin{escolha}
\item  mantido em escravidão ou servidão.

\item  preso, detido ou exilado arbitrariamente.

\item  submetido à tortura nem a tratamento cruel.

\item  privado de sua nacionalidade arbitrariamente.
\end{escolha}


\num{15} Leia o texto.

\begin{quote}
\textbf{Sistemas agroflorestais}

Os Sistemas Agroflorestais (SAFs) permitem aos agricultores familiares conciliar a produção de alimentos com a gestão das riquezas naturais de cada bioma. Na Bahia, {[}...{]} indígenas da etnia Pataxó, do território Barra Velha, município de Porto Seguro, a 629 km de Salvador, receberam do grupo ambiental Natureza Bela apoio para a implantação do corredor ecológico Monte Pascoal-Pau-Brasil, com a recuperação de uma área de mais de 50 hectares de terra, por meio da produção agroecológica de alimentos no Sistema Agroflorestal (SAF). {[}...{]}

\fonte{CONAFER. Agricultores Pataxó aliam produção de alimentos com reflorestamento via SAFs. Disponível em: \emph{https://conafer.org.br/povos-conafer-agricultores-pataxo-aliam-producao-de-alimentos-com-reflorestamento-via-safs/}. Acesso em: 11 mar. 2023.}
\end{quote}

\pagebreak
Devido à prática de agricultura ecológica, a população Pataxó necessita,
primordialmente, de

\begin{escolha}
\item  empréstimos bancários.

\item  qualificação na área industrial.

\item  áreas completamente florestadas.

\item  território disponível para plantio.
\end{escolha}

\num{16} Leia o texto.

\begin{quote}
Nas décadas de 1820 e 1830, o avanço impessoal e poderoso da
máquina e do mercado começou a deixá-los [os trabalhadores pobres] de lado. Na melhor das
hipóteses, este fato fazia com que homens independentes se tornassem
dependentes, e que as pessoas se transformassem em ``mãos''. Na pior das
hipóteses, e a mais frequente, criava multidões de desclassificados,
empobrecidos e famintos tecelões manuais, tecelões mecânicos etc., cuja
miséria gelava o sangue do economista mais insensível.

\fonte{E. Hobsbawm. \textit{A era das revoluções}. São Paulo: Paz e Terra, 2011.}
\end{quote}

De acordo com o texto, o avanço da mecanização industrial na Europa, no
período destacado, foi um responsávelpor

\begin{escolha}
\item  minar as atividades laborais de trabalhadores especializados.

\item  garantir pleno emprego aos trabalhadores urbanos.

\item  incentivar a criatividade no ambiente laboral.

\item  oferecer boas condições de vida.
\end{escolha}

\num{17} Leia o texto.

\begin{quote}
As tensões entre judeus e não judeus atingiram um nível crítico, e em 1948, uma Grã-Bretanha exausta transferiu a questão para as Nações Unidas, que votaram a favor da divisão da região em dois países. Os judeus concordaram, enquanto os árabes manifestaram sua discordância.

\fonte{Texto escrito para este material.}
\end{quote}

A situação apresentada faz referência à disputa

\begin{escolha}
\item  política entre Iraque e Irã.

\item  bélica entre a Síria e o Líbano.

\item  territorial entre Israel e Palestina.

\item  diplomática entre Iêmen e Arábia Saudita.
\end{escolha}

\num{18} Leia o texto.

\begin{quote}
\textbf{Os curdos}

Os curdos, descendentes da Pérsia antiga, constituem hoje a maior nação apátrida do mundo, com uma população estimada entre 30 e 40 milhões de pessoas. Sua presença na região montanhosa do Curdistão, que se estende por parte dos territórios de Irã, Iraque, Síria, Armênia e Turquia, remonta a 4.300 a.C. Com cerca de 500 mil quilômetros quadrados, o Curdistão é uma região histórico-cultural habitada pelos curdos. A maior concentração de curdos encontra-se no sudeste turco, com cerca de 20 milhões de pessoas.

\fonte{Fonte de pesquisa: Folha de S.Paulo. Entenda quem são os curdos, povo no centro da disputa entre Turquia e EUA. Disponível em: \emph{https://www1.folha.uol.com.br/mundo/2019/10/entenda-quem-sao-os-curdos-a-maior-nacao-apatrida-do-mundo.shtml}. Acesso em: 05 maio 2023.}
\end{quote}

A característica apátrida do povo curdo denota a ausência de

\begin{escolha}
\item  uma cultura nacional.

\item  um território próprio.

\item  um idioma comum.

\item  um dialeto.
\end{escolha}