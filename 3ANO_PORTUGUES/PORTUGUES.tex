\section{1. O gênero fábula}\label{muxf3dulo-1}

\protect\hypertarget{_Hlk127342659}{}{}Neste módulo, por meio do
trabalho com o gênero textual fábula, espera-se que os alunos consigam
localizar informações explícitas, inferir informações implícitas,
inferir o sentido de palavras ou expressões e reconhecer o significado
de palavras derivadas com base em seus afixos.
\protect\hypertarget{_Hlk127342929}{}{}

\colorsec{Habilidades do SAEB}

\begin{itemize}
  \item Identificar a ideia central o texto.
  \item Localizar informação explícita.
  \item Inferir informações implícitas em textos.
  \item Inferir o sentido de palavras ou expressões em textos.
  \item Reconhecer em textos o significado de palavras derivadas a partir de
seus afixos.
\end{itemize}

\colorsec{Habilidades da BNCC}

\begin{itemize}
  \item EF35LP03, EF15LP03, EF35LP04, EF35LP05, EF03LP10.
\end{itemize}

\textbf{\textless{}início de boxe de conteúdo\textgreater{}}

\subsection{Conteúdo}\label{conteuxfado}

Histórias que têm animais como personagens existem há muito tempo. Você
já leu ou escutou alguma história na qual os animais se comportam como
os humanos?

As fábulas são narrativas com essas características, com personagens
como animais, plantas ou objetos com características humanas, e
geralmente trazem um ensinamento, uma moral ou um conselho.

Esopo, que viveu na Grécia Antiga, é autor de diversos textos desse
gênero.

As fábulas são consideradas um gênero literário e são uma das mais
antigas formas de se contar uma história. Elas podem ser escritas
em prosa (texto em parágrafos) ou em versos. Os títulos normalmente se
referem às personagens, e o tempo e o espaço relacionam-se ao ambiente
delas. A linguagem apresenta-se de modos simples, objetivo e direto. Sua
estrutura apresenta começo, desenvolvimento e fim.

\textbf{\textless{}fim do boxe de conteúdo\textgreater{}}

\subsection{Atividades}\label{atividades}

Leia, agora, a fábula ``O leão e o mosquito'', de Esopo.

\coment{Oriente os alunos a realizar, primeiramente, a leitura silenciosa do
texto, seguida de leitura em voz alta. Essa estratégia de leitura
possibilita que o aluno desenvolva fluência leitora, facilitando a
compreensão global do texto lido. As atividades introdutórias são de
compreensão do texto e têm a finalidade de desenvolver habilidades de
constatar, localizar informações, realizar simples inferências, deduzir
significados de termos do texto e inferir o tempo em que ocorre a
narrativa.}

\begin{quote}
\textbf{O leão e o mosquito}

Um leão ficou com raiva de um mosquito que não parava de zumbir ao redor
de sua cabeça, mas o mosquito não deu a mínima.

--- Você está achando que vou ficar com medo de você, só porque você
pensa que é rei? --- Disse ele, altivo, e em seguida voou para o leão e
deu uma picada ardida no seu focinho.

Indignado, o leão deu uma patada no mosquito, mas a única coisa que
conseguiu foi arranhar-se com as próprias garras. O mosquito continuou
picando o leão, que começou a urrar como um louco.

No fim, exausto, enfurecido e
coberto de feridas provocadas por seus próprios dentes e garras, o leão
se rendeu. O mosquito foi embora zumbindo, para contar a todo o mundo que
tinha vencido o leão, mas entrou direto numa teia de aranha. Ali, o
vencedor do rei dos animais encontrou seu triste fim, comido por uma
aranha minúscula.

\emph{Muitas vezes o menor de nossos inimigos é o mais terrível.}

\fonte{O leão e o mosquito. Disponível em:
\emph{http://www.dominiopublico.gov.br/download/texto/me000589.pdf}. Acesso em:
13 fev. 2023.}
\end{quote}

\subsubsection{1. }\label{section}

Quem são os personagens da fábula?
\coment{O leão, o mosquito e a aranha.}

\_\_\_\_\_\_\_\_\_\_\_\_\_\_\_\_\_\_\_\_\_\_\_\_\_\_\_\_\_\_\_\_\_\_\_\_\_\_\_\_\_\_\_\_\_\_\_\_\_\_\_\_\_\_\_\_\_\_\_\_\_\_\_\_\_\_\_\_\_\_\_\_\_\_\_\_\_\_\_\_\_\_\_\_\_\_\_\_\_\_\_\_\_\_\_\_\_\_\_\_\_\_\_\_\_\_\_\_\_\_\_\_\_\_\_\_\_\_\_\_\_\_\_\_\_\_\_\_\_\_\_\_\_\_\_\_\_\_\_\_

\subsubsection{2. }\label{section-1}

Por que o leão ficou com raiva?
\coment{O leão ficou com raiva porque o mosquito
picou seu focinho.}

\_\_\_\_\_\_\_\_\_\_\_\_\_\_\_\_\_\_\_\_\_\_\_\_\_\_\_\_\_\_\_\_\_\_\_\_\_\_\_\_\_\_\_\_\_\_\_\_\_\_\_\_\_\_\_\_\_\_\_\_\_\_\_\_\_\_\_\_\_\_\_\_\_\_\_\_\_\_\_\_\_\_\_\_\_\_\_\_\_\_\_\_\_\_\_\_\_\_\_\_\_\_\_\_\_\_\_\_\_\_\_\_\_\_\_\_\_\_\_\_\_\_\_\_\_\_\_\_\_\_\_\_\_\_\_\_\_\_\_\_

\subsubsection{3. }\label{section-2}

O que houve com o mosquito finalmente?
\coment{O mosquito foi comido por uma
aranha minúscula.}

\_\_\_\_\_\_\_\_\_\_\_\_\_\_\_\_\_\_\_\_\_\_\_\_\_\_\_\_\_\_\_\_\_\_\_\_\_\_\_\_\_\_\_\_\_\_\_\_\_\_\_\_\_\_\_\_\_\_\_\_\_\_\_\_\_\_\_\_\_\_\_\_\_\_\_\_\_\_\_\_\_\_\_\_\_\_\_\_\_\_\_\_\_\_\_\_\_\_\_\_\_\_\_\_\_\_\_\_\_\_\_\_\_\_\_\_\_\_\_\_\_\_\_\_\_\_\_\_\_\_\_\_\_\_\_\_\_\_\_\_

\_\_\_\_\_\_\_\_\_\_\_\_\_\_\_\_\_\_\_\_\_\_\_\_\_\_\_\_\_\_\_\_\_\_\_\_\_\_\_\_\_\_\_\_\_\_\_\_\_\_\_\_\_\_\_\_\_\_\_\_\_\_\_\_\_\_\_\_\_\_

\subsubsection{4. }\label{section-3}

Em sua opinião, onde ocorreu a história?
\coment{É muito provável que os alunos
respondam que a história pode ter ocorrido na floresta.}

\_\_\_\_\_\_\_\_\_\_\_\_\_\_\_\_\_\_\_\_\_\_\_\_\_\_\_\_\_\_\_\_\_\_\_\_\_\_\_\_\_\_\_\_\_\_\_\_\_\_\_\_\_\_\_\_\_\_\_\_\_\_\_\_\_\_\_\_\_\_\_\_\_\_\_\_\_\_\_\_\_\_\_\_\_\_\_\_\_\_\_\_\_\_\_\_\_\_\_\_\_\_\_\_\_\_\_\_\_\_\_\_\_\_\_\_\_\_\_\_\_\_\_\_\_\_\_\_\_\_\_\_\_\_\_\_\_\_\_\_

\_\_\_\_\_\_\_\_\_\_\_\_\_\_\_\_\_\_\_\_\_\_\_\_\_\_\_\_\_\_\_\_\_\_\_\_\_\_\_\_\_\_\_\_\_\_\_\_\_\_\_\_\_\_\_\_\_\_\_\_\_\_\_\_\_\_\_\_\_\_

\subsubsection{5. }\label{section-4}

Qual é a moral da história? Transcreva no espaço a seguir.
\coment{Muitas vezes o menor de nossos inimigos é o
mais terrível.}

\_\_\_\_\_\_\_\_\_\_\_\_\_\_\_\_\_\_\_\_\_\_\_\_\_\_\_\_\_\_\_\_\_\_\_\_\_\_\_\_\_\_\_\_\_\_\_\_\_\_\_\_\_\_\_\_\_\_\_\_\_\_\_\_\_\_\_\_\_\_\_\_\_\_\_\_\_\_\_\_\_\_\_\_\_\_\_\_\_\_\_\_\_\_\_\_\_\_\_\_\_\_\_\_\_\_\_\_\_\_\_\_\_\_\_\_\_\_\_\_\_\_\_\_\_\_\_\_\_\_\_\_\_\_\_\_\_\_\_\_

\_\_\_\_\_\_\_\_\_\_\_\_\_\_\_\_\_\_\_\_\_\_\_\_\_\_\_\_\_\_\_\_\_\_\_\_\_\_\_\_\_\_\_\_\_\_\_\_\_\_\_\_\_\_\_\_\_\_\_\_\_\_\_\_\_\_\_\_\_\_

\subsubsection{6. }\label{section-5}

Reescreva as frases a seguir, trocando as palavras em destaque por
sinônimos, ou seja, palavras diferentes que têm sentido
semelhante.

\coment{Para o desenvolvimento desta atividade, coloque à disposição dos alunos
diferentes dicionários para consulta.}

a) O mosquito continuou picando o leão, que começou a \textbf{urrar}
como um louco. \protect\hypertarget{_Hlk127196857}{}{}Sugestão de
resposta: O mosquito continuou picando o leão, que começou a
\textbf{rugir} como um louco.

\_\_\_\_\_\_\_\_\_\_\_\_\_\_\_\_\_\_\_\_\_\_\_\_\_\_\_\_\_\_\_\_\_\_\_\_\_\_\_\_\_\_\_\_\_\_\_\_\_\_\_\_\_\_\_\_\_\_\_\_\_\_\_\_\_\_\_\_\_\_\_\_\_\_\_\_\_\_\_\_\_\_\_\_\_\_\_\_\_\_\_\_\_\_\_\_\_\_\_\_\_\_\_\_\_\_\_\_\_\_\_\_\_\_\_\_\_\_\_\_\_\_\_\_\_\_\_\_\_\_\_\_\_\_\_\_\_\_\_\_

\_\_\_\_\_\_\_\_\_\_\_\_\_\_\_\_\_\_\_\_\_\_\_\_\_\_\_\_\_\_\_\_\_\_\_\_\_\_\_\_\_\_\_\_\_\_\_\_\_\_\_\_\_\_\_\_\_\_\_\_\_\_\_\_\_\_\_\_\_\_

\textbf{b)} \protect\hypertarget{_Hlk127196878}{}{}No fim, exausto,
\textbf{enfurecido} e coberto de feridas provocadas por seus próprios
dentes e garras, o leão se rendeu. Sugestão de resposta: No fim,
exausto, \textbf{furioso} e coberto de feridas provocadas por seus
próprios dentes e garras, o leão se rendeu.

\_\_\_\_\_\_\_\_\_\_\_\_\_\_\_\_\_\_\_\_\_\_\_\_\_\_\_\_\_\_\_\_\_\_\_\_\_\_\_\_\_\_\_\_\_\_\_\_\_\_\_\_\_\_\_\_\_\_\_\_\_\_\_\_\_\_\_\_\_\_\_\_\_\_\_\_\_\_\_\_\_\_\_\_\_\_\_\_\_\_\_\_\_\_\_\_\_\_\_\_\_\_\_\_\_\_\_\_\_\_\_\_\_\_\_\_\_\_\_\_\_\_\_\_\_\_\_\_\_\_\_\_\_\_\_\_\_\_\_\_

\textbf{\_\_\_\_\_\_\_\_\_\_\_\_\_\_\_\_\_\_\_\_\_\_\_\_\_\_\_\_\_\_\_\_\_\_\_\_\_\_\_\_\_\_\_\_\_\_\_\_\_\_\_\_\_\_\_\_\_\_\_\_\_\_\_\_\_\_\_\_\_\_}

\textbf{c)} Ali, o vencedor do rei dos animais encontrou seu triste fim,
comido por uma aranha \textbf{minúscula}. Sugestão de resposta: Ali, o
vencedor do rei dos animais encontrou seu triste fim, comido por uma
aranha \textbf{pequenina}.

\_\_\_\_\_\_\_\_\_\_\_\_\_\_\_\_\_\_\_\_\_\_\_\_\_\_\_\_\_\_\_\_\_\_\_\_\_\_\_\_\_\_\_\_\_\_\_\_\_\_\_\_\_\_\_\_\_\_\_\_\_\_\_\_\_\_\_\_\_\_\_\_\_\_\_\_\_\_\_\_\_\_\_\_\_\_\_\_\_\_\_\_\_\_\_\_\_\_\_\_\_\_\_\_\_\_\_\_\_\_\_\_\_\_\_\_\_\_\_\_\_\_\_\_\_\_\_\_\_\_\_\_\_\_\_\_\_\_\_\_

\_\_\_\_\_\_\_\_\_\_\_\_\_\_\_\_\_\_\_\_\_\_\_\_\_\_\_\_\_\_\_\_\_\_\_\_\_\_\_\_\_\_\_\_\_\_\_\_\_\_\_\_\_\_\_\_\_\_\_\_\_\_\_\_\_\_\_\_\_\_

\subsubsection{7. }\label{section-6}

Marque com um X a alternativa que explica a atitude do mosquito.

( ) Queria punir o leão porque ele foi agressivo.

( ) Queria que o leão passasse vergonha diante dos outros animais.

( x ) Era arrogante e queria provar que era mais corajoso que o leão.

\subsubsection{8. }\label{section-7}

Muitas palavras são formadas a partir de outras. As que são formadas são
palavras \textbf{derivadas} de outras palavras, que são denominadas
\textbf{primitivas}. Observe o quadro a seguir e, em seguida, faça as
atividades propostas.

\begin{longtable}[]{@{}l@{}}
\toprule
\begin{minipage}[t]{0.97\columnwidth}\raggedright\strut
\textbf{Palavra derivada Palavra primitiva}

Livraria Livro

Floreira Flor

Apavorar Pavor\strut
\end{minipage}\tabularnewline
\bottomrule
\end{longtable}

a) Separe as palavras a seguir em sílabas e escreva um substantivo ou um
adjetivo derivado para cada uma delas.

\begin{itemize}
\item
  Susto: \_\_sus-to\_\_\_\_\_\_\_\_\_\_\_\_\_\_\_
  \_assustar\_\_\_\_\_\_\_\_\_\_\_\_\_\_\_\_\_
\item
  Barba: \_\_\_\_\_bar-ba\_\_\_\_\_\_\_\_\_\_\_
  \_barbeiro\_\_\_\_\_\_\_\_\_\_\_\_\_\_
\item
  Saco: \_\_\_\_\_sa-co\_\_\_\_\_\_\_\_\_\_\_
  sacola\_\_\_\_\_\_\_\_\_\_\_\_\_\_\_\_\_\_
\item
  Pedra: \_\_\_pe-dra\_\_\_\_\_\_\_\_\_\_\_\_
  pedreiro\_\_\_\_\_\_\_\_\_\_\_\_\_\_\_\_
\item
  Manteiga: \_man-tei-ga\_\_\_\_\_\_\_\_\_\_ amanteigado
  \_\_\_\_\_\_\_\_\_\_\_\_\_\_
\end{itemize}

\subsubsection{9. }\label{section-8}

Escreva o substantivo primitivo das palavras a seguir.

a) Boleira: \_\_\_\_\_\_bolo\_\_\_\_\_\_\_\_\_\_

b) Coveiro: \_\_\_\_\_\_cova\_\_\_\_\_\_\_\_\_\_

c) Beleza: \_\_\_belo\_\_\_\_\_\_\_\_\_\_\_\_\_

d) Cabeluda: \_\_\_\_cabelo\_\_\_\_\_\_\_\_\_

e) Mangueira: \_\_\_\_\_\_manga\_\_\_\_\_\_\_\_\_\_\_\_\_

\subsubsection{10. }\label{section-9}

Ligue cada palavra derivada à respectiva palavra primitiva.

Empedrado Maçã

Padeiro Pimenta

Macieira Pedra

Apimentada Pão

\subsubsection{11. }\label{section-10}

Você lembra o que são sinônimos e antônimos? Leia o quadro para
relembrar.

\coment{Estas atividades facilitam o desenvolvimento da habilidade de inferir
significados de palavras desconhecidas sem recorrer ao dicionário.}

\begin{longtable}[]{@{}l@{}}
\toprule
\begin{minipage}[t]{0.97\columnwidth}\raggedright\strut
\begin{quote}
\textbf{Sinônimo}: palavra que apresenta significado semelhante ao de
outra.

\textbf{Antônimo}: palavra que apresenta significado contrário ao de
outra.
\end{quote}\strut
\end{minipage}\tabularnewline
\bottomrule
\end{longtable}

a) Leia a frase a seguir e marque as palavras que poderiam substituir as
palavras em destaque por terem significados semelhantes.

\begin{longtable}[]{@{}l@{}}
\toprule
O leão estava \textbf{triste} e \textbf{fraco}.\tabularnewline
\bottomrule
\end{longtable}

( x ) infeliz - debilitado ( ) feliz - gordo ( ) saltitante - forte

b) Escreva palavras antônimas de:

\begin{itemize}
\item
  moderno: antigo.
\item
  quente: frio.
\item
  dia: noite.
\item
  dentro: fora.
\item
  claro: escuro.
\item
  cair: levantar.
\end{itemize}

\subsubsection{12. }\label{section-11}

No caça-palavra a seguir, encontre palavras com sentido contrário aos das
palavras do quadro.

\begin{longtable}[]{@{}l@{}}
\toprule
\textbf{sair abrir descer acordar branco bonito}\tabularnewline
\bottomrule
\end{longtable}

E T E E W D T T B H L U

A F N I A N E H I S D D

O E E I D I O P E R E C

H I D C S T E N H E A E

H O N U H D N O S E A J

D V B M Y A T P E Y M N

Y I U F C R R W D R O F

R H D C T E A O O E O I

O T D A T S R O H N T A

Y C E O L F D O R M I R

D V M U O T S E C S S N

E T H H A Y L R E L F S

\subsubsection{13. }\label{section-12}

É possível formar antônimos usando \textbf{prefixos}, ou seja, pequenas inserções
que são feitas antes das palavras primitivas. Observe os exemplos a seguir.

\begin{longtable}[]{@{}l@{}}
\toprule
Constante -- \textbf{In}constante Continuar -
\textbf{Des}continuar\tabularnewline
\bottomrule
\end{longtable}

Acrescente \textbf{in-}, \textbf{im-} ou \textbf{des-} às palavras a seguir e crie antônimos.

a) Atar: desatar

b) Próprio: impróprio

c) Afogar: desafogar

d) Satisfeito: insatisfeito

e) Amar: desamar

f) Iludir: desiludir

g) Parcial: imparcial

h) Preparada: despreparada

\subsection{Treino}\label{treino}

\subsubsection{1 }\label{section-13}

Leia a fábula.

\begin{quote}
\textbf{O ratinho, o gato e o galo}

Certa manhã, um ratinho saiu do buraco pela primeira vez. Queria
conhecer o mundo e travar relações com tanta coisa bonita de que falavam
seus amigos. Admirou a luz do sol, o verdor das árvores, a correnteza
dos rios, a habitação dos homens. E acabou entrando no quintal duma casa
da roça.

--- Sim, senhor! É interessante isto!

Examinou tudo minuciosamente, farejou a tulha de milho e a estrebaria.
Em seguida, notou no terreiro um certo animal de belo pelo, que dormia
sossegado ao sol. Em seguida, notou no terreiro um certo animal de belo
pelo, que dormia sossegado ao sol. Aproximou-se dele e farejou-o, sem
receio nenhum. Nisto, aparece um galo, que bate as asas e canta.
{[}...{]}.

\fonte{O ratinho, o gato e o galo. Disponível em:
\emph{http://www.dominiopublico.gov.br/download/texto/me000589.pdf}. Acesso em:
14 fev. 2023.}
\end{quote}

Onde a segunda parte da história se passa?

(A) No buraco dos ratinhos.

(B) Em uma árvore do campo.

(C) Em um rio com correnteza.

(D) Em um espaço de roça.

\protect\hypertarget{_Hlk127278651}{}{}SAEB: Inferir informações implícitas em textos.
BNCC: EF35LP04 -- Inferir informações implícitas nos textos lidos.

(A) Incorreta. O ratinho saiu do buraco para explorar o mundo.

(B) Incorreta. Esse não é o local em que se passa a história.

(C) Incorreta. O texto menciona que o ratinho admirou a correnteza dos
rios, e não que a história se passa em um rio com correnteza.

(D) Correta. O ratinho saiu do buraco para explorar e acabou no quintal
de uma casa de roça.

\subsubsection{2 }\label{section-14}

Leia a parlenda.

\begin{quote}
\textbf{Hoje é domingo}

Hoje é domingo
Pede cachimbo
O cachimbo é de barro
Bate no jarro
O jarro é de ouro
Bate no touro
O touro é \textbf{valente}
Bate na gente
A gente é fraco
Cai no buraco
O buraco é fundo
Acabou-se o mundo

\fonte{Domínio público.}
\end{quote}

Que palavra a seguir apresenta sentido semelhante ao da palavra
``valente'', do texto?

a) Destemido.

b) Covarde.

c) Inteligente.

d) Doido.

Saeb: Inferir o sentido de palavras ou expressões em textos.
BNCC: EF35LP05 -- Inferir o sentido de palavras ou expressões
desconhecidas em textos, com base no contexto da frase ou do texto.

(A) Correta. A palavra ``destemido'' pode ser usada como sinônimo de
``valente'', pois é um adjetivo próprio de pessoas ou animais
guerreiros, audazes, ousados, corajosos.

(B) Incorreta. A palavra ``covarde'' é antônima da palavra ``valente''.

(C) Incorreta. A palavra ``inteligente'' é sinônima da palavra
``sabido''.

(D) Incorreta. A palavra ``doido'' teria como sinônimo ``maluco'' ou
``biruta'', por exemplo, mas não é adequado ao contexto.

\subsubsection{3. }\label{section-15}

Leia um trecho de quadrinha.

\begin{quote}
\textbf{Pombinha branca}

Pombinha branca,
O que está fazendo?
Lavando a roupa
Do casamento.
A roupa é suja
É cor de rosa
Pombinha branca
É \textbf{preguiçosa}.

\fonte{Folclore popular.}
\end{quote}

A palavra ``preguiçosa'' é classificada como

(A) primitiva, pois é possível formar outras palavras a partir dela.

(B) derivada, pois foi formada a partir de uma palavra primitiva.

(C) derivada, pois não apresenta relação com outras palavras.

(D) primitiva, pois é uma palavra que não deriva de outras.

Saeb: Reconhecer em textos o significado de palavras derivadas a partir de seus afixos.
BNCC: EF03LP10 -- Reconhecer prefixos e sufixos produtivos na formação de
palavras derivadas de substantivos, de adjetivos e de verbos,
utilizando-os para compreender palavras e para formar novas palavras.

(A) Incorreta. A palavra ``preguiçosa'' não é primitiva.

(B) Correta. A palavra ``preguiçosa'' é derivada, uma vez que foi
formada a partir da palavra primitiva ``preguiça''.

(C) Incorreta. A palavra apresenta relação direta com a palavra
``preguiça'' (primitiva).

(D) Incorreta. A palavra é derivada diretamente de ``preguiça''.

\section{2. O texto dramático}\label{muxf3dulo-2}

\coment{Neste módulo, será abordado o texto dramático, seu contexto de produção
e circulação, além de sua forma composicional. Espera-se que os alunos
leiam e compreendam texto do campo artístico-literário; reconheçam o
tema principal do texto; percebam o enredo expresso em textos
dramáticos; localizem informações explícitas no texto dramático; infiram
o sentido de palavras no texto, com base no contexto de trecho do texto;
assim como analisem os efeitos de sentido de verbos de enunciação.}

\colorsec{Habilidades do SAEB}

\begin{itemize}
  \item Reconhecer diferentes gêneros textuais.
  \item Identificar elementos constitutivos de textos narrativos.
  \item Identificar as marcas de organização de textos dramáticos.
  \item Analisar os efeitos de sentido de verbos de enunciação.
\end{itemize}

\colorsec{Habilidades da BNCC}

\begin{itemize}
  \item EF35LP03, EF15LP03, EF35LP04, EF35LP05, EF03LP10.
\end{itemize}

\subsection{Conteúdo}\label{conteuxfado-1}

\textbf{\textless{}início do boxe de conteúdo\textgreater{}}

https://www.pexels.com/pt-br/foto/agindo-atuacao-substituto-irmao-5801566/

\includegraphics[width=2.43750in,height=3.65625in]{media/image1.jpeg}

Uma história, para ser encenada no teatro, deve ser transformada em
\textbf{roteiro}, isto é, em um texto com uma estrutura adequada para
ser utilizada pelos atores na hora da encenação, sob a orientação de um
diretor.

No roteiro de texto teatral, também chamado de \textbf{texto dramático},
estão descritos os diálogos dos personagens e as rubricas, que consistem
nas orientações acerca da montagem da cena, do figurino e de todas as
situações que necessariamente devem ocorrer ao longo da peça.

Em textos dramáticos, o enredo é transmitido por meio das ações e dos
diálogos dos personagens, que, nas encenações, são representados por
atores. A representação dos personagens e de suas ações em peças
teatrais e filmes, inseridos em diferentes cenários, espaços e tempo,
torna possível o imaginado ficar mais próximo de quem o representa e do
público que prestigia a atuação.

Normalmente, um roteiro de texto dramático apresenta os seguintes
elementos: título, lista de personagens, organização do cenário,
rubricas.

\textbf{\textless{}fim do boxe de conteúdo\textgreater{}}

\subsection{Atividades}\label{atividades-1}

Leia o trecho da peça teatral para resolver as atividades de 1 a 6.

\coment{Realize uma leitura compartilhada do texto. Chame a atenção dos alunos
para as rubricas no texto e para sua função de orientar a dramatização
(como indicações das formas de falar, caminhar, gesticular; indicação de
características como altura da voz, ritmo). Solicite aos alunos que
observem quem são os personagens e qual é o papel de cada um.}

Inserir imagem ao lado do texto:
https://pixabay.com/pt/vectors/violino-violinista-banda-bandsman-154220/

\includegraphics[width=2.61389in,height=2.83858in]{media/image2.png}

\begin{quote}
\textbf{Zé Betovi e Nhô Mozarte}

\textit{Entra Nhô Mozarte com seu livro embaixo do braço olhando em
sua volta e fala em voz alta:}

\emph{Nhô Mozarte}: Aqui está bem mais tranquilo. Pelo menos não tem
nenhuma obra por perto. Com aquele barulho danado das máquinas eu não
estava conseguindo me concentrar.

(Em seguida, senta-se no banco da praça e começa a ler.)

\emph{Zé Betovi}: Tchau, mãe! Tô indo lá na praça tocar um pouco.

(Zé Betovi caminha em
direção à praça com seu violino na mão e resolve se sentar perto de uma
árvore, pois o calor estava muito grande naquele dia. Nem notou a
presença de Nhô Mozarte sentado no banco, entretido lendo seu livro, e
começa a tocar. Quando Nhô Mozarte escuta a música, para de ler e
procura ver de onde vem aquele som. Avista Zé Betovi sentado tocando.
Então se levanta devagar, procurando não fazer barulho, e vai em direção
a ele. Quando Zé Betovi acaba de tocar, aplaude com entusiasmo.)

\emph{Nhô Mozarte}: Bravo, meu jovem! Que maravilha! Estou admirado de
ver um garoto de sua idade tocando violino. E um instrumento que não é
fácil. Os jovens assim como você, principalmente nos dias de hoje,
preferem as guitarras, baterias, violão.

\emph{Zé Betovi}: Obrigado. É mesmo... Eu também gosto dos outros
instrumentos. Tenho violão e teclado e toco de vez em quando, mas o meu
preferido mesmo é esse aqui (mostra o violino).

\emph{Nhô Mozarte}: Um artista completo! (Expressa admiração.) Então, se
você toca violino, é porque aprecia a música clássica.

\emph{Zé Betovi}: Gosto. E nem tinha como eu não gostar. Lá em casa
tanto meu pai como minha mãe adoram. Na verdade, eu cresci ouvindo. Meu
avô, o pai de minha mãe, tocava muito bem de ouvido e nem sabia ler
partitura.

\emph{Nhô Mozarte}: E você? Toca de ouvido como seu avô?

\emph{Zé Betovi}: Das duas formas. Minha mãe me colocou em aulas de
música desde que eu era bem pequeno. Ela achou que ia ser bom pra mim.
E, quando eu fiz seis anos, escolhi o violino.

\emph{Nhô Mozarte}: Sua mãe fez muito bem. A música é importante na vida
de todo mundo. Ela nos ajuda em tantas coisas... Mas me diga, meu rapaz:
você sabe quem é o autor da música que estava tocando há pouco?

\emph{Zé Betovi}: Villa Lobos.

{[}...{]}

\fonte{Marluzi Moreira de Carvalho. Teatro na escola. Zé Betovi e Nhô Mozarte.
Disponível em:
\emph{www.teatronaescola.com/index.php/banco-de-pecas/category/infantil-ou-infanto-juvenil-2}.
Acesso em: 15 fev. 2023. (Adaptado.)}
\end{quote}

\subsubsection{1. }\label{section-16}

Quem são os personagens desse texto? Zé Betovi e Nhô Mozarte.

\_\_\_\_\_\_\_\_\_\_\_\_\_\_\_\_\_\_\_\_\_\_\_\_\_\_\_\_\_\_\_\_\_\_\_\_\_\_\_\_\_\_\_\_\_\_\_\_\_\_\_\_\_\_\_\_\_\_\_\_\_\_\_\_\_\_\_\_\_\_\_\_\_\_\_\_\_\_\_\_\_\_\_\_\_\_\_\_\_\_\_\_\_\_\_\_\_\_\_\_\_\_\_\_\_\_\_\_\_\_\_\_\_\_\_\_\_\_\_\_\_\_\_\_\_\_\_\_\_\_\_\_\_\_\_\_\_\_\_\_

\subsubsection{2. }\label{section-17}

As rubricas no texto servem para orientar a dramatização. Sublinhe cada
uma delas.

\subsubsection{3. }\label{section-18}

Qual é o instrumento preferido de Zé Betovi? O violino.

\protect\hypertarget{_Hlk127362551}{}{}\_\_\_\_\_\_\_\_\_\_\_\_\_\_\_\_\_\_\_\_\_\_\_\_\_\_\_\_\_\_\_\_\_\_\_\_\_\_\_\_\_\_\_\_\_\_\_\_\_\_\_\_\_\_\_\_\_\_\_\_\_\_\_\_\_\_\_\_\_\_\_\_\_\_\_\_\_\_\_\_\_\_\_\_\_\_\_\_\_\_\_\_\_\_\_\_\_\_\_\_\_\_\_\_\_\_\_\_\_\_\_\_\_\_\_\_\_\_\_\_\_\_\_\_\_\_\_\_\_\_\_\_\_\_\_\_\_\_\_\_

\subsubsection{4. }\label{section-19}

Os nomes dos personagens foram inspirados em dois músicos famosos. Você
consegue identificar quem são eles? Mozart e Beethoven.

\_\_\_\_\_\_\_\_\_\_\_\_\_\_\_\_\_\_\_\_\_\_\_\_\_\_\_\_\_\_\_\_\_\_\_\_\_\_\_\_\_\_\_\_\_\_\_\_\_\_\_\_\_\_\_\_\_\_\_\_\_\_\_\_\_\_\_\_\_\_

\subsubsection{5. }\label{section-20}

O texto apresentado é teatral. Com qual objetivo esses textos são
escritos? O objetivo de se escrever textos desse tipo é que sejam
encenados.

\_\_\_\_\_\_\_\_\_\_\_\_\_\_\_\_\_\_\_\_\_\_\_\_\_\_\_\_\_\_\_\_\_\_\_\_\_\_\_\_\_\_\_\_\_\_\_\_\_\_\_\_\_\_\_\_\_\_\_\_\_\_\_\_\_\_\_\_\_\_\_\_\_\_\_\_\_\_\_\_\_\_\_\_\_\_\_\_\_\_\_\_\_\_\_\_\_\_\_\_\_\_\_\_\_\_\_\_\_\_\_\_\_\_\_\_\_\_\_\_\_\_\_\_\_\_\_\_\_\_\_\_\_\_\_\_\_\_\_\_

\subsubsection{6. }\label{section-21}

Releia o trecho a seguir.

\begin{quote}
\emph{Nhô Mozarte}: Bravo, meu jovem! Que maravilha! Estou admirado de
ver um garoto de sua idade tocando violino. {[}...{]} Os jovens assim
como você, principalmente nos dias de hoje, preferem as
\textbf{guitarras}, \textbf{baterias}, \textbf{violão}.
\end{quote}

No espaço a seguir, faça um desenho de cada instrumento musical
destacado e escreva o nome de cada um deles abaixo do respectivo desenho.

\textless{}Criar espaço para desenho dos alunos.\textgreater{}

No conto ``João e Maria'', narra-se a história de dois irmãos que são
abandonados pelo pai e pela madrasta na floresta. Leia um trecho desse
conto e resolva as atividade de 7 a 13.

Inserir imagem de floresta:
https://unsplash.com/pt-br/fotografias/6YHlHIVROzg

\includegraphics[width=4.08333in,height=2.76048in]{media/image3.jpeg}

\begin{quote}
\textbf{João e Maria}

Às margens de uma extensa mata, existia, há muito tempo, uma cabana
pobre, feita de troncos de árvore, na qual morava um lenhador com sua segunda esposa e
seus dois filhinhos, nascidos do primeiro casamento. O garoto chamava-se João e a
menina, Maria.

A vida sempre fora difícil na casa do lenhador, mas naquela época as
coisas haviam piorado ainda mais: não havia pão para todos.

--- Minha mulher, o que será de nós? Acabaremos todos por morrer
de necessidade. E as crianças serão as primeiras.

--- Há uma solução --- disse a madrasta, que era muito
malvada. --- Amanhã, daremos a João e Maria um pedaço de pão, depois os
levaremos à mata e lá os abandonaremos.

{[}...{]}

Irmãos Grimm. João e Maria. Disponível em:
\emph{www.dominiopublico.gov.br/download/texto/me000589.pdf}. Acesso em: 15
fev. 2023.

\subsubsection{7. }\label{section-22} Faça o que se pede a seguir.

a) Indique quem são os personagens dessa história.

João, Maria, o pai e a madrasta.

\_\_\_\_\_\_\_\_\_\_\_\_\_\_\_\_\_\_\_\_\_\_\_\_\_\_\_\_\_\_\_\_\_\_\_\_\_\_\_\_\_\_\_\_\_\_\_\_\_\_\_\_\_\_\_\_\_\_\_\_\_\_\_\_\_\_\_\_\_\_\_\_\_\_\_\_\_\_\_\_\_\_\_\_\_\_\_\_\_\_\_\_\_\_\_\_\_\_\_\_\_\_\_\_\_\_\_\_\_\_\_\_\_\_\_\_\_\_\_\_\_\_\_\_\_\_\_\_\_\_\_\_\_\_\_\_\_\_\_\_

b) Explique por que a madrasta sugeriu abandonar João e Maria.

A madrasta deu essa sugestão, porque não havia comida para todos.

\_\_\_\_\_\_\_\_\_\_\_\_\_\_\_\_\_\_\_\_\_\_\_\_\_\_\_\_\_\_\_\_\_\_\_\_\_\_\_\_\_\_\_\_\_\_\_\_\_\_\_\_\_\_\_\_\_\_\_\_\_\_\_\_\_\_\_\_\_\_\_\_\_\_\_\_\_\_\_\_\_\_\_\_\_\_\_\_\_\_\_\_\_\_\_\_\_\_\_\_\_\_\_\_\_\_\_\_\_\_\_\_\_\_\_\_\_\_\_\_\_\_\_\_\_\_\_\_\_\_\_\_\_\_\_\_\_\_\_\_

c) Indique onde os pais planejavam abandonar as crianças.

Eles planejavam abandonar as crianças na mata.

\_\_\_\_\_\_\_\_\_\_\_\_\_\_\_\_\_\_\_\_\_\_\_\_\_\_\_\_\_\_\_\_\_\_\_\_\_\_\_\_\_\_\_\_\_\_\_\_\_\_\_\_\_\_\_\_\_\_\_\_\_\_\_\_\_\_\_\_\_\_\_\_\_\_\_\_\_\_\_\_\_\_\_\_\_\_\_\_\_\_\_\_\_\_\_\_\_\_\_\_\_\_\_\_\_\_\_\_\_\_\_\_\_\_\_\_\_\_\_\_\_\_\_\_\_\_\_\_\_\_\_\_\_\_\_\_\_\_\_\_

d) Marque a alternativa correta em relação ao que mostra o trecho
selecionado do conto ``João e Maria''.

( ) A situação de tristeza que tomou conta do pai e da madrasta depois
de eles abandonarem as crianças.

( x ) Os motivos que levaram o pai e a madrasta a pensarem em deixar as
crianças na floresta.

\subsubsection{8. }\label{section-23}

Releia a história e sublinhe no texto:

a) de vermelho, a fala do pai.

b) de verde, a fala da madrasta.

\subsubsection{9. }\label{section-24}

Narrador é aquele que conta a história. Assinale a frase abaixo que
pertence ao narrador.

( x ) O garoto chamava-se João e a menina, Maria.

( ) Minha mulher, o que será de nós?

\subsubsection{10. }\label{section-25}

Releia o trecho a seguir.

\begin{quote}
--- Há uma solução --- disse a madrasta, que era muito malvada.
--- Amanhã daremos a João e Maria um pedaço de pão, depois os levaremos
à mata e lá os abandonaremos.
\end{quote}

Como o leitor sabe quem está falando?

Pelo comentário do narrador depois da fala indicada pelo travessão.

\_\_\_\_\_\_\_\_\_\_\_\_\_\_\_\_\_\_\_\_\_\_\_\_\_\_\_\_\_\_\_\_\_\_\_\_\_\_\_\_\_\_\_\_\_\_\_\_\_\_\_\_\_\_\_\_\_\_\_\_\_\_\_\_\_\_\_\_\_\_\_\_\_\_\_\_\_\_\_\_\_\_\_\_\_\_\_\_\_\_\_\_\_\_\_\_\_\_\_\_\_\_\_\_\_\_\_\_\_\_\_\_\_\_\_\_\_\_\_\_\_\_\_\_\_\_\_\_\_\_\_\_\_\_\_\_\_\_\_\_

\subsubsection{11. }\label{section-26}

Quem conta a história de João e Maria? Assinale a alternativa correta.

\coment{Explique aos alunos que, em algumas narrativas, pode acontecer de o
narrador não indicar quem fala; a identificação ocorre pela sequência do
discurso e pela apresentação por meio dos verbos de enunciação.}

( ) Um narrador que participa da história (narração em primeira pessoa).

( x ) Um narrador que não participa das ações (narração em terceira
pessoa).

\subsubsection{12. }\label{section-27}

Releia o trecho a seguir.

\begin{quote}
--- Amanhã daremos a João e Maria um pedaço de pão, depois os
levaremos à mata e lá os abandonaremos.
\end{quote}

a) Circule o sinal usado para destacar a fala da madrasta.

b) Qual é o nome desse sinal que você circulou? Travessão.

\_\_\_\_\_\_\_\_\_\_\_\_\_\_\_\_\_\_\_\_\_\_\_\_\_\_\_\_\_\_\_\_\_\_\_\_\_\_\_\_\_\_\_\_\_\_\_\_\_\_\_\_\_\_\_\_\_\_\_\_\_\_\_\_\_\_\_

\subsubsection{13. }\label{section-28}

Os verbos de enunciação são verbos que introduzem a fala. Tendo isso em
vista, releia este trecho:

\begin{quote}
--- Há uma solução --- disse a madrasta, que era muito malvada.
\end{quote}

\coment{Caso julgue pertinente, incentive a observação de como ficaria esta fala
em um discurso indireto, para que os alunos percebam os efeitos de sentido produzidos
pelos verbos de enunciação no discurso direto.}

Qual palavra indica o que a madrasta fez? A forma verbal ``disse''.

\_\_\_\_\_\_\_\_\_\_\_\_\_\_\_\_\_\_\_\_\_\_\_\_\_\_\_\_\_\_\_\_\_\_\_\_\_\_\_\_\_\_\_\_\_\_\_\_\_\_\_\_\_\_\_\_\_\_\_\_\_\_\_\_\_\_\_

\subsection{Treino}\label{treino-1}

\subsubsection{1. }\label{section-29}

Leia o trecho de um texto, atentando-se às falas dos personagens.

\begin{quote}
\textbf{A cabra cabriola}
\textbf{Cena 1}
(Maria brinca no pátio e a mãe entra.)

\emph{Mãe}: Maria, minha filhinha, agora vou trabalhar! Você vai ficar
quietinha. Não saia a passear, pois a cabra cabriola anda por este
lugar!

(Maria nem escuta, continua brincando.)

\emph{Mãe}: Maria, sua teimosa! Faça o favor de escutar: se você for
sequestrada, se a cabra a pegar, não tenho dinheiro e joias para o
resgate pagar!

\emph{Maria}: Ah, mamãe, não acredito nessa história. É inventada! Mas
pode ir. Aqui fico neste batente, sentada. {[}...{]}

(A mãe sai e Maria vai passear.)

\emph{Maria}: Obedecer? Ah, quem disse? Vou sair a passear, caçar
borboletas, ninhos, correr, pular e brincar, até pegar passarinhos para
em gaiolas criar!

{[}...{]}

\fonte{Lourdes Ramalho. Teatro na escola. A cabra cabriola. Disponível
em:
\emph{www.teatronaescola.com/index.php/banco-de-pecas/category/lourdes-ramalho}.
Acesso em: 09 fev. 2023. (Adaptado.)}
\end{quote}

O texto apresentado é

(A) uma lenda.

(B) uma poesia.

(C) um texto teatral.

(D) uma fábula.

SAEB: Reconhecer diferentes gêneros textuais.
BNCC: EF35LP24 -- Identificar funções do texto dramático (escrito para ser
encenado) e sua organização por meio de diálogos entre personagens e
marcadores das falas das personagens e de cena.

(A) Incorreta. O texto não explica acontecimentos misteriosos ou
sobrenaturais.

(B) Incorreta. O texto não está organizado em versos.

(C) Correta. O texto apresenta estrutura do texto teatral, marcada pela
descrição dos movimentos de cena (texto secundário indicado entre
parênteses) e pela organização do enredo em falas.

(D) Incorreta. Nas fábulas, não há descrição de movimentações de cena.

\subsubsection{2. }\label{section-30}

Leia um trecho de texto teatral inspirado em uma fábula de Esopo.

\begin{quote}
\textbf{A toupeira avarenta}

Personagens: toupeira, tatu, avestruz.

Cenário: um campo.

(Toupeira está cavando um buraco. É observada, de longe, por um tatu.)

\emph{Toupeira}: Meu tesouro, cadê você, meu tesouro?

\emph{Tatu} (à parte): Ora, ora, ora...

\emph{Toupeira} (falando a uma barra de ouro que acaba de tirar do
buraco): Ah, aí está você: tudo o que tenho é esta bela barra de ouro.

\emph{Tatu} (à parte): Ora, ora, ora...

\emph{Toupeira} (enterrando novamente a barra de ouro): Bem, já chega.
Amanhã eu volto pra ver você de novo...

{[}...{]}

\fonte{José Carlos Aragão. \emph{No palco, todo mundo vira bicho:} novas
fábulas de Esopo adaptadas para teatro. São Paulo: Planeta do Brasil,
2007. p. 39.}
\end{quote}

Textos como esse são semelhantes:

(A) ao roteiro de cinema.

(B) à entrevista.

(C) à reportagem.

(D) ao jornal de rádio.

SAEB: Identificar as marcas de organização de textos dramáticos.
BNCC: EF35LP24 -- Identificar funções do texto dramático (escrito para
ser encenado) e sua organização por meio de diálogos entre personagens e
marcadores das falas das personagens e de cena.

(A) Correta. Roteiros cinematográficos, assim como as peças teatrais,
são elaborados para serem encenados por atores em filmes, com marcas
no texto que são tipicamente feitas para esse fim.

(B) Incorreta. Entrevista consiste em gênero informativo que apresenta
entrevistador e entrevistado.

(C) Incorreta. Reportagens são textos informativos veiculados por
diversos meios de comunicação e têm como conteúdo informações reais e
atuais.

(D) Incorreta. Jornal de rádio transmite as notícias da atualidade por
meio de emissões radiofônicas.

\subsubsection{3. }\label{section-31}

Leia um trecho do conto ``Água da vida''.

\begin{quote}
\textbf{Água da vida}

Houve, uma vez, um rei muito poderoso, que vivia feliz e tranquilo em
seu reino. Um belo dia, adoeceu gravemente e ninguém tinha esperanças de
que escapasse. Ele tinha três filhos {[}...{]}.

Encontravam-se eles no jardim do castelo a chorar e, de repente, viram
surgir à sua frente um velho de aspecto venerável, que indagou a causa
de tamanha tristeza. Disseram-lhe que estavam aflitos porque o pai
estava gravemente enfermo e os médicos já não tinham esperanças de o
salvar.

O velho, então, disse-lhes:

--- Eu conheço um remédio muito eficaz, que poderá curá-lo; é a famosa
água da vida. Mas é muito difícil obtê-la.

{[}...{]}

\fonte{Irmãos Grimm. A água da vida. Disponível em:
\emph{https://www.grimmstories.com/pt/grimm\_contos/a\_agua\_da\_vida}. Acesso
em: 16 fev. 2023.}
\end{quote}

Em relação ao narrador do conto, ele

(A) participa da história.

(B) não participa da história.

(C) é personagem da história.

(D) realiza ações da história.

SAEB: Identificar elementos constitutivos de textos narrativos.
BNCC: EF35LP26 -- Ler e compreender, com certa autonomia, narrativas
ficcionais que apresentem cenários e personagens, observando os
elementos da estrutura narrativa: enredo, tempo, espaço, personagens,
narrador e a construção do discurso indireto e discurso direto.

(A) Incorreta. Não existe ``eu'' ou ``nós'' no conto para se afirmar
que o narrador participe da história.

(B) Correta. O narrador não realiza as ações da história, mas narra os
acontecimentos.

(C) Incorreta. O narrador não é um personagem da história.

(D) Incorreta. O narrador não pratica nenhuma ação narrada.

\section{3. Sinais de pontuação}\label{muxf3dulo-3}

\coment{Neste módulo, espera-se que os alunos identifiquem os sinais de
pontuação e analisem os efeitos de sentido decorrentes do uso da
pontuação em diversos tipos de textos.}

\colorsec{Habilidades do SAEB}

\begin{itemize}
  \item Analisar elementos constitutivos de gêneros textuais diversos.
  \item Reconhecer os usos da pontuação.
  \item Analisar os efeitos de sentido decorrentes do uso da pontuação.
\end{itemize}

\colorsec{Habilidades da BNCC}

\begin{iteize}
  \item EF03LP07, EF03LP16.
\end{itemize}

\subsection{Conteúdo}\label{conteuxfado-2}

\textbf{https://pixabay.com/pt/illustrations/sinais-de-pontua\%c3\%a7\%c3\%a3o-palavra-l\%c3\%adngua-2999583/}

\includegraphics[width=3.41667in,height=3.41667in]{media/image4.jpeg}

Os sinais de pontuação têm variadas funções em um texto escrito e ajudam
bastante a compreender a construção de sentidos pretendida por quem
escreve um texto. Para entender melhor, leia um trecho do conto ``O
rouxinol do imperador'', atentando-se à pontuação.

\begin{quote}
\textbf{O rouxinol do imperador}

{[}...{]}

Um dia, um {[}livro{]} chegou às mãos do imperador. O soberano o leu e
ficou, ao mesmo tempo, surpreso e enfurecido. Mandou logo chamar o
primeiro-ministro.

--- Incrível! No bosque que faz divisa com os jardins imperiais, vive um
rouxinol cujo canto é incomparável, e eu o desconheço! Tive que ler um
livro estrangeiro para aprender que a maior maravilha de meu país é um
pássaro de voz de ouro, e não este meu soberbo palácio! Diga-me, por que
não fui informado?

--- Eu também ignorava o fato, meu senhor --- respondeu o
primeiro-ministro, assustado com a ira do imperador. --- Mas vou
descobri-lo.

--- E que seja muito breve. Nesta noite mesmo o rouxinol deverá cantar
somente para mim.

{[}...{]}

\fonte{Hans Christian Andersen. O rouxinol do imperador. Disponível em:
\emph{http://www.dominiopublico.gov.br/download/texto/me000589.pdf}.
Acesso em: 16 fev. 2023.}
\end{quote}

O texto apresenta vários sinais de pontuação: ponto final (\textbf{.}),
travessão (\textbf{---}), ponto de exclamação (\textbf{!}) e ponto de
interrogação (\textbf{?}). Perceba que o travessão marca a voz das
personagens; e esses diálogos fazem com que o leitor tenha a sensação de
assistir à conversa, como se acontecesse no momento em que lê. O ponto
de exclamação intensifica os sentimentos de indignação. O ponto de
interrogação indica a entonação utilizada ao se fazer pergunta. O ponto
final marca o fim de uma frase.

\textbf{\textless{}fim do boxe de conteúdo\textgreater{}}

\subsection{Atividades}\label{atividades-2}

\subsubsection{1. }\label{section-32}

Leia o texto a seguir. Depois, responda aos itens propostos.

Inserir imagem ao lado do texto:
https://pixabay.com/pt/vectors/lobo-bonitinho-animal-personagem-1454420/

\includegraphics[width=1.97917in,height=3.16667in]{media/image5.png}

\begin{quote}
\textbf{O lobo e o cordeiro}

Um lobo estava bebendo água num riacho. Um cordeirinho chegou e também
começou a beber, um pouco mais para baixo.

O lobo arreganhou os dentes e disse ao cordeiro:

--- Como é que você tem a ousadia de vir sujar a água que estou bebendo?

--- Como sujar? --- Respondeu o cordeiro. --- A água corre daí para cá,
logo eu não posso estar sujando sua água.

--- Não me responda! --- Tornou o lobo furioso.

--- Há seis meses seu pai me fez a mesma coisa!

--- Há seis meses eu nem tinha nascido. Como é que eu posso ter culpa
disso? --- Respondeu o cordeiro.

--- Mas você estragou todo o meu pasto --- Replicou o lobo.

--- Como é que posso ter estragado seu pasto, se nem dentes eu tenho?

O lobo, não tendo mais como culpar o cordeiro, não disse mais nada:
pulou sobre ele e o devorou.

\fonte{O lobo e o cordeiro. Disponível em:
\emph{http://www.dominiopublico.gov.br/download/texto/me000589.pdf}.
Acesso em: 16 fev. 2023.}
\end{quote}

a) Quem são os personagens do texto? O lobo e o cordeiro.

\_\_\_\_\_\_\_\_\_\_\_\_\_\_\_\_\_\_\_\_\_\_\_\_\_\_\_\_\_\_\_\_\_\_\_\_\_\_\_\_\_\_\_\_\_\_\_\_\_\_\_\_\_\_\_\_\_\_\_\_\_\_\_\_\_\_\_\_\_\_\_\_\_\_\_\_\_\_\_\_\_\_\_\_\_\_\_\_\_\_\_\_\_\_\_\_\_\_\_\_\_\_\_\_\_\_\_\_\_\_\_\_\_\_\_\_\_\_\_\_\_\_\_\_\_\_\_\_\_\_\_\_\_\_\_\_\_\_\_\_

b) Onde a história acontece? Provavelmente, na floresta, em um riacho.

\_\_\_\_\_\_\_\_\_\_\_\_\_\_\_\_\_\_\_\_\_\_\_\_\_\_\_\_\_\_\_\_\_\_\_\_\_\_\_\_\_\_\_\_\_\_\_\_\_\_\_\_\_\_\_\_\_\_\_\_\_\_\_\_\_\_\_\_\_\_\_\_\_\_\_\_\_\_\_\_\_\_\_\_\_\_\_\_\_\_\_\_\_\_\_\_\_\_\_\_\_\_\_\_\_\_\_\_\_\_\_\_\_\_\_\_\_\_\_\_\_\_\_\_\_\_\_\_\_\_\_\_\_\_\_\_\_\_\_\_

c) Qual é o assunto tratado no texto? Trata-se da história de um lobo
que queria encontrar motivos para matar um cordeiro.

\_\_\_\_\_\_\_\_\_\_\_\_\_\_\_\_\_\_\_\_\_\_\_\_\_\_\_\_\_\_\_\_\_\_\_\_\_\_\_\_\_\_\_\_\_\_\_\_\_\_\_\_\_\_\_\_\_\_\_\_\_\_\_\_\_\_\_\_\_\_\_\_\_\_\_\_\_\_\_\_\_\_\_\_\_\_\_\_\_\_\_\_\_\_\_\_\_\_\_\_\_\_\_\_\_\_\_\_\_\_\_\_\_\_\_\_\_\_\_\_\_\_\_\_\_\_\_\_\_\_\_\_\_\_\_\_\_\_\_\_

\_\_\_\_\_\_\_\_\_\_\_\_\_\_\_\_\_\_\_\_\_\_\_\_\_\_\_\_\_\_\_\_\_\_\_\_\_\_\_\_\_\_\_\_\_\_\_\_\_\_\_\_\_\_\_\_\_\_\_\_\_\_\_\_\_\_\_\_\_\_

\subsubsection{2. }\label{section-33}

Releia este trecho do texto.

\begin{quote}
--- Não me responda! --- Tornou o lobo furioso.
--- Há seis meses seu pai me fez a mesma coisa!
--- Há seis meses eu nem tinha nascido, como é que eu posso ter culpa disso? --- respondeu o cordeiro.
\end{quote}

a) Que sinais de pontuação aparecem em final de frase nesse trecho? Ponto final,
ponto de exclamação e ponto de interrogação.

\protect\hypertarget{_Hlk127460239}{}{}\_\_\_\_\_\_\_\_\_\_\_\_\_\_\_\_\_\_\_\_\_\_\_\_\_\_\_\_\_\_\_\_\_\_\_\_\_\_\_\_\_\_\_\_\_\_\_\_\_\_\_\_\_\_\_\_\_\_\_\_\_\_\_\_\_\_\_\_\_\_\_\_\_\_\_\_\_\_\_\_\_\_\_\_\_\_\_\_\_\_\_\_\_\_\_\_\_\_\_\_\_\_\_\_\_\_\_\_\_\_\_\_\_\_\_\_\_\_\_\_\_\_\_\_\_\_\_\_\_\_\_\_\_\_\_\_\_\_\_\_

\_\_\_\_\_\_\_\_\_\_\_\_\_\_\_\_\_\_\_\_\_\_\_\_\_\_\_\_\_\_\_\_\_\_\_\_\_\_\_\_\_\_\_\_\_\_\_\_\_\_\_\_\_\_\_\_\_\_\_\_\_\_\_\_\_\_\_\_\_\_

b) Explique a função de cada um desses sinais de pontuação. Usa-se o
ponto final quando se quer finalizar uma frase ou um período. Utiliza-se
o ponto de exclamação para enfatizar o que foi dito. Usa-se o ponto de
interrogação quando se formula uma pergunta.

\coment{Explique aos alunos que frases terminadas com ponto final denominam-se
\textbf{frases declarativas}; as terminadas com ponto de interrogação
denominam-se \textbf{frases interrogativas}; as que terminam com ponto
de exclamação denominam-se \textbf{frases exclamativas}.}

\_\_\_\_\_\_\_\_\_\_\_\_\_\_\_\_\_\_\_\_\_\_\_\_\_\_\_\_\_\_\_\_\_\_\_\_\_\_\_\_\_\_\_\_\_\_\_\_\_\_\_\_\_\_\_\_\_\_\_\_\_\_\_\_\_\_\_\_\_\_\_\_\_\_\_\_\_\_\_\_\_\_\_\_\_\_\_\_\_\_\_\_\_\_\_\_\_\_\_\_\_\_\_\_\_\_\_\_\_\_\_\_\_\_\_\_\_\_\_\_\_\_\_\_\_\_\_\_\_\_\_\_\_\_\_\_\_\_\_\_

\_\_\_\_\_\_\_\_\_\_\_\_\_\_\_\_\_\_\_\_\_\_\_\_\_\_\_\_\_\_\_\_\_\_\_\_\_\_\_\_\_\_\_\_\_\_\_\_\_\_\_\_\_\_\_\_\_\_\_\_\_\_\_\_\_\_\_\_\_\_

\_\_\_\_\_\_\_\_\_\_\_\_\_\_\_\_\_\_\_\_\_\_\_\_\_\_\_\_\_\_\_\_\_\_\_\_\_\_\_\_\_\_\_\_\_\_\_\_\_\_\_\_\_\_\_\_\_\_\_\_\_\_\_\_\_\_\_\_\_\_

c) Que função tem o segundo travessão que aparece após uma fala?

A segunda ocorrência do travessão marca o fim da fala do personagem e o
início da intervenção do narrador.

\begin{quote}
\_\_\_\_\_\_\_\_\_\_\_\_\_\_\_\_\_\_\_\_\_\_\_\_\_\_\_\_\_\_\_\_\_\_\_\_\_\_\_\_\_\_\_\_\_\_\_\_\_\_\_\_\_\_\_\_\_\_\_\_\_\_\_\_\_\_\_

\_\_\_\_\_\_\_\_\_\_\_\_\_\_\_\_\_\_\_\_\_\_\_\_\_\_\_\_\_\_\_\_\_\_\_\_\_\_\_\_\_\_\_\_\_\_\_\_\_\_\_\_\_\_\_\_\_\_\_\_\_\_\_\_\_\_\_

\_\_\_\_\_\_\_\_\_\_\_\_\_\_\_\_\_\_\_\_\_\_\_\_\_\_\_\_\_\_\_\_\_\_\_\_\_\_\_\_\_\_\_\_\_\_\_\_\_\_\_\_\_\_\_\_\_\_\_\_\_\_\_\_\_\_\_
\end{quote}

\subsubsection{3. }\label{section-34}

Releia o fim da história.

\begin{quote}
O lobo, não tendo mais como culpar o cordeiro, não disse mais nada:
pulou sobre ele e o devorou.
\end{quote}

\begin{itemize}
\item
  O que essa frase expressa? Marque a alternativa correta.
\end{itemize}

( ) Dúvida.

( ) Surpresa.

( x ) Afirmação.

\subsubsection{4. }\label{section-35}

Leia o trecho de um conto tradicional.

\textbf{\textless{}Arte: colocar cor nos sinais destacados com
grifa-texto.\textgreater{}}

Inserir imagem ao lado do texto:
https://www.istockphoto.com/br/vetor/branca-de-neve-ilustra\%C3\%A7\%C3\%A3o-vetorial-gm521993146-91504715?utm\_source=pixabay\&utm\_medium=affiliate\&utm\_campaign=SRP\_illustration\_sponsored\&utm\_content=https\%3A\%2F\%2Fpixabay.com\%2Fpt\%2Fillustrations\%2Fsearch\%2Fbranca\%2520de\%2520neve\%2F\&utm\_term=branca+de+neve

\includegraphics[width=2.71875in,height=2.71875in]{media/image6.jpeg}

\begin{quote}
\textbf{Branca de Neve}

{[}...{]}

Alguns meses depois, o desejo da rainha foi atendido. Ela deu à luz uma
menina de cabelos bem pretos, pele branca e face rosada. O nome dado à
princesinha foi Branca de Neve.

Mas, quando nasceu a menina, a rainha morreu. Passado um ano, o rei se
casou novamente. Sua esposa era lindíssima, mas muito vaidosa, invejosa
e cruel\textbf{.}

Um feiticeiro lhe dera um espelho mágico, ao qual todos os dias ela
perguntava, com vaidade\textbf{:}

\textbf{---} Espelho, espelho meu, diga-me se há no mundo mulher mais bela do que
eu.

E o espelho respondia:

--- Em todo o mundo, minha querida rainha, não existe beleza maior.

{[}...{]}

\fonte{Branca de Neve. Disponível em:
\emph{http://www.dominiopublico.gov.br/download/texto/me000589.pdf}. Acesso em:
16 fev. 2023.}
\end{quote}

\begin{enumerate}
\def\labelenumi{\alph{enumi})}
\item
  O que a esposa do rei perguntava todos os dias ao espelho? Ela
  perguntava se no mundo havia mulher mais bela do que ela.
\end{enumerate}

\_\_\_\_\_\_\_\_\_\_\_\_\_\_\_\_\_\_\_\_\_\_\_\_\_\_\_\_\_\_\_\_\_\_\_\_\_\_\_\_\_\_\_\_\_\_\_\_\_\_\_\_\_\_\_\_\_\_\_\_\_\_\_\_

\_\_\_\_\_\_\_\_\_\_\_\_\_\_\_\_\_\_\_\_\_\_\_\_\_\_\_\_\_\_\_\_\_\_\_\_\_\_\_\_\_\_\_\_\_\_\_\_\_\_\_\_\_\_\_\_\_\_\_\_\_\_\_\_

\begin{enumerate}
\def\labelenumi{\alph{enumi})}
\item
  Dos sinais de pontuação que estão destacados no texto, qual indica o
  início da fala do personagem? O travessão.
\end{enumerate}

\begin{quote}
\protect\hypertarget{_Hlk127463829}{}{}\_\_\_\_\_\_\_\_\_\_\_\_\_\_\_\_\_\_\_\_\_\_\_\_\_\_\_\_\_\_\_\_\_\_\_\_\_\_\_\_\_\_\_\_\_\_\_\_\_\_\_\_\_\_\_\_\_\_\_\_\_\_\_\_\_\_\_

c) Qual sinal anuncia que o personagem vai falar? Os dois-pontos.

\_\_\_\_\_\_\_\_\_\_\_\_\_\_\_\_\_\_\_\_\_\_\_\_\_\_\_\_\_\_\_\_\_\_\_\_\_\_\_\_\_\_\_\_\_\_\_\_\_\_\_\_\_\_\_\_\_\_\_\_\_\_\_\_\_\_\_

d) Se a frase ``--- Espelho, espelho meu, diga-me se há no mundo mulher
mais bela do que eu.'' fosse uma pergunta, como ela seria escrita?
Marque a alternativa correta.
\end{quote}

( x ) --- Espelho, espelho meu, há no mundo mulher mais bela do que eu?

( ) --- Espelho, espelho meu, diga-me se há no mundo mulher mais bela do
que eu!

\subsection{Treino}\label{treino-2}

\subsubsection{1.}\label{section-36}

Leia o anúncio da campanha de vacinação contra o sarampo.

\begin{quote}
https://www.pmna.ms.gov.br/noticias/saude/sarampo-filhos-de-seis-meses-a-menores-de-um-ano-de-idade-que-irao-viajar-para-municipios-em-situacao-de-surto-ativo-devem-vacinar
\end{quote}

\includegraphics[width=4.68681in,height=2.81736in]{media/image7.jpeg}

A frase do anúncio é:

\begin{enumerate}
\def\labelenumi{(\Alph{enumi})}
\item
  declarativa.
\item
  exclamativa.
\item
  interrogativa.
\item
  negativa.
\end{enumerate}

\begin{quote}
SAEB: Analisar os efeitos de sentido decorrentes do uso da pontuação.
BNCC: EF03LP07 -- Identificar a função na leitura e usar na escrita ponto
final, ponto de interrogação, ponto de exclamação e, em diálogos
(discurso direto), dois-pontos e travessão.

(A) Incorreta. Frases declarativas terminam com ponto final.

(B) Correta. A frase apresenta ponto de exclamação.

(C) Incorreta. A frase não é uma pergunta.

(D) Incorreta. A frase não expressa negação.
\end{quote}

\subsubsection{2. }\label{section-37}

Leia o trecho extraído da história ``O gato de botas''.

\begin{quote}
\textbf{O gato de botas}

Um lavrador trabalhara muito, durante a vida toda, ganhando sempre o
suficiente para o sustento da família. Quando faleceu, deixou sua
herança para os filhos: um sítio, um burrinho e um gato.

Ao filho mais velho coube o sítio; ao segundo, o burrinho; e o caçula
ficou com o gato. Este último, nada satisfeito com o que lhe coubera,
resmungou: ``Meus irmãos sobreviverão honestamente. Mas e eu? O que vou
fazer? Talvez possa jantar o gato e com o couro fazer um tamborim. Mas e
depois?''

O gato logo endireitou as orelhas, querendo ouvir melhor um assunto de
tamanho interesse. Então, percebendo que precisava agir, foi dizendo:

--- Não se desespere, patrãozinho, pois eu tenho um plano. Consiga-me um
par de botas e um saco de pano, e deixe o resto comigo.

{[}...{]}

\fonte{O gato de botas. Disponível em:
\emph{http://www.dominiopublico.gov.br/download/texto/me000589.pdf}.
Acesso em: 17 fev. 2023.}
\end{quote}

Que sinal de pontuação indica a presença de um diálogo no trecho?
\end{quote}

\begin{enumerate}
\def\labelenumi{(\Alph{enumi})}
\item
  A vírgula.
\item
  O ponto final.
\item
  O travessão.
\item
  O ponto de interrogação.
\end{enumerate}

SAEB: Reconhecer os usos da pontuação.
BNCC: EF03LP07 -- Identificar a função na leitura e usar na escrita ponto
final, ponto de interrogação, ponto de exclamação e, em diálogos
(discurso direto), dois-pontos e travessão.

(A) Incorreta. A vírgula não serve para indicar diálogos, mas para
separar elementos dentro de uma mesma frase.

(B) Incorreta. O ponto final é um sinal de pontuação que encerra o
período.

(C) Correta. O travessão é um sinal de pontuação usado especialmente no
início de cada fala no discurso direto.

(D) Incorreta. O ponto de interrogação é usado para indicar uma
pergunta.

\subsubsection{3.}\label{section-38}

Leia o trecho de um manual.

\begin{quote}
\textbf{Manual Aventura Científica}

{[}...{]}

\textbf{1)} Destaque as cartas.
\textbf{2)} Embaralhe as cartas EU QUERO SABER e deixe-as em um monte com a
face Eu Quero Saber virada para cima, no lugar marcado no tabuleiro.
\textbf{3)} Embaralhe as cartas A-HÁ e deixe-as em um monte, com a face A-HÁ
virada para cima, no lugar marcado no tabuleiro.
\textbf{4)} Deixe as peças dos quadros espalhadas ao lado do tabuleiro, com o
lado da ilustração virada para cima, ao alcance de todos os jogadores.
\textbf{5)} Cada jogador escolhe um peão e coloca-o no local marcado no
tabuleiro.

{[}...{]}

\fonte{Manual Aventura Científica. Disponível em:
\emph{https://estrela.vteximg.com.br/arquivos/Manual-Aventura-Cientifica-Show-da-Luna.pdf}. Acesso em: 17 fev. 2023.}
\end{quote}

O objetivo desse texto é

(A) dar informações relacionadas à ciência.

(B) mostrar as instruções relativas a um jogo.

(C) convencer o leitor a jogar um jogo de aventura.

(D) ajudar o leitor a solucionar problemas de um jogo.
\end{quote}

SAEB: Analisar elementos constitutivos de gêneros textuais diversos.
BNCC EF03LP16 -- Identificar e reproduzir, em textos injuntivos
instrucionais (receitas, instruções de montagem, digitais ou impressos),
a formatação própria desses textos (verbos imperativos, indicação de
passos a ser seguidos) e a diagramação específica dos textos desses
gêneros (lista de ingredientes ou materiais e instruções de execução --
"modo de fazer").

(A) Incorreta. O trecho apresenta instruções de um jogo, e não
informações relacionadas à ciência.

(B) Correta. O manual mostra instruções de como jogar o jogo ``Aventura
científica''.

(C) Incorreta. O manual não tem como finalidade convencer o leitor, já
que ele apenas apresenta instruções do jogo.

(D) Incorreta. O texto ajuda o leitor a jogar o jogo, e não a solucionar
problemas.

\section{4. Argumentação}\label{muxf3dulo-4}

\coment{Neste módulo, espera-se que os alunos leiam
e compreendam autonomamente texto do campo da vida pública; relacionem a
imagem no texto à mensagem (linguagens verbal e não verbal); relacionem
a finalidade do texto às estratégias de convencimento; identifiquem a
função social do texto, reconhecendo para que serve e a quem se destina;
identifiquem a ideia central do texto, compreendendo-o globalmente; e
infiram informações implícitas no texto.}

\colorsec{Habilidades do SAEB}

\begin{itemize}
  \item Analisar o uso de recursos de persuasão em textos verbais e/ou
multimodais.
  \item Analisar os efeitos de sentido de recursos multissemióticos em
textos que circulam em diferentes suportes.
  \item Julgar a eficácia de argumentos em textos.
\end{itemize}

\colorsec{Habilidades da BNCC}

\begin{itemize}
  \item EF03LP19.
\end{itemize}

\subsection{Conteúdo}\label{conteuxfado-3}

\textbf{https://pixabay.com/pt/photos/paris-fran\%c3\%a7a-cidade-cidades-urbano-195327/}

\includegraphics[width=4.83333in,height=3.62500in]{media/image8.jpeg}

Anúncio publicitário (ou propaganda) é um gênero textual que
tem como finalidade divulgar uma ideia, um produto, um valor ou um
conceito. Essa divulgação é feita para persuadir, isto é, convencer o
público em relação ao ao que está sendo veiculado, como a comprar um
produto, a realizar uma ação ou a adotar um comportamento.

Como o principal objetivo do anúncio é persuadir o leitor,
todos os elementos nele presentes são muito bem pensados para esse fim. Para atingir
seu objetivo, quem o elabora emprega elementos como
combinação de cores, disposição de elementos visuais, jogo de palavras e
imagens, por exemplo.

A linguagem normalmente é objetiva e o vocabulário utilizado é dirigido
ao público ao qual o anúncio se destina.

Os anúncios podem ser constituídos de \emph{slogans} (frases de efeito)
e apresentam também a instituição ou a marca responsável pela sua
veiculação (logomarca).

Esse gênero textual pode ser divulgado na televisão, em revistas,
jornais, \emph{outdoors}, cartazes e anúncios na internet.

\subsection{Atividades}\label{atividades-3}

Analise o anúncio e resolvas as atividades de 1 a 6.

\coment{Explore com os alunos o cartaz, a imagem que o compõe e os elementos
verbais. Avalie com eles os motivos da escolha do produtor ao dar ênfase
a alguns elementos do texto verbal, e se esse recurso foi efetivo na
divulgação do que se pretendia.}

%Paulo: Inserir ilustração 1, que será produzida.

\includegraphics[width=3.11354in,height=4.44792in]{media/image9.jpeg}

\num{1} Qual é o objetivo desse anúncio publicitário?

Espera-se que os alunos compreendam que o objetivo é incentivar as
pessoas a cuidarem dos animais de estimação.

\_\_\_\_\_\_\_\_\_\_\_\_\_\_\_\_\_\_\_\_\_\_\_\_\_\_\_\_\_\_\_\_\_\_\_\_\_\_\_\_\_\_\_\_\_\_\_\_\_\_\_\_\_\_\_\_\_\_\_\_\_\_\_\_

\_\_\_\_\_\_\_\_\_\_\_\_\_\_\_\_\_\_\_\_\_\_\_\_\_\_\_\_\_\_\_\_\_\_\_\_\_\_\_\_\_\_\_\_\_\_\_\_\_\_\_\_\_\_\_\_\_\_\_\_\_\_\_\_

\num{2} Para quem esse anúncio é destinado? É destinado aos donos de animais
de estimação.

\_\_\_\_\_\_\_\_\_\_\_\_\_\_\_\_\_\_\_\_\_\_\_\_\_\_\_\_\_\_\_\_\_\_\_\_\_\_\_\_\_\_\_\_\_\_\_\_\_\_\_\_\_\_\_\_\_\_\_\_\_\_\_\_

\_\_\_\_\_\_\_\_\_\_\_\_\_\_\_\_\_\_\_\_\_\_\_\_\_\_\_\_\_\_\_\_\_\_\_\_\_\_\_\_\_\_\_\_\_\_\_\_\_\_\_\_\_\_\_\_\_\_\_\_\_\_\_\_

\num{3} Segundo o anúncio, por que os animais não devem ser abandonados? Os animais não devem ser abandonados, porque eles sentem fome, frio e medo.

\_\_\_\_\_\_\_\_\_\_\_\_\_\_\_\_\_\_\_\_\_\_\_\_\_\_\_\_\_\_\_\_\_\_\_\_\_\_\_\_\_\_\_\_\_\_\_\_\_\_\_\_\_\_\_\_\_\_\_\_\_\_\_\_

\_\_\_\_\_\_\_\_\_\_\_\_\_\_\_\_\_\_\_\_\_\_\_\_\_\_\_\_\_\_\_\_\_\_\_\_\_\_\_\_\_\_\_\_\_\_\_\_\_\_\_\_\_\_\_\_\_\_\_\_\_\_\_\_

\num{4} O cachorro que aparece no anúncio aparenta estar

\begin{escolha}
  \item feliz.
  \item triste.
  \item cansado.
  \item eufórico.
\end{escolha}

\coment{O cachorro aparenta estar triste.}

\num{5} Em sua opinião, na frase ``Cachorro não é brinquedo'', por que a
palavra \textbf{não} está com uma cor diferente do restante do texto?
Resposta pessoal.

\_\_\_\_\_\_\_\_\_\_\_\_\_\_\_\_\_\_\_\_\_\_\_\_\_\_\_\_\_\_\_\_\_\_\_\_\_\_\_\_\_\_\_\_\_\_\_\_\_\_\_\_\_\_\_\_\_\_\_\_\_\_\_\_

\_\_\_\_\_\_\_\_\_\_\_\_\_\_\_\_\_\_\_\_\_\_\_\_\_\_\_\_\_\_\_\_\_\_\_\_\_\_\_\_\_\_\_\_\_\_\_\_\_\_\_\_\_\_\_\_\_\_\_\_\_\_\_\_

\_\_\_\_\_\_\_\_\_\_\_\_\_\_\_\_\_\_\_\_\_\_\_\_\_\_\_\_\_\_\_\_\_\_\_\_\_\_\_\_\_\_\_\_\_\_\_\_\_\_\_\_\_\_\_\_\_\_\_\_\_\_\_\_

\num{6} Em sua opinião, o anúncio está cumprindo sua finalidade? Justifique
sua resposta. Resposta pessoal.

\_\_\_\_\_\_\_\_\_\_\_\_\_\_\_\_\_\_\_\_\_\_\_\_\_\_\_\_\_\_\_\_\_\_\_\_\_\_\_\_\_\_\_\_\_\_\_\_\_\_\_\_\_\_\_\_\_\_\_\_\_\_\_\_

\_\_\_\_\_\_\_\_\_\_\_\_\_\_\_\_\_\_\_\_\_\_\_\_\_\_\_\_\_\_\_\_\_\_\_\_\_\_\_\_\_\_\_\_\_\_\_\_\_\_\_\_\_\_\_\_\_\_\_\_\_\_\_\_

\_\_\_\_\_\_\_\_\_\_\_\_\_\_\_\_\_\_\_\_\_\_\_\_\_\_\_\_\_\_\_\_\_\_\_\_\_\_\_\_\_\_\_\_\_\_\_\_\_\_\_\_\_\_\_\_\_\_\_\_\_\_\_\_

\num{7} Veja a seguir mais um exemplo de cartaz de campanha publicitária.

https://www.comunicaquemuda.com.br/monstros-avisam-que-lavar-as-maos-protege-de-doencas/

\includegraphics[width=3.27083in,height=4.76432in]{media/image10.jpeg}

a) Qual é o objetivo desse anúncio?

O objetivo é orientar a população a
higienizar as mãos.

\protect\hypertarget{_Hlk127604103}{}{}\_\_\_\_\_\_\_\_\_\_\_\_\_\_\_\_\_\_\_\_\_\_\_\_\_\_\_\_\_\_\_\_\_\_\_\_\_\_\_\_\_\_\_\_\_\_\_\_\_\_\_\_\_\_\_\_\_\_\_\_\_\_\_\_

\_\_\_\_\_\_\_\_\_\_\_\_\_\_\_\_\_\_\_\_\_\_\_\_\_\_\_\_\_\_\_\_\_\_\_\_\_\_\_\_\_\_\_\_\_\_\_\_\_\_\_\_\_\_\_\_\_\_\_\_\_\_\_\_

b) Você acha importante lavar as mãos? Por quê?

Resposta pessoal. Espera-se que os alunos respondam que sim, porque é um
hábito que pode impedir a propagação de diversas doenças.

\_\_\_\_\_\_\_\_\_\_\_\_\_\_\_\_\_\_\_\_\_\_\_\_\_\_\_\_\_\_\_\_\_\_\_\_\_\_\_\_\_\_\_\_\_\_\_\_\_\_\_\_\_\_\_\_\_\_\_\_\_\_\_\_

\_\_\_\_\_\_\_\_\_\_\_\_\_\_\_\_\_\_\_\_\_\_\_\_\_\_\_\_\_\_\_\_\_\_\_\_\_\_\_\_\_\_\_\_\_\_\_\_\_\_\_\_\_\_\_\_\_\_\_\_\_\_\_\_

\_\_\_\_\_\_\_\_\_\_\_\_\_\_\_\_\_\_\_\_\_\_\_\_\_\_\_\_\_\_\_\_\_\_\_\_\_\_\_\_\_\_\_\_\_\_\_\_\_\_\_\_\_\_\_\_\_\_\_\_\_\_\_\_

\_\_\_\_\_\_\_\_\_\_\_\_\_\_\_\_\_\_\_\_\_\_\_\_\_\_\_\_\_\_\_\_\_\_\_\_\_\_\_\_\_\_\_\_\_\_\_\_\_\_\_\_\_\_\_\_\_\_\_\_\_\_\_\_

c) Segundo o anúncio, o que é preciso usar para lavar as mãos e proteger a saúde?

Água e sabão.

\_\_\_\_\_\_\_\_\_\_\_\_\_\_\_\_\_\_\_\_\_\_\_\_\_\_\_\_\_\_\_\_\_\_\_\_\_\_\_\_\_\_\_\_\_\_\_\_\_\_\_\_\_\_\_\_\_\_\_\_\_\_\_\_

\_\_\_\_\_\_\_\_\_\_\_\_\_\_\_\_\_\_\_\_\_\_\_\_\_\_\_\_\_\_\_\_\_\_\_\_\_\_\_\_\_\_\_\_\_\_\_\_\_\_\_\_\_\_\_\_\_\_\_\_\_\_\_\_

d) O que a imagem da mão está representando?

A imagem representa os micróbios prejudiciais que ficam nas mãos quando não estão higienizadas.

\_\_\_\_\_\_\_\_\_\_\_\_\_\_\_\_\_\_\_\_\_\_\_\_\_\_\_\_\_\_\_\_\_\_\_\_\_\_\_\_\_\_\_\_\_\_\_\_\_\_\_\_\_\_\_\_\_\_\_\_\_\_\_\_

\_\_\_\_\_\_\_\_\_\_\_\_\_\_\_\_\_\_\_\_\_\_\_\_\_\_\_\_\_\_\_\_\_\_\_\_\_\_\_\_\_\_\_\_\_\_\_\_\_\_\_\_\_\_\_\_\_\_\_\_\_\_\_\_

e) \emph{Slogan} é uma frase curta que se destina a prender a atenção do
público. Qual é \emph{o slogan} presente nesse anúncio?

Afaste os bichos, lave as mãos.

\_\_\_\_\_\_\_\_\_\_\_\_\_\_\_\_\_\_\_\_\_\_\_\_\_\_\_\_\_\_\_\_\_\_\_\_\_\_\_\_\_\_\_\_\_\_\_\_\_\_\_\_\_\_\_\_\_\_\_\_\_\_\_\_

\_\_\_\_\_\_\_\_\_\_\_\_\_\_\_\_\_\_\_\_\_\_\_\_\_\_\_\_\_\_\_\_\_\_\_\_\_\_\_\_\_\_\_\_\_\_\_\_\_\_\_\_\_\_\_\_\_\_\_\_\_\_\_\_

f) Assinale as frases que mostram características que podem ser
observadas no \emph{slogan} do cartaz.

( ) Frase longa com excesso de informação.

( x ) Frase curta e de fácil memorização.

( x ) Letras com destaque, para chamar a atenção do leitor.

( ) Conteúdo não relacionado ao objeto principal do anúncio.

\subsubsection{8}\label{section-41}

Com o anúncio presente na atividade anterior, forme dupla com um colega. Juntos, criem outro \emph{slogan} para a
campanha. Registre o resultado no espaço a seguir.

\coment{Lembre os alunos de fazerem uso do imperativo nas formas verbais
utilizadas como recurso para convencer o leitor.}

\begin{longtable}[]{@{}l@{}}
\toprule
\begin{minipage}[t]{0.97\columnwidth}\raggedright\strut
\_\_\_\_\_\_\_\_\_\_\_\_\_\_\_\_\_\_\_\_\_\_\_\_\_\_\_\_\_\_\_\_\_\_\_\_\_\_\_\_\_\_\_\_\_\_\_\_\_\_\_\_\_\_\_\_\_\_\_\_\_\_\_

\_\_\_\_\_\_\_\_\_\_\_\_\_\_\_\_\_\_\_\_\_\_\_\_\_\_\_\_\_\_\_\_\_\_\_\_\_\_\_\_\_\_\_\_\_\_\_\_\_\_\_\_\_\_\_\_\_\_\_\_\_\_\_

\_\_\_\_\_\_\_\_\_\_\_\_\_\_\_\_\_\_\_\_\_\_\_\_\_\_\_\_\_\_\_\_\_\_\_\_\_\_\_\_\_\_\_\_\_\_\_\_\_\_\_\_\_\_\_\_\_\_\_\_\_\_\_

\_\_\_\_\_\_\_\_\_\_\_\_\_\_\_\_\_\_\_\_\_\_\_\_\_\_\_\_\_\_\_\_\_\_\_\_\_\_\_\_\_\_\_\_\_\_\_\_\_\_\_\_\_\_\_\_\_\_\_\_\_\_\_

\_\_\_\_\_\_\_\_\_\_\_\_\_\_\_\_\_\_\_\_\_\_\_\_\_\_\_\_\_\_\_\_\_\_\_\_\_\_\_\_\_\_\_\_\_\_\_\_\_\_\_\_\_\_\_\_\_\_\_\_\_\_\_\strut
\end{minipage}\tabularnewline
\bottomrule
\end{longtable}

\subsubsection{9}\label{section-42}

Encontre, no caça-palavra a seguir, palavras que representam
características do texto publicitário.

CONVENCER --- LOGOMARCA --- \textit{SLOGAN} --- TÍTULO

\begin{longtable}[]{@{}llllllllllll@{}}
\toprule
K & T & R & S & E & Y & A & E & T & E & T & F\tabularnewline
\midrule
\endhead
P & D & A & T & S & T & O & O & E & E & U & S\tabularnewline
A & T & P & I & Í & E & H & N & E & C & P & P\tabularnewline
S & E & I & A & D & T & S & T & M & E & C & L\tabularnewline
A & A & A & G & S & R & U & C & N & D & O & A\tabularnewline
M & T & O & W & I & E & S & L & E & G & N & U\tabularnewline
T & S & C & T & U & R & N & A & O & A & V & T\tabularnewline
E & P & N & O & R & T & A & M & A & D & E & N\tabularnewline
O & P & S & L & O & G & A & N & T & A & N & E\tabularnewline
E & H & O & S & A & R & I & E & T & B & C & O\tabularnewline
N & W & P & E & C & N & W & R & E & T & E & C\tabularnewline
H & Y & B & A & R & Y & U & O & I & T & R & S\tabularnewline
\bottomrule
\end{longtable}

\subsubsection{10}\label{section-43}

Leia, agora, outro anúncio de campanha publicitária.

%Paulo: Inseriri a ilustração 2, que será produzida.

a) Qual é o objetivo da campanha anunciada?

Trata-se de uma campanha para arrecadação de brinquedos e roupas.

\protect\hypertarget{_Hlk127709586}{}{}\_\_\_\_\_\_\_\_\_\_\_\_\_\_\_\_\_\_\_\_\_\_\_\_\_\_\_\_\_\_\_\_\_\_\_\_\_\_\_\_\_\_\_\_\_\_\_\_\_\_\_\_\_\_\_\_\_\_\_\_\_\_\_\_

\_\_\_\_\_\_\_\_\_\_\_\_\_\_\_\_\_\_\_\_\_\_\_\_\_\_\_\_\_\_\_\_\_\_\_\_\_\_\_\_\_\_\_\_\_\_\_\_\_\_\_\_\_\_\_\_\_\_\_\_\_\_\_\_

b) Qual é o \textit{slogan} presente nesse anúncio?

Doe um brinquedo, ganhe um sorriso!

c) Onde serão coletados os brinquedos e as roupas?

Na Estação Cultural e na Secretaria de Educação.

\_\_\_\_\_\_\_\_\_\_\_\_\_\_\_\_\_\_\_\_\_\_\_\_\_\_\_\_\_\_\_\_\_\_\_\_\_\_\_\_\_\_\_\_\_\_\_\_\_\_\_\_\_\_\_\_\_\_\_\_\_\_\_\_

\_\_\_\_\_\_\_\_\_\_\_\_\_\_\_\_\_\_\_\_\_\_\_\_\_\_\_\_\_\_\_\_\_\_\_\_\_\_\_\_\_\_\_\_\_\_\_\_\_\_\_\_\_\_\_\_\_\_\_\_\_\_\_\_

d) Em sua opinião, por que doar brinquedos garante sorrisos?

\coment{Resposta pessoal. Aproveite o momento e explique aos alunos que doar é
um ato de amor ao próximo e evita o acúmulo desnecessário. Além disso, no caso
específico da campanha, as crianças que recebem brinquedos ficam alegres.}

\_\_\_\_\_\_\_\_\_\_\_\_\_\_\_\_\_\_\_\_\_\_\_\_\_\_\_\_\_\_\_\_\_\_\_\_\_\_\_\_\_\_\_\_\_\_\_\_\_\_\_\_\_\_\_\_\_\_\_\_\_\_\_\_

\_\_\_\_\_\_\_\_\_\_\_\_\_\_\_\_\_\_\_\_\_\_\_\_\_\_\_\_\_\_\_\_\_\_\_\_\_\_\_\_\_\_\_\_\_\_\_\_\_\_\_\_\_\_\_\_\_\_\_\_\_\_\_\_

\subsection{Treino}\label{treino-3}

\subsubsection{1.}\label{section-44}

Leia o cartaz de conscientização.

https://www.bombinhas.sc.gov.br/noticias/ver/2014/11/campanha-de-vacinacao-contra-sarampo-e-paralisia-infantil-inicia-neste-sabado

\includegraphics[width=3.95833in,height=3.95833in]{media/image12.png}

\fonte{Disponível em:
\emph{https://www.bombinhas.sc.gov.br/noticias/ver/2014/11/campanha-de-vacinacao-contra-sarampo-e-paralisia-infantil-inicia-neste-sabado}.
Acesso em: 18 fev. 2023.}

Compreende-se que o cartaz é parte de uma campanha sobre a importância

(A) de desenhar com liberdade e criatividade.

(B) de ter a própria ``turma'', ou seja, bons amigos.

(C) da amizade e das brincadeiras com os amigos.

(D) da vacinação contra o sarampo e a paralisia infantil.

SAEB: Analisar o uso de recursos de persuasão em textos verbais e/ou multimodais.
BNCC: EF03LP19 -- Identificar e discutir o propósito do uso de recursos de
persuasão (cores, imagens, escolha de palavras, jogo de palavras,
tamanho de letras) em textos publicitários e de propaganda, como
elementos de convencimento.

(A) Incorreta. Embora haja a presença de desenhos no anúncio, esse não é
o objetivo da campanha, visto que a conscientização é referente à
vacinação.

(B) Incorreta. A ``turma da vacinação'' é referente à conscientização
sobre a necessidade de se vacinar contra doenças específicas, não em
relação à importância de ter amizades.

(C) Incorreta. A finalidade da campanha é promover a conscientização sobre a
importância da vacinação contra o sarampo e a paralisia infantil.

(D) Correta. No cartaz, compreende-se que os elementos verbais e não
verbais apontam para a importância da vacinação contra o sarampo e a
paralisia infantil, dado o anúncio ``Vacinação contra o sarampo e a
paralisia infantil'', bem como o slogan ``Vem pra turma da vacinação''.

\subsubsection{2. }\label{section-45}

Observe o cartaz de uma campanha.

https://i.pinimg.com/originals/73/ce/78/73ce78802f6062e3f50dfe73c65931b4.jpg

\includegraphics[width=2.84375in,height=3.36642in]{media/image13.jpeg}

Uma análise correta sobre elemento do cartaz é que

(A) a presença da água reforça a ideia de que é preciso aguar as plantas com regularidade.

(B) o sinal de corte sobre determinadas imagens aponta para aquilo que deve ser evitado.

(C) a representação do mosquito não está relacionada ao texto principal do anúncio.

(D) a imagem das larvas refere-se à necessidade de manter vivos os mosquitos na natureza.

SAEB: Analisar os efeitos de sentido de recursos multissemiótico em textos que circulam em diferentes suportes.
BNCC: EF03LP19 -- Identificar e discutir o propósito do uso de recursos de
persuasão (cores, imagens, escolha de palavras, jogo de palavras,
tamanho de letras) em textos publicitários e de propaganda, como
elementos de convencimento.

(A) Incorreta. A presença da água no cartaz reforça a ideia de que se deve ter cuidado com a água parada, que se torna um criadouro de mosquitos da dengue.

(B) Correta. O xis vermelho que ``corta'' determinados elementos indica que as situações relacionadas a eles devem ser evitadas na tentativa de se combater o mosquito da dengue.

(C) Incorreta. A representação do mosquito relaciona-se diretamente ao texto ``Combata o mosquito''.

(D) Incorreta. A imagem das larvas na água relembra que se devem evitar os criadouros do mosquito.

\subsubsection{3. }\label{section-46}

Analise este cartaz.

https://vitalvereador.wordpress.com/2012/01/29/campanha-vamos-tirar-o-planeta-do-sufoco/

\includegraphics[width=6.14514in,height=2.43088in]{media/image14.jpeg}

\fonte{Disponível em:
\emph{https://vitalvereador.wordpress.com/2012/01/29/campanha-vamos-tirar-o-planeta-do-sufoco/}.
Acesso em: 18 fev. 2023.}

Com esse cartaz, o objetivo pretendido

(A) não pode ser alcançado.

(B) pode ser alcançado, mas não está claro o que o motiva.

(C) está alcançado na medida em que fica clara a ideia defendida.

(D) seria alcançado se o texto fosse dirigido diretamenteo ao leitor.

SAEB: Julgar a eficácia de argumentos em textos.
BNCC: EF03LP19 -- Identificar e discutir o propósito do uso de recursos de
persuasão (cores, imagens, escolha de palavras, jogo de palavras,
tamanho de letras) em textos publicitários e de propaganda, como
elementos de convencimento.

(A) Incorreta. O cartaz mostra-se eficiente.

(B) Incorreta. Está, sim, claro o que motiva o cartaz.

(C) Correta. A ideia de defender o meio ambiente com a diminuição do consumo de plástico está bem clara no cartaz.

(D) Incorreta. Já há texto do cartaz diretamente dirigido ao leitor.

\section{5. O texto em versos}\label{muxf3dulo-5}

\coment{Neste módulo, espera-se que os alunos
localizem informações explícitas no poema; identifiquem a função social
do texto, reconhecendo a sua função, onde circula, quem o produziu e a quem se destina;
reconheçam características do poema, como estrofes e versos, rimas nos
versos; notem a relação entre textos; reconheçam o sentido figurado de
palavras e expressões utilizadas no poema; infiram o sentido de
palavras, com base no contexto de trecho de texto.}

\colorsec{Habilidades do SAEB}

\begin{itemize}
  \item Reconhecer diferentes modos de organização composicional de
textos em versos.
  \item Analisar a construção de sentidos de textos em versos com base
em seus elementos constitutivos.
\end{itemize}

\colorsec{Habilidades da BNCC}

\begin{itemize}
EF35LP16, EF03LP26, EF35LP27, EF35LP31.
\end{itemize}

\subsection{Conteúdo}\label{conteuxfado-4}

\href{https://www.istockphoto.com/br/vetor/escritora-cria-poesia-com-pena-no-papel-gm1346126990-423980947?utm_source=pixabay\&utm_medium=affiliate\&utm_campaign=SRP_vector_sponsored\&utm_content=https\%3A\%2F\%2Fpixabay.com\%2Fpt\%2Fvectors\%2Fsearch\%2Fpoema\%2F\%3Fmanual_search\%3D1\&utm_term=poema}{https://www.istockphoto.com/br/vetor/escritora-cria-poesia-com-pena-no-papel-gm1346126990-423980947?utm\_source=pixabay\&utm\_medium=affiliate\&utm\_campaign=SRP\_vector\_sponsored\&utm\_content=https\%3A\%2F\%2Fpixabay.com\%2Fpt\%2Fvectors\%2Fsearch\%2Fpoema\%2F\%3Fmanual\_search\%3D1\&utm\_term=poema\includegraphics[width=4.80139in,height=3.19792in]{media/image15.jpeg}}

O \textbf{poema} é o texto organizado em
versos. Sua principal característica é explorar a linguagem de modo
diferenciado, isto é, utiliza a sonoridade das palavras (apresenta rima,
repetição de sons e ritmo) e utiliza termos e expressões no sentido
figurado -- com significado diferente daquele que normalmente apresenta.

A cada uma das linhas de um poema damos o nome de \textbf{verso}. Alguns poemas
podem ser organizados em conjuntos de versos separados por uma linha em
branco. Esses conjuntos de versos denominam-se \textbf{estrofes}. Nas
estrofes, pode ou não haver rimas entre os versos que as compõem.

Em um texto poético, é necessário fazer a distinção entre o poeta, ou
seja, o escritor, do \textbf{eu que fala no texto}. Em texto narrativos, por
exemplo, quem dá a voz denomina-se narrador, mas, nos poemas, a voz
é a do \textbf{eu lírico}, que pode expressar
sentimentos, ideias e emoções que permitem envolver o leitor e
despertar nele as mais diferentes sensações.

Além disso, a escolha
cuidadosa das palavras e a forma como são organizadas no texto conferem
beleza ao poema e também são responsáveis por atrair o interesse dos
leitores.

O poema pode apresentar as mais variadas estruturas. A mais tradicional tem
estrutura na vertical, isto é, um verso embaixo do outro. Os versos
poderão ser agrupados em estrofes, e as estrofes são, geralmente,
separadas por uma linha em branco.

É o poeta quem escolhe o título (se houver) e o tamanho do poema. É bom lembrar que
as escolhas realizadas pelo
poeta levam em consideração que, normalmente, o texto poético procura
sensibilizar o leitor, provocando sentimentos e emoções no momento da
leitura.

Na elaboração do texto, o poeta pode utilizar muitos recursos poéticos,
como rima, recursos visuais e sonoros, imagens poéticas ou
onomatopeias, por exemplo.

\subsection{Atividades}\label{atividades-4}

Leia o poema e resolva as atividades de 1 a 7.

Inserir imagem do relógio ao lado do poema.
https://www.pexels.com/pt-br/foto/relogio-analogico-preto-e-amarelo-3283142/

\includegraphics[width=4.46875in,height=2.93485in]{media/image16.jpeg}

\begin{quote}
\textbf{O relógio}

Preso à parede, sozinho 

\coment,{lá
gente
eternamente
está
lar
verdadeiras
ligeiras
demorar
enfim
chora
demora
fim}

Alvo dos olhos de toda 

\coment,{lá
gente
eternamente
está
lar
verdadeiras
ligeiras
demorar
enfim
chora
demora
fim}

Minutos e horas, 

\coment,{lá
gente
eternamente
está
lar
verdadeiras
ligeiras
demorar
enfim
chora
demora
fim}

Alma do tempo, batendo 

\coment.{lá
gente
eternamente
está
lar
verdadeiras
ligeiras
demorar
enfim
chora
demora
fim}


Na intimidade de todo 

\coment{lá
gente
eternamente
está
lar
verdadeiras
ligeiras
demorar
enfim
chora
demora
fim}

Se as alegrias são 

\coment,{lá
gente
eternamente
está
lar
verdadeiras
ligeiras
demorar
enfim
chora
demora
fim}

As horas correm, voam 

\coment{lá
gente
eternamente
está
lar
verdadeiras
ligeiras
demorar
enfim
chora
demora
fim}

Para a ventura não 

\coment.{lá
gente
eternamente
está
lar
verdadeiras
ligeiras
demorar
enfim
chora
demora
fim}


Gelam os risos, e quando 

\coment{lá
gente
eternamente
está
lar
verdadeiras
ligeiras
demorar
enfim
chora
demora
fim}

Dor ou tristeza nos olhos 

\coment,{lá
gente
eternamente
está
lar
verdadeiras
ligeiras
demorar
enfim
chora
demora
fim}

Calmo, o relógio para e 

\coment,{lá
gente
eternamente
está
lar
verdadeiras
ligeiras
demorar
enfim
chora
demora
fim}

A hora parece que não tem 

\coment.{lá
gente
eternamente
está
lar
verdadeiras
ligeiras
demorar
enfim
chora
demora
fim}

\fonte{Francisca Julia e Julio da Silva. \textbf{Alma infantil:} versos para
uso das escolas. São Paulo/Rio de Janeiro: Editora Livraria Magalhães,
1912. p. 25-26.}
\end{quote}

\textbf{Vocabulário}
Ligeiras: rápidas, velozes.
Ventura: felicidade.

\num{1} Qual é ó título do poema?

O título do poema é ``O relógio''.

\_\_\_\_\_\_\_\_\_\_\_\_\_\_\_\_\_\_\_\_\_\_\_\_\_\_\_\_\_\_\_\_\_\_\_\_\_\_\_\_\_\_\_\_\_\_\_\_\_\_\_\_\_\_\_\_\_\_\_\_\_\_\_\_

\_\_\_\_\_\_\_\_\_\_\_\_\_\_\_\_\_\_\_\_\_\_\_\_\_\_\_\_\_\_\_\_\_\_\_\_\_\_\_\_\_\_\_\_\_\_\_\_\_\_\_\_\_\_\_\_\_\_\_\_\_\_\_\_

\num{2} Quem escreveu o poema?

Francisca Julia e Julio da Silva.

\_\_\_\_\_\_\_\_\_\_\_\_\_\_\_\_\_\_\_\_\_\_\_\_\_\_\_\_\_\_\_\_\_\_\_\_\_\_\_\_\_\_\_\_\_\_\_\_\_\_\_\_\_\_\_\_\_\_\_\_\_\_\_\_

\_\_\_\_\_\_\_\_\_\_\_\_\_\_\_\_\_\_\_\_\_\_\_\_\_\_\_\_\_\_\_\_\_\_\_\_\_\_\_\_\_\_\_\_\_\_\_\_\_\_\_\_\_\_\_\_\_\_\_\_\_\_\_\_

\num{3} Em que livro esse poema foi publicado?

O livro em que o poema foi publicado intitula-se \textit{Alma infantil}.

\_\_\_\_\_\_\_\_\_\_\_\_\_\_\_\_\_\_\_\_\_\_\_\_\_\_\_\_\_\_\_\_\_\_\_\_\_\_\_\_\_\_\_\_\_\_\_\_\_\_\_\_\_\_\_\_\_\_\_\_\_\_\_\_

\_\_\_\_\_\_\_\_\_\_\_\_\_\_\_\_\_\_\_\_\_\_\_\_\_\_\_\_\_\_\_\_\_\_\_\_\_\_\_\_\_\_\_\_\_\_\_\_\_\_\_\_\_\_\_\_\_\_\_\_\_\_\_\_

\num{4} De acordo com o poema, se as alegrias são verdadeiras, as horas passam

\begin{escolha}
  \item lentamente.
  \item calmamente.
  \item rapidamente.
  \item vagarosamente.
\end{escolha}

\coment{De acordo com o poema, se as alegrias são verdadeiras, as horas passam rapidamente.}

\subsubsection{5}\label{section-48}

Leia este verbete de dicionário.

\begin{quote}
\textbf{Gelar} (ge.lar)
\textit{v.}

\textbf{1.} Esfriar(-se) muito.
Gelou o suco antes de servir.
Estava sem meias, seus pés gelaram.

\textbf{2.} Adquirir dura consistência pela ação do frio; congelar(-se).
O inverno gelou o lago.
As vinhas gelaram por causa do frio.]

\textbf{3.} Causar forte sensação de frio; resfriar [td. : O vento frio gelou -lhe o nariz.] [int. : Os quartos do hotel não tinham aquecedores, e os hóspedes gelavam.]

\textbf{4.} Causar ou sentir muito medo; apavorar(-se).
A terrível cena gelou a mulher.
Diante do assaltante, o homem gelou de medo.

\textbf{5.} Tornar (alguém) abatido, desanimado.
A censura do diretor gelou o aluno.
Gelou quando a moça desmanchou o namoro.

\textbf{6.} Fazer perder ou perder o calor humano ou o ardor dos sentimentos.
As decepções da vida gelam o entusiasmo.
Sentindo-se injustiçado, gelou toda a família.
Diante do desprezo do marido, seu coração gelou.

\textbf{7.} Forçar a interrupção ou a suspensão de; interromper(-se); suspender(-se).
A falta de verbas gelou as bolsas de pós-graduação.
Com o mau tempo, o passeio gelou.]

\fonte Aulete digital. Disponível em: \emph{https://aulete.com.br/gelar}. Acesso em: 19 fev. 2023.
\end{quote}

Agora, releia uma estrofe do poema.

\begin{quote}
Gelam os risos, e quando enfim

Dor ou tristeza nos olhos chora,

Calmo, o relógio para e demora,

A hora parece que não tem fim.
\end{quote}

Em sua opinião, a palavra ``gelam'' tem o mesmo sentido que qual item do verbete? Justifique sua resposta.

\coment{O verbo ``gelar'' foi empregado com o sentido expresso no item 7 do verbete.}

\_\_\_\_\_\_\_\_\_\_\_\_\_\_\_\_\_\_\_\_\_\_\_\_\_\_\_\_\_\_\_\_\_\_\_\_\_\_\_\_\_\_\_\_\_\_\_\_\_\_\_\_\_\_\_\_\_\_\_\_\_\_\_\_

\_\_\_\_\_\_\_\_\_\_\_\_\_\_\_\_\_\_\_\_\_\_\_\_\_\_\_\_\_\_\_\_\_\_\_\_\_\_\_\_\_\_\_\_\_\_\_\_\_\_\_\_\_\_\_\_\_\_\_\_\_\_\_\_

\_\_\_\_\_\_\_\_\_\_\_\_\_\_\_\_\_\_\_\_\_\_\_\_\_\_\_\_\_\_\_\_\_\_\_\_\_\_\_\_\_\_\_\_\_\_\_\_\_\_\_\_\_\_\_\_\_\_\_\_\_\_\_\_

\_\_\_\_\_\_\_\_\_\_\_\_\_\_\_\_\_\_\_\_\_\_\_\_\_\_\_\_\_\_\_\_\_\_\_\_\_\_\_\_\_\_\_\_\_\_\_\_\_\_\_\_\_\_\_\_\_\_\_\_\_\_\_\_

\subsubsection{6}\label{section-49}

Como é a divisão do poema em versos e estrofes?

\coment{O poema tem um total de doze verssos, divididos igualmente em três estrofes de quatro versos.}

\_\_\_\_\_\_\_\_\_\_\_\_\_\_\_\_\_\_\_\_\_\_\_\_\_\_\_\_\_\_\_\_\_\_\_\_\_\_\_\_\_\_\_\_\_\_\_\_\_\_\_\_\_\_\_\_\_\_\_\_\_\_\_\_

\_\_\_\_\_\_\_\_\_\_\_\_\_\_\_\_\_\_\_\_\_\_\_\_\_\_\_\_\_\_\_\_\_\_\_\_\_\_\_\_\_\_\_\_\_\_\_\_\_\_\_\_\_\_\_\_\_\_\_\_\_\_\_\_

\_\_\_\_\_\_\_\_\_\_\_\_\_\_\_\_\_\_\_\_\_\_\_\_\_\_\_\_\_\_\_\_\_\_\_\_\_\_\_\_\_\_\_\_\_\_\_\_\_\_\_\_\_\_\_\_\_\_\_\_\_\_\_\_
\end{quote}

\subsubsection{7}\label{section-50}

Como se dão as rimas no poema?

\coment{Os pares de palavras que rimam entre si são estes: lá/está; gente/eternamente;
lar/demorar; verdadeiras/ligeiras; enfim/fim; chora/demora.}

Leia o poema e resolva as atividades de 8 a 12.

\comnet{Promova uma leitura expressiva desse poema. Desse modo, estará
desenvolvendo a fluência, identificação e apreciação de textos em
versos.}

Inserir imagem ao lado do poema:
https://pixabay.com/pt/vectors/gato-scottish-fold-gatinho-gatinha-7753428/

\includegraphics[width=2.29455in,height=2.45833in]{media/image17.png}

\begin{quote}
\textbf{Uma amiguinha}

É inteligente e graciosa;

Mais limpa, que ela, não há:

Focinhito cor‑de‑rosa,

E chama‑se Resedá.


Muito orgulhosa e faceira,

Não quer saber da cozinha,

E, à sesta, sob a roseira,

Dorme um sono de rainha.


Gosta do sol, ama as flores,

Corre por todo o jardim,

E tem, no dorso, em três cores,

A maciez do cetim.


{[}...{]}


É toda mimos da sorte,

Gatinha de estimação,

Defende‑a, contra o mais forte,

Das patas vivo arranhão.


Mas é boazinha e correta;

Não provoca ásperos tratos;

Somente mostra‑se inquieta,

Se escuta rumor de ratos.
 

Então --- adeus, gentileza! ---

É toda instinto animal,

De um salto, atira‑se à presa...

E é como as outras, tal qual.

\fonte{Zalina Rolim. \textbf{Livro das crianças}. Disponível em:
\emph{https://www.unicamp.br/iel/memoria/Ensaios/LiteraturaInfantil/10Zalina.htm}.
Acesso em: 11 fev. 2023.}

\textbf{Vocabulário}
Focinhito: focinho.
Dorso: costas.
Cetim: tipo de tecido muito macio.
Inquieta: agitada.
Rumor: ruído.

\num{8} O título do texto é ``Uma amiguinha''. Quem é a amiguinha?

A amiguinha é a gatinha Resedá.

\_\_\_\_\_\_\_\_\_\_\_\_\_\_\_\_\_\_\_\_\_\_\_\_\_\_\_\_\_\_\_\_\_\_\_\_\_\_\_\_\_\_\_\_\_\_\_\_\_\_\_\_\_\_\_\_\_\_\_\_\_\_\_\_

\_\_\_\_\_\_\_\_\_\_\_\_\_\_\_\_\_\_\_\_\_\_\_\_\_\_\_\_\_\_\_\_\_\_\_\_\_\_\_\_\_\_\_\_\_\_\_\_\_\_\_\_\_\_\_\_\_\_\_\_\_\_\_\_

\num{9} Quantas estrofes e quantos versos há nesse trecho do poema?

Nesse trecho do poema, aparecem seis estrofes, e cada uma tem quatro versos.

\_\_\_\_\_\_\_\_\_\_\_\_\_\_\_\_\_\_\_\_\_\_\_\_\_\_\_\_\_\_\_\_\_\_\_\_\_\_\_\_\_\_\_\_\_\_\_\_\_\_\_\_\_\_\_\_\_\_\_\_\_\_\_\_

\_\_\_\_\_\_\_\_\_\_\_\_\_\_\_\_\_\_\_\_\_\_\_\_\_\_\_\_\_\_\_\_\_\_\_\_\_\_\_\_\_\_\_\_\_\_\_\_\_\_\_\_\_\_\_\_\_\_\_\_\_\_\_\_

\num{10} Transcreva uma rima presente no poema.

Sugestão de resposta: rimam entre si, no poema, ``graciosa'' e ``cor-de-rosa''.

\_\_\_\_\_\_\_\_\_\_\_\_\_\_\_\_\_\_\_\_\_\_\_\_\_\_\_\_\_\_\_\_\_\_\_\_\_\_\_\_\_\_\_\_\_\_\_\_\_\_\_\_\_\_\_\_\_\_\_\_\_\_\_\_

\_\_\_\_\_\_\_\_\_\_\_\_\_\_\_\_\_\_\_\_\_\_\_\_\_\_\_\_\_\_\_\_\_\_\_\_\_\_\_\_\_\_\_\_\_\_\_\_\_\_\_\_\_\_\_\_\_\_\_\_\_\_\_\_

\num{11} Marque com um X as palavras que rimam com \textbf{graciosa} e
\textbf{cor-de-rosa}.

( x ) misteriosa ( ) mimada ( x ) sedosa ( ) quietinha

\num{12} Normalmente, representam-se os sons das rimas por letras maiúculas. Observe:

\begin{quote}
É inteligente e graci\textbf{osa}; \textbf{A}

Mais limpa, que ela, não h\textbf{á}: \textbf{B}

Focinhito cor‑de‑r\textbf{osa}, \textbf{A}

E chama‑se Resed\textbf{á}. \textbf{B}

Isso significa que, nessa estrofe, o primeiro verso rima com o terceiro, e o segundo rima com o quarto. A letra \textbf{A}, nesse caso, representa o som repetido no final do primeiro e do terceiro verso: \textbf{osa}. Já a letra \textbf{B} represente o som repetido no final do segundo e do quarto verso: \textbf{á}.

Agora, releia outra estrofe do poema.

\begin{quote}[]{@{}l@{}}
Mas é boazinha e correta;

Não provoca ásperos tratos;

Somente mostra-se inquieta,

Se escuta rumor de ratos.\strut
\end{quote}

Usando as letras maiúsculas C e D, represente o esquema de rimas dessa estrofe.

\coment{CDCD.}

\begin{quote}
\_\_\_\_\_\_\_\_\_\_\_\_\_\_\_\_\_\_\_\_\_\_\_\_\_\_\_\_\_\_\_\_\_\_\_\_\_\_\_\_\_\_\_\_\_\_\_\_\_\_\_\_\_\_\_\_\_\_\_\_\_\_\_\_

\_\_\_\_\_\_\_\_\_\_\_\_\_\_\_\_\_\_\_\_\_\_\_\_\_\_\_\_\_\_\_\_\_\_\_\_\_\_\_\_\_\_\_\_\_\_\_\_\_\_\_\_\_\_\_\_\_\_\_\_\_\_\_\_
\end{quote}

\subsection{Treino}\label{treino-4}

\num{1} Leia o poema.

\begin{quote}
\textbf{Amigos por toda a parte}

Manhã de primavera:

Nos ares voa um cântico festivo ---

Leve rumor de voz, barulho vivo,

Ao sol, que reverbera.


{[}...{]}


Por toda a parte flores!

Áureas, roxas, azuis, brancas, vermelhas...

E, em zumbidora orquestra, andam abelhas

Correndo os arredores.


Gorjeiam passarinhos...

E Lídia vai seguindo alegremente,

Num bem‑estar de espírito contente,

Ao longo dos caminhos.


Orla, um ribeiro, a mata,

Alvo, entre margens de veludo eterno;

O gaio azul do céu de um brilho terno

Nas águas se retrata.


Serena paz bendita,

Como um perfume, estende‑se por tudo...

E, olhos abertos, cauteloso e mudo,

Fiel a cauda agita.


E os olhos tão suaves

De Lídia, e os doces lábios cor de rosa,

Riem‑se à luz do sol, fina e radiosa,

E ao cântico das aves.

\fonte{Zalina Rolim. \textbf{Livro das crianças}. Disponível em:
\emph{https://www.unicamp.br/iel/memoria/Ensaios/LiteraturaInfantil/10Zalina.htm}.
Acesso em: 11 fev. 2023.}
\end{quote}

Nesse trecho do poema, há

(A) versos todos do mesmo tamanho.

(B) estrofes com número diferentes de versos.

(C) quatro versos em cada estrofe.

(D) estrofes de verso único.

SAEB: Reconhecer diferentes modos de organização composicional de textos em versos.
BNCC: EF35LP27 -- Ler e compreender, com certa autonomia, textos em
versos, explorando rimas, sons e jogos de palavras, imagens poéticas
(sentidos figurados) e recursos visuais e sonoros.

(A) Incorreta. Os versos têm tamanhos diferentes.

(B) Incorreta. Todas as estrofes têm o mesmo número de versos.

(C) Correta. Cada estrofe tem quatro versos.

(D) Nenhuma estrofe tem apenas um verso.

\num{2} Leia um trecho do poma ``Infância'', de Casimiro de Abreu.

\begin{quote}
\textbf{Infância}

{[}...{]}

Ó anjo da loura trança,

És criança,

A vida começa a rir.

--- Vive e folga descansada,

Descuidada

Das tristezas do porvir.

\fonte. Casimiro de Abreu. \textbf{Infância}. Disponível em:
\emph{www.dominiopublico.gov.br/download/texto/wk000394.pdf}. Acesso em: 20
fev. 2023.
\end{quote}

\textbf{Vocabulário}
Porvir: futuro.

Versos que rimam entre si, nesse trecho do poema, são

(A) ``Ó anjo da loura trança'' e ``És criança''.

(B) ``A vida começa a rir'' e ``Vive e folga descansada''.

(C) ``Descuidada'' e ``Das tristezas do porvir''.

(D) ``Ó anjo da loura trança'' e ``Vive e folga descansada''.

SAEB: Reconhecer diferentes modos de organização composicional de textos em versos.
BNCC: EF35LP27 -- Ler e compreender, com certa autonomia, textos em
versos, explorando rimas, sons e jogos de palavras, imagens poéticas
(sentidos figurados) e recursos visuais e sonoros.

(A) Correta. As palavras finais ``trança'' e ``criança'' rimam entre si.

(B) Incorreta. As palavras finais ``rir'' e ``descansada'' não rimae entre si.

(C) Incorreta. A palavra ``Descuidada'' e a palavra final ``porvir'' não rimam entre si.

(D) Incorreta. As palavras finais ``trança'' e ``descansada'' não rimam entre si.

\num{3} Leia o poema.

\begin{quote}
\textbf{O meu retrato}

Aquele retrato lá:

O corpo, as pernas, o braço,

Sou eu mesmo traço a traço,

Tão parecido ele está.


Acham bonito? Talvez...

O nariz é um pouco chato...

Mas, que importa? É o meu retrato;

Foi o vovô quem o fez.

{[}...{]}

Francisca Julia e Julio da Silva. \textbf{Alma infantil:} versos para uso das escolas.
Disponível em: \emph{https://digital.bbm.usp.br/bitstream/bbm/4556/1/033579_COMPLETO.pdf}.
Acesso em: 17 abr. 2023.

A inversão das palavras que ocorre no último verso da primeira estrofe

(A) dificulta a compreensão do poema, pois cria uma incoerência.

(B) enfatiza o sentido do verbo e cria uma rima com o primeiro verso.

(C) gera um diferencial no poema com algo que não é típico de textos como esse.

(D) é do mesmo tipo que a inversão que ocorre no último verso da segunda estrofe.

SAEB: Analisar a construção de sentidos de textos em versos com base em seus elementos constitutivos.
BNCC: EF35LP31 -- Identificar, em textos versificados, efeitos de sentido
decorrentes do uso de recursos rítmicos e sonoros e de metáforas.

(A) Incorreta. Não há dificuldade em se compreender a inversão operada no verso.

(B) Correta. A inversão, que coloca ``está'' no final do verso, enfatiza o sentido do verbo para o verso e cria uma rima de final de verso com o primeiro, em que ocorre a palavra ``lá''.

(C) Incorreta. Em poemas, é típico haver inversões de palavras com objetivos rítmicos, estéticos ou semânticos.

(D) Incorreta. Não há inversão no último verso da segunda estrofe.

\section{6. Discurso direto e discurso indireto}\label{muxf3dulo-6}

\colorsec{Habilidade do SAEB}

\begin{itemize}
  \item - Identificar as variedades linguísticas em textos.
\end{itemize}

\colorsec{Habilidades da BNCC}

\begin{itemize}
  \item EF35LP22, EF35LP30.
\end{itemize}

\subsection{Conteúdo}\label{conteuxfado-5}

https://pixabay.com/pt/vectors/falar-pessoas-conversa\%c3\%a7\%c3\%a3o-amigo-7647863/

\includegraphics[width=3.01042in,height=3.01042in]{media/image18.png}

O \textbf{discurso direto} é a transcrição fiel da fala de um personagem na
narração, sem intervenção do narrador. Esse tipo de
discurso normalmente é precedido pelo travessão (sinal de pontuação que
indica quando se inicia a fala de um personagem, quando ocorre mudança
de interlocutores e quando existe mudança para o narrador mediante um
verbo de elocução). Além disso, discurso direto pode aparecer entre aspas.

Normalmente, o discurso direto é iniciado por verbos de elocução, tais como: falar,
dizer, comentar, perguntar, responder, observar, murmurar, exclamar,
gritar, aconselhar. Os verbos de elocução podem ser seguidos por dois-pontos.

O discurso direto também pode ser usado para dar voz aos personagens,
propiciando ao leitor notar variadas reações diante dos acontecimentos
que estão sendo narrados. Assim sendo, o discurso direto torna a
narração mais dinâmica e atraente para o leitor.

Na escrita, as marcas do discurso direto podem ser: dois-pontos (\textbf{:}),
aspas (\textbf{``''}) e travessão (\textbf{---}). Veja um exemplo a seguir.

\begin{quote}
João disse:
--- A Brisa é uma gatinha muito fofa!
\end{quote}

Além disso, é no discurso direto que o narrador mais tem chance de mostrar
possíveis variedades da língua.

Já o \textbf{discurso indireto} consiste em uma reprodução do conteúdo das falas dos
personagens, em vez de uma transcrição exata delas, com as palavras do
narrador. Ele atua como intermediário, muitas vezes incluindo também
emoções, reações, sentimentos ou marcas de personalidade do personagem.
Os verbos de elocução também aparecem, mas a estrutura é diferente.
Veja um exemplo a seguir.

\begin{quote}
João disse que a Brisa é uma gatinha muito fofa.
\end{quote}


\subsection{Atividades}\label{atividades-5}

Leia o conto a seguir, de Monteiro Lobato, e resolva as atividades de 1 a 8.

https://pixabay.com/pt/vectors/tartaruga-animal-desenho-animado-151431/


\includegraphics[width=2.90902in,height=2.95828in]{media/image19.png}

\begin{quote}
\textbf{O cágado na festa do céu}

Certa vez houve uma grande festa no céu, para a qual foram convidados os
bichos da floresta. Todos se encaminharam para lá, e o cágado também, mas
ele era vagaroso demais, de modo que andava, andava e não chegava
nunca.

A festa era só de três dias, e o cágado nada de chegar. Desanimado, pediu
a uma garça que o conduzisse carregado nas costas. A garça respondeu: ``Pois não'', e
o cágado montou.

A garça foi subindo, subindo, subindo; de vez em quando perguntava ao
cágado se estava vendo a terra.

--- Estou, sim, mas lá longe.

A garça subia mais e mais.

--- E agora?

--- Agora já não vejo o menor sinalzinho da terra.

A garça, então, que era uma malvada, fez uma reviravolta no ar,
derrubando o cágado. Coitado! Começou a cair com velocidade cada vez
maior. Enquanto caía, murmurava:

\emph{Se eu desta escapar,}
\emph{léu, léu, léu,}
\emph{se eu desta escapar,}
\emph{nunca mais ao céu me deixarei levar.}

Nisto, avistou lá embaixo a terra. Gritou:

--- Distanciem-se, pedras e paus, senão eu os esmagarei! As pedras e os paus
se afastaram e o cágado caiu. Mesmo assim arrebentou-se todo, em cem
pedaços.

Quem via tinha dó do coitado. Afinal de contas, aquela
desgraça tinha acontecido só porque o cágado teimou em comparecer à festa do
céu. De pena, um ser mágico da floresta junto os pedacinhos.
É por isso que o cágado tem a casca feita de pedacinhos emendados uns
nos outros.

\fonte Monteiro Lobato. O cágado na festa do céu. Adaptado.
\end{quote}

\coment{Se os alunos não souberem, explique que o cágado é um animal parecido
com uma tartaruga ou com um jabuti.}

\num{1} Quais são os personagens da história? Garça, cágado e um ser mágico da floresta.

\_\_\_\_\_\_\_\_\_\_\_\_\_\_\_\_\_\_\_\_\_\_\_\_\_\_\_\_\_\_\_\_\_\_\_\_\_\_\_\_\_\_\_\_\_\_\_\_\_\_\_\_\_\_\_\_\_\_\_\_\_\_\_\_

\_\_\_\_\_\_\_\_\_\_\_\_\_\_\_\_\_\_\_\_\_\_\_\_\_\_\_\_\_\_\_\_\_\_\_\_\_\_\_\_\_\_\_\_\_\_\_\_\_\_\_\_\_\_\_\_\_\_\_\_\_\_\_\_

\num{2} Qual é o local onde se passa a história? Na terra e no céu.

\_\_\_\_\_\_\_\_\_\_\_\_\_\_\_\_\_\_\_\_\_\_\_\_\_\_\_\_\_\_\_\_\_\_\_\_\_\_\_\_\_\_\_\_\_\_\_\_\_\_\_\_\_\_\_\_\_\_\_\_\_\_\_\_

\_\_\_\_\_\_\_\_\_\_\_\_\_\_\_\_\_\_\_\_\_\_\_\_\_\_\_\_\_\_\_\_\_\_\_\_\_\_\_\_\_\_\_\_\_\_\_\_\_\_\_\_\_\_\_\_\_\_\_\_\_\_\_\_

\num{3} Por que o cágado não chegava nunca à festa? Por que era vagaroso
demais.

\protect\hypertarget{_Hlk127856302}{}{}\_\_\_\_\_\_\_\_\_\_\_\_\_\_\_\_\_\_\_\_\_\_\_\_\_\_\_\_\_\_\_\_\_\_\_\_\_\_\_\_\_\_\_\_\_\_\_\_\_\_\_\_\_\_\_\_\_\_\_\_\_\_\_\_

\_\_\_\_\_\_\_\_\_\_\_\_\_\_\_\_\_\_\_\_\_\_\_\_\_\_\_\_\_\_\_\_\_\_\_\_\_\_\_\_\_\_\_\_\_\_\_\_\_\_\_\_\_\_\_\_\_\_\_\_\_\_\_\_

\num{4} Por que o cágado teve a casca feita de pedacinhos emendados uns nos
outros? Isso aconteceu, porque um ser mágico teve dó do que aconteceu com o cágado e juntou outra
vez os pedaços.

\_\_\_\_\_\_\_\_\_\_\_\_\_\_\_\_\_\_\_\_\_\_\_\_\_\_\_\_\_\_\_\_\_\_\_\_\_\_\_\_\_\_\_\_\_\_\_\_\_\_\_\_\_\_\_\_\_\_\_\_\_\_\_\_

\_\_\_\_\_\_\_\_\_\_\_\_\_\_\_\_\_\_\_\_\_\_\_\_\_\_\_\_\_\_\_\_\_\_\_\_\_\_\_\_\_\_\_\_\_\_\_\_\_\_\_\_\_\_\_\_\_\_\_\_\_\_\_\_

\num{5} Quem conta a história? Assinale a alternativa correta.

( ) Um narrador que participa da história -- narração em primeira pessoa.

( x ) Um narrador que não participa da história -- narraçã em terceira pessoa.

\num{6} No trecho a seguir, quem está participando do diálogo? A garça e o
cágado.

\begin{longtable}[]{@{}l@{}}
\toprule
\begin{minipage}[t]{0.97\columnwidth}\raggedright\strut
A garça foi subindo, subindo, subindo; de vez em quando perguntava ao
cágado se estava vendo a terra.

--- Estou, sim, mas lá longe.

A garça subia mais e mais.

--- E agora?

--- Agora já não vejo o menor sinalzinho da terra.\strut
\end{minipage}\tabularnewline
\bottomrule
\end{longtable}

\protect\hypertarget{_Hlk127806075}{}{}\textbf{\_\_\_\_\_\_\_\_\_\_\_\_\_\_\_\_\_\_\_\_\_\_\_\_\_\_\_\_\_\_\_\_\_\_\_\_\_\_\_\_\_\_\_\_\_\_\_\_\_\_\_\_\_\_\_\_\_\_\_\_\_\_\_\_}

\textbf{\_\_\_\_\_\_\_\_\_\_\_\_\_\_\_\_\_\_\_\_\_\_\_\_\_\_\_\_\_\_\_\_\_\_\_\_\_\_\_\_\_\_\_\_\_\_\_\_\_\_\_\_\_\_\_\_\_\_\_\_\_\_\_\_}

\num{7} Em sua opinião, como seria a história se o diálogo apresentado no trecho selecionado na atividade anterior fosse formado
somente pelas falas dos personagens, sem a intervenção do narrador?
Resposta pessoal. Espera-se que os alunos respondam que o leitor não
saberia de que forma os personagens falaram, uma vez que não existiriam
os comentários.

\textbf{\_\_\_\_\_\_\_\_\_\_\_\_\_\_\_\_\_\_\_\_\_\_\_\_\_\_\_\_\_\_\_\_\_\_\_\_\_\_\_\_\_\_\_\_\_\_\_\_\_\_\_\_\_\_\_\_\_\_\_\_\_\_\_\_}

\textbf{\_\_\_\_\_\_\_\_\_\_\_\_\_\_\_\_\_\_\_\_\_\_\_\_\_\_\_\_\_\_\_\_\_\_\_\_\_\_\_\_\_\_\_\_\_\_\_\_\_\_\_\_\_\_\_\_\_\_\_\_\_\_\_\_}

\textbf{\_\_\_\_\_\_\_\_\_\_\_\_\_\_\_\_\_\_\_\_\_\_\_\_\_\_\_\_\_\_\_\_\_\_\_\_\_\_\_\_\_\_\_\_\_\_\_\_\_\_\_\_\_\_\_\_\_\_\_\_\_\_\_\_}

\textbf{\_\_\_\_\_\_\_\_\_\_\_\_\_\_\_\_\_\_\_\_\_\_\_\_\_\_\_\_\_\_\_\_\_\_\_\_\_\_\_\_\_\_\_\_\_\_\_\_\_\_\_\_\_\_\_\_\_\_\_\_\_\_\_\_}

\num{8} Ainda sobre o trecho selecionado na atividade 6, qual é o efeito expresso pelo discurso direto? O discurso
direto apresenta ao leitor a conversa dos personagens como se estivesse
ocorrendo naquele momento.

\textbf{\_\_\_\_\_\_\_\_\_\_\_\_\_\_\_\_\_\_\_\_\_\_\_\_\_\_\_\_\_\_\_\_\_\_\_\_\_\_\_\_\_\_\_\_\_\_\_\_\_\_\_\_\_\_\_\_\_\_\_\_\_\_\_\_}

\protect\hypertarget{_Hlk127808036}{}{}\textbf{\_\_\_\_\_\_\_\_\_\_\_\_\_\_\_\_\_\_\_\_\_\_\_\_\_\_\_\_\_\_\_\_\_\_\_\_\_\_\_\_\_\_\_\_\_\_\_\_\_\_\_\_\_\_\_\_\_\_\_\_\_\_\_\_}

\textbf{\_\_\_\_\_\_\_\_\_\_\_\_\_\_\_\_\_\_\_\_\_\_\_\_\_\_\_\_\_\_\_\_\_\_\_\_\_\_\_\_\_\_\_\_\_\_\_\_\_\_\_\_\_\_\_\_\_\_\_\_\_\_\_\_}

Leia a piadinha a seguir para resolver as atividades de 9 a 11.

Inserir imagem ao lado do texto:
https://cdn.pixabay.com/photo/2017/07/18/21/09/ice-cream-2517064\_960\_720.png

\includegraphics[width=1.61997in,height=3.13542in]{media/image20.png}

\begin{quote}
\textbf{Sorvete de macaxeira}

O garoto chega na sorveteria e pergunta:

--- Tem sorvete de macaxeira?

O vendedor responde:

--- Não.

No dia seguinte:

--- Tem sorvete de macaxeira?

--- Não.

No outro dia:

--- Tem sorvete de macaxeira?

--- Não.

No outro dia:

--- Tem sorvete de macaxeira?

--- Tem!

--- Eca!

\fonte{Domínio público.}
\end{quote}

\num{9} Qual é a finalidade desse texto? O gênero
textual anedota é humorístico e tem por objetivo provocar o riso no
leitor.


\num{10} Releia o trecho a seguir.

\begin{quote}
O garoto chega na sorveteria e pergunta:

--- Tem sorvete de macaxeira?

O vendedor responde:

--- Não.
\end{quote}

a) Quais são os verbos que introduzem as falas dos personagens? 

\coment{Os verbos são ``perguntar'' (na forma ``pegunta'') e ``responder'' (na forma ``responde''.}

\begin{quote}
\_\_\_\_\_\_\_\_\_\_\_\_\_\_\_\_\_\_\_\_\_\_\_\_\_\_\_\_\_\_\_\_\_\_\_\_\_\_\_\_\_\_\_\_\_\_\_\_\_\_\_\_\_\_\_\_\_\_\_\_\_\_\_\_

\_\_\_\_\_\_\_\_\_\_\_\_\_\_\_\_\_\_\_\_\_\_\_\_\_\_\_\_\_\_\_\_\_\_\_\_\_\_\_\_\_\_\_\_\_\_\_\_\_\_\_\_\_\_\_\_\_\_\_\_\_\_\_\_
\end{quote}

b) Forme dupla com um colega e reescreva o trecho usando o discurso
indireto. Sugestão de resposta: O garoto chega na sorveteria e pergunta
ao vendedor se tem sorvete de macaxeira. O vendedor responde que não.

\begin{quote}
\_\_\_\_\_\_\_\_\_\_\_\_\_\_\_\_\_\_\_\_\_\_\_\_\_\_\_\_\_\_\_\_\_\_\_\_\_\_\_\_\_\_\_\_\_\_\_\_\_\_\_\_\_\_\_\_\_\_\_\_\_\_\_\_

\_\_\_\_\_\_\_\_\_\_\_\_\_\_\_\_\_\_\_\_\_\_\_\_\_\_\_\_\_\_\_\_\_\_\_\_\_\_\_\_\_\_\_\_\_\_\_\_\_\_\_\_\_\_\_\_\_\_\_\_\_\_\_\_

\_\_\_\_\_\_\_\_\_\_\_\_\_\_\_\_\_\_\_\_\_\_\_\_\_\_\_\_\_\_\_\_\_\_\_\_\_\_\_\_\_\_\_\_\_\_\_\_\_\_\_\_\_\_\_\_\_\_\_\_\_\_\_\_

\_\_\_\_\_\_\_\_\_\_\_\_\_\_\_\_\_\_\_\_\_\_\_\_\_\_\_\_\_\_\_\_\_\_\_\_\_\_\_\_\_\_\_\_\_\_\_\_\_\_\_\_\_\_\_\_\_\_\_\_\_\_\_\_
\end{quote}

\num{11} O nome de um vegetal muito usado no Brasil tem variações ao longo do território.
Na piada, aparece o nome \textbf{macaxeira}. Você reconhece essa forma? Se sim, que outros
nomes esses vegetal recebe? Se não, pesquise e escreva como esse vegeral é conhecido por
você?

\coment{Os nomes são mandioca, macaxeira e aipim.}

\subsection{Treino}\label{treino-5}

Leia o texto para responder às questões 1 e 2.

\begin{quote}
\textbf{Cinderela}

Há muito tempo, aconteceu que a esposa de um rico
comerciante adoeceu gravemente e, sentindo seu fim se
aproximar, chamou sua única rebenta e disse:

--- Minha querida, continue piedosa e boa menina que
Deus a protegerá sempre. Lá do céu olharei por você e estarei
sempre a seu lado.

Mal acabou de dizer isso, fechou os olhos e morreu.

A jovem ia todos os dias visitar o túmulo da mãe,
sempre chorando muito.

Veio o inverno, e a neve cobriu o túmulo com seu alvo
manto. Chegou a primavera, e o sol derreteu a neve. Foi então
que o viúvo resolveu se casar outra vez.

A nova esposa trouxe suas duas filhas, ambas louras e
bonitas --- mas só exteriormente. As duas tinham a alma feia
e cruel.

A partir desse momento, dias difíceis começaram para
a pobre enteada.

{[}...{]}

\fonte{Cinderela. Disponível em: \emph{www.dominiopublico.gov.br/download/texto/me000589.pdf}.
Acesso em: 18 abr. 2023. (Adaptado.)}

\num{1} A palavra ``rebenta'', que aparece no texto, é uma variação de

\begin{escolha}
\item ``arrebenta''.
\item ``rebento''.
\item ``filha''.
\item ``broto''.
\end{escolha}

SAEB: Identificar as variedades linguísticas em textos.
BNCC: EF35LP30 -- Diferenciar discurso indireto e discurso direto,
determinando o efeito de sentido de verbos de enunciação e explicando o
uso de variedades linguísticas no discurso direto, quando for o caso.

(A) Incorreta. Apesar da semelhança e da origem comum, o substantivo ``rebenta'' não é uma variação para a forma ``arrebenta'', do verbo ``arrebentar''.

(B) Incorreta. A palavra ``rebenta'' é uma flexão de ``rebento'', mas não uma variação.

(C) Correta. As palavras ``rebenta'' e ``filha'', nesse contexto, são sinônimas.

(D) Incorreta. O sentido de ``broto'' é um possível para a palavra ``rebento'', mas não é o caso do contexto.

\num{2} Na linguagem coloquial, é mais usada a forma ``loiras'', palavra que tem uma variante no texto. Essa variante é

\begin{escolha}
\item mais antiga.
\item mais correta.
\item menos usada na escrita.
\item mais comum no dia a dia.
\end{escolha}

SAEB: Identificar as variedades linguísticas em textos.
BNCC: EF35LP22 -- Perceber diálogos em textos narrativos, observando o
efeito de sentido de verbos de enunciação e, se for o caso, o uso de
variedades linguísticas no discurso direto.

(A) Correta. A forma ``louro'' é anterior ao aparecimento de ``loiro''.

(B) Incorreta. Tanto a forma ``louro'' quanto a forma ``loiro'' são igualmente corretas.

(C) Incorreta. Na escrita, ainda é mais comum a forma ``louro''.

(D) Incorreta. No dia a dia, a forma ``loiro'' é mais comum que ``louro''.

\num{3} Leia a piada.

\begin{quote}
Um compadre chega na casa do outro, que assistia à TV. Bate nas costas do amigo e pergunta:

--- E aí, cumpadi, firme?

O outro compadre responde:

--- Não, cumpadi. É futebor.

\fonte{Domínio público.}
\end{quote}

O humor da piada se deve ao fato de que

\begin{escolha}
\item os dois compadres utilizam variantes diferentes da língua.
\item o segundo compadre não reconheceu a palavra ``firme'', dita pelo outro.
\item o primeiro compadre não sabia que programa era aquele a que o amigo assitia na TV quando ele chegou.
\item houve uma confusão entre ``firme'' e ``filme'', porque as duas palavras são pronunciadas da mesma maneira.
\end{escolha}

SAEB: Identificar as variedades linguísticas em textos.
BNCC: EF35LP30 -- Diferenciar discurso indireto e discurso direto,
determinando o efeito de sentido de verbos de enunciação e explicando o
uso de variedades linguísticas no discurso direto, quando for o caso.

(A) Incorreta. Os dois compadres utilizam as mesmas variantes da língua.

(B) Incorreta. Apesar de ter confundido seu significado, o segundo compadre reconheceu a palavra ``firme''.

(C) Incorreta. A pergunta do primeiro compadre não era sobre o programa na TV, apesar de o outro compadre ter entendido isso.

(D) Correta. O humor da piada deve-se ao fato de que o segundo compadre relacionou a forma ``firme'' à palavra ``filme'', porque as duas palavras, na variedade linguística em questão, são pronunciadas da mesma forma.

\section{7. Adjetivos e advérbios}\label{muxf3dulo-7}

\begin{quote}
Neste módulo, os alunos vão identificar adjetivos em textos e os
substantivos a que se referem, assim como relacionar substantivos e
adjetivos, observando a concordância de gênero e número.
\end{quote}

\colorsec{Habilidades do SAEB}

\begin{itemize}
  \item Analisar os efeitos de sentido decorrentes do uso dos adjetivos.
  \item Analisar os efeitos de sentido decorrentes do uso dos advérbios.
\end{itemize}

\colorsec{Habilidades da BNCC}

\begin{itemize}
  \item EF03LP09, EF03LP23.
\end{itemize}

\subsection{Conteúdo}\label{conteuxfado-6}

\begin{quote}
Provavelmente, você já deve ter notado que tudo o que existe no planeta
tem um nome. Imagine a seguinte situação: você está deitado na cama,
olhando para a televisão, e a televisão provavelmente está na
estante. Você pode agora estar usando uma calça e um par de tênis.
Tudo isso tem um nome específico.

A palavra que nomeia algo é o \textbf{substantivo}, mas, muitas vezes,
para melhorar a expressividade do que dizemos ou escrevemos, é preciso
qualificar os substantivos. A palavra que qualifica um substantivo é o
\textbf{adjetivo}.

A palavra ``cachorro'' é um substantivo que dá nome a um animal
doméstico, e podem ser atribuídas a essa palavra muitas características:
``branco'', ``fofo'', ``bravo'', ``engraçado'', por exemplo.

Outro tipo de palavra que se liga sempre a uma outra é o \textbf{advérbio},
que pode modificar um \textbf{verbo}, um \textbf{adjetivo} ou \textbf{outro
advérbio}. Em uma frase como ``O cachorro correu demais e cansou'', o advérbio
``demais'' modifica a forma verbal ``correu''. Já em uma frase como ``O cachorro
está muito feliz'', o advérbio ``muito'' modifica o adjetivo ``feliz''. Por fim,
em uma frase como ``O cachorro ficou bem mais contente'', o advérbio ``mais''
modifica o adjetivo ``contente'' e é modificado pelo advérbio ``bem''.

\subsection{Atividades}\label{atividades-6}

Leia a carta de um leitor publicada em uma revista especializada em
divulgação científica dirigida a crianças. Depois, resolva as atividades de 1 a 7.

\begin{quote}
\textbf{Lagartos}

Olá, equipe da CHC! Ficamos extremamente impressionados com o artigo ``Por que o lagarto balança tanto a cabeça?'' publicado na edição 244 da CHC. O artigo aborda o comportamento do lagarto-cinzento, que utiliza sua cabeça para se comunicar com outros membros da mesma espécie, informando seu gênero. Esses répteis habitam uma variedade de ambientes naturais, incluindo rochas, águas subterrâneas, solo e árvores. Apreciamos muito o texto e ficaríamos muito felizes se nossa carta fosse selecionada para publicação. Abraços cordiais.

Alunos do 3º ano da Escola Municipal Mestre Carlos Gomes.

\fonte{Fonte de pesquisa: CHC. Lagartos. Fala Aqui! Disponível em:
\emph{https://chc.org.br/artigo/fala-aqui/#:textasciitilde:text=Ficamos\%20muito\%20impressionados\%20com\%20o,se\%20s\%C3\%A3o\%20machos\%20ou\%20f\%C3\%AAmeas}.
Acesso em: 21 fev. 2023.}
\end{quote}

\numn{1} Qual é o assunto dessa carta? Os alunos ficaram muito impressionados
com o texto ``Por que o lagarto balança tanto a cabeça?'', publicado na
CHC 244, que fala sobre o lagarto-cinzento.

\num{2} A quem a carta foi destinada? A carta foi destinada à equipe da revista CHC.

\num{3} Quem é o remetente da carta? São os alunos do 3º ano da Escola Municipal
  Mestre Carlos Gomes.

\num{4} Transcreva o trecho da carta em que os leitores expressam sua opinião.
\coment{Ficamos extremamente impressionados com o artigo ``Por que o lagarto balança tanto a cabeça?'' publicado na edição 244 da CHC.}

\num{5} Do trecho que você transcreveu, que adjetivo e que advérbio são responsáveis por demonstrar a opinião positiva dos leitores da revista?
\coment{O adjetivo é ``impressionados'', modificado pelo advérbio ``extremamente''.}

\num{6} Como esse adjetivo e esse advérbio se relacionam entre si?
\coment{O advérbio modifica o adjetivo.}

\num{7} Sobre esse adjetivo que você identificou na atividade anterior, assinale a alternativa que contém um adjetivo que poderia substituí-lo sem alteração de sentido?

( x ) maravilhados

( ) chateados

( ) decepcionados

Leia um trecho do poema ``Meiguice'', de Adelina Lopes Vieira, para resolver as atividades de 8 a 10.

\begin{quote}
Inserir imagem ao lado do texto:
\url{https://pixabay.com/pt/illustrations/gato-filhote-de-cachorro-7347316/}

\includegraphics[width=3.41667in,height=3.07500in]{media/image23.png}

\textbf{Meiguice}

{[}...{]}

Deram à linda Clarisse

uma gatinha mimosa,

tão branca, tão carinhosa,

tão engraçada, tão mansa

que a encantadora criança

por nome lhe pôs Meiguice.

{[}...{]}

\fonte{Adelina Lopes Vieira. Meiguice. Disponível em:
\emph{http://www.dominiopublico.gov.br/download/texto/wk000075.pdf}.
Acesso em: 21 fev. 2023.}
\end{quote}

\num{8} Que nome Clarisse deu à gatinha que ganhou? Meiguice.

\_\_\_\_\_\_\_\_\_\_\_\_\_\_\_\_\_\_\_\_\_\_\_\_\_\_\_\_\_\_\_\_\_\_\_\_\_\_\_\_\_\_\_\_\_\_\_\_\_\_\_\_\_\_\_\_\_\_\_\_\_\_\_\_

\num{9} Por que Clarisse deu esse nome à gatinha? Clarisse deu esse nome, porque a gatinha era
branca, carinhosa, engraçada e mansa.

\_\_\_\_\_\_\_\_\_\_\_\_\_\_\_\_\_\_\_\_\_\_\_\_\_\_\_\_\_\_\_\_\_\_\_\_\_\_\_\_\_\_\_\_\_\_\_\_\_\_\_\_\_\_\_\_\_\_\_\_\_\_\_\_

\_\_\_\_\_\_\_\_\_\_\_\_\_\_\_\_\_\_\_\_\_\_\_\_\_\_\_\_\_\_\_\_\_\_\_\_\_\_\_\_\_\_\_\_\_\_\_\_\_\_\_\_\_\_\_\_\_\_\_\_\_\_\_\_

\_\_\_\_\_\_\_\_\_\_\_\_\_\_\_\_\_\_\_\_\_\_\_\_\_\_\_\_\_\_\_\_\_\_\_\_\_\_\_\_\_\_\_\_\_\_\_\_\_\_\_\_\_\_\_\_\_\_\_\_\_\_\_\_

\_\_\_\_\_\_\_\_\_\_\_\_\_\_\_\_\_\_\_\_\_\_\_\_\_\_\_\_\_\_\_\_\_\_\_\_\_\_\_\_\_\_\_\_\_\_\_\_\_\_\_\_\_\_\_\_\_\_\_\_\_\_\_\_

\num{10} Que adjetivos aparecem no texto? E que advérbio?
Os adjetivos são: ``linda'', ``mimosa'', ``branca'', ``carinhosa'', ``engraçada'', ``mansa'' e ``encantadora''. O advérbio é ``tão''.

\subsubsection{3. }\label{section-59}

\begin{quote}
Relacione os substantivos aos adjetivos que podem caracterizá-los.

\Paulo, precisa fazer caixinhas em duas colunas para serem relacionadas entre si.

ruas quentinho

sapato peludo

bolo esburacadas

cachorro alto

vestido ruivo

cabelo curto
\end{quote}

\subsection{Treino}\label{treino-6}

1. Leia um trecho de um conto popular.

\begin{quote}
\textbf{A guardadora de patos}

Era uma vez uma velha, muito velhinha, toda corcovada, que vivia com o
seu bando de patos num lugar deserto, no meio das montanhas, onde tinha
uma linda casinha. O sítio estava cercado de uma grande floresta [aonde] a
velha ia todas as manhãs, servindo-se de uma muleta para poder andar.
Trabalhava ali horas e horas com uma força extraordinária para a sua
idade; cortava a erva para os patos, que muito gostavam disso
{[}...{]}.

\fonte{Irmãos Grimm. Contos dos Irmãos Grimm. Rio de Janeiro: Livraria Garnier,
1932, p. 7. Disponível em: \emph{https://digital.bbm.usp.br/handle/bbm/7812}.
Acesso em: 22 fev. 2023.}
\end{quote}

\textbf{Vocabulário}
Corcovada: corcunda.

Uma das palavras que, no texto, caracterizam a guardadora de
patos é

(A) ``deserto''.

(B) ``velhinha''.

(C) ``linda''.

(D) ``grande''.

SAEB: Analisar os efeitos de sentido decorrentes do uso dos adjetivos.
BNCC: EF03LP09 -- Identificar, em textos, adjetivos e sua função de
atribuição de propriedades aos substantivos.

(A) Incorreta. A palavra ``deserto'' caracteriza o espaço em que se
passa a narrativa.

(B) Correta. Uma das palavras que caracterizam a
guardadora de patos, a personagem principal da história, é ``velhinha''.

(C) Incorreta. O adjetivo ``linda'' caracteriza a casa da personagem.

(D) Incorreta. O adjetivo ``grande'' caracteriza o espaço da narrativa.

2. Leia uma carta do leitor.

\begin{quote}
\textbf{Fama de guloso}

Nós, estudantes do quinto ano, tivemos a oportunidade de ler um texto sobre a fama de D. João VI como um homem guloso. Achamos essa curiosidade muito interessante e entendemos por que os artistas o retrataram como um indivíduo mais gordinho nos retratos e nos livros de história. Ficamos imaginando as festividades que o rei costumava promover naquela época.

\fonte{Fonte de pesquisa: CHC. Fama de guloso. Fala aqui! Disponível em:
\emph{http://chc.org.br/artigo/fala-aqui-303/}. Acesso em: 22 fev. 2023.}

Uma das características de D. João VI citadas no trecho é que ele era

(A) gordinho, segundo os retratos da época.

(B) guloso, segundo os estudantes do quinto ano.

(C) famoso, já que ele era o rei de Portugal.

(D) festeiro, promovendo muitas festas naquele época.

SAEB: Analisar os efeitos de sentido decorrentes do uso dos adjetivos.
BNCC: EF03LP23 -- Analisar o uso de adjetivos em cartas dirigidas a
veículos da mídia impressa ou digital (cartas do leitor ou de reclamação
a jornais ou revistas), digitais ou impressas.

(A) Correta. Segundo os leitores, os artistas da época retratava D. João VI como gordinho.

(B) Incorreta. A informação de que D. João VI era guloso não tem como fonte os estudantes do quinto ano, mas um texto que eles leram na revista.

(C) Incorreta. Fala-se em uma fama de ser guloso de D. João VI, mas isso não o caracteriza como famoso por ser o rei de Portugal.

(D) Incorreta. Apesar de se falar em festas promovidas por D. João VI, não se caracteriza o rei como festeiro.

3. Leia uma carta de leitor e a resposta que ela recebeu por parte da revista.

\begin{quote}
\textbf{Cabeça de jacaré}

Na última edição da revista, tinha um texto sobre a cabeça do jacaré.
Muito legal! Mas fiquei pensando: como a gente sabe que o crânio do jacaré
é mais longo que o crânio do crocodilo?

\textit{Oi, Pedro! Quem dá as informações para os textos da CHC são cientistas.
Cada um é especializado em uma área. O texto que você leu veio de um
especialista em crocodilos e jacarés. Até a próxima!}

\fonte{Fonte de pesquisa: CHC. Cabeça de jacaré. Fala aqui! Disponível em:
\emph{http://chc.org.br/artigo/fala-aqui/}. Acesso em: 22 fev. 2023.}
\end{quote}

No texto, o advérbio ``mais'' é responsável por

(A) alterar o sentido de outro advérbio.

(B) qualificar um substantivo.

(C) criar uma comparação de superioridade.

(D) diminuir a força do sentido de um adjetivo.

SAEB: Analisar os efeitos de sentido decorrentes do uso dos advérbios.
Não há correspondência com a BNCC do terceiro ano.

(A) Incorreta. O advérbio está ligado a um adjetivo.

(B) Incorreta. O advérbio está modificando um adjetivo.

(C) Correta. Por meio da expressão ``mais... que...'', em que aparece o advérbio ``mais'', cria-se uma comparação de superioridade.

(D) Incorreta. O advérbio tem sentido de intensificador.

\section{8. Fato e opinião}\label{muxf3dulo-8}

\begin{quote}
Neste módulo, espera-se que os alunos leiam e compreendam o texto,
identifiquem e localizem informações no texto e façam distinção entre
fatos e opiniões em textos.
\end{quote}

\subsection{Conteúdo}\label{conteuxfado-7}

\begin{quote}
https://pixabay.com/pt/illustrations/coment\%c3\%a1rios-relat\%c3\%b3rio-de-volta-2990424/

\includegraphics[width=3.98294in,height=2.65514in]{media/image24.jpeg}

\textbf{Fato ou opinião?}

\textbf{Fato} é algo que ocorreu e pode ser comprovado de algum modo,
por meio de determinado documento, números, vídeo ou registro. Observe o
exemplo:
\end{quote}

\begin{longtable}[]{@{}l@{}}
\toprule
\begin{minipage}[t]{0.97\columnwidth}\raggedright\strut
\begin{quote}
Há mais alunos matriculados em escolas públicas de Fortaleza - CE.
\end{quote}\strut
\end{minipage}\tabularnewline
\bottomrule
\end{longtable}

\begin{quote}
A informação consiste em um fato, uma vez que há registros dos casos e,
por meio deles, é possível fazer uma afirmação.

\textbf{Opinião} consiste em uma interpretação do fato, isto é, uma
forma pessoal de olhar o fato. Essa opinião pode ser diferente de pessoa
para pessoa a depender de muitos fatores. Veja o exemplo:
\end{quote}

\begin{longtable}[]{@{}l@{}}
\toprule
\begin{minipage}[t]{0.97\columnwidth}\raggedright\strut
\begin{quote}
Se aquela moça fosse mais alta, poderia ser modelo.
\end{quote}\strut
\end{minipage}\tabularnewline
\bottomrule
\end{longtable}

\begin{quote}
Leia o início de uma notícia para entender melhor o que é fato e o que é
opinião.

\textbf{https://chc.org.br/fofos-preguicosos-e-comiloes/}

\textbf{Fofos, preguiçosos e comilões}

\includegraphics[width=3.61389in,height=2.71042in]{media/image25.jpeg}

Sua aparência engana: apesar de lembrarem ursos de pelúcia, os coalas
são parentes dos cangurus, gambás e outros animais conhecidos
como~marsupiais. Encontrados na Austrália, esses mamíferos passam o dia
inteiro deitados em árvores, comendo folhas e dormindo. Que vida boa,
não é? {[}...{]}

FOFOS, preguiçosos e comilões. CHC.

Disponível em: \url{https://chc.org.br/fofos-preguicosos-e-comiloes/}.
Acesso em: 22 fev. 2023.

Note que o que está em verde apresenta um fato, uma afirmação, uma
certeza. O que está em roxo é uma opinião, pois não se sabe com certeza
se os coalas têm uma vida boa.
\end{quote}

\subsection{Atividades}\label{atividades-7}

\subsubsection{1. }\label{section-60}

\begin{quote}
O ECA é o Estatuto da Criança e do Adolescente que tem como finalidade
proteger e garantir o direito dos jovens. Em 2019, o ECA completou 29
anos desde sua criação.

\textbf{29 anos do ECA}

\includegraphics[width=5.90556in,height=2.30417in]{media/image26.jpeg}

16/07/2019~

No dia 13 de julho, o~Estatuto~da Criança e do Adolescente, o ECA,
completou mais um ano. E como estamos falando de aniversário, é
importante refletir sobre o que realmente se pode comemorar. Como quase
todo mundo já sabe, apesar do apelido engraçado, o ECA é coisa muito
séria. Trata-se do conjunto de normas criadas para garantir direitos e
proteger a criança e o adolescente contra a violência, o trabalho
infantil, a discriminação ou preconceito de qualquer tipo, humilhações
ou crueldades.

Nestes 29 anos, muita coisa boa aconteceu. Houve redução da mortalidade
infantil por crimes, a previsão do amplo acesso ao ensino fundamental, a
criação do plano nacional de educação e a implantação de testes
obrigatórios para recém-nascidos.

\protect\hypertarget{_Hlk128033654}{}{}Além disso, o~Estatuto~sofreu
alterações com o objetivo fazer cumprir o que nele está escrito. Foi
criado o Cadastro Nacional de Pessoas Desaparecidas, a idade mínima para
que a criança possa viajar desacompanhada dos pais passou para 16 anos
{[}...{]}.

Na hora de celebrar uma~legislação~tão importante, não se pode esquecer
os desafios que ainda permanecem. Por isso, é preciso conhecer bem o ECA
e entendê-lo, para defendê-lo!

PLENARINHO. 29 anos do ECA. Disponível em:
https://plenarinho.leg.br/index.php/2019/07/29-anos-eca/. Acesso em: 23
fev. 2023.

a) Segundo informações do texto, o que é o ECA? É o conjunto de normas
criadas para garantir direitos e proteger a criança e o adolescente
contra a violência, o trabalho infantil, a discriminação ou preconceito
de qualquer tipo, humilhações ou crueldades.

\_\_\_\_\_\_\_\_\_\_\_\_\_\_\_\_\_\_\_\_\_\_\_\_\_\_\_\_\_\_\_\_\_\_\_\_\_\_\_\_\_\_\_\_\_\_\_\_\_\_\_\_\_\_\_\_\_\_\_\_\_\_\_\_

\_\_\_\_\_\_\_\_\_\_\_\_\_\_\_\_\_\_\_\_\_\_\_\_\_\_\_\_\_\_\_\_\_\_\_\_\_\_\_\_\_\_\_\_\_\_\_\_\_\_\_\_\_\_\_\_\_\_\_\_\_\_\_\_

\_\_\_\_\_\_\_\_\_\_\_\_\_\_\_\_\_\_\_\_\_\_\_\_\_\_\_\_\_\_\_\_\_\_\_\_\_\_\_\_\_\_\_\_\_\_\_\_\_\_\_\_\_\_\_\_\_\_\_\_\_\_\_\_

\_\_\_\_\_\_\_\_\_\_\_\_\_\_\_\_\_\_\_\_\_\_\_\_\_\_\_\_\_\_\_\_\_\_\_\_\_\_\_\_\_\_\_\_\_\_\_\_\_\_\_\_\_\_\_\_\_\_\_\_\_\_\_\_

b) Qual foi a data de publicação dessa notícia? A data foi 16/07/2019~

\_\_\_\_\_\_\_\_\_\_\_\_\_\_\_\_\_\_\_\_\_\_\_\_\_\_\_\_\_\_\_\_\_\_\_\_\_\_\_\_\_\_\_\_\_\_\_\_\_\_\_\_\_\_\_\_\_\_\_\_\_\_\_\_

\_\_\_\_\_\_\_\_\_\_\_\_\_\_\_\_\_\_\_\_\_\_\_\_\_\_\_\_\_\_\_\_\_\_\_\_\_\_\_\_\_\_\_\_\_\_\_\_\_\_\_\_\_\_\_\_\_\_\_\_\_\_\_\_

c) Em que ano o ECA completou 29 anos? 2019

\_\_\_\_\_\_\_\_\_\_\_\_\_\_\_\_\_\_\_\_\_\_\_\_\_\_\_\_\_\_\_\_\_\_\_\_\_\_\_\_\_\_\_\_\_\_\_\_\_\_\_\_\_\_\_\_\_\_\_\_\_\_\_\_

\_\_\_\_\_\_\_\_\_\_\_\_\_\_\_\_\_\_\_\_\_\_\_\_\_\_\_\_\_\_\_\_\_\_\_\_\_\_\_\_\_\_\_\_\_\_\_\_\_\_\_\_\_\_\_\_\_\_\_\_\_\_\_\_

d) Quais foram as coisas boas que aconteceram nesses 29 anos de ECA?
Houve redução da mortalidade infantil por crimes, a previsão do amplo
acesso ao ensino fundamental, a criação do plano nacional de educação e
a implantação de testes obrigatórios para recém-nascidos.

\_\_\_\_\_\_\_\_\_\_\_\_\_\_\_\_\_\_\_\_\_\_\_\_\_\_\_\_\_\_\_\_\_\_\_\_\_\_\_\_\_\_\_\_\_\_\_\_\_\_\_\_\_\_\_\_\_\_\_\_\_\_\_\_

\_\_\_\_\_\_\_\_\_\_\_\_\_\_\_\_\_\_\_\_\_\_\_\_\_\_\_\_\_\_\_\_\_\_\_\_\_\_\_\_\_\_\_\_\_\_\_\_\_\_\_\_\_\_\_\_\_\_\_\_\_\_\_\_

\protect\hypertarget{_Hlk128040463}{}{}\_\_\_\_\_\_\_\_\_\_\_\_\_\_\_\_\_\_\_\_\_\_\_\_\_\_\_\_\_\_\_\_\_\_\_\_\_\_\_\_\_\_\_\_\_\_\_\_\_\_\_\_\_\_\_\_\_\_\_\_\_\_\_\_

\_\_\_\_\_\_\_\_\_\_\_\_\_\_\_\_\_\_\_\_\_\_\_\_\_\_\_\_\_\_\_\_\_\_\_\_\_\_\_\_\_\_\_\_\_\_\_\_\_\_\_\_\_\_\_\_\_\_\_\_\_\_\_\_

\_\_\_\_\_\_\_\_\_\_\_\_\_\_\_\_\_\_\_\_\_\_\_\_\_\_\_\_\_\_\_\_\_\_\_\_\_\_\_\_\_\_\_\_\_\_\_\_\_\_\_\_\_\_\_\_\_\_\_\_\_\_\_\_
\end{quote}

\begin{enumerate}
\def\labelenumi{\alph{enumi})}
\item
  Segundo o texto, o que é necessário para defender o ECA? É preciso
  conhecer bem e entendê-lo.
\end{enumerate}

\begin{quote}
\_\_\_\_\_\_\_\_\_\_\_\_\_\_\_\_\_\_\_\_\_\_\_\_\_\_\_\_\_\_\_\_\_\_\_\_\_\_\_\_\_\_\_\_\_\_\_\_\_\_\_\_\_\_\_\_\_\_\_\_\_\_\_\_

\_\_\_\_\_\_\_\_\_\_\_\_\_\_\_\_\_\_\_\_\_\_\_\_\_\_\_\_\_\_\_\_\_\_\_\_\_\_\_\_\_\_\_\_\_\_\_\_\_\_\_\_\_\_\_\_\_\_\_\_\_\_\_\_

\_\_\_\_\_\_\_\_\_\_\_\_\_\_\_\_\_\_\_\_\_\_\_\_\_\_\_\_\_\_\_\_\_\_\_\_\_\_\_\_\_\_\_\_\_\_\_\_\_\_\_\_\_\_\_\_\_\_\_\_\_\_\_\_

g) Numere as frases a seguir de acordo com a ordem das informações do
texto.

( 3 ) Nestes 29 anos, muita coisa boa aconteceu.

( 1 ) No dia 13 de julho, o~Estatuto~da Criança e do Adolescente, o ECA,
completou mais um ano.

( 4 ) Além disso, o~Estatuto~sofreu alterações com o objetivo fazer
cumprir o que nele está escrito.

( 2 ) Como quase todo mundo já sabe, apesar do apelido engraçado, o ECA
é coisa muito séria.

( 5 ) Por isso, é preciso conhecer bem o ECA e entendê-lo, para
defendê-lo!
\end{quote}

\subsubsection{2. }\label{section-61}

\begin{quote}
Releia este trecho do texto.
\end{quote}

\begin{longtable}[]{@{}l@{}}
\toprule
\begin{minipage}[t]{0.97\columnwidth}\raggedright\strut
Além disso, o~Estatuto~sofreu alterações com o objetivo fazer cumprir o
que nele está escrito. Foi criado o Cadastro Nacional de Pessoas
Desaparecidas, a idade mínima para que a criança possa viajar
desacompanhada dos pais passou para 16 anos {[}...{]}.

Na hora de celebrar uma~legislação~tão importante, não se pode esquecer
os desafios que ainda permanecem. Por isso, é preciso conhecer bem o ECA
e entendê-lo, para defendê-lo!\strut
\end{minipage}\tabularnewline
\bottomrule
\end{longtable}

\begin{quote}
a) Escreva o trecho que expressa:

Auxilie os alunos a distinguir fato de opinião, pois esse assunto é um
relativamente complexo para eles, mas é\\
fundamental que, gradativamente, eles se\\
habituem a realizar esse tipo de análise.
\end{quote}

\begin{itemize}
\item
  O \textbf{fato} que foi informado: o~Estatuto~sofreu alterações com o
  objetivo fazer cumprir o que nele está escrito.
\end{itemize}

\begin{quote}
\_\_\_\_\_\_\_\_\_\_\_\_\_\_\_\_\_\_\_\_\_\_\_\_\_\_\_\_\_\_\_\_\_\_\_\_\_\_\_\_\_\_\_\_\_\_\_\_\_\_\_\_\_\_\_\_\_\_\_\_\_\_\_\_

\_\_\_\_\_\_\_\_\_\_\_\_\_\_\_\_\_\_\_\_\_\_\_\_\_\_\_\_\_\_\_\_\_\_\_\_\_\_\_\_\_\_\_\_\_\_\_\_\_\_\_\_\_\_\_\_\_\_\_\_\_\_\_\_

\_\_\_\_\_\_\_\_\_\_\_\_\_\_\_\_\_\_\_\_\_\_\_\_\_\_\_\_\_\_\_\_\_\_\_\_\_\_\_\_\_\_\_\_\_\_\_\_\_\_\_\_\_\_\_\_\_\_\_\_\_\_\_\_
\end{quote}

\begin{itemize}
\item
  A \textbf{opinião} dos autores do texto: Na hora de celebrar uma
  legislação tão importante, não se pode esquecer os desafios que ainda
  permanecem.
\end{itemize}

\begin{quote}
\_\_\_\_\_\_\_\_\_\_\_\_\_\_\_\_\_\_\_\_\_\_\_\_\_\_\_\_\_\_\_\_\_\_\_\_\_\_\_\_\_\_\_\_\_\_\_\_\_\_\_\_\_\_\_\_\_\_\_\_\_\_\_\_

\_\_\_\_\_\_\_\_\_\_\_\_\_\_\_\_\_\_\_\_\_\_\_\_\_\_\_\_\_\_\_\_\_\_\_\_\_\_\_\_\_\_\_\_\_\_\_\_\_\_\_\_\_\_\_\_\_\_\_\_\_\_\_\_

\_\_\_\_\_\_\_\_\_\_\_\_\_\_\_\_\_\_\_\_\_\_\_\_\_\_\_\_\_\_\_\_\_\_\_\_\_\_\_\_\_\_\_\_\_\_\_\_\_\_\_\_\_\_\_\_\_\_\_\_\_\_\_\_
\end{quote}

\begin{itemize}
\item
  O \textbf{motivo} da opinião dos autores do texto: Por isso, é preciso
  conhecer bem o ECA e entendê-lo, para defendê-lo!
\end{itemize}

\begin{quote}
\_\_\_\_\_\_\_\_\_\_\_\_\_\_\_\_\_\_\_\_\_\_\_\_\_\_\_\_\_\_\_\_\_\_\_\_\_\_\_\_\_\_\_\_\_\_\_\_\_\_\_\_\_\_\_\_\_\_\_\_\_\_\_\_

\_\_\_\_\_\_\_\_\_\_\_\_\_\_\_\_\_\_\_\_\_\_\_\_\_\_\_\_\_\_\_\_\_\_\_\_\_\_\_\_\_\_\_\_\_\_\_\_\_\_\_\_\_\_\_\_\_\_\_\_\_\_\_\_

\_\_\_\_\_\_\_\_\_\_\_\_\_\_\_\_\_\_\_\_\_\_\_\_\_\_\_\_\_\_\_\_\_\_\_\_\_\_\_\_\_\_\_\_\_\_\_\_\_\_\_\_\_\_\_\_\_\_\_\_\_\_\_\_
\end{quote}

\subsubsection{3. }\label{section-62}

Leia mais um trecho de texto relacionado ao ECA.

\begin{quote}
Se julgar pertinente, após a leitura do trecho, comente com os alunos
que, antes de o Estatuto da Criança e do Adolescente (ECA) ser
promulgado, era comum ver crianças trabalhando em vez de estudar ou
brincar.

https://pixabay.com/pt/illustrations/diversidade-igualdade-crian\%c3\%a7as-5392891/

\includegraphics[width=2.70833in,height=3.25006in]{media/image27.png}

Inserir cores, conforme modelo.

{[}...{]}

A família, base da sociedade, é lugar onde os vínculos afetivos são
fortalecidos. Crianças são seres com diversas particularidades em fase
de desenvolvimento, por isso elas têm de brincar, estudar, tem que ser
amadas e devem se expressar. Esse é o período de ser protegida dentro
dessa condição de pessoa em desenvolvimento e que tem prioridade
absoluta. O estatuto valoriza esses vínculos e elos que são fundamentais
para o florescimento humano.

{[}...{]}

Disponível em:
\url{https://www.novaserrana.mg.gov.br/portal/noticias/0/3/3550/estatuto-da-crianca-e-do-adolescente-completa-30-anos}.
Acesso em: 23 fev. 2023.

a) O que é família, segundo informações do texto? A família, base da
sociedade, é lugar onde os vínculos afetivos são fortalecidos

\_\_\_\_\_\_\_\_\_\_\_\_\_\_\_\_\_\_\_\_\_\_\_\_\_\_\_\_\_\_\_\_\_\_\_\_\_\_\_\_\_\_\_\_\_\_\_\_\_\_\_\_\_\_\_\_\_\_\_\_\_\_\_\_

\_\_\_\_\_\_\_\_\_\_\_\_\_\_\_\_\_\_\_\_\_\_\_\_\_\_\_\_\_\_\_\_\_\_\_\_\_\_\_\_\_\_\_\_\_\_\_\_\_\_\_\_\_\_\_\_\_\_\_\_\_\_\_\_

\_\_\_\_\_\_\_\_\_\_\_\_\_\_\_\_\_\_\_\_\_\_\_\_\_\_\_\_\_\_\_\_\_\_\_\_\_\_\_\_\_\_\_\_\_\_\_\_\_\_\_\_\_\_\_\_\_\_\_\_\_\_\_\_

b) Conforme informações do trecho, as crianças são seres em fase de
desenvolvimento. Quais são os direitos delas?

\_\_\_\_\_\_\_\_\_\_\_\_\_\_\_\_\_\_\_\_\_\_\_\_\_\_\_\_\_\_\_\_\_\_\_\_\_\_\_\_\_\_\_\_\_\_\_\_\_\_\_\_\_\_\_\_\_\_\_\_\_\_\_\_

\_\_\_\_\_\_\_\_\_\_\_\_\_\_\_\_\_\_\_\_\_\_\_\_\_\_\_\_\_\_\_\_\_\_\_\_\_\_\_\_\_\_\_\_\_\_\_\_\_\_\_\_\_\_\_\_\_\_\_\_\_\_\_\_

\_\_\_\_\_\_\_\_\_\_\_\_\_\_\_\_\_\_\_\_\_\_\_\_\_\_\_\_\_\_\_\_\_\_\_\_\_\_\_\_\_\_\_\_\_\_\_\_\_\_\_\_\_\_\_\_\_\_\_\_\_\_\_\_

c) Dos trechos destacados, qual deles expressa um fato? Qual deles
expressa uma opinião?

O trecho destacado em verde expressa um fato. O trecho destacado em azul
expressa uma opinião.

\_\_\_\_\_\_\_\_\_\_\_\_\_\_\_\_\_\_\_\_\_\_\_\_\_\_\_\_\_\_\_\_\_\_\_\_\_\_\_\_\_\_\_\_\_\_\_\_\_\_\_\_\_\_\_\_\_\_\_\_\_\_\_\_

\_\_\_\_\_\_\_\_\_\_\_\_\_\_\_\_\_\_\_\_\_\_\_\_\_\_\_\_\_\_\_\_\_\_\_\_\_\_\_\_\_\_\_\_\_\_\_\_\_\_\_\_\_\_\_\_\_\_\_\_\_\_\_\_

\_\_\_\_\_\_\_\_\_\_\_\_\_\_\_\_\_\_\_\_\_\_\_\_\_\_\_\_\_\_\_\_\_\_\_\_\_\_\_\_\_\_\_\_\_\_\_\_\_\_\_\_\_\_\_\_\_\_\_\_\_\_\_\_
\end{quote}

\subsection{Treino}\label{treino-7}

\subsubsection{1. }\label{section-63}

\begin{quote}
(Fácil) Leia um trecho de uma entrevista com a escritora Ruth Rocha.

www1.folha.uol.com.br/folhinha/2014/09/1510852-computador-nao-faz-com-que-se-leia-menos-diz-ruth-rocha-leia-entrevista.shtml?mobile

\textbf{`Computador não faz com que se leia menos', diz Ruth Rocha; leia
entrevista}

{[}\ldots{}{]}

Folhinha - A infância mudou nesses 45 anos?

Ruth Rocha - As crianças são muito parecidas. Por isso, livros infantis

mais antigos e contos de fadas ainda encantam gente do mundo todo.

{[}\ldots{}{]}

Usar o computador faz com que as crianças leiam menos?

Não acho. Nunca se vendeu ou produziu tanto livro. Na minha época,

não tínhamos opções, meus colegas não conversavam sobre literatura e as

escolas não tinham bibliotecas. {[}\ldots{}{]}

MOLINERO, Bruno. `Computador não faz com que se leia menos', diz Ruth
Rocha; leia entrevista. Folhinha, 6 set. 2014. Disponível em:
www1.folha.uol.com.br/folhinha/2014/09/1510852-computador-nao-faz-com-que-se-leia-menos-diz-ruth-rocha-leia-entrevista.shtml?mobile.
Acesso em: 23 fev. 2023.

Ao ler a entrevista, pode-se identificar que a autora está

(A) explicando o funcionamento das bibliotecas das escolas.

(B) dando a sua opinião sobre os leitores infantis.

(C) relembrando a produção de livros da época da sua infância.

(D) explicando o fato da falta de conversa sobre literatura com seus
colegas.

Saeb D13 - Estabelecer, no interior de um texto, relação entre um fato e
uma opinião relativa a este fato.

BNCC: Não há correspondência.

(A) Incorreta. A autora apenas cita que não havia bibliotecas em sua
escola.

(B) Correta. A autora está dando suas opiniões sobre os leitores
infantis, perpassando temas como o uso do computador e se isso afeta a
leitura.

(C) Incorreta. Esse é um argumento que a autora utiliza para falar como
ela acredita que as crianças de hoje leem mais do que antigamente.

(D) Incorreta. A autora apenas menciona esse assunto. Tanto as perguntas
do jornalista como as respostas da entrevistada giram em torno das
opiniões da escritora a respeito da leitura das crianças.
\end{quote}

\subsubsection{2. }\label{section-64}

\begin{quote}
(Médio) Leia um trecho da reportagem a seguir.

\textbf{Professores do Rio usam as redes sociais para compartilhar
aulas}

No dia seguinte que soube da hashtag \#ComparilheUmaAula, o professor de
matemática Deivison de Albuquerque estava com um vídeo no Facebook
compartilhando uma aula {[}...{]} voltada para os alunos do 9º ano, que
estão com as aulas suspensas. {[}...{]}

``A minha área, matemática, requer uma certa constância de estudos.
Estamos em um momento complicado, que não sabemos quando retornaremos às
aulas regulares. É muito importante manter os estudos, procurar
videoaulas, para não ficar zerado'', diz o professor da Escola Municipal
Alberto José Sampaio, na Pavuna, na zona norte do Rio.

TOKARNIA, Mariana. Professores do Rio usam as redes sociais para
compartilhar aulas. EBC. Disponível em:
https://agenciabrasil.ebc.com.br/educacao/noticia/2020-03/professores-do-rio-usam-redes-sociais-para-compartilhar-aulas.
Acesso em: 26 mar. 2020.

\protect\hypertarget{_Hlk128058360}{}{}No segundo parágrafo, o trecho
que está entre aspas mostra a

(A) opinião do autor da reportagem.

(B) opinião do professor de matemática.

(C) explicação sobre a hashtag criada.

(D) introdução das videoaulas.

Saeb D13 - Estabelecer, no interior de um texto, relação entre um fato e
uma opinião relativa a este fato.

BNCC: Não há correspondência.

(A) Incorreta. Não existe, na reportagem, a opinião do autor da
reportagem.

(B) Correta. As aspas indicam a opinião do professor de matemática.

(C) Incorreta. O professor apenas comenta sobre a importância de se
manter os estudos enquanto as aulas regulares não voltam.

(D) Incorreta. Inexiste introdução em relação às videoaulas na
reportagem.
\end{quote}

\subsubsection{3.}\label{section-65}

\begin{quote}
(Difícil) Leia o texto a seguir.

\textbf{Livro infantil fala sobre sustentabilidade na Amazônia}

O programa Tarde Nacional {[}...{]} falou sobre a obra infantil Viagem
Amazônica. A entrevistada foi a autora e ilustradora do livro, Gabriela
Brioschi.

Ela contou como as diversas questões da Amazônia são abordadas na
narrativa por meio das aventuras da GabyGaby, personagem principal do
livro.~

Viagem Amazônica é o primeiro de uma coleção de 4 livros que falam da
cultura regional e das lendas das diferentes regiões do país. A obra
integra o SustentaMundo, um projeto cultural e educativo que estimula as
crianças a refletirem sobre temas relacionados à educação ambiental e
emocional.

RÁDIOS EBC. Livro infantil fala sobre sustentabilidade na Amazônia.
Disponível em:
http://radios.ebc.com.br/tarde-nacional-amazonia/2019/07/livro-infantil-fala-sobre-sustentabilidade-na-amazonia.
Acesso em: 23 fev. 2023.

Um trecho que apresenta um fato sobre o conteúdo do livro é

(A) ``A entrevistada foi a autora e ilustradora do livro, Gabriela
Brioschi''.

(B) ``{[}...{]} falam da cultura regional e das lendas das diferentes
regiões do país''.

(C) ``A obra integra o SustentaMundo, um projeto cultural e educativo
que estimula as crianças''.

(D) ``O programa Tarde Nacional {[}...{]} falou sobre a obra infantil
Viagem Amazônica''.

Saeb D13 - Estabelecer, no interior de um texto, relação entre um fato e
uma opinião relativa a este fato.

BNCC: Não há correspondência.

(A) Incorreta. Nesse trecho é expresso o nome da autora e ilustradora do
livro, não sendo um fato sobre o livro.

(B) Correta. Em ``Viagem Amazônica é o primeiro de uma coleção de 4
livros que falam da cultura regional e das lendas das diferentes regiões
do país'', há apresentação de um fato, assim como em ``as diversas
questões da Amazônia são abordadas na narrativa por meio das aventuras
da GabyGaby''.

(C) Incorreta. O trecho aborda o projeto cultural do qual a obra faz
parte, não citando fatos do livro.

(D) Incorreta. O trecho mostra o nome da obra infantil, não abordando os
fatos do conteúdo do livro.
\end{quote}

\section{Módulo 9}\label{muxf3dulo-9}

\begin{quote}
Neste módulo, os alunos vão ler e analisar gráficos e tabelas,
reconhecendo afunção deles.
\end{quote}

\subsection{Conteúdo}\label{conteuxfado-8}

\begin{quote}
\textbf{Gráficos e tabelas}

\textbf{Tabela} consiste em um gênero textual que tem a finalidade de
organizar informações e dados numéricos em linhas e colunas, de forma a
facilitar a compreensão, pelo leitor, das informações e dos dados
apresentados.

Pode-se usar as tabelas para organizar variados tipos de informações
diárias, como: horário das aulas, resultados de jogos, etc.

As tabelas podem ser: elaboradas e preenchidas à mão; elaboradas e
preenchidas no computador; elaboradas no computador e preenchidas à mão.

Existem muitos programas de computador que podem ser usados para
elaborar tabelas. O mais comum e apropriado é o Excel.

Observe uma tabela com os horários de serviços da prefeitura de Sengés
nos dias de jogos do Brasil na Copa do Mundo de 2022.

\includegraphics[width=4.56744in,height=2.66667in]{media/image28.png}

Disponível em:
\url{https://www.senges.pr.gov.br/portal/noticias/horario-de-servicos-da-prefeitura-serao-diferenciados-em-dias-de-jogos-do-br/}.
Acesso em: 24 fev. 2023.

Os \textbf{gráficos}, por sua vez, podem ser encontrados em todos os
contextos em que haja necessidade de manipulação de dados, ou seja,
precisem informar dados para transmitir informações ao leitor. Podem ser
veiculados em jornais, revistas, almanaques, artigos científicos, livros
técnicos, \emph{sites} de variedades e diversos outros meios, impressos
ou digitais.

Quem elabora um gráfico, precisa entender muito bem os dados para poder
``traduzi-los'' ao leitor. Um gráfico pode ser feito manualmente, mas
também há ferramentas (programas e aplicativos) em computadores que
facilitam esse trabalho.

Há vários tipos de gráficos. Veja alguns:

Arte: inserir os modelos de gráficos conforme modelos a seguir.
\end{quote}

\begin{longtable}[]{@{}l@{}}
\toprule
\begin{minipage}[t]{0.97\columnwidth}\raggedright\strut
Gráfico de colunas{{[}CHART{]}}

Gráfico de barras

{{[}CHART{]}}

Gráfico de \emph{pizza}

{{[}CHART{]}}\strut
\end{minipage}\tabularnewline
\bottomrule
\end{longtable}

\subsection{Atividades}\label{atividades-8}

\begin{quote}
Explicar aos alunos que este gráfico é conhecido como gráfico de barras.
Caso julgue pertinente, mostre aos alunos como se elabora um gráfico
usando um programa de computador.
\end{quote}

\subsubsection{1. }\label{section-66}

\begin{quote}
Analise o gráfico a seguir, observando sua função e as informações
apresentadas.

\includegraphics[width=5.68530in,height=4.10069in]{media/image29.png}

IBGE. Tipos de gráficos no ensino. Disponível em:
\textless{}https://educa.ibge.gov.br/professores/educa-recursos/20773-tipos-de-graficos-no-ensino.html\textgreater{}.
Acesso em: 24 fev. 2023.

a) Qual é a finalidade desse gráfico? A finalidade do gráfico é
apresentar ao leitor quanto cada região brasileira produz de leite.

\_\_\_\_\_\_\_\_\_\_\_\_\_\_\_\_\_\_\_\_\_\_\_\_\_\_\_\_\_\_\_\_\_\_\_\_\_\_\_\_\_\_\_\_\_\_\_\_\_\_\_\_\_\_\_\_\_\_\_\_\_\_\_\_

\_\_\_\_\_\_\_\_\_\_\_\_\_\_\_\_\_\_\_\_\_\_\_\_\_\_\_\_\_\_\_\_\_\_\_\_\_\_\_\_\_\_\_\_\_\_\_\_\_\_\_\_\_\_\_\_\_\_\_\_\_\_\_\_

\_\_\_\_\_\_\_\_\_\_\_\_\_\_\_\_\_\_\_\_\_\_\_\_\_\_\_\_\_\_\_\_\_\_\_\_\_\_\_\_\_\_\_\_\_\_\_\_\_\_\_\_\_\_\_\_\_\_\_\_\_\_\_\_

b) Como você chegou a essa conclusão? Espera-se que os alunos que pelo
título do gráfico ``ranking da produtividade de leite'' e pelos nomes
das regiões que estão abaixo das barras: ``Sul, Sudeste, Centro-Oeste,
Nordeste e Norte''.

\_\_\_\_\_\_\_\_\_\_\_\_\_\_\_\_\_\_\_\_\_\_\_\_\_\_\_\_\_\_\_\_\_\_\_\_\_\_\_\_\_\_\_\_\_\_\_\_\_\_\_\_\_\_\_\_\_\_\_\_\_\_\_\_

\_\_\_\_\_\_\_\_\_\_\_\_\_\_\_\_\_\_\_\_\_\_\_\_\_\_\_\_\_\_\_\_\_\_\_\_\_\_\_\_\_\_\_\_\_\_\_\_\_\_\_\_\_\_\_\_\_\_\_\_\_\_\_\_

\_\_\_\_\_\_\_\_\_\_\_\_\_\_\_\_\_\_\_\_\_\_\_\_\_\_\_\_\_\_\_\_\_\_\_\_\_\_\_\_\_\_\_\_\_\_\_\_\_\_\_\_\_\_\_\_\_\_\_\_\_\_\_\_

c) O que as barras presentes no gráfico indicam? As barras indicam que a
diferença entre Nordeste e Norte é menor do que entre Sul e
Centro-Oeste.

Explique aos alunos que um dos recursos gráfico-visuais presente no
gráfico citado são as barras correspondentes a cada região brasileira
que apresentam, em escala, a quantidade de leite produzido em cada uma
delas. Assim, compreende-se, pela altura das barras, que a diferença de
leite produzido entre as regiões Norte e Nordeste é menor do que entre
as regiões Sul e Centro-Oeste, tendo em vista a barra referente à região
Sul é muito maior do que a do Centro-Oeste.

\_\_\_\_\_\_\_\_\_\_\_\_\_\_\_\_\_\_\_\_\_\_\_\_\_\_\_\_\_\_\_\_\_\_\_\_\_\_\_\_\_\_\_\_\_\_\_\_\_\_\_\_\_\_\_\_\_\_\_\_\_\_\_\_

\_\_\_\_\_\_\_\_\_\_\_\_\_\_\_\_\_\_\_\_\_\_\_\_\_\_\_\_\_\_\_\_\_\_\_\_\_\_\_\_\_\_\_\_\_\_\_\_\_\_\_\_\_\_\_\_\_\_\_\_\_\_\_\_

\_\_\_\_\_\_\_\_\_\_\_\_\_\_\_\_\_\_\_\_\_\_\_\_\_\_\_\_\_\_\_\_\_\_\_\_\_\_\_\_\_\_\_\_\_\_\_\_\_\_\_\_\_\_\_\_\_\_\_\_\_\_\_\_

d) Segundo os dados do gráfico, a região Sul está em qual colocação no
\emph{ranking} de maior produtor de leite? Marque a alternativa correta.

( x ) primeiro lugar

( ) segundo lugar

( ) terceiro lugar

( ) quarto lugar

e) A leitura do gráfico é feita da esquerda para a direito. Então, os
resultados são lidos ( x ) do maior produtor de leite (Sul) para o menor
(Norte).

( ) do menor produtor de leite (Norte) para o maior (Sul).

f) Houve regiões com resultados iguais na pesquisa? Justifique sua
resposta. Não. Ao olhar as barras, compreende-se que cada uma tem uma
altura diferente, portanto, os resultados da pesquisa feita pelo gráfico
apontam dados diferentes.

\_\_\_\_\_\_\_\_\_\_\_\_\_\_\_\_\_\_\_\_\_\_\_\_\_\_\_\_\_\_\_\_\_\_\_\_\_\_\_\_\_\_\_\_\_\_\_\_\_\_\_\_\_\_\_\_\_\_\_\_\_\_\_\_

\_\_\_\_\_\_\_\_\_\_\_\_\_\_\_\_\_\_\_\_\_\_\_\_\_\_\_\_\_\_\_\_\_\_\_\_\_\_\_\_\_\_\_\_\_\_\_\_\_\_\_\_\_\_\_\_\_\_\_\_\_\_\_\_

\_\_\_\_\_\_\_\_\_\_\_\_\_\_\_\_\_\_\_\_\_\_\_\_\_\_\_\_\_\_\_\_\_\_\_\_\_\_\_\_\_\_\_\_\_\_\_\_\_\_\_\_\_\_\_\_\_\_\_\_\_\_\_\_

\_\_\_\_\_\_\_\_\_\_\_\_\_\_\_\_\_\_\_\_\_\_\_\_\_\_\_\_\_\_\_\_\_\_\_\_\_\_\_\_\_\_\_\_\_\_\_\_\_\_\_\_\_\_\_\_\_\_\_\_\_\_\_\_
\end{quote}

\subsubsection{2. }\label{section-67}

\begin{quote}
Observe a tabela a seguir das dez empresas mais reclamadas no Procon do
Rio de Janeiro.

Explique aos alunos que o Procon é um órgão que orienta e auxilia os
consumidores na busca pelos seus direitos. Existem Procons espalhados
por todo o território brasileiro.

https://www.minhaoperadora.com.br/2018/07/teles-sao-as-maisreclamadas-no-procon-do-rio-de-janeiro.html

\includegraphics[width=2.72917in,height=3.70833in]{media/image30.jpeg}

Disponível em: \protect\hypertarget{_Hlk128145216}{}{}\textless{}
https://www.minhaoperadora.com.br/2018/07/teles-sao-as-mais-reclamadas-no-procon-do-rio-de-janeiro.html\textgreater{}.
Acesso em: 24 fev. 2023.

a) Qual é o título da tabela? Procon do Rio de Janeiro

\_\_\_\_\_\_\_\_\_\_\_\_\_\_\_\_\_\_\_\_\_\_\_\_\_\_\_\_\_\_\_\_\_\_\_\_\_\_\_\_\_\_\_\_\_\_\_\_\_\_\_\_\_\_\_\_\_\_\_\_\_\_\_\_

\_\_\_\_\_\_\_\_\_\_\_\_\_\_\_\_\_\_\_\_\_\_\_\_\_\_\_\_\_\_\_\_\_\_\_\_\_\_\_\_\_\_\_\_\_\_\_\_\_\_\_\_\_\_\_\_\_\_\_\_\_\_\_\_

b) Segundo dados da tabela, qual foi a empresa mais reclamada pelos
cariocas? A Operadora Oi, com 104 reclamações.

\_\_\_\_\_\_\_\_\_\_\_\_\_\_\_\_\_\_\_\_\_\_\_\_\_\_\_\_\_\_\_\_\_\_\_\_\_\_\_\_\_\_\_\_\_\_\_\_\_\_\_\_\_\_\_\_\_\_\_\_\_\_\_\_

\_\_\_\_\_\_\_\_\_\_\_\_\_\_\_\_\_\_\_\_\_\_\_\_\_\_\_\_\_\_\_\_\_\_\_\_\_\_\_\_\_\_\_\_\_\_\_\_\_\_\_\_\_\_\_\_\_\_\_\_\_\_\_\_

\_\_\_\_\_\_\_\_\_\_\_\_\_\_\_\_\_\_\_\_\_\_\_\_\_\_\_\_\_\_\_\_\_\_\_\_\_\_\_\_\_\_\_\_\_\_\_\_\_\_\_\_\_\_\_\_\_\_\_\_\_\_\_\_

c) E qual foi a empresa com o menor número de reclamações? A Cedae, com
41 reclamações.

\protect\hypertarget{_Hlk128145702}{}{}\_\_\_\_\_\_\_\_\_\_\_\_\_\_\_\_\_\_\_\_\_\_\_\_\_\_\_\_\_\_\_\_\_\_\_\_\_\_\_\_\_\_\_\_\_\_\_\_\_\_\_\_\_\_\_\_\_\_\_\_\_\_\_\_

\_\_\_\_\_\_\_\_\_\_\_\_\_\_\_\_\_\_\_\_\_\_\_\_\_\_\_\_\_\_\_\_\_\_\_\_\_\_\_\_\_\_\_\_\_\_\_\_\_\_\_\_\_\_\_\_\_\_\_\_\_\_\_\_

\_\_\_\_\_\_\_\_\_\_\_\_\_\_\_\_\_\_\_\_\_\_\_\_\_\_\_\_\_\_\_\_\_\_\_\_\_\_\_\_\_\_\_\_\_\_\_\_\_\_\_\_\_\_\_\_\_\_\_\_\_\_\_\_

d) Em sua opinião, essa tabela facilita a organização das informações?
Justifique sua resposta. Espera-se que os alunos respondam que sim, uma
vez que ela organiza e resume os dados.

\_\_\_\_\_\_\_\_\_\_\_\_\_\_\_\_\_\_\_\_\_\_\_\_\_\_\_\_\_\_\_\_\_\_\_\_\_\_\_\_\_\_\_\_\_\_\_\_\_\_\_\_\_\_\_\_\_\_\_\_\_\_\_\_

\_\_\_\_\_\_\_\_\_\_\_\_\_\_\_\_\_\_\_\_\_\_\_\_\_\_\_\_\_\_\_\_\_\_\_\_\_\_\_\_\_\_\_\_\_\_\_\_\_\_\_\_\_\_\_\_\_\_\_\_\_\_\_\_

\_\_\_\_\_\_\_\_\_\_\_\_\_\_\_\_\_\_\_\_\_\_\_\_\_\_\_\_\_\_\_\_\_\_\_\_\_\_\_\_\_\_\_\_\_\_\_\_\_\_\_\_\_\_\_\_\_\_\_\_\_\_\_\_
\end{quote}

\subsection{Treino}\label{treino-8}

\subsubsection{1.}\label{section-68}

(Fácil) Emerson mediu o comprimento dos filhotinhos de sua gatinha,
anotando os dados em uma tabela.

\begin{quote}
Arte: compor tabela conforme modelo
\end{quote}

\begin{longtable}[]{@{}ll@{}}
\toprule
& \textbf{Comprimento em centímetros}\tabularnewline
\midrule
\endhead
Filhote 1 & 10\tabularnewline
Filhote 2 & 11\tabularnewline
Filhote 3 & 12\tabularnewline
Filhote 4 & 9\tabularnewline
Filhote 5 & 10\tabularnewline
\bottomrule
\end{longtable}

Qual foi a medida de comprimento que apareceu mais vezes?

(A) 10 cm.

(B) 12 cm.

(C) 9 cm.

(D) 11 cm.

Saeb D3 - Estabelecer relação entre informações num texto ou entre
diferentes textos.

BNCC: Não há correspondência.

(A) Correta. A medida que mais apareceu foi 10, nos filhotinhos 1 e 5.

(B) Incorreta. A medida 12 cm aparece apenas uma vez.

(C) Incorreta. A medida 9 cm aparece apenas uma vez.

(D) Incorreta. A medida 11 cm aparece apenas uma vez.

\subsubsection{2. }\label{section-69}

(Médio) Fernanda realizou uma pesquisa em sua escola para descobrir quem
joga mais vôlei: meninas ou meninos.

\begin{quote}
Arte: fazer gráfico conforme o modelo. Trocar homens e mulheres por
meninas e meninos. Tirar porcentagem dos números, deixar 53 e 47
\end{quote}

\includegraphics[width=2.60244in,height=2.32558in]{media/image31.png}

Pela pesquisa de Fernanda,

(A) as meninas jogam mais, pois tem 6 a mais que os meninos.

(B) os meninos jogam mais, pois tem 6 a mais que as meninas.

(C) as meninas jogam mais, pois tem 7 a mais que os meninos.

(D) os meninos jogam mais, pois tem 7 a mais que as meninas.

\begin{quote}
Saeb D1 - Localizar informações num texto.

BNCC: Não há correspondência.

(A) Correta. No gráfico, o gênero que mais joga é o das meninas por ser
o maior número. E a diferença entre eles é 6.

(B) Incorreta. O aluno se confundiu ao dizer qual o gênero que mais
joga.

(C) Incorreta. O aluno se confundiu ao dizer a diferença entre os
gêneros.

(D) Incorreta. O aluno se confundiu ao dizer a diferença entre os
gêneros e confundiu a diferença entre eles.
\end{quote}

\subsubsection{3. }\label{section-70}

(Difícil) Em 2015, o IBGE, Instituto Brasileiro de Geografia e
Estatística, fez uma pesquisa em relação à cor, raça e etnia das pessoas
por autodeclaração, ou seja, o próprio entrevistado respondeu a qual
grupo pertence. Observe no gráfico o resultado dessa pesquisa.

\textbf{\textless{}
https://educa.ibge.gov.br/criancas/brasil/nosso-povo/19624-cor-ou-raca.html\textgreater{}.
}

\includegraphics[width=5.51042in,height=3.98702in]{media/image32.jpeg}

\protect\hypertarget{_Hlk40547643}{}{}

Fonte: IBGE. Cor ou raça. Disponível em: \textless{}
https://educa.ibge.gov.br/criancas/brasil/nosso-povo/19624-cor-ou-raca.html\textgreater{}.
Acesso em 24 fev. 2023.

Conforme informações presentes no gráfico, a população do Brasil é
resultado da mistura de diversas raças, e

(A) os de cor parda representam a minoria.

(B) na pesquisa, várias pessoas se autodeclararam amarelas.

(C) metade da população respondeu que é da cor branca.

(D) a cada 100 pessoas entrevistadas, menos de 1 é indígena.

\begin{quote}
Saeb D3 - Estabelecer relação entre informações num texto ou entre
diferentes textos.

BNCC: Não há correspondência.

(A) Incorreta. A cor parda representa 45 pessoas de cada 100
entrevistada.

(B) Incorreta. Menos de 1 por cento se autodeclararam amarelas.

(C) Incorreta. Para ser metade, teriam que ser 50 pessoas e não 45, como
mostra o gráfico.

(D) Correta. Conforme o gráfico, é possível perceber como é a
diversidade populacional brasileira, e por meio da análise dos dados,
afirmar que de cada 100 pessoas entrevistadas menos de 1 é indígena.
\end{quote}

\section{Módulo 10}\label{muxf3dulo-10}

\begin{quote}
Nesta seção, os alunos vão analisar trecho de texto e compreender a
função dos pronomes pessoais; identificar pronomes pessoais no contexto
apresentado; recuperar relações entre partes do texto, identificando
substituições lexicais por pronomes pessoais; identificar em textos a
concordância entre substantivo ou pronome pessoal e verbo (concordância
verbal).
\end{quote}

\subsection{Conteúdo}\label{conteuxfado-9}

\begin{quote}
\url{https://www.istockphoto.com/br/vetor/ilustra\%C3\%A7\%C3\%A3o-conceitual-de-vetor-plano-isom\%C3\%A9trico-3d-do-vocabul\%C3\%A1rio-online-gm1357666281-431507042?utm_source=pixabay\&utm_medium=affiliate\&utm_campaign=SRP_illustration_sponsored\&utm_content=https\%3A\%2F\%2Fpixabay.com\%2Fpt\%2Fillustrations\%2Fsearch\%2Fl\%25C3\%25A9xico\%2F\%3Fmanual_search\%3D1\&utm_term=l\%C3\%A9xico}

\textbf{Pronomes pessoais}

\includegraphics[width=4.81250in,height=3.20531in]{media/image33.jpeg}

Pronome consiste na palavra que substitui ou acompanha um nome, um
substantivo. Pode apresentar variação de gênero e número, de acordo com
o termo ao qual se refere, e desempenhar variadas funções em um texto.

Observe a seguir:
\end{quote}

\begin{longtable}[]{@{}l@{}}
\toprule
\begin{minipage}[t]{0.97\columnwidth}\raggedright\strut
\begin{itemize}
\item
  eu: 1ª pessoa do singular - nós: 1ª pessoa do plural;
\item
  tu: 2ª pessoa do singular - vós: 2ª pessoa do plural;
\item
  ele/ela: 3ª pessoa do singular - eles/elas: 3ª pessoa do plural.
\end{itemize}\strut
\end{minipage}\tabularnewline
\bottomrule
\end{longtable}

\begin{longtable}[]{@{}l@{}}
\toprule
\begin{minipage}[t]{0.97\columnwidth}\raggedright\strut
\begin{quote}
Os pronomes que apontam as pessoas do discurso denominam-se
\textbf{pronomes pessoais}:
\end{quote}

\begin{itemize}
\item
  quem fala -- 1ª pessoa;
\item
  com quem se fala -- 2ª pessoa;
\item
  de quem se fala -- 3ª pessoa.
\end{itemize}\strut
\end{minipage}\tabularnewline
\bottomrule
\end{longtable}

\begin{quote}
A utilização do pronome pessoal permite que o falante indique ao seu
interlocutor aquilo de que se fala (um ser, um objeto, entre outos), que
pode ou não estar presente no momento da interação.

O uso do pronome \textbf{tu} comumente acontece na região Sul do país e
em determinadas localidades das regiões Sudeste, Norte e Nordeste.

Todavia, na oralidade, nem sempre se flexiona o informal verbo na
segunda pessoa. É comum ouvir ``tu comeu'', ``tu sorriu'', que consiste
na variação linguística que ocorre no país.

As palavras utilizadas para nos dirigirmos às pessoas denominam-se
\textbf{pronomes de tratamento}. É o caso de \textbf{você}, uso mais
informal, e de \textbf{senhor} ou \textbf{senhora}, uso mais formal. São
utilizados do mesmo modo que os pronomes pessoais.
\end{quote}

\subsection{Atividades}\label{atividades-9}

\subsubsection{1. }\label{section-71}

\begin{quote}
Leia a narrativa a seguir. Sugere-se propor inicialmente uma leitura
silenciosa. Após a leitura, retomar os aspectos que chamaram a atenção
de cada aluno. Pode-se fazer, em seguida, uma leitura compartilhada,
parágrafo a parágrafo, conversando sobre as palavras desconhecidas e as
impressões relacionadas ao texto.

https://www.istockphoto.com/br/vetor/castelo-com-arquitetura-do-pal\%C3\%A1cio-majestic-e-conto-de-fadas-como-cen\%C3\%A1rio-em-desenho-gm1384445266-443752317?utm\_source=pixabay\&utm\_medium=affiliate\&utm\_campaign=SRP\_illustration\_sponsored\&utm\_content=https\%3A\%2F\%2Fpixabay.com\%2Fpt\%2Fillustrations\%2Fsearch\%2Fprincesa\%2F\%3Fmanual\_search\%3D1\&utm\_term=princesa

\includegraphics[width=3.15101in,height=2.21875in]{media/image34.jpeg}
\end{quote}

A princesa e a ervilha

~ ~ Era uma vez um príncipe que queria se casar com uma princesa, mas
uma princesa de verdade, de sangue real mesmo. Viajou pelo mundo
inteiro, à procura da princesa dos seus sonhos, mas todas as que ele
encontrava tinha algum defeito. Não é que faltassem princesas, não.
Havia de sobra, mas a dificuldade era saber se realmente eram de sangue
real. E o príncipe retornou ao seu castelo, muito triste e~desiludido,
pois queria muito casar com uma princesa de verdade.

~ ~ Uma noite desabou uma tempestade~medonha. Chovia desabaladamente,
com trovoadas, raios e relâmpagos. Um espetáculo tremendo!

~ ~ De repente bateram à porta do castelo, e o rei em pessoa foi
atender, pois os criados estavam ocupados enxugando as salas cujas
janelas foram abertas pela tempestade.

~ ~ Era uma moça, que dizia ser uma princesa. Mas estava encharcada de
tal maneira, os cabelos escorrendo, as roupas grudadas ao corpo, os
sapatos quase desmanchando\ldots{} que era difícil acreditar que fosse
realmente uma princesa real.

~ ~ A moça tanto afirmou que era uma princesa que a rainha pensou numa
forma de provar se o que ela dizia era verdade.

~ ~ Ordenou que sua criada de confiança empilhasse vinte colchões no
quarto de hóspedes e colocou sob eles uma ervilha. Aquela seria a cama
da ``princesa''.

~ ~ A moça estranhou a altura da cama, mas conseguiu, com a ajuda e uma
escada, se deitar.

~ ~ No dia seguinte, a rainha perguntou como ela havia dormido.

~ ~ -- Oh! Não consegui dormir -- respondeu a moça -- havia algo duro na
minha cama, e me deixou até manchas roxas no corpo!

~ ~ \protect\hypertarget{_Hlk128236047}{}{}O rei, a rainha e o príncipe
se olharam com surpresa. A moça era realmente uma princesa! Só mesmo uma
princesa verdadeira teria pele tão sensível para sentir um grão de
ervilha sob vinte colchões!!!

~ ~ O príncipe casou com a princesa, feliz da vida, e a ervilha foi
enviada para um museu, e ainda deve estar por lá \ldots{}

ANDERSEN, Hans Christian. \textbf{A princesa e a ervilha.} Disponível
em:
\url{https://alfabetizacao.mec.gov.br/images/conta-pra-mim/livros/versao_digital/a_princesa_e_a_ervilha_versao_digital.pdf}.
Acesso em: 25 fev. 2023.

a) Quem são as personagens da história? O príncipe, a princesa, a
rainha, o rei e os empregados.

\_\_\_\_\_\_\_\_\_\_\_\_\_\_\_\_\_\_\_\_\_\_\_\_\_\_\_\_\_\_\_\_\_\_\_\_\_\_\_\_\_\_\_\_\_\_\_\_\_\_\_\_\_\_\_\_\_\_\_\_\_\_\_\_

\_\_\_\_\_\_\_\_\_\_\_\_\_\_\_\_\_\_\_\_\_\_\_\_\_\_\_\_\_\_\_\_\_\_\_\_\_\_\_\_\_\_\_\_\_\_\_\_\_\_\_\_\_\_\_\_\_\_\_\_\_\_\_\_

\_\_\_\_\_\_\_\_\_\_\_\_\_\_\_\_\_\_\_\_\_\_\_\_\_\_\_\_\_\_\_\_\_\_\_\_\_\_\_\_\_\_\_\_\_\_\_\_\_\_\_\_\_\_\_\_\_\_\_\_\_\_\_\_

b) Há quantos parágrafos no texto? 11 parágrafos

\_\_\_\_\_\_\_\_\_\_\_\_\_\_\_\_\_\_\_\_\_\_\_\_\_\_\_\_\_\_\_\_\_\_\_\_\_\_\_\_\_\_\_\_\_\_\_\_\_\_\_\_\_\_\_\_\_\_\_\_\_\_\_\_

c) Qual era o desejo do príncipe? Ele queria se casar com uma princesa
de sangue real.

\_\_\_\_\_\_\_\_\_\_\_\_\_\_\_\_\_\_\_\_\_\_\_\_\_\_\_\_\_\_\_\_\_\_\_\_\_\_\_\_\_\_\_\_\_\_\_\_\_\_\_\_\_\_\_\_\_\_\_\_\_\_\_\_

\_\_\_\_\_\_\_\_\_\_\_\_\_\_\_\_\_\_\_\_\_\_\_\_\_\_\_\_\_\_\_\_\_\_\_\_\_\_\_\_\_\_\_\_\_\_\_\_\_\_\_\_\_\_\_\_\_\_\_\_\_\_\_\_

d) Por que a rainha ordenou que a criada colocasse um grão de ervilha na
cama da princesa? Para se certificar de que a moça era uma princesa de
verdade.

\_\_\_\_\_\_\_\_\_\_\_\_\_\_\_\_\_\_\_\_\_\_\_\_\_\_\_\_\_\_\_\_\_\_\_\_\_\_\_\_\_\_\_\_\_\_\_\_\_\_\_\_\_\_\_\_\_\_\_\_\_\_\_\_

\_\_\_\_\_\_\_\_\_\_\_\_\_\_\_\_\_\_\_\_\_\_\_\_\_\_\_\_\_\_\_\_\_\_\_\_\_\_\_\_\_\_\_\_\_\_\_\_\_\_\_\_\_\_\_\_\_\_\_\_\_\_\_\_

e) Como o rei, a rainha e o príncipe tiveram certeza de que ela era uma
princesa de verdade? Somente uma princesa verdadeira teria pele tão
sensível para sentir um grão de ervilha sob vinte colchões.

\_\_\_\_\_\_\_\_\_\_\_\_\_\_\_\_\_\_\_\_\_\_\_\_\_\_\_\_\_\_\_\_\_\_\_\_\_\_\_\_\_\_\_\_\_\_\_\_\_\_\_\_\_\_\_\_\_\_\_\_\_\_\_\_

\_\_\_\_\_\_\_\_\_\_\_\_\_\_\_\_\_\_\_\_\_\_\_\_\_\_\_\_\_\_\_\_\_\_\_\_\_\_\_\_\_\_\_\_\_\_\_\_\_\_\_\_\_\_\_\_\_\_\_\_\_\_\_\_

\_\_\_\_\_\_\_\_\_\_\_\_\_\_\_\_\_\_\_\_\_\_\_\_\_\_\_\_\_\_\_\_\_\_\_\_\_\_\_\_\_\_\_\_\_\_\_\_\_\_\_\_\_\_\_\_\_\_\_\_\_\_\_\_

\subsubsection{2. }\label{section-72}

Releia o trecho a seguir:

\begin{longtable}[]{@{}l@{}}
\toprule
\begin{minipage}[t]{0.97\columnwidth}\raggedright\strut
~ ~

A moça tanto afirmou que era uma princesa que a rainha pensou numa forma
de provar se o que ela dizia era verdade.\strut
\end{minipage}\tabularnewline
\bottomrule
\end{longtable}

\begin{itemize}
\item
  A palavra ela no trecho se refere
\end{itemize}

( x ) à moça ( ) à rainha

\subsubsection{3. }\label{section-73}

Siga o exemplo e escreva a forma verbal que corresponde à pessoa
destacada e complete as frases.

\begin{longtable}[]{@{}l@{}}
\toprule
\textbf{Eu dormi} sob a ervilha.\tabularnewline
\bottomrule
\end{longtable}

a) Nós \_\_dormimos\_\_\_\_\_\_\_\_\_\_\_\_\_\_ sob a ervilha.

b) Vocês \_\_\_\_dormiram\_\_\_\_\_\_\_\_\_\_ sob a ervilha.

c) Eles \_\_\_\_\_dormiram\_\_\_\_\_\_\_\_\_\_\_\_ sob a ervilha.

d) Você \_\_\_\_\_dormiu\_\_\_\_\_\_\_\_\_\_\_\_ sob a ervilha.

e) Ele \_\_\_\_\_\_\_dormiu\_\_\_\_\_\_\_ sob a ervilha

\subsubsection{4. }\label{section-74}

Substitua os termos destacados pelos pronomes pessoais adequados.

a) \textbf{O príncipe} casou com a princesa. Ele

\_\_\_\_\_\_\_\_\_\_\_\_\_\_\_\_\_\_\_\_\_\_\_\_\_\_\_\_\_\_\_\_\_\_\_\_\_\_\_\_\_\_\_\_\_\_\_\_\_\_\_\_\_\_\_\_\_\_\_\_\_\_\_\_

b) \textbf{O rei e a rainha} gostaram da princesa\textbf{.} Eles

\_\_\_\_\_\_\_\_\_\_\_\_\_\_\_\_\_\_\_\_\_\_\_\_\_\_\_\_\_\_\_\_\_\_\_\_\_\_\_\_\_\_\_\_\_\_\_\_\_\_\_\_\_\_\_\_\_\_\_\_\_\_\_\_

c) \textbf{O rei, a rainha e o príncipe} se olharam com surpresa. Eles

\_\_\_\_\_\_\_\_\_\_\_\_\_\_\_\_\_\_\_\_\_\_\_\_\_\_\_\_\_\_\_\_\_\_\_\_\_\_\_\_\_\_\_\_\_\_\_\_\_\_\_\_\_\_\_\_\_\_\_\_\_\_\_\_

d) \textbf{Eu e a princesa} casamos. Nós

\_\_\_\_\_\_\_\_\_\_\_\_\_\_\_\_\_\_\_\_\_\_\_\_\_\_\_\_\_\_\_\_\_\_\_\_\_\_\_\_\_\_\_\_\_\_\_\_\_\_\_\_\_\_\_\_\_\_\_\_\_\_\_\_

\subsubsection{5. }\label{section-75}

Releia as frases da atividade anterior, substituindo os termos
destacados pelos pronomes. A substituição dos pronomes altera os verbos?
Espera-se que os alunos percebam que não, os verbos permanecem, tendo em
vista que os pronomes substituem os termos destacados sem alterar as
pessoas do discurso.

\subsubsection{6. }\label{section-76}

Qual é a função dos termos que você usou na atividade 4?

\begin{quote}
( ) Indicar uma ação.

( x ) Substituir os nomes.
\end{quote}

\subsection{Treino}\label{treino-9}

\subsubsection{1.}\label{section-77}

\begin{quote}
(Fácil) Leia a charadinha a seguir.
\end{quote}

\begin{longtable}[]{@{}l@{}}
\toprule
\begin{minipage}[t]{0.97\columnwidth}\raggedright\strut
\begin{quote}
Por que a mula não usa chapéu? Porque ela não tem cabeça.
\end{quote}\strut
\end{minipage}\tabularnewline
\bottomrule
\end{longtable}

\begin{quote}
A palavra ``ela'' se refere a

(A) ``chapéu''.

(B) ``cabeça''.

(C) ``mula''.

(D) ``tem''.

Saeb D2 - Inferir uma afirmação implícita num texto.

BNCC Não há correspondência.

(A) Incorreta. A palavra ``ela'' não pode retomar ``chapéu''.

(B) Correta. A palavra ``ela'' não pode retomar ``cabeça''.

(C) Correta. O pronome ``ela'' está substituindo a palavra ``mula'',

(D) Incorreta. Pronomes retomam substantivos, e não verbos.
\end{quote}

\subsubsection{2. }\label{section-78}

\begin{quote}
(Médio) Leia a campanha publicitária a seguir.

\includegraphics[width=5.20833in,height=7.05556in]{media/image35.jpeg}

GOVERNO DE OURINHOS. Vigilância sanitária alerta frentistas sobre a
campanha educativa ``Não passe do limite! Complete o tanque só até o
automático''. Disponível em:
\textless{}www.ourinhos.sp.gov.br/noticia/1910/vigilancia-sanitaria-alerta-frentistas-sobre-a-campanha-educativa-nao-passe-do-limite-complete-o-tanque-so-ate-o-automatico/\textgreater{}.
Acesso em: 25 fev. 2023.

Na frase ``Esta ação evita'', a palavra ``esta'' está se referindo

(A) ao ``desperdício de combustível''.

(B) a ``não passe do limite''.

(C) a ``complete o tanque até o automático''.

(D) a ``danos à saúde do consumidor''.

Saeb D2 - Inferir uma afirmação implícita num texto.

BNCC Não há correspondência.

(A) Incorreta. O ``desperdício de combustível'' é uma das consequências
a serem evitadas com essa ação.

(B) Incorreta. O pronome não tem relação com essa frase.

(C) Correta. O pronome demonstrativo ``esta'', refere-se a ``completar o
tanque só até o automático''.

(D) Incorreta. O pronome não se refere a ``danos à saúde''.
\end{quote}

\subsubsection{3. }\label{section-79}

(Difícil) Leia o trecho da reportagem sobre pais e filhos.

\begin{quote}
\textbf{Assando pães e pizzas, pais e filhos descobrem prazer de
cozinhar juntos}

A vida de Sofia Santos Vasconcelos, 10, era diferente e quase sem graça,
como ela diz. Entre receitas e pratos, podia, no máximo, ajudar com
colheradas de açúcar e farinha. Com a quarentena e mais tempo ao lado da
mãe, tudo mudou da água para o vinho.

{[}...{]}

Tudo começou comigo e meu marido pensando em como ocupar o tempo dela.
Um dia vi uma receita muito legal de pizza com farinha de amêndoas.
Resolvemos fazer'', lembra a mãe. ``Ficou maravilhosa, e passei a
incentivá-la a buscar mais''.

FRANCO, Marcella. Assando pães e pizzas, pais e filhos descobrem prazer
de cozinhar juntos.

Disponível em:
https://www1.folha.uol.com.br/folhinha/2020/05/assando-paes-e-pizzas-pais-e-filhos-descobrem-prazer-de-cozinhar-juntos.shtml.
Acesso em: 25 fev. 2023.

No trecho ``passei a incentivá-la a buscar mais'', ``la'' está sendo
usado no lugar

da palavra

(A) ``receita''.

(B) ``mãe''.

(C) ``Sofia''.

(D) ``maravilhosa''.

Saeb D2 - Inferir uma afirmação implícita num texto.

BNCC Não há correspondência.

(A) Incorreta. O pronome não está retomando ``receita''.

(B) Incorreta. O pronome não está retomando ``mãe''.

(C) Correta. ``la'' está substituindo ``Sofia''.

(D) Incorreta. O pronome não pode retomar um substantivo.
\end{quote}

\section{Simulado 1}\label{simulado-1}

\subsubsection{1. }\label{section-80}

\begin{quote}
Leia o trecho da notícia.

\textbf{Peça primorosa da Cia. Delas, ``Maria e os Insetos''; retrata
cientista como heroína}

Depois de ``Mary e os Monstros Marinhos'', a Cia. Delas volta com o tema
da invisibilidade das mulheres na ciência em ``Maria e os Insetos''.
Dirigida por uma das integrantes do grupo, Thaís Medeiros, a peça marca
a maturidade da companhia de mulheres, que estreou em 2001 trazendo o
universo de Clarice Lispector para crianças.

Com gestos coreografados que bem desenham mundos interiores​, as atrizes
Fernanda Castello Branco, Julia Ianina e Paula Weinfeld se revezam de
modo dinâmico no papel da protagonista, Maria Sibylla Merian,
ilustradora científica nascida na Alemanha, em 1647 --- uma mulher que
foi tão pioneira quanto esquecida em tudo o que fez.

ROMEU, Gabriela. Peça primorosa da Cia. Delas, ``Maria e os Insetos.
Disponível em:
https://guia.folha.uol.com.br/crianca/2020/02/peca-primorosa-da-cia-delas-maria-e-os-insetos-retrata-cientista-como-heroina.shtml.
Acesso em: 26 fev. 2023.

\textbf{primorosa}: encantadora.

O tema do texto é

(A) a Cia. Delas.

(B) a peça ``Maria e os Insetos''.

(C) o tema ``Mary e os monstros marinhos''.

(D) o grupo de Thaís Medeiros.

Saeb D1 - Localizar informações num texto.

BNCC EF35LP03: Identificar a ideia central do texto, demonstrando
compreensão global.

(A) Incorreta. Cia. Delas são os criadores da peça.

(B) Correta. A notícia trata em relação à peça ``Maria e os Insetos'',
criada pela Cia. Delas.

(C) Incorreta. ``Mary e os monstros marinhos'' é uma das peças criada
pela mesma companhia.

(D) Incorreta. Thaís Medeiros faz parte da Cia. Delas, grupo que criou a
peça ``Maria e os insetos'', tema da notícia.
\end{quote}

\subsubsection{2. }\label{section-81}

O texto a seguir conta a história de um acordo feito entre quatro
animais. Leia-o.

\textbf{O leão, a vaca, a cabra e a ovelha}

Um leão, uma vaca, uma cabra e uma ovelha combinaram de caçar juntos e
repartir o que conseguissem. {[}\ldots{}{]} logo, repartiram a carne em
quatro partes. o leão se apossou da primeira parte, dizendo:

--- esta é minha, como combinamos.

O LEÃO, a vaca, a cabra e a ovelha. Disponível em:
\textless{}www.dominiopublico.gov.br/download/texto/me000589.pdf\textgreater{}.
Acesso em: 27 fev. 2023.

Com quem ficou a primeira parte carne? Marque a alternativa correta.

\begin{quote}
(A) Com a ovelha.

(B) Com a cabra.

(C) Com a vaca.

(D) Com o leão.

Saeb D2 - Inferir uma afirmação implícita num texto.

BNCC EF15LP03: Localizar informações explícitas em textos.

(A) Incorreta. A ovelha não ficou com a minha primeira parte, foi o
leão.

(B) Incorreta. A cabra não ficou com a primeira parte da carne.

(C) Incorreta. A vaca não ficou com a primeira parte da carne.

(D) Correta. O texto deixa explícito que o animal que ficou com a
primeira parte da carne foi o leão, como visto em: ``o leão se apossou
da primeira parte''.
\end{quote}

\subsubsection{3. }\label{section-82}

\begin{quote}
Viagens de Gulliver é um livro em que se relata as viagens da personagem
principal, Gulliver. Leia um trecho.

\textbf{Viagens de Gulliver}

{[}\ldots{}{]} Desconheço qual tivesse sido a sorte dos meus
companheiros de lancha, nem dos que se salvaram do escolho,
{[}\ldots{}{]} mas desconfio que pereceram todos; quanto a mim nadei ao
acaso e fui levado para a terra pelo vento e pela maré. De vez em quando
estendia as pernas para ver se encontrava o fundo; por fim, estando
quase exausto, tomei pé. Por então, o temporal amainara {[}\ldots{}{]}
caminhei perto de meia légua pelo mar, antes que pusesse pé em terra
firme.

SWIFT, Jonathan. Viagens de Gulliver. Disponível em:

www.dominiopublico.gov.br/download/texto/ph000001.pdf. Acesso em: 27
fev. 2023.

\textbf{escolho}: rocha

\textbf{pereceram}: morreram

\textbf{amainara}: acalmara

Com base na leitura do trecho, a lancha

(A) colidiu com uma rocha e afundou.

(B) levou a personagem à praia.

(C) tinha só um tripulante.

(D) navegava em um dia de tempo bom.

Saeb D2 - Inferir uma afirmação implícita num texto.

BNCC EF35LP04: Inferir informações implícitas nos textos lidos.

(A) Correta. A lancha que levava a tripulação colidiu com uma rocha, e
afundou, o que pode ser inferido pela leitura de ``mas desconfio que
pereceram todos; quanto a mim nadei ao acaso e fui levado para a terra
pelo vento e pela maré''.

(B) Incorreta. A personagem chegou a nado até o local, visto que a
lancha afundou.

(C) Incorreta. Não tinha somente um tripulante na lancha.

(D) Incorreta. A lancha navegada em um dia de chuva.
\end{quote}

\subsubsection{4. }\label{section-83}

\begin{quote}
O conto ``Os sete corvos'' narra a história de um casal que teve os sete
filhos transformados em corvos. Leia o início do conto.

\textbf{Os sete corvos}

Era uma vez um homem que tinha sete filhos, todos meninos, e vivia
suspirando por uma menina. Afinal, um dia, a mulher anunciou-lhe que
estava mais uma vez esperando criança.

No tempo certo, quando ela deu à luz, veio uma menina. Foi imensa a
alegria deles. Mas, ao mesmo tempo, ficaram muito preocupados, pois a
recém-nascida era pequena e fraquinha, e precisava ser batizada com
urgência.

Então, o pai mandou um dos filhos ir bem depressa até a fonte e trazer
água para o batismo. O menino foi correndo e, atrás dele, seus seis
irmãos. Chegando lá, cada um queria encher o cântaro primeiro; na
disputa, o cântaro caiu na água e desapareceu.

{[}\ldots{}{]}

IRMÃOS GRIMM. \textbf{Os sete corvos}. Disponível em:

www.dominiopublico.gov.br/download/texto/me001614.pdf. Acesso em: 26
fev. 2023.

\textbf{cântaro}: jarra para guardar água.

A expressão ``estava mais uma vez esperando criança'' significa que

(A) os filhos não tinham voltado.

(B) o pai e a mãe esperavam mais um menino.

(C) a mulher estava grávida.

(D) a mulher não sabia onde era a fonte.

Saeb D5 - Inferir o sentido de uma palavra ou expressão a partir do
contexto imediato.

BNCC EF35LP05 - Inferir o sentido de palavras ou expressões
desconhecidas em textos, com base no contexto da frase ou do texto.

(A) Incorreta. Essa informação não se refere à expressão destacada.

(B) Incorreta. No texto, verifica-se que os pais estavam, dessa vez,
esperando uma menina.

(C) Correta. Infere-se que, ao estar ``esperando criança'', a mãe estava
grávida novamente.

(D) Incorreta. Essa informação não se refere à expressão destacada.
\end{quote}

\section{Simulado 2}\label{simulado-2}

\subsubsection{1. }\label{section-84}

\begin{quote}
Leia os dois textos a seguir, que apresentam trechos finais de duas
narrativas.

\textbf{I}

Chapeuzinho vermelho

{[}\ldots{}{]} --- Vovó, como são grandes os seus dentes!

--- É para te comer!

E assim dizendo, o malvado lobo se atirou sobre Chapeuzinho Vermelho e a
comeu.

PERRAULT, Charles. Chapeuzinho vermelho. Disponível em:

\href{http://www.dominiopublico.gov.br/download/texto/me000589.pdf}{www.dominiopublico.gov.br/download/texto/me000589.pdf}.
Acesso em: 27 fev. 2023.

\textbf{II}

\textbf{Chapeuzinho vermelho}

{[}\ldots{}{]} --- Oh, vovozinha, que boca enorme você tem!

--- É para engolir você melhor!!!

Assim dizendo, o lobo mau deu um pulo e, num movimento só, comeu a pobre
Chapeuzinho Vermelho. {[}\ldots{}{]}

Algumas horas mais tarde, um caçador passou em frente à casa da vovó,
ouviu o barulho e pensou: ``Olha só como a velhinha ronca! Estará
passando mal!? Vou dar uma espiada.''

Abriu a porta, chegou perto da cama e\ldots{} quem ele viu?

O lobo, que dormia como uma pedra, com uma enorme barriga parecendo um
grande balão!

O caçador ficou bem satisfeito. Há muito tempo estava procurando esse
lobo, que já matara

muitas ovelhas e cordeirinhos.

--- Afinal você está aqui, velho malandro! Sua carreira terminou. Já vai
ver! {[}\ldots{}{]}

IRMÃOS GRIMM. Chapeuzinho vermelho. Disponível em:

www.dominiopublico.gov.br/download/texto/me000589.pdf. Acesso em: 27
fev. 2023.

Os dois textos estabelecem uma relação entre si, sendo que o texto II
apresenta características do gênero textual

(A) reconto, porque traz novas informações à primeira versão.

(B) anedota, porque tem o objetivo de provocar o riso do leitor.

(C) poema visual, porque a história é narrada por meio de imagens.

(D) fábula, porque apenas as personagens animais falam.

Saeb D3 - Estabelecer relação entre informações num texto ou entre
diferentes textos.

BNCC EF35LP29: Identificar, em narrativas, cenário, personagem central,
conflito gerador, resolução e o ponto de vista com base no qual
histórias são narradas, diferenciando narrativas em primeira e terceira
pessoas.

(A) Correta. A versão de ``Chapeuzinho vermelho'' dos Irmãos Grimm
mostra um novo final para a narrativa tradicional, na qual a história
termina quando o Lobo Mau come a Chapeuzinho vermelho, caracterizando um
reconto.

(B) Incorreta. O texto II não apresenta uma construção de humor.

(C) Incorreta. Não há características de poema visual.

(D) Incorreta. O texto apresenta personagens humanos que falam, como a
Chapeuzinho Vermelho e o caçador.
\end{quote}

\subsubsection{2. }\label{section-85}

\begin{quote}
Leia o trecho que traz um diálogo entre os personagens Lola, o Gerente e
Gouveia, da peça teatral de Artur Azevedo, A capital federal.

\textbf{A capital federal}

{[}\ldots{}{]}

--- Cena VI ---

{[}\ldots{}{]}

Lola (Entrando.) --- Então? Estou esperando há uma hora!...

{[}\ldots{}{]}

Lola --- {[}\ldots{}{]} O que eu quero é falar ao Gouveia!

O Gerente --- Já o mandei chamar. (Vendo o Gouveia que desce a escada.)
E ele aí vem descendo a escada. {[}\ldots{}{]} (Sai.)

Gouveia (Que tem descido.) --- {[}\ldots{}{]} Não te disse que não me
procurasses aqui? Este hotel...

AZEVEDO, Artur. A capital federal. Disponível em:

www.dominiopublico.gov.br/download/texto/bn000020.pdf. Acesso em: 27
fev. 2023.

Peças teatrais devem ser encenadas por atores em teatros. Elas são
parecidas com

(A) entrevista.

(B) reportagem.

(C) roteiro de cinema.

(D) jornal de rádio.

Saeb D3 - Estabelecer relação entre informações num texto ou entre
diferentes textos.

BNCC EF35LP29: Identificar, em narrativas, cenário, personagem central,
conflito gerador, resolução e o ponto de vista com base no qual
histórias são narradas, diferenciando narrativas em primeira e terceira
pessoas.

(A) Incorreta. Entrevistas são gêneros informativos que podem ser
veiculados por variados meios de comunicação e têm entrevistador e
entrevistado.

(B) Incorreta. Reportagens são textos informativos veiculados por vários
meios de comunicação, os quais têm informações reais e atuais.

(C) Correta. Roteiros de cinema, como as peças teatrais, são escritos
para serem encenados por atores em filmes.

(D) Incorreta. Jornal radiofônico é aquele que transmite as notícias da
atualidade por meio de uma rádio.
\end{quote}

\subsubsection{3. }\label{section-86}

\begin{quote}
Leia o trecho a seguir que mostra uma entrevista com um químico que
trabalha no Centro de Tecnologia de Chocolate da Nestlé, na Inglaterra.

\textbf{Doutor chocolate}

\textbf{CHC:\emph{~Para muitos, o senhor tem o melhor emprego do mundo.
O senhor come chocolate todos os dias?}}\\
Josélio Vieira:~Além de ser o melhor emprego do mundo, trabalhar com
chocolate é fascinante do ponto de vista científico. As tecnologias
estão cada vez mais sofisticadas! Não digo que como chocolates todos os
dias, mas muitas das reuniões envolvem degustar várias amostras.
{[}...{]}

\emph{\textbf{Como é seu dia a dia?}}\\
Sou responsável pela busca de novas tecnologias e por fornecer
direcionamento e conselhos técnicos a projetos de desenvolvimento de
produtos. Meu dia a dia é repleto de reuniões com técnicos internos e
externos. {[}...{]}

Doutor chocolate. \textbf{CHC}. Disponível em:
\url{https://chc.org.br/doutor-chocolate/}. Acesso em: 27 fev. 2023.

Na entrevista, os trechos em negrito são as

(A) perguntas do entrevistador.

(B) respostas do entrevistado.

(C) informações sobre o chocolate.

(D) falas dos personagens.

Saeb D4 - Identificar o tema central do texto.

BNCC EF35LP16: Identificar e reproduzir, em notícias, manchetes, lides e
corpo de notícias simples para público infantil e cartas de reclamação
(revista infantil), digitais ou impressos, a formatação e diagramação
específica de cada um desses gêneros, inclusive em suas versões orais.

.

(A) Correta. Em entrevistas, as perguntas feitas pelo entrevistador
normalmente são apresentadas em negrito para que possam ser
diferenciadas das respostas do entrevistado.

(B) Incorreta. As respostas do entrevistado normalmente são apresentadas
da forma normal, sem estar em negrito.

(C) Incorreta. Os trechos em negrito não mostram esse assunto.

(D) Incorreta. Essa é uma característica de um texto narrativo e não de
um texto informativo como a entrevista.
\end{quote}

\subsubsection{4. }\label{section-87}

\begin{quote}
Observe como fazer uma peteca de palha de banana no texto a seguir.

\textbf{Descubra como fazer uma peteca usando palha de banana}

Não é só de palha de milho que vivem as petecas pelo Brasil. Em Abadia
(Minas Gerais), as meninas usam a casca da bananeira para
confeccioná-las.

Siga os passos {[}\ldots{}{]}:

1. Pegue um pouco de cascas da bananeira.

2. Dobre uma parte da casca até que fique com um volume pequeno.

3. Corte um outro pedaço da casca e embrulhe numa trouxinha o volume da
casca que está

dobrado.

4. Amarre com fios da própria casca ou barbante para amarrar a ponta da
peteca.

5. Depois de amarrada, coloque penas {[}\ldots{}{]} na ponta.

E pronto.

MEIRELLES, Renata. Descubra como fazer uma peteca usando palha de
banana. EBC. Disponível

em:
www.ebc.com.br/infantil/2016/11/descubra-como-fazer-uma-peteca-usando-palha-de-banana.

Acesso em: 27 fev. 2023 (Adaptado).

confeccioná-las: fabricar, produzir, fazer.

Qual é a finalidade do texto?

(A) Explicar que as petecas foram inventadas em Abadia, Minas Gerais.

(B) Instruir como se faz uma peteca com a casca da bananeira.

(C) Mostrar como se faz uma peteca de palha de milho.

(D) Explicar a origem das petecas de palha de milho.

Saeb D7 - Relacionar, na compreensão do texto, informações textuais com
conhecimentos de senso comum.

BNCC EF03LP11: Ler e compreender, com autonomia, textos injuntivos
instrucionais (receitas, instruções de montagem etc.), com a estrutura
própria desses textos (verbos imperativos, indicação de passos a ser
seguidos) e mesclando palavras, imagens e recursos gráfico- visuais,
considerando a situação comunicativa e o tema/assunto do texto.

(A) Incorreta. A finalidade do texto, além de ensinar a se fazer petecas
de palha de bananeira, é mostrar que a origem delas é de Abadia, Minas
Gerais, e não explicar a invenção do brinquedo.

(B) Correta. Pela leitura do trecho, pode-se compreender que a
finalidade do texto é instruir o leitor a montar uma peteca a partir da
casca de bananeira, o que pode ser observado pelo uso das instruções
passo a passo.

(C) Incorreta. O texto mostra como se faz uma peteca com a palha de
bananeira, e não de palha de milho.

(D) Incorreta. O texto apresenta a origem e o modo de se confeccionar
uma peteca de palha de banana, em oposição à peteca de palha de milho.
\end{quote}

\section{Simulado 3}\label{simulado-3}

\subsubsection{1. }\label{section-88}

\begin{quote}
Leia o texto a seguir.

\textbf{O assustador livro vermelho}

No final de 2014, foi publicado um dos livros mais aterrorizantes que
nós já lemos: o livro vermelho. Não é uma história de terror comum, com
monstros ou facas, e sim a lista dos animais brasileiros ameaçados de
extinção. Pesquisadores estudaram 12.256 animais de todos os tipos e
viram que quase uma espécie em cada dez está ameaçada. {[}...{]}

No Brasil, a instituição responsável por realizar esta pesquisa é~o
Instituto Chico Mendes de Conservação da Biodiversidade~(ICMBio). Um
trabalho \textbf{ingrato}, não acha? Mas muito importante. Afinal,
conhecendo exatamente as ameaças à nossa fauna, podemos pensar em
soluções para preservá-la.

O assustador livro vermelho. CHC. Disponível em:
\url{https://chc.org.br/o-assustador-livro-vermelho/}. Acesso em: 28
fev. 2023.

A palavra em destaque é formada pela junção de duas partes. A mesma
formação de palavras ocorre na palavra

(A) triste.

(B) felizes.

(C) informal.

(D) inveja.

Saeb D5 - Inferir o sentido de uma palavra ou expressão a partir do
contexto imediato.

BNCC EF03LP10 - Reconhecer prefixos e sufixos produtivos na formação de
palavras derivadas de substantivos, de adjetivos e de verbos,
utilizando-os para compreender palavras e para formar novas palavras.

(A) Incorreta. A palavra ``triste'' não é formada pela junção de duas
partes como acontece em ``ingrato''.

(B) Incorreta. A palavra ``felizes'' não é formada pela junção de duas
partes, como acontece em ``ingrato''.

(C) Correta. A palavra destacada é feita pela junção do prefixo ``in''
mais a palavra ``grato''. Assim, o mesmo ocorre em ``informal'' em que
se tem a junção de ``in'' com a palavra ``formal'', dando a ela um novo
sentido.

(D) Incorreta. A palavra ``inveja'' não é formada pela junção de duas
partes, como acontece na palavra ``ingrato''. O aluno que marca essa
alternativa reparou apenas no começo de ambas as palavras, que
apresentam ``in'', sem compreender que na palavra ``inveja'' não há a
junção de ``in'' mais ``veja''.
\end{quote}

\subsubsection{2. }\label{section-89}

\begin{quote}
O trecho a seguir é o início do conto \textbf{Ali Babá e os quarenta
ladrões}, leia para compreender como o personagem convivia com todos.

\textbf{Ali Babá e os quarenta ladrões}

Numa distante cidade do Oriente, vivia um homem bom e justo, chamado Ali
Babá.

Ali Babá era muito pobre. Morava numa tenda, entre um vasto deserto e um
grande oásis. Para sustentar a mulher, Samira, e os quatro filhos, Ali
Babá oferecia seus serviços às caravanas de mercadores que passavam por
ali. Estava sempre pronto para cuidar dos camelos, lavá-los, escová-los
e dar-lhes água e alimento.

Os ricos comerciantes já conheciam Ali Babá e gostavam muito de seu
serviço. Ele sempre cobrava o preço justo pelo trabalho, porém, muitas
vezes, os mercadores davam-lhe mais, pois sabiam que ele vivia em
dificuldades.

{[}\ldots{}{]}

ALI BABÁ e os quarenta ladrões. Disponível em:
www.dominiopublico.gov.br/download/texto/me001614.pdf. Acesso em: 28
fev. 2023.

\textbf{oásis}: pequena região fértil no deserto, com presença de água.

\textbf{caravanas}: grupos de mercadores viajantes.

Considerando-se o trecho, Ali Babá

\protect\hypertarget{_Hlk128465624}{}{}(A) trabalhava para os
comerciantes ricos.

(B) cuidava dos camelos de Samira e seus quatro filhos.

(C) cobrava um preço mais alto dos mercadores.

(D) entregava água e alimento para os mercadores venderem.

Saeb D2 - Inferir uma afirmação implícita num texto.

BNCC EF35LP04 - Inferir informações implícitas nos textos lidos.

(A) Correta. Ali Babá trabalhava para comerciantes ricos, informação
implícita no trecho ``Os ricos comerciantes já conheciam Ali Babá e
gostavam muito de seu serviço''.

(B) Incorreta. Os camelos eram dos ricos comerciantes.

(C) Incorreta. Os mercadores davam dinheiro a mais para Ali Babá por
terem simpatia com ele.

(D) Incorreta. A água e o alimento eram para os camelos dos viajantes.
\end{quote}

\subsubsection{3. }\label{section-90}

\begin{quote}
Leia o trecho que conta a história de um reino e seus moradores.

\textbf{No reino das letras felizes}

Num lugar muito distante, existia um reino silencioso, habitado apenas
por letras. Elas eram muito desunidas: vivia cada uma para si, e nunca
se reuniam para formar uma palavra sequer.

HECK, Lenira Almeida. No reino das letras felizes. Disponível em:

www.dominiopublico.gov.br/download/texto/eu00005a.pdf. Acesso em: 28
fev. 2023.

No reino silencioso moravam

(A) letras felizes.

(B) letras desunidas.

(C) animais silenciosos.

(D) palavras formadas.

Saeb D1 - Localizar informações num texto.

BNCC EF15LP03 - Localizar informações explícitas em textos.

(A) Incorreta. O reino era das letras felizes.

(B) Correta. Os habitantes do ``reino silencioso'' eram letras
desunidas.

(C) Incorreta. Não há animais mencionados.

(D) Incorreta. As palavras não moravam nesse reino.

\textbf{4.}

O conto ``O avô e o netinho'', de Figueiredo Pimentel, conta a história
de tio Benedito.

\textbf{O avô e o netinho}

{[}...{]}

Bastante velho já, fatigado por uma longa existência de trabalhos e
canseiras, exausto de forças e doente de velhice -- porque a velhice é,
também, uma doença -- estava tio Benedito, o bom e estimado velhote tio
Benedito: oitenta anos pesavam-lhe às costas, como um grande fardo que
ele a custo carregasse.

Na sua mocidade, e mesmo durante parte da velhice, ninguém trabalhara
mais que ele, honesto sempre, mourejando, dia e noite, para sustento de
sua família.

Não podendo fazer serviço algum, alquebrado pela idade, veio morar em
casa de Augusto, seu filho mais moço, já com um filhinho de três para
quatro anos, o pequenino e interessante Luís, vivo e esperto como
poucos.

PIMENTEL, Figueiredo. O avô e o netinho. Disponível em:

www.dominiopublico.gov.br/download/texto/bn000137.pdf. Acesso em: 28
fev. 2023.

fatigado: cansado

mourejando: trabalhando muito

alquebrado: abatido; cansado

A ideia central do trecho apresentado é

(A) mostrar a família do tio Benedito.

(B) mostrar o que é a velhice.

(C) explicar o título do conto.

(D) apresentar o tio Benedito.

Saeb D4 - Identificar o tema central do texto.

BNCC EF35LP03 - Identificar a ideia central do texto, demonstrando
compreensão global.

(A) Incorreta. Só se mencionou a família para mostrar com quem ele vai
morar.

(B) Incorreta. A intenção é mostrar quem é a personagem; as descrições
sobre a velhice são feitas com base na situação da personagem.

(C) Incorreta. O neto Luís é citado, mas não há esclarecimentos sobre a
relação que os dois têm para o conto ter o título ``o avô e o netinho''.

(D) Correta. Os trechos selecionados do conto apresentam o personagem
tio Benedito.
\end{quote}

\section{Simulado 4}\label{simulado-4}

\subsubsection{1. }\label{section-91}

\begin{quote}
Leia o cartaz referente a uma forma de se prevenir contra o Covid-19
para responder à questão.

https://diamantina.mg.gov.br/coronavirus-covid-19-em-diamantina-2/

\includegraphics[width=4.68750in,height=4.68750in]{media/image36.png}

Glossário:

\textbf{consumir}: usar.

\textbf{descarte}: jogar no lixo.

PREFEITURA DE DIAMANTINA. Coronavírus -- Covid-19 em Diamantina.
Disponível em:

https://diamantina.mg.gov.br/coronavirus-covid-19-em-diamantina-2/.
Acesso em: 28 fev. 2023.

Com base na leitura do cartaz, as mãos devem ser lavadas

(A) duas vezes após receber a entrega.

(B) depois de usar o produto.

(C) ao descartar a embalagem no lixo.

(D) depois da entrega e antes de usar o produto.

Saeb D6 - Utilizar informações oferecidas por um glossário, verbete de
dicionário ou texto informativo na compreensão ou interpretação do
texto.

BNCC EF15LP03: Localizar informações explícitas em textos.

(A) Incorreta. Embora o cartaz indique que se lave as mãos duas vezes,
nessa ação não ocorre após receber a entrega, mas sim uma vez ao receber
a entrega e outra antes de consumir o produto.

(B) Incorreta. As mãos devem ser lavadas antes de consumir o produto, e
não depois.

(C) Incorreta. No cartaz, não há informações para lavar as mãos ao jogar
descartar a embalagem no lixo.

(D) Correta. Ao ler o cartaz da campanha, identifica-se que ela
incentiva e instrui o leitor a lavar as mãos duas vezes, após a pessoa
receber a entrega e antes de usar o produto.

\textbf{2.}

Leia o início do conto maravilhoso ``Os quatro irmãos espertos''.

\textbf{Os quatro irmãos espertos}

Era uma vez um pobre homem que tinha quatro filhos; muito lhe custou a
educá-los, mas enfim sempre o conseguiu. Quando já estavam crescidos,
disse-lhes:

``Meus ricos filhos, não posso de modo algum continuar a sustentá-los.
Precisam, pois, ir pelo mundo fora e aprender um ofício para que possam
ganhar a vida.''

Depois de lhes ter ainda dado a cada um meio quilo de pão e um bom
cajado, despediram-se do pai e saíram juntos da cidade. {[}\ldots{}{]}

OS QUATRO irmãos espertos. \textbf{Contos dos Irmãos Grimm}. Rio de
Janeiro: Livraria Garnier, 1932, p. 59. Disponível em:
https://digital.bbm.usp.br/handle/bbm/7812. Acesso em: 28 fev. 2023.

\textbf{Glossário}

ofício: trabalho.

A história é narrada

(A) por um dos irmãos da história.

(B) pelos quatro irmãos, personagens da história.

(C) por um narrador-observador.

(D) pelo personagem ``pobre homem''.

Saeb D6 - Utilizar informações oferecidas por um glossário, verbete de
dicionário ou texto informativo na compreensão ou interpretação do
texto.

BNCC EF35LP26: Ler e compreender, com certa autonomia, narrativas
ficcionais que apresentem cenários e personagens, observando os
elementos da estrutura narrativa: enredo, tempo, espaço, personagens,
narrador e a construção do discurso indireto e discurso direto.

(A) Incorreta. Não existe narrador-personagem.

(B) Incorreta. Os irmãos não narram a história.

(C) Correta. A história é apresentada por meio de um
narrador-observador.

(D) Incorreta. O narrador da história é observador.
\end{quote}

\subsubsection{3. }\label{section-92}

\begin{quote}
Os objetivos dos jornais são informar a população. Leia a capa do jornal
a seguir.

https://auniao.pb.gov.br/noticias/primeira-pagina/capa-26-02.2023

\includegraphics[width=4.47917in,height=8.00000in]{media/image37.png}

A UNIÃO. Capa: 26.02.2023. Disponível em:
https://auniao.pb.gov.br/noticias/primeira-pagina/capa-26-02.2023.
Acesso em: 28 fev. 2023.

O trecho ``Vândalos depredam patrimônio público para venda de
materiais'' é uma

(A) manchete de jornal.

(B) chamada de jornal.

(C) informação sobre esporte.

(D) notícia sobre saúde.

Saeb D3 - Estabelecer relação entre informações num texto ou entre
diferentes textos.

BNCC EF35LP16 - Identificar e reproduzir, em notícias, manchetes, lides
e corpo de notícias simples para público infantil e cartas de reclamação
(revista infantil), digitais ou impressos, a formatação e diagramação
específica de cada um desses gêneros, inclusive em suas versões orais.

(A) Correta. A manchete de uma notícia aparece sempre em negrito e com
letras maiores que o restante do texto.

(B) Incorreta. A chamada está logo após a manchete, em letras menores e
introduz aquilo que foi apresentado na manchete.

(C) Incorreta. A informação contida na manchete não é sobre esporte,
refere-se à \protect\hypertarget{_Hlk128473874}{}{}ação de vândalos que
depredaram patrimônio público para venda de materiais.

(D) Incorreta. Não se trata de uma notícia sobre saúde, e sim sobre a
ação de vândalos que depredaram patrimônio público para venda de
materiais.

\textbf{4.}

Observe as frutas e, com base em seus nomes, responda à questão.

\href{https://www.istockphoto.com/br/foto/ma\%C3\%A7\%C3\%A3-vermelha-com-folha-isolada-no-fundo-branco-gm185262648-19966694?phrase=ma\%C3\%A7\%C3\%A3}{https://www.istockphoto.com/br/foto/ma\%C3\%A7\%C3\%A3-vermelha-com-folha-isolada-no-fundo-branco-gm185262648-19966694?phrase=ma\%C3\%A7\%C3\%A3\includegraphics[width=2.98958in,height=2.98958in]{media/image38.jpeg}}

https://www.istockphoto.com/br/foto/mel\%C3\%A3o-honeydew-gm146890371-13998859?phrase=mel\%C3\%A3o

\includegraphics[width=3.33412in,height=2.64722in]{media/image39.jpeg}

https://mail.google.com/mail/u/0/?tab=rm\&ogbl\#search/EF03lp10/FMfcgxwHMGLSpFBBGCZGcQSzrSfLHRQZ?projector=1\&messagePartId=0.1

\includegraphics[width=3.76501in,height=2.67500in]{media/image40.jpeg}

A partir da aplicação do sufixo -eiro/-eira nos nomes das frutas, tem-se
o nome das árvores em que esses frutos nascem. Elas são,
respectivamente,

(A) maçãzeira, melãozeiro e mamãozeiro.

(B) maceiro, meleiro, mameiro.

(C) macieira, meloeiro, mamoeiro.

(D) maçãozeira, melãoeiro, mamãoeiro.

Saeb D4 - Identificar o tema central do texto.

BNCC EF03LP10: Reconhecer prefixos e sufixos produtivos na formação de
palavras derivadas de substantivos, de adjetivos e de verbos,
utilizando-os para compreender palavras e para formar novas palavras.

(A) Incorreta. O aluno acrescentou os sufixo -zeiro/-zeira em vez do
sufixo -eiro / -eira.

(B) Incorreta. O aluno retirou incorretamente alguns fonemas para formar
o nome das árvores.

(C) Correta. Maçã, melão e mamão, palavras relativas às imagens, formam:
macieira, meloeiro e mamoeiro.

(D) Incorreta. O aluno apenas acrescentou o sufixo às palavras, sem
entender a regra.
\end{quote}

\section{REFERÊNCIAS}\label{referuxeancias}

\begin{quote}
BRASIL. Ministério da Educação. \textbf{Base nacional comum curricular}:
educação é a base.

BRASIL. Ministério da Educação. \textbf{Pacto nacional pela
alfabetização na idade certa}. Disponível em:
\textless{}http://www.serdigital.com.br/gerenciador/clientes/ceel/material/149.pdf\textgreater{}.
Acesso em: fev. 2023.

KAUFMAN, Ana María; RODRÍGUEZ, María Helena. \textbf{Escola, leitura e
produção de textos}. Porto Alegre: Artmed, 1995.

KOCH, Ingedore G. Villaça. \textbf{Ler e escrever}: estratégias de
produção textual. São Paulo: Contexto, 2010.

LERNER, Delia. \textbf{Ler e escrever na escola}: o real, o possível e o
necessário. Porto Alegre: Artmed, 2002.

NÓBREGA, Maria José. \textbf{Ortografia}. São Paulo: Melhoramentos,
2013.
\end{quote}

\section{SITES}\label{sites}

\begin{quote}
\textbf{Ciência Hoje das Crianças}. Disponível em:
\textless{}http://chc.org.br/\textgreater{}. Acesso em: fev. 2023.

\textbf{DOMÍNIO PÙBLICO}. Disponível em:
\url{http://www.dominiopublico.gov.br/pesquisa/PesquisaObraForm.jsp}.
Acesso em: 28 fev. 2023.

\textbf{RÁDIOS EBC}. Disponível em: https://radios.ebc.com.br/. Acesso
em: 23 fev. 2023.

\textbf{PORTAL Teatro na escola}. Disponível em:
\url{https://www.teatronaescola.com/}. Acesso em: 28 fev. 2023.
\end{quote}
