\chapter{Respostas}
\pagestyle{plain}
\footnotesize

\pagecolor{gray!40}

\section*{Módulo 1 – Treino}

\begin{enumerate}
\item
SAEB: Inferir informações implícitas em textos. BNCC: EF35LP04 -- Inferir informações implícitas nos textos lidos. (A) Incorreta. O ratinho saiu do buraco para explorar o mundo. (B) Incorreta. Esse não é o local em que se passa a história. (C) Incorreta. O texto menciona que o ratinho admirou a correnteza dos rios, e não que a história se passa em um rio com correnteza. (D) Correta. O ratinho saiu do buraco para explorar e acabou no quintal de uma casa de roça.

\item
Saeb: Inferir o sentido de palavras ou expressões em textos.
BNCC: EF35LP05 -- Inferir o sentido de palavras ou expressões
desconhecidas em textos, com base no contexto da frase ou do texto.
(A) Correta. A palavra ``destemido'' pode ser usada como sinônimo de
``valente'', pois é um adjetivo próprio de pessoas ou animais
guerreiros, audazes, ousados, corajosos.
(B) Incorreta. A palavra ``covarde'' é antônima da palavra ``valente''.
(C) Incorreta. A palavra ``inteligente'' é sinônima da palavra
``sabido''.
(D) Incorreta. A palavra ``doido'' teria como sinônimo ``maluco'' ou
``biruta'', por exemplo, mas não é adequado ao contexto.

\item
Saeb: Reconhecer em textos o significado de palavras derivadas a partir de seus afixos.
BNCC: EF03LP10 -- Reconhecer prefixos e sufixos produtivos na formação de
palavras derivadas de substantivos, de adjetivos e de verbos,
utilizando-os para compreender palavras e para formar novas palavras.
(A) Incorreta. A palavra ``preguiçosa'' não é primitiva.
(B) Correta. A palavra ``preguiçosa'' é derivada, uma vez que foi
formada a partir da palavra primitiva ``preguiça''.
(C) Incorreta. A palavra apresenta relação direta com a palavra
``preguiça'' (primitiva).
(D) Incorreta. A palavra é derivada diretamente de ``preguiça''.
\end{enumerate}

\section*{Módulo 2 – Treino}

\begin{enumerate}
\item
SAEB: Reconhecer diferentes gêneros textuais.
BNCC: EF35LP24 -- Identificar funções do texto dramático (escrito para ser
encenado) e sua organização por meio de diálogos entre personagens e
marcadores das falas das personagens e de cena.
(A) Incorreta. O texto não explica acontecimentos misteriosos ou
sobrenaturais.
(B) Incorreta. O texto não está organizado em versos.
(C) Correta. O texto apresenta estrutura do texto teatral, marcada pela
descrição dos movimentos de cena (texto secundário indicado entre
parênteses) e pela organização do enredo em falas.
(D) Incorreta. Nas fábulas, não há descrição de movimentações de cena.

\item
SAEB: Identificar as marcas de organização de textos dramáticos.
BNCC: EF35LP24 -- Identificar funções do texto dramático (escrito para
ser encenado) e sua organização por meio de diálogos entre personagens e
marcadores das falas das personagens e de cena.
(A) Correta. Roteiros cinematográficos, assim como as peças teatrais,
são elaborados para serem encenados por atores em filmes, com marcas
no texto que são tipicamente feitas para esse fim.
(B) Incorreta. Entrevista consiste em gênero informativo que apresenta
entrevistador e entrevistado.
(C) Incorreta. Reportagens são textos informativos veiculados por
diversos meios de comunicação e têm como conteúdo informações reais e
atuais.
(D) Incorreta. Jornal de rádio transmite as notícias da atualidade por
meio de emissões radiofônicas.

\item
SAEB: Identificar elementos constitutivos de textos narrativos.
BNCC: EF35LP26 -- Ler e compreender, com certa autonomia, narrativas
ficcionais que apresentem cenários e personagens, observando os
elementos da estrutura narrativa: enredo, tempo, espaço, personagens,
narrador e a construção do discurso indireto e discurso direto.
(A) Incorreta. Não existe ``eu'' ou ``nós'' no conto para se afirmar
que o narrador participe da história.
(B) Correta. O narrador não realiza as ações da história, mas narra os
acontecimentos.
(C) Incorreta. O narrador não é um personagem da história.
(D) Incorreta. O narrador não pratica nenhuma ação narrada.
\end{enumerate}

\section*{Módulo 3 – Treino}

\begin{enumerate}
\item
SAEB: Analisar os efeitos de sentido decorrentes do uso da pontuação.
BNCC: EF03LP07 -- Identificar a função na leitura e usar na escrita ponto
final, ponto de interrogação, ponto de exclamação e, em diálogos
(discurso direto), dois-pontos e travessão.
(A) Incorreta. Frases declarativas terminam com ponto final.
(B) Correta. A frase apresenta ponto de exclamação.
(C) Incorreta. A frase não é uma pergunta.
(D) Incorreta. A frase não expressa negação.

\item
SAEB: Reconhecer os usos da pontuação.
BNCC: EF03LP07 -- Identificar a função na leitura e usar na escrita ponto
final, ponto de interrogação, ponto de exclamação e, em diálogos
(discurso direto), dois-pontos e travessão.
(A) Incorreta. A vírgula não serve para indicar diálogos, mas para
separar elementos dentro de uma mesma frase.
(B) Incorreta. O ponto final é um sinal de pontuação que encerra o
período.
(C) Correta. O travessão é um sinal de pontuação usado especialmente no
início de cada fala no discurso direto.
(D) Incorreta. O ponto de interrogação é usado para indicar uma
pergunta.

\item
SAEB: Analisar elementos constitutivos de gêneros textuais diversos.
BNCC EF03LP16 -- Identificar e reproduzir, em textos injuntivos
instrucionais (receitas, instruções de montagem, digitais ou impressos),
a formatação própria desses textos (verbos imperativos, indicação de
passos a ser seguidos) e a diagramação específica dos textos desses
gêneros (lista de ingredientes ou materiais e instruções de execução --
"modo de fazer").
(A) Incorreta. O trecho apresenta instruções de um jogo, e não
informações relacionadas à ciência.
(B) Correta. O manual mostra instruções de como jogar o jogo ``Aventura
científica''.
(C) Incorreta. O manual não tem como finalidade convencer o leitor, já
que ele apenas apresenta instruções do jogo.
(D) Incorreta. O texto ajuda o leitor a jogar o jogo, e não a solucionar
problemas.
\end{enumerate}

\section*{Módulo 4 – Treino}

\begin{enumerate}
\item
SAEB: Analisar o uso de recursos de persuasão em textos verbais e/ou multimodais.
BNCC: EF03LP19 -- Identificar e discutir o propósito do uso de recursos de
persuasão (cores, imagens, escolha de palavras, jogo de palavras,
tamanho de letras) em textos publicitários e de propaganda, como
elementos de convencimento.
(A) Incorreta. Embora haja a presença de desenhos no anúncio, esse não é
o objetivo da campanha, visto que a conscientização é referente à
vacinação.
(B) Incorreta. A ``turma da vacinação'' é referente à conscientização
sobre a necessidade de se vacinar contra doenças específicas, não em
relação à importância de ter amizades.
(C) Incorreta. A finalidade da campanha é promover a conscientização sobre a
importância da vacinação contra o sarampo e a paralisia infantil.
(D) Correta. No cartaz, compreende-se que os elementos verbais e não
verbais apontam para a importância da vacinação contra o sarampo e a
paralisia infantil, dado o anúncio ``Vacinação contra o sarampo e a
paralisia infantil'', bem como o slogan ``Vem pra turma da vacinação''.

\item
SAEB: Analisar os efeitos de sentido de recursos multissemiótico em textos que circulam em diferentes suportes.
BNCC: EF03LP19 -- Identificar e discutir o propósito do uso de recursos de
persuasão (cores, imagens, escolha de palavras, jogo de palavras,
tamanho de letras) em textos publicitários e de propaganda, como
elementos de convencimento.
(A) Incorreta. A presença da água no cartaz reforça a ideia de que se deve ter cuidado com a água parada, que se torna um criadouro de mosquitos da dengue.
(B) Correta. O xis vermelho que ``corta'' determinados elementos indica que as situações relacionadas a eles devem ser evitadas na tentativa de se combater o mosquito da dengue.
(C) Incorreta. A representação do mosquito relaciona-se diretamente ao texto ``Combata o mosquito''.
(D) Incorreta. A imagem das larvas na água relembra que se devem evitar os criadouros do mosquito.

\item
SAEB: Julgar a eficácia de argumentos em textos.
BNCC: EF03LP19 -- Identificar e discutir o propósito do uso de recursos de
persuasão (cores, imagens, escolha de palavras, jogo de palavras,
tamanho de letras) em textos publicitários e de propaganda, como
elementos de convencimento.
(A) Incorreta. O cartaz mostra-se eficiente.
(B) Incorreta. Está, sim, claro o que motiva o cartaz.
(C) Correta. A ideia de defender o meio ambiente com a diminuição do consumo de plástico está bem clara no cartaz.
(D) Incorreta. Já há texto do cartaz diretamente dirigido ao leitor.
\end{enumerate}

\section*{Módulo 5 – Treino}

\begin{enumerate}
\item
SAEB: Reconhecer diferentes modos de organização composicional de textos em versos.
BNCC: EF35LP27 -- Ler e compreender, com certa autonomia, textos em
versos, explorando rimas, sons e jogos de palavras, imagens poéticas
(sentidos figurados) e recursos visuais e sonoros.
(A) Incorreta. Os versos têm tamanhos diferentes.
(B) Incorreta. Todas as estrofes têm o mesmo número de versos.
(C) Correta. Cada estrofe tem quatro versos.
(D) Nenhuma estrofe tem apenas um verso.

\item
SAEB: Reconhecer diferentes modos de organização composicional de textos em versos.
BNCC: EF35LP27 -- Ler e compreender, com certa autonomia, textos em
versos, explorando rimas, sons e jogos de palavras, imagens poéticas
(sentidos figurados) e recursos visuais e sonoros.
(A) Correta. As palavras finais ``trança'' e ``criança'' rimam entre si.
(B) Incorreta. As palavras finais ``rir'' e ``descansada'' não rimae entre si.
(C) Incorreta. A palavra ``Descuidada'' e a palavra final ``porvir'' não rimam entre si.
(D) Incorreta. As palavras finais ``trança'' e ``descansada'' não rimam entre si.

\item
SAEB: Analisar a construção de sentidos de textos em versos com base em seus elementos constitutivos.
BNCC: EF35LP31 -- Identificar, em textos versificados, efeitos de sentido
decorrentes do uso de recursos rítmicos e sonoros e de metáforas.
(A) Incorreta. Não há dificuldade em se compreender a inversão operada no verso.
(B) Correta. A inversão, que coloca ``está'' no final do verso, enfatiza o sentido do verbo para o verso e cria uma rima de final de verso com o primeiro, em que ocorre a palavra ``lá''.
(C) Incorreta. Em poemas, é típico haver inversões de palavras com objetivos rítmicos, estéticos ou semânticos.
(D) Incorreta. Não há inversão no último verso da segunda estrofe.
\end{enumerate}

\section*{Módulo 6 – Treino}

\begin{enumerate}
\item
SAEB: Identificar as variedades linguísticas em textos.
BNCC: EF35LP30 -- Diferenciar discurso indireto e discurso direto,
determinando o efeito de sentido de verbos de enunciação e explicando o
uso de variedades linguísticas no discurso direto, quando for o caso.
(A) Incorreta. Apesar da semelhança e da origem comum, o substantivo ``rebenta'' não é uma variação para a forma ``arrebenta'', do verbo ``arrebentar''.
(B) Incorreta. A palavra ``rebenta'' é uma flexão de ``rebento'', mas não uma variação.
(C) Correta. As palavras ``rebenta'' e ``filha'', nesse contexto, são sinônimas.
(D) Incorreta. O sentido de ``broto'' é um possível para a palavra ``rebento'', mas não é o caso do contexto.

\item
SAEB: Identificar as variedades linguísticas em textos.
BNCC: EF35LP22 -- Perceber diálogos em textos narrativos, observando o
efeito de sentido de verbos de enunciação e, se for o caso, o uso de
variedades linguísticas no discurso direto.
(A) Correta. A forma ``louro'' é anterior ao aparecimento de ``loiro''.
(B) Incorreta. Tanto a forma ``louro'' quanto a forma ``loiro'' são igualmente corretas.
(C) Incorreta. Na escrita, ainda é mais comum a forma ``louro''.
(D) Incorreta. No dia a dia, a forma ``loiro'' é mais comum que ``louro''.

\item
SAEB: Identificar as variedades linguísticas em textos.
BNCC: EF35LP30 -- Diferenciar discurso indireto e discurso direto,
determinando o efeito de sentido de verbos de enunciação e explicando o
uso de variedades linguísticas no discurso direto, quando for o caso.
(A) Incorreta. Os dois compadres utilizam as mesmas variantes da língua.
(B) Incorreta. Apesar de ter confundido seu significado, o segundo compadre reconheceu a palavra ``firme''.
(C) Incorreta. A pergunta do primeiro compadre não era sobre o programa na TV, apesar de o outro compadre ter entendido isso.
(D) Correta. O humor da piada deve-se ao fato de que o segundo compadre relacionou a forma ``firme'' à palavra ``filme'', porque as duas palavras, na variedade linguística em questão, são pronunciadas da mesma forma.
\end{enumerate}

\section*{Módulo 7 – Treino}

\begin{enumerate}
\item
SAEB: Analisar os efeitos de sentido decorrentes do uso dos adjetivos.
BNCC: EF03LP09 -- Identificar, em textos, adjetivos e sua função de
atribuição de propriedades aos substantivos.
(A) Incorreta. A palavra ``deserto'' caracteriza o espaço em que se
passa a narrativa.
(B) Correta. Uma das palavras que caracterizam a
guardadora de patos, a personagem principal da história, é ``velhinha''.
(C) Incorreta. O adjetivo ``linda'' caracteriza a casa da personagem.
(D) Incorreta. O adjetivo ``grande'' caracteriza o espaço da narrativa.

\item
SAEB: Analisar os efeitos de sentido decorrentes do uso dos adjetivos.
BNCC: EF03LP23 -- Analisar o uso de adjetivos em cartas dirigidas a
veículos da mídia impressa ou digital (cartas do leitor ou de reclamação
a jornais ou revistas), digitais ou impressas.
(A) Correta. Segundo os leitores, os artistas da época retratava D. João VI como gordinho.
(B) Incorreta. A informação de que D. João VI era guloso não tem como fonte os estudantes do quinto ano, mas um texto que eles leram na revista.
(C) Incorreta. Fala-se em uma fama de ser guloso de D. João VI, mas isso não o caracteriza como famoso por ser o rei de Portugal.
(D) Incorreta. Apesar de se falar em festas promovidas por D. João VI, não se caracteriza o rei como festeiro.

\item
SAEB: Analisar os efeitos de sentido decorrentes do uso dos advérbios.
Não há correspondência com a BNCC do terceiro ano.
(A) Incorreta. O advérbio está ligado a um adjetivo.
(B) Incorreta. O advérbio está modificando um adjetivo.
(C) Correta. Por meio da expressão ``mais... que...'', em que aparece o advérbio ``mais'', cria-se uma comparação de superioridade.
(D) Incorreta. O advérbio tem sentido de intensificador.
\end{enumerate}

\section*{Simulado 1}

\begin{enumerate}
\item
SAEB: Identificar a ideia central o texto.
BNCC: EF35LP03 -- Identificar a ideia central do texto, demonstrando
compreensão global.
(A) Incorreta. A Companhia Delas é a criadora do espetáculo, mas o tema do texto é o espetáculo em si.
(B) Correta. A notícia trata da peça ``Maria e os Insetos''.
(C) Incorreta. ``Mary e os monstros marinhos'' é uma das peças criadas
pela mesma companhia, mencionada no texto, mas não é o tema.
(D) Incorreta. Thaís Medeiros faz parte da Cia. Delas, grupo que criou a
peça ``Maria e os insetos'', tema da notícia.

\item
SAEB: Localizar informação explícita.
BNCC: EF15LP03 -- Localizar informações explícitas em textos.
(A) Incorreta. A ovelha não recebeu sua parte.
(B) Incorreta. A cabra ficou sem carne.
(C) Incorreta. Nenhuma parte foi dada à vaca.
(D) Correta. Explicitamente, o leão se apossou de todas as quatro partes da carne.

\item
SAEB: Inferir informações implícitas em textos.
BNCC: EF35LP04 -- Inferir informações implícitas nos textos lidos.
(A) Correta. Está implícita a ideia de que o narrador-personagem não esteve com seus companheiros depois do incidente..
(B) Incorreta. O narrador-personagem não morreu afogado.
(C) Incorreta. O narrador-personagem não teve notícias de seus companheiros.
(D) Incorreta. O narrador-personagem precisou caminhar pelo mar até chegar à terra firme.

\item
SAEB: Inferir o sentido de palavras ou expressões em textos.
BNCC: EF35LP05 -- Inferir o sentido de palavras ou expressões
desconhecidas em textos, com base no contexto da frase ou do texto.
(A) Incorreta. Essa informação não se refere à expressão destacada.
(B) Incorreta. No texto, verifica-se que os pais estavam, daquela vez,
esperando uma menina.
(C) Correta. Entende-se que, ao estar ``esperando criança'', a mãe estava
grávida.
(D) Incorreta. Essa informação não se refere à expressão destacada.

\item
SAEB: Identificar a ideia central o texto. BNCC: EF35LP03 --- Identificar a ideia central do texto, demonstrando compreensão global a) Incorreta. Não há menção a postos de saúde. b) Incorreta. A afirmação é oposta ao que informa a reportagem. c) Correta. O foco do texto é exatamente a queda na cobertura vacinal do país. d) Incorreta. O texto informa que poliomielite e outras doenças correm o risco de voltar ao Brasil.

\item
SAEB: Localizar informação explícita. BNCC: EF15LP03 --- Localizar informações explícitas em textos a) Incorreta. O texto é claro ao falar que não se compreende a cultura dos indígenas. b) Incorreta. O texto é enfático ao declarar que não foram reduzidos com mágica. c) Correta. O texto afirma que um dos piores motivos para a redução dos indígenas seja o apagamento de sua história. d) Incorreta. O texto é sobre a diminuição e não sobre o crescimento da população indígena.

\item
SAEB: Inferir informações implícitas em textos. BNCC: EF35LP04 --- Inferir informações implícitas nos textos lidos. a) Incorreta. Olívia pode ser enfermeira, recepcionista ou cozinheira, entre outras profissões, por exemplo. b) Incorreta. Não há qualquer indicação do dia da semana em que o diálogo ocorreu. c) Incorreta. Nada no diálogo demonstra que Sara não trabalha. d) Correta. Olívia pediu desculpas por não poder ir à casa da amiga na quinta, o que pressupõe uma combinação prévia.

\item
SAEB: Inferir o sentido de palavras ou expressões em textos. BNCC: EF35LP05 --- Inferir o sentido de palavras ou expressões desconhecidas em textos, com base no contexto da frase ou do texto. a) Incorreta. A informação é que a chuva do litoral de São Paulo superou a de Petrópolis. b) Correta. O texto diz que a chuva do litoral foi considerada um novo recorde. c) Incorreta. Não há menção à influência do derretimento das geleiras nas chuvas. d) Incorreta. O texto diz que as chuvas torrenciais serão cada vez mais frequentes no país.

\item
SAEB: Reconhecer em textos o significado de palavras derivadas a partir de seus afixos. BNCC: EF03LP10 --- Reconhecer prefixos e sufixos produtivos na formação de palavras derivadas de substantivos, de adjetivos e de verbos, utilizando-os para compreender palavras e para formar novas palavras. a) Correta. A palavra ``passarinho'' é um diminutivo, enquanto ``anteontem'' contém um prefixo que indica anterioridade. b) Incorreta. Na frase, só há um diminutivo. c) Incorreta. As palavras formadas por sufixação, na frase, não são diminutivos. d) Incorreta. Na frase, só há um diminutivo.

\item
SAEB: Identificar as marcas de organização de textos dramáticos. BNCC: EF35LP24 --- Identificar funções do texto dramático (escrito para ser encenado) e sua organização por meio de diálogos entre personagens e marcadores das falas das personagens e de cena. a) Correta. A organização do texto dramático é em atos e cenas descritos detalhadamente. b) Incorreta. O texto dramático pode ser um monólogo, com apenas uma atriz ou um ator. c) Incorreta. O texto dramático precisa descrever em detalhes a situação a ser encenada de modo que atores, cenógrafos, iluminadores, figurinistas consigam traduzi-la para a plateia. d) Incorreta. O narrador é dispensável, já que a história não é contada, mas representada pelos artistas em cena.

\item
SAEB: Analisar os efeitos de sentido de verbos de enunciação. BNCC: EF35LP30 --- Diferenciar discurso indireto e discurso direto, determinando o efeito de sentido de verbos de enunciação e explicando o uso de variedades linguísticas no discurso direto, quando for o caso. a) Incorreta. O verbo ``implorar'' não foi empregado para satisfazer um desejo. b) Correta. O verbo ``implorar'' foi utilizado como uma súplica, traduzindo sua necessidade. c) Incorreta. O verbo ``implorar'', na situação descrita, sugere algo além do querer --- necessitar. d) Incorreta. Em nenhuma hipótese, o verbo ``implorar'' pode ser
entendido com o sentido de ``exigir''.

\item
SAEB: Reconhecer os usos da pontuação. BNCC: EF35LP07 --- Utilizar, ao produzir um texto, conhecimentos linguísticos e gramaticais, tais como ortografia, regras básicas de concordância nominal e verbal, pontuação (ponto final, ponto de exclamação, ponto de interrogação, vírgulas em enumerações) e pontuação do discurso direto, quando for o caso. a) Incorreta. Os parênteses não iniciam frases; são utilizados para isolar alguma informação. b) Incorreta. Os dois-pontos podem ser utilizados para indicar falas, mas não aparecem no início de frases. c) Incorreta. O ponto de interrogação aparece no final da fala para indicar pergunta. d) Correta. Uma das funções do travessão é indicar o início da fala de cada personagem.

\item
SAEB: Analisar o uso de recursos de persuasão em textos verbais e/ou multimodais. BNCC: EF03LP19 --- Identificar e discutir o propósito do uso de recursos de persuasão (cores, imagens, escolha de palavras, jogo de palavras, tamanho de letras) em textos publicitários e de propaganda, como elementos de convencimento. a) Incorreta. Na ocasião, a população estava em guerra contra o mosquito e precisava do esforço de todos para vencê-lo. b) Correta. Ninguém recebe o mosquito como um hóspede, mas, ao dar condições para ele morar e se reproduzir, é como se o fizesse. c) Incorreta. Não há jogo, são ordens diretas sobre o que fazer. d) Incorreta. São orientações claras sobre como proceder com o lixo para evitar que se torne um criadouro do mosquito.

\item
SAEB: Reconhecer diferentes modos de organização composicional de textos em versos BNCC: EF35LP31 --- Identificar, em textos versificados, efeitos de sentido decorrentes do uso de recursos rítmicos e sonoros e de metáforas. a) Incorreta. O sentido é literal. b) Incorrera. O sentido é literal. c) Correta. A jovem pianista é como uma flor, o que sugere beleza e delicadeza. d) Incorreta. O sentido é literal.

\item
SAEB: Analisar os efeitos de sentido decorrentes do uso dos advérbios. Não há correspondência com a BNCC do terceiro ano. a) Correta. A fala do amigo ironiza o fato de Rodolfo ter chegado tarde para o almoço. b) Incorreta. A fala não é incoerente. c) Incorreta. Não há ambiguidade no uso desse advérbio. d) Incorreta. A fala pode ser perfeitamente bem compreendida.
\end{enumerate}

\section*{Simulado 2}

\begin{enumerate}
\item
SAEB: Reconhecer diferentes gêneros textuais.
BNCC: EF35LP29 -- Identificar, em narrativas, cenário, personagem central,
conflito gerador, resolução e o ponto de vista com base no qual
histórias são narradas, diferenciando narrativas em primeira e terceira
pessoas.
(A) Incorreta. Não se trata de uma fábula, pois não contém uma moral.
(B) Incorreta. Não se trata de um poema, mas de um texto em prosa.
(C) Correta. A história da Chapeuzinho Vermelho é narrada em um conto.
(D) Incorreta. Não se trata de uma anedota, pois não tem o objetivo de fazer rir.

\item
SAEB: Identificar as marcas de organização de textos dramáticos.
BNCC: EF35LP24 -- Identificar funções do texto dramático (escrito para ser encenado) e sua organização
por meio de diálogos entre personagens e marcadores das falas das personagens e de cena.
(A) Incorreta. Textos dramáticos não apresentam narrador.
(B) Correta. O desenrolar por meio de diálogo é uma marca característica do texto dramático.
(C) Incorreta. A presença de personagens não é exclusividade do texto dramático.
(D) Incorreta. Nível de informatividade não é uma marca a ser avaliada no texto dramático.

\item
SAEB: Analisar elementos constitutivos de gêneros textuais diversos.
BNCC: EF35LP16 -- Identificar e reproduzir, em notícias, manchetes, lides e
corpo de notícias simples para público infantil e cartas de reclamação
(revista infantil), digitais ou impressos, a formatação e diagramação
específica de cada um desses gêneros, inclusive em suas versões orais.
(A) Correta. Em entrevistas, as perguntas feitas pelo entrevistador
normalmente são apresentadas em negrito para que possam ser
diferenciadas das respostas do entrevistado.
(B) Incorreta. As respostas do entrevistado normalmente são apresentadas
sem estar em negrito.
(C) Incorreta. Os trechos em negrito não mostram esse assunto.
(D) Incorreta. Essa é uma característica de um texto narrativo, e não de
um texto informativo como a entrevista.

\item
SAEB: Inferir o sentido de palavras ou expressões em textos.
BNCC: EF35LP05 -- Inferir o sentido de palavras ou expressões desconhecidas em textos, com base no
contexto da frase ou do texto.
(A) Incorreta. Os verbos ``confeccionar'' e ``purificar'' não são próximos semanticamente.
(B) Correta. O verbo ``fabricar'' pode ser considerado um sinônimo de ``confeccionar''.
(C) Incorreta. Os verbos ``confeccionar'' e ``configurar'' não são próximos semanticamente.
(D) Incorreta. Os verbos ``confeccionar'' e ``organizar'' não são próximos semanticamente.

\item
SAEB: Analisar os efeitos de sentido decorrentes do uso dos adjetivos. BNCC: EF03LP09 --- Identificar, em textos, adjetivos e sua função de atribuição de propriedades aos substantivos. a) Correta. No primeiro caso, a colocação do adjetivo antes do substantivo dá ideia de antiguidade; cachorro velho = idoso; grande homem = notável; homem grande = alto. b) Incorreta. ``Cachorro velho'' não significa companheiro antigo, assim como ``grande homem'' não o qualifica como rico. c) Incorreta. ``Velho cachorro'' não dá sentido de inteligência ao animal, assim como ``grande homem'' não quer dizer que a pessoa seja alta, e tampouco ``homem grande'' quer dizer que ele seja respeitável. d) Incorreta. ``Cachorro velho'' não significa cachorro inteligente; da mesma forma, ``homem grande'' não quer dizer que ele seja rico.

\item
SAEB: Identificar as variedades linguísticas em textos. Não há correspondência com a BNCC do terceiro ano. a) Incorreta. O pronome ``você'' ocorre em praticamente todo o Brasil. b) Correta. As mudanças aconteceram no decorrer do tempo. c) Incorreta. Médicos, surfistas e professores, por exemplo, apesar de grupos sociais distintos, acompanharam essas mudanças. d) Incorreta. As alterações têm relação com o tempo, e não com a situação de interlocução.

\item
SAEB: Analisar a construção de sentidos de textos em versos com base em seus elementos constitutivos. BNCC: EF35LP23 --- Apreciar poemas e outros textos versificados, observando rimas, aliterações e diferentes modos de divisão dos versos, estrofes e refrões e seu efeito de sentido. a) Incorreta. Não há rimas intercaladas. b) Incorreta. Não há metáforas. c) Correta. A aliteração, por meio da repetição do fonema S, provoca um efeito sonoro que sugere o guizo da serpente. d) Incorreta. A repetição do fonema S não provoca sensação de força.

\item
SAEB: Identificar a ideia central o texto. BNCC: EF35LP03 --- Identificar a ideia central do texto, demonstrando compreensão global. a) Correta. O texto chama a atenção para a influência humana que levou o planeta ao aquecimento global e à necessidade fazer adaptações. b) Incorreta. O título enfoca a necessidade de fazer adaptações; portanto os eventos climáticos não deixarão de acontecer espontaneamente. c) Incorreta. O texto cita estados onde eventos extremos ocorreram, mas não menciona qualquer dado relacionado à geografia. d) Incorreta. A ideia central do texto é oposta a essa afirmação.

\item
SAEB: Reconhecer diferentes gêneros textuais. BNCC: EF03LP11 --- Ler e compreender, com autonomia, textos injuntivos instrucionais (receitas, instruções de montagem etc.), com a estrutura própria desses textos (verbos imperativos, indicação de passos a serem seguidos) e mesclando palavras, imagens e recursos gráfico-visuais, considerando a situação comunicativa e o tema/assunto do texto. a) Correta. Textos injuntivos podem se valer do imperativo. b) Incorreta. Não há narrador nos textos injuntivos. c) Incorreta. Adjetivos não fazem parte das características específicas dos textos injuntivos. d) Incorreta. Não há exposição de fatos, mas passos a serem seguidos.

\item
SAEB: Identificar elementos constitutivos de textos narrativos BNCC: EF35LP26 --- Ler e compreender, com certa autonomia, narrativas ficcionais que apresentem cenários e personagens, observando os elementos da estrutura narrativa: enredo, tempo, espaço, personagens, narrador e a construção do discurso indireto e discurso direto. a) Incorreta. Verbos sempre no presente não são obrigatórios nos textos narrativos, assim como adjetivos, fatos e opiniões. b) Incorreta. Locuções adverbiais não são elementos específicos do texto narrativo. c) Incorreta. As histórias dos textos narrativos podem ser fictícias ou reais. d) Correta. Todos os elementos apresentados são características das narrativas.

\item
SAEB: Analisar os efeitos de sentido decorrentes do uso da pontuação. BNCC: EF03LP07 --- Identificar a função na leitura e usar na escrita ponto final, ponto de interrogação, ponto de exclamação e, em diálogos (discurso direto), dois-pontos e travessão. a) Incorreta. A exclamação pelo nascimento de alguém não exprime tristeza; exclamar que existe uma aranha na cozinha não demonstra desprezo. b) Incorreta. Um dia lindo não é motivo para susto, assim como afirmar, exclamando, ``não acredito que você teve a coragem de fazer isso comigo'' não transmite alívio. c) Correta. É uma alegria o nascimento de um irmão; é um susto encontrar uma aranha na cozinha; um dia lindo é sempre motivo de admiração; indignação exprime o sentimento de ter sido desrespeitado. d) Incorreta. Exclamar sobre o nascimento de um irmãozinho não traduz medo; encontrar uma aranha na cozinha não traz tristeza; a frase exclamativa como no exemplo não demonstra alívio.

\item
SAEB: Analisar os efeitos de sentido de recursos multissemióticos em textos que circulam em diferentes suportes BNCC: EF15LP04 --- Identificar o efeito de sentido produzido pelo uso de recursos expressivos gráfico-visuais em textos multissemióticos. a) Incorreta. A campanha é nacional. b) Incorreta. A vacinação, segundo o próprio anúncio, é para todos. c) Correta. A frase ``Vacina é vida'' demonstra que a campanha de vacinação é uma celebração da vida, o que se indica, não verbalmente, por meio do uso de muitas cores. d) Incorreta. A vacinação é voltada para várias faixas etárias.

\item
SAEB: Reconhecer em textos o significado de palavras derivadas a partir de seus afixos. BNCC: EF03LP10 --- Reconhecer prefixos e sufixos produtivos na formação de palavras derivadas de substantivos, de adjetivos e de verbos, utilizando-os para compreender palavras e para formar novas palavras. a) Incorreta. A palavra ``cirurgião'' não é formada por um sufixo de aumentativo. b) Incorreta. A palavra ``mão'' não é formada por um sufixo de aumentativo. c) Correta. A partir de ``pai'', a palavra ``paizão'' é formada pelo acréscimo de um sufixo de aumentativo. d) Incorreta. A palavra ``tubarão'' não é formada por um sufixo de aumentativo.

\item
SAEB: Localizar informação explícita. BNCC: EF15LP03 --- Localizar informações explícitas em textos. a) Incorreta. Em nenhuma parte do texto, aparece a informação de que Mamãe queria ser cantora. b) Incorreta. O texto diz que Mamãe ``fica horas e horas'' metendo as mãos na água. c) Incorreta. O texto não dá nenhuma informação a esse respeito. d) Correta. O texto afirma que ela lavava roupa para ajudar nas despesas da casa.

\item
SAEB: Julgar a eficácia de argumentos em textos Não há correspondência com a BNCC do terceiro ano. a) Correta. Argumentos de especialistas ou figuras de autoridade dão credibilidade às informações. b) Incorreta. O texto sequer menciona a palavra ``poupança''. c) Incorreta. A argumentação apresentada é oposta à apresentada na alternativa. d) Incorreta. Devemos confiar em informações de especialistas, já que eles dominam o assunto de suas áreas de atuação.
\end{enumerate}

\section*{Simulado 3}

\begin{enumerate}
\item
SAEB: Reconhecer em textos o significado de palavras derivadas a partir de seus afixos.
BNCC: EF03LP10 -- Reconhecer prefixos e sufixos produtivos na formação de
palavras derivadas de substantivos, de adjetivos e de verbos,
utilizando-os para compreender palavras e para formar novas palavras.
(A) Incorreta. O prefixo ``in-'' não reforça o sentido de ``grato''.
(B) Incorreta. O prefixo ``in-'' não ressalta o sentido de ``grato''.
(C) Incorreta. O prefixo ``in-'' não reitera o sentido de ``grato''.
(D) Correta. A palavra ``ingrato'', graças ao prefixo ``in-'', tem sentido contrário ao da palavra ``grato''.

\item
SAEB: Inferir informações implícitas em textos.
BNCC: EF35LP04 -- Inferir informações implícitas nos textos lidos.
(A) Correta. Ali Babá trabalhava para comerciantes ricos, informação
implícita no trecho ``Os ricos comerciantes já conheciam Ali Babá e
gostavam muito de seu serviço''.
(B) Incorreta. Os camelos eram dos ricos comerciantes.
(C) Incorreta. Os mercadores davam dinheiro a mais para Ali Babá por
terem simpatia com ele.
(D) Incorreta. A água e o alimento eram para os camelos dos viajantes.

\item
SAEB: Reconhecer os usos da pontuação.
BNCC: EF03LP07 -- Identificar a função na leitura e usar na escrita ponto final, ponto
de interrogação, ponto de exclamação e, em diálogos (discurso direto), dois-pontos e
travessão.
(A) Incorreta. Os dois-pontos podem ser usados para explicar o sentido de um termo, mas não é o caso do texto.
(B) Incorreta. Os dois-pontos podem ser usados para acrescentar uma informação, mas não é o caso do texto.
(C) Incorreta. O discurso indireto prescinde do uso dos dois-pontos.
(D) Correta. Como é comum, os dois-pontos foram usados para indicar discurso direto.

\item
SAEB: Analisar os efeitos de sentido decorrentes do uso dos adjetivos.
BNCC: EF03LP09 -- Identificar, em textos, adjetivos e sua função de atribuição de propriedades
aos substantivos.
(A) Incorreta. O narrador dá conta de descrever tio Benedito como um velhote bastante debilitado.
(B) Correta. Esses adjetivos e toda a descrição do texto mostram tio Benedito como alguém que se desgastou muito na vida.
(C) Incorreta. O vigor da mocidade já havia abandonado tio Benedito.
(D) Correta. Tio Benedito estava mais para um velho do que para um jovenzinho.

\item
SAEB: Inferir informações implícitas em textos. BNCC: EF35LP04 --- Inferir informações implícitas nos textos lidos. a) Incorreta. Não há beleza em seres vivos aquáticos aprisionados em \emph{pets} e sacolas plásticas. b) Incorreta. Garrafas plásticas não proporcionam proteção aos peixes; pelo contrário, elas os impedem de se locomoverem e se alimentarem adequadamente. c) Correta. A ilustração mostra como os resíduos plásticos podem aprisionar habitantes aquáticos, prejudicando suas vidas. d) Incorreta. Apesar de mostrar vários tipos de animais e plantas aquáticas, o destaque fica por conta das sacolas e \emph{pets} que não deveriam estar presentes naquele local.

\item
SAEB: Reconhecer diferentes gêneros textuais. BNCC: EF03LP19 --- Identificar e discutir o propósito do uso de recursos de persuasão (cores, imagens, escolha de palavras, jogo de palavras, tamanho de letras) em textos publicitários e de propaganda, como elementos de convencimento. a) Incorreta. Não há argumentações históricas no texto. b) Correta. O anúncio de venda é predominantemente descritivo. c) Incorreta. Não há foco em ação narrada no anúncio. d) Incorreta. Não há uma dissertação estruturada no anúncio.

\item
SAEB: Identificar elementos constitutivos de textos narrativos. BNCC: EF35LP26 --- Ler e compreender, com certa autonomia, narrativas ficcionais que apresentem cenários e personagens, observando os elementos da estrutura narrativa: enredo, tempo, espaço, personagens, narrador e a construção do discurso indireto e discurso direto. a) Correta. Logo no primeiro parágrafo, quem narra a história diz que procurou a sua tia. b) Incorreta. A tia faz parte da história, mas não a está contando. c) Incorreta. A prima é mencionada apenas nesse trecho do texto. d) Incorreta. Fica claro no primeiro parágrafo que a sobrinha é quem conta a história.

\item
SAEB: Identificar as marcas de organização de textos dramáticos. BNCC: EF35LP24 --- Identificar funções do texto dramático (escrito para ser encenado) e sua organização por meio de diálogos entre personagens e marcadores das falas das personagens e de cena. a) Correta. Descrição detalhada dos personagens (características físicas, figurinos, gestos) e detalhamento de cenas (diálogos, movimento dos personagens) são elementos do texto que nasce para ser encenado. b) Incorreta. Poema concreto não tem qualquer relação com detalhamento de cenas e personagens. c) Incorreta. Texto publicitário visa a convencer, persuadir o público-alvo a fazer alguma coisa; dispensa descrição de cenas e personagens.
d) Incorreta. Conto realista não pede descrição de cenas.

\item
SAEB: Analisar elementos constitutivos de gêneros textuais diversos. BNCC: EF03LP18 --- Ler e compreender, com autonomia, cartas dirigidas a veículos da mídia impressa ou digital (cartas de leitor e de reclamação a jornais, revistas) e notícias, dentre outros gêneros do campo jornalístico, de acordo com as convenções do gênero carta e considerando situação comunicativa e o tema/assunto do texto. a) Incorreta. Charge é texto verbal e não verbal. b) Correta. Notícia apresenta título, subtítulo, narra fatos atuais de maneira impessoal, não utiliza adjetivos, é informativo. c) Incorreta. Artigo de opinião é um texto dissertativo-argumentativo. d) Incorreta. Reportagem é um texto longo, que pode incluir entrevistas,
gráficos, análises de dados estatísticos etc.

\item
SAEB: Reconhecer os usos da pontuação. BNCC: EF35LP07 --- Utilizar, ao produzir um texto, conhecimentos linguísticos e gramaticais, tais como ortografia, regras básicas de concordância nominal e verbal, pontuação (ponto final, ponto de exclamação, ponto de interrogação, vírgulas em enumerações) e pontuação do discurso direto, quando for o caso. a) Correta. Os itens da lista de compras foram separados por vírgulas; as exclamações expressaram surpresa/admiração. b) Incorreta. Não há vocativos no diálogo. c) Incorreta. Sujeito e predicado não foram separados por vírgula; tampouco a exclamação expressa indignação. d) Incorreta. A palavra ``Nossa'' pode ser utilizada como interjeição, mas não como conjunção.

\item
SAEB: Identificar as variedades linguísticas em textos BNCC: BNEF35LP22 --- Perceber diálogos em textos narrativos, observando o efeito de sentido de verbos de enunciação e, se for o caso, o uso de variedades linguísticas no discurso direto. a) Incorreta. Não tem a ver com a passagem do tempo, mas com a região. b) Correta. A palavra muda de acordo com a região onde é falada. c) Incorreta. Não se trata de vocábulos atrelados a grupos sociais. d) Incorreta. Os vocábulos apresentados independem da situação.

\item
SAEB: Analisar os efeitos de sentido decorrentes do uso dos advérbios. BNCC --- Não há correspondência com a BNCC do quarto ano. a) Incorreta. ``Preocupados'' não é advérbio. b) Incorreta. ``Sempre'' é um advérbio de tempo, não remete à dor. c) Correta. ``Copiosamente'' muda o sentido do verbo \emph{chorar} e sugere muita dor. d) Incorreta. ``Infelizmente'' indica uma situação adversa, mas não um desespero.

\item
SAEB: Analisar os efeitos de sentido decorrentes do uso da pontuação. BNCC: EF03LP07 --- Identificar a função na leitura e usar na escrita ponto final, ponto de interrogação, ponto de exclamação e, em diálogos (discurso direto), dois-pontos e travessão. a) Incorreta. Travessão não serve para cortar sentenças. b) Correta. Uma das funções do travessão no meio de frases é destacar alguma informação. c) Incorreta. Os travessões não estão separando doces de salgados. d) Incorreta. A informação de que havia doces caseiros já estava clara em ``principalmente caseiros''.

\item
SAEB: Analisar o uso de recursos de persuasão em textos verbais e/ou multimodais. BNC: EF03LP19 --- Identificar e discutir o propósito do uso de recursos de persuasão (cores, imagens, escolha de palavras, jogo de palavras, tamanho de letras) em textos publicitários e de propaganda, como elementos de convencimento. a) Incorreta. A frase não é um elemento persuasivo, mas uma ordem para não fazer. b) Incorreta. O apoio da Prefeitura é importante, mas não persuasivo. c) Incorreta. Ele faz parte das suas escolhas do outro, então o outro pode escolher abandonar o cãozinho. d) Correta. A imagem dá uma sensação ruim imaginar a tristeza e o medo que o cãozinho está sentindo ao se sentir abandonado pelos tutores em quem confiava.

\item
SAEB: Analisar os efeitos de sentido decorrentes do uso dos adjetivos BNCC: EF03LP09 --- Identificar, em textos, adjetivos e sua função de atribuição de propriedades aos substantivos. a) Incorreta. Trata-se de advérbio. b) Correta. \emph{Azedinha} é uma das características da fruta. c) Incorreta. \emph{Praticamente} é advérbio d) Incorreta. A palavra é um verbo.
\end{enumerate}

\section*{Simulado 4}

\begin{enumerate}
\item
SAEB: Analisar o uso de recursos de persuasão em textos verbais e/ou multimodais.
BNCC: EF03LP19: Identificar e discutir o propósito do uso de recursos de persuasão (cores,
imagens, escolha de palavras, jogo de palavras, tamanho de letras) em textos publicitários
e de propaganda, como elementos de convencimento.
(A) Incorreta. Embora o cartaz indique que se lavem as mãos duas vezes,
essa ação não ocorre após receber a entrega, mas sim uma vez ao receber
a entrega e outra antes de consumir o produto.
(B) Incorreta. As mãos devem ser lavadas antes de consumir o produto, e
não depois.
(C) Incorreta. No cartaz, não há informações para lavar as mãos ao descartar a embalagem.
(D) Correta. Ao ler o cartaz da campanha, identifica-se que ela
incentiva e instrui o leitor a lavar as mãos duas vezes, após a pessoa
receber a entrega e antes de usar o produto.

\item
SAEB: Identificar elementos constitutivos de textos narrativos.
BNCC: EF35LP26 -- Ler e compreender, com certa autonomia, narrativas
ficcionais que apresentem cenários e personagens, observando os
elementos da estrutura narrativa: enredo, tempo, espaço, personagens,
narrador e a construção do discurso indireto e discurso direto.
(A) Incorreta. Não existe narrador-personagem na narrativa.
(B) Incorreta. Os irmãos não narram a história.
(C) Correta. A história é apresentada por meio de um narrador-observador.
(D) Incorreta. O narrador da história é observador.

\item
SAEB: Inferir o sentido de palavras ou expressões em textos.
BNCC: EF35LP05 -- Inferir o sentido de palavras ou expressões desconhecidas em textos, com base no
contexto da frase ou do texto.
(A) Incorreta. O sentido de ``ganhar'', na expressão, não é o literal.
(B) Correta. A expressão ``ganhar a vida'' relaciona-se à capacidade de alguém de sustentar a própria vida.
(C) Incorreta. A expressão não está relacionada a conseguir uma ocupação, mas a conseguir o sustento a partir da ocupação.
(D) Incorreta. O sentido da expressão está voltado para a própria pessoa, e não para o sustento de outra.

\item
SAEB: Reconhecer em textos o significado de palavras derivadas a partir de seus afixos.
BNCC: EF03LP10 -- Reconhecer prefixos e sufixos produtivos na formação de
palavras derivadas de substantivos, de adjetivos e de verbos,
utilizando-os para compreender palavras e para formar novas palavras.
(A) Incorreta. Há alterações no radical da palavra que não estão contempladas nessas opções.
(B) Incorreta. O aluno retirou incorretamente alguns fonemas para formar
os nomes das árvores.
(C) Correta. Nas imagens, tem-se: maçã, melão e mamão. Os nomes das respectivas árvores são: macieira, meloeiro e mamoeiro.
(D) Incorreta. O sufixo foi simplesmente acrescentado ao nome da fruta, sem que se entendesse a regra.

\item
SAEB: Inferir o sentido de palavras ou expressões em textos. BNCC: EF35LP05 --- Inferir o sentido de palavras ou expressões desconhecidas em textos, com base no contexto da frase ou do texto. a) Correta. Tem o sentido de segurar com força, fazer de tudo para não perder algo ou alguém. b) Incorreta. O verbo ``agarrar'' não sugere descansar. c) Incorreta. O sentido é de se esforçar, e não de brigar. d) Incorreta. O sentido de ``agarrar'' não sugere seguir em frente.

\item
SAEB: Analisar os efeitos de sentido de recursos multissemióticos em textos que circulam em diferentes suportes. BNCC: EF15LP04 --- Identificar o efeito de sentido produzido pelo uso de recursos expressivos gráfico-visuais em textos multissemióticos. a) Incorreta. Ao lado do mapa, o Maranhão aparece na terceira posição. b) Incorreta. Ao lado do mapa, o Ceará aparece na quarta posição. c) Correta. Ao lado do mapa do Brasil, mostra-se a Bahia em primeiro lugar e, no quadro com os locais mais beneficiados, a escola aparece na primeira posição. d) Incorreta. Ao lado do mapa, o Amazonas aparece na quinta posição.

\item
SAEB: Julgar a eficácia de argumentos em textos. Não há correspondência com a BNCC do terceiro ano. a) Incorreta. O título do referido artigo é ``Aprender a partir da leitura em voz alta dos adultos'' e mostra que ele defende a importância de ler em voz alta para crianças. b) Incorreta. O texto diz claramente que ``as autoras defendem {[}...{]} que a leitura leva ao desenvolvimento de uma linguagem mais complexa''. c) Incorreta. O título do referido artigo é ``Aprender a partir da leitura em voz alta dos adultos'' e mostra que ele defende a importância de ler em voz alta para crianças. d) Correta. O artigo foi elaborado justamente para enfatizar a importância de ler em voz alta para crianças.

\item
SAEB: Reconhecer diferentes modos de organização composicional de textos em versos. BNCC: EF35LP23 --- Apreciar poemas e outros textos versificados, observando rimas, aliterações e diferentes modos de divisão dos versos, estrofes e refrões e seu efeito de sentido. a) Correta. O primeiro poema tem rimas nos versos 2, 4 e 6, enquanto o segundo não apresenta rimas. b) Incorreta. Os dois têm pelo menos uma estrofe. c) Incorreta. O segundo não tem rimas, enquanto o primeiro tem rimas nos versos 2, 4 e 6. d) Incorreta. Os dois têm pelo menos uma estrofe.

\item
SAEB: Reconhecer em textos o significado de palavras derivadas a partir de seus afixos. BNCC: EF03LP10 --- Reconhecer prefixos e sufixos produtivos na formação de palavras derivadas de substantivos, de adjetivos e de verbos, utilizando-os para compreender palavras e para formar novas palavras. a) Incorreta. O prefixo ``ultra'' traduz uma ideia oposto à de diminutivo. b) Incorreta. A palavra tem sufixo, e não prefixo, mas também não é de diminutivo. c) Correta. Em ``microplásticos'', o prefixo ``micro'' indica diminutivo.
d) Incorreta. A palavra não tem afixo.

\item
SAEB: Reconhecer diferentes gêneros textuais. BNCC: EF03LP12 --- Ler e compreender, com autonomia, cartas pessoais e diários, com expressão de sentimentos e opiniões, dentre outros gêneros do campo da vida cotidiana, de acordo com as convenções do gênero carta e considerando a situação comunicativa e o tema/assunto. a) Incorreta. O texto em questão não tem as características de uma notícia. b) Incorreta. Apesar de conter informações sobre determinado assunto, o objetivo do texto não é informar o leitor. c) Correta. Reconhecem-se, no texto, as características de um texto escrito para um diário, inclusive com a colocação da data. d) Incorreta. Não há uma interlocução estabelecida no texto em questão.

\item
SAEB: Identificar elementos constitutivos de textos narrativos. Não há correspondência com a BNCC do terceiro ano. a) Correta. De fato, como em um texto narrativo, a notícia apresenta a narração de um fato. b) Incorreta. Não há, na notícia em questão, a construção de um cenário. c) Incorreta. Não se mostra, na notícia em questão, a interação entre personagens. d) Incorreta. O fato narrado aparece em um tempo apenas.

\item
SAEB: Analisar elementos constitutivos de gêneros textuais diversos. BNCC: EF03LP16 --- Identificar e reproduzir, em textos injuntivos instrucionais (receitas, instruções de montagem, digitais ou impressos), a formatação própria desses textos (verbos imperativos, indicação de passos a serem seguidos) e a diagramação específica dos textos desses gêneros (lista de ingredientes ou materiais e instruções de execução -- "modo de fazer"). a) Incorreta. Usar metáforas não é uma característica de textos injuntivos. b) Incorreta. Descrições detalhadas não são características de textos injuntivos. c) Correta. A presença da lista de passos a seguir é uma característica de textos injuntivos. d) Incorreta. As informações dos textos injuntivos são bem explícitas e claras. 

\item
SAEB: Reconhecer os usos da pontuação. BNCC: EF03LP07 --- Identificar a função na leitura e usar na escrita ponto final, ponto de interrogação, ponto de exclamação e, em diálogos (discurso direto), dois-pontos e travessão. a) Incorreta. A vírgula sinaliza uma pausa. b) Correta. O ponto de interrogação é indicativo de pergunta. c) Incorreta. Os dois-pontos indicam fala ou enumeração a seguir. d) Incorreta. O ponto de exclamação é utilizado para indicar surpresa, susto, indignação, por exemplo.

\item
SAEB: Analisar os efeitos de sentido decorrentes do uso dos adjetivos. BNCC: EF03LP09 --- Identificar, em textos, adjetivos e sua função de atribuição de propriedades aos substantivos. a) Correta. Os adjetivos caracterizam substantivos com abundância no primeiro parágrafo. b) Incorreta. Não são as conjunções que diferenciam os dois textos. c) Incorreta. A diferença está na presença de adjetivos em um dos dois parágrafos. d) Incorreta. Não há interjeições no segundo texto.

\item
SAEB: Analisar os efeitos de sentido de verbos de enunciação. BNCC: EF35LP22 --- Perceber diálogos em textos narrativos, observando o efeito de sentido de verbos de enunciação e, se for o caso, o uso de variedades linguísticas no discurso direto. a) Incorreta. Verbos de enunciação não se vinculam a cenários. b) Incorreta. Verbos de enunciação não se referem a tempo. c) Incorreta. Verbos de enunciação não têm conexão com o assunto do diálogo. d) Correta. Verbos de enunciação emprestam emoção à fala das personagens para melhor compreensão da atmosfera do diálogo.
\end{enumerate}