\chapter{Respostas}
\pagestyle{plain}
\footnotesize

\pagecolor{gray!40}

\colorsec{Módulo 1 – Treino}

\begin{enumerate}
\item
SAEB: Inferir informações implícitas em textos. BNCC: EF35LP04 -- Inferir informações implícitas nos textos lidos. (A) Incorreta. O ratinho saiu do buraco para explorar o mundo. (B) Incorreta. Esse não é o local em que se passa a história. (C) Incorreta. O texto menciona que o ratinho admirou a correnteza dos rios, e não que a história se passa em um rio com correnteza. (D) Correta. O ratinho saiu do buraco para explorar e acabou no quintal de uma casa de roça.

\item
Saeb: Inferir o sentido de palavras ou expressões em textos.
BNCC: EF35LP05 -- Inferir o sentido de palavras ou expressões
desconhecidas em textos, com base no contexto da frase ou do texto.
(A) Correta. A palavra ``destemido'' pode ser usada como sinônimo de
``valente'', pois é um adjetivo próprio de pessoas ou animais
guerreiros, audazes, ousados, corajosos.
(B) Incorreta. A palavra ``covarde'' é antônima da palavra ``valente''.
(C) Incorreta. A palavra ``inteligente'' é sinônima da palavra
``sabido''.
(D) Incorreta. A palavra ``doido'' teria como sinônimo ``maluco'' ou
``biruta'', por exemplo, mas não é adequado ao contexto.

\item
Saeb: Reconhecer em textos o significado de palavras derivadas a partir de seus afixos.
BNCC: EF03LP10 -- Reconhecer prefixos e sufixos produtivos na formação de
palavras derivadas de substantivos, de adjetivos e de verbos,
utilizando-os para compreender palavras e para formar novas palavras.
(A) Incorreta. A palavra ``preguiçosa'' não é primitiva.
(B) Correta. A palavra ``preguiçosa'' é derivada, uma vez que foi
formada a partir da palavra primitiva ``preguiça''.
(C) Incorreta. A palavra apresenta relação direta com a palavra
``preguiça'' (primitiva).
(D) Incorreta. A palavra é derivada diretamente de ``preguiça''.
\end{enumerate}

\colorsec{Módulo 2 – Treino}

\begin{enumerate}
\item
SAEB: Reconhecer diferentes gêneros textuais.
BNCC: EF35LP24 -- Identificar funções do texto dramático (escrito para ser
encenado) e sua organização por meio de diálogos entre personagens e
marcadores das falas das personagens e de cena.
(A) Incorreta. O texto não explica acontecimentos misteriosos ou
sobrenaturais.
(B) Incorreta. O texto não está organizado em versos.
(C) Correta. O texto apresenta estrutura do texto teatral, marcada pela
descrição dos movimentos de cena (texto secundário indicado entre
parênteses) e pela organização do enredo em falas.
(D) Incorreta. Nas fábulas, não há descrição de movimentações de cena.

\item
SAEB: Identificar as marcas de organização de textos dramáticos.
BNCC: EF35LP24 -- Identificar funções do texto dramático (escrito para
ser encenado) e sua organização por meio de diálogos entre personagens e
marcadores das falas das personagens e de cena.
(A) Correta. Roteiros cinematográficos, assim como as peças teatrais,
são elaborados para serem encenados por atores em filmes, com marcas
no texto que são tipicamente feitas para esse fim.
(B) Incorreta. Entrevista consiste em gênero informativo que apresenta
entrevistador e entrevistado.
(C) Incorreta. Reportagens são textos informativos veiculados por
diversos meios de comunicação e têm como conteúdo informações reais e
atuais.
(D) Incorreta. Jornal de rádio transmite as notícias da atualidade por
meio de emissões radiofônicas.

\item
SAEB: Identificar elementos constitutivos de textos narrativos.
BNCC: EF35LP26 -- Ler e compreender, com certa autonomia, narrativas
ficcionais que apresentem cenários e personagens, observando os
elementos da estrutura narrativa: enredo, tempo, espaço, personagens,
narrador e a construção do discurso indireto e discurso direto.
(A) Incorreta. Não existe ``eu'' ou ``nós'' no conto para se afirmar
que o narrador participe da história.
(B) Correta. O narrador não realiza as ações da história, mas narra os
acontecimentos.
(C) Incorreta. O narrador não é um personagem da história.
(D) Incorreta. O narrador não pratica nenhuma ação narrada.
\end{enumerate}

\colorsec{Módulo 3 – Treino}

\begin{enumerate}
\item
SAEB: Analisar os efeitos de sentido decorrentes do uso da pontuação.
BNCC: EF03LP07 -- Identificar a função na leitura e usar na escrita ponto
final, ponto de interrogação, ponto de exclamação e, em diálogos
(discurso direto), dois-pontos e travessão.
(A) Incorreta. Frases declarativas terminam com ponto final.
(B) Correta. A frase apresenta ponto de exclamação.
(C) Incorreta. A frase não é uma pergunta.
(D) Incorreta. A frase não expressa negação.

\item
SAEB: Reconhecer os usos da pontuação.
BNCC: EF03LP07 -- Identificar a função na leitura e usar na escrita ponto
final, ponto de interrogação, ponto de exclamação e, em diálogos
(discurso direto), dois-pontos e travessão.
(A) Incorreta. A vírgula não serve para indicar diálogos, mas para
separar elementos dentro de uma mesma frase.
(B) Incorreta. O ponto final é um sinal de pontuação que encerra o
período.
(C) Correta. O travessão é um sinal de pontuação usado especialmente no
início de cada fala no discurso direto.
(D) Incorreta. O ponto de interrogação é usado para indicar uma
pergunta.

\item
SAEB: Analisar elementos constitutivos de gêneros textuais diversos.
BNCC EF03LP16 -- Identificar e reproduzir, em textos injuntivos
instrucionais (receitas, instruções de montagem, digitais ou impressos),
a formatação própria desses textos (verbos imperativos, indicação de
passos a ser seguidos) e a diagramação específica dos textos desses
gêneros (lista de ingredientes ou materiais e instruções de execução --
"modo de fazer").
(A) Incorreta. O trecho apresenta instruções de um jogo, e não
informações relacionadas à ciência.
(B) Correta. O manual mostra instruções de como jogar o jogo ``Aventura
científica''.
(C) Incorreta. O manual não tem como finalidade convencer o leitor, já
que ele apenas apresenta instruções do jogo.
(D) Incorreta. O texto ajuda o leitor a jogar o jogo, e não a solucionar
problemas.
\end{enumerate}

\colorsec{Módulo 4 – Treino}

\begin{enumerate}
\item
SAEB: Analisar o uso de recursos de persuasão em textos verbais e/ou multimodais.
BNCC: EF03LP19 -- Identificar e discutir o propósito do uso de recursos de
persuasão (cores, imagens, escolha de palavras, jogo de palavras,
tamanho de letras) em textos publicitários e de propaganda, como
elementos de convencimento.
(A) Incorreta. Embora haja a presença de desenhos no anúncio, esse não é
o objetivo da campanha, visto que a conscientização é referente à
vacinação.
(B) Incorreta. A ``turma da vacinação'' é referente à conscientização
sobre a necessidade de se vacinar contra doenças específicas, não em
relação à importância de ter amizades.
(C) Incorreta. A finalidade da campanha é promover a conscientização sobre a
importância da vacinação contra o sarampo e a paralisia infantil.
(D) Correta. No cartaz, compreende-se que os elementos verbais e não
verbais apontam para a importância da vacinação contra o sarampo e a
paralisia infantil, dado o anúncio ``Vacinação contra o sarampo e a
paralisia infantil'', bem como o slogan ``Vem pra turma da vacinação''.

\item
SAEB: Analisar os efeitos de sentido de recursos multissemiótico em textos que circulam em diferentes suportes.
BNCC: EF03LP19 -- Identificar e discutir o propósito do uso de recursos de
persuasão (cores, imagens, escolha de palavras, jogo de palavras,
tamanho de letras) em textos publicitários e de propaganda, como
elementos de convencimento.
(A) Incorreta. A presença da água no cartaz reforça a ideia de que se deve ter cuidado com a água parada, que se torna um criadouro de mosquitos da dengue.
(B) Correta. O xis vermelho que ``corta'' determinados elementos indica que as situações relacionadas a eles devem ser evitadas na tentativa de se combater o mosquito da dengue.
(C) Incorreta. A representação do mosquito relaciona-se diretamente ao texto ``Combata o mosquito''.
(D) Incorreta. A imagem das larvas na água relembra que se devem evitar os criadouros do mosquito.

\item
SAEB: Julgar a eficácia de argumentos em textos.
BNCC: EF03LP19 -- Identificar e discutir o propósito do uso de recursos de
persuasão (cores, imagens, escolha de palavras, jogo de palavras,
tamanho de letras) em textos publicitários e de propaganda, como
elementos de convencimento.
(A) Incorreta. O cartaz mostra-se eficiente.
(B) Incorreta. Está, sim, claro o que motiva o cartaz.
(C) Correta. A ideia de defender o meio ambiente com a diminuição do consumo de plástico está bem clara no cartaz.
(D) Incorreta. Já há texto do cartaz diretamente dirigido ao leitor.
\end{enumerate}

\colorsec{Módulo 5 – Treino}

\begin{enumerate}
\item
SAEB: Reconhecer diferentes modos de organização composicional de textos em versos.
BNCC: EF35LP27 -- Ler e compreender, com certa autonomia, textos em
versos, explorando rimas, sons e jogos de palavras, imagens poéticas
(sentidos figurados) e recursos visuais e sonoros.
(A) Incorreta. Os versos têm tamanhos diferentes.
(B) Incorreta. Todas as estrofes têm o mesmo número de versos.
(C) Correta. Cada estrofe tem quatro versos.
(D) Nenhuma estrofe tem apenas um verso.

\item
SAEB: Reconhecer diferentes modos de organização composicional de textos em versos.
BNCC: EF35LP27 -- Ler e compreender, com certa autonomia, textos em
versos, explorando rimas, sons e jogos de palavras, imagens poéticas
(sentidos figurados) e recursos visuais e sonoros.
(A) Correta. As palavras finais ``trança'' e ``criança'' rimam entre si.
(B) Incorreta. As palavras finais ``rir'' e ``descansada'' não rimae entre si.
(C) Incorreta. A palavra ``Descuidada'' e a palavra final ``porvir'' não rimam entre si.
(D) Incorreta. As palavras finais ``trança'' e ``descansada'' não rimam entre si.

\item
SAEB: Analisar a construção de sentidos de textos em versos com base em seus elementos constitutivos.
BNCC: EF35LP31 -- Identificar, em textos versificados, efeitos de sentido
decorrentes do uso de recursos rítmicos e sonoros e de metáforas.
(A) Incorreta. Não há dificuldade em se compreender a inversão operada no verso.
(B) Correta. A inversão, que coloca ``está'' no final do verso, enfatiza o sentido do verbo para o verso e cria uma rima de final de verso com o primeiro, em que ocorre a palavra ``lá''.
(C) Incorreta. Em poemas, é típico haver inversões de palavras com objetivos rítmicos, estéticos ou semânticos.
(D) Incorreta. Não há inversão no último verso da segunda estrofe.
\end{enumerate}

\colorsec{Módulo 6 – Treino}

\begin{enumerate}
\item
SAEB: Identificar as variedades linguísticas em textos.
BNCC: EF35LP30 -- Diferenciar discurso indireto e discurso direto,
determinando o efeito de sentido de verbos de enunciação e explicando o
uso de variedades linguísticas no discurso direto, quando for o caso.
(A) Incorreta. Apesar da semelhança e da origem comum, o substantivo ``rebenta'' não é uma variação para a forma ``arrebenta'', do verbo ``arrebentar''.
(B) Incorreta. A palavra ``rebenta'' é uma flexão de ``rebento'', mas não uma variação.
(C) Correta. As palavras ``rebenta'' e ``filha'', nesse contexto, são sinônimas.
(D) Incorreta. O sentido de ``broto'' é um possível para a palavra ``rebento'', mas não é o caso do contexto.

\item
SAEB: Identificar as variedades linguísticas em textos.
BNCC: EF35LP22 -- Perceber diálogos em textos narrativos, observando o
efeito de sentido de verbos de enunciação e, se for o caso, o uso de
variedades linguísticas no discurso direto.
(A) Correta. A forma ``louro'' é anterior ao aparecimento de ``loiro''.
(B) Incorreta. Tanto a forma ``louro'' quanto a forma ``loiro'' são igualmente corretas.
(C) Incorreta. Na escrita, ainda é mais comum a forma ``louro''.
(D) Incorreta. No dia a dia, a forma ``loiro'' é mais comum que ``louro''.

\item
SAEB: Identificar as variedades linguísticas em textos.
BNCC: EF35LP30 -- Diferenciar discurso indireto e discurso direto,
determinando o efeito de sentido de verbos de enunciação e explicando o
uso de variedades linguísticas no discurso direto, quando for o caso.
(A) Incorreta. Os dois compadres utilizam as mesmas variantes da língua.
(B) Incorreta. Apesar de ter confundido seu significado, o segundo compadre reconheceu a palavra ``firme''.
(C) Incorreta. A pergunta do primeiro compadre não era sobre o programa na TV, apesar de o outro compadre ter entendido isso.
(D) Correta. O humor da piada deve-se ao fato de que o segundo compadre relacionou a forma ``firme'' à palavra ``filme'', porque as duas palavras, na variedade linguística em questão, são pronunciadas da mesma forma.
\end{enumerate}

\colorsec{Módulo 7 – Treino}

\begin{enumerate}
\item
SAEB: Analisar os efeitos de sentido decorrentes do uso dos adjetivos.
BNCC: EF03LP09 -- Identificar, em textos, adjetivos e sua função de
atribuição de propriedades aos substantivos.
(A) Incorreta. A palavra ``deserto'' caracteriza o espaço em que se
passa a narrativa.
(B) Correta. Uma das palavras que caracterizam a
guardadora de patos, a personagem principal da história, é ``velhinha''.
(C) Incorreta. O adjetivo ``linda'' caracteriza a casa da personagem.
(D) Incorreta. O adjetivo ``grande'' caracteriza o espaço da narrativa.

\item
SAEB: Analisar os efeitos de sentido decorrentes do uso dos adjetivos.
BNCC: EF03LP23 -- Analisar o uso de adjetivos em cartas dirigidas a
veículos da mídia impressa ou digital (cartas do leitor ou de reclamação
a jornais ou revistas), digitais ou impressas.
(A) Correta. Segundo os leitores, os artistas da época retratava D. João VI como gordinho.
(B) Incorreta. A informação de que D. João VI era guloso não tem como fonte os estudantes do quinto ano, mas um texto que eles leram na revista.
(C) Incorreta. Fala-se em uma fama de ser guloso de D. João VI, mas isso não o caracteriza como famoso por ser o rei de Portugal.
(D) Incorreta. Apesar de se falar em festas promovidas por D. João VI, não se caracteriza o rei como festeiro.

\item
SAEB: Analisar os efeitos de sentido decorrentes do uso dos advérbios.
Não há correspondência com a BNCC do terceiro ano.
(A) Incorreta. O advérbio está ligado a um adjetivo.
(B) Incorreta. O advérbio está modificando um adjetivo.
(C) Correta. Por meio da expressão ``mais... que...'', em que aparece o advérbio ``mais'', cria-se uma comparação de superioridade.
(D) Incorreta. O advérbio tem sentido de intensificador.
\end{enumerate}

\colorsec{Simulado 1}

\begin{enumerate}
\item
SAEB: Identificar a ideia central o texto.
BNCC: EF35LP03 -- Identificar a ideia central do texto, demonstrando
compreensão global.
(A) Incorreta. A Companhia Delas é a criadora do espetáculo, mas o tema do texto é o espetáculo em si.
(B) Correta. A notícia trata da peça ``Maria e os Insetos''.
(C) Incorreta. ``Mary e os monstros marinhos'' é uma das peças criadas
pela mesma companhia, mencionada no texto, mas não é o tema.
(D) Incorreta. Thaís Medeiros faz parte da Cia. Delas, grupo que criou a
peça ``Maria e os insetos'', tema da notícia.

\item
SAEB: Localizar informação explícita.
BNCC: EF15LP03 -- Localizar informações explícitas em textos.
(A) Incorreta. A ovelha não recebeu sua parte.
(B) Incorreta. A cabra ficou sem carne.
(C) Incorreta. Nenhuma parte foi dada à vaca.
(D) Correta. Explicitamente, o leão se apossou de todas as quatro partes da carne.

\item
SAEB: Inferir informações implícitas em textos.
BNCC: EF35LP04 -- Inferir informações implícitas nos textos lidos.
(A) Correta. Está implícita a ideia de que o narrador-personagem não esteve com seus companheiros depois do incidente..
(B) Incorreta. O narrador-personagem não morreu afogado.
(C) Incorreta. O narrador-personagem não teve notícias de seus companheiros.
(D) Incorreta. O narrador-personagem precisou caminhar pelo mar até chegar à terra firme.

\item
SAEB: Inferir o sentido de palavras ou expressões em textos.
BNCC: EF35LP05 -- Inferir o sentido de palavras ou expressões
desconhecidas em textos, com base no contexto da frase ou do texto.
(A) Incorreta. Essa informação não se refere à expressão destacada.
(B) Incorreta. No texto, verifica-se que os pais estavam, daquela vez,
esperando uma menina.
(C) Correta. Entende-se que, ao estar ``esperando criança'', a mãe estava
grávida.
(D) Incorreta. Essa informação não se refere à expressão destacada.

\item

\item

\item

\item

\item

\item

\item

\item

\item

\item

\item

\end{enumerate}

\colorsec{Simulado 2}

\begin{enumerate}
\item
SAEB: Reconhecer diferentes gêneros textuais.
BNCC: EF35LP29 -- Identificar, em narrativas, cenário, personagem central,
conflito gerador, resolução e o ponto de vista com base no qual
histórias são narradas, diferenciando narrativas em primeira e terceira
pessoas.
(A) Incorreta. Não se trata de uma fábula, pois não contém uma moral.
(B) Incorreta. Não se trata de um poema, mas de um texto em prosa.
(C) Correta. A história da Chapeuzinho Vermelho é narrada em um conto.
(D) Incorreta. Não se trata de uma anedota, pois não tem o objetivo de fazer rir.

\item
SAEB: Identificar as marcas de organização de textos dramáticos.
BNCC: EF35LP24 -- Identificar funções do texto dramático (escrito para ser encenado) e sua organização
por meio de diálogos entre personagens e marcadores das falas das personagens e de cena.
(A) Incorreta. Textos dramáticos não apresentam narrador.
(B) Correta. O desenrolar por meio de diálogo é uma marca característica do texto dramático.
(C) Incorreta. A presença de personagens não é exclusividade do texto dramático.
(D) Incorreta. Nível de informatividade não é uma marca a ser avaliada no texto dramático.

\item
SAEB: Analisar elementos constitutivos de gêneros textuais diversos.
BNCC: EF35LP16 -- Identificar e reproduzir, em notícias, manchetes, lides e
corpo de notícias simples para público infantil e cartas de reclamação
(revista infantil), digitais ou impressos, a formatação e diagramação
específica de cada um desses gêneros, inclusive em suas versões orais.
(A) Correta. Em entrevistas, as perguntas feitas pelo entrevistador
normalmente são apresentadas em negrito para que possam ser
diferenciadas das respostas do entrevistado.
(B) Incorreta. As respostas do entrevistado normalmente são apresentadas
sem estar em negrito.
(C) Incorreta. Os trechos em negrito não mostram esse assunto.
(D) Incorreta. Essa é uma característica de um texto narrativo, e não de
um texto informativo como a entrevista.

\item
SAEB: Inferir o sentido de palavras ou expressões em textos.
BNCC: EF35LP05 -- Inferir o sentido de palavras ou expressões desconhecidas em textos, com base no
contexto da frase ou do texto.
(A) Incorreta. Os verbos ``confeccionar'' e ``purificar'' não são próximos semanticamente.
(B) Correta. O verbo ``fabricar'' pode ser considerado um sinônimo de ``confeccionar''.
(C) Incorreta. Os verbos ``confeccionar'' e ``configurar'' não são próximos semanticamente.
(D) Incorreta. Os verbos ``confeccionar'' e ``organizar'' não são próximos semanticamente.

\item
\item
\item
\item
\item
\item
\item
\item
\item
\item
\item
\end{enumerate}

\colorsec{Simulado 3}

\begin{enumerate}
\item
SAEB: Reconhecer em textos o significado de palavras derivadas a partir de seus afixos.
BNCC: EF03LP10 -- Reconhecer prefixos e sufixos produtivos na formação de
palavras derivadas de substantivos, de adjetivos e de verbos,
utilizando-os para compreender palavras e para formar novas palavras.
(A) Incorreta. O prefixo ``in-'' não reforça o sentido de ``grato''.
(B) Incorreta. O prefixo ``in-'' não ressalta o sentido de ``grato''.
(C) Incorreta. O prefixo ``in-'' não reitera o sentido de ``grato''.
(D) Correta. A palavra ``ingrato'', graças ao prefixo ``in-'', tem sentido contrário ao da palavra ``grato''.

\item
SAEB: Inferir informações implícitas em textos.
BNCC: EF35LP04 -- Inferir informações implícitas nos textos lidos.
(A) Correta. Ali Babá trabalhava para comerciantes ricos, informação
implícita no trecho ``Os ricos comerciantes já conheciam Ali Babá e
gostavam muito de seu serviço''.
(B) Incorreta. Os camelos eram dos ricos comerciantes.
(C) Incorreta. Os mercadores davam dinheiro a mais para Ali Babá por
terem simpatia com ele.
(D) Incorreta. A água e o alimento eram para os camelos dos viajantes.

\item
SAEB: Reconhecer os usos da pontuação.
BNCC: EF03LP07 -- Identificar a função na leitura e usar na escrita ponto final, ponto
de interrogação, ponto de exclamação e, em diálogos (discurso direto), dois-pontos e
travessão.
(A) Incorreta. Os dois-pontos podem ser usados para explicar o sentido de um termo, mas não é o caso do texto.
(B) Incorreta. Os dois-pontos podem ser usados para acrescentar uma informação, mas não é o caso do texto.
(C) Incorreta. O discurso indireto prescinde do uso dos dois-pontos.
(D) Correta. Como é comum, os dois-pontos foram usados para indicar discurso direto.

\item
SAEB: Analisar os efeitos de sentido decorrentes do uso dos adjetivos.
BNCC: EF03LP09 -- Identificar, em textos, adjetivos e sua função de atribuição de propriedades
aos substantivos.
(A) Incorreta. O narrador dá conta de descrever tio Benedito como um velhote bastante debilitado.
(B) Correta. Esses adjetivos e toda a descrição do texto mostram tio Benedito como alguém que se desgastou muito na vida.
(C) Incorreta. O vigor da mocidade já havia abandonado tio Benedito.
(D) Correta. Tio Benedito estava mais para um velho do que para um jovenzinho.

\item
\item
\item
\item
\item
\item
\item
\item
\item
\item
\item
\end{enumerate}

\colorsec{Simulado 4}

\begin{enumerate}
\item
SAEB: Analisar o uso de recursos de persuasão em textos verbais e/ou multimodais.
BNCC: EF03LP19: Identificar e discutir o propósito do uso de recursos de persuasão (cores,
imagens, escolha de palavras, jogo de palavras, tamanho de letras) em textos publicitários
e de propaganda, como elementos de convencimento.
(A) Incorreta. Embora o cartaz indique que se lavem as mãos duas vezes,
essa ação não ocorre após receber a entrega, mas sim uma vez ao receber
a entrega e outra antes de consumir o produto.
(B) Incorreta. As mãos devem ser lavadas antes de consumir o produto, e
não depois.
(C) Incorreta. No cartaz, não há informações para lavar as mãos ao descartar a embalagem.
(D) Correta. Ao ler o cartaz da campanha, identifica-se que ela
incentiva e instrui o leitor a lavar as mãos duas vezes, após a pessoa
receber a entrega e antes de usar o produto.

\item
SAEB: Identificar elementos constitutivos de textos narrativos.
BNCC: EF35LP26 -- Ler e compreender, com certa autonomia, narrativas
ficcionais que apresentem cenários e personagens, observando os
elementos da estrutura narrativa: enredo, tempo, espaço, personagens,
narrador e a construção do discurso indireto e discurso direto.
(A) Incorreta. Não existe narrador-personagem na narrativa.
(B) Incorreta. Os irmãos não narram a história.
(C) Correta. A história é apresentada por meio de um narrador-observador.
(D) Incorreta. O narrador da história é observador.

\item
SAEB: Inferir o sentido de palavras ou expressões em textos.
BNCC: EF35LP05 -- Inferir o sentido de palavras ou expressões desconhecidas em textos, com base no
contexto da frase ou do texto.
(A) Incorreta. O sentido de ``ganhar'', na expressão, não é o literal.
(B) Correta. A expressão ``ganhar a vida'' relaciona-se à capacidade de alguém de sustentar a própria vida.
(C) Incorreta. A expressão não está relacionada a conseguir uma ocupação, mas a conseguir o sustento a partir da ocupação.
(D) Incorreta. O sentido da expressão está voltado para a própria pessoa, e não para o sustento de outra.

\item
SAEB: Reconhecer em textos o significado de palavras derivadas a partir de seus afixos.
BNCC: EF03LP10 -- Reconhecer prefixos e sufixos produtivos na formação de
palavras derivadas de substantivos, de adjetivos e de verbos,
utilizando-os para compreender palavras e para formar novas palavras.
(A) Incorreta. Há alterações no radical da palavra que não estão contempladas nessas opções.
(B) Incorreta. O aluno retirou incorretamente alguns fonemas para formar
os nomes das árvores.
(C) Correta. Nas imagens, tem-se: maçã, melão e mamão. Os nomes das respectivas árvores são: macieira, meloeiro e mamoeiro.
(D) Incorreta. O sufixo foi simplesmente acrescentado ao nome da fruta, sem que se entendesse a regra.

\item
\item
\item
\item
\item
\item
\item
\item
\item
\item
\item
\end{enumerate}