\chapter{Respostas}
\pagestyle{plain}
\footnotesize

\pagecolor{gray!40}

\section*{Módulo 1 – Treino}

\begin{enumerate}
\item
a) Correta. A senha do cartão identifica o dono do cartão para que não haja fraudes.
b) Incorreta. A senha do cartão de crédito não mede nenhuma grandeza.
c) Incorreta. A senha do cartão de crédito não quantifica nenhum grupo de objetos.
d) Incorreta. A senha do cartão de crédito não segue uma ordem lógica e sequencial.
SAEB: Reconhecer o que os números naturais indicam em diferentes
situações: quantidade, ordem, medida ou código de identificação.
BNCC: EF02MA01 - Comparar e ordenar números naturais (até a ordem de
centenas) pela compreensão de características do sistema de numeração decimal (valor
posicional e função do zero).

\item
a) Incorreta. O aluno não entendeu que teria que diferenciar entre pares e ímpares.
b) Incorreta. O aluno indicou o 20° número ao invés do 30°.
c) Correta. Como temos a metade de números pares, basta multiplicar o 30 por 2.
d) Incorreta. O aluno indicou o 45° número ao invés do 30°.
SAEB: Identificar a posição ordinal de um objeto ou termo em uma sequência (1º, 2º etc.).
BNCC: EF02MA01 - Comparar e ordenar números naturais (até a ordem de centenas) pela compreensão de características do sistema de numeração decimal (valor
posicional e função do zero).

\item
a) Correta. A ordem das dezenas está representeada pelo número 3.
b) Incorreta. A ordem das dezenas está representeada pelo número 8.
c) Incorreta. A ordem das dezenas está representeada pelo número 4.
d) Incorreta. A ordem das dezenas está representeada pelo número 2.
SAEB: Identificar a ordem ocupada por um algarismo ou seu valor posicional (ou valor relativo) em um número natural de até 3 ordens. BNCC: EF02MA03 - Comparar quantidades de objetos de dois conjuntos, por estimativa e/ou por correspondência (um a um, dois a dois, entre outros), para indicar ``tem mais'', ``tem menos'' ou ``tem a mesma quantidade'', indicando, quando for o caso, quantos a mais e quantos a menos.
\end{enumerate}

\section*{Módulo 2 – Treino}

\begin{enumerate}
\item
a) Incorreta. O aluno subtraiu os números.
b) Incorreta. O aluno errou a adição da ordem das unidades.
c) Correta. A soma dos dois números é igual a 236 + 132 = 368.
d) Incorreta. O aluno errou a adição da ordem das dezenas.
SAEB: Resolver problemas de adição ou de subtração, envolvendo
números naturais de até 3 ordens, com os significados de juntar,
acrescentar, separar ou retirar.
BNCC: EF02MA06 - Resolver e elaborar problemas de adição e de subtração, envolvendo números de até três ordens, com os significados de juntar, acrescentar, separar,
retirar, utilizando estratégias pessoais.

\item
a) Correta. A soma dos números resulta em 452.
b) Incorreta. A soma dos números resulta em 462.
c) Incorreta. A soma dos números resulta em 472.
d) Incorreta. A soma dos números resulta em 482.
SAEB: Compor ou decompor números naturais de até 3 ordens por
meio de diferentes adições.
BNCC: EF02MA04 -- Compor e decompor números naturais de até três ordens,
com suporte de material manipulável, por meio de diferentes adições.

\item
a) Incorreta. O aluno adicionou somente os municípios do Paraná e de Santa Catarina.
b) Incorreta. O aluno adicionou somente os municípios de Santa Catarina e os do Rio Grande do Sul.
c) Incorreta. O aluno adicionou somente os municípios de Paraná e os do Rio Grande do Sul.
d) Correta. O aluno adicionou corretamente, obtendo: 399 + 295 + 497 = 1191.
SAEB: Calcular o resultado de adições e subtrações, envolvendo número naturais de até 3 ordens.
BNCC: EF02MA06 -- Resolver e elaborar problemas de adição e de subtração,
envolvendo números de até três ordens, com os significados de juntar, acrescentar, separar,
retirar, utilizando estratégias pessoais.
\end{enumerate}

\section*{Módulo 3 – Treino}

\begin{enumerate}
\item
a) Incorreta. Os braços da menina não podem calcular o espaço ocupado pela lousa.
b) Correta. Os braços da menina medem o comprimento entre a borda da lousa e o ponto marcado.
c) Incorreta. Os braços da menina não podem medir a massa da lousa
d) Incorreta. Os braços da menina não podem calcular o tempo.
SAEB: Identificar a medida de comprimento, da capacidade ou da
massa de objetos, dada a imagem de um instrumento de medida.
BNCC: EF02MA16 -- Estimar, medir e comparar comprimentos de lados de
salas (incluindo contorno) e de polígonos, utilizando unidades de medida não padronizadas e
padronizadas (metro, centímetro e milímetro) e instrumentos adequados.

\item
a) Incorreta. O marcador do tubo marca bem mais de 10 mL.
b) Incorreta. O marcador do tubo marca mais de 20 mL.
c) Correta. O marcador marca um pouquinho mais do que 40 mL; logo, mesmo que baixasse um pouco o nível com a retirada do conta-gotas, ele subiria novamente acima de 30 mL.
d) Incorreta. É impossível fazer uma estimativa tão precisa nesse caso, apenas com a observação.
SAEB: Estimar/inferir medida de comprimento, capacidade ou
massa de objetos, utilizando unidades de medida convencionais ou não ou
medir comprimento, capacidade ou massa de objetos.
BNCC: EF02MA17 -- Estimar, medir e comparar capacidade e massa,
utilizando estratégias pessoais e unidades de medida não padronizadas ou padronizadas (litro, mililitro,
grama e quilograma).

\item
a) Incorreta. O aluno inverteu A e B, além de adicionar somente os lados desconhecidos.
b) Incorreta. O aluno adicionou os lados, mas esqueceu de adicionar os valores de A e B.
c) Correta. Os lados A e B são iguais aos seus correspondentes e a adição dos lados é igual a 2 + 2 + 5 + 2 + 2 + 5 = 18.
d) Incorreta. O aluno inverteu os valores de A e B.
SAEB: Comparar comprimentos, capacidades ou massas ou ordenar
imagens de objetos com base na comparação visual de seus comprimentos,
capacidades ou massas.
Estimar/inferir medida de comprimento, capacidade ou massa de
objetos, utilizando unidades de medida convencionais ou não ou medir
comprimento, capacidade ou massa de objetos.
BNCC: EF02MA16 -- Estimar, medir e comparar comprimentos de lados de
salas (incluindo contorno) e de polígonos, utilizando unidades de medida não padronizadas e
padronizadas (metro, centímetro e milímetro) e instrumentos adequados.
\end{enumerate}

\section*{Módulo 4 – Treino}

\begin{enumerate}
\item
a) Incorreta. O aluno contou dois dias a menos.
b) Incorreta. O aluno contou dois dias a menos
c) Correta. O aluno contou corretamente a quantidade de dias.
d) Incorreta. O aluno contou um dia a mais.
SAEB: Determinar a data de início, a data de término ou a
duração de um acontecimento entre duas datas.
BNCC: EF02MA18 -- Indicar a duração de intervalos de tempo entre duas
datas, como dias da semana e meses do ano, utilizando calendário, para planejamentos e organização
de agenda.

\item
a) Incorreta. Segunda-feira será dia 17.
b) Incorreta. Terça-feira será dia 18.
c) Incorreta. Quarta-feira será dia 19.
d) Correta. Quinta-feira será dia 20.
SAEB: Identificar datas, dias da semana ou meses do ano em calendário ou escrever uma data, apresentando o dia, o mês e o ano.
BNCC: EF02MA18 -- Indicar a duração de intervalos de tempo entre duas datas, como dias da semana e meses do ano, utilizando calendário, para planejamentos e organização de agenda.

\item
a) Incorreta. O aluno contou 15 minutos a menos.
b) Correta. O aluno contou corretamente 30 minutos até fazer 10 horas, mais 15 minutos do número 3.
c) Incorreta. O aluno contou 5 minutos a mais.
d) Incorreta. O aluno contou 15 minutos a mais.
SAEB: Determinar o horário de início, o horário de término ou a
duração de um acontecimento.
BNCC: EF02MA19 -- Medir a duração de um intervalo de tempo por meio de
relógio digital e registrar o horário do início e do fim do intervalo.
\end{enumerate}

\section*{Módulo 5 – Treino}

\begin{enumerate}
\item
a) Incorreta. O aluno adicionou o troco à cédula de R\$ 200,00.
b) Incorreta. O aluno desconsiderou o valor do troco.
c) Correta. O aluno subtraiu o valor do troco da cédula dada por Claudia: 200 -- 87 = 113.
d) Incorreta. O aluno somente adicionou as cédulas de troco, desconsiderando a cédula dada.
SAEB: Relacionar valores de moedas e/ou cédulas do sistema
monetário brasileiro, com base nas imagens desses objetos.
Resolver problemas que envolvam moedas e/ou cédulas do sistema
monetário brasileiro.
BNCC: EF02MA20 -- Estabelecer a equivalência de valores entre moedas e cédulas do sistema monetário brasileiro para resolver situações cotidianas.

\item
a) Incorreta. O aluno esqueceu de acrescentar uma das cédulas de 100.
b) Incorreta. O aluno achou que as moedas de 5 centavos valiam 5 reais, além de esquecer uma das cédulas de 100,00.
c) Correta. O aluno adicionou corretamente: 100 + 100 + 1 + 0,05 + 0,05 = 201,10
d) Incorreta. O aluno achou que as moedas de 5 centavos valiam 5 reais.
SAEB: Relacionar valores de moedas e/ou cédulas do sistema
monetário brasileiro, com base nas imagens desses objetos.
Resolver problemas que envolvam moedas e/ou cédulas do sistema monetário brasileiro.
BNCC: EF02MA20 -- Estabelecer a equivalência de valores entre moedas e
cédulas do sistema monetário brasileiro para resolver situações cotidianas.

\item
a) Incorreta. A adição perfaz 350,00.
b) Incorreta. A adição perfaz 226,00.
c) Incorreta. A adição perfaz 226,00
d) Correta. A adição perfaz 326,00.
SAEB: Relacionar valores de moedas e/ou cédulas do sistema
monetário brasileiro, com base nas imagens desses objetos.
Resolver problemas que envolvam moedas e/ou cédulas do sistema
monetário brasileiro.
BNCC: EF02MA20 -- Estabelecer a equivalência de valores entre moedas e cédulas do sistema monetário brasileiro para resolver situações cotidianas.
\end{enumerate}

\section*{Módulo 6 – Treino}

\begin{enumerate}
\item
a) Incorreta. Cair de um lugar alto, caso não se tenha segurança, é certeza, em virtude da gravidade.
b) Incorreta. Um raio em dia de chuva é muito provável de acontecer.
c) Incorreta. Dormir todos os dias é certeza, ainda que ao longo da vida possamos passar algumas poucas noites acordados.
d) Correta. Em contato com o fogo a queimadura é certa; logo, não se queimar nesse caso é impossível.
SAEB:2 Classificar resultados de eventos cotidianos aleatórios como ``pouco prováveis'', ``muito prováveis'', ``certos'' ou ``impossíveis''.BNCC: EF02MA21 -- Classificar resultados de eventos cotidianos aleatórios como ``pouco prováveis'', ``muito prováveis'', ``improváveis'' e ``impossíveis''.

\item
a) Incorreta. Situação pouco provável. Quanto mais treinamos, melhores somos em determinada atividade.
b) Incorreta. Situação impossível. Sorrir quando feliz é uma reação
automática do corpo.
c) Correta. É muito provável que alguém se queime ao manipular fogo de
forma descuidada.
d) Incorreta. Situação pouco provável. Em dias frios só suaremos se
praticarmos exercícios físicos intensos.
SAEB: Classificar resultados de eventos cotidianos aleatórios como ``pouco prováveis'', ``muito prováveis'', ``certos'' ou ``impossíveis''. BNCC: EF02MA21 -- Classificar resultados de eventos cotidianos aleatórios como ``pouco prováveis'', ``muito prováveis'', ``improváveis'' e ``impossíveis''.

\item
a) Correta. Situação pouco provável. Geralmente fazemos isso todos os
dias; algumas pessoas podem ficar alguns dias sem fazer.
b) Incorreta. Impossível. Por menor que seja a dor, a pele ralada sempre gera incômodo.
c) Incorreta. Impossível. Ao praticar esportes o atrito aumenta a
temperatura corporal e o suor é ativado automaticamente para regular.
d) Incorreta. Impossível. Não tem como ficar seco estando em contato direto com a água.
SAEB:2 Classificar resultados de eventos cotidianos aleatórios como ``pouco prováveis'', ``muito prováveis'', ``certos'' ou ``impossíveis''. BNCC: EF02MA21 -- Classificar resultados de eventos cotidianos aleatórios como ``pouco prováveis'', ``muito prováveis'', ``improváveis'' e ``impossíveis''.
\end{enumerate}

\section*{Módulo 7 – Treino}

\begin{enumerate}
\item
a) Correta. A nota mais alta de Jéssica foi em Artes.
b) Incorreta. Em ciências, Jéssica tirou a terceira nota mais alta.
c) Incorreta. Em Geografia, assim como em História, Jéssica tirou a nota mais baixa.
d) Incorreta. Em matemática, Jéssica tirou a segunda nota mais alta.
SAEB: Ler/identificar ou comparar dados estatísticos ou
informações expressas em tabelas (simples ou de dupla entrada).
BNCC: EF02MA22 -- Comparar informações de pesquisas apresentadas por meio
de tabelas de dupla entrada e em gráficos de colunas simples ou barras, para melhor
compreender aspectos da realidade próxima.

\item
a) Incorreta. O chocolate foi o mais vendido.
b) Incorreta. O pudim foi o terceiro mais vendido.
c) Correta. O quindim foi o menos vendido.
d) Incorreta. O sorvete foi o segundo mais vendido.
SAEB: Ler/identificar ou comparar dados estatísticos ou
informações expressas em tabelas (simples ou de dupla entrada).
BNCC: EF02MA22 -- Comparar informações de pesquisas apresentadas por meio
de tabelas de dupla entrada e em gráficos de colunas simples ou barras, para melhor
compreender aspectos da realidade próxima.

\item
a) Incorreta. O aluno adicionou os preços mais altos.
b) Incorreta. O aluno adicionou os preços da loja 2.
c) Incorreta. O aluno adicionou os preços da loja 1.
d) Correta. O aluno adicionou os preços mais baixos: 125 + 70 + 650 + 86 + 23 = R\$ 954,00.
SAEB: Ler/identificar ou comparar dados estatísticos ou
informações expressas em tabelas (simples ou de dupla entrada).
BNCC: EF02MA22 -- Comparar informações de pesquisas apresentadas por meio
de tabelas de dupla entrada e em gráficos de colunas simples ou barras, para melhor
compreender aspectos da realidade próxima.
\end{enumerate}

\section*{Módulo 8 – Treino}

\begin{enumerate}
\item
a) Incorreta. O aluno dividiu os pares pela metade.
b) Incorreta. O aluno entendeu pares de meia e pés de meia como sendo a
mesma coisa.
c) Correta. O aluno multiplicou os pares por 2, entendendo que cada par
tem dois pés de meia.
d) Incorreta. O aluno multiplicou os pares de meia por três.
SAEB: Resolver problemas de multiplicação ou de divisão (por 2,
3, 4 ou 5), envolvendo números naturais, com os significados de formação
de grupos iguais ou proporcionalidade (incluindo dobro, metade, triplo
ou terça parte).
BNCC: EF02MA08 -- Resolver e elaborar problemas envolvendo dobro, metade, triplo e terça parte,
com o suporte de imagens ou material manipulável, utilizando estratégias
pessoais.

\item
a) Incorreta. O aluno não multiplicou o valor da prateleira por nenhum fator.
b) Incorreta. O aluno dobrou o valor da prateleira.
c) Incorreta. O aluno triplicou o valor da prateleira.
d) Correta. O aluno quadruplicou o valor da prateleira.
SAEB: Resolver problemas de multiplicação ou de divisão (por 2,
3, 4 ou 5), envolvendo números naturais, com os significados de formação
de grupos iguais ou proporcionalidade (incluindo dobro, metade, triplo
ou terça parte).
BNCC: EF02MA08 -- Resolver e elaborar problemas envolvendo dobro, metade, triplo e terça parte,
com o suporte de imagens ou material manipulável, utilizando estratégias
pessoais.

\item
a) Correta. Mário percorreu 250 km a mais, que é exatamente a metade do percurso original.
b) Incorreta. O aluno entendeu que, adicionando a metade do valor original, resultaria no dobro.
c) Incorreta. O aluno entendeu que, adicionando a metade do valor original, resultaria no triplo.
d) Incorreta. O aluno entendeu que, adicionando a metade do valor original, resultaria no quádruplo.
SAEB: Analisar argumentações sobre a resolução de problemas de adição,
subtração, multiplicação ou divisão envolvendo números naturais.
BNCC: EF02MA08 -- Resolver e elaborar problemas envolvendo dobro, metade,
triplo e terça parte, com o suporte de imagens ou material manipulável, utilizando estratégias
pessoais.
\end{enumerate}

\section*{Simulado 1}

\begin{enumerate}
\item
a) Correta. O número que aparece na porta de entrada da casa representa a identificação do endereço da casa.
b) Incorreta. O número não representa a medida da casa.
c) Incorreta. O número não corresponde a 01 unidade de casa; sendo assim, não representa uma quantidade.
d) Incorreta. Mesmo se tratando de um número de identificação de endereço, ele não expressa a posição da casa em relação as outras.
SAEB: Reconhecer o que os números naturais indicam em diferentes
situações: quantidade, ordem, medida ou código de identificação.
BNCC: EF02MA01 -- Comparar e ordenar números naturais (até a ordem de
centenas) pela compreensão de características do sistema de numeração
decimal (valor posicional e função do zero).

\item
a) Incorreta. A fila se inicia em Alex, por isso ele está na primeira posição da fila.
b) Correta. Beatriz está na segunda posição da fila.
c) Incorreta. Carlos está na terceira posição da fila.
d) Incorreta. Daniel está na quarta posição da fila.
SAEB: Identificar a posição ordinal de um objeto ou termo em uma
sequência (1º, 2º etc.).
BNCC: EF02MA01 -- Comparar e ordenar números naturais (até a ordem de
centenas) pela compreensão de características do sistema de numeração
decimal (valor posicional e função do zero).

\item
a) Incorreta. 123 é o menor número possível de se formar com os algarismos 1,2 e 3.
b) Incorreta. 231 não é o maior número possível de se formar com os algarismos 1,2 e 3.
c) Incorreta. 312 não é o maior número possível de se  formar com os algarismos 1,2 e 3.
d) Correta. 321 é o maior número possível de se formar com os algarismos 1,2 e 3.
SAEB: Escrever números naturais de até 3 ordens em sua
representação por algarismos ou em língua materna ou associar o registro
numérico de números naturais de até 3 ordens ao registro em língua
materna.
BNCC: EF02MA01 -- Comparar e ordenar números naturais (até a ordem de
centenas) pela compreensão de características do sistema de numeração
decimal (valor posicional e função do zero).

\item
a) Incorreta. Nesse caso, efetuou-se a soma incorretamente: 45 + 75 = 110; depois; a subtração foi realizada corretamente: 110 -- 15 = 95.
b) Correta. Nesse caso, a soma e a subtração foram efetuadas corretamente: 45 + 75 = 120 e 120 -- 15 = 105.
c) Incorreta. Efetuou-se a soma corretamente: 45 + 75 = 120; depois
subtraiu-se incorretamente: 120 -- 15 = 110.
d) Incorreta. Efetuou-se somente a soma, sem que se fizesse a subtração: 45 + 75 = 120.
SAEB: Calcular o resultado de adições e subtrações, envolvendo
número naturais de até 3 ordens.
BNCC: EF02MA06 -- Resolver e elaborar problemas de adição e de subtração,
envolvendo números de até três ordens, com os significados de juntar,
acrescentar, separar, retirar, utilizando estratégias pessoais ou
convencionais.

\item
a) Incorreta. A composição das cédulas gera um valor de 180 reais.
b) Incorreta. A composição das cédulas gera um valor de 152 reais.
c) Correta. A composição das cédulas gera um valor de 175 reais.
d) Incorreta. A composição das cédulas gera um valor de 182 reais.
SAEB: Compor ou decompor números naturais de até 3 ordens por
meio de diferentes adições.
BNCC: EF02MA04 -- Compor e decompor números naturais de até três ordens,
com suporte de material manipulável, por meio de diferentes adições.

\item
a) Incorreta. A ovelha não é o animal mais baixo, pois ela é mais alta
que o porco e o pato.
b) Correta. A vaca é o animal mais pesado.
c) Incorreta. O pato é o animal mais baixo, a vaca é o animal mais alto.
d) Incorreta. O porco não é o animal leve; o pato é o animal mais leve.
SAEB: Comparar comprimentos, capacidades ou massas ou ordenar
imagens de objetos com base na comparação visual de seus comprimentos,
capacidades ou massas.
BNCC: EF02MA16 -- Estimar, medir e comparar comprimentos de lados de
salas (incluindo contorno) e de polígonos, utilizando unidades de medida
não padronizadas e padronizadas (metro, centímetro e milímetro) e
instrumentos adequados. EF02MA17 - Estimar, medir e comparar capacidade
e massa, utilizando estratégias pessoais e unidades de medida não
padronizadas ou padronizadas (litro, mililitro, grama e quilograma).

\item
a) Incorreta. 5 metros equivalem a 25 passos de Mariana.
b) Incorreta. 6 metros equivalem a 30 passos de Mariana.
c) Incorreta. 7 metros equivalem a 35 passos de Mariana.
d) Correta. 8 metros equivalem a 40 passos de Mariana, pois a cada 5 passos temos 1 metro; então, em 40 passos, teremos 8 metros.
SAEB: Estimar/inferir medida de comprimento, capacidade ou massa
de objetos, utilizando unidades de medida convencionais ou não ou medir
comprimento, capacidade ou massa de objetos.
BNCC: EF02MA16 -- Estimar, medir e comparar comprimentos de lados de
salas (incluindo contorno) e de polígonos, utilizando unidades de medida
não padronizadas e padronizadas (metro, centímetro e milímetro) e
instrumentos adequados. EF02MA17 - Estimar, medir e comparar capacidade
e massa, utilizando estratégias pessoais e unidades de medida não
padronizadas ou padronizadas (litro, mililitro, grama e quilograma).

\item
a) Correta. Daniel acorda às 6 horas da manhã e entre o horário de acordar e o horário de trabalhar passam-se dez horas.
b) Incorreta. Ele toma banho às 8 horas da noite, não às 10 horas. Entre o horário de trabalhar e o horário de tomar banho passam-se quatro horas.
c) Incorreta. Ele faz o lanche da tarde às 5 horas, não às 3 horas. Entre o horário de lanchar e o horário de tomar banho passam-se três horas.
d) Incorreta. Ele trabalha às 4 horas da tarde, não às 8 horas da noite. Entre o horário de acordar e o horário de lanchar passam-se onze horas.
SAEB: Identificar sequência de acontecimentos relativos a um dia.
BNCC: EF02MA19 -- Medir a duração de um intervalo de tempo por meio de
relógio digital e registrar o horário do início e do fim do intervalo.

\item
a) Incorreta. A quinta-feira de duas semanas atrás caiu no dia 3.
b) Correta. A quinta-feira da semana anterior caiu no dia 10.
c) Incorreta. A quinta-feira da semana posterior caiu no dia 24.
d) Incorreta. A quinta-feira de duas semanas posteriores caiu no dia 31.
SAEB: Identificar datas, dias da semana ou meses do ano em
calendário ou escrever uma data, apresentando o dia, o mês e o ano.
BNCC: EF02MA18 -- Indicar a duração de intervalos de tempo entre duas
datas, como dias da semana e meses do ano, utilizando calendário, para
planejamentos e organização de agenda.

\item
a) Incorreta. O aluno considerou 50 centavos a menos.
b) Correta. São 14 moedas de 1 real, 9 moedas de 50 centavos, 8 moedas
de 25 centavos, 10 moedas de 10 centavos e 10 moedas de 5 centavos =
14,00 + 4,50 + 2,00 + 1,00 + 0,50 = 22 reais.
c) Incorreta. O aluno considerou 50 centavos a mais.
d) Incorreta. O aluno considerou 1 real a mais.
SAEB: Relacionar valores de moedas e/ou cédulas do sistema
monetário brasileiro, com base nas imagens desses objetos.
BNCC: Estabelecer a equivalência de valores entre moedas e
cédulas do sistema monetário brasileiro para resolver situações
cotidianas.

\item
a) Incorreta. Não é muito provável que Gabriela consiga abrir o portão
na primeira tentativa, uma vez que há 5 chaves diferentes e ela não sabe
qual delas é a correta.
b) Incorreta. Não é provável que Gabriela consiga abrir o portão na
primeira tentativa, já que há 5 chaves diferentes e ela não sabe qual
delas é a correta.
c) Correta. É pouco provável que Gabriela consiga abrir o portão na
primeira tentativa, pois há 5 chaves diferentes e ela não sabe qual
delas é a correta.
d) Incorreta. Não é impossível que Gabriela consiga abrir o portão na
primeira tentativa, porque a chave correta está entre as outras chaves
no molho.
SAEB: Classificar resultados de eventos cotidianos aleatórios
como ``pouco prováveis'', ``muito prováveis'', ``certos'' ou
``impossíveis''.
BNCC: EF02MA21 -- Classificar resultados de eventos cotidianos aleatórios
como ``pouco prováveis'', ``muito prováveis'', ``improváveis'' e ``impossíveis''.

\item
a) Incorreta. Alana foi a pessoa que fez menos pontos.
b) Correta. Bruno fez 56 pontos e foi o jogador que mais pontuou;
portanto, é o ganhador da brincadeira.
c) Incorreta. Carla fez 48 pontos, porém Bruno fez mais pontos do que ela.
d) Incorreta. Denílson fez apenas 39 pontos e não foi o suficiente para ser o vencedor.
SAEB: Ler/identificar ou comparar dados estatísticos ou
informações expressas em tabelas (simples ou de dupla entrada).
BNCC: EF02MA22 -- Comparar informações de pesquisas apresentadas por meio
de tabelas de dupla entrada e em gráficos de colunas simples ou barras,
para melhor compreender aspectos da realidade próxima.

\item
a) Incorreta. O basquete só teve 400 votos.
b) Correta. O futsal foi o esporte com maior número de votos, obtendo 700 votos.
c) Incorreta. O handebol só obteve 300 votos.
d) Incorreta. O vôlei teve 500 votos, porém outros esportes tiveram mais votos.
SAEB: Ler/identificar ou comparar dados estatísticos expressos
em gráficos (barras simples, colunas simples ou pictóricos).
BNCC: EF02MA22 -- Comparar informações de pesquisas apresentadas por meio
de tabelas de dupla entrada e em gráficos de colunas simples ou barras,
para melhor compreender aspectos da realidade próxima.

\item
a) Incorreta. Ela teria pagado a terça parte do valor da compra dos ingressos se ela
tivesse ido somente com 2 amigas e se ela tivesse pagado só o ingresso
dela.
b) Incorreta. Ela teria pagado o dobro do valor do ingresso se ela
tivesse pagado somente o ingresso dela e de outra amiga.
c) Incorreta. Ela teria pagado o triplo do valor do ingresso, se ela
tivesse pagado somente o ingresso dela e de outras duas amigas.
d) Correta. Como ela pagou o ingresso dela e de outras 3 amigas, ela
pagou o valor de 4 ingressos, ou seja, ela pagou quatro vezes o valor de um ingresso.
SAEB: Resolver problemas de multiplicação ou de divisão (por 2,
3, 4 ou 5), envolvendo números naturais, com os significados de formação
de grupos iguais ou proporcionalidade (incluindo dobro, metade, triplo
ou terça parte).
BNCC: EF02MA08 - Resolver e elaborar
problemas envolvendo dobro, metade, triplo e terça parte, com o suporte de imagens ou material manipulável, utilizando estratégias pessoais.

\item
a) Incorreta. O aluno considerou que cada pilha tivesse 10 livros, obtendo 10 * 2 = 20.
b) Incorreta. O aluno considerou que cada pilha tivesse 11 livros, obtendo 11 * 2 = 22.
c) Correta. Multiplicando por dois o número de livros que há em cada pilha, obtemos 12 * 2 = 24.
d) Incorreta. O aluno considerou que cada pilha tivesse 13 livros, obtendo 13 * 2 = 26.
SAEB: Analisar argumentações sobre a resolução de problemas de
adição, subtração, multiplicação ou divisão envolvendo números naturais.
BNCC: EF02MA07 -- Resolver e elaborar problemas de multiplicação (por 2,
3, 4 e 5) com a ideia de adição de parcelas iguais por meio de
estratégias e formas de registro pessoais, utilizando ou não suporte de
imagens e/ou material manipulável. EF02MA08 - Resolver e elaborar
problemas envolvendo dobro, metade, triplo e terça parte, com o suporte
de imagens ou material manipulável, utilizando estratégias pessoais.
\end{enumerate}

\section*{Simulado 2}

\begin{enumerate}
\item
a) Incorreta. Artur fez 5 pontos e não foi a maior quantidade de pontos do jogo.
b) Correta. Bruna fez 9 pontos e foi a maior quantidade de pontos do jogo.
c) Incorreta. César fez 7 pontos e não foi a maior quantidade de pontos do jogo.
d) Incorreta. Dulce fez 8 pontos e não foi a maior quantidade de pontos do jogo.
SAEB: Comparar ou ordenar quantidades de objetos (até 2 ordens).
BNCC: EF02MA03 - Comparar quantidades de objetos de dois conjuntos, por
estimativa e/ou por correspondência (um a um, dois a dois, entre
outros), para indicar ``tem mais'', ``tem menos'' ou ``tem a mesma
quantidade'', indicando, quando for o caso, quantos a mais e quantos a
menos.

\item
a) Correta. O boné é o item mais barato da lista, custando R\$ 35,00.
b) Incorreta. A camiseta é mais cara que o boné, pois R\$ 42,00 é mais do que R\$ 35,00.
c) Incorreta. A mochila não é o item mais barato da loja, pois R\$ 97,00 é mais do que R\$ 42,00 - que é maior que R\$ 35,00.
d) Incorreta. O tênis é o item mais caro da loja, pois R\$ 108,00 é o maior de todos os valores.
SAEB: Comparar ou ordenar números naturais de até 3 ordens com
ou sem suporte da reta numérica.
BNCC: EF02MA03 - Comparar quantidades de objetos de dois conjuntos, por
estimativa e/ou por correspondência (um a um, dois a dois, entre
outros), para indicar ``tem mais'', ``tem menos'' ou ``tem a mesma
quantidade'', indicando, quando for o caso, quantos a mais e quantos a
menos.

\item
a) Incorreta. O algarismo das centenas é maior que 1; assim, não
pode ser o próprio número 1. O algarismo das dezenas está correto, é 3.
O algarismo das unidades é menor que 9; logo, não pode ser o número 9. A senha não é o número 139.
b) Correta. O algarismo das centenas é maior que 1 e menor que 3, ou
seja, 2. O algarismo das dezenas é menor que 4 e maior que 2, ou seja,
3. O algarismo das unidades é maior que 7 e menor que 9, ou seja, 8. A
senha é o número 238.
c) Incorreta. O algarismo das centenas está correto, é 2. O algarismo
das dezenas é menor que 4, dessa forma não pode ser o próprio número 4.
O algarismo das unidades é maior que 7, sendo assim não pode ser o
número 7. A senha não é o número 247.
d) Incorreta. O algarismo das centenas é menor que 3, dessa forma não
pode ser o próprio número 3. O algarismo das dezenas é maior que 2,
dessa maneira não pode ser o número 2. O algarismo das unidades é maior
que 7, consequentemente não pode ser o número 7. A senha não é o número
327.
SAEB: Identificar a ordem ocupada por um algarismo ou seu valor
posicional (ou valor relativo) em um número natural de até 3 ordens.
BNCC: EF02MA01 - Comparar e ordenar números naturais (até a ordem de
centenas) pela compreensão de características do sistema de numeração
decimal (valor posicional e função do zero).

\item
a) Incorreta. Para ficar com 11 pirulitos, Gabriel precisava ter dado 7
pirulitos para seu primo, não 5.
b) Incorreta. Para ficar com 12 pirulitos, Gabriel precisava ter dado 6
pirulitos para seu primo, não 5.
c) Correta. A subtração de 18 -- 5 é igual a 13; portanto, Gabriel ficou
com 13 pirulitos depois de ter dado 5 a seu primo.
d) Incorreta. Para ficar com 14 pirulitos, Gabriel precisava ter dado
somente 4 pirulitos para seu primo, não 5.
SAEB: Resolver problemas de adição ou de subtração, envolvendo
números naturais de até 3 ordens, com os significados de juntar,
acrescentar, separar ou retirar.
BNCC: EF02MA06 - Resolver e elaborar problemas de adição e de subtração,
envolvendo números de até três ordens, com os significados de juntar,
acrescentar, separar, retirar, utilizando estratégias pessoais ou convencionais.

\item
a) Incorreta. O resultado da adição é 124.
b) Incorreta. O resultado da adição é 128.
c) Correta. O resultado da adição é 132.
d) Incorreta. O resultado da adição é 136.
SAEB: Calcular o resultado de adições e subtrações, envolvendo
número naturais de até 3 ordens.
BNCC: EF02MA06 - Resolver e elaborar problemas de adição e de subtração,
envolvendo números de até três ordens, com os significados de juntar,
acrescentar, separar, retirar, utilizando estratégias pessoais ou
convencionais.

\item
a) Incorreta. O termômetro é um instrumento utilizado para medir temperatura.
b) Incorreta. A balança é um instrumento utilizado para medir massa.
c) Correta. A trena é um instrumento utilizado para medir comprimento
em metros, sendo adequada para medir o comprimento da quadra da escola.
d) Incorreta. A régua, por mais que seja um instrumento utilizado para
medir comprimento, é ideal para medir distâncias em centímetros.
SAEB: Identificar a medida de comprimento, da capacidade ou da
massa de objetos, dada a imagem de um instrumento de medida.
BNCC: EF02MA16 - Estimar, medir e comparar comprimentos de lados de
salas (incluindo contorno) e de polígonos, utilizando unidades de medida
não padronizadas e padronizadas (metro, centímetro e milímetro) e
instrumentos adequados. EF02MA17 - Estimar, medir e comparar capacidade
e massa, utilizando estratégias pessoais e unidades de medida não
padronizadas ou padronizadas (litro, mililitro, grama e quilograma).

\item
a) Incorreta O copo graduado usa sua própria capacidade (volume) para
medir o espaço ocupado por líquidos.
b) Incorreta. O comprimento é a medida obtida por réguas e trenas.
c) Incorreta. A massa é a medida obtida por balanças.
d) Correta. O tempo é a medida obtida por relógios.
SAEB: Reconhecer unidades de medida e/ou instrumentos utilizados
para medir comprimento, tempo, massa ou capacidade.
BNCC: BNCC - EF02MA16 - Estimar, medir e comparar comprimentos de lados
de salas (incluindo contorno) e de polígonos, utilizando unidades de
medida não padronizadas e padronizadas (metro, centímetro e milímetro) e
instrumentos adequados.

\item
a) Correta. Do dia 2 ao dia 22 terão passado exatamente 3 semanas.
b) Incorreta. Do dia 9 ao dia 24 terão passado 2 semanas e 2 dias, não 3 semanas.
c) Incorreta. Do dia 16 ao dia 29 terão passado somente 2 semanas.
d) Incorreta. Do dia 23 ao dia 31 o tempo transcorrido será de apenas 1 semana e 2 dias,não de 3 semanas.
SAEB: Determinar a data de início, a data de término ou a
duração de um acontecimento entre duas datas.
BNCC: EF02MA18 - Indicar a duração de intervalos de tempo entre
duas datas, como dias da semana e meses do ano, utilizando calendário,
para planejamentos e organização de agenda.

\item
a) Incorreta. Consideraram-se somente 45 minutos; nesse caso, o avião deveria ter pousado às 08:45.
b) Incorreta. Considerou-se somente 1 hora; assim, o avião deveria ter pousado às 09:00.
c) Correta. O avião decolou às 8:00 e pousou 1 hora e 45 minutos depois, às 09:45.
d) Incorreta. Considerou-se o intervalo de 2 horas e 15 minutos; o avião, nesse caso, deveria ter
pousado às 10:15.
SAEB: Determinar o horário de início, o horário de término ou a
duração de um acontecimento.
BNCC: EF02MA19 - Medir a duração de um intervalo de tempo por meio de
relógio digital e registrar o horário do início e do fim do intervalo.

\item
a) Incorreta. Efetuou-se a subtração de 100 -- 57 incorretamente.
b) Correta. Calculou-se corretamente o troco, chegando-se ao valor de 43 reais.
c) Incorreta. Efetuou-se a subtração de 100 -- 57 incorretamente.
d) Incorreta. Considerou-se simplesmente o valor que Leonardo pagou pelo brinquedo.
SAEB: Resolver problemas que envolvam moedas e/ou cédulas do
sistema monetário brasileiro. 
BNCC: EF02MA20 - Estabelecer a
equivalência de valores entre moedas e cédulas do sistema monetário
brasileiro para resolver situações cotidianas.

\item
a) Correta. É muito provável que saia uma goma vermelha., uma vez que há
muito mais gomas vermelhas do que pretas.
b) Incorreta. É mais provável que saia uma goma vermelha, pois há
pouquíssimas gomas pretas em relação às vermelhas.
c) Incorreta. É muito provável que saia uma goma vermelha, já que há
mais gomas vermelhas do que pretas.
d) Incorreta. Não é impossível que saia uma goma preta, já que há gomas
pretas entre as opções.
SAEB: Classificar resultados de eventos cotidianos aleatórios
como ``pouco prováveis'', ``muito prováveis'', ``certos'' ou
``impossíveis''.
BNCC: EF02MA21 - Classificar resultados de eventos cotidianos aleatórios
como ``pouco prováveis'', ``muito prováveis'', ``improváveis'' e
``impossíveis''.

\item
a) Correta. Sofia encontrou 6 borboletas e 5 joaninhas; então, ao todo, foram 11 borboletas e joaninhas.
b) Incorreta. 6 + 5 é diferente de 13.
c) Incorreta. 6 + 5 é diferente de 14.
d) Incorreta. 6 + 5 é diferente de 15.
SAEB: Ler/identificar ou comparar dados estatísticos ou
informações expressas em tabelas (simples ou de dupla entrada).
BNCC: EF02MA22 - Comparar informações de pesquisas apresentadas por meio
de tabelas de dupla entrada e em gráficos de colunas simples ou barras,
para melhor compreender aspectos da realidade próxima.

\item
a) Incorreta. Amarelo não é a cor de que os alunos mais gostam; verde é a cor preferida.
b) Correta. Azul teve 25 votos.
c) Incorreta. Verde não foi a cor menos votada; vermelho foi a cor menos votada.
d) Incorreta. Vermelho teve menos de 20 votos.
SAEB: Ler/identificar ou comparar dados estatísticos expressos
em gráficos (barras simples, colunas simples ou pictóricos).
BNCC: EF02MA22 - Comparar informações de pesquisas apresentadas por meio
de tabelas de dupla entrada e em gráficos de colunas simples ou barras,
para melhor compreender aspectos da realidade próxima.

\item
a) Incorreta. R\$10,00 é metade do valor que Marcelo tem.
b) Incorreta. R\$20,00 é o próprio valor que Marcelo tem ou o valor que
está faltando para comprar a bola.
c) Incorreta. R\$30,00 é o valor que Marcelo tem mais metade do que ele
tem; ou seja, 20 + 10 = 30.
d) Correta. R\$40,00 é o dobro do valor que Marcelo tem; ou seja, a bola
custa 40 reais.
SAEB: Resolver problemas de multiplicação ou de divisão (por 2,
3, 4 ou 5), envolvendo números naturais, com os significados de formação
de grupos iguais ou proporcionalidade (incluindo dobro, metade, triplo
ou terça parte).
BNCC: EF02MA07 - Resolver e elaborar problemas de multiplicação (por 2,
3, 4 e 5) com a ideia de adição de parcelas iguais por meio de
estratégias e formas de registro pessoais, utilizando ou não suporte de
imagens e/ou material manipulável. EF02MA08 - Resolver e elaborar
problemas envolvendo dobro, metade, triplo e terça parte, com o suporte
de imagens ou material manipulável, utilizando estratégias pessoais.

\item
a) Incorreta. 8 é a quantidade de tomates que vem em cada saco.
b) Incorreta. 16 seria a quantidade de tomates que ela teria se tivesse comprado apenas 2 sacos.
c) Correta. Como cada saco tem 8 tomates e ela comprou 3 sacos, ao todo ela comprou 3 x 8 = 24 tomates.
d) Incorreta. 32 seria a quantidade de tomates que ela teria se tivesse comprado 4 sacos.
SAEB: Analisar argumentações sobre a resolução de problemas de
adição, subtração, multiplicação ou divisão envolvendo números naturais.
BNCC: EF02MA07 - Resolver e elaborar problemas de multiplicação (por 2,
3, 4 e 5) com a ideia de adição de parcelas iguais por meio de
estratégias e formas de registro pessoais, utilizando ou não suporte de
imagens e/ou material manipulável.
\end{enumerate}

\section*{Simulado 3}

\begin{enumerate}
\item
a) Incorreta. O número que aparece no visor da balança não é um código
de identificação, pois esse valor muda conforme o objeto que está sendo
pesado.
b) Correta. O número que aparece no visor da balança representa uma
medida, pois a balança é um instrumento de medir massa.
c) Incorreta. O número que aparece no visor da balança não corresponde à
quantidade de maçãs, mas sim ao peso delas.
d) Incorreta. O número que aparece no visor da balança não expressa a
posição da maçã em lugar algum.
SAEB: Reconhecer o que os números naturais indicam em diferentes
situações: quantidade, ordem, medida ou código de identificação.
BNCC: EF02MA01 -- Comparar e ordenar números naturais (até a ordem de
centenas) pela compreensão de características do sistema de numeração
decimal (valor posicional e função do zero).

\item
a) Incorreta. A atleta de número 07 está na quarta posição na corrida.
b) Incorreta. A atleta de número 12 está na segunda posição na corrida.
c) Correta. O atleta de número 31 está na primeira posição na corrida.
d) Incorreta. O atleta de número 72 está na última posição na corrida.
SAEB: Identificar a posição ordinal de um objeto ou termo
em uma sequência (1º, 2º etc.).
BNCC: EF02MA01 -- Comparar e ordenar números naturais (até a ordem de
centenas) pela compreensão de características do sistema de numeração
decimal (valor posicional e função do zero).

\item
a) Incorreta. O resultado da soma e da subtração é 310.
b) Incorreta. O resultado da soma e da subtração é 320.
c) Incorreta. O resultado da soma e da subtração é 290.
d) Correta. O resultado da soma e da subtração é 300.
SAEB: Compor ou decompor números naturais de até 3 ordens por
meio de diferentes adições.
BNCC: EF02MA04 -- Compor e decompor números naturais de até três ordens,
com suporte de material manipulável, por meio de diferentes adições.

\item
a) Incorreta. Fez a subtração incorretamente: 48 -- 10 = 38.
b) Correta. Fez a subtração corretamente: 48 -- 9 = 39.
c) Incorreta. Fez a subtração incorretamente: 48 -- 8 = 40.
d) Incorreta. Fez a subtração incorretamente: 48 -- 7 = 41.
SAEB: Resolver problemas de adição ou de subtração, envolvendo
números naturais de até 3 ordens, com os significados de juntar,
acrescentar, separar ou retirar.
BNCC: EF02MA06 - Resolver e elaborar problemas de adição e de subtração,
envolvendo números de até três ordens, com os significados de juntar,
acrescentar, separar, retirar, utilizando estratégias pessoais ou
convencionais.

\item
a) Incorreta. A ordem: Alce -- Flamingo -- Girafa -- Jacaré não
apresenta os animais do menor para o maior, apenas repete a ordem que
foi apresentada no enunciado.
b) Correta. O Flamingo é o menor, com 95 de altura; o segundo é o Jacaré
com 110; o terceiro é o Alce, com 150 e, por fim, a Girafa, com 170 de
altura.
c) Incorreta. A ordem Girafa -- Alce -- Jacaré -- Flamingo apresenta
os animais do maior para o menor.
d) Incorreta. A ordem Jacaré -- Flamingo -- Alce -- Girafa apresenta
uma ordem aleatória.
SAEB: Comparar comprimentos, capacidades ou massas ou ordenar
imagens de objetos com base na comparação visual de seus comprimentos,
capacidades ou massas.
BNCC: EF02MA17 - Estimar, medir e comparar capacidade
e massa, utilizando estratégias pessoais e unidades de medida não
padronizadas ou padronizadas (litro, mililitro, grama e quilograma).

\item
a) Incorreta. A quantidade de líquido presente nessa jarra sugere que
foram adicionados apenas 2 copos de água.
b) Incorreta. A quantidade de líquido presente nessa jarra sugere que
foram adicionados apenas 3 copos de água.
c) Incorreta. A quantidade de líquido presente nessa jarra sugere que
foram adicionados apenas 4 copos de água.
d) Correta. A quantidade de líquido presente nessa jarra sugere que
foram adicionados 5 copos de água
SAEB: Estimar/inferir medida de comprimento, capacidade ou massa
de objetos, utilizando unidades de medida convencionais ou não ou medir
comprimento, capacidade ou massa de objetos.
BNCC: EF02MA16 - Estimar, medir e comparar comprimentos de lados de
salas (incluindo contorno) e de polígonos, utilizando unidades de medida
não padronizadas e padronizadas (metro, centímetro e milímetro) e
instrumentos adequados. EF02MA17 - Estimar, medir e comparar capacidade
e massa, utilizando estratégias pessoais e unidades de medida não
padronizadas ou padronizadas (litro, mililitro, grama e quilograma).

\item
a) Incorreta. A sequência de atividades apresentada segue uma ordem
aleatória, que não é comum de acontecer: estudar -- escovar os dentes --
dormir -- tomar banho -- almoçar -- brincar.
b) Incorreta. A sequência de atividades apresentada segue uma ordem
aleatória, que não é comum de acontecer: almoçar -- tomar banho -- brincar
-- estudar -- dormir -- escovar os dentes.
c) Correta. A rotina diária de uma criança normalmente segue esta sequência
de atividades: escovar os dentes -- estudar -- almoçar - brincar -- tomar banho -- dormir.
d) Incorreta. A sequência de atividades apresentada segue uma ordem
aleatória que não é comum de acontecer: tomar banho -- dormir -- escovar
os dentes -- brincar -- estudar - almoçar.
SAEB: Identificar sequência de acontecimentos relativos a um dia.
BNCC: EF02MA19 -- Medir a duração de um intervalo de tempo por meio de
relógio digital e registrar o horário do início e do fim do intervalo.

\item
a) Correta. O dia 6 de julho de 2023 cai numa quinta-feira; portanto; a
fruta que o filho de Beatriz irá comer é abacaxi.
b) Incorreta. A banana é a fruta da terça-feira e o dia 6 de julho cai
em uma quinta-feira.
c) Incorreta. A laranja é a fruta da segunda-feira e o dia 6 de julho
cai em uma quinta-feira.
d) Incorreta. A pera é a fruta da quarta-feira e o dia 6 de julho cai em uma quinta-feira.
SAEB: Identificar datas, dias da semana ou meses do ano em
calendário ou escrever uma data, apresentando o dia, o mês e o ano.
BNCC: EF02MA18 - Indicar a duração de intervalos de tempo entre duas
datas, como dias da semana e meses do ano, utilizando calendário, para
planejamentos e organização de agenda.

\item
a) Incorreta. Considerou-se apenas uma cédula de cada valor, ou seja: 20 +
10 + 5 + 2 = 37 reais.
b) Incorreta. Desconsiderara-se uma cédula de 5 e uma cédula de 2, obtendo-se
20 + 10 + 5 + 5 + 2 = 42 reais.
c) Correta. A mãe de Brenda usou uma cédula de 20, uma de 10, três
cédulas de 5 e duas cédulas de 2 para pagar o lanche da padaria; sendo
assim, gastaram ao todo 20 + 10 + 5 + 5 + 5 + 2 + 2 = 49 reais.
d) Incorreta. Considerou-se que uma das cédulas de 5 fosse uma cédula de 10
reais, obtendo-se 20 + 10 + 10 + 5 + 5 + 2 + 2 = 54 reais.
SAEB: Relacionar valores de moedas e/ou cédulas do sistema
monetário brasileiro, com base nas imagens desses objetos.
BNCC: EF02MA20 -- Estabelecer a equivalência de valores entre moedas e
cédulas do sistema monetário brasileiro para resolver situações
cotidianas.

\item
a) Incorreta. A soma de 6 cédulas de 20 reais, 1 cédula de 50 reais e 1
cédula de 10 reais equivale a 180 reais, não a 160 reais.
b) Incorreta. A soma de 3 cédulas de 50 reais, 1 cédula de 20 reais e 1
cédula de 5 reais equivale a 175 reais, não a 160 reais.
c) Incorreta. A soma de 2 cédulas de 50 reais, 2 cédulas de 10 reais e 2
cédulas de 5 reais equivale a 150 reais, não a 160 reais.
d) Correta. A soma de 1 cédula de 100 reais, 1 cédula de 50 reais e 1
cédula de 10 reais equivale a 160 reais.
SAEB: Resolver problemas que envolvam moedas e/ou cédulas do
sistema monetário brasileiro. BNCC: EF02MA20 - Estabelecer a
equivalência de valores entre moedas e cédulas do sistema monetário
brasileiro para resolver situações cotidianas.

\item
a) Incorreta. Visto que há outras bolinhas em grande quantidade, assim
como as bolinhas verdes, torna-se, a cada lançamento, mais improvável
que sejam lançadas apenas bolinhas verdes.
b) Correta. As bolinhas verdes e amarelas são maioria.
c) Incorreta. As bolinhas vermelhas estão presentes em grande
quantidade, tornando provável o seu lançamento.
d) Incorreta. Mesmo que as bolinhas estejam em quantidades semelhantes,
torna-se apenas possível que sejam lançadas 3 bolinhas de cada cor, não certo.
SAEB: Classificar resultados de eventos cotidianos aleatórios
como ``pouco prováveis'', ``muito prováveis'', ``certos'' ou
``impossíveis''.
BNCC: EF02MA21 - Classificar resultados de eventos cotidianos aleatórios
como ``pouco prováveis'', ``muito prováveis'', ``improváveis'' e
``impossíveis''.

\item
a) Incorreta. 5 reais é o preço do bolo somente.
b) Incorreta. 8 reais é o preço do bolo e do refrigerante somente.
c) Correta. Se comer um bolo, um pão de queijo e um refrigerante, Tiago
vai pagar na vendinha 5 + 2 + 3 = R\$ 10,00.
d) Incorreta. 13 reais é o preço do bolo, do salgado e do refrigerante.
SAEB: Ler/identificar ou comparar dados estatísticos ou
informações expressas em tabelas (simples ou de dupla entrada).
BNCC: EF02MA22 - Comparar informações de pesquisas apresentadas por meio
de tabelas de dupla entrada e em gráficos de colunas simples ou barras,
para melhor compreender aspectos da realidade próxima.

\item
a) Incorreta. Artes é a segunda matéria preferida entre os estudantes do 2° ano.
b) Incorreta. Ciências é a terceira entre as matérias de preferência dos estudantes.
c) Incorreta. Matemática é a matéria de que os estudantes menos gostam.
d) Correta. Português é a matéria preferida entre os estudantes do 2° ano.
SAEB: Ler/identificar ou comparar dados estatísticos expressos
em gráficos (barras simples, colunas simples ou pictóricos).
BNCC: EF02MA22 - Comparar informações de pesquisas apresentadas por meio
de tabelas de dupla entrada e em gráficos de colunas simples ou barras,
para melhor compreender aspectos da realidade próxima.

\item
a) Incorreta. Um terço dos cupcakes são de mirtilo.
b) Incorreta. Os cupcakes de creme estão na mesma quantidade dos cupcakes de mirtilo.
c) Incorreta. Um terço dos cupcakes são de creme.
d) Incorreta. Um terço dos cupcakes são de morango; ou seja, 3 de 9.
SAEB: Resolver problemas de multiplicação ou de divisão (por 2,
3, 4 ou 5), envolvendo números naturais, com os significados de formação
de grupos iguais ou proporcionalidade (incluindo dobro, metade, triplo
ou terça parte).
BNCC: EF02MA08 - Resolver e elaborar
problemas envolvendo dobro, metade, triplo e terça parte, com o suporte
de imagens ou material manipulável, utilizando estratégias pessoais.

\item
a) Correta. 256 litros de leite divididos por 2 litros em cada garrafa geram 128 garrafas de leite.
b) Incorreta. Esse valor não condiz com a divisão de 256 por 2.
c) Incorreta. Esse valor não condiz com a divisão de 256 por 2.
d) Incorreta. 256 seria a quantidade de garrafas se cada uma tivesse 1 litro, não 2 litros.
SAEB: Analisar argumentações sobre a resolução de problemas de
adição, subtração, multiplicação ou divisão envolvendo números naturais.
BNCC: EF02MA07 - Resolver e elaborar problemas de multiplicação (por 2,
3, 4 e 5) com a ideia de adição de parcelas iguais por meio de
estratégias e formas de registro pessoais, utilizando ou não suporte de
imagens e/ou material manipulável.
\end{enumerate}

\section*{Simulado 4}

\begin{enumerate}
\item
a) Incorreta. Seis e cinco não é a forma correta de se representar por extenso o número 65.
b) Incorreta. Seiscentos e cinco é a forma de escrever o número 605, não 65.
c) Correta. Sessenta e cinco é a forma correta de se representar por extenso o número 65.
d) Incorreta. Setenta e cinco é a forma extensa equivalente ao número 75, não 65.
SAEB: Escrever números naturais de até 3 ordens em sua
representação por algarismos ou em língua materna ou associar o registro
numérico de números naturais de até 3 ordens ao registro em língua
materna.
BNCC: EF02MA01 - Comparar e ordenar números naturais (até a ordem de
centenas) pela compreensão de características do sistema de numeração
decimal (valor posicional e função do zero).

\item
a) Incorreta. A caixa 1 tem apenas um macaron a mais do que a caixa 2, não cinco.
b) Incorreta. A caixa 2 tem uma quantidade menor de macarons do que a
caixa 1, por isso não é possível que tenha dois a mais.
c) Correta. A caixa 2 tem um macaron a menos do que a caixa 1, pois a
caixa 1 tem 11 macarons e a caixa 2 tem 10.
d) Incorreta. As duas caixas possuem quantidades diferentes de macarons,
a caixa 1 tem 11 macarons e a caixa 2 tem 10 macarons.
SAEB: Comparar ou ordenar quantidades de objetos (até 2 ordens).
BNCC: EF02MA03 - Comparar quantidades de objetos de dois conjuntos, por
estimativa e/ou por correspondência (um a um, dois a dois, entre
outros), para indicar ``tem mais'', ``tem menos'' ou ``tem a mesma
quantidade'', indicando, quando for o caso, quantos a mais e quantos a
menos.

\item
a) Correta. O resultado da subtração é 75: 280 -- 120 -- 60 -- 25 = 75.
b) Incorreta. O resultado da subtração é 65: 250 -- 45 -- 55 -- 85 = 65.
c) Incorreta. O resultado da subtração é 85: 220 -- 75 -- 40 -- 20 = 85.
d) Incorreta. O resultado da subtração é 55: 200 -- 65 -- 50 -- 30 = 55.
SAEB: Calcular o resultado de adições e subtrações, envolvendo
número naturais de até 3 ordens.
BNCC: EF02MA06 -- Resolver e elaborar problemas de adição e de subtração,
envolvendo números de até três ordens, com os significados de juntar,
acrescentar, separar, retirar, utilizando estratégias pessoais ou convencionais.

\item
a) Incorreta. Somaram-se as 5 guloseimas por pacote apenas 10 vezes: 5 + 5 +
5 + 5 + 5 + 5 + 5 + 5 + 5 + 5 = 50.
b) Incorreta. Somaram-se as 5 guloseimas por pacote apenas 11 vezes: 5 + 5 +
5 + 5 + 5 + 5 + 5 + 5 + 5 + 5 + 5 = 55.
c) Correta. A composição do total de guloseimas de que a professora precisa
para montar os 12 pacotes é a soma de 5 guloseimas por pacote, ou seja,
5 + 5 + 5 + 5 + 5 + 5 + 5 + 5 + 5 + 5 + 5 + 5 = 60.
d) Incorreta. Somaram-se as 5 guloseimas por pacote 13 vezes: 5 + 5 + 5 + 5 +
5 + 5 + 5 + 5 + 5 + 5 + 5 + 5 + 5 = 65.
SAEB: Compor ou decompor números naturais de até 3 ordens por meio de diferentes adições.
BNCC: EF02MA04 - Compor e decompor números naturais de até três ordens,
com suporte de material manipulável, por meio de diferentes adições.

\item
a) Incorreta. O centímetro é uma medida de comprimento mais adequadamente medida por réguas.
b) Correta. O litro é uma medida de capacidade adequadamente medida por copos medidores.
c) Incorreta. O metro é uma medida de comprimento mais adequadamente medida por réguas.
d) Incorreta. O quilograma é uma medida de massa facilmente medida por balanças.
SAEB: Identificar a medida de comprimento, da capacidade ou da
massa de objetos, dada a imagem de um instrumento de medida.
BNCC: EF02MA17 - Estimar, medir e comparar capacidade
e massa, utilizando estratégias pessoais e unidades de medida não
padronizadas ou padronizadas (litro, mililitro, grama e quilograma).

\item
a) Incorreta. Por mais que as mãos possam ser usadas para medir
comprimentos ou comparar massas, não não são consideradas instrumentos que
pesam objetos com precisão.
b) Correta. A balança é o instrumento de medida indicado para a medição de massas.
c) Incorreta. O copo medidor é utilizado para medir a capacidade, ou
seja, usa-se a capacidade do copo para medir o espaço ocupado por líquidos.
d) Incorreta. A régua é usada para medir comprimentos não muito grandes.
SAEB: Reconhecer unidades de medida e/ou instrumentos utilizados
para medir comprimento, tempo, massa ou capacidade.
BNCC: EF02MA17 - Estimar, medir e comparar capacidade
e massa, utilizando estratégias pessoais e unidades de medida não
padronizadas ou padronizadas (litro, mililitro, grama e quilograma).

\item
a) Incorreta. Os dias 5 e 6 compõem o primeiro fim de semana do mês de maio.
b) Incorreta. Os dias 18 e 19 compõem o terceiro fim de semana do mês de maio, não o último.
c) Correta. Os dias 25 e 26 compõem o último fim de semana do mês de maio.
d) Incorreta. Por mais que os dias 30 e 31 sejam os últimos dias do mês de maio, não caem no sábado e no domingo.
SAEB: Determinar a data de início, a data de término ou a
duração de um acontecimento entre duas datas.
BNCC: EF02MA18 - Indicar a duração de intervalos de tempo entre
duas datas, como dias da semana e meses do ano, utilizando calendário,
para planejamentos e organização de agenda.

\item
a) Incorreta. Considerou que ele chegou ao mercado às 11:00, não às 12:00.
b) Incorreta. Considerou-se que ele chegou ao mercado às 11:30, não às 12:00.
c) Correta. Ele chegou no mercado às 12:00 e saiu 2 horas e 35 minutos depois, às 14:35.
d) Incorreta. Considerou-se que ele chegou ao mercado às 12:30, não às 12:00.
SAEB: Determinar o horário de início, o horário de término ou a
duração de um acontecimento.
BNCC: EF02MA19 - Medir a duração de um intervalo de tempo por meio de
relógio digital e registrar o horário do início e do fim do intervalo.

\item
a) Correta. Raíssa tem uma cédula de 50 reais, uma cédula de 5 reais,
uma cédula de 10 reais, cinco cédulas de 2 reais, duas moedas de 1 real,
uma moeda de 50 centavos e 3 moedas de 25 centavos; ou seja, ela tem 50
+ 5 + 10 + 10 + 2 + 0,50 + 0,75 = 78,25. Portanto, a boneca de 78 reais
é o único brinquedo que ela pode comprar e lhe restarão 25 centavos.
b) Incorreta. Raíssa tem apenas 78,25 reais; para comprar a casinha,
faltam 26 reais e 75 centavos.
c) Incorreta. Raíssa tem apenas 78,25 reais; para comprar os patins
faltam 17 reais e 75 centavos.
d) Incorreta. Raíssa tem apenas 78,25 reais; para comprar o videogame
faltam 421 reais e 75 centavos.
SAEB: Relacionar valores de moedas e/ou cédulas do sistema
monetário brasileiro, com base nas imagens desses objetos.
BNCC: EF02MA20 - Estabelecer a equivalência de valores entre moedas e
cédulas do sistema monetário brasileiro para resolver situações
cotidianas.

\item
a) Incorreta. Efetuou-se a subtração de 300 -- 249 incorretamente, obtendo-se o valor de 48 reais.
b) Incorreta. Efetuou-se a subtração de 300 -- 249 incorretamente, obtendo-se o valor de 49 reais.
c) Incorreta. Efetuou-se a subtração de 300 -- 249 incorretamente, obtendo-se o valor de 50 reais.
d) Correta. Luís entregou 3 notas de 100 reais, ou seja, 300 reais; o patinete custou R\$ 249,00. A subtração de 300,00 -- 249,00 = 51 reais.
SAEB: Resolver problemas que envolvam moedas e/ou cédulas do
sistema monetário brasileiro. 
BNCC: EF02MA20 -- Estabelecer a
equivalência de valores entre moedas e cédulas do sistema monetário
brasileiro para resolver situações cotidianas.

\item
A) Incorreta. Há três possibilidades dentre seis de se sortear um número
ímpar, não sendo pouco provável, mas sim provável de ocorrer, uma
probabilidade de 50\%.
B) Incorreta. É muito improvável que se sorteie o número seis duas vezes
consecutivas, sendo 1 possibilidade dentre 36 outras.
C) Incorreta. É impossível que se sorteie o número 8, pois este não se
encontra na numeração do dado.
D) Correto. É correto afirmar que se pode sortear um número maior que 3,
visto que, dentre 1 e 6, aparecem o 4, o 5 e o 6, que são maiores que 3.
SAEB: Classificar resultados de eventos cotidianos aleatórios
como ``pouco prováveis'', ``muito prováveis'', ``certos'' ou
``impossíveis''.
BNCC: EF02MA21 - Classificar resultados de eventos cotidianos aleatórios
como ``pouco prováveis'', ``muito prováveis'', ``improváveis'' e
``impossíveis''.

\item
a) Incorreta. Duas motos resultam em 4 rodas, não em 3.
b) Incorreta. Dois carros possuem 8 rodas e três motos, 6 rodas.
c) Correta. Seis rodas mais 4 rodas resultam em 10 rodas -- a quantidade de rodas de um caminhão.
d) Incorreta. O ônibus tem 2 rodas a mais que um carro, não 3.
SAEB: Ler/identificar ou comparar dados estatísticos ou
informações expressas em tabelas (simples ou de dupla entrada).
BNCC: EF02MA22 -- Comparar informações de pesquisas apresentadas por meio
de tabelas de dupla entrada e em gráficos de colunas simples ou barras,
para melhor compreender aspectos da realidade próxima.

\item
a) Correta. As brincadeiras Esconde-esconde e Corda de Pular
recebem a mesma quantidade de votos.
b) Incorreta. A brincadeira Batata-quente é a segunda em ordem de
preferência.
c) Incorreta. O videogame é a brincadeira predileta.
d) Incorreta. O jogo da damas recebeu a menor quantidade de votos,
sendo o menos escolhido.
SAEB: Ler/identificar ou comparar dados estatísticos expressos
em gráficos (barras simples, colunas simples ou pictóricos).
BNCC: EF02MA22 -- Comparar informações de pesquisas apresentadas por meio
de tabelas de dupla entrada e em gráficos de colunas simples ou barras,
para melhor compreender aspectos da realidade próxima.

\item
a) Incorreta. 9 é a quantidade de torradinhas que cabem de cada vez na assadeira.
b) Incorreta. 27 seria a quantidade de torradinhas se ela tivesse levado 3 fornadas ao forno, não 5.
c) Correta. Como a vovó levou 5 fornadas ao forno e cada fornada tinha 9 torradinhas, ao todo foram assadas 5 x 9 = 45 torradinhas.
d) Incorreta. 54 seria a quantidade de torradinhas se ela tivesse levado 1 fornada a mais ao forno.
SAEB: Resolver problemas de multiplicação ou de divisão (por 2,
3, 4 ou 5), envolvendo números naturais, com os significados de formação
de grupos iguais ou proporcionalidade (incluindo dobro, metade, triplo
ou terça parte).
BNCC: EF02MA07 - Resolver e elaborar problemas de multiplicação (por 2,
3, 4 e 5) com a ideia de adição de parcelas iguais por meio de
estratégias e formas de registro pessoais, utilizando ou não suporte de
imagens e/ou material manipulável.

\item
a) Incorreta. 10 é a quantidade de queijos entregue em cada
mercearia para revenda.
b) Incorreta. Se Dona Cida fez 150 queijos e cada mercearia recebeu 10,
então temos que fazer a divisão de 150 por 10, que é igual a 15.
Portanto, na cidade, 15 mercearias receberam os queijos de Dona Cida para
revenda.
c) Incorreta. 25 seria o número de mercearias recebendo os queijos se Dona Cida
tivesse feito 100 queijos a mais para revenda.
d) Incorreta. 30 seria o número de mercearias com queijo para revenda se Dona Cida
tivesse feito 300 queijos para revenda, não 150.
SAEB: Analisar argumentações sobre a resolução de problemas de
adição, subtração, multiplicação ou divisão envolvendo números naturais.
BNCC: EF02MA07 - Resolver e elaborar problemas de multiplicação (por 2,
3, 4 e 5) com a ideia de adição de parcelas iguais por meio de
estratégias e formas de registro pessoais, utilizando ou não suporte de
imagens e/ou material manipulável. EF02MA08 - Resolver e elaborar
problemas envolvendo dobro, metade, triplo e terça parte, com o suporte
de imagens ou material manipulável, utilizando estratégias pessoais.
\end{enumerate}

\blankpage