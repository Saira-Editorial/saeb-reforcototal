\chapter{Respostas}
\pagestyle{plain}
\footnotesize

\pagecolor{gray!40}

\colorsec{Matemática – Módulo 1 – Treino}

\begin{enumerate}
\item

\item

\item
\end{enumerate}

\colorsec{Matemática – Módulo 2 – Treino}

\begin{enumerate}
\item

\item

\item
\end{enumerate}

\colorsec{Matemática – Módulo 3 – Treino}

\begin{enumerate}
\item

\item

\item
\end{enumerate}

\colorsec{Matemática – Módulo 4 – Treino}

\begin{enumerate}
\item

\item

\item
\end{enumerate}

\colorsec{Matemática – Módulo 5 – Treino}

\begin{enumerate}
\item

\item

\item
\end{enumerate}

\colorsec{Matemática – Módulo 6 – Treino}

\begin{enumerate}
\item

\item

\item
\end{enumerate}

\colorsec{Matemática – Módulo 7 – Treino}

\begin{enumerate}
\item

\item

\item
\end{enumerate}

\colorsec{Matemática – Módulo 8 – Treino}

\begin{enumerate}
\item

\item

\item
\end{enumerate}

\colorsec{Matemática – Módulo 9 – Treino}

\begin{enumerate}
\item

\item

\item
\end{enumerate}

\colorsec{Matemática – Módulo 10 – Treino}

\begin{enumerate}
\item

\item

\item
\end{enumerate}

\colorsec{Matemática – Módulo 11 – Treino}

\begin{enumerate}
\item

\item

\item
\end{enumerate}

\colorsec{Educação Física – Módulo 1 –  Treino}

\begin{enumerate}
\item
SAEB: Identificar as diferentes valências físicas necessárias à
realização de práticas corporais (jogos eletrônicos, lutas, práticas
corporais de aventura, ginásticas, esportes e dança).
BNCC: EF67EF06 - Analisar as transformações na organização e na prática
dos esportes em suas diferentes manifestações (profissional e
comunitário/lazer).
a) Correta. de fato, a valência física trabalhada pelo atleta da imagem 
é a força, porque ele está vencendo uma caga externa composta por anilhas
e barra.
b) Incorreta. O levantamento olímpico trabalha a força, não a
resistência.
c) Incorreta. A velocidade não é praticada nesse esporte.
d) Incorreta. Nesse esporte, a flexibilidade não é priorizada.

\item
SAEB: Analisar as transformações históricas, o processo de
esportivização e a midiatização das práticas corporais, com ênfase nas
lutas.
BNCC: EF89EF18 - Discutir as transformações históricas, o processo de
esportivização e a midiatização de uma ou mais lutas, valorizando e
respeitando as culturas de origem.
a) Incorreta. Não há alusão, no texto, à criação de
federações ou confederações da luta marajoara. 
b) Correta. A luta marajoara faz parte de um evento oficial de
competição, os Jogos Estudantis Paraenses.
c) Incorreta. Não há alusão, no texto, à alteração do objetivo da luta
marajoara. 
d) Incorreta. A semelhança entre a luta marajoara e o
wrestling não define essa prática corporal como esporte.

\item
SAEB: Identificar o valor do patrimônio urbano e natural
nas vivências das práticas corporais de aventura urbana e na natureza.
BNCC: EF67EF06 - Analisar as transformações na organização e na prática
dos esportes em suas diferentes manifestações (profissional e
comunitário/lazer).
a) Incorreta. Não há alusão, no texto, ao propósito de formar novos
atletas.
b) Incorreta. Não há alusão, no texto, ao uso do skate como meio
de locomoção. 
c) Correta. O segundo parágrafo é explícito quanto à demanda da 
população por um espaço adequado para a prática do esporte.
d) Incorreta. Não há alusão, no texto, ao propósito de criação de novos investimentos da prefeitura.
\end{enumerate}

\colorsec{Educação Física – Módulo 2 –  Treino}

\begin{enumerate}
\item
SAEB: Diferenciar os esportes com base nos critérios de sua lógica
interna.
BNCC: EF67EF12 - Planejar e utilizar estratégias para aprender elementos
constitutivos das danças urbanas.
a) Incorreta. A disputa por medalhas olímpicas não caracteriza se uma
modalidade é um esporte oficial ou não.
b) Incorreta. A dança urbana não foi modalidade esportiva de olimpíadas 
anteriores. A estreia ocorrerá em 2024.
c) Incorreta. A participação de homens e mulheres não caracteriza o
breaking como esporte oficial.
d) Correta. No texto, afirma-se que, além das Olimpíadas, os atletas
devem participar de outras competições oficiais.

\item
SAEB: Diferenciar as danças urbanas, seus elementos constitutivos
e seu valor cultural nas demais manifestações da dança.
BNCC: EF67EF13 - Diferenciar as danças urbanas das demais manifestações
da dança, valorizando e respeitando os sentidos e significados
atribuídos a eles por diferentes grupos sociais.
a) Correta. De acordo com o texto, no hip hop o rap, o grafite e
o break são formas de expressão das dificuldades dos moradores da 
periferia.
b) Incorreta. A origem do hip hop está mais associada à expressão das 
adversidades das pessoas da periferia do que ao incentivo à dança.
c) Incorreta. A origem do hip hop está mais associada à expressão das 
adversidades das pessoas da periferia do que à criação de uma nova
modalidade de dança.
d) Incorreta. Não há, no texto, referências ao hip hop como forma de 
assistência social do governo.

\item
SAEB: Diferenciar as danças urbanas, seus elementos constitutivos
e seu valor cultural nas demais manifestações da dança.
BNCC: EF67EF13 - Diferenciar as danças urbanas das demais manifestações
da dança, valorizando e respeitando os sentidos e significados
atribuídos a eles por diferentes grupos sociais.
a) Incorreta. A origem asiática do k-pop não implica que essa modalidade
seja considerada dança urbana.
b) Incorreta. A prática da dança em grupo não tem relação com sua
classificação com dança urbana.
c) Incorreta. Os movimentos corporais são constitutivos de quaisquer 
danças, de modo que essa característica não é suficiente para caracterizar 
o hip hop como danças urbanas.
d) Correta. O k-pop se inspirou em elementos do hip hop, como os
movimentos de dança do break e o rap.
\end{enumerate}

\colorsec{Educação Física – Módulo 3 –  Treino}

\begin{enumerate}
\item

\item

\item
\end{enumerate}

\colorsec{Ciências da Natureza – Módulo 1 –  Treino}

\begin{enumerate}
\item

\item

\item
\end{enumerate}

\colorsec{Ciências da Natureza – Módulo 2 –  Treino}

\begin{enumerate}
\item

\item

\item
\end{enumerate}

\colorsec{Ciências da Natureza – Módulo 3 –  Treino}

\begin{enumerate}
\item

\item

\item
\end{enumerate}

\colorsec{Simulado 1}

\begin{enumerate}
\item
\item
\item
\item
\item
\item
\item
\item
\item
\item
\item
\item
\item
\item
\item
\end{enumerate}

\colorsec{Simulado 2}

\begin{enumerate}
\item
\item
\item
\item
\item
\item
\item
\item
\item
\item
\item
\item
\item
\item
\item
\end{enumerate}

\colorsec{Simulado 3}

\begin{enumerate}
\item
\item
\item
\item
\item
\item
\item
\item
\item
\item
\item
\item
\item
\item
\item
\end{enumerate}

\colorsec{Simulado 4}

\begin{enumerate}
\item
\item
\item
\item
\item
\item
\item
\item
\item
\item
\item
\item
\item
\item
\item
\end{enumerate}