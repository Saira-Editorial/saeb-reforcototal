\chapter{Respostas}
\pagestyle{plain}
\footnotesize

\pagecolor{gray!40}

\colorsec{Matemática – Módulo 1 – Treino}

\begin{enumerate}
\item

\item

\item
\end{enumerate}

\colorsec{Matemática – Módulo 2 – Treino}

\begin{enumerate}
\item

\item

\item
\end{enumerate}

\colorsec{Matemática – Módulo 3 – Treino}

\begin{enumerate}
\item

\item

\item
\end{enumerate}

\colorsec{Matemática – Módulo 4 – Treino}

\begin{enumerate}
\item

\item

\item
\end{enumerate}

\colorsec{Matemática – Módulo 5 – Treino}

\begin{enumerate}
\item

\item

\item
\end{enumerate}

\colorsec{Matemática – Módulo 6 – Treino}

\begin{enumerate}
\item

\item

\item
\end{enumerate}

\colorsec{Matemática – Módulo 7 – Treino}

\begin{enumerate}
\item

\item

\item
\end{enumerate}

\colorsec{Matemática – Módulo 8 – Treino}

\begin{enumerate}
\item

\item

\item
\end{enumerate}

\colorsec{Matemática – Módulo 9 – Treino}

\begin{enumerate}
\item

\item

\item
\end{enumerate}

\colorsec{Matemática – Módulo 10 – Treino}

\begin{enumerate}
\item

\item

\item
\end{enumerate}

\colorsec{Matemática – Módulo 11 – Treino}

\begin{enumerate}
\item

\item

\item
\end{enumerate}

\colorsec{Educação Física – Módulo 1 –  Treino}

\begin{enumerate}
\item
SAEB: Identificar as diferentes valências físicas necessárias à
realização de práticas corporais (jogos eletrônicos, lutas, práticas
corporais de aventura, ginásticas, esportes e dança).
BNCC: EF67EF06 - Analisar as transformações na organização e na prática
dos esportes em suas diferentes manifestações (profissional e
comunitário/lazer).
a) Correta. de fato, a valência física trabalhada pelo atleta da imagem 
é a força, porque ele está vencendo uma caga externa composta por anilhas
e barra.
b) Incorreta. O levantamento olímpico trabalha a força, não a
resistência.
c) Incorreta. A velocidade não é praticada nesse esporte.
d) Incorreta. Nesse esporte, a flexibilidade não é priorizada.

\item
SAEB: Analisar as transformações históricas, o processo de
esportivização e a midiatização das práticas corporais, com ênfase nas
lutas.
BNCC: EF89EF18 - Discutir as transformações históricas, o processo de
esportivização e a midiatização de uma ou mais lutas, valorizando e
respeitando as culturas de origem.
a) Incorreta. Não há alusão, no texto, à criação de
federações ou confederações da luta marajoara. 
b) Correta. A luta marajoara faz parte de um evento oficial de
competição, os Jogos Estudantis Paraenses.
c) Incorreta. Não há alusão, no texto, à alteração do objetivo da luta
marajoara. 
d) Incorreta. A semelhança entre a luta marajoara e o
wrestling não define essa prática corporal como esporte.

\item
SAEB: Identificar o valor do patrimônio urbano e natural
nas vivências das práticas corporais de aventura urbana e na natureza.
BNCC: EF67EF06 - Analisar as transformações na organização e na prática
dos esportes em suas diferentes manifestações (profissional e
comunitário/lazer).
a) Incorreta. Não há alusão, no texto, ao propósito de formar novos
atletas.
b) Incorreta. Não há alusão, no texto, ao uso do skate como meio
de locomoção. 
c) Correta. O segundo parágrafo é explícito quanto à demanda da 
população por um espaço adequado para a prática do esporte.
d) Incorreta. Não há alusão, no texto, ao propósito de criação de novos investimentos da prefeitura.
\end{enumerate}

\colorsec{Educação Física – Módulo 2 –  Treino}

\begin{enumerate}
\item
SAEB: Diferenciar os esportes com base nos critérios de sua lógica
interna.
BNCC: EF67EF12 - Planejar e utilizar estratégias para aprender elementos
constitutivos das danças urbanas.
a) Incorreta. A disputa por medalhas olímpicas não caracteriza se uma
modalidade é um esporte oficial ou não.
b) Incorreta. A dança urbana não foi modalidade esportiva de olimpíadas 
anteriores. A estreia ocorrerá em 2024.
c) Incorreta. A participação de homens e mulheres não caracteriza o
breaking como esporte oficial.
d) Correta. No texto, afirma-se que, além das Olimpíadas, os atletas
devem participar de outras competições oficiais.

\item
SAEB: Diferenciar as danças urbanas, seus elementos constitutivos
e seu valor cultural nas demais manifestações da dança.
BNCC: EF67EF13 - Diferenciar as danças urbanas das demais manifestações
da dança, valorizando e respeitando os sentidos e significados
atribuídos a eles por diferentes grupos sociais.
a) Correta. De acordo com o texto, no hip hop o rap, o grafite e
o break são formas de expressão das dificuldades dos moradores da 
periferia.
b) Incorreta. A origem do hip hop está mais associada à expressão das 
adversidades das pessoas da periferia do que ao incentivo à dança.
c) Incorreta. A origem do hip hop está mais associada à expressão das 
adversidades das pessoas da periferia do que à criação de uma nova
modalidade de dança.
d) Incorreta. Não há, no texto, referências ao hip hop como forma de 
assistência social do governo.

\item
SAEB: Diferenciar as danças urbanas, seus elementos constitutivos
e seu valor cultural nas demais manifestações da dança.
BNCC: EF67EF13 - Diferenciar as danças urbanas das demais manifestações
da dança, valorizando e respeitando os sentidos e significados
atribuídos a eles por diferentes grupos sociais.
a) Incorreta. A origem asiática do k-pop não implica que essa modalidade
seja considerada dança urbana.
b) Incorreta. A prática da dança em grupo não tem relação com sua
classificação com dança urbana.
c) Incorreta. Os movimentos corporais são constitutivos de quaisquer 
danças, de modo que essa característica não é suficiente para caracterizar 
o hip hop como danças urbanas.
d) Correta. O k-pop se inspirou em elementos do hip hop, como os
movimentos de dança do break e o rap.
\end{enumerate}

\colorsec{Educação Física – Módulo 3 –  Treino}

\begin{enumerate}
\item
Saeb: Avaliar a multiplicidade de padrões de estética corporal
disseminados pela mídia, que geram uma prática excessiva de exercícios e
o uso de recursos ergogênicos.
BNCC: EF89EF09 -- Problematizar a prática excessiva de exercícios físicos
e o uso de medicamentos para a ampliação do rendimento ou
potencialização das transformações corporais, bem como os efeitos do
exercício físico para saúde e sua ausência, relacionada ao sedentarismo
e ao aparecimento de doenças.
a) Correta. A bulimia é caracterizada pela ingestão excessiva de alimentos
seguida pela indução de vômito pelo próprio indivíduo, para evitar
ganho de peso.
b) Incorreta. A anorexia se caracteriza por hábitos alimentares 
alterados, voltados sempre à perda de peso: privação de alimentos, 
dietas restritivas extremas, longos períodos em jejum, uso de 
medicamentos laxativos e inibidores de apetite, além de outros 
comportamentos, como excesso de práticas esportivas e vômitos
induzidos. 
c) Incorreta. A vigorexia é um transtorno no qual o indivíduo
tem de si mesmo uma imagem distorcida: embora tenha compleição
física forte e musculosa, diante do espelho ele se vê como fraco
e franzino.
d) Incorreta. A ortorexia é a obsessão pela alimentação saudável.

\item
SAEB: Avaliar os problemas presentes nos esportes e abordados pela
mídia, tais como doping, violência ou corrupção.
BNCC: EF89EF09 -- Problematizar a prática excessiva de exercícios físicos
e o uso de medicamentos para a ampliação do rendimento ou
potencialização das transformações corporais, bem como os efeitos do
exercício físico para saúde e sua ausência, relacionada ao sedentarismo
e ao aparecimento de doenças.
a) Incorreta. O treino em excesso é definido como \textit{overtrainig},
e não como \textit{doping} ou dopagem.
b) Correta. A ingestão ou aplicação de substâncias proibidas caracteriza
o \textit{doping} ou dopagem, como se pode verificar no próprio texto.
c) Incorreta. A prática de burlar as regras não caracteriza o 
\textit{doping} ou dopagem, que está associado especificamente ao uso
de substâncias ilícitas. 
d) Incorreta. Quem evita o uso de anabolizantes proibidos não está
incorrendo na prática do \textit{doping} ou dopagem.

\item
SAEB: Avaliar a relação entre as práticas corporais e a promoção da
saúde.
BNCC: EF89EF08 -- Discutir, analisar e refletir criticamente as
transformações históricas dos padrões de desempenho, saúde e beleza,
considerando a forma como são apresentados nos diferentes meios
(científico, midiático etc.), identificando e reconhecendo a influência
da mídia nos padrões de comportamento do/no corpo.
a) Incorreta. Segundo o texto, as práticas esportivas aumentam a
longevidade.
b) Incorreta. Segundo o texto, as práticas esportivas contribuem para a 
redução das taxas de gordura, trocada por massa magra.
c) Correta. Segundo o texto, as práticas esportivas melhoram a 
qualidade de vida.  
d) Incorreta. Não há referências, no texto, à redução de reações químicas
no corpo.
\end{enumerate}

\colorsec{Ciências da Natureza – Módulo 1 –  Treino}

\begin{enumerate}
\item
BNCC: EF09CI01 -- Investigar as mudanças de estado físico da matéria e explicar essas transformações com base no modelo de constituição submicroscópica.
a) Incorreta. A matéria com um todo possui volume; dessa forma essa propriedade não justifica o fato de o lixo flutuar na superfície da água. Um exemplo é a areia que possui volume, mas sedimenta e é encontrada no fundo do mar.
b) Incorreta. A maleabilidade refere-se à capacidade de a matéria ser
  moldada, o que não é a propriedade que faz com que esse lixo possa ser catado pelos membros da ONG.
c) Incorreta. O brilho não é uma propriedade relevante para que o lixo seja catado na superfície da água, já que ele não influencia diretamente na densidade do material.
d) Correta. É graças à diferença de densidade do lixo e da água que esses resíduos boiam na superfície e podem ser catados pelos membros
  da ONG.

\item
BNCC: EF09CI03 -- Identificar modelos que descrevem a estrutura da matéria (constituição do átomo e composição de moléculas
simples) e reconhecer sua evolução histórica.
a) Incorreta, pois os elétrons não se encontram no núcleo atômico;
b) Incorreta, pois os elétrons são partículas que apresentam cargas
  negativas;
c) Correta, pois elétrons de fato são partículas negativas que se
  encontram na eletrosfera capazes de se movimentar e liberar energia;
d) Incorreta, pois os elétrons são ficam presos ao núcleo, e sim livres
  na eletrosfera.

\item
BNCC: EF09CI06 -- Classificar as radiações
eletromagnéticas por suas frequências, fontes e aplicações, discutindo e
avaliando as implicações de seu uso em controle remoto, telefone celular, raio X, forno de micro-ondas, fotocélulas etc.
a) Correta. De fato, a transmissão Wi-Fi ocorre por meio de ondas eletromagnéticas que são capturadas por materiais que possuem boa capacidade elétrica, como é o caso do metal, presente nos espelhos.
b) Incorreta. A transmissão do sinal Wi-Fi não ocorre por meio de ondas mecânicas, e sim por meio de ondas eletromagnéticas.
c) Incorreta. Por se tratar de uma propagação por meio de ondas eletromagnéticas, o Wi-Fi é capaz de se propagar no vacúo.
d) Incorreta. O sinal Wi-Fi não é transmitido com base no mesmo tipo de onda que o som, já que o Wi-Fi é transmitido através de ondas eletromagnéticas e o som, de ondas mecânicas.
\end{enumerate}

\colorsec{Ciências da Natureza – Módulo 2 –  Treino}

\begin{enumerate}
\item
BNCC: EF09CI08 -- Associar os gametas à transmissão
das características hereditárias, estabelecendo relações entre
ancestrais e descendentes.
a) Correta. A hereditariedade trata das características e
  informações genéticas transmitidas a cada geração. Ao estudar o código
  genético para entender sobre doenças e outras características da raça
  humana, as relações ancestrais-descendentes estão sendo desmistificadas.
b) Incorreta. A paleontologia trata do estudo de seres vivos que
  habitaram a Terra em um passado remoto.
c) Incorreta. A conservação trata do estudo de técnicas alternativas
  que resultem no uso sustentável dos recursos e na preservação das
  espécies.
d) Incorreta. A taxonomia trata de descrever, identificar e nomear
  seres vivos a partir de critérios estabelecidos.

\item
BNCC: EF09CI10 -- Comparar as ideias evolucionistas de
Lamarck e Darwin apresentadas em textos científicos e históricos,
identificando semelhanças e diferenças entre essas ideias e sua
importância para explicar a diversidade biológica.
a) Incorreta. A alternativa descreve o conceito descrito por
  Lamarck, no qual os indivíduos se modificam para se adaptar ao
  ambiente. Seleção natural é um conceito próximo à ideia de Darwin, que
  menciona que o ambiente seleciona os indivíduos mais aptos.
b) Incorreta. A alternativa descreve o conceito de Lamarck, que,
  posteriormente, foi estudado e modernizado por Darwin, dando origem à
  teoria que utilizamos hoje: a seleção natural.
c) Correta. As mudanças ambientais levam a condições ambientais
  ruins, como alimento escasso. Com isso, obter energia fica
  mais difícil e, por consequência, o investimento de energia em
  ornamentos (como cores fortes) também diminui. Sendo assim, o ambiente
  seleciona apenas os pássaros capazes de sobreviver sob tais condições.
d) Incorreta. Lamarck não acredita que o ambiente selecionava as
  espécies; na verdade, sua teoria falava sobre a mudanças das espécies
  frente às imprevisibilidades do ambiente.

\item
BNCC: EF09CI12 -- Justificar a importância das unidades de
conservação para a preservação da biodiversidade e do patrimônio
nacional, considerando os diferentes tipos de unidades (parques,
reservas e florestas nacionais), as populações humanas e as atividades a
eles relacionados.
a) Incorreta. A Caatinga é um bioma com um dos níveis de
  biodiversidade mais altos.
b) Correta. A Caatinga apresenta alta biodiversidade, sendo 15\% dela
  endêmica; portanto, diante das mudanças ambientais, a construção de
  unidades de conservação nesse bioma deve ser priorizada para garantir a proteção adequada das espécies.
c) Incorreta. O comércio de espécies exóticas é uma medida que iria
  prejudicar a biodiversidade presente na Caatinga, estando em posição
  antagônica ao objetivo das unidades de conservação.
d) Incorreta. Com a construção de indústrias o habitat de diversas
  espécies seria prejudicado, favorecendo a perda de biodiversidade e,
  portanto, estando em posição antagônica ao objetivo das unidades de conservação.
\end{enumerate}

\colorsec{Ciências da Natureza – Módulo 3 –  Treino}

\begin{enumerate}
\item

\item

\item
\end{enumerate}

\colorsec{Simulado 1}

\begin{enumerate}
\item
\item
\item
\item
\item
\item
\item
\item
\item
\item
\item
\item
\item
\item
\item
\end{enumerate}

\colorsec{Simulado 2}

\begin{enumerate}
\item
\item
\item
\item
\item
\item
\item
\item
\item
\item
\item
\item
\item
\item
\item
\end{enumerate}

\colorsec{Simulado 3}

\begin{enumerate}
\item
\item
\item
\item
\item
\item
\item
\item
\item
\item
\item
\item
\item
\item
\item
\end{enumerate}

\colorsec{Simulado 4}

\begin{enumerate}
\item
\item
\item
\item
\item
\item
\item
\item
\item
\item
\item
\item
\item
\item
\item
\end{enumerate}