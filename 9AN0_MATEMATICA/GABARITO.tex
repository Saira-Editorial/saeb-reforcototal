\chapter{Respostas}
\pagestyle{plain}
\footnotesize

\pagecolor{gray!40}

\colorsec{Matemática – Módulo 1 – Treino}

\begin{enumerate}
\item
SAEB: Converter uma representação de um número 
racional positivo para outra representação.
a) Incorreta. x = 2. 
b) Incorreta. x = 2.
c) Incorreta. x = 2.  
d) 2\textsuperscript{3}x3\textsuperscript{4}x5\textsuperscript{X} \rightarrow 
(3+1)x(4+1)x(x+1) = 60 \rightarrow 4x5x(x+1)=60 
20x(x=1)=60 \rightarrow x+1=3 \therefore x=2
Dessa forma, x = 2.

\item
SAEB: Identificar um número natural como primo, composto, 
``múltiplo/fator de'' ou ``divisor de'' ou identificar a decomposição de
um número natural em fatores primos ou relacionar as propriedades aritméticas
(primo, composto, ``múltiplo/fator de'' ou ``divisor de'') de um
número natural à sua decomposição em fatores primos.
a) Incorreta. 10 é múltiplo de 5 e não é ímpar.
b) Incorreta. 3 é ímpar e não é múltiplo de 5.
c) Incorreta. 2 é par e não é múltiplo de 6.
d) Correta. Como 12 é par, todo múltiplo deste número terá o fator 2.

\item
SAEB: Comparar ou ordenar números reais, com ou sem suporte da reta
numérica, ou aproximar número reais para múltiplos de potência de 10
mais próxima.
BNCC: EF09MA02 -- Reconhecer um número irracional como um número real cuja representação decimal é infinita e não periódica, e estimar a localização de alguns deles na reta numérica.
a) Incorreta. A letra que indica a localização do número 1,6 é D.
b) Incorreta. A letra que indica a localização do número 1,6 é D.
c) Incorreta. A letra que indica a localização do número 1,6 é D.
d) Correta. Entre os números 1 e 2, existem 9 marcações, e a letra D está
na sexta marcação, o que corresponde a 1,6.
\end{enumerate}

\colorsec{Matemática – Módulo 2 – Treino}

\begin{enumerate}
\item
SAEB: Resolver problemas que envolvam as ideias de múltiplo, 
divisor, máximo divisor comum ou mínimo múltiplo comum.
a) Incorreta. Eles se encontrarão em uma quarta.
b) Incorreta. Eles se encontrarão em uma quarta.
c) Correta. MMC (4,6) = 12 dias. Após 7 dias será sexta-feira, e podemos 
continuar contando (o 8º dia é sábado; o 9º, domingo; o 10º, segunda;
o 11º, terça; e o 12º, quarta).
d) Incorreta. Eles se encontrarão em uma quarta.

\item
SAEB: Resolver problemas que envolvam as ideias de múltiplo, 
divisor, máximo divisor comum ou mínimo múltiplo comum.
a) Correta. MDC (120, 150) = 30, sendo assim temos 150 : 30 = 5 e 120 : 30 = 4. Julinha montará 30 pacotes de cada doce. Os pacotes de brigadeiro terão 5 doces; os de beijinhos, 4.
b) Incorreta. Julinha montará 30 pacotes de cada doce. Os pacotes de brigadeiro terão 5 doces; os de beijinhos, 4.
c) Julinha montará 30 pacotes de cada doce. Os pacotes de brigadeiro terão 5 doces; os de beijinhos, 4.
d) Julinha montará 30 pacotes de cada doce. Os pacotes de brigadeiro terão 5 doces; os de beijinhos, 4. 

\item
SAEB: Resolver problemas de adição, subtração, multiplicação, 
divisão, potenciação ou radiciação envolvendo número reais, inclusive
notação científica.
BNCC: EF09MA04 -- Resolver e elaborar problemas com números reais, 
inclusive em notação científica, envolvendo diferentes operações.
a) Incorreta. 150 bilionésimo de m pode ser representado por 1,5 x 10\textsuperscript{--7}m.
b) Correta. 150 bilionésimo de m pode ser representado por:
150 x \frac{1}{1000000000} = 150 x \frac{1}{10\textsuperscript{9}} 
= 150 x 10\textsuperscript{--9} = 1,5 x 10\textsuperscript{--7}m.
c) Incorreta. 150 bilionésimo de m pode ser representado por 1,5 x 10\textsuperscript{--7}m.
d) Incorreta. 150 bilionésimo de m pode ser representado por 1,5 x 10\textsuperscript{--7}m.
\end{enumerate}

\colorsec{Matemática – Módulo 3 – Treino}

\begin{enumerate}
\item
SAEB: Determinar uma fração geratriz para uma dízima periódica.
a) Incorreta. 1)  A dízima periódica 0,126126... é representada pela 
fração \frac{14}{111}.
b) Correta. 0,126126... = \frac{126}{999} = \frac{14}{111}.
c)  Incorreta. 1)  A dízima periódica 0,126126... é representada pela 
fração \frac{14}{111}.
d)  Incorreta. 1)  A dízima periódica 0,126126... é representada pela 
fração \frac{14}{111}.

\item
SAEB: Identificar frações equivalentes.
a) Incorreta. x = 36.
b) Incorreta. x = 36.
c) Correta. Como o valor 111 é igual a 3 x 37, então basta fazer 
3 x 12 = 36, desta forma x = 36.
d) Incorreta. x = 36.

\item
SAEB: Representar frações menores ou maiores que a unidade por 
meio de representações pictóricas ou associar frações a representações 
pictóricas.
a) Incorreta. Juca consumiu \frac{2}{5} do total de pedaços. 
b) Correta. \frac{2}{3} de 12 são 8 pedaços e \frac{1}{2} de 12 são 6
pedaços, ou seja, Juca comeu 14 pedaços dos 35 disponíveis. A fração que
corresponde à quantidade de pedaços que ela consumiu foi \frac{14}{35} 
= \frac{2}{5}.
c) Incorreta. Juca consumiu \frac{2}{5} do total de pedaços.
d) Incorreta. Juca consumiu \frac{2}{5} do total de pedaços.
\end{enumerate}

\colorsec{Matemática – Módulo 4 – Treino}

\begin{enumerate}
\item
SAEB: Resolver problemas que envolvam porcentagens, incluindo os
que lidam com acréscimos e decréscimos simples, aplicação de
percentuais sucessivos e determinação das taxas percentuais.
BNCC: EF09MA05 -- Resolver e elaborar problemas que envolvam porcentagens, com a ideia de aplicação de percentuais sucessivos e a determinação das taxas percentuais, preferencialmente com o uso de tecnologias digitais, no contexto da educação financeira.
a) Incorreta. A segunda parcela foi de R\$ 1.505,00.
b) Incorreta. A segunda parcela foi de R\$ 1.505,00.
c) Correta. Como foram pagos 30\% na entrada, sobraram 70\% para a segunda 
parcela. Para saber o valor da segunda parcela precisamos calcular 70\% de
2150, da seguinte maneira: 0,70 x 2150 = 1505.
d) Incorreta. A segunda parcela foi de R\$ 1.505,00.

\item
SAEB: Resolver problemas que envolvam porcentagens, incluindo os
que lidam com acréscimos e decréscimos simples, aplicação de
percentuais sucessivos e determinação das taxas percentuais.
BNCC: EF09MA05 -- Resolver e elaborar problemas que envolvam porcentagens, com a ideia de aplicação de percentuais sucessivos e a determinação das taxas percentuais, preferencialmente com o uso de tecnologias digitais, no contexto da educação financeira.
a) Correta. Hugo comprou 150 figurinhas, mas apenas 60 não eram repetidas.
Sendo assim ele usou efetivamente \frac{60}{150} = 0,40 = 40\%.
b) Incorreta. Hugo usou efetivamente \frac{60}{150} = 0,40 = 40\%.
c) Incorreta. Hugo usou efetivamente \frac{60}{150} = 0,40 = 40\%.
d) Incorreta. Hugo usou efetivamente \frac{60}{150} = 0,40 = 40\%.

\item
SAEB: Resolver problemas que envolvam porcentagens, incluindo os
que lidam com acréscimos e decréscimos simples, aplicação de
percentuais sucessivos e determinação das taxas percentuais.
BNCC: EF09MA05 -- Resolver e elaborar problemas que envolvam porcentagens, com a ideia de aplicação de percentuais sucessivos e a determinação das taxas percentuais, preferencialmente com o uso de tecnologias digitais, no contexto da educação financeira.
a) Incorreta. O valor pago foi de R\$ 425,60.
b) Correta. Houve dois descontos sucessivos. O primeiro foi de 20\%; o outro,
de 5\%. (1 -- 0,2) x (1 -- 0,05) x R\$ 560 = 0,8 x 0,95 x R\$ 560 = R\$ 425,60.
c) Incorreta. O valor pago foi de R\$ 425,60.
d) Incorreta. O valor pago foi de R\$ 425,60.
\end{enumerate}

\colorsec{Matemática – Módulo 5 – Treino}

\begin{enumerate}
\item
SAEB: Resolver uma equação polinomial de 1o grau.
a) Incorreta. O custo total de Carlos foi de R\$ 90,00.
b) Incorreta. O custo total de Carlos foi de R\$ 90,00.
c) Correta. Como são R\$ 16,00 por hora mais R\$ 10,00 fixos de taxa, 
podemos escrever a seguinte expressão: C = 10 +16x, na qual, ao substituir 
o x por 5, teremos C = 10 + 16 x 5 = 90.
d) Incorreta. O custo total de Carlos foi de R\$ 90,00.

\item
SAEB: Resolver problemas que possam ser representados por sistema de equações de 1o grau com duas incógnitas.
a) Incorreta. Existem 40 carros no estacionamento.
b) Correta. Observe:

\begin{displaymath}
\left\{\begin{matrix}
C + M = 50 &  & \\ 
4C + 2M = 160 &  & 
\end{matrix}\right.

\sim 

\left\{\begin{matrix}
-2C - 2M = -100 &  & \\ 
4C + 2M = 160 &  & 
\end{matrix}\right.

\rightarrow

2C =60 

\rightarrow

C = 30
\end{displaymath}

Há 30 carros no estacionamento.

c) Incorreta. Existem 40 carros no estacionamento. 
d) Incorreta. Existem 40 carros no estacionamento. 

\item
SAEB: Resolver uma equação polinomial de 1o grau
a) Incorreta. O produto desses três números é 1320.
b) Incorreta. O produto desses três números é 1320.
c) Correta. Observe a resolução a seguir:

x + (x+1) + (x+2) = 33
3x = 30
x = 10
x+1 = 11
x+2 = 12
10 x 11 x 12 = 1320

d) Incorreta. O produto desses três números é 1320.} 
\end{enumerate}

\colorsec{Matemática – Módulo 6 – Treino}

\begin{enumerate}
\item

\item

\item
\end{enumerate}

\colorsec{Matemática – Módulo 7 – Treino}

\begin{enumerate}
\item

\item

\item
\end{enumerate}

\colorsec{Matemática – Módulo 8 – Treino}

\begin{enumerate}
\item

\item

\item
\end{enumerate}

\colorsec{Matemática – Módulo 9 – Treino}

\begin{enumerate}
\item

\item

\item
\end{enumerate}

\colorsec{Matemática – Módulo 10 – Treino}

\begin{enumerate}
\item

\item

\item
\end{enumerate}

\colorsec{Matemática – Módulo 11 – Treino}

\begin{enumerate}
\item

\item

\item
\end{enumerate}

\colorsec{Matemática – Módulo 12 – Treino}

\begin{enumerate}
\item

\item

\item
\end{enumerate}

\colorsec{Matemática – Módulo 13 – Treino}

\begin{enumerate}
\item

\item

\item
\end{enumerate}

\colorsec{Matemática – Módulo 14 – Treino}

\begin{enumerate}
\item

\item

\item
\end{enumerate}

\colorsec{Matemática – Módulo 15 – Treino}

\begin{enumerate}
\item

\item

\item
\end{enumerate}

\colorsec{Educação Física – Módulo 1 –  Treino}

\begin{enumerate}
\item
SAEB: Identificar as diferentes valências físicas necessárias à
realização de práticas corporais (jogos eletrônicos, lutas, práticas
corporais de aventura, ginásticas, esportes e dança).
BNCC: EF67EF06 - Analisar as transformações na organização e na prática
dos esportes em suas diferentes manifestações (profissional e
comunitário/lazer).
a) Correta. de fato, a valência física trabalhada pelo atleta da imagem 
é a força, porque ele está vencendo uma caga externa composta por anilhas
e barra.
b) Incorreta. O levantamento olímpico trabalha a força, não a
resistência.
c) Incorreta. A velocidade não é praticada nesse esporte.
d) Incorreta. Nesse esporte, a flexibilidade não é priorizada.

\item
SAEB: Analisar as transformações históricas, o processo de
esportivização e a midiatização das práticas corporais, com ênfase nas
lutas.
BNCC: EF89EF18 - Discutir as transformações históricas, o processo de
esportivização e a midiatização de uma ou mais lutas, valorizando e
respeitando as culturas de origem.
a) Incorreta. Não há alusão, no texto, à criação de
federações ou confederações da luta marajoara. 
b) Correta. A luta marajoara faz parte de um evento oficial de
competição, os Jogos Estudantis Paraenses.
c) Incorreta. Não há alusão, no texto, à alteração do objetivo da luta
marajoara. 
d) Incorreta. A semelhança entre a luta marajoara e o
wrestling não define essa prática corporal como esporte.

\item
SAEB: Identificar o valor do patrimônio urbano e natural
nas vivências das práticas corporais de aventura urbana e na natureza.
BNCC: EF67EF06 - Analisar as transformações na organização e na prática
dos esportes em suas diferentes manifestações (profissional e
comunitário/lazer).
a) Incorreta. Não há alusão, no texto, ao propósito de formar novos
atletas.
b) Incorreta. Não há alusão, no texto, ao uso do skate como meio
de locomoção. 
c) Correta. O segundo parágrafo é explícito quanto à demanda da 
população por um espaço adequado para a prática do esporte.
d) Incorreta. Não há alusão, no texto, ao propósito de criação de novos investimentos da prefeitura.
\end{enumerate}

\colorsec{Educação Física – Módulo 2 –  Treino}

\begin{enumerate}
\item
SAEB: Diferenciar os esportes com base nos critérios de sua lógica
interna.
BNCC: EF67EF12 - Planejar e utilizar estratégias para aprender elementos
constitutivos das danças urbanas.
a) Incorreta. A disputa por medalhas olímpicas não caracteriza se uma
modalidade é um esporte oficial ou não.
b) Incorreta. A dança urbana não foi modalidade esportiva de olimpíadas 
anteriores. A estreia ocorrerá em 2024.
c) Incorreta. A participação de homens e mulheres não caracteriza o
breaking como esporte oficial.
d) Correta. No texto, afirma-se que, além das Olimpíadas, os atletas
devem participar de outras competições oficiais.

\item
SAEB: Diferenciar as danças urbanas, seus elementos constitutivos
e seu valor cultural nas demais manifestações da dança.
BNCC: EF67EF13 - Diferenciar as danças urbanas das demais manifestações
da dança, valorizando e respeitando os sentidos e significados
atribuídos a eles por diferentes grupos sociais.
a) Correta. De acordo com o texto, no hip hop o rap, o grafite e
o break são formas de expressão das dificuldades dos moradores da 
periferia.
b) Incorreta. A origem do hip hop está mais associada à expressão das 
adversidades das pessoas da periferia do que ao incentivo à dança.
c) Incorreta. A origem do hip hop está mais associada à expressão das 
adversidades das pessoas da periferia do que à criação de uma nova
modalidade de dança.
d) Incorreta. Não há, no texto, referências ao hip hop como forma de 
assistência social do governo.

\item
SAEB: Diferenciar as danças urbanas, seus elementos constitutivos
e seu valor cultural nas demais manifestações da dança.
BNCC: EF67EF13 - Diferenciar as danças urbanas das demais manifestações
da dança, valorizando e respeitando os sentidos e significados
atribuídos a eles por diferentes grupos sociais.
a) Incorreta. A origem asiática do k-pop não implica que essa modalidade
seja considerada dança urbana.
b) Incorreta. A prática da dança em grupo não tem relação com sua
classificação com dança urbana.
c) Incorreta. Os movimentos corporais são constitutivos de quaisquer 
danças, de modo que essa característica não é suficiente para caracterizar 
o hip hop como danças urbanas.
d) Correta. O k-pop se inspirou em elementos do hip hop, como os
movimentos de dança do break e o rap.
\end{enumerate}

\colorsec{Educação Física – Módulo 3 –  Treino}

\begin{enumerate}
\item
Saeb: Avaliar a multiplicidade de padrões de estética corporal
disseminados pela mídia, que geram uma prática excessiva de exercícios e
o uso de recursos ergogênicos.
BNCC: EF89EF09 -- Problematizar a prática excessiva de exercícios físicos
e o uso de medicamentos para a ampliação do rendimento ou
potencialização das transformações corporais, bem como os efeitos do
exercício físico para saúde e sua ausência, relacionada ao sedentarismo
e ao aparecimento de doenças.
a) Correta. A bulimia é caracterizada pela ingestão excessiva de alimentos
seguida pela indução de vômito pelo próprio indivíduo, para evitar
ganho de peso.
b) Incorreta. A anorexia se caracteriza por hábitos alimentares 
alterados, voltados sempre à perda de peso: privação de alimentos, 
dietas restritivas extremas, longos períodos em jejum, uso de 
medicamentos laxativos e inibidores de apetite, além de outros 
comportamentos, como excesso de práticas esportivas e vômitos
induzidos. 
c) Incorreta. A vigorexia é um transtorno no qual o indivíduo
tem de si mesmo uma imagem distorcida: embora tenha compleição
física forte e musculosa, diante do espelho ele se vê como fraco
e franzino.
d) Incorreta. A ortorexia é a obsessão pela alimentação saudável.

\item
SAEB: Avaliar os problemas presentes nos esportes e abordados pela
mídia, tais como doping, violência ou corrupção.
BNCC: EF89EF09 -- Problematizar a prática excessiva de exercícios físicos
e o uso de medicamentos para a ampliação do rendimento ou
potencialização das transformações corporais, bem como os efeitos do
exercício físico para saúde e sua ausência, relacionada ao sedentarismo
e ao aparecimento de doenças.
a) Incorreta. O treino em excesso é definido como \textit{overtrainig},
e não como \textit{doping} ou dopagem.
b) Correta. A ingestão ou aplicação de substâncias proibidas caracteriza
o \textit{doping} ou dopagem, como se pode verificar no próprio texto.
c) Incorreta. A prática de burlar as regras não caracteriza o 
\textit{doping} ou dopagem, que está associado especificamente ao uso
de substâncias ilícitas. 
d) Incorreta. Quem evita o uso de anabolizantes proibidos não está
incorrendo na prática do \textit{doping} ou dopagem.

\item
SAEB: Avaliar a relação entre as práticas corporais e a promoção da
saúde.
BNCC: EF89EF08 -- Discutir, analisar e refletir criticamente as
transformações históricas dos padrões de desempenho, saúde e beleza,
considerando a forma como são apresentados nos diferentes meios
(científico, midiático etc.), identificando e reconhecendo a influência
da mídia nos padrões de comportamento do/no corpo.
a) Incorreta. Segundo o texto, as práticas esportivas aumentam a
longevidade.
b) Incorreta. Segundo o texto, as práticas esportivas contribuem para a 
redução das taxas de gordura, trocada por massa magra.
c) Correta. Segundo o texto, as práticas esportivas melhoram a 
qualidade de vida.  
d) Incorreta. Não há referências, no texto, à redução de reações químicas
no corpo.
\end{enumerate}

\colorsec{Ciências da Natureza – Módulo 1 –  Treino}

\begin{enumerate}
\item
BNCC: EF09CI01 -- Investigar as mudanças de estado físico da matéria e explicar essas transformações com base no modelo de constituição submicroscópica.
a) Incorreta. A matéria com um todo possui volume; dessa forma essa propriedade não justifica o fato de o lixo flutuar na superfície da água. Um exemplo é a areia que possui volume, mas sedimenta e é encontrada no fundo do mar.
b) Incorreta. A maleabilidade refere-se à capacidade de a matéria ser
  moldada, o que não é a propriedade que faz com que esse lixo possa ser catado pelos membros da ONG.
c) Incorreta. O brilho não é uma propriedade relevante para que o lixo seja catado na superfície da água, já que ele não influencia diretamente na densidade do material.
d) Correta. É graças à diferença de densidade do lixo e da água que esses resíduos boiam na superfície e podem ser catados pelos membros
  da ONG.

\item
BNCC: EF09CI03 -- Identificar modelos que descrevem a estrutura da matéria (constituição do átomo e composição de moléculas
simples) e reconhecer sua evolução histórica.
a) Incorreta, pois os elétrons não se encontram no núcleo atômico;
b) Incorreta, pois os elétrons são partículas que apresentam cargas
  negativas;
c) Correta, pois elétrons de fato são partículas negativas que se
  encontram na eletrosfera capazes de se movimentar e liberar energia;
d) Incorreta, pois os elétrons são ficam presos ao núcleo, e sim livres
  na eletrosfera.

\item
BNCC: EF09CI06 -- Classificar as radiações
eletromagnéticas por suas frequências, fontes e aplicações, discutindo e
avaliando as implicações de seu uso em controle remoto, telefone celular, raio X, forno de micro-ondas, fotocélulas etc.
a) Correta. De fato, a transmissão Wi-Fi ocorre por meio de ondas eletromagnéticas que são capturadas por materiais que possuem boa capacidade elétrica, como é o caso do metal, presente nos espelhos.
b) Incorreta. A transmissão do sinal Wi-Fi não ocorre por meio de ondas mecânicas, e sim por meio de ondas eletromagnéticas.
c) Incorreta. Por se tratar de uma propagação por meio de ondas eletromagnéticas, o Wi-Fi é capaz de se propagar no vacúo.
d) Incorreta. O sinal Wi-Fi não é transmitido com base no mesmo tipo de onda que o som, já que o Wi-Fi é transmitido através de ondas eletromagnéticas e o som, de ondas mecânicas.
\end{enumerate}

\colorsec{Ciências da Natureza – Módulo 2 –  Treino}

\begin{enumerate}
\item
BNCC: EF09CI08 -- Associar os gametas à transmissão
das características hereditárias, estabelecendo relações entre
ancestrais e descendentes.
a) Correta. A hereditariedade trata das características e
  informações genéticas transmitidas a cada geração. Ao estudar o código
  genético para entender sobre doenças e outras características da raça
  humana, as relações ancestrais-descendentes estão sendo desmistificadas.
b) Incorreta. A paleontologia trata do estudo de seres vivos que
  habitaram a Terra em um passado remoto.
c) Incorreta. A conservação trata do estudo de técnicas alternativas
  que resultem no uso sustentável dos recursos e na preservação das
  espécies.
d) Incorreta. A taxonomia trata de descrever, identificar e nomear
  seres vivos a partir de critérios estabelecidos.

\item
BNCC: EF09CI10 -- Comparar as ideias evolucionistas de
Lamarck e Darwin apresentadas em textos científicos e históricos,
identificando semelhanças e diferenças entre essas ideias e sua
importância para explicar a diversidade biológica.
a) Incorreta. A alternativa descreve o conceito descrito por
  Lamarck, no qual os indivíduos se modificam para se adaptar ao
  ambiente. Seleção natural é um conceito próximo à ideia de Darwin, que
  menciona que o ambiente seleciona os indivíduos mais aptos.
b) Incorreta. A alternativa descreve o conceito de Lamarck, que,
  posteriormente, foi estudado e modernizado por Darwin, dando origem à
  teoria que utilizamos hoje: a seleção natural.
c) Correta. As mudanças ambientais levam a condições ambientais
  ruins, como alimento escasso. Com isso, obter energia fica
  mais difícil e, por consequência, o investimento de energia em
  ornamentos (como cores fortes) também diminui. Sendo assim, o ambiente
  seleciona apenas os pássaros capazes de sobreviver sob tais condições.
d) Incorreta. Lamarck não acredita que o ambiente selecionava as
  espécies; na verdade, sua teoria falava sobre a mudanças das espécies
  frente às imprevisibilidades do ambiente.

\item
BNCC: EF09CI12 -- Justificar a importância das unidades de
conservação para a preservação da biodiversidade e do patrimônio
nacional, considerando os diferentes tipos de unidades (parques,
reservas e florestas nacionais), as populações humanas e as atividades a
eles relacionados.
a) Incorreta. A Caatinga é um bioma com um dos níveis de
  biodiversidade mais altos.
b) Correta. A Caatinga apresenta alta biodiversidade, sendo 15\% dela
  endêmica; portanto, diante das mudanças ambientais, a construção de
  unidades de conservação nesse bioma deve ser priorizada para garantir a proteção adequada das espécies.
c) Incorreta. O comércio de espécies exóticas é uma medida que iria
  prejudicar a biodiversidade presente na Caatinga, estando em posição
  antagônica ao objetivo das unidades de conservação.
d) Incorreta. Com a construção de indústrias o habitat de diversas
  espécies seria prejudicado, favorecendo a perda de biodiversidade e,
  portanto, estando em posição antagônica ao objetivo das unidades de conservação.
\end{enumerate}

\colorsec{Ciências da Natureza – Módulo 3 –  Treino}

\begin{enumerate}
\item
BNCC: EF09CI14 -- Descrever a composição e a estrutura
do Sistema Solar (Sol, planetas rochosos, planetas gigantes gasosos e
corpos menores), assim como a localização do Sistema Solar na nossa
Galáxia (a Via Láctea) e dela no Universo (apenas uma galáxia dentre
bilhões).
a) Incorreta. Júpiter é um planeta gasoso.
b) Incorreta. Dos planetas citados, apenas a Terra faz parte dos planetas rochosos.
c) Correta. De fato os planetas citados fazem parte dos rochosos.
d) Incorreta. Os planetas citados são os gasosos.

\item
BNCC: EF09CI17 -- Analisar o ciclo evolutivo do Sol (nascimento, vida e morte) baseado no conhecimento das etapas de evolução de estrelas de diferentes dimensões e os efeitos desse processo
no nosso planeta.
a) Incorreta. A presença de elementos pesados é
  característica do fim da vida da estrela.
b) Incorreta. A presença de ferro na superfície e no interior de uma
  estrela é característica do fim da vida dela.
c) Correta. De fato, a presença de grande quantidade de energia,
  poeira cósmica e gás são os elementos presentes no início da vida de uma estrela.
d) Incorreta. A presença de uma Gigante vermelha significa que a estrela está em crescimento.

\item
BNCC: EF09CI16 -- Selecionar argumentos sobre a
viabilidade da sobrevivência humana fora da Terra, com base nas
condições necessárias à vida, nas características dos planetas e nas
distâncias e nos tempos envolvidos em viagens interplanetárias e interestelares.
a) Correta. De fato a distância em que o planeta se encontra de sua
  estrela é o principal ponto que os cientistas observam para
  identificar planetas em zonas habitáveis no espaço.
b) Incorreta. A órbita dos satélites naturais de um planeta não
  necessariamente se relaciona coma formação de água naquele planeta.
c) Incorreta. Ao aumentar a distância entre um planeta e sua estrela,
  a quantidade de radiação e calor podem levar o planeta a superaquecer, o que não favorece o surgimento da vida como conhecemos.
d) Incorreta. Apesar de a inclinação do planeta ser um ponto
  extremamente relevante, a radiação que um planeta deve receber deve
  ser proporcional a sua posição e tamanho, para que haja temperaturas amenas para o surgimento da vida.
\end{enumerate}

\colorsec{Simulado 1}

\begin{enumerate}
\item
\item
\item
\item
\item
\item
\item
\item
\item
\item
\item
\item
\item
\item
\item
\end{enumerate}

\colorsec{Simulado 2}

\begin{enumerate}
\item
\item
\item
\item
\item
\item
\item
\item
\item
\item
\item
\item
\item
\item
\item
\end{enumerate}

\colorsec{Simulado 3}

\begin{enumerate}
\item
\item
\item
\item
\item
\item
\item
\item
\item
\item
\item
\item
\item
\item
\item
\end{enumerate}

\colorsec{Simulado 4}

\begin{enumerate}
\item
\item
\item
\item
\item
\item
\item
\item
\item
\item
\item
\item
\item
\item
\item
\end{enumerate}