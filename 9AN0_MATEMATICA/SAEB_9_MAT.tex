\pagestyle{mat}
\chapter{Conjuntos Numéricos}
\markboth{Módulo 1}{}

\colorsec{Habilidades do SAEB}

\begin{itemize}
  \item Escrever números racionais (representação fracionária ou decimal finita) em sua representação por algarismos ou em língua materna ou associar o registro numérico ao registro em língua materna.
  \item Compor ou decompor números racionais positivos (representação
decimal finita) na forma aditiva, ou em suas ordens, ou em adições e
multiplicações.
  \item Identificar números racionais ou irracionais.
  \item Comparar ou ordenar números reais, com ou sem suporte da reta
numérica, ou aproximar número reais para múltiplos de potência de 10
mais próxima.
  \item Converter uma representação de um número racional positivo para outra
representação.
  \item Identificar um número natural como primo, composto, ``múltiplo/fator
de'' ou ``divisor de'' ou identificar a decomposição de um número natural
em fatores primos ou relacionar as propriedades aritméticas (primo,
composto, “múltiplo/fator de” ou “divisor de”) de um número natural à
sua decomposição em fatores primos.
\end{itemize} 

\colorsec{Habilidade da BNCC}

\begin{itemize}
  \item EF09MA02.
\end{itemize}

\conteudo{Os números racionais e irracionais são duas categorias de números
importantes na matemática.

Os números racionais são todos aqueles que podem ser escritos na forma
de fração, ou seja, como a razão entre dois números inteiros. Eles podem
ser representados no formato decimal finito ou infinito periódico, como
1/2, 0,75 e 0,333\ldots Os números racionais incluem também os números
inteiros e fracionários.

Por outro lado, os números irracionais são aqueles que não podem ser
expressos na forma de fração. Eles possuem infinitas casas decimais não
periódicas, como pi (\Pi), a raiz quadrada de 2 (\sqrt{2}) e a constante
de Euler (e). Os números irracionais não podem ser representados como
uma fração simples de dois números inteiros.

Uma característica importante dos números irracionais é que eles são
infinitos e não repetitivos, o que os torna fundamentais para a
geometria e para outras áreas da matemática.

Em resumo, os números racionais são aqueles que podem ser escritos na
forma de fração, enquanto os números irracionais são aqueles que não
podem. Ambos os tipos de números são importantes e estão presentes em
diversas áreas da matemática e da ciência.}

\colorsec{Atividades}

\num{1} Escreva se os números abaixo são \textbf{Racionais(Q)} ou 
\textbf{Irracionais (I)}.

\begin{escolha}
  \item 0,25

  \item \Pi
  
  \item \frac{2}{7}
  
  \item \sqrt{7}
  
  \item 2,1230012579\ldots{}
\end{escolha}

\coment{
a) Decimal exato é racional.
b) O número pi é irracional.
c) Toda fração é racional.
d) A raiz quadrada de 7 não é exata, está entre os números naturais 2 e
  3, pois 2\textsuperscript{2} = 4 e 3\textsuperscript{2} = 9. Toda raiz quadrada não exata é irracional.
e) Este número é uma dízima \textbf{não} periódica, ou seja,
  irracional.}

\num{2} Localize na reta aproximadamente os locais correspondentes aos
valores abaixo:

\begin{figure}
\centering
\includegraphics[width=4.10036in,height=0.88341in]{./_SAEB_9_MAT/media/image4.png}
\caption{Imagem em preto e branco Descrição gerada automaticamente com
confiança média}
\end{figure}

Construir uma reta graduada com a de cima, preferencialmente com uma
reta em cima com uma seta para o lado direito.

\begin{enumerate}
  \item \frac{2}{5}

  \item -- 1,8
  
  \item \sqrt{5}
\end{enumerate}

\coment{
\begin{figure}
\centering
\includegraphics[width=4.10036in,height=0.88341in]{./_SAEB_9_MAT/media/image4.png}
\caption{Imagem em preto e branco Descrição gerada automaticamente com
confiança média}
\end{figure}}
%Fazer a figura colocando a resposta com a seta e letra.}

\num{3} Em um jogo realizado em sala de aula, o professor colocou
diversos números racionais e irracionais em uma urna. A primeira parte
do jogo consistia em cada grupo de alunos sortear 5 números sem reposição;
posteriormente, eles deveriam colocar os números sorteados em ordem
crescente. Os números sorteados por um destes grupos foram: 
7; \frac{9}{4} ; \sqrt{8} ; 0,75 e \frac{2}{3}. Escreva-os em ordem crescente.

%deixar box médio para resolução
\coment{Considerando as aproximações \frac{9}{4} = 2,25; \sqrt{8} \approx 2,83;
0,75; e \frac{2}{3} = 0,66\ldots{} Temos: \frac{2}{3}; 0,75 ; \frac{9}{4} ; 
\sqrt{8} ; 7. 

\num{4} Complete a tabela.

\begin{table}[]
\begin{tabular}{ll}
\rowcolor[HTML]{34CDF9} 
{\color[HTML]{000000} Representação decimal ou fracionário} & {\color[HTML]{000000} Descrição na língua materna (escrito por extenso)} \\
\textbackslash{}frac\{2\}\{100\} &  \\
 & Três décimos \\
0,003 &  \\
 & Dois milésimos \\
\textbackslash{}frac\{51\}\{10\} & 
\end{tabular}
\end{table}

\coment{
\begin{table}[]
\begin{tabular}{ll}
\rowcolor[HTML]{34CDF9} 
{\color[HTML]{000000} Representação decimal ou fracionário} & {\color[HTML]{000000} Descrição na língua materna (escrito por extenso)} \\
\textbackslash{}frac\{2\}\{100\} & Dois centésimos \\
\textbackslash{}frac\{3\}\{10\} ou 0,3 & Três décimos \\
0,003 & Três milésimos \\
\textbackslash{}frac\{2\}\{1000\} ou 0,002 & Dois milésimos \\
\textbackslash{}frac\{51\}\{10\} & Cinquenta e um décimos
\end{tabular}
\end{table}
}

\num{5} Os números naturais podem ser primos ou compostos. Considerando
os números naturais 87 e 29, como podemos classificá-los? E quais os
critérios você utilizou para definir a classificação?

%deixar box pequeno para resolução
\coment{O número 89 é composto, pois pode ser escrito como 3x29, 
e o número 29 é primo, pois não tem outro divisor primo.}

\num{6} Jorge adora propor desafios matemáticos para seus netos. No
último domingo disse para seu neto que em 2023 ele completou a idade
correspondente ao número de divisores positivos do número 5040. Quantos
aos completou Jorge no ano de 2023?

%deixar box médio para resolução
\coment{Decompondo o número 5040 temos que: com isso o número de divisores
positivos são: (4+1)x(2+1)x(1+1)x(1+1) = 5x3x2x2=60. Desta
forma Jorge completou 60 anos em 2023.}

\num{7} Complete a tabela.

\begin{table}[]
\begin{tabular}{ll}
\rowcolor[HTML]{9698ED} 
Representação decimal & Representação fracionária \\
 & \textbackslash{}frac\{24\}\{10\} \\
2,712 &  \\
0,0003 &  \\
0,25 &  \\
 & \textbackslash{}frac\{7\}\{8\}
\end{tabular}
\end{table}

\coment{
\begin{table}[]
\begin{tabular}{ll}
\rowcolor[HTML]{9698ED} 
Representação decimal & Representação fracionária \\
2,4 & \textbackslash{}frac\{24\}\{10\} \\
2,712 & \textbackslash{}frac\{2712\}\{1000\} \\
0,0003 & \textbackslash{}frac\{3\}\{10.000\} \\
0,25 & \textbackslash{}frac\{25\}\{100\} \\
0,875 & \textbackslash{}frac\{7\}\{8\}
\end{tabular}
\end{table}
} 

\num{8} Decomponha os números abaixo em fatores primos.

\begin{escolha}
  \item 256

  \item 1250

  \item 4800

\end{escolha}

\coment{}
a) 256 = 2\textsuperscript{8}
b) 1250 = 2x5\textsuperscript{4}
c) 4800 = 2\textsuperscript{6} x 3 x 5\textsuperscript{2}
}

\num{9} Escreva o número que corresponde a cada fatoração em números
primos.

\begin{escolha}
\item 2\textsuperscript{3} x 3\textsuperscript{2} x 5\textsuperscript{1}  
\item 3\textsuperscript{3} x 5\textsuperscript{2} x 7\textsuperscript{1}
\end{escolha}

\coment{
a) 360 
b) 4725
}

\num{10} Ao fazer a decomposição de 336 em fatores primos, Thiago chegou
à seguinte expressão 2\textsuperscript{x}x3x7. Qual valor Thiago deve 
colocar no lugar de x?

%deixar box pequeno para resolução
\coment{Como o número 336 = 2\textsuperscript{4} x 3 x 7, temos que X 
é igual a 4.}

\colorsec{Treino}

\num{1} Se o número 2\textsuperscript{3}·3\textsuperscript{4}·5\textsuperscript{X} tem exatamente 60 divisores positivos então o valor de X é:

\begin{escolha}
  \item 5
  \item 4
  \item 3
  \item 2
\end{escolha}

\coment{SAEB: Converter uma representação de um número 
racional positivo para outra representação.
a) Incorreta. x = 2. 
b) Incorreta. x = 2.
c) Incorreta. x = 2.  
d) 2\textsuperscript{3}x3\textsuperscript{4}x5\textsuperscript{X} \rightarrow 
(3+1)x(4+1)x(x+1) = 60 \rightarrow 4x5x(x+1)=60 
20x(x=1)=60 \rightarrow x+1=3 \therefore x=2
Dessa forma, x = 2.}

\num{2} Qual das 4 afirmações a seguir contém uma afirmação verdadeira? 

\begin{escolha}
\item
  Todo múltiplo de 5 é ímpar.
\item
  Todo número ímpar é múltiplo de 5.
\item
  Todo número par é múltiplo de 6.
\item
  Todo múltiplo de 12 é um número par. 
\end{escolha}

\coment{SAEB: Identificar um número natural como primo, composto, 
``múltiplo/fator de'' ou ``divisor de'' ou identificar a decomposição de
um número natural em fatores primos ou relacionar as propriedades aritméticas
(primo, composto, ``múltiplo/fator de'' ou ``divisor de'') de um
número natural à sua decomposição em fatores primos.

a) Incorreta. 10 é múltiplo de 5 e não é ímpar.
b) Incorreta. 3 é ímpar e não é múltiplo de 5.
c) Incorreta. 2 é par e não é múltiplo de 6.
d) Correta. Como 12 é par, todo múltiplo deste número terá o fator 2.}

\num{3} Na reta numérica abaixo, há quatro valores indicados pelas
letras A, B, C e D. Qual destas letras indica a localização do número
1,6?

\begin{figure}
\centering
\includegraphics[width=3.66667in,height=0.85833in]{./_SAEB_9_MAT/media/image31.png}
\caption{Linha do tempo Descrição gerada automaticamente}
\end{figure}

\begin{escolha}
  \item A
  \item B
  \item C
  \item D
\end{enumerate}

\coment{SAEB: Comparar ou ordenar números reais, com ou sem suporte da reta
numérica, ou aproximar número reais para múltiplos de potência de 10
mais próxima.
BNCC: EF09MA02 -- Reconhecer um número irracional como um número real cuja representação decimal é infinita e não periódica, e estimar a localização de alguns deles na reta numérica.
a) Incorreta. A letra que indica a localização do número 1,6 é D.
b) Incorreta. A letra que indica a localização do número 1,6 é D.
c) Incorreta. A letra que indica a localização do número 1,6 é D.
d) Correta. Entre os números 1 e 2, existem 9 marcações, e a letra D está
na sexta marcação, o que corresponde a 1,6.}

\pagestyle{mat}
\chapter{Operações}
\markboth{Módulo 2}{}

\colorsec{Habilidades do SAEB}

\begin{itemize}

  \item Calcular o resultado de adições, subtrações, multiplicações ou divisões
envolvendo número reais.
  \item Calcular o resultado de potenciação ou radiciação envolvendo números
reais.
  \item Resolver problemas de adição, subtração, multiplicação, divisão,
potenciação ou radiciação envolvendo número reais, inclusive notação
científica.
  \item Resolver problemas de contagem cuja resolução envolva a aplicação do
princípio multiplicativo.
  \item Resolver problemas que envolvam as ideias de múltiplo, divisor, máximo
divisor comum ou mínimo múltiplo comum.  

\end{itemize} 

\colorsec{Habilidades da BNCC} 

\begin{itemize}
  \item EF09MA03, EF09MA04.
\end{itemize}

\conteudo{O conjunto dos números reais é a união entre o conjunto dos números
racionais e o conjunto dos números irracionais; dessa forma, o número
real pode ser um número racional ou um número irracional. Por isso, esse
conjunto também contempla o dos números naturais e o dos números
inteiros.

O conjunto dos números reais é o mais utilizado no cotidiano, como na
realização de medições, no cálculo de funções matemáticas, no estudo de
grandezas da física e da química, entre outras situações.

As operações envolvendo os números reais são:

\begin{itemize}
  \item Adição

  \item Subtração

  \item Multiplicação

  \item Divisão

  \item Potenciação

  \item Radiciação
\end{itemize}

A \textbf{potenciação} é a operação matemática que representa a
multiplicação de fatores iguais. Ou seja, usamos a potenciação quando um
número é multiplicado por ele mesmo várias vezes.

Notação:

\begin{figure}
\centering
\includegraphics[width=1.46875in,height=0.63399in]{./_SAEB_9_MAT/media/image32.jpeg}
\caption{potenciação}
\end{figure}

Sendo a \neq 0, temos:

a: Base (número que está sendo multiplicado por ele mesmo).

n: Expoente (número de vezes que o número é multiplicado)

Para melhor entender a potenciação, no caso do número
4\textsuperscript{3} (quatro elevado à terceira potência ou quatro
elevado ao cubo), tem-se:

4\textsuperscript{3} = 4 x 4 x 4 = 16 x 4 = 64

A \textbf{Radiciação} é o método matemático inverso à potenciação. Enquanto os
cálculos com potências são determinados pela multiplicação de elementos
iguais sucessivas vezes, a radiciação procura quais são esses elementos.

Por exemplo: se 8\textsuperscript{2} = 64, podemos dizer que a raiz quadrada de
64 é 8.

\colorsec{Atividades}

\num{1} Urano é o sétimo planeta do Sol e é o terceiro maior no sistema
solar. Foi descoberto por William Herschel em 1781. Tem um diâmetro
equatorial de 51.800 quilômetros (32.190 milhas) e orbita o Sol uma vez
a cada 84,01 anos da Terra. Tem uma distância média do Sol de
2.870.990.000 quilômetros.

Escreva a distância média de Urano ao Sol em notação Científica.

%deixar box pequeno para resolução
\coment{2,87 x 10\textsuperscript{9}km.}

\num{2} Jorge foi fazer uma entrega, em um condomínio, de 50 caixas
pesando 60kg cada caixa. Para cumprir essa tarefa, utilizou o elevador,
que tinha a capacidade máxima de 800kg. Jorge pesa 100kg. Por segurança,
o zelador não deixa que o limite de capacidade do elevador seja ultrapassado.

Quantas vezes será necessário usar o elevador para levar todas as
caixas?

%deixar box médio para resolução
\coment{50 x 60 = 3000 kg de carga. Como Jorge tem 100 kg, é preciso considerar
que o limite de peso que pode ser colocado no elevador será de, no máximo, 700
kg. Para carregar 700 kg, Jorge deverá levar no máximo 11 caixas por vez.
50 : 11 = 4,54 aproximadamente, ou seja, 5 viagens.}

\num{3} Para fazer a publicação e um determinado livro, estimou-se um
custo inicial de R\$ 250.000,00 e ainda um custo de R\$ 10,00 por
unidade. Elisa quer publicar um livro com uma tiragem de 5.000
exemplares. Qual será o custo de cada exemplar?

%deixar box médio para resolução
\coment{R\$ 250.000,00 + (5.000,00 x R\$ 10) = R\$ 300.000,00. 
Como serão publicados, 5.000 exemplares, é preciso
dividir o valor ou seja:  R\$ 300.000,00/5.000 = R\$ 60. 
Portanto, o custo por unidade é de R\$ 60,00.}

\num{4} A cantina da escola optou por vender lanches saudáveis e fez uma
negociação com o distribuidor de maçãs da região. A negociação fez com
que a cada duas maçãs fosse pago o valor de R\$ 0,80. O preço
de venda será de R\$ 1,00 por unidade.

Pretende-se um lucro mínimo semanal de R\$ 90,00 com as maçãs, dados os
custos operacionais internos.

Quantas maçãs precisam ser vendidas?

%deixar box médio para resolução
\coment{Como o custo por unidade é de R\$ 0,40 e o valor de venda é R\$ 1,00,
o lucro por unidade é de R\$ 0,60. Dessa forma, para obter lucro mínimo 
semanal de R\$ 90,00, divide-se o valor pretendido (R\$ 90) pelo lucro 
por unidade (R\$ 0,60): 90/0,60 = 150 maçãs.}

\num{5} Uma xícara com leite pesa 420 g, porém se bebermos metade do
leite ele passa a pesar 230 g.

Qual o peso da xícara vazia?

%deixar box pequeno para resolução
\coment{Ao analisar a diferença do peso da xícara cheia e com metade do leite,
percebe-se que houve redução de 190 g, ou seja, esta é massa ocupada
por metade do leite. Conclui-se, dessa forma, que o leite tem a massa de
190 x 2 = 380g e a xícara 420 -- 380 = 40g.}

\num{6} Leia o texto abaixo.

\begin{quote}
A NGC 4151 está localizada a cerca de 43 milhões de anos-luz
da Terra e se enquadra entre as galáxias jovens que possuem um buraco
negro em intensa atividade. Mas ela não é só lembrada por esses
quesitos. A NGC 4151 é conhecida por astrônomos como o `olho de Sauron',
uma referência ao vilão do filme \textit{O Senhor dos Anéis}.
\end{quote}

\fonte{Folha de São Paulo. Galáxia herda nome de vilão do filme \textit{O Senhor dos Anéis}. Disponível em: http://www1.folha.uol.com.br/ciencia/887260-galaxia-herda-nome-de-vilao-do-filme-o-senhor-dos-aneis.shtml}
Acesso em: 8 mai.2023.}

Para facilitar cálculos, números são utilizados em notação científica.
A notação científica do número \textbf{43 milhões de anos-luz}, do texto 
acima, está corretamente representada na alternativa:

\begin{escolha}

  \item 4,3 x 10\textsuperscript{7}

  \item 43 x 10\textsuperscript{7}

  \item 4,3 x 10\textsuperscript{--7}

  \item 43 x 10\textsuperscript{--7} 

\end{escolha}

\coment{43 milhões representam 43.000.000, ou seja, 43 X 10\textsuperscript{6}
ou 4,3 x 10\textsuperscript{7}.}

\num{7} Um professor de Educação Física quer mesclar os jogos que são
oferecidos para suas turmas e com isso estabeleceu uma regularidade de
repetição dos esportes das turmas. As partidas de futebol acontecem a 
cada 30 dias, as de basquete a cada 45 dias e as de handebol, a cada 60 dias.

Todas as modalidades foram oferecidas no dia de hoje. Quando essa
situação ocorrerá novamente?

%deixar box pequeno para resolução
\coment{Determinando o MMC (30, 45, 60) = 180
A situação ocorrerá novamente após 180 dias.}

\num{8} Durante uma campanha de arrecadação de alimentos na escola,
a turma da Larissa arrecadou: 288 pacotes de feijão, 96 pacotes de açúcar,
360 pacotes de arroz e 240 pacotes de fubá. Larissa recorreu a seus
conhecimentos matemáticos para organizar pacotes que cada família
beneficiária receberia.

Quantos pacotes de feijão foram colocados em cada pacote para entregar,
considerando que o objetivo era atender a maior quantidade possível de
famílias?

%deixar box médio para resolução
\coment{MDC (96, 240, 288, 360) = 24. Dessa forma, podemos determinar que, em
cada pacote, foram colocados 288/24 = 12 pacotes de feijão.}

\num{9} Uma costureira dispõe de dois pedaços de fita de comprimentos
1,20m e 1,80m. Ela pretende cortar essas fitas em pedaços menores e 
iguais, com o maior tamanho possível. Quantos pedaços menores de fita 
e de máximo tamanho possível ela conseguirá?

%deixar box pequeno para resolução
\coment{MDC (1,20 e 1,80) = 0,60. Ela conseguirá cortar 2 pedaços da 
fita de 1,20 e 3 pedaços da fita de 1,80. Os pedaços serão de 60 cm.}

\num{0} Simplificando a expressão \sqrt[5]{\frac{2\textsuperscript{16}+2\textsuperscript{18}}{10}} encontramos

\begin{escolha}
\item 2
\item 4
\item 6
\item 8
\end{escolha}

\coment{
a) Incorreta. A expressão simplificada é igual a 2\textsuperscript{3}.
b) Incorreta. A expressão simplificada é igual a 2\textsuperscript{3}.
c) Incorreta. A expressão simplificada é igual a 2\textsuperscript{3}.
d) Correta.
\sqrt[5]{\frac{2\textsuperscript{16}+2\textsuperscript{18}}{10}} = \sqrt[5]{\frac{2\textsuperscript{15} x (2\textsuperscript{1} + 2\textsuperscript{3})}{10}} = \sqrt[5]{\frac{2\textsuperscript{15} x 10}{10}} = 2\textsuperscript{3}
}

\colorsec{Treino}

\num{1} Fernando faz caminhada a cada 4 dias. Thiago, vizinho de
Fernando, faz caminhada no mesmo parque, a cada 6 dias. Considerando que
Fernando e Thiago se encontraram sexta fazendo caminhada, em qual dia da
semana ocorrerá o próximo encontro entre eles?

\begin{escolha}
\item Segunda
\item Terça
\item Quarta
\item Sexta
\end{escolha}

\coment{SAEB: Resolver problemas que envolvam as ideias de múltiplo, 
divisor, máximo divisor comum ou mínimo múltiplo comum.

a) Incorreta. Eles se encontrarão em uma quarta.
b) Incorreta. Eles se encontrarão em uma quarta.
c) Correta. MMC (4,6) = 12 dias. Após 7 dias será sexta-feira, e podemos 
continuar contando (o 8º dia é sábado; o 9º, domingo; o 10º, segunda;
o 11º, terça; e o 12º, quarta).
d) Incorreta. Eles se encontrarão em uma quarta.}

\num{2} Para arrecadar dinheiro para a festa de formatura do 9º ano,
Julinha quer vender doces. Ela preparou 150 brigadeiros e 120 beijinhos,
e quer vender pacotes separados desses doces. Para facilitar as contas,
ela quer ter o mesmo número de pacotes de brigadeiros e de beijinhos, e
cada um desses pacotes deve conter o maior número possível de guloseimas. 
Quantos pacotes ela deve preparar? E quantos doces deve conter cada 
pacote?   

\begin{escolha}
\item
  \textbf{30 pacotes de cada doce, 5 brigadeiros e 4 beijinhos.}
\item
  10 pacotes de cada doce, 15 brigadeiros, 12 beijinhos.
\item
  15 pacotes de cada doce, 10 brigadeiros, 12 beijinhos.
\item
  90 pacotes de cada doce, 3 brigadeiros, 3 beijinhos.
\end{escolha}

\coment{SAEB: Resolver problemas que envolvam as ideias de múltiplo, 
divisor, máximo divisor comum ou mínimo múltiplo comum.

a) Correta. MDC (120, 150) = 30, sendo assim temos 150 : 30 = 5 e 120 : 30 = 4. Julinha montará 30 pacotes de cada doce. Os pacotes de brigadeiro terão 5 doces; os de beijinhos, 4.
b) Incorreta. Julinha montará 30 pacotes de cada doce. Os pacotes de brigadeiro terão 5 doces; os de beijinhos, 4.
c) Julinha montará 30 pacotes de cada doce. Os pacotes de brigadeiro terão 5 doces; os de beijinhos, 4.
d) Julinha montará 30 pacotes de cada doce. Os pacotes de brigadeiro terão 5 doces; os de beijinhos, 4. 
}

\num{3} Leia o texto a seguir para responder à pergunta.

\begin{quote}
Cientistas australianos podem ter descoberto algo surpreendente nas
profundezas do mar: criaturas misteriosas, aparentemente vivas e tão
pequenas que chegam ao limite do que é necessário para que exista vida
independente.

Seus descobridores, da Universidade de Queensland, afirmam que esses
seres são novas formas de micróbios. Céticos, no entanto, veem a
descoberta como uma provável nova desilusão na caçada às menores formas
de vida.

Os pesquisadores as chamam de \textit{nanobes}. O ``nano'' vem de seu 
comprimento, de 20 a 150 nanômetros (bilionésimos de metro).
\end{quote}

\fonte{William J. Broad. Equipe diz ter descoberto menor forma de vida.
Disponível em: https://www1.folha.uol.com.br/fsp/ciencia/fe1901200001.htm.
Acesso em: 9 mai. 2023.}

O maior tamanho do nanobe em notação científica está corretamente
representado na alternativa:

\begin{escolha}
\item 15 x 10\textsuperscript{--7} m.
\item 1,5 x 10\textsuperscript{--7} m.
\item 150 x 10\textsuperscript{--9} m.
\item 1,5 x 10\textsuperscript{--9} m.
\end{escolha}

\coment{SAEB: Resolver problemas de adição, subtração, multiplicação, 
divisão, potenciação ou radiciação envolvendo número reais, inclusive
notação científica.
BNCC: EF09MA04 -- Resolver e elaborar problemas com números reais, 
inclusive em notação científica, envolvendo diferentes operações.
a) Incorreta. 150 bilionésimo de m pode ser representado por 1,5 x 10\textsuperscript{--7}m.
b) Correta. 150 bilionésimo de m pode ser representado por:
150 x \frac{1}{1000000000} = 150 x \frac{1}{10\textsuperscript{9}} 
= 150 x 10\textsuperscript{--9} = 1,5 x 10\textsuperscript{--7}m.
c) Incorreta. 150 bilionésimo de m pode ser representado por 1,5 x 10\textsuperscript{--7}m.
d) Incorreta. 150 bilionésimo de m pode ser representado por 1,5 x 10\textsuperscript{--7}m.
}

\pagestyle{mat}
\chapter{Frações (associadas a imagens e fração geratriz)}
\markboth{Módulo 3}{}

\colorsec{Habilidades do SAEB}

\begin{itemize}

  \item Representar frações menores ou maiores que a unidade por meio de
representações pictóricas ou associar frações a representações pictóricas.
  \item Identificar frações equivalentes.
  \item Determinar uma fração geratriz para uma dízima periódica.   

\end{itemize} 

\conteudo{
A fração geratriz é aquela cujo resultado será uma dízima periódica 
(número decimal periódico) quando calculamos seu valor decimal.

Os números decimais periódicos apresentam um ou mais algarismos que se
repetem infinitamente. Esse algarismo ou algarismos que se repetem
representam o período do número.

A fração geratriz pode ser determinada por meio de operações com equações
ou de algumas regras práticas que podem facilitar seu processo de
identificação.

O processo por equação pode ser realizado da seguinte forma:

\begin{itemize}
  \item Igualar a dízima periódica a uma incógnita, por exemplo x, de
forma a escrever uma equação do 1º grau;

  \item Multiplicar ambos os lados da equação por um múltiplo de 10
(a depender do número de casas em que ocorre a repetição dos termos);

  \item Subtrair a equação encontrada da equação inicial;

  \item Isolar a incógnita.
\end{itemize}

\textbf{Exemplo 1}

0,232323...

x = 0,232323...

100x = 23,232323...

100x -- x = 23,23232323... -- 0,232323...

99x = 23

x = \frac{23}{99}\

\textbf{Exemplo 2} 

0,5123123123...
Neste exemplo, temos um dígito após a vírgula que não se repete (5). 
É necessário, assim, fazer dois passos, para que apenas os
dígitos que se repetem fiquem após a vírgula.

x = 0,5123123123...

10x = 5,123123123...

10000x = 5123,123123...

10000x -- 10x = 5123,123123... -- 5,123123123...

9990x = 5118

x = \frac{5118}{9990}
}


\colorsec{Atividades}

\num{1} Dados dois números reais x e y, tais que x = 2,333... e y =
0,151515... são dízimas periódicas. Qual fração representa a soma de x
e y?

%deixar box grande para resolução
\coment{
x = 2,333\ldots = 2 + \frac{1}{3} = \frac{7}{3} e
y = 0,1515\ldots = \frac{15}{99} = \frac{5}{33}. 
Fazendo x + y = \frac{7}{3} + \frac{5}{33} = \frac{77 + 5}{33} 
= \frac{82}{33}\).
}

\num{2} A representação do número 2,123123123... pode ser colocada pela
fração:

%deixar box médio para resolução
\coment{
2,123123123 = 2 + \frac{123}{999} = \frac{1998 + 123}{999} = \frac{2121}{999} = \frac{707}{333}
}

\num{3} Beatriz precisa encontrar uma fração na forma irredutível que
represente o número 5,151515... . Qual será a fração que ela deve
encontrar?

%deixar box médio para resolução
\coment{5,151515\ldots = 5 + \frac{15}{99} = 5 + \frac{5}{33} = \frac{165 + 5}{33} = \frac{170}{33}}

\num{4} Na imagem abaixo, faça um x sobre os bombons de forma a
representar a fração equivalente a \frac{5}{9}.

%pedido do autor: Fazer uma imagem semelhante a imagem abaixo sem os pontilhados, o aluno irá preencher a resposta sobre a imagem. 

\begin{figure}
\centering
\includegraphics[width=2.9871in,height=1.23958in]{./_SAEB_9_MAT/media/image43.png}
\caption{Desenho de cachorro Descrição gerada automaticamente com
confiança baixa}
\end{figure}

\coment{

%Pedido do autor: Montar imagem como a de baixo

\begin{figure}
\centering
\includegraphics[width=3.55137in,height=1.38021in]{./_SAEB_9_MAT/media/image44.png}
\caption{Desenho de cachorro Descrição gerada automaticamente com
confiança baixa}
\end{figure}

\num{5} Quando Artur chegou em casa, encontrou uma pizza com alguns pedaços
já consumidos. Como estava com fome, ele também comeu alguns pedaços.
Qual fração de pizza Artur comeu?

%Pedido do autor: montar imagem como a de baixo

\begin{figure}
\centering
\includegraphics[width=2.75521in,height=0.88894in]{./_SAEB_9_MAT/media/image45.png}
\caption{Uma imagem contendo Seta Descrição gerada automaticamente}
\end{figure}

\coment{Por meio da imagem, pode-se verificar que, quando Artur chegou em
casa, a pizza ainda tinha 8 pedaços de 10 (porque dois haviam sido 
comidos). Depois de Artur comer, ainda restavam 5 fatias, de
maneira que podemos concluir que Artur consumiu 3, ou seja,
\frac{3}{10}.}

\num{6} A fração de \frac{2}{7}, ao ser somada com a fração
representada na figura a seguir, será igual a:

\begin{figure}
\centering
\includegraphics[width=1.26562in,height=1.16258in]{./_SAEB_9_MAT/media/image46.png}
\caption{Foto preta e branca de um guarda-chuva Descrição gerada
automaticamente com confiança média}
\end{figure}

%deixar box pequeno para resolução
\coment{\frac{2}{7} + \frac{5}{8} = \frac{16 + 35}{56} = \frac{51}{56}}

\num{7} Qual das imagens abaixo representa 4 décimos?

\begin{escolha}

\item
  \begin{figure}
  \centering
  \includegraphics[width=1.0625in,height=1.28788in]{./_SAEB_9_MAT/media/image47.png}
  \caption{Gráfico, Gráfico de barras Descrição gerada automaticamente}
  \end{figure}

\item
  \includegraphics[width=1.07728in,height=1.29167in]{./_SAEB_9_MAT/media/image48.png}

\item
  \includegraphics[width=1.00521in,height=1.25518in]{./_SAEB_9_MAT/media/image49.png}

\item
  \includegraphics[width=0.9751in,height=1.25in]{./_SAEB_9_MAT/media/image50.png}
\end{escolha}

\coment{Observando as imagens apresentadas, verifica-se que todas elas
são divididas em dez colunas. Para obter a proporção solicitada no
enunciado, basta verificar o número de colunas pintadas em relação
ao total de 10.    
a) Incorreta. Nesta alternativa, seis colunas estão pintadas, de modo que
a proporção representada é de \frac{6}{10}. 
a) Correta. Nesta alternativa, quatro colunas estão pintadas, de modo que
a proporção representada é de \frac{4}{10}.
c) Incorreta. Nesta alternativa, três colunas estão pintadas, de modo que
a proporção representada é de \frac{3}{10}.
d) Incorreta. Nesta alternativa, duas colunas estão pintadas, de modo que
a proporção representada é de \frac{2}{10}.
}

\num{8} Considerando o diagrama abaixo, qual fração representa a
quantidade de crianças que preferem hamster como animal de estimação?

\includegraphics[width=3.85937in,height=1.64922in]{./_SAEB_9_MAT/media/image51.png}

\coment{Pela contagem, o total de 17 bolinhas representa 34 crianças.
Dessa forma, 12 crianças que preferem hamsters. A fração é \frac{12}{34} = \frac{6}{17}.}

\num{9} No diagrama abaixo, é possível observar a quantidade de peixes
capturados por Angélica, Pedro, Melissa e Leandro.

\includegraphics[width=3.65104in,height=1.34767in]{./_SAEB_9_MAT/media/image52.png}

Qual a fração equivalente ao número de peixes do Pedro em relação a
todos?

%deixar box pequeno para resolução
\coment{De acordo com o diagrama, o total de peixes de Angélica, Pedro, 
Melissa e Leandro é de 16. Pedro tem 3 peixes. Dessa forma, a fração da
quantidade de peixes que Pedro tem em relação ao total é
\frac{3}{16}. 

\num{10} O professor de André montou uma grande lista de exercícios 
e avisou aos alunos que alguns deles iriam cair na avaliação. Alice
já resolveu \frac{4}{5} dos exercícios, Bruno \frac{2}{7}, Augusto 
\frac{12}{15} e Cadu \frac{12}{10}.

Considerando essas informações, quais são os alunos que resolveram a mesma
quantidade de exercícios?

\coment{Augusto resolveu \frac{12}{15} -- isto é, \frac{4}{5} -- dos 
exercícios. Cadu fez \frac{12}{10}, ou seja,\frac{6}{5}. As frações
equivalentes são, portanto, as de Alice e Augusto.} 

\colorsec{Treino}

\num{1} A dízima periódica 0,126126... é representada pela fração:

\begin{escolha}

\item \frac{126}{1000}
\item \frac{14}{111}
\item \frac{126}{990}
\item \frac{63}{99}

\end{escolha}

\coment{SAEB: Determinar uma fração geratriz para uma dízima periódica.

a) Incorreta. 1)  A dízima periódica 0,126126... é representada pela 
fração \frac{14}{111}.
b) Correta. 0,126126... = \frac{126}{999} = \frac{14}{111}.
c)  Incorreta. 1)  A dízima periódica 0,126126... é representada pela 
fração \frac{14}{111}.
d)  Incorreta. 1)  A dízima periódica 0,126126... é representada pela 
fração \frac{14}{111}.
}

\num{2} O número x, de forma a que as frações
\frac{12}{37} = \frac{x}{111} sejam equivalentes, é:

\begin{escolha}

\item 12
\item 24
\item 36
\item 48

\end{escolha}

\coment{SAEB: Identificar frações equivalentes.

a) Incorreta. x = 36.
b) Incorreta. x = 36.
c) Correta. Como o valor 111 é igual a 3 x 37, então basta fazer 
3 x 12 = 36, desta forma x = 36.
d) Incorreta. x = 36.
}

\num{3} Observando os pedaços de chocolate na imagem a seguir, percebemos 
que há três tipos espalhados: chocolate meio amargo, chocolate ao leite e
chocolate branco. 

\begin{figure}
\centering
\includegraphics[width=4.54097in,height=1.51544in]{./_SAEB_9_MAT/media/image53.png}
\caption{Desenho de personagens Descrição gerada automaticamente com
confiança baixa}
\end{figure}

\href{https://br.freepik.com/vetores-gratis/conjunto-de-varias-fatias-de-chocolate_10155086.htm\#page=2\&query=chocolate\&position=49\&from_view=search\&track=sph}{Free
Vector \textbar{} Vetor grátis conjunto de várias fatias de chocolate
(freepik.com)}

Juca comeu \frac{2}{3} do chocolate meio amargo e \frac{1}{2} do
chocolate ao leite.

A fração correspondente à quantidade que Juca comeu em relação ao total
é:

\begin{escolha}

\item
  \includegraphics[width=2.11685in,height=0.29169in]{./_SAEB_9_MAT/media/image54.png}
\item
  \includegraphics[width=2.10852in,height=0.30003in]{./_SAEB_9_MAT/media/image55.png}
\item
  \includegraphics[width=2.08351in,height=0.25002in]{./_SAEB_9_MAT/media/image56.png}
\item
  \includegraphics[width=2.10852in,height=0.30003in]{./_SAEB_9_MAT/media/image57.png}
\end{escolha}

%Essas imagens da alternativa foram feitas pelo autor, podem ser mantidas ou alteradas para se adequar ao projeto.

\coment{SAEB: Representar frações menores ou maiores que a unidade por 
meio de representações pictóricas ou associar frações a representações 
pictóricas.

a) Incorreta. Juca consumiu \frac{2}{5} do total de pedaços. 
b) Correta. \frac{2}{3} de 12 são 8 pedaços e \frac{1}{2} de 12 são 6
pedaços, ou seja, Juca comeu 14 pedaços dos 35 disponíveis. A fração que
corresponde à quantidade de pedaços que ela consumiu foi \frac{14}{35} 
= \frac{2}{5}.
c) Incorreta. Juca consumiu \frac{2}{5} do total de pedaços.
d) Incorreta. Juca consumiu \frac{2}{5} do total de pedaços.
}

\pagestyle{mat}
\chapter{Porcentagem Aumentos e Descontos}
\markboth{Módulo 4}{}

\colorsec{Habilidades do SAEB}

\begin{itemize}

  \item Resolver problemas que envolvam porcentagens, incluindo os que 
  lidam com acréscimos e decréscimos simples, aplicação de percentuais
  sucessivos e determinação de taxas percentuais.   

\end{itemize} 

\colorsec{Habilidade da BNCC}

\begin{itemize}
  \item EF09MA05.
\end{itemize}

\conteudo{Para determinar aumentos ou descontos sucessivos, utilizamos o 
\textbf{fator de correção}. O fator de correção é dado por F = (1 + i), 
onde i é a taxa percentual de maneira unitária.

Observe os exemplos a seguir.

\begin{itemize}
\item
  3\% = 0,03
\item
  12\% = 0,12
\item
  50\% = 0,5
\end{itemize}

A composição de aumentos e descontos sucessivos pode ser composta por
vários aumentos e vários descontos.

Por exemplo: como calcular o aumento total de um produto que 
passou por dois aumentos sucessivos -- um de 10\% e outro de 15\%?

O percentual total não depende do valor do produto, por isso podemos
fazer:

(1 + 0,1) x (1 + 0,15) = 1,265 \rightarrow 1 + 0,265 , ou seja, um aumento 
de 26,5\% no total.
}

\colorsec{Atividades}

\num{1} Em um período de alta inflação, os itens da cesta básica tiveram
um aumento de 75\%.

Considerando que o preço inicial a cesta básica era de R\$ 60,00, qual
o preço no final do referido período?

%deixar box pequeno para resolução
\coment{Calculando 75\% de 60, teremos 0,75 x 60=45, ou seja, no final
do período teremos o valor de R\$ 105,00}.

\num{2} Para comprar um tênis no valor de R\$ 250,00, existem duas opções
para pagar, que são as seguintes:

\begin{itemize}
  \item \textbf{Plano I}: Pagamento à vista com 20\% de desconto;

  \item \textbf{Plano II}: Pagamento em duas parcelas iguais sem aumento, a primeira no dia da compra e a segunda um mês depois.
\end{itemize}

Embora o Plano II seja apresentado ao público como vantajoso, o valor
parcelado é mais alto que o do Plano I, pago à vista.

Qual a taxa de juros aplicada no pagamento parcelado?

%deixar box grande para resolução
\coment{O valor do tênis pode até não ser considerado no cálculo, mas, para
facilitar para os estudantes, vamos utilizá-lo.

No Plano I, há desconto de 20\% sobre o valor de R\$ 250,00. Nesse caso,
o valor pago seria de R\$ 200,00, pois seriam dados R\$ 50,00 de desconto.

No Plano II, o comprador tem que desembolsar metade do valor no dia da
compra, ou seja, pagaria no ato R\$ 125,00 e mais R\$ 125,00
após 30 dias, pagando um total de R\$ 250,00. Como, nesse caso,
\textit{as duas parcelas são iguais}, o valor pago em relação ao que 
seria o valor à vista é de \frac{125}{75} = 1,667, ou seja, 66,7\%
a mais. 
}

\num{3} Ao longo do ano de 2017 o preço de um artigo esportivo foi
reajustado da seguinte forma: de 15 de março a 15 de abril sofreu um
aumento de 30\%; de 15 de março a 15 de maio, 56\%; de 15 de março a 15
de junho, 48,2\% e de 15 de março a 15 de julho, 90\%.

\begin{figure}
\centering
\includegraphics[width=3.25in,height=2.18333in]{./_SAEB_9_MAT/media/image58.png}
\caption{Uma imagem contendo Retângulo Descrição gerada automaticamente}
\end{figure}

Considerando que o preço em 15/5 era R\$ 120,00, qual foi preço praticado
em 15/7, aproximadamente?

%deixar box médio para resolução
\coment{Como em 15/5 o preço estava com 56\% de aumento, podemos retroagir
para o valor em 15/3. Para isso, dividimos 120 por 1,56 e obtemos o valor de
aproximadamente 76,92. Agora podemos fazer o aumento de 90\% que
determinará o valor procurado. Assim: 76,92 x 1,9 = 146,15.}

\num{4} Observe os valores na tabela. 

\begin{table}[]
\begin{tabular}{|lll|}
\hline
\multicolumn{1}{|c}{\textbf{Ano}} & \multicolumn{1}{c}{\textbf{Escola A}} & \multicolumn{1}{c|}{\textbf{Escola B}} \\ \hline
\textbf{2022} & R\$ 1000,00 & R\$ 1500,00 \\ \hline
\textbf{2023} & R\$ 1150,00 & R\$ 1680,00 \\ \hline
\end{tabular}
\end{table}

Determine qual escola teve o maior aumento percentual nas mensalidades
de 2022 para 2023 e diga qual foi o percentual de aumento.

%deixar box médio para resolução
\coment{Calculando a razão da diferença pelo valor inicial teremos:

Escola A: \frac{150}{1000} = 0,15
Escola B: \frac{180}{1500} = 0,12
O maior aumento foi o da escola A: 15\%.}

\num{5} O \textbf{Ideb} é o Índice de Desenvolvimento da
Educação Básica e seu objetivo é medir a qualidade do ensino da
educação básica no Brasil.

O gráfico abaixo mostra os resultados do Ideb nacional por biênio de
2005 a 2015.

\includegraphics[width=4.91667in,height=2.86449in]{./_SAEB_9_MAT/media/image63.wmf}

No ano de 2021 o índice do Ensino Médio no Brasil foi de 4,2. Qual foi 
o percentual de evolução em relação ao ano de 2015?

%deixar box pequeno para resolução
\coment{O índice era de 3,7 e aumentou para 4,2. Verifica-se, portanto,
um aumento de 0,5 o que representa \frac{0,5}{3,7} \cong 1,1351\ldots\ 
ou seja, um aumento de aproximadamente 13,5\%.}

\num{6} Leia o textp abaixo para responder à questão.

\begin{quote}

\textbf{Disparidade de rendimento entre sexos permanece alta
apesar do maior ganho para mulheres}

Além da valorização do salário-mínimo, houve aumento real do rendimento
médio de todas as fontes na comparação entre 2010 e 2000. Em 2010, o
rendimento médio era de R\$ 1.587 para os homens e R\$ 1.074 entre as
mulheres.

\end{quote}

\fonte{Instituto Brasileiro de Geografia e Estatística. Censo 2010.
Disponível em: https://censo2010.ibge.gov.br/noticias-censo?busca=1&id=1&idnoticia=2747&t=estatisticas-genero-mostram-como-mulheres-vem-ganhando-espaco-realidade-socioeconomica-pais&view=noticia.
Acesso em: 10 mai. 2023.}

Qual deve ser o desconto sobre o rendimento médio dos homens para
atingir o valor do salário das mulheres?

%deixar box médio para resolução
\coment{A diferença entre os salários é 1587 -- 1074 = 513. Calculando a
razão \frac{513}{1587} \cong 0,3232\ldots{}, ou seja, aproximadamente
32,3\%.}

\num{7} Observe o gráfico a seguir. 

\begin{figure}
\centering
\includegraphics[width=2.91782in,height=2.79636in]{./_SAEB_9_MAT/media/image64.jpg}
\caption{Gráfico, Gráfico de barras Descrição gerada automaticamente}
\end{figure}

\fonte{Pretrtobras. Disponível em: https://petrobras.com.br/data/files/55/43/0A/53/165954104F528454893851A8/producao-petroleo-VALE-ESTE.jpg. Acesso em: 10 mai. 2023.}

Considerando a previsão realizada em 2013 para atingir a produção de 3,9
milhões de barris por dia, qual deveria ser o percentual de aumento em
relação a 2013?

%deixar box pequeno para resolução
\coment{Calculando a razão \frac{3,9}{2,32} = 1,681\ldots{}. Sendo assim
temos uma previsão de aumento de 68,1\%.}

\num{8} Calcule o percentual total de aumento sobre um produto que teve
dois aumentos sucessivos de 20\%.

%deixar box pequeno para resolução
\coment{(1 + 0,2) x (1 + 0,2) = 1,44. Assim podemos afirmar que tivemos um
aumento de 44\%.}

\num{9} Calcule o percentual total de desconto sobre um produto que teve
dois descontos sucessivos de 10\%.

%deixar box pequeno para resolução
\coment{(1 - 0,1) x (1 - 0,1) = 0,81. Assim podemos afirmar que tivemos um
desconto de 19\%.}

\num{10} Um produto teve um aumento de 20\% e posteriormente um desconto
de 15\%. Considerando o valor final, houve aumento ou desconto? Em que 
porcentagem?

%deixar box pequeno para resolução
\coment{(1 +0,2) x (1 - 0,15) = 1,2 · 0,85 = 1,02. Pode-se afirmar, assim,
que houve aumento de 2\%.}

\colorsec{Treino}

\num{1} Jorge comprou uma televisão por R\$ 2.150,00 para pagar em duas
parcelas. A primeira parcela corresponde a 30\% do preço da
televisão. O restante foi pago após 30 dias.

Quanto Jorge pagou pela segunda parcela?

\begin{escolha}

\item R\$ 645,00
\item R\$ 1.290,00
\item R\$ 1.505,00
\item R\$ 2.120,00
\end{escolha}

\coment{SAEB: Resolver problemas que envolvam porcentagens, incluindo os
que lidam com acréscimos e decréscimos simples, aplicação de
percentuais sucessivos e determinação das taxas percentuais.
BNCC: EF09MA05 -- Resolver e elaborar problemas que envolvam porcentagens, com a ideia de aplicação de percentuais sucessivos e a determinação das taxas percentuais, preferencialmente com o uso de tecnologias digitais, no contexto da educação financeira.

a) Incorreta. A segunda parcela foi de R\$ 1.505,00.
b) Incorreta. A segunda parcela foi de R\$ 1.505,00.
c) Correta. Como foram pagos 30\% na entrada, sobraram 70\% para a segunda 
parcela. Para saber o valor da segunda parcela precisamos calcular 70\% de
2150, da seguinte maneira: 0,70 x 2150 = 1505.
d) Incorreta. A segunda parcela foi de R\$ 1.505,00.}

\num{2} Hugo comprou 150 figurinhas para o seu álbum da Copa do Mundo de
2022. Porém, dessas 150 figurinhas, 90 eram repetidas. As figurinhas que
efetivamente foram utilizadas representam qual porcentagem do total de
figurinhas que ele comprou?

\begin{escolha}

\item 40\%
\item 50\%
\item 60\%
\item 90\%

\end{escolha}

\coment{SAEB: Resolver problemas que envolvam porcentagens, incluindo os
que lidam com acréscimos e decréscimos simples, aplicação de
percentuais sucessivos e determinação das taxas percentuais.
BNCC: EF09MA05 -- Resolver e elaborar problemas que envolvam porcentagens, com a ideia de aplicação de percentuais sucessivos e a determinação das taxas percentuais, preferencialmente com o uso de tecnologias digitais, no contexto da educação financeira.

a) Correta. Hugo comprou 150 figurinhas, mas apenas 60 não eram repetidas.
Sendo assim ele usou efetivamente \frac{60}{150} = 0,40 = 40\%.
b) Incorreta. Hugo usou efetivamente \frac{60}{150} = 0,40 = 40\%.
c) Incorreta. Hugo usou efetivamente \frac{60}{150} = 0,40 = 40\%.
d) Incorreta. Hugo usou efetivamente \frac{60}{150} = 0,40 = 40\%.}

\num{3} Kríssia quer comprar à vista uma mochila que custa R\$ 560,00. 
Para pagamento à vista, a loja oferece um desconto de 5\%. Além disso, 
a última peça da loja tembém tem desconto de 20\%.

Ao chegar ao caixa, Kríssia foi surpreendida, pois o desconto de
pagamento à vista foi dado sobre o preço com o desconto de
última peça. Qual foi o valor pago?

\begin{escolha}

\item R\$ 532
\item R\$ 425,60
\item R\$ 420,00
\item R\$ 404,32

\end{escolha}

\coment{SAEB: Resolver problemas que envolvam porcentagens, incluindo os
que lidam com acréscimos e decréscimos simples, aplicação de
percentuais sucessivos e determinação das taxas percentuais.
BNCC: EF09MA05 -- Resolver e elaborar problemas que envolvam porcentagens, com a ideia de aplicação de percentuais sucessivos e a determinação das taxas percentuais, preferencialmente com o uso de tecnologias digitais, no contexto da educação financeira.

a) Incorreta. O valor pago foi de R\$ 425,60.
b) Correta. Houve dois descontos sucessivos. O primeiro foi de 20\%; o outro,
de 5\%. (1 -- 0,2) x (1 -- 0,05) x R\$ 560 = 0,8 x 0,95 x R\$ 560 = R\$ 425,60.
c) Incorreta. O valor pago foi de R\$ 425,60.
d) Incorreta. O valor pago foi de R\$ 425,60.}

\pagestyle{mat}
\chapter{Equação do 1º grau}
\markboth{Módulo 5}{}

\colorsec{Habilidades do SAEB}

\begin{itemize}

  \item Resolver uma equação polinomial de 1o grau.
  \item Inferir uma equação, inequação polinomial de 1o grau ou um sistema de
equações de 1o grau com duas incógnitas que modela um problema.
  \item Associar uma equação polinomial de 1o grau com duas variáveis a uma
reta no plano cartesiano.
  \item Resolver problemas que possam ser representados por sistema de
equações de 1o grau com duas incógnitas. 

\end{itemize} 

\conteudo{\textbf{Equação do primeiro grau} é toda expressão algébrica que
possui incógnita com grau 1. É uma sentença matemática que pode ser
escrita, do tipo ax + b = 0, em que a e b são números reais, e a é
diferente de 0.

O objetivo de escrever uma equação do 1º grau é encontrar qual é o valor
da incógnita que satisfaz a equação. Esse valor é conhecido como 
\textbf{solução} ou \textbf{raiz} da equação.

As incógnitas podem, na definição da equação, ser restritas a um
conjunto numérico cuja raiz seja um elemento. Caso não pertença ao
conjunto universo proposto, dizemos que o conjunto solução é vazio.

O que define o grau de uma equação é o expoente da incógnita. Sendo
assim, quando o expoente da incógnita possui grau 1, temos uma equação
do 1º grau. Observe o exemplo a seguir. 

\begin{itemize}
  
  \item 3x + 5 = 29 (equação do 1º grau com uma incógnita, x)
  
  \item y + 5 = - 3y (equação do 1º grau com uma incógnita, y)
  
  \item 2x -- 3y + 5 = 0 (equação do 1º grau com duas incógnitas, x e y)

\end{itemize}

O maior desafio é ``transformar'' os problemas em uma equação e
resolvê-la. Seguem algumas dicas para resolver problemas
matemáticos em geral:

\begin{itemize}
  \item Leia o problema cuidadosamente: certifique-se de que entende o que
o problema está pedindo. Leia o problema mais de uma vez, se necessário;

  \item Identifique as informações importantes: sublinhe ou destaque as
informações que são relevantes para a resolução do problema. Selecione com
cuidado o que você está tentando encontrar e quais dados você tem;

  \item Desenhe um diagrama: um diagrama pode ajudar a visualizar o
problema e ajudar na solução. Isso pode ser especialmente útil em
problemas geométricos;

  \item Use palavras-chave: muitas vezes as palavras usadas em um
problema podem dar uma indicação de que tipo de operação matemática deve
ser usada. Por exemplo, ``mais'' pode indicar adição, ``menos'' pode
indicar subtração, ``vezes'' pode indicar multiplicação e
``dividido por'' pode indicar divisão;

  \item Escreva equações: transforme as informações importantes do
problema em equações matemáticas. Use variáveis para representar
quantidades desconhecidas;

  \item Simplifique: tente simplificar as equações matemáticas tanto
quanto possível. Isso pode ajudar a reduzir a complexidade do problema;

  \item Resolva a equação: use as propriedades matemáticas para resolver
as equações. Lembre-se de seguir as regras corretas de operações
matemáticas;

  \item Verifique sua resposta: certifique-se de que sua resposta faz
sentido no contexto do problema. Verifique se você respondeu a todas as
partes do problema e se a resposta está em um formato adequado.
\end{itemize}
}

\colorsec{Atividades}

\num{1} Com o objetivo de juntar dinheiro para a formatura, Jorge começou a
produzir doces para revender. Cada receita é composta de \frac{4}{5} kg de 
castanha e \frac{1}{5} kg de açúcar.

Quando Jorge começou a produção, o quilo de castanha custava R\$ 20,00; o do
açúcar, R\$ 4,00. Recentemente, o quilo do açúcar teve aumento e passou a
custar R\$ 4,40. Para manter o custo original com a produção de uma receita, 
Jorge terá que negociar um desconto com o fornecedor de castanha.

Nas condições estabelecidas, o novo valor do quilo de castanha deverá ser
reduzido para qual valor?

%deixar box médio para resolução
\voment{Primeiramente calcula-se o custo da receita inicial:

\frac{4}{5} x 20 + \frac{1}{5} x 4 = 16 + 0,8 = R\$ 16,80

Agora vamos calcular o valor da castanha para manter o custo de R\$
16,80.

\frac{4}{5} x X + \frac{1}{5} x 4,4 = R\$ 16,80 \rightarrow
\frac{4}{5} x X + 0,88 = R\$ 16,80 \rightarrow R\$ 19,90.

Portanto, o valor do quilo da castanha deverá ser reduzido para R\$ 19,90.

\num{2} Um condomínio grande de uma cidade tem três entradas de água. A
vazão de cada entrada é apresentada na tabela a seguir.

\begin{table}[]
\begin{tabular}{|cc}
\hline
\textbf{Entrada de água} & \multicolumn{1}{c|}{\textbf{Vazão (litro/minuto)}} \\ \hline
\textbf{A} & 50 \\ \hline
\textbf{B} & 60 \\ \hline
\textbf{C} & 90 \\ \hline
\textbf{Vazão total} & 200 \\ \hline
\end{tabular}
\end{table}

Em função do aumento do número de moradores e considerando o histórico
do consumo de água, o síndico do condomínio julgou prudente aumentar em
100\% a vazão total de água nas entradas. Para esse propósito,
as vazões das entradas A e B foram aumentadas ao máximo, para 80 L/min e
100 L/min, respectivamente.

Para que a vazão de água das três entradas juntas seja aumentada em
100\%, em quantos litros por minuto deve ser aumentada a vazão da entrada C?

%deixar box grande para resolução

\coment{A vazão total era de 200 litros por minuto e precisa dobrar. 
Como as entradas A e B passaram a ser de 80 litros por minuto e 100 litros por
minuto, respectivamente, temos:

400 = 80 + 100 + x

220 = x

A entrada C era de 90 L/ min , então 220 -- 90 = 130 L/min, ou seja, um
aumento de 130 L/min.}

\num{3} Um vendedor vendeu cadeiras a R\$ 50,00 cada e mesas a R\$ 120,00
cada. No total, o total arrecadado com a venda foi de R\$ 4.800,00. O número 
de cadeiras vendidas é igual a quatro vezes o número de mesas vendidas.

Chamando-se o número de cadeiras vendidas de x e o número de mesas
vendidas de y, crie o sistema que representa, em linguagem
matemática, essa situação e resolva-o.

%deixar box médio para resolução

\coment{

\left\{\begin{matrix}
x = 4y &  & \\ 
50x + 120y = 4800 &  & 
\end{matrix}\right.

50 x 4y + 120y = 4800 \rightarrow y = 15 \therefore x = 60}


\num{4} Dois lutadores MMA de categorias diferentes foram desafiados para
disputar o cinturão de uma categoria intermediária às suas atuais. Um
deles tinha 91 kg e o outro tinha 77 kg. Eles foram submetidos a uma
intensa jornada de treinos, em que, a cada semana que passava, o primeiro
perdia 1,2 kg, enquanto o outro ganhava 0,6 kg. Logo após esse período de
preparação, a diferença de massa entre os dois atletas era de apenas 4 kg.

Quantas semanas, no mínimo, teve esse período de preparação dos
lutadores?

%deixar box médio para resolução
\coment{Lutador A: 91 -- 1,2x
Lutador B: 77 + 0,6x

Lutador A -- Lutador B = 4
(91 -- 1,2x) -- (77 + 0,6x) = 4
91 -- 1,2x -- 77 -- 0,6x = 4
-1,8x = 4 + 77 -- 91
-1,8x = -10
x = 5,56
Passaram-se 5 semanas e 4 dias aproximadamente.}

\num{5} Marcos pede aos seus pais que tripliquem o valor que recebe de mesada.
Eles aceitam, mas impõem uma condição: reduzir 5 reais do novo valor. Depois 
dessa negociação, fica acordado que Marcos receberá R\$ 60,00. 
Formule a equação que expressa essa situação.

%deixar box pequeno para resolução
\coment{Considerando x como valor inicial da mesada de Marcos, 3x corresponde
ao total que ele pretende receber. Reduzindo-se os R\$ 5,00, o total será
igual a 60, da seguinte maneira: 3x -- 5 = 60.}

\num{6} Paulo é garçom de um badalado de um restaurante. Ele recebe, por mês,
R\$ 650,00 mais R\$ 20,00 por hora extra que trabalha. Considerando que,
neste mês, ele recebeu um total de R\$ 3150,00, quantas horas extras ele 
trabalhou?

%Tirei a imagem deste exercício

\coment{650 + 20x = 3150
20x = 2500
x = 125
Paulo trabalhou 125 horas extras.}

\num{7} O reservatório de uma chácara estava cheio de água. O proprietário 
da chácara usou \frac{2}{3} desse conteúdo para encher a piscina e, em seguida,
adicionou 3000 litros de água ao reservatório. Com isso, o conteúdo do
reservatório passou a ocupar a metade de sua capacidade inicial. Qual a
capacidade total do reservatório?

%deixar box médio para resolução

\coment{ 
\frac{2}{3}x + 3000 = \frac{x}{2}
\frac{x}{6} = 3000
x = 18.000
}

\num{8} Carol e Alexandre têm, juntos, R\$ 1000. O dobro do
valor de Alexandre corresponde ao triplo do valor de Carol. Qual o valor
que cada um possui?

%deixar box pequeno para resolução

\coment{
  \left\{\begin{matrix}
C + A = 1000 &  & \\ 
2A = 3C &  & 
\end{matrix}\right.

A = 1000 -- C
2 x (1000 -- C) = 3C
2000 - 2C = 3C
2000 = 5C
C = 400
A = 600
}

\num{9} Um estacionamento cobra R\$ 8,00 pelas primeiras duas horas e mais
R\$ 1,50 pelas horas subsequentes. Considerando que foram gastos R\$
14,00, qual foi o tempo de permanência?

%deixar box pequeno para resolução

\coment{ 
1,5x + 8 = 14
1,5x = 6
x = 4 horas
}

\num{10} Nico viaja 350 quilômetros para ir de carro de sua casa à cidade
onde moram seus avós. Em uma dessas viagens, após alguns quilômetros,
ele parou para almoçar. A seguir, percorreu o triplo da quantidade de
quilômetros que havia percorrido antes de parar.

%deixar box pequeno para resolução

\coment{Consideremo o 1º trecho como x. O 2º trecho corresponde ao triplo
de x, ou seja, 3x. Somando os dois, encontra-se o valor de 350.

x + 3x = 350
4x = 350
x = 87,5 km até a parada. 
Depois, Nico percorreu 350 -- 87,5 = 262,5 km.}

\colorsec{Treino}

\num{1} Aos domingos é comum em algumas cidades termos ciclovias, onde as
famílias inteiras participam de passeios ciclísticos. Uma locadora de bicicleta
cobra R\$ 16,00 por hora de aluguel de uma bicicleta. Além disso,
também cobra uma taxa fixa de manutenção de R\$ 10,00. Carlos alugou uma
bicicleta por 5 horas. Qual o custo ele teve?

\begin{escolha}

  \item R\$ 80,00

  \item R\$ 85,00

  \item R\$ 90,00

  \item R\$ 130,00

\end{escolha}

\coment{SAEB: Resolver uma equação polinomial de 1o grau.

a) Incorreta. O custo total de Carlos foi de R\$ 90,00.
b) Incorreta. O custo total de Carlos foi de R\$ 90,00.
c) Correta. Como são R\$ 16,00 por hora mais R\$ 10,00 fixos de taxa, 
podemos escrever a seguinte expressão: C = 10 +16x, na qual, ao substituir 
o x por 5, teremos C = 10 + 16 x 5 = 90.
d) Incorreta. O custo total de Carlos foi de R\$ 90,00.}

\num{2} Um estacionamento tem 50 veículos entre carros e motos. Sabendo que o
total de pneus do estacionamento é igual a 160, qual o número de
carros?

\begin{escolha}

  \item 20

  \item 30

  \item 40

  \item 50

\end{enumerate}

\coment{SAEB: Resolver problemas que possam ser representados por sistema de equações de 1o grau com duas incógnitas.

a) Incorreta. Existem 40 carros no estacionamento.
b) Correta. Observe:

\begin{displaymath}
\left\{\begin{matrix}
C + M = 50 &  & \\ 
4C + 2M = 160 &  & 
\end{matrix}\right.

\sim 

\left\{\begin{matrix}
-2C - 2M = -100 &  & \\ 
4C + 2M = 160 &  & 
\end{matrix}\right.

\rightarrow

2C =60 

\rightarrow

C = 30
\end{displaymath}

Há 30 carros no estacionamento.

c) Incorreta. Existem 40 carros no estacionamento. 
d) Incorreta. Existem 40 carros no estacionamento. 
}

\num{3} A soma de três números inteiros consecutivos é 33. Qual é o produto
entre esses três números?

\begin{escolha}
  \item 110

  \item 120

  \item 1320

  \item 4200
\end{escolha}

\coment{SAEB: Resolver uma equação polinomial de 1o grau

a) Incorreta. O produto desses três números é 1320.
b) Incorreta. O produto desses três números é 1320.
c) Correta. Observe a resolução a seguir:

x + (x+1) + (x+2) = 33
3x = 30
x = 10
x+1 = 11
x+2 = 12
10 x 11 x 12 = 1320

d) Incorreta. O produto desses três números é 1320.} 

\pagestyle{mat}
\chapter{Sequências e Expressões Algébricas}
\markboth{Módulo 6}{}

\colorsec{Habilidades do SAEB}

\begin{itemize}

  \item Identificar uma representação algébrica para o padrão ou a
regularidade de uma sequência de números racionais ou representar
algebricamente o padrão ou a regularidade de uma sequência de
números racionais. 
  \item Identificar representações algébricas equivalentes. 
  \item Resolver problemas que envolvam cálculo do valor numérico de
expressões algébricas. 

\end{itemize} 

\conteudo{\textbf{Sequências numéricas} são conjuntos ordenados de números
que seguem um padrão ou uma regra específica. Essas sequências podem ser 
finitas ou infinitas e podem ser classificadas de diferentes maneiras,
dependendo de suas características.

Algumas das sequências numéricas mais comuns incluem as \textbf{sequências
aritméticas}, que seguem uma regra de adição ou subtração constante entre
seus termos sucessivos, e as \textbf{sequências geométricas}, que seguem uma
regra de multiplicação ou divisão constante entre seus termos
sucessivos.

Outros tipos de sequências numéricas incluem as \textbf{sequências harmônicas},
as \textbf{sequências de Fibonacci} e as \textbf{sequências de números 
primos}. Cada uma dessas sequências tem propriedades e características 
únicas que podem ser estudadas e exploradas por matemáticos e estudantes.

O estudo de sequências numéricas é importante em muitas áreas da
matemática, bem como em outras disciplinas, como física, engenharia e
ciência da computação. As sequências numéricas são usadas para modelar e
descrever fenômenos naturais, para resolver problemas matemáticos e para
desenvolver algoritmos eficientes para uma variedade de aplicativos.

\textbf{Lei de formação}

Há sequências de vários elementos, como meses, nomes, dias da
semana, entre outros. Quando envolve números, a sequência numérica pode
ter uma ``regra'' específica. Podemos formar a sequência de números
pares, números ímpares, números primos, múltiplos de 8 etc.

A sequência numérica pode ser representada por meio de uma \textbf{lei de
formação}. Isso nada mais é que a lista dos elementos da sequência
numérica que seguem a mesma regra. Observe o exemplo a seguir.

Na sequência $a\textsubscript{n}= 3n + 1$, calcula-se o 1º termo com n = 1,
o 2º termo com n=2 e assim sucessivamente. A sequência será 4, 7, 10, 13, 16, 19, \ldots{}

\textbf{Classificação da sequência numérica}

A sequencia numérica pode ser classificada como: crescente, decrescente
ou constante.}

\colorsec{Atividades}

\num{1} Matheus montou figuras com palitos de fósforo. Na 1ª figura, montou
um triângulo e, nas etapas seguintes, foi acrescentando outros triângulos,
conforme se pode observar na imagem a seguir.

\includegraphics[width=4.44792in,height=1.53053in]{./_SAEB_9_MAT/media/image74.wmf}
%Montar figura semelhante, lembrando que a quantidade é importante

Qual a quantidade de palitos de fósforo necessários e suficientes para a
construção da 6ª figura?

%deixar box grande para resolução

\coment{A 1ª figura é composta por 3 palitos; a 2ª figura, por 9; a 3ª,
por 15. Observa-se, portanto, que sempre são acrescentados 6 palitos, 
de modo que uma das expressões que representa a quantidade de palitos é 
$y = 6n -- 3$, onde n é a posição da figura. Para n = 6 teremos: 
$y = 6 x 6 -- 3 = 33$ palitos.}

\num{2} Em um grupo de crianças, 210 bombons foram distribuídos
para cada uma, na forma de uma sequência crescente, da criança de menor
estatura para a de maior estatura. Ao colocarmos as crianças nessa
ordem, percebeu-se que a segunda criança ganhou 5 bombons, a quinta
ganhou 11 e sétimo 15. Complete a tabela abaixo para encontrar a
quantidade de bombons de cada criança, desde que sempre seja regular.

% Please add the following required packages to your document preamble:
% \usepackage[table,xcdraw]{xcolor}
% If you use beamer only pass "xcolor=table" option, i.e. \documentclass[xcolor=table]{beamer}
\begin{table}[]
\begin{tabular}{|l|l|l|l|l|l|l|l|l|l|}
\hline
\rowcolor[HTML]{CBCEFB} 
1ª & 2ª & 3ª & 4ª & 5ª & 6ª & 7ª & 8ª & 9ª & 10ª \\ \hline
 & 5 &  &  & 11 &  & 15 &  &  &  \\ \hline
\end{tabular}
\end{table}

\coment{A regularidade deve despertar a percepção de que, entre os 2 
primeiros números já preenchidos, temos a diferença de 6, mas, ao olhar 
o 5º e 7º, essa diferença se reduz para 4, ou seja, a mudança de um para
outro será de apenas 2 unidades. Não há problemas se os alunos usarem o
método de tentativa e erro para resolver a questão.

% Please add the following required packages to your document preamble:
% \usepackage[table,xcdraw]{xcolor}
% If you use beamer only pass "xcolor=table" option, i.e. \documentclass[xcolor=table]{beamer}
\begin{table}[]
\begin{tabular}{|l|l|l|l|l|l|l|l|l|l|}
\hline
\rowcolor[HTML]{CBCEFB} 
1ª & 2ª & 3ª & 4ª & 5ª & 6ª & 7ª & 8ª & 9ª & 10ª \\ \hline
{\color[HTML]{FE0000} 3} & 5 & {\color[HTML]{FE0000} 7} & {\color[HTML]{FE0000} 9} & 11 & {\color[HTML]{FE0000} 13} & 15 & {\color[HTML]{FE0000} 17} & {\color[HTML]{FE0000} 19} & {\color[HTML]{FE0000} 21} \\ \hline
\end{tabular}
\end{table}

\num{3} Observe a quantidade de lugares em cada configuração:

%Montar uma imagem como a de baixo. É importante ter a mesma quantidade de cadeiras e não pode se redonda, precisam ser retangulares

\includegraphics[width=5.72917in,height=1.3in]{./_SAEB_9_MAT/media/image76.wmf}

Se mantivermos o padrão, quantos lugares teremos no total se
acrescentarmos as configurações 4 e 5?

%deixar box grande para resolução
\coment{Na Configuração 1, há 4 lugares; na 2, há 6 lugares que, somados
aos 4 lugares anteriores, resultam em 10; na 3, há 8 lugares que, somados
aos 10 anteriores, completam 18 lugares.

Percebe-se que a próxima configuração a partir da Configuração 3 sempre 
terá duas cadeiras a mais, da seguinte maneira: 4, 6, 8, 10, 12. Essas 
são as quantidades de cadeiras em cada configuração. Somando teremos:
4 + 6 + 8 + 10 + 12 = 40 lugares.}

\num{4} Uma fábrica trabalha com vários modelos e tamanhos de mesas. As mesas
são todas acompanhadas com uma certa quantidade de poltronas a depender
do tamanho da mesa.

%Montar uma tabela como a de baixo. É importante ter a mesma quantidade
de cadeiras.

\includegraphics[width=4.28125in,height=1.50833in]{./_SAEB_9_MAT/media/image77.wmf}

O primeiro modelo vem acompanhado de 3 poltronas; o segundo, de 6; o 
terceiro, de 9 poltronas -- e assim sucessivamente. Ao adquirir uma
unidade de cada um dos 10 primeiros modelos de mesa circular, quantas
poltronas estarão disponíveis?

%deixar box médio para resolução
\coment{A sequência de poltronas pode ser observada da seguinte maneira:
3, 6, 9, 12, 15, 18, 21, 24, 27, 30. Somando, encontraremos: 3 + 6 + 9 +
12 + 15 + 18 + 21 + 24 + 27 + 30 = 165.}

\num{5} Felizmente os animais não são mais usados nos espetáculos de circo,
que seguem interessando ao público com outras atrações. Uma delas é a 
\textbf{pirâmide humana}, em que uma pessoa no topo é sustentada por duas
outras, que são sustentadas por mais três, e assim sucessivamente. Quantas
pessoas são necessárias para formar uma pirâmide com 5 linhas de pessoas,
da base ao topo?

\begin{figure}
\centering
\includegraphics[width=1.71354in,height=1.56302in]{./_SAEB_9_MAT/media/image82.jpeg}
\caption{Pirâmide criadas por pessoas desenho - Royalty-free Pirâmide
Humana arte vetorial}
\end{figure}

\href{https://www.istockphoto.com/pt/vetorial/pir\%C3\%A2mide-criadas-por-pessoas-desenho-gm499201238-79908399?phrase=pir\%C3\%A2mide\%20humana}{Pirâmide
Criadas Por Pessoas Desenho - Arte vetorial de stock e mais imagens de
Pirâmide Humana - Pirâmide Humana, Esboço, Pirâmide - Estrutura
construída - iStock (istockphoto.com)}

%deixar box pequeno para resolução

\coment{Considerando as informações do enunciado e a imagem apresentada, 
quando são usadas 4 linhas, são necessárias 10 pessoas para formar a 
pirâmide. Na próxima linha, haverá mais 5 pessoas, completando um total
de 15.}

\num{6} Qual o número de cadeiras necessárias ao utilizar 10 mesas na
configuração da imagem abaixo:

\includegraphics[width=4.6875in,height=0.8in]{./_SAEB_9_MAT/media/image83.wmf}

%Montar imagem semelhante, número de cadeiras é importante.

%deixar box pequeno para resolução

\coment{Considerando a imagem apresentada, observa-se uma sequência numérica
de cadeiras organizada da seguinte forma: 1 mesa, 4 cadeiras; 2 mesas, 6
cadeiras; 3 mesas, 8 cadeiras.
É possível estabelecer uma expressão de quantidade de cadeiras de acordo
com a quantidade de mesas, por exemplo: $y = 2 x n + 2$, com n representando 
a quantidade de mesas. Portanto: $y = 2 · 10 + 2 = 22$ cadeiras.}

\num{7} Escreva mais 3 números da sequência 18, 30, 42, 54.

\coment{Para adicionar mais três números à sequência, basta perceber que
a diferença entre 18, 30, 42, 54 é sempre 12 em 12. Portanto, os três
números solicitados são 66, 78, 90.}

\num{8} Observe a imagem a seguir. 

%Montar imagem semelhante, número de cadeiras é importante.

\includegraphics[width=5.68333in,height=1.33333in]{./_SAEB_9_MAT/media/image84.wmf}

Complete a tabela a seguir.

% Please add the following required packages to your document preamble:
% \usepackage[table,xcdraw]{xcolor}
% If you use beamer only pass "xcolor=table" option, i.e. \documentclass[xcolor=table]{beamer}
\begin{table}[]
\begin{tabular}{|
>{\columncolor[HTML]{CBCEFB}}c |c|c|}
\hline
\textbf{Configuração} & \cellcolor[HTML]{CBCEFB}\textbf{Número de mesas} & \cellcolor[HTML]{CBCEFB}\textbf{Número de lugares} \\ \hline
\textbf{1} &  &  \\ \hline
\textbf{2} &  &  \\ \hline
\textbf{3} &  &  \\ \hline
\textbf{4} &  &  \\ \hline
\textbf{5} &  &  \\ \hline
\textbf{n} &  &  \\ \hline
\end{tabular}
\end{table}

\coment{
% Please add the following required packages to your document preamble:
% \usepackage[table,xcdraw]{xcolor}
% If you use beamer only pass "xcolor=table" option, i.e. \documentclass[xcolor=table]{beamer}
\begin{table}[]
\begin{tabular}{|
>{\columncolor[HTML]{CBCEFB}}c |c|c|}
\hline
\textbf{Configuração} & \cellcolor[HTML]{CBCEFB}\textbf{Número de mesas} & \cellcolor[HTML]{CBCEFB}\textbf{Número de lugares} \\ \hline
\textbf{1} & 1 & 4 \\ \hline
\textbf{2} & 2 & 6 \\ \hline
\textbf{3} & 3 & 8 \\ \hline
\textbf{4} & 4 & 10 \\ \hline
\textbf{5} & 5 & 12 \\ \hline
\textbf{n} & n & 2n + 2 \\ \hline
\end{tabular}
\end{table}
}

\num{9} Qual é a diferença entre o décimo e o quinto termos da sequência 
a\textsubscript{n} = 3n + 2?

%deixar box pequeno para resolução
\coment{$n = 10 32$
$n = 5 17$
$32 -- 17 = 15$}

\num{10} Observe a sequência: 0, 1, 4, 9, 16, 25, \ldots{}
Qual expressão pode gerar essa sequência?

%deixar box pequeno para resolução
\coment{A sequência é composta pelos quadrados dos números naturais:
$1\textsuperscript{2}$, $2\textsuperscript{2}$, 
$3\textsuperscript{2}$ \ldots{} y = $n\textsuperscript{2}$}

\colorsec{Treino}

\num{1} Renato colecionou figurinhas da Copa do Mundo. Se ele ganhar mais 8
figurinhas, a quantidade ficará igual ao dobro da quantidade de figurinhas 
que Rodrigo tem menos 12. Se Rodrigo possui 20 figurinhas, então o número de
figurinhas que Renato tem é igual a:

\begin{escolha}

  \item 40

  \item 44

  \item 52

  \item 60

\end{escolha}

\coment{SAEB: Resolver problemas que envolvam cálculo do valor numérico de expressões algébricas.

a) Incorreta. Renato tem 44 figurinhas.
b) Correta. Sendo x a quantidade de figurinhas de Renata:
$x + 8 = 2 x 20 + 12$
$x + 8 = 40 + 12$
$x + 8 = 52$
$x = 52 -- 8$
$x = 44$
c) Incorreta. Renato tem 44 figurinhas.
d) Incorreta. Renato tem 44 figurinhas.}

\num{2} Caio e Bruna têm, juntos, R\$ 21.000,00 para dar de entrada em um
carro. Caio tem três quartos da quantia de Bruna. Quanto dinheiro Bruna tem?

\begin{escolha}
  \item R\$ 14.200,00

  \item R\$ 13.500,00

  \item R\$ 9.000,00

  \item R\$ 12.000,00
\end{escolha}

\coment{SAEB: Resolver problemas que envolvam cálculo do valor numérico de
expressões algébricas. 

a) Incorreta. Bruna tem R\$ 12.000,00. 
b) Incorreta. Bruna tem R\$ 12.000,00.
c) Incorreta. Bruna tem R\$ 12.000,00.
d) Correta. Seja x a quantidade de dinheiro da Bruna, e \frac{3}{4}x a
quantidade de dinheiro de Caio.
$x + \frac{3}{4}x = 21000$
$\frac{4 + 3}{4}x = 21000$
$7x = 84000$
$x = 12000$.}

\num{3} Um vendedor recebe pagamento composto por uma parte fixa de R\$
850,00 mais um bônus de R\$ 60,00 a cada peça vendida. Se em um
determinado mês ele recebeu o salário de R\$ 1870,00, a
quantidade de produtos vendidos foi igual a:

\begin{escolha}
  \item 15

  \item 16

  \item 17

  \item 18
\end{escolha}

\coment{SAEB: Resolver problemas que envolvam cálculo do valor numérico de expressões algébricas.

a) Incorreta. Foram vendidos 17 produtos.
b) Incorreta. Foram vendidos 17 produtos.
c) Correta. Montando a equação, sabemos que:
%Paulo: a linha abaixo é um teste que fiz com "ambiente matemático", para verificarmos se funciona bem. Em vez de colocar cada passagem em uma linha, usei a seta
\begin{displaymath} 60x + 850 = 1870 \rightarrow 60x = 1870 - 850 \rightarrow
60x = 1020 \rightarrow x = 17
\end{displaymath}
d) Incorreta. Foram vendidos 17 produtos.}

\pagestyle{mat}
\chapter{Equações do 2º grau}
\markboth{Módulo 7}{}

\colorsec{Habilidades do SAEB}

\begin{itemize}

  \item Inferir uma equação polinomial de 2o grau que modela um problema.
  \item Resolver uma equação polinomial de 2o grau.
  \item Resolver problemas que possam ser representados por equações
polinomiais de 2o grau.   

\end{itemize} 

\colorsec{Habilidade da BNCC}

\begin{itemize}
  \item EF09MA09.
\end{itemize}

\conteudo{

Chamamos de equação do segundo grau as equações do tipo ax² + bx + c = 0
com a, b e c ∈ R, em que \textbf{a} é diferente de zero.

As equações do 2º grau podem ter duas raízes reais iguais, duas raízes
reais distintas ou nenhuma raiz real. O que determina como as raízes são
é o valor encontrado para o discriminante \mathrm{\Delta}.

A maneira mais comum de determinar as raízes da equação do 2º grau é
através da famosa fórmula de Bhaskara:

$\mathrm{\Delta} = b^{2} - 4 x a x c$

$x = \frac{- b \pm \sqrt{\mathrm{\Delta}}}{2 x a}$

Se $\mathrm{\Delta} \textgreater{}0$, então a equação admite duas raízes distintas em
R;

Se $\mathrm{\Delta} = 0$, então a equação admite duas raízes iguais em R;

Se $\mathrm{\Delta} \textless{}0$, ou seja, se $\mathrm{\Delta}$ for 
negativo, a equação não admite solução em R.

As equações do 2º grau podem ser completas ou incompletas. Elas são
incompletas quando o valor referente a \textbf{b} ou \textbf{c} for
ausente, ou seja, igual a zero. Lembrando que o valor de \textbf{a} tem
que ser diferente de zero.

Os parâmetros da equação são:

\begin{enumerate}
  \item \textbf{a} -- coeficiente principal

  \item \textbf{b} -- coeficiente secundário

  \item \textbf{c} -- termo independente
\end{enumerate}

Observe o exemplo a seguir.

\begin{itemize}
\item
  2x\textsuperscript{2} + 6x + 3 = 0 (essa é uma equação do segundo grau)
\item
  Chamamos \textbf{a}, \textbf{b} e \textbf{c} de coeficientes; \textbf{a}
  é sempre coeficiente de \textbf{x²}; \textbf{b} é sempre coeficiente de 
  \textbf{x}; e \textbf{c} é sempre coeficiente do termo independente.
\end{itemize}

Dessa forma:

\begin{enumerate}
\item
  $3x\textsuperscript{2} + 4x + 1 = 0: é uma equação do segundo grau, 
  com a = 3, b = 4, c = 1;
\item
  x² -- x -- 1 = 0: é uma equação incompleta com grau 2, com a = 1, 
  b= --1, c = --1;
\item
  x\textsuperscript{2} -- 5x = 0: também é uma equação incompleta de grau 2, com a = 9, b = --5, c = 0;
\item
  5x\textsuperscript{2} -- 4 = 0: equação do segundo grau, com a = 5, 
  b = 0, c = --4$.
\end{enumerate}
}

\colorsec{Atividades}

\num{1} 
\begin{quote}
O IMC (Índice de Massa Corporal) é um padrão internacional de cálculo
da obesidade de um indivíduo adotado pela OMS (Organização Mundial da
Saúde). O método, desenvolvido pelo belga Lambert Quételet no fim do
século XIX, é a forma mais fácil de saber se uma pessoa está com o peso
ideal ou não.

A altura (calculada em metros) e o peso (calculado em quilogramas) do
indivíduo são os dois fatores levados em conta no cálculo do IMC. Para
calcularmos o índice, basta dividirmos o peso de uma pessoa pela sua
altura ao quadrado.
\end{quote}

\fonte{Mundo Educação. IMC. Disponível em: https://mundoeducacao.uol.com.br/saude-bem-estar/imc.htm. Acesso em: 15 mai. 2023}

A fórmula do \textbf{IMC} é a mesma para todas as pessoas e pode ser
escrita de seguinte maneira:

$h\textsuperscript{2} x IMC - P = 0$

sendo \textbf{h} a altura da pessoa em metros (m) e \textbf{P} o seu peso 
em quilogramas (kg).

Se uma pessoa possui \textbf{IMC} igual a 40kg/m\textsuperscript{2} e está
com peso igual a 120 kg, qual é sua altura?

%deixar box pequeno para resolução
\coment{Substituindo os valores na fórmula dada, chegamos a:

$h\textsuperscript{2} x 40 - 120 = 0 \rightarrow

h = \sqrt{\frac{120}{40}} = \sqrt{3}

\therefore h \cong 1,70m$} 

\num{2} Calcule os valores das raízes das equações de 2º grau:

\begin{escolha}

  \item $x\textsuperscript{2} + 6x + 8 = 0

  \item x\textsuperscript{2} - 5x - 24 = 0$

\end{escolha}

%deixar box médio para resolução

\coment{
a) Por fatoração podemos escrever a equação $x\textsuperscript{2} + 6x + 8 = (x+4) x (x+2) = 0$.
Com isso, temos como raízes $- 4 e - 2. S={-4, -2}$.

b) Por fatoração podemos escrever a equação $x\textsuperscript{2} - 5x - 24 = 
(x+8)x (x-3) = 0$.
Com isso, temos como raízes $- 8 e 3. S = {- 8, 3}$.} 

\num{3} Ao quadrado de um número x você adiciona 7 e obtém sete vezes o
número x, menos 3. Quais são as raízes dessa equação?

%deixar box pequeno para resolução

\coment{$x\textsuperscript{2} + 7 = 7x - 3 \rightarrow

x\textsuperscript{2} -7x + 10 = 0 \rightarrow

(x - 5)·(x - 2) = 0 \rightarrow

x = 5 ou x = 2$}

\num{4} Suponha que a área de 4.225 km\textsuperscript{2} é delimitada por um
retângulo. Se o comprimento da área excede em sua largura x, formule uma 
equação que permite determinar essa largura x.

\coment{Largura x Comprimento

$x \cdot (x + 100) = 4225 \rightarrow

x\textsuperscript{2} + 100x - 4225 = 0$
}

\num{5} Uma escola pretende colocar o piso das salas de aula como na
figura:

\includegraphics[width=1.70833in,height=1.75833in]{./_SAEB_9_MAT/media/image104.wmf}

Cada piso é formado por quatro retângulos iguais de lados 10 cm e (x +
10) cm respectivamente, e um quadrado de lado igual a x cm.

Sabendo-se que a área de cada piso equivale a 900 cm\textsuperscript{2},
qual é o valor de x, em centímetros?

\coment{$4 \cdot 10 \cdot (x+10) + x\textsuperscript{2} = 900 \rightarrow
40x + 400 + x\textsuperscript{2} = 900 \rightarrow
x\textsuperscript{2} + 40x - 500 = 0 \rightarrow
\mathrm{\Delta} = 40\textsuperscript{2} - 4 \cdot 1 \cdot -500 = 3600 \rightarrow
x = \frac{-40\pm\sqrt{3600}}{2\cdot1} \rightarrow$
x = -50 (não convém) ou x = 10}

\num{6} Um retângulo tem 204\textsuperscript{2}.

\includegraphics[width=2.175in,height=1.29167in]{./_SAEB_9_MAT/media/image107.wmf}

Para uma atividade escolar, precisa-se que o retângulo seja refeito com 3
cm a mais no comprimento e 2 cm a mais na largura e com isso a
superfície aumentou em 76 cm\textsuperscript{2}.

Nessas condições, quais são os dois valores possíveis do comprimento?

%deixar box grande para resolução

\coment{

Do enunciado, temos:
$
\left\{\begin{matrix}
x \cdot y =204 (i)
\\ (x+3) \cdot (y+2) = 204 + 76 (i)
\end{matrix}\right.
$

Da equação (ii),
$xy + 2x + 3y + 6 = 280$

Como $xy =204$,

$204 + 2x + 3y + 6 = 280 \rightarrow
2x + 3y = 70 \rightarrow
y = \frac{70-2x}{3}$

Substituindo $y = \frac{70-2x}{3}$ na equação (i),

$x \cdot \left ( \frac{70-2x}{3} \right ) = 204

x\textsuperscript{2} - 35x + 306 = 0$

Resolvendo a equação acima, obtemos x = 17 ou x = 18.}

\num{7} Um grupo de amigos decidiu alugar uma chácara em um feriado
prolongado para comemorar a formatura. O valor de R\$ 3600,00 seria igualmente
dividido por todos. Devido a alguns problemas financeiros, oito alunos
que estavam no grupo desistiram, e a parte que cada um do grupo deveria
pagar aumentou R\$ 75,00.

Quantos alunos faziam parte do grupo inicialmente?

%deixar box grande para resolução

\coment{Seja x o número de alunos e y o valor de cada aluno, temos 
duas situações:

$\left\{\begin{matrix}
\frac{3600}{x} = y
\\ \frac{3600}{x-8} = y + 75
\end{matrix}\right.$

Substituindo a primeira equação na segunda, temos:

$\frac{3600}{x-8} = \frac{3600}{x} + 75 \rightarrow
\frac{3600 \cdot x}{x \cdot (x-8)} = \frac{3600 \cdot (x - 8)}{x \cdot (x-8)} + \frac {75 \cdot x \cdot (x - 8)}{x \cdot (x-8)} \rightarrow
3600x - 3600x + 28800 - 75x\textsuperscript{2} + 600x = 0 \rightarrow
- 75x\textsuperscript{2} + 600x + 28800 = 0 (\div 75) \rightarrow
- x\textsuperscript{2} + 8x + 384 = 0$ 

Aplicando soma e produto temos:
$\left\{\begin{matrix}
x = -16
\\x = 24 
\end{matrix}\right.$

Logo, o total de alunos da turma é 24.}

\num{8} Augusto tem 6 anos e Gabriela tem 5. Daqui a quantos anos o produto
de suas idades será igual a 42?

%deixar box pequeno para resolução

\coment{$(x + 6) \cdot (x + 5) = 42
x\textsuperscript{2} + 11x - 12 = 0$
x = - 11 ou x = 1
O produto de suas idades será igual a 42 daqui a 1 ano.}

\num{9} Um café da manhã que seria dado de presente no valor de R\$ 360,00
deveria ser comprado por um grupo de amigos que contribuíram em partes
iguais. Como 4 deles desistiram, os outros precisaram aumentar a sua
participação em R\$ 15,00 cada um. Qual era a quantidade inicial de 
participantes?

%deixar box pequeno para resolução

\coment{Sendo x igual ao número de rapazes e y igual à quantia que cada um deve
disponibilizar inicialmente, pode-se escrever:

$ xy = 360 \rightarrow y = \frac{360}{x}$

Após a desistência de 4 rapazes, a quantia que cada um deve que
disponibilizar aumentou R\$ 15,00, ou seja:

$(x-4) \cdot (y + 15) = 360 \rightarrow xy - 4y + 15x - 60 = 360$

Sabendo o valor de xy e de y conforme a relação inicial, pode-se
substituir:

$ xy - 4y + 15x - 60 = 360 \\

360 - 4 \cdot \frac{360}{x} + 15x - 60 = 360 \\

- 4 \cdot \frac{360}{x} + 15x - 60 = 0 \\

- 1440 + 15x\textsuperscript{2} - 60x = 0 \\

15x\textsuperscript{2} - 60x - 1440 = 0  \\

x\textsuperscript{2} - 4x - 96 = 0  \\

\delta = (-4)\textsuperscript{2} - 4 \cdot 1 \cdot (-96) = 400 \\

x = \frac{4 \pm 20}{2}  \\

x_1 = 12 \\

x_2 = - 8 $

Como é impossível ter uma quantidade negativa de pessoas, conclui-se que
o número inicial de rapazes era 12.}

\num{10} Jandira usou a seguinte frase: ``Um número natural x cujo quadrado
aumentado do seu dobro é igual a 15''. Que número é este?

\coment{$x^2 + 2x = 15 \\
x² + 2x -- 15 = 0 $ \\
x = - 5 ou x = 3 
O número é 3.}

\colorsec{Treino}

\num{1} Em uma empresa, o custo de produção, em milhares de reais, de n peças
iguais é calculado pela expressão C(n) = n² -- n + 10.

Se o custo foi de 82 mil reais, então, o número de caçambas utilizadas
na produção foi

a) 6

b) 7

c) 8

d) 9

Professor

BNCC: EF09MA09 SAEB: 9A2.4

n² - n + 10 = 82

n² - n -72 = 0

n = 9 ou n = - 8 (Não convém)

Alternativa d

\num{2} Eloisa multiplicou a idade atual de seu filho pela idade que ele terá
daqui a 8 anos e obteve como resultado 20 anos.

Qual é a idade atual do filho de Eloisa?

a) 2 anos.

b) 5 anos.

c) 7 anos.

d) 9 anos.

Professor

BNCC: EF09MA09 SAEB: 9A2.4

x · (x + 8) = 20

x² + 8x -- 20 = 0

x = 2 ou x = - 10

Resposta 2 anos.

Alternativa a

\num{3} As idades de dois irmãos são as raízes da equação: x² -- 24x + 144 =
0. Com isso, podemos afirmar que:

a) Eles são gêmeos.

b) Um deles ainda não nasceu.

c) Os dois ainda não nasceram.

d) Um é mais velho do que o outro um ano.

Professor

BNCC: EF09MA09 SAEB: 9A2.4

(x -- 12)² = 0

x = 12

Única raiz, ou seja, são gêmeos.

\pagestyle{mat}
\chapter{Grandezas Direta e Inversamente Proporcionais}
\markboth{Módulo 8}{}

\colorsec{Habilidades do SAEB}

\begin{itemize}

  \item Resolver problemas que envolvam variação de proporcionalidade direta
ou inversa entre duas ou mais grandezas, inclusive escalas, divisões
proporcionais e taxa de variação.   

\end{itemize} 

\colorsec{Habilidades da BNCC}

\begin{itemize}
  \item EF09MA07, EF09MA08.
\end{itemize}

\conteudo{
    [Grandeza Diretamente Proporcional]

Dizemos que duas grandezas x e y são diretamente proporcionais quando a
razão (divisão) entre x e y sempre dá o mesmo resultado, ou seja, é
constante.

$$\frac{x}{y} = k$$

k é a constante de proporcionalidade. Em outras palavras, duas grandezas
são diretamente proporcionais quando uma aumenta e a outra também
aumenta na mesma proporção (quando uma triplica, a outra também
triplica, por exemplo); quando uma diminui, a outra também diminui na
mesma proporção (quando uma cai pela metade, a outra também tem o mesmo
comportamento).

Podemos pegar alguns exemplos do cotidiano, tais como a quantidade de
ovos de uma receita, para dobrar a receita devemos dobrar também a
quantidade de ingredientes. Uma pessoa que recebe apenas pela quantidade
de horas trabalhada ao triplicar a quantidade de horas triplicará o
valor recebido.

Exemplo:

Com a velocidade constante, veja a relação entre deslocamento tempo

  **Deslocamento (m)**   3   6   9   12   15
  ---------------------- --- --- --- ---- ----
  **Tempo (s)**          1   2   3   4    5

[Grandeza Inversamente Proporcional]{.underline}

Dizemos que duas grandezas x e y são inversamente proporcionais quando o
produto entre x e y é constante.

x ⋅ y = k

Este valor k é denominado constante de proporção.

Em outras palavras, duas grandezas são inversamente proporcionais quando
uma aumenta e a outra diminui na mesma proporção (uma dobra e a outra
cai pela metade, por exemplo).

Um exemplo de grandezas inversamente proporcionais é a velocidade e o
tempo, pois ao dobrar a velocidade você tem o tempo reduzido.

Exemplo:

Com o mesmo índice de rendimento de produtividade de cada máquina

  **Máquinas**       2    4    8    16
  ------------------ ---- ---- ---- ----
  **Tempo (dias)**   80   40   20   10
}

\colorsec{Atividades}

1) Uma máquina produz 5 peças a cada 40 segundos. A tabela a seguir
mostra a quantidade de peças que são produzidas em termo da quantidade
de segundos.

\begin{figure}
\centering
\includegraphics[width=3.41667in,height=0.55in]{./_SAEB_9_MAT/media/image132.png}
\caption{Uma imagem contendo Tabela Descrição gerada automaticamente}
\end{figure}

Montar imagem semelhante

Depois de 20 minutos, qual a quantidade de peças produzidas?

BNCC: EF09MA07 e EF09MA08 SAEB: 9A2.1

Em 20 minutos temos 20 · 60 = 1 200 segundos, ou seja, olhando a tabela
podemos perceber que se a cada 200 segundos a máquina produz 25 peças em
1 200 segundos produzirá 6 · 25 = 150 peças.

2) Uma obra será realizada em uma cidade do interior de São Paulo. A
empresa contratada fez uma planilha com a previsão de todos os gastos
com a execução dessa obra. Foi planejado executar a obra em 16 dias com
25 operários trabalhando 6 horas por dia. Porém, o engenheiro verificou
alguns itens não previstos no projeto fizeram com que a obra tivesse o
triplo da dificuldade inicialmente prevista..

Com o replanejamento viu a necessidade de dobrar o número de operários e
que trabalhassem 8 horas por dia.

Qual o prazo para concluir a obra?

Deixar 4 linhas

BNCC: EF09MA07 e EF09MA08 SAEB: 9A2.1

\begin{longtable}[]{@{}llll@{}}
\toprule\noalign{}
Dias & Horas por dia & Funcionários & Dificuldade \\
\midrule\noalign{}
\endhead
\bottomrule\noalign{}
\endlastfoot
16 & 6 & 25 & d \\
x & 8 & 50 & 3d \\
\end{longtable}

Portanto, o total de dias necessários será 18.

3) Uma família que tem sua própria horta, sabe que são utilizados 2
litros de água por hora para manutenção da irrigação por gotejamento.
Sabe-se que o reservatório de água tem 20 litros de capacidade. Com isso
podemos garantir que a água é suficiente pra manter a horta irrigada por
quantas horas?

Deixar 2 linhas

BNCC: EF09MA07 e EF09MA08 SAEB: 9A2.1

Basta calcular a razão \(\frac{20}{2} = 10\) horas.

4) No período de seca é comum termos incêndios em matas. Em um destes
incêndios de grandes proporções, foram chamados 60 bombeiros para
realizar o rescaldo numa área de 400 m². Considerando que estes
bombeiros demoraram 96 horas para controlar as chamas, quantos bombeiros
teriam sido necessários para controlar as chamas em 50 horas?

BNCC: EF09MA07 e EF09MA08 SAEB: 9A2.1

\begin{longtable}[]{@{}lll@{}}
\toprule\noalign{}
Bombeiros & Área & Tempo \\
\midrule\noalign{}
\endhead
\bottomrule\noalign{}
\endlastfoot
40 & 400 & 96 \\
x & 400 & 60 \\
\end{longtable}

Pela área está fixa não precisamos considerá-la.

5) Uma pequena fábrica de móveis de madeira tem dois funcionários
especialistas na produção artesanal de um tipo de cadeira. O Lucas
fabrica 20 cadeiras do modelo em 3 dias de 4 horas de trabalho por dia.
No entanto o Marcel fabrica 15 cadeiras do modelo em 8 dias de 2 horas
de trabalho por dia.

Eles receberam uma encomenda de 250 cadeiras, para atender esta demanda
se dedicaram em trabalhar 6 horas por dia. Quantos dias serão necessário
para concluir o trabalho?

Deixar 4 linhas

BNCC: EF09MA07 e EF09MA08 SAEB: 9A2.1

Lucas fabrica 20 cadeiras em 3 · 4 = 12 horas, Marcel fabrica 15
cadeiras em 8 · 2 = 16 horas.

O número total de horas necessárias para concluir o trabalho é igual a:

6) Para administrar um medicamento infantil normalmente a dosagem
depende da massa corpórea da criança. Veja a distribuição da tabela:

\begin{longtable}[]{@{}ll@{}}
\toprule\noalign{}
\textbf{Massa (kg)} & \textbf{Dosagem (mL)} \\
\midrule\noalign{}
\endhead
\bottomrule\noalign{}
\endlastfoot
Até 5 & 4 \\
10 & 6 \\
15 & 9 \\
\end{longtable}

Seguindo a proporcionalidade qual deve ser a dose para uma criança de 25
kg?

Deixar 4 linhas

BNCC: EF09MA07 e EF09MA08 SAEB: 9A2.1

Pela tabela notamos que a cada 5kg aumenta-se 3 ml na dosagem, sendo
assim para 25 kg deve-se tomar 15 mL.

7) Um comerciante compra maças pagando R\$ 7,50 para cada 3 kg e as
revende ao preço de R\$ 30,00 para cada 6 kg. Para que ele obtenha um
lucro de R\$ 420,00 qual quantidade ele deve comprar e revender?

Deixar 4 linhas

BNCC: EF09MA07 e EF09MA08 SAEB: 9A2.1

Para 6 kg de maça ele gasta R\$ 15,00 e recebe R\$ 30,00. Desta forma,
percebemos que ele recebe o dobro do valor gasto, sendo seu lucro metade
do valor empregado. Sendo assim ele deve comprar R\$ 420,00 em maças que
receberá o mesmo valor em lucro.

\(\frac{420}{2,5} = 168\) kg

8) Fernando quis fazer um teste sobre seu gasto de combustível. Para tal
ele em cada abastecimento, colocou a mesma quantidade de combustível e
anotou o valor pago pelo abastecimento. Posteriormente, calculou a
distância percorrida com aquela quantidade de combustível e construiu o
seguinte quadro:

\begin{longtable}[]{@{}lll@{}}
\toprule\noalign{}
\textbf{Tipo de combustível} & \textbf{Valor pago (R\$)} &
\textbf{Distância percorrida (km)} \\
\midrule\noalign{}
\endhead
\bottomrule\noalign{}
\endlastfoot
Gasolina comum & 210 & 350 \\
Gasolina aditivada & 270 & 360 \\
Etanol & 175 & 250 \\
\end{longtable}

Decidiu verificar o rendimento, ou seja, custo por quilômetro
percorrido. Qual foi o combustível mais econômico?

Deixar 4 linhas

BNCC: EF09MA07 e EF09MA08 SAEB: 9A2.1

Fazendo as divisões de valor por km temos:

Gasolina comum: 0,60

Gasolina aditivada: 0,75

Etanol: 0,70

O combustível mais econômico foi a gasolina comum.

9) Um grupo de cartógrafos decide imprimir um mapa de regiões de
preservação permanente. Eles gostariam que, no mapa, a distância entre
dois pontos seja de 2 cm Sabendo que a distância real é de,
aproximadamente, 12 km qual deve ser a escala utilizada no mapa?

Deixar 4 linhas

BNCC: EF09MA07 e EF09MA08 SAEB: 9A2.1

A razão deverá ser \(\frac{2}{1200000} = \frac{1}{600000}\), ou seja,
1:600000.

10) Uma oca de 50 m² de área é construída em 8 dias com o trabalho de 10
pessoas,

quantos dias serão necessários para fazer uma oca de 80 m² se indígenas
trabalharem na sua construção?

Deixar 3 linhas

BNCC: EF09MA07 e EF09MA08 SAEB: 9A2.1

\begin{longtable}[]{@{}lll@{}}
\toprule\noalign{}
Área & Dias & Índios \\
\midrule\noalign{}
\endhead
\bottomrule\noalign{}
\endlastfoot
50 & 8 & 10 \\
80 & x & 16 \\
\end{longtable}

\(\frac{8}{x} = \frac{50}{80} \cdot \frac{16}{10} \rightarrow \frac{8}{x} = \frac{1}{1} \cdot \frac{2}{2} \rightarrow x = 8\),
a quantidade de dias permanecerá a mesma.

\colorsec{Treino}

1) Um campeonato entre escolas foi promovido. Com isso separou-se
algumas salas que foram organizadas. Para arrumar tais salas, seis
pessoas trabalharam por três dias.

Para que a mesma quantidade total de salas de aula ficasse pronta em um
único dia, o número de pessoas a mais que teriam que ajudar na
arrumação, trabalhando no mesmo ritmo das anteriores, qual quantidade de
pessoas a mais ?

a) 18

b) 15

c) 12

d) 8

BNCC: EF09MA07 e EF09MA08 SAEB: 9A2.1

Basta perceber que por ser inversamente proporcional ao dividir por 3 o
tempo devemos multiplicar por 3 o número de pessoas. Desta forma
precisa-se de 18 pessoas, mas como o enunciado pede apenas a quantidade
de aumento, temos que 18 -- 6 =12 pessoas.

2) Para realizar inscrição em um evento, 3 pessoas atenderam 80 alunos
em 4 horas.

Se houvesse 4 pessoas atendendo os alunos no mesmo ritmo, quantas horas
eles levariam para atender 160 alunos?

a) 3 horas

b) 5 horas

c) 6 horas

d) 8 horas

e) 9 horas

BNCC: EF09MA07 e EF09MA08 SAEB: 9A2.1

\begin{longtable}[]{@{}lll@{}}
\toprule\noalign{}
Pessoas & Alunos & Tempo \\
\midrule\noalign{}
\endhead
\bottomrule\noalign{}
\endlastfoot
3 & 80 & 4 \\
4 & 160 & x \\
\end{longtable}

A grandeza alunos em relação ao tempo é direta, mas a grandezas pessoas
que atendem é inversamente proporcional em relação ao tempo.

\[\frac{4}{x} = \frac{80}{160} \cdot \frac{4}{3} \rightarrow \frac{4}{x} = \frac{4}{6} \rightarrow x = 6.\]

Alternativa c.

3) Augusto gasta 50 minutos para ir dirigindo de casa ao trabalho com
uma velocidade média de 80km/h. A uma velocidade média de 50km/h o tempo
gasto por ele é aproximadamente:

A) 10 minutos.

B) 25 minutos.

C) 31 minutos.

D) 64 minutos.

BNCC: EF09MA07 e EF09MA08 SAEB: 9A2.1

Como a velocidade e tempo são inversamente proporcionais temos:

\(\frac{80}{50} = \frac{50}{x} \rightarrow x = \frac{2500}{80} = 31,25\)
minutos.

Alternativa c.

\pagestyle{mat}
\chapter{Função do 1º e 2º grau}
\markboth{Módulo 9}{}

\colorsec{Habilidades do SAEB}

\begin{itemize}

  \item Associar uma das representações de uma função afim ou quadrática a
outra de suas representações (tabular, algébrica, gráfica) ou associar uma
situação que envolva função afim ou quadrática a uma das suas
representações (tabular, algébrica, gráfica).
  \item Resolver problemas que envolvam função afim.  

\end{itemize} 

\colorsec{Habilidade da BNCC}

\begin{itemize}
  \item EF09MA06.
\end{itemize}


Função do 1º e 2º grau

\conteudo{...}

A função do 1º grau é uma função linear que pode ser escrita na forma
f(x) = ax + b, onde a e b são constantes reais. É uma função que
representa uma reta no plano cartesiano e tem uma inclinação constante.
Ela é usada para modelar situações que envolvem uma relação de
proporcionalidade direta entre duas variáveis.

Já a função do 2º grau é uma função quadrática que pode ser escrita na
forma f(x) = ax² + bx + c, onde a, b e c são constantes reais, com a
diferente de zero. É uma função que representa uma curva no plano
cartesiano e pode ter um ponto máximo ou mínimo. Ela é usada para
modelar situações que envolvem uma relação de proporcionalidade indireta
entre duas variáveis.

\ul{Função do 1º grau -- Itens importantes}

\begin{itemize}
\item
  É do f(x) = ax + b. O coeficiente a é chamado de angular e b de
  coeficiente linear.
\item
  Gráfico é formada por uma reta, dados dois pontos no plano cartesiano
  é possível determinar o gráfico da função.
\item
  A função pode ser crescente (a \textgreater0), decrescente
  (a\textless0).
\end{itemize}

\ul{Função do 2º grau -- Itens importantes}

\begin{itemize}
\item
  É do f(x) = ax² + bx + c.
\item
  Gráfico é formada por uma parábola. A parábola pode ter concavidade
  para cima (a\textgreater0) ou para baixo (a\textless0).
\item
  A função tem ponto de máxima quando a parábola tem a concavidade para
  baixo, a função tem ponto de mínimo quando a parábola tem a
  concavidade para cima.
\end{itemize}

\colorsec{Atividades}

1) Uma empresa de transportes calcula o preço a ser cobrado de acordo
com a distância percorrida entre a coleta e a entrega dos objetos. O
preço total a pagar (P) depende da distância (d) a ser percorrida,
acrescido de um valor fixo de R\$ 400,00, referente ao carregamento e à
descarga dos objetos.

Parte dos valores são vistos nesta tabela

\begin{longtable}[]{@{}lllll@{}}
\toprule\noalign{}
\textbf{Representação parcial do quadro disponível na empresa} & & &
& \\
\midrule\noalign{}
\endhead
\bottomrule\noalign{}
\endlastfoot
Distância percorrida (km) & 10 & 20 & 30 & ... \\
Preço total a pagar (R\$) & 430 & 460 & 490 & ... \\
\end{longtable}

Escreva a função que pode modelar o preço.

Deixar 3 linhas

BNCC: EF09MA06 SAEB: 9A2.5

Podemos verificar pela tabela que o valor aumenta de 30 em 30 a cada
aumento de 10 km, sendo assim temos que o aumento será de 3 em 3 por
unidade.

Com isso temos:

\[P\left( d \right) = 3 \cdot d + 400\]

2) Veja o quadro a seguir e determine a regra matemática que relaciona x
e y.

\begin{longtable}[]{@{}lllll@{}}
\toprule\noalign{}
x & --1 & 2 & 3 & 5 \\
\midrule\noalign{}
\endhead
\bottomrule\noalign{}
\endlastfoot
y & --4 & 5 & 8 & 14 \\
\end{longtable}

Deixar 3 linhas

BNCC: EF09MA06 SAEB: 9A2.5

Primeiramente podemos imaginar a tabela de maneira que tenha o
distanciamento de uma unidade para cada valor de x.

\begin{longtable}[]{@{}llllllll@{}}
\toprule\noalign{}
x & --1 & 0 & 1 & 2 & 3 & 4 & 5 \\
\midrule\noalign{}
\endhead
\bottomrule\noalign{}
\endlastfoot
y & --4 & --1 & 2 & 5 & 8 & 11 & 14 \\
\end{longtable}

Com a tabela ampliada fica claro o aumento de 3 unidades em y a cada
aumento de 1 unidade em x. Essa regularidade nos aponta que em é função
afim do tipo y = ax+b e o valor de a é 3. Para determinar o valor de b
basta ajustar o valor:

Para x = 0, por exemplo, y = 3·0 + b = --1 (dado da tabela) 0 + b = --
1, ou seja, b = -- 1.

y = 3x -- 1

3) Considere o gráfico da função real representado no plano cartesiano a
seguir.

\begin{figure}
\centering
\includegraphics[width=1.97543in,height=2.20513in]{./_SAEB_9_MAT/media/image137.png}
\caption{Forma Descrição gerada automaticamente}
\end{figure}

Determine a função afim, f(x).

Deixar 4 linhas

BNCC: EF09MA06 SAEB: 9A2.5

Como o eixo y é ``cortado'' no valor 4, temos que b = 4, para determinar
o valor de a vamos substituir na função, pois pelo gráfico sabemos que
f(2) = 0 (raiz da função).

f(2) = a · 2 + 4 = 0 →2a = - 4 → a = - 2. f(x) = - 2x + 4.

4) Para se deslocar em uma cidade de interior que não possui transporte
de carro por aplicativo utiliza-se uma das empresas de transporte.

\begin{longtable}[]{@{}lll@{}}
\toprule\noalign{}
& \textbf{Empresa 1} & \textbf{Empresa 2} \\
\midrule\noalign{}
\endhead
\bottomrule\noalign{}
\endlastfoot
\textbf{Taxa fixa (R\$)} & 5 & 7 \\
\textbf{Taxa por quilômetro (R\$/km)} & 0,45 & 0,35 \\
\end{longtable}

Baseado nestes preços para qual distância o valor das duas empresas se
iguala.

Deixar 4 linhas

BNCC: EF09MA06 SAEB: 9A2.5

Empresa1: P\textsubscript{1} = 0,45x + 5

Empresa2: P\textsubscript{2} = 0,35x + 7

P\textsubscript{1} = P\textsubscript{2} → 0,45x + 5 = 0,35x + 7

0,10x = 2 → x = 20 km

5) Carlos trabalha como segurança, cobrando uma taxa fixa de 150,00 mais
20,00 por hora. Roberto, na mesma função, cobra uma taxa fixa de 120,00
mais 25,00 por hora. O tempo máximo para contratarmos o Roberto, de tal
forma que não seja mais caro que o de Carlos será, em horas, igual a

a)

b)

c)

d)

BNCC: EF09MA06 SAEB: 9A2.5

Carlos: P\textsubscript{C} = 20x + 150

Roberto: P\textsubscript{R} = 25x + 120

P\textsubscript{C} = P\textsubscript{R} → 20x + 150 = 25x + 120

30 = 5x → x = 6.

\textbf{6) O corte transversal de um túnel, de pista única, em que a
base tem 30 m de largura e} a altura máxima é de 10 m, o formato é um
arco de parábola, conforme representado na ilustração e no gráfico a
seguir, sendo o vértice da parábola.

\begin{figure}
\centering
\includegraphics[width=5.56667in,height=2.24198in]{./_SAEB_9_MAT/media/image143.png}
\caption{Imagem de vídeo game Descrição gerada automaticamente com
confiança média}
\end{figure}

Escreva uma função que modele essa parábola.

BNCC: EF09MA06 SAEB: 9A1.6

f(x) = a· (x -- x\textsubscript{1})·(x -- x\textsubscript{2}),
considerando x\textsubscript{1} = 0 e x\textsubscript{2} = 30 temos
então:

f(x) = a · (x -- 0)·(x -- 30) = a·(x² - 30x). Como a parábola é
simétrica temos que a altura é determinada quando x = 15.

f(15) = a·(15² - 30·15) = 10

a·(225 -- 450) = 10 → a = \(- \ \frac{10}{225} = \  - \ \frac{2}{45}\).

\(f\left( a \right) = - \ \frac{2}{45}(x^{2} - 30x)\).

7) Em uma indústria, o custo de produção em mil reais de x mil unidades
de determinado produto, é dada pela função Calcule o custo para a
produção de 100 unidades.

BNCC: EF09MA06 SAEB: 9A1.7

Calculando o custo para x = 100 temos:

C(100) = 0,1·(100)² - 4·100 + 70 = 0,1·10000 -- 400 + 70 = 670.

8) A concentração de um medicamento de certa medicação, em varia de
acordo com a função em que é o tempo decorrido, em horas, após a
ingestão da medicação, durante um período de observação de horas.
Determine a concentração após 12 horas.

BNCC: EF09MA06 SAEB: 9A1.7

C = 6·12 -- 0,25·12² = 72 -- 36 = 36.

9) Qual função representa o gráfico abaixo?

\begin{figure}
\centering
\includegraphics[width=2.14105in,height=1.76667in]{./_SAEB_9_MAT/media/image150.jpeg}
\caption{Função do 1 grau e seu gráfico (Função Afim) -}
\end{figure}

BNCC: EF09MA06 SAEB: 9A1.8

Sabemos que f(2) = 2 e f(5) = 4. Dado o gráfico sabemos que a função é
do tipo f(x)=ax+b.

f(2)=2a + b = 2

f(5)=5a + b = 4

Resolvendo o sistema
\(\left\{ \begin{matrix} 2a + b = 2 \\ 5a + b = 4 \\ \end{matrix} \rightarrow 3a = 2 \rightarrow a = \frac{2}{3} \right.\ \).

\[5 \cdot \frac{2}{3} + b = 4 \rightarrow \frac{10}{3} - 4 = - b\]

\(b = \frac{2}{3}\).

\(f\left( x \right) = \frac{2}{3}x + \frac{2}{3}\).

10) O deslocamento de um balão é dado pelo gráfico abaixo. Sabe-se que a
altura máxima foi atingida na metade do tempo. A altura em metros, do
balão, está em função do tempo em horas, através da fórmula . Qual é
altura máxima?

\begin{figure}
\centering
\includegraphics[width=2.62523in,height=2.05851in]{./_SAEB_9_MAT/media/image154.png}
\caption{Diagrama Descrição gerada automaticamente}
\end{figure}

BNCC: EF09MA06 SAEB: 9A2.4

Como a altura máxima foi atingida no tempo de 4 horas. h(4) =
\(- \frac{3}{4}4^{2} + 6 \cdot 4 = - 12 + 24 = 12\) horas.

\colorsec{Treino}

\num{1} Uma prestadora de serviços cobra pela visita à residência do
cliente e pelo tempo necessário para realizar o serviço na residência. O
valor da visita é R\$ 60 e o valor da hora para realização do serviço é
R\$ 35. Uma expressão que indica o valor a ser pago (P) em função das
horas (h) necessárias à execução do serviço é:

A) P = 40h

B) P = 60h

C) P = 35 + 60h

D) P = 60 + 35h

BNCC: EF09MA06 SAEB: 9A1.5

Dado que a função tem um valor que varia por hora e uma parte fixa, taxa
de visita. P(x)=35h+60.

Alternativa D

\num{2} Uma aplicação tem valorização determinada pelo gráfico abaixo,
diante dos dados determine o valor no tempo 10 anos.

\begin{figure}
\centering
\includegraphics[width=2.80858in,height=2.3252in]{./_SAEB_9_MAT/media/image155.png}
\caption{Forma Descrição gerada automaticamente com confiança média}
\end{figure}

\begin{enumerate}

\item
  260 000
\item
  300 000
\item
  320 000
\item
  400 000
\end{enumerate}

BNCC: EF09MA06 SAEB: 9A1.5 Pelo gráfico podemos concluir que temos uma
função linear em que a cada 2 anos há uma aumento de 40 000. Com isso em
10 anos teremos um aumento de cinco vezes maior, ou seja, 5·40 000 = 200
000. Alternativa D.

\num{3} Baseado no gráfico apresentado abaixo:

\begin{figure}
\centering
\includegraphics[width=1.72515in,height=1.5668in]{./_SAEB_9_MAT/media/image156.png}
\caption{Diagrama Descrição gerada automaticamente}
\end{figure}

Qual é o ponto de máximo?

\begin{enumerate}

\item
  (1,0)
\item
  (2,1)
\item
  (3,0)
\item
  (4,-3)
\end{enumerate}

BNCC: EF09MA06 SAEB: 9A2.4 Pelo gráfico podemos localizar o ponto mais
alto do gráfico e determinar seu par ordenado que é (2,1). Alternativa
B.

\pagestyle{mat}
\chapter{Geometria Plana I}
\markboth{Módulo 10}{}

\colorsec{Habilidades do SAEB}

\begin{itemize}

  \item Identificar, no plano cartesiano, figuras obtidas por uma ou mais
transformações geométricas (reflexão, translação, rotação).
  \item Relacionar o número de vértices, faces ou arestas de prismas ou
pirâmides, em função do seu polígono da base.
  \item Relacionar objetos tridimensionais às suas planificações ou vistas.
  \item Classificar polígonos em regulares e não regulares.
  \item Reconhecer polígonos semelhantes ou as relações existentes entre
ângulos e lados correspondentes nesses tipos de polígonos.
  \item Reconhecer circunferência/círculo como lugares geométricos, seus
elementos (centro, raio, diâmetro, corda, arco, ângulo central, ângulo
inscrito).
  \item Construir/desenhar figuras geométricas planas ou espaciais que
satisfaçam condições dadas.
  \item Resolver problemas que envolvam relações entre os elementos de uma
circunferência/círculo (raio, diâmetro, corda, arco, ângulo central, ângulo
inscrito).

\end{itemize} 

\colorsec{Habilidades da BNCC}

\begin{itemize}
  \item EF09MA11, EF09MA12.
\end{itemize}

\conteudo{

\begin{itemize}

  \item Plano Cartesiano

\begin{figure}
\centering
\includegraphics[width=1.8125in,height=1.8125in]{./_SAEB_9_MAT/media/image157.jpeg}
\caption{Plano cartesiano: o que é, como fazer, quadrantes}
\end{figure}

Montar nova figura

O plano cartesiano tem os pontos indicado par ordenado (a, b). O valor
\textbf{a} é a abscissa e \textbf{b} é a ordenada. Por exemplo, o ponto
A é (2, \num{3}.

  \item Polígonos

Polígonos são formas geométricas planas compostas por segmentos de reta
que se unem em vértices. Eles são classificados como convexos ou
não-convexos.

Exemplos de polígonos convexo e não convexo.

\includegraphics[width=2.32292in,height=1.19387in]{./_SAEB_9_MAT/media/image160.png}

Fazer figura semelhante

O polígono é convexo quando quaisquer dois pontos pertencentes ao
interior do polígono ao serem unidos formando um segmento de reto, este
segmento pertence totalmente ao polígono (não passa por fora, observe a
figura).

\begin{figure}
\centering
\includegraphics[width=3.07204in,height=1.70833in]{./_SAEB_9_MAT/media/image161.png}
\caption{Forma, Polígono Descrição gerada automaticamente}
\end{figure}

Fazer figura semelhante

  \item Relações de Euler

A relação de Euler nos ajuda a definir a quantidade de vértices, arestas
e faces.

\begin{figure}
\centering
\includegraphics[width=3.61979in,height=2.66667in]{./_SAEB_9_MAT/media/image162.png}
\caption{Forma Descrição gerada automaticamente}
\end{figure}

Refazer a imagem

  \item Elementos da circunferência

\begin{figure}
\centering
\includegraphics[width=1.53027in,height=1.57292in]{./_SAEB_9_MAT/media/image163.png}
\caption{Partes de uma Circunferência com Diagramas - Neurochispas}
\end{figure}

Fazer imagem semelhante

\begin{itemize}
\item
  Tangente -- um ponto em comum com a circunferência
\item
  Secante -- dois pontos em comum com a circunferência
\item
  Corda -- segmento de reta interior a circunferência que não passa pelo
  centro.
\item
  Diâmetro -- segmento de reta interior a circunferência que passa pelo
  centro.
\item
  Raio -- segmento de reta entre o centro a e borda da circunferência.
\end{itemize}

\end{itemize}
} 

\colorsec{Atividades}

\num{1} Escreva as coordenadas dos A, B e C.

\begin{figure}
\centering
\includegraphics[width=2.58356in,height=2.95026in]{./_SAEB_9_MAT/media/image164.png}
\caption{Gráfico, Gráfico de dispersão Descrição gerada automaticamente}
\end{figure}

Montar uma figura semelhante.

Deixar 2 linhas

BNCC: - SAEB: 9G1.1

A = (--1, 2); B = (2, 1) e C = (1, --3)

\num{2} O par ordenado de números que representa a igreja é:

\begin{figure}
\centering
\includegraphics[width=2.7in,height=1.92673in]{./_SAEB_9_MAT/media/image165.png}
\caption{Gráfico, Gráfico de dispersão Descrição gerada automaticamente}
\end{figure}

Montar figura semelhante

Deixar 2 linhas

BNCC: - SAEB: 9G1.1

A igreja está localizado no ponto ( - 3, 2)

\num{3} Uma formiga sai de um ponto x, anda 6 metros para a esquerda, 5
metros para cima, 2 metros para a direita, 2 metros para baixo, 6 metros
para a esquerda e 3 metros para baixo, chegando ao ponto y. Qual a
distância entre x e y?

Deixar 2 linhas

BNCC: - SAEB: 9G1.1

\begin{figure}
\centering
\includegraphics[width=3.54197in,height=2.80024in]{./_SAEB_9_MAT/media/image166.png}
\caption{Diagrama, Forma Descrição gerada automaticamente}
\end{figure}

A distância será de 10 metros

\num{4} Relacione as duas colunas ligando as planificações aos respectivos
objetos tridimensionais

\begin{figure}
\centering
\includegraphics[width=3.45in,height=2.11404in]{./_SAEB_9_MAT/media/image167.png}
\caption{Desenho de uma casa Descrição gerada automaticamente}
\end{figure}

Montar imagem semelhante

BNCC: - SAEB: 9G1.3

\begin{figure}
\centering
\includegraphics[width=3.45in,height=2.11404in]{./_SAEB_9_MAT/media/image167.png}
\caption{Desenho de uma casa Descrição gerada automaticamente}
\end{figure}

Montar figura semelhante

\num{5} Um sólido geométrico convexo é formado por 8 faces triangulares, 10
faces quadrangulares e 12 faces hexagonais. Quantos vértices possui esse
sólido geométrico?

BNCC: - SAEB: 9G1.2

Sabemos que o total de faces é igual a 8 + 10 + 12 = 30, para determinar
o número de arestas somamos todas e dividimos por 2, pois elas cada
aresta é compartilhada entre duas faces.

8 · 3 + 10 · 4 + 12 · 6 = 24 + 40 + 72 = 136 → 136 ÷ 2 = 68 arestas.
Pela relação de Euler V + F = A + 2 → V + 30 = 68 + 2 → V = 40.

\num{6} Escreva os respectivos nomes dos segmentos
\(\overline{\mathbf{\text{MN}}}\) e \(\overline{\mathbf{\text{RS}}}\) e
das retas \textbf{a} e \textbf{b}:

\begin{figure}
\centering
\includegraphics[width=4.87542in,height=2.25019in]{./_SAEB_9_MAT/media/image168.png}
\caption{Diagrama Descrição gerada automaticamente}
\end{figure}

BNCC: - SAEB: 9G1.8

\(\overline{\mathbf{\text{MN}}}\) é corda

\(\overline{\mathbf{\text{RS}}}\) é o diâmetro

A reta a é secante

A reta b é tangente

\num{7} Considere as figuras planificadas I e II

\begin{figure}
\centering
\includegraphics[width=3.93333in,height=2.18333in]{./_SAEB_9_MAT/media/image169.png}
\caption{Forma, Retângulo Descrição gerada automaticamente}
\end{figure}

Montar figura semelhante

Estas planificações ao serem montadas de maneira tridimensional irão
gerar respectivamente:

\begin{enumerate}

\item
  Prisma, Prisma
\item
  Prisma, Cone
\item
  Pirâmide, Prisma
\item
  Prima, Pirâmide
\end{enumerate}

BNCC: - SAEB: 9G1.3

A figura I irá gerar um prisma triangular e a figura II uma pirâmide
quadrangular.

Alternativa D.

\num{8} Uma pirâmide com base hexagonal possui quantas faces e quantos
vértices?

Deixar uma linha

BNCC: - SAEB: 9G1.2

Uma pirâmide com base hexagonal terá seis faces triangulares (laterais),
mais a base hexagonal, ou seja, serão 7 faces. Em relação aos vértices
terá os seis vértices da base e mais o vértice do topo, desta forma
serão 7 vértices.

\num{9} Uma cuia indígena é um tipo de cuia feita por indígenas, utilizando
técnicas e materiais tradicionais de suas culturas. As cuias indígenas
podem ser feitas de diferentes materiais, como cabaça, madeira, bambu,
taquara, entre outros, dependendo da região e da cultura de cada povo
indígena.

As cuias indígenas são usadas para diversas finalidades, dependendo da
cultura em que são produzidas. Um exemplo de uso é armazenar e servir
alimentos e bebidas.

A parte superior de uma determinada cuia tem a forma de uma
circunferência de 60 cm de comprimento.

\includegraphics[width=2.31667in,height=1.40833in]{./_SAEB_9_MAT/media/image170.wmf}

Fazer figura semelhante

Qual a medida em centímetros do raio dessa circunferência?

Use

a) 10

b) 12

c) 15

d) 20

BNCC: - SAEB: 9G2.7

Como o comprimento da circunferência é dado por
\(C = 2\pi r = 2 \cdot 3 \cdot r = 60 \rightarrow r = 10\) cm.

Alternativa A.

\num{10} Nomeio os ângulos ACD e ABD como um ângulo central ou ângulo
inscrito.

\includegraphics[width=2.425in,height=2.4in]{./_SAEB_9_MAT/media/image172.wmf}

Montar figura semelhante com os pontos A, B, C e D. Não é necessário
deixar este pontilhado da parte de baixo.

ACD é um ângulo \_\_\_\_\_\_\_\_\_\_\_\_\_\_\_

ABD é um ângulo \_\_\_\_\_\_\_\_\_\_\_\_\_\_\_

BNCC: - SAEB: 9G1.8

ACD é um ângulo \textbf{central}. Porque o vértice está no centro da
circunferência.

ABD é um ângulo \textbf{inscrito}. Porque o vértice está sobre a
circunferência.

\colorsec{Treino}

\num{1} Considere um poliedro convexo que possui oito vértices e apenas faces
triangulares e quadrangulares. Sabe-se que o número de faces
triangulares desse poliedro é quatro vezes maior do que o número de
faces quadrangulares. Determine a quantidade de arestas presentes nesse
poliedro.

Sabe-se que Vértice + Face = Aresta + 2

a)

b)

c)

d)

BNCC: - SAEB: 9G1.2

Sejam e respectivamente, o número de faces triangulares e o número de
faces quadrangulares. Logo, como temos

em que é o número de ares

Ad\textbf{pela }Relação de Euler, temos

A resposta é

\num{2} Uma pirâmide que possui 6 vértices e 6 faces tem como base um:

a) triângulo

b) quadrado

c) pentágono

d) hexágono

BNCC: - SAEB: 9G1.2

Por ser pirâmide o sólido obrigatoriamente terá um vértice no topo e o
restante na base, ou seja, 5 na base. Isso implica que a base é um
pentágono.

Alternativa C.

\num{3} Dos polígonos a seguir, indique aquele que é sempre regular.

a) Retângulo

b) Triângulo isósceles

c) Trapézio

d) Quadrado

BNCC: - SAEB: 9G1.4

Entre todas as figuras listadas temos que apenas o quadrado tem ângulos
internos iguais e lados iguais, por isso o único regular.

\pagestyle{mat}
\chapter{Geometria II}
\markboth{Módulo 11}{}

\colorsec{Habilidades do SAEB}

\begin{itemize}

  \item Identificar propriedades e relações existentes entre os elementos de um
triângulo (condição de existência, relações de ordem entre as medidas dos
lados e as medidas dos ângulos internos, soma dos ângulos internos,
determinação da medida de um ângulo interno ou externo).
  \item Classificar triângulos ou quadriláteros em relação aos lados ou aos
ângulos internos.
  \item Identificar retas ou segmentos de retas concorrentes, paralelos ou
perpendiculares.
  \item Identificar relações entre ângulos formados por retas paralelas cortadas
por uma transversal.
  \item Resolver problemas que envolvam relações entre ângulos formados por
retas paralelas cortadas por uma transversal, ângulos internos ou externos
de polígonos ou cevianas (altura, bissetriz, mediana, mediatriz) de
polígonos.
  \item Resolver problemas que envolvam relações métricas do triângulo
retângulo, incluindo o teorema de Pitágoras.
  \item Resolver problemas que envolvam polígonos semelhantes.
  \item Resolver problemas que envolvam aplicação das relações de
proporcionalidade abrangendo retas paralelas cortadas por transversais.
  \item Determinar o ponto médio de um segmento de reta ou a distância entre
dois pontos quaisquer, dadas as coordenadas desses pontos no plano
cartesiano.

\end{itemize} 

\colorsec{Habilidades da BNCC}

\begin{itemize}
  \item EF09MA10, EF09MA13, EF09MA14, EF09MA16. 
\end{itemize}

\conteudo{

\begin{itemize}
        \item \textbf{Triângulos}

Os triângulos podem ser classificados pela medida do lado ou pelos
ângulos:

\ul{Lado}

\begin{itemize}
\item
  Triângulo escaleno: triângulo que possui todos os lados com medidas
  diferentes.
\item
  Triângulos isósceles: triângulo que possui dois lados com medidas
  iguais.
\item
  Triângulo equilátero: triângulo que possui todos os lados com medidas
  iguais.
\end{itemize}

\ul{Ângulo}

\begin{itemize}
\item
  Triângulo acutângulo: possui todos os ângulos internos menores que
  90°.
\item
  Triângulo obtusângulo: possui exatamente um ângulo interno maior que
  90°.
\item
  Triângulo retângulo: possui exatamente um ângulo interno igual a 90°.
\end{itemize}

\ul{Propriedades}

Estas são as principais propriedades dos triângulos:

\begin{itemize}
\item
  A soma dos ângulos internos de qualquer triângulo é sempre igual a
  180°.
\item
  A soma dos ângulos externos de qualquer triângulo é sempre igual a
  360°.
\item
  A soma das medidas de dois lados de um triângulo é sempre maior que a
  medida do terceiro lado. Essa propriedade é chamada de desigualdade
  triangular, trata-se da condição de existência do triângulo.
\end{itemize}

        \item \textbf{Paralelas cortadas por uma transversal}

Duas retas distintas são paralelas quando possuem a mesma inclinação, ou
seja, possuem o mesmo coeficiente angular. Além disso, a distância entre
elas é sempre a mesma e não possuem pontos em comum (não se cruzam).

Seja duas retas r e s, paralelas entre si e uma transversal t, não
perpendicular a r e s. Temos que os 8 (oito) ângulos formados pela reta
transversal com as retas r e s, quatro deles serão agudos (α) e
congruentes (mesma medida), os outros quatro serão obtusos (β) e
congruentes. Além disso, os ângulos obtusos e agudos serão suplementares
(medem 180°).

\includegraphics[width=1.96354in,height=1.3968in]{./_SAEB_9_MAT/media/image186.png}
\begin{itemize}

}

Refazer a figura

\colorsec{Atividades}

\num{1} Determine o valor de x e os valores dos ângulos A e B.

\begin{figure}
\centering
\includegraphics[width=2.83333in,height=1.60939in]{./_SAEB_9_MAT/media/image187.jpeg}
\caption{Lista de Exercícios sobre retas paralelas cortadas por uma
transversal}
\end{figure}

Montar figura semelhante, colocando o ângulo 2x -- 60º = A e o ângulo
x/2 + 30º = B

BNCC: EF09MA14 SAEB: 9G1.10

Pela figura sabemos que os ângulos alternos externos são congruentes, ou
seja, A = B.

\[2x - 60 = \frac{x}{2} + 30\]

\[2x - \frac{x}{2} = 30 + 60 \rightarrow \frac{4x - x}{2} = 90 \rightarrow \frac{3}{2}x = 90 \rightarrow x = 60\]

X = 60º, A = B = 60º.

\num{2} Um retângulo foi desenhado em um plano cartesiano, com altura
determinada pelos vértices dos pontos A e B de coordenadas (1, 3) e (1,
\num{4}, respectivamente. Além disso, sabe-se que um ponto M de
coordenadas (3, 3) é o ponto médio da base do retângulo.

Qual é a área desse retângulo?

Deixar 3 linhas

BNCC: EF09MA16 SAEB: 9G2.3

Construindo o retângulo com vértices em A e B, podemos perceber que sua
altura será igual a 1 unidade. A distância entre o ponto A (1,3) e M
(3,3) é de 2 unidades, dado que M é ponto médio da base, podemos afirmar
que a base tem 4 unidades.

Assim para determinar a área, basta fazer 1 · 4 = 4 unidades quadradas
de área.

\num{3} Veja o texto:

\textbf{Futebol e Matemática}

A bola de futebol utilizada na Copa de 2010, na África do Sul, ganhou o
apelido de "Jabulani" e foi criticada pelos jogadores por ser mais
simples do que as bolas utilizadas desde a década de 1970, que
apresentam uma estrutura icosaedro truncado com 32 faces compostas por
12 pentágonos regulares e 20 hexágonos regulares. Cada um dos 60
vértices do icosaedro truncado é formado pela união de um pentágono e
dois hexágonos, resultando em 90 arestas para o sólido.

Qual soma dos ângulos internos do pentágono e hexágono respectivamente?

~Deixar 3 linhas

BNCC: - SAEB: 9G2.3

A soma dos ângulos internos é dada por:

S = (n -- 2)· 180º

Para pentágono temos: S = 3 · 180 = 540º

Para hexágono temos: S = 4 · 180 = 720º

\num{4} Um estudante fez um polígono inscrito em um círculo. Para executar
essa tarefa, ele dividiu o círculo em cinco arcos de mesma medida, e
marcou os pontos A, B, C, D e E.

\begin{figure}
\centering
\includegraphics[width=2.45291in,height=2.3178in]{./_SAEB_9_MAT/media/image188.png}
\caption{1631935324022\_.jpg}
\end{figure}

Determine a medida de AOB.

~Deixar 3 linhas

BNCC: - SAEB: 9G2.7

Como a circunferência foi dividida em 5 arcos, basta fazer 360 ÷ 5 =
72º, com AOB é ângulo central a medida do ângulo e arco é a mesma.

\num{5} Um polígono com ângulos centrais medindo 40º tem quantos lados? Qual
a medida do ângulo interno deste mesmo polígono?

~Deixar 3 linhas

BNCC: EF09MA11 SAEB: 9G2.7

Como o ângulo central tem medida 40º, podemos determinar o número de
lados do polígono utilizando o Ângulo de 360º: 360 ÷ 40 = 9. O polígono
tem 9 lados. A soma dos ângulo internos é dada por S = 7 · 180 = 1260,
logo cada ângulo interno é dado por 1260 ÷ 9 = 140º.

\num{6} Entre as medidas dos palitos dadas abaixo quais dela impossibilitam a
formação de um triângulo?

I - Três palitos de 7 cm cada;

II - Dois palitos de 7 cm e um palito de 15 cm;

III - Dois palitos de 10 cm e um palito de 13 cm;

IV - Um palito de 10 cm, um de 13 cm e outro de 15 cm.

~Deixar 5 linhas

BNCC: - SAEB: 9G1.5

Para testar a condição de existência de um triângulo, é preciso
verificar se a soma das medidas de dois lados quaisquer é maior do que a
medida do terceiro lado.

II - Dois palitos de 7 cm e um palito de 15 cm:

A soma das medidas de dois lados quaisquer é igual a 14 cm, que é menor
do que a medida do terceiro lado, que é de 15 cm. Portanto, não é
possível construir um triângulo com dois palitos de 7 cm e um palito de
15 cm.

\num{7} João montou um triângulo retângulo utilizando régua e compasso. Os
catetos formados tinham as medidas 12 cm e 16 cm. Com isso qual será a
medida do 3º lado?

~Deixar 4 linhas

BNCC: EF09MA13 SAEB: 9G2.4

Pelo teorema de Pitágoras temos:

x² = 12² + 16²

x² = 144 + 256

x² = 400 → x = 20.

Pode-se também utilizar da ideia do triângulo 3, 4 e 5 e perceber que é
semelhante ao triângulo do problema com razão de proporção 4. Ou seja,
bastaria fazer 4 · 5 = 20.

\num{8} Ao tentar construí um triângulo escaleno, ou seja, como todos os
lados diferentes quais são as opções para o terceiro lado, com medida
inteira, sabendo que os dois primeiros lados têm a medida 3 e 4?

~Deixar 5 linhas

BNCC: - SAEB: 9G1.5

Para construir um triângulo escaleno, é preciso que os três lados tenham
medidas diferentes. Se dois lados desse triângulo tiverem medidas
respectivamente iguais a 3 e 4, o terceiro lado só poderá ter uma medida
que não seja 3 e nem 4.

Descartamos o valor 1, pois a soma com 3 daria 4 que a medida de um dos
lados, pelo critério de existência do triângulo podemos usar os valores:
2, 5 e 6.

\num{9} Terrenos estão localizados entre as ruas das rosas e rua das flores,
sabendo que o proprietário pode optar por fazer a frente da casa par
qualquer uma das flores. Qual a medida, em metros, dos lotes na rua das
Flores?

\begin{figure}
\centering
\includegraphics[width=2.89167in,height=1.99596in]{./_SAEB_9_MAT/media/image189.png}
\caption{Diagrama, Histograma Descrição gerada automaticamente}
\end{figure}

Montar uma figura semelhante

BNCC: - SAEB: 9G2.6

Podemos calcular através do Teorema de Tales. Fazendo a proporção para
determinar os valores:

\[\frac{20}{x} = \frac{50}{64} \rightarrow 50x = 1\ 280 \rightarrow x = 25,6\]

\[\frac{60}{x} = \frac{50}{64} \rightarrow 50x = 3840 \rightarrow x = 76,8\]

\num{10} Determine o valor de α, sabendo que temos uma figura formada por 2
quadrados e um triângulo equilátero.

\begin{figure}
\centering
\includegraphics[width=2.90025in,height=1.35012in]{./_SAEB_9_MAT/media/image190.png}
\caption{Forma, Polígono Descrição gerada automaticamente}
\end{figure}

Montar uma figura semelhante

BNCC: - SAEB: 9G2.7

Dado a formação com dois quadrados e um triângulo equilátero temos:

\begin{figure}
\centering
\includegraphics[width=2.975in,height=1.53351in]{./_SAEB_9_MAT/media/image191.png}
\caption{Forma, Polígono Descrição gerada automaticamente}
\end{figure}

Temos que x = 60º e y = 90º. Desta forma α será igual a:

α + 90 + 90 + 60 = 360, pois fecha a circunferência. Logo podemos
afirmar que α = 120º.

\colorsec{Treino}

\num{1} Um portão é formado por barras de fecho:

\includegraphics[width=3.36667in,height=1.51667in]{./_SAEB_9_MAT/media/image192.png}{]}

Fazer figura semelhante

Determine a medida da barra que está na diagonal.

\begin{enumerate}

\item
  2
\item
  2,5
\item
  3
\item
  4
\end{enumerate}

BNCC: EF09MA13 SAEB: 9G2.4

Pelo Teorema de Pitágoras temos: x² = 1,5² + 2² → x² = 2,25 + 4 = 6,25 →
x = 2,5 m.

Alternativa B.

\num{2} Qual figura representa um trapézio escaleno?

\begin{figure}
\centering
\includegraphics[width=3.44197in,height=1.9335in]{./_SAEB_9_MAT/media/image193.png}
\caption{Diagrama, Forma, Polígono Descrição gerada automaticamente}
\end{figure}

Montar figura semelhante

\begin{enumerate}

\item
  I
\item
  II
\item
  III
\item
  IV
\end{enumerate}

BNCC: EF09MA13 SAEB: 9G1.6

Para ser escaleno devemos ter todos os ângulos e lados diferentes e isso
ocorre apenas no item IV.

\num{3} Observe esta imagem que apresenta paralelas cortadas por resta
transversais.

\begin{figure}
\centering
\includegraphics[width=3.65833in,height=1.84637in]{./_SAEB_9_MAT/media/image194.jpeg}
\caption{Exercícios sobre retas paralelas cortadas por uma transversal -
Toda Matéria}
\end{figure}

Fazer imagem semelhante

Determine o valor de x.

\begin{enumerate}

\item
  12º
\item
  20º
\item
  24º
\item
  48º
\end{enumerate}

BNCC: EF09MA10 SAEB: 9G2.3

\begin{figure}
\centering
\includegraphics[width=4.15in,height=2.15991in]{./_SAEB_9_MAT/media/image195.png}
\caption{Diagrama Descrição gerada automaticamente}
\end{figure}

Fazer imagem semelhante

Primeiramente traçamos retas auxiliares cortando a estrutura nos
vértices. Aplicando a regra de transversais podemos seguir os seguintes
passos:

O valor 25º pode ser colocado como parte do ângulo de 43º, ou seja,
sobra 18º que pode ser comparado a parte do lado de 3x, por outro lado
temos a medida 54º que também é parte.

Assim temos:

3x = 54 + 18 = 72 → x = 24º.

Alternativa C.

\pagestyle{mat}
\chapter{Deslocamento usando coordenadas}
\markboth{Módulo 12}{}

\colorsec{Habilidades do SAEB}

\begin{itemize}

  \item Descrever ou esboçar deslocamento de pessoas e/ou de objetos em
representações bidimensionais (mapas, croquis etc.), plantas de
ambientes ou vistas, de acordo com condições dadas.   

\end{itemize} 

\conteudo{

\textbf{Localização e movimentação}

Interpretar e construir representações espaciais são habilidades
fundamentais para a vida cotidiana das pessoas em diversas situações.
Compreender noções como trajetória, direção e sentido é essencial para
se localizar e comunicar posições e deslocamentos, tanto em ambientes
menores, como a sala de aula ou a própria casa, quanto em ambientes
maiores, como a cidade.

Ao interpretar uma representação espacial, é preciso selecionar
referências para se localizar e entender as informações presentes. Essas
referências podem ser pontos de referência físicos, como um edifício
alto ou uma praça, ou pontos de referência abstratos, como um cruzamento
ou uma interseção de ruas.

Já ao construir uma representação espacial, é necessário ter em mente as
noções de trajetória, direção e sentido para indicar com precisão a
posição e o deslocamento de objetos ou pessoas. Além disso, é importante
utilizar símbolos e legendas adequados para tornar a representação clara
e compreensível.

Em resumo, interpretar e construir representações espaciais, localizar
objetos e comunicar posições e deslocamentos são habilidades
fundamentais para a vida cotidiana. O entendimento de noções como
trajetória, direção e sentido é essencial para se localizar e
interpretar informações em ambientes de dimensões menores e maiores.

\textbf{Trajetória}

A trajetória é o caminho percorrido por um objeto, pessoa ou evento em
uma sequência de pontos, desde o ponto de partida até o ponto de
chegada. Ao longo da trajetória, podem existir pontos fixos que servem
como referência para a localização e orientação.

A trajetória pode apresentar deslocamentos em linha reta ou em linha
curva, dependendo do percurso que é seguido. A direção indica a
orientação em que o deslocamento ocorre, enquanto o sentido indica a
orientação em relação ao ponto de partida.

É importante entender e interpretar a trajetória, a direção e o sentido
em diversas situações do cotidiano, como na locomoção em uma cidade ou
na navegação em um mapa. Compreender essas noções é fundamental para se
orientar e se deslocar com segurança e eficiência.

Por exemplo, o deslocamento da Praça da Sé na cidade de São Paulo até a
NeoQuímica Arena (Arena Corinthians).

\begin{figure}
\centering
\includegraphics[width=3.39062in,height=1.68534in]{./_SAEB_9_MAT/media/image196.png}
\caption{Mapa Descrição gerada automaticamente}
\end{figure}

Disponível em:
\url{https://www.google.com.br/maps/dir/Pra\%C3\%A7a+da+S\%C3\%A9+-+S\%C3\%A9,+S\%C3\%A3o+Paulo+-+SP,+01001-000/Arena+Corinthians+-+Avenida+Miguel+Ign\%C3\%A1cio+Curi+-+Artur+Alvim,+S\%C3\%A3o+Paulo+-+SP/@-23.5331433,-46.5950426,12.74z/data=!3m1!5s0x94ce66de952b5e23:0x3333a5705850af9c!4m14!4m13!1m5!1m1!1s0x94ce59abaaae4233:0xd9186faf714bc5b1!2m2!1d-46.6342009!2d-23.5503099!1m5!1m1!1s0x94ce66dec98fb855:0xf2b061ffbcd2ecf8!2m2!1d-46.4742676!2d-23.5453221!3e0}
}

\colorsec{Atividades}

1. Gustavo e Heitor utilizaram um tabuleiro quadriculado como tabuleiro
de um ``jogo da velha'', no qual as linhas são identificadas por letras,
e, as colunas, por números. Durante uma partida, Gustavo decidiu começar
o jogo pela casa destacada em cinza.

\includegraphics[width=1.5918in,height=1.55847in]{./_SAEB_9_MAT/media/image197.png}

Construir a figura semelhante

Qual é a coordenada que o Gustavo decidiu começar a jogada?

BNCC: - SAEB 9G2.1

Ele começou pela casa G3.

2. Observando um tabuleiro de xadrez com algumas peças escreva a posição
da peça
\includegraphics[width=0.27083in,height=0.27639in]{./_SAEB_9_MAT/media/image198.png}

no tabuleiro.

\begin{figure}
\centering
\includegraphics[width=2.40278in,height=2.27161in]{./_SAEB_9_MAT/media/image199.png}
\caption{Gráfico, Gráfico de dispersão Descrição gerada automaticamente}
\end{figure}

Montar figura semelhante, posição da peça influencia na resposta.

BNCC: - SAEB 9G2.1

A peça está na casa I3.

3. Veja o mapa abaixo:

\begin{figure}
\centering
\includegraphics[width=2.74306in,height=2.00374in]{./_SAEB_9_MAT/media/image200.png}
\caption{Diagrama Descrição gerada automaticamente}
\end{figure}

Montar figura semelhante, posição influencia na resposta.

Eloisa disse para sua amiga Lara que mora numa rua entre as avenidas A e
B e entre as ruas da igreja e da locadora. Sendo assim, qual rua Eloisa
mora?

BNCC: - SAEB 9G2.1

Entre as avenidas A e B temos as ruas 2 e 4, mas entre a Igreja e
locadora temos a rua 4, logo é a rua da casa da Eloisa.

4. Considerando o mapa da região onde Lucas mora escreva a localização
da escola.

\begin{figure}
\centering
\includegraphics[width=1.71528in,height=1.71528in]{./_SAEB_9_MAT/media/image201.png}
\caption{Diagrama Descrição gerada automaticamente}
\end{figure}

Montar uma imagem semelhante

Deixe 1 linhas (pode ser na lateral da imagem)

BNCC: - SAEB 9G2.1

A escola está na coluna 3, linha E.

5. O mapa abaixo é de um bairro, em que cada quadrado representa um
quarteirão, cuja a medida do lado de cada quadrado é de 100m.

\begin{figure}
\centering
\includegraphics[width=2.4375in,height=1.82292in]{./_SAEB_9_MAT/media/image202.png}
\caption{Imagem em branco e preto Descrição gerada automaticamente com
confiança média}
\end{figure}

Fazer imagem semelhante, é importante os dados

Maria saiu da esquina indicada pelo ponto Q e percorreu o seguinte
percurso:

• caminhou 500 metros na direção Norte; • depois caminhou 200 metros na
direção Oeste; • depois caminhou 300 metros na direção Sul; • e,
finalmente, caminhou mais 200 metros na direção Oeste.

Ao final desse percurso, Maria chegou na esquina indicada pela letra:

\begin{enumerate}

\item
  R
\item
  T
\item
  S
\item
  P
\end{enumerate}

BNCC: - SAEB 9G2.1

\begin{figure}
\centering
\includegraphics[width=1.50277in,height=1.45139in]{./_SAEB_9_MAT/media/image203.png}
\caption{Tela de celular Descrição gerada automaticamente com confiança
baixa}
\end{figure}

Fazer imagem semelhante, é importante os dados

Alternativa C.

6. Luiz Felipe criou uma planta da sua casa para um trabalho da escola.

\begin{figure}
\centering
\includegraphics[width=2.52083in,height=1.18147in]{./_SAEB_9_MAT/media/image204.png}
\caption{Diagrama Descrição gerada automaticamente}
\end{figure}

Fazer imagem semelhante, é importante os dados

Ao entrar pela porta da sala virar a esquerda, seguir até o final do
corredor e virar a esquerda novamente, a qual cômodo Luiz Felipe foi?

BNCC: - SAEB 9G2.1

\begin{figure}
\centering
\includegraphics[width=3.28362in,height=1.58347in]{./_SAEB_9_MAT/media/image205.png}
\caption{Diagrama Descrição gerada automaticamente}
\end{figure}

Fazer imagem semelhante, é importante os dados

Deixar 1 linha

Luiz Felipe chegou ao Quarto1.

7. João e Marcos resolveram disputar uma corrida diferente, entre os
pontos A e B, partindo simultaneamente de A e deslocando-se
rigorosamente sobre as linhas tracejadas das alamedas.

\begin{figure}
\centering
\includegraphics[width=3.00694in,height=1.36847in]{./_SAEB_9_MAT/media/image206.png}
\caption{Diagrama Descrição gerada automaticamente}
\end{figure}

Fazer imagem semelhante, é importante os dados

Qual a diferença, em metros, da distância percorrida pelo João e Marcos?

Deixar 2 linhas

BNCC: - SAEB 9G2.1

A distância percorrida pelo amigo da bicicleta foi: 150 + 150 + 250 +
250 + 150 + 150 = 1 100 metros. O amigo a pé 250 + 250 = 500 m.
Diferença entre eles: 1 100 -- 500 = 600 metros.

8. O mapa de uma região de uma cidade do interior está representado na
figura abaixo.

\begin{figure}
\centering
\includegraphics[width=2.96528in,height=2.96528in]{./_SAEB_9_MAT/media/image207.png}
\caption{Diagrama Descrição gerada automaticamente}
\end{figure}

Fazer imagem semelhante, é importante os dados

Deixar 1 linha

Jorge saiu da praça central e, orientando-se por esse mapa, caminhou 3
quadras na direção oeste e, depois, 2 quadras na direção sul. Em qual
local Jorge chegou?

BNCC: - SAEB 9G2.1

Jorge chegou ao Posto de Saúde.

\begin{figure}
\centering
\includegraphics[width=1.95139in,height=1.84648in]{./_SAEB_9_MAT/media/image208.png}
\caption{Diagrama Descrição gerada automaticamente}
\end{figure}

9. Carlos precisa sair do ponto A e chegou no ponto D. Faça a indicação
de qual menor caminho para interligar ponto A ao ponto D.

\begin{figure}
\centering
\includegraphics[width=2.40278in,height=1.53858in]{./_SAEB_9_MAT/media/image209.jpeg}
\caption{Calendário Descrição gerada automaticamente}
\end{figure}

Fazer imagem semelhante, é importante os dados

Deixar 2 linhas

BNCC: - SAEB 9G2.1

Carlos pode ter saído do ponto A pela Avenida Orla por 1 quarteirão,
virar a direita na Rua Paraguai e seguir até o cruzamento da Avenida
Projetada.

10. Um turista caminhou por todo os pontos turísticos, a distância entre
os pontos estão em quilômetros.

\begin{figure}
\centering
\includegraphics[width=2.77778in,height=2.5in]{./_SAEB_9_MAT/media/image210.png}
\caption{Mapa Descrição gerada automaticamente}
\end{figure}

Fazer imagem semelhante, é importante os dados

Saindo de A, passando por B, C, D, E e retornando ao ponto A. Qual a
distância total percorrida, em km?

Deixar 2 linhas

BNCC: - SAEB 9G2.1

90 + 153 + 121 + 239 + 117 = 720 km.

\colorsec{Treino}

1. Observe a imagem:

Fazer imagem semelhante, é importante os dados

Qual a distância entre a casa da Silva e Paula?

\begin{enumerate}

\item
  100 m
\item
  200 m
\item
  300 m
\item
  400 m
\end{enumerate}

BNCC: - SAEB 9G2.1

Alternativa D. São 2 quadras ao Sul e duas quadras ao Leste

2. Qual é a melhor maneira de determinar sua localização exata em um
mapa? a) Usando um sistema de orientação, como uma bússola ou GPS. b)
Estimando sua posição com base em pontos de referência próximos. c)
Adivinhando sua posição com base em sua experiência anterior na área. d)
Olhando para o mapa e tentando comparar com o ambiente ao seu redor.

BNCC: - SAEB 9G2.1

A resposta correta é a alternativa A.

Usando um sistema de orientação, como uma bússola ou GPS.

3. Determine a distância entre a casa de Lucas e a escola.

\begin{figure}
\centering
\includegraphics[width=3.29167in,height=1.79167in]{./_SAEB_9_MAT/media/image212.png}
\caption{Diagrama Descrição gerada automaticamente}
\end{figure}

Fazer imagem semelhante, é importante os dados

a) 3 Norte e 2 Leste

b) 2 Norte e 3 Leste

c) 5 Sul e 2 Oeste

d) 3 Norte e 4 Leste

BNCC: - SAEB 9G2.1

Alternativa A. Subiu 3 quadras Norte e 2 quartas Leste.

\pagestyle{mat}
\chapter{Estatística}
\markboth{Módulo 13}{}

\colorsec{Habilidades do SAEB}

\begin{itemize}

  \item Identificar os indivíduos (universo ou população alvo da pesquisa), as
variáveis e os tipos de variáveis (quantitativas ou categóricas) em um
conjunto de dados.
  \item Representar ou associar os dados de uma pesquisa estatística ou de um
levantamento em listas, tabelas (simples ou de dupla entrada) ou gráficos
(barras simples ou agrupadas, colunas simples ou agrupadas, pictóricos,
de linhas, de setores, ou em histograma).
  \item Inferir a finalidade da realização de uma pesquisa estatística ou de um
levantamento, dada uma tabela (simples ou de dupla entrada) ou gráfico
(barras simples ou agrupadas, colunas simples ou agrupadas, pictóricos,
de linhas, de setores ou em histograma) com os dados dessa pesquisa. 
  \item Interpretar o significado das medidas de tendência central (média
aritmética simples, moda e mediana) ou da amplitude. 
  \item Calcular os valores de medidas de tendência central de uma pesquisa
estatística (média aritmética simples, moda ou mediana). 
  \item Resolver problemas que envolvam dados estatísticos apresentados em
tabelas (simples ou de dupla entrada) ou gráficos (barras simples ou
agrupadas, colunas simples ou agrupadas, pictóricos, de linhas, de setores
ou em histograma). 
  \item Argumentar ou analisar argumentações/conclusões com base nos dados
apresentados em tabelas (simples ou de dupla entrada) ou gráficos (barras
simples ou agrupadas, colunas simples ou agrupadas, pictóricos, de linhas,
de setores ou em histograma). 
  \item Explicar/descrever os passos para a realização de uma pesquisa
estatística ou de um levantamento.

\end{itemize} 

\colorsec{Habilidades da BNCC}

\begin{itemize}
  \item EF09MA21, EF09MA22.
\end{itemize}

\conteudo{
A moda, a média e a mediana são medidas de tendência central usadas na
estatística para resumir um conjunto de dados.

A moda é o valor mais comum em um conjunto de dados, ou seja, é o valor
que aparece com mais frequência. É útil para dados discretos e pode ser
usada em combinação com outras medidas de tendência central.

A média é a soma de todos os valores em um conjunto de dados dividida
pelo número de valores. É a medida mais comum de tendência central e é
útil para dados contínuos. No entanto, é sensível a valores extremos,
que podem distorcer o resultado.

A mediana é o ``valor do meio'' em um conjunto de dados quando eles são
colocados em ordem crescente ou decrescente. É uma medida robusta de
tendência central, ou seja, é menos afetada por valores extremos do que
a média. A mediana é útil para dados com valores extremos ou
distribuição assimétrica. Para calcular a mediana organizam-se os dados
de forma crescente ou decrescente. Esta lista é o ROL de dados. Após,
verificamos se a quantidade de dados no ROL é par ou ímpar.

\begin{itemize}
\item
  Se a quantidade de dados no ROL é ímpar, a mediana é o valor do meio,
  da posição central.
\item
  Se a quantidade de dados no ROL é par, a mediana é a média aritmética
  dos valores centrais.
\end{itemize}

Em geral, a escolha da medida de tendência central depende da natureza
dos dados e do objetivo da análise estatística. Cada medida pode ser
útil em diferentes contextos e situações.
}

\colorsec{Atividades}

\begin{enumerate}

\tightlist
\item
  Complete a frase
\end{enumerate}

João no 1º trimestre tirou as notas parciais 5,0 na prova 1 e 3,0 na
prova 2. Portanto sua média foi \_\_\_. No entanto, se tivesse tirado
\_\_\_ na segunda prova, sua média seria 6,0.

BNCC: EF09MA21 SAEB: 9E1.4

A média é: \(\frac{5 + 3}{2} = 4\). Com a segunda prova igual a 6,0 a
média seria \(\frac{5 + 6}{2} = 5,5\).

2. Carlos e Anderson tiveram a mesma nota na disciplina de Álgebra.
Carlos tirou as seguintes notas7,0 e 5,0. Anderson tirou 8,0 em uma das
provas, qual foi a nota na segunda prova?

Deixar 2 linhas

BNCC: EF09MA21 SAEB: 9E1.4

A média de Carlos é: \(\frac{7 + 5}{2} = 6\). Anderson em uma das provas
tirou 8,0. A média de Anderson também é 6, sendo assim temos que
\(\frac{x + 8}{2} = 6 \rightarrow x + 8 = 12 \rightarrow x = 4\).

3. Construa um gráfico de setor com base na tabela a seguir,
representando-a em porcentagem.

\begin{longtable}[]{@{}ll@{}}
\toprule\noalign{}
\textbf{Marca} & \textbf{Pessoas} \\
\midrule\noalign{}
\endhead
\bottomrule\noalign{}
\endlastfoot
A & 30 \\
B & 70 \\
C & 45 \\
D & 55 \\
\end{longtable}

Deixar espaço de 5 linhas

BNCC: EF09MA22 SAEB: 9E1.4

\begin{figure}
\centering
\includegraphics[width=2.41667in,height=2.43637in]{./_SAEB_9_MAT/media/image213.png}
\caption{Gráfico, Gráfico de pizza Descrição gerada automaticamente}
\end{figure}

4. Uma loja fez um levantamento para saber como estava distribuído o
atendimento dos clientes ao longo do dia para poder decidir sobre novas
contratações para atender o horário com maior fluxo de clientes.

\begin{figure}
\centering
\includegraphics[width=3.32598in,height=2.58268in]{./_SAEB_9_MAT/media/image214.jpg}
\caption{1632731228879\_1622599524064\_3568a1d3-2f50-427b-8eea-c95bc06e8bfe.jpg}
\end{figure}

Monta figura semelhante

Qual o horário que deve ter mais funcionários para atender?

BNCC: EF09MA22 SAEB: 9E1.4

Dado a distribuição dos horários é perceptível que a grande maioria dos
clientes frequentam a loja das 8h às 12h.

5. Foi realizada uma pesquisa sobre a possibilidade de alterar o horário
de início das aulas de uma grande escola das 7h30 para às 7h. 3700
famílias responderam a pesquisa e tiveram ruas respostas tabuladas e
colocadas em um gráfico de setores.

\begin{figure}
\centering
\includegraphics[width=3.34677in,height=1.96063in]{./_SAEB_9_MAT/media/image215.jpg}
\caption{1632731350522\_1622599544538\_67a4b9ce-09b6-4c36-83bb-b178065a1133.jpg}
\end{figure}

Fazer uma nova imagem

Quantas pessoas concordaram com a mudança?

a) 925

b) 814

c) 740

d) 555

BNCC: EF09MA22 SAEB: 9E2.1

Precisamos calcular 22\% de 3700 = 0,22 · 3700 = 814 pessoas.

6. Em uma aula de Matemática os estudantes receberam a tarefa de medir a
largura da sala. Após a medição, os estudantes colocaram os resultados
encontrados na lousa.

5,35 5,21 5,74 5,19 4,96 5,80 5,56 5,94 5,56 4,98 6,10 5,38 5,74 5,65

O professor ao perceber que os alunos não foram muito precisos em suas
medidas propôs uma nova proposta, perguntando quais são os valores
relativos à moda e mediana?

Deixar 4 linhas

BNCC: EF09MA22 SAEB: 9E1.5

Primeiramente precisamos organizar o rol de informações

4,96 4,98 5,19 5,21 5,35 5,38 5,56 5,56 5,58 5,74 5,74 5,74 5,94 6,1

Com isso podemos facilmente perceber que a moda é 5,74 , porque se
repete mais vezes. Como temos uma quantidade par de elementos, pois
temos 14 medições a mediana é dada pela média aritmética entre o 7º e 8º
elemento do rol, que neste caso ambos tem o valor 5,56 , sendo assim a
mediana é 5,56.

7. Uma cidade investiu em tratamento de água de chuva. A análise da água
tradada revelou os dados apresentados no gráfico da quantidade em m³ x
mês.

\begin{figure}
\centering
\includegraphics[width=4.95586in,height=2.50694in]{./_SAEB_9_MAT/media/image216.jpg}
\caption{1632731306179\_1622601258507\_1318b98e-1363-40cc-af30-a8e7cb6c0ecc.jpg}
\end{figure}

Fazer novo gráfico

\begin{enumerate}

\setcounter{enumi}{1}
\tightlist
\item
  Qual mês registrou a menor quantidade e qual registrou a maior
  quantidade de água de chuva tratada nesse período?
\end{enumerate}

Deixar 1 linha

\begin{enumerate}

\setcounter{enumi}{2}
\tightlist
\item
  Sabendo que em junho foram tratados 6 m³ de água, qual a porcentagem
  de aumento para o mês de setembro?
\end{enumerate}

Deixar 3 linhas

BNCC: EF09MA22 SAEB: 9E2.3

\begin{enumerate}

\setcounter{enumi}{3}
\item
  A menor quantidade foi registrada no mês de maio com 2m³ e a maior foi
  em outubro com 11m³.
\item
  Como em junho foram tratados 6 m³ de água e em setembro foram tratados
  10 m³ tivemos um aumento de 4 m³ o que representa um aumento
  aproximado de 66,67\%
\end{enumerate}

8. O gráfico a seguir representa a quantidade de casas que uma
imobiliária alugou em cada mês em uma determinada cidade.

\begin{figure}
\centering
\includegraphics[width=3.21165in,height=1.77953in]{./_SAEB_9_MAT/media/image217.jpg}
\caption{1632730655835\_1622599672828\_f1b500eb-5b86-4a39-b7df-d128c164df02.jpg}
\end{figure}

Montar gráfico semelhante.

Qual é a mediana da quantidade de casas alugadas?

BNCC: EF09MA22 SAEB: 9E1.5

Primeiramente precisamos organizar o rol de informações

10 , 15 , 15 , 20 , 30 , 35 . Por ter 6 elementos a mediana é
determinada pela média aritmética entre os elementos centrais que são 15
e 20, ou seja, a mediana é dada pelo valor 17,5.

9. Uma auditoria de uma montadora do grande ABC paulista levantou os
dados da tabela:

\begin{figure}
\centering
\includegraphics[width=3.41667in,height=0.85417in]{./_SAEB_9_MAT/media/image218.png}
\caption{Tabela Descrição gerada automaticamente}
\end{figure}

Montar tabela

O gráfico que melhor representa a quantidade da produção vendida por
essas três montadoras é:

Montar os gráficos de cada alternativa, importante os títulos dos eixos

a)

\begin{figure}
\centering
\includegraphics[width=1.85016in,height=1.57514in]{./_SAEB_9_MAT/media/image219.png}
\caption{Gráfico, Gráfico de barras Descrição gerada automaticamente}
\end{figure}

b)

\begin{figure}
\centering
\includegraphics[width=1.84183in,height=1.43346in]{./_SAEB_9_MAT/media/image220.png}
\caption{Gráfico, Gráfico de barras Descrição gerada automaticamente}
\end{figure}

c)

\begin{figure}
\centering
\includegraphics[width=2.01684in,height=1.90017in]{./_SAEB_9_MAT/media/image221.png}
\caption{Gráfico, Gráfico de barras Descrição gerada automaticamente}
\end{figure}

d)

\begin{figure}
\centering
\includegraphics[width=1.89183in,height=2.26686in]{./_SAEB_9_MAT/media/image222.png}
\caption{Gráfico, Gráfico de barras Descrição gerada automaticamente}
\end{figure}

BNCC: EF09MA21 SAEB: 9E1.2

A alternativa que apresenta os valores correspondente ao da tabela é a
alternativa C. É importante verificar que o gráfico representa a
quantidade vendida e não a quantidade produzida. Caso seja marcada a
alternativa B ficará perceptível que foi levado em consideração a
quantidade produzida.

10. Uma loja de queijos fez a contagem de queijos canastra vendidos ao
longo de 2022 em forma do pictograma abaixo:

\begin{figure}
\centering
\includegraphics[width=2.97917in,height=1.71639in]{./_SAEB_9_MAT/media/image223.png}
\caption{Uma imagem contendo Texto Descrição gerada automaticamente}
\end{figure}

Refazer o gráfico, sugiro usar a imagem de queijos meia cura.

Qual a quantidade de queijos vendidos por mês?

BNCC: EF09MA21 SAEB: 9E2.1

Setembro: 200 + 100 = 300

Outubro: 3 · 200 + 50 = 650

Novembro: 2 · 200 + 150 = 550

Dezembro: 4 · 200 = 800

\colorsec{Treino}

\begin{enumerate}

\setcounter{enumi}{5}
\tightlist
\item
  Um studio de danças fez pesquisa sobre as suas turmas
\end{enumerate}

\begin{figure}
\centering
\includegraphics[width=3.04861in,height=1.34722in]{./_SAEB_9_MAT/media/image224.png}
\caption{Tabela Descrição gerada automaticamente}
\end{figure}

Fazer nova tabela

Qual a quantidade de alunos que fazem Ioga ou Dança no período da tarde?

a) 15

b) 8

c) 23

d) 30

BNCC: EF09MA21 SAEB: 9E1.2

Pela tabela temos que 15 alunos fazem Dança no período da tarde e 8
fazem Ioga no período da tarde. Sendo assim a reposta correta é 23,
alternativa C.

2. Veja o gráfico sobre formas de pagamento

\begin{figure}
\centering
\includegraphics[width=3.75in,height=2.34772in]{./_SAEB_9_MAT/media/image225.jpeg}
\caption{Pix e celular empurram comércio on-line \textbar{} Empresas
\textbar{} Valor Econômico}
\end{figure}

\url{https://s2.glbimg.com/cULAdhFJ_T233T8LcZE3axW8tAA=/984x0/smart/filters:strip_icc()/i.s3.glbimg.com/v1/AUTH_63b422c2caee4269b8b34177e8876b93/internal_photos/bs/2022/7/A/DB4wyNRGAeb1vhvBiDJw/arte25emp-101-market-b6.jpg}
\textless Acessado 15/03/2023 -- Jornal Valor Econômico\textgreater{}

Quais dos meios de pagamento apresenta maior perspectiva de crescimento?

a) Cartão de crédito

b) Pix

c) Boleto

d) Wallets

BNCC: EF09MA21 SAEB: 9E2.2

Analisando os dados pode-se perceber que a expectativa de crescimento de
pagamento por pix apresenta um crescimento esperado acima de 230 \%,
enquanto os outros apresentam crescimento aproximados:

Cartão: 69\%, Wallets 77\% e Boleto 47\%.

3. O uso de internet pelo celular tem crescido significativamente nos
últimos anos, impulsionado pela popularidade cada vez maior dos
smartphones e pela melhoria da infraestrutura de telecomunicações. Isso
permite que as pessoas acessem a internet em qualquer lugar e a qualquer
momento, facilitando o acesso à informação e a comunicação.

\begin{figure}
\centering
\includegraphics[width=3.50825in,height=2.33333in]{./_SAEB_9_MAT/media/image226.png}
\caption{Gráfico, Gráfico de linhas Descrição gerada automaticamente}
\end{figure}

Disponível em
\url{https://g1.globo.com/economia/tecnologia/noticia/2019/08/28/uso-da-internet-no-brasil-cresce-e-70percent-da-populacao-esta-conectada.ghtml}
\textless acessado em 15/03/2023\textgreater{}

Baseado nas informações e no gráfico podemos inferir que:

a) Estão sem alterações as escolhas para acessar internet.

b) O uso de computadores para acesso a internet tem crescido de maneira
considerável.

c) O uso de celulares como único meio de acesso a internet é o que mais
cresce segundo o gráfico.

d) Todos os meios de acesso a internet estão estáveis desde 2014.

BNCC: EF09MA21 SAEB: 9E2.2

O gráfico apresenta um crescimento da curva de acesso exclusivo pelo
celular quando comparado com os outros meios de acesso.

Alternativa c.

\pagestyle{mat}
\chapter{Unidades de Medida}
\markboth{Módulo 14}{}

\colorsec{Habilidades do SAEB}

\begin{itemize}

  \item Resolver problemas que envolvam medidas de grandezas (comprimento,
massa, tempo, temperatura, capacidade ou volume) em que haja
conversões entre unidades mais usuais. 
  \item Resolver problemas que envolvam perímetro de figuras planas. 
  \item Resolver problemas que envolvam área de figuras planas. 
  \item Resolver problemas que envolvam volume de prismas retos ou cilindros
retos.  

\end{itemize} 

\colorsec{Habilidades da BNCC}

\begin{itemize}
  \item EF09MA18, EF09MA19. 
\end{itemize}

\conteudo{...}

As unidades de medidas de comprimento surgem para padronizar as medidas
de distância. Existem várias unidades de medidas de comprimento, a
utilizada no sistema internacional de unidades é o metro, e seus
múltiplos (quilômetro, hectômetro e decâmetro) e submúltiplos
(decímetro, centímetro milímetro).

Além das unidades de medidas de comprimento apresentadas, existem outras
como as que utilizam o corpo como parâmetro: o palmo, o pé, a polegada.
Ainda, há aquelas que não são do sistema internacional, mas são
utilizadas a depender da região ou de medidas astronômicas, como a
légua, a jarda, a milha e o ano-luz."

\ul{Mudança de unidade de medida linear}

\begin{figure}
\centering
\includegraphics[width=1.63366in,height=0.76567in]{./_SAEB_9_MAT/media/image227.jpeg}
\caption{Interface gráfica do usuário Descrição gerada automaticamente}
\end{figure}

Refazer imagem

\ul{Mudança de unidades de medida: volume e capacidade}

\begin{enumerate}

\tightlist
\item
  Volume
\end{enumerate}

\includegraphics[width=2.87318in,height=0.92079in]{./_SAEB_9_MAT/media/image228.png}

Refazer imagem

\begin{enumerate}

\setcounter{enumi}{1}
\tightlist
\item
  Capacidade
\end{enumerate}

\begin{figure}
\centering
\includegraphics[width=2.73657in,height=1.06931in]{./_SAEB_9_MAT/media/image229.jpeg}
\caption{Tabela Descrição gerada automaticamente com confiança média}
\end{figure}

Refazer imagem

IMPORTANTE

1 m³ = 1 000 Litros

1 cm³ = 1 mL

1 dm³ = 1 Litro

\colorsec{Atividades}

1. Nanômetro (nm) é uma unidade de medida que equivale a um bilionésimo
de 1 metro e que tem grande relevância na indústria de semicondutores.
Essa é a escala usada para medir dimensões no interior de qualquer
microchip: módulos de memória, SSDs, processadores e GPUs.

Disponível em:
\url{https://www.techtudo.com.br/noticias/2016/10/o-que-sao-nanometros-e-por-que-eles-sao-tao-importantes-na-tecnologia.ghtml}
\textless Acessado em 14/03/2023\textgreater.

Represente o valor de 250 nanômetros em m.

Deixar 1 linha

BNCC: EF09MA18 SAEB: 9M2.1

Como um nanômetro é a bilionésima parte de 1 metro podemos calcular que
250 nanômetros é o mesmo que \(250 \cdot 10^{- 9} = 0,00000025\) m ou
\(2,5 \cdot 10^{- 7}\) m.

2. Observe o infográfico abaixo:

\begin{figure}
\centering
\includegraphics[width=4.24063in,height=5.13386in]{./_SAEB_9_MAT/media/image230.jpg}
\caption{1632730699587\_1622599822531\_c9b273b3-95a9-4823-810e-4142d64b374f.jpg}
\end{figure}

Fazer figura semelhante

Baseado nos dados abaixo e sabendo que o vírus Influenza mede 100 nm, o
vírus da febre amarela mede 20 nm e Staphylococcus que mede 1 000 nm.
Destes elementos citados quais podem ser vistos apenas por microscópio
eletrônico e quais podem também ser vistos por microscópico óptico?

Deixar 1 linha

~BNCC: EF09MA18 SAEB: 9M2.1

O vírus Influenza e de febre amarela pela escala apenas no microscópico
eletrônico, mas o Staphyloccus pode ser visto em ambos, pois tem mais de
100 nm.

3. Uma distância muito famosa e a distância entre a Terra e a Lua que é
384 400 km. Qual é esta medida em metros?

Deixar 1 linha

~BNCC: EF09MA18 SAEB: 9M2.1

Para transformar medidas de km para m basta multiplicar por 1 000, ou
seja, a distância entre a Terra e a Lua é 384 400 000 metros.

4. Uma estrela famosa é a Próxima Centauri, a estrela mais próxima do
sistema solar. Ela está localizada na constelação de Centaurus e tem uma
distância de cerca de 4,2 anos-luz da Terra. Próxima Centauri é uma
estrela anã vermelha, com uma magnitude aparente de 11, e é parte do
sistema estelar triplo Alpha Centauri. É conhecida por abrigar um
planeta potencialmente habitável.

Um ano-luz é a distância que a luz viaja no vácuo durante um ano
juliano, que tem uma duração média de 365,25 dias. Essa distância é de
cerca de 9,46 trilhões de quilômetros ou 5,88 trilhões de milhas. O
ano-luz é uma unidade de medida usada principalmente em astronomia para
descrever distâncias entre estrelas e galáxias.

A distância da Próxima Centauri em relação a Terra corresponde a quantos
dias completos?

Deixar 3 linhas

~BNCC: EF09MA18 SAEB: 9M2.1

Um ano-luz é a distância percorrida pela luz em um ano juliano, que tem
365,25 dias. Portanto, para calcular o número de dias em 4,2 anos-luz,
podemos multiplicar 4,2 por 365,25:

4,2 anos-luz · 365,25 dias/ano = 1 534,05 dias

Então, 4,2 anos-luz correspondem a aproximadamente 1 534 dias inteiros.

5. Observando uma lata de suco, verificamos que seu conteúdo é de 355
ml. Esta mesma quantidade pode ser expressa em \_\_\_ litro(s). Complete
a a frase anterior.

~BNCC: EF09MA19 SAEB: 9M2.1

Como 1 litro é o mesmo que 1 000 ml, o valor que completa a frase é
0,355 .

6. Junior fez uma piscina em sua casa, sendo que a piscina tem 1,5 metro
de profundidade, 3 metros de comprimento e 2 de largura. Caso decida por
colocar água até a altura de 1,2 m para evitar desperdício de água,
quantos litros serão necessários?

Deixar 3 linhas

~BNCC: EF09MA19 SAEB: 9M2.1

Para calcular o volume de água necessário basta multiplicar as três
medidas 3 · 2 · 1,2 = 7,2 m³ = 7 200 litros, pois 1 m³ = 1 000 litros.

7. Antigamente as embalagens que armazenavam óleo de soja eram
cilíndricas e de metal. Sabendo que na lata cabiam 900 ml, considerando
que o raio era de 3 cm e considerando \(\pi = 3\), calcule a altura
mínima necessária.

Deixar 3 linhas

~BNCC: EF09MA19 SAEB: 9M2.1

\[V = 900\ ml\]

\(900 = 3 \cdot 3^{2} \cdot h \rightarrow h = \frac{900}{27} \cong 33,3\ cm\)

8. Uma caixa d\textquotesingle água em formato de bloco retangular tem
capacidade para 1 500 L. Considerando-se que ela seja completamente
cheia 3 vezes por dia, quantos metros cúbicos de água são usados para
enchê-la durante um mês de 30 dias?

Deixar 3 linhas

~BNCC: EF09MA19 SAEB: 9M2.1

1 500 L = 1,5 m³. Como a caixa é completamente cheia 3 vezes ao dia por
30 dias, temos:

1,5 · 3 · 30 = 135 m³.

9. Calcule a área do polígono:

\begin{figure}
\centering
\includegraphics[width=1.77871in,height=1.52778in]{./_SAEB_9_MAT/media/image231.jpg}
\caption{1632731864945\_1622599638003\_99dff173-8647-4b54-9717-4e0fcaca2a1b.jpg}
\end{figure}

Deixar 4 linhas.

Fazer figura semelhante

~BNCC: EF09MA19 SAEB: 9M2.3

Para determinar a área vamos dividir a figura.

\begin{figure}
\centering
\includegraphics[width=1.65556in,height=1.32674in]{./_SAEB_9_MAT/media/image232.png}
\caption{Gráfico Descrição gerada automaticamente}
\end{figure}

A área do triângulo pode ser calculada por:

\(\frac{6 \cdot 3}{2} = 9\) cm²

A área do retângulo: 6 · 2 = 12 cm², somando área 1 com área 2 temos: 21
cm².

10. Janaina quer fazer uma festa para comemorar seu aniversário. Em suas
contas em média as pessoas irão consumir em média dois copos de 300 ml,
sabendo que Janaina tem a expectativa de 70 pessoas, quantas garrafas de
2 litros ela deverá comprar?

Deixar 3 linhas

~BNCC: EF09MA19 SAEB: 9M2.1

Calculando a expectativa de pessoas temos:

70 · 2 · 300 = 42 000 ml = 42 litros = 21 garrafas de 2 litros.

\colorsec{Treino}

1. Para fazer um determinado prato, Alessandra precisa de 1 kg de carne
para cada receita. Ao tirar o pacote de carne da geladeira, vê que ele
tem apenas 625 gramas. De quantos gramas de carne ela ainda precisa para
fazer duas receitas?

a) 1375 gramas.

b) 1750 gramas.

c) 950 gramas.

d) 967 gramas.

~BNCC: EF09MA19 SAEB: 9M2.1

Para fazer duas receitas, problema vemos que se consideremos Que serão
necessários 2kg, mas por hora temos 625 g, ou 0,625 kg. Desta forma
falta:

2 000 -- 625 = 1375 gramas faltando.

Alternativa A.

2. Uma piscina recebe tratamento diário com um produto na proporção de
50 gramas para cada 1 000 litros. A piscina tem as seguintes medidas:
1,0 m x 2,0 m x 3,5 m. Qual o volume o piscineiro deve levar em
consideração para colocar o produto na piscina e qual a quantidade de
produto respectivamente ele deve utilizar?

\begin{figure}
\centering
\includegraphics[width=2.7in,height=2.17742in]{./_SAEB_9_MAT/media/image233.jpeg}
\caption{A Swimming pool. 3D rendered Illustration. Isolated on white.}
\end{figure}

\url{https://www.shutterstock.com/pt/image-illustration/swimming-pool-3d-rendered-illustration-isolated-73423618}

a) 5,5 m³, 275 g.

b) 7 m³, 350 g.

c) 6,5 m³, 325 g.

d) 3,5 m³, 700 g.

~BNCC: EF09MA19 SAEB: 9M2.4

O volume da piscina é 1 · 2 · 3,5 = 7 m³, ou seja, 7 000 litros. O
produto é usado na proporção de 50 gramas a cada 1 000 litros, sendo
assim, precisará de 7 · 50 = 350 gramas.

Alternativa B.

3. Uma caixa em forma de cilindro reto foi instalada em um condomínio
para resolver um problema de falta d'água. Percebeu-se no projeto que os
engenheiros calcularam errado a demanda de uso de água do condomínio e
que seria necessário mais um reservatório de 20 000 litros. Sabendo-se
que a caixa tem 1,5 metros de raio e altura de 3 metros, podemos afirmar
que após a instalação a demanda:

\begin{figure}
\centering
\includegraphics[width=3.29444in,height=2.91667in]{./_SAEB_9_MAT/media/image234.jpeg}
\caption{Water tank vector. wallpaper. water tank on white background.
free space for text. copy space.}
\end{figure}

\url{https://www.shutterstock.com/pt/image-vector/water-tank-vector-wallpaper-on-white-1346364413}

a) foi atendida e sobrou 250 litros.

b) foi atendida e sobrou 1 195 litros.

c) não foi atendida, pois faltou 250 litros.

d) não foi atendida, pois faltou 13 250 litros.

~BNCC: EF09MA19 SAEB: 9M2.4

Calculando o volume do cilindro instalado como reservatório temos:

\(V = {1,5}^{2} \cdot 3,14 \cdot 3 = 21,195\ m³\), ou seja, 21 195
litros, o que atendeu a demanda com a sobra de 1 195 litros. Alternativa
B.

\pagestyle{mat}
\chapter{Probabilidade}
\markboth{Módulo 15}{}

\colorsec{Habilidades do SAEB}

\begin{itemize}

  \item Resolver problemas que envolvam a probabilidade de ocorrência de um
resultado em eventos aleatórios equiprováveis independentes ou
dependentes.   

\end{itemize}

\colorsec{Habilidade da BNCC}

\begin{itemize}
  \item EF09MA20. 
\end{itemize}

\conteudo{
Probabilidade é o estudo das chances de obtenção de cada resultado de um
experimento aleatório. A essas chances são atribuídos os números reais
do intervalo entre 0 e 1. Resultados mais próximos de 1 têm mais chances
de ocorrer. Além disso, a probabilidade também pode ser apresentada na
forma percentual.

Usamos com notação para probabilidade de ocorrer um evento A como sendo
P(A) e pela definição este valor está entre 0 e 1 incluindo estes.

\ul{Experimento aleatório}: É qualquer experiência cujo resultado não
seja conhecido, por exemplo, observar a face voltada para cima de um
dado, ou de uma moeda lançada. \ul{Espaço Amostral}: Conjunto formado
por todos os resultados possíveis.

\ul{Evento}: É um subconjunto do espaço amostral.

\ul{Cálculo da probabilidade}

Seja um evento A, a probabilidade de A ocorrer é dado por:

\[P\left( A \right) = \frac{Quantidade\ de\ casos\ favoráveis\ ao\ evento}{\text{Total\ de\ possibilidades}}\]

Exemplo:

Qual a probabilidade em um lançamento de um dado honesto sair um número
maior que 1?

Evento: Sair número maior que 1, \{2, 3, 4, 5 e 6\}

Espaço Amostral: \{1, 2, 3, 4, 5 e 6\}

\(P\left( A \right) = \frac{5}{6}\).
}

\colorsec{Atividades}

1. Um dado foi construído com duas faces 4 e nenhuma face 3. Ao lançar o
dado qual a probabilidade de sair o número 4?

Deixar duas linhas

~BNCC: EF09MA19 SAEB: 9E2.4

Para calcular esta probabilidade temos que levar em consideração que
entre as 6 faces tem duas delas com o número 4, sendo assim,
\(P\left( A \right) = \frac{2}{6} = 33,3\%.\)

2. Lançando um dado e uma moeda, qual a probabilidade de obter Cara e um
número maior que 2?

Deixar duas linhas

~BNCC: EF09MA19 SAEB: 9E2.4

Para calcular esta probabilidade temos que levar em consideração que
entre as 6 faces tem 4 delas com o número maior que 2 e a chance de sair
cara na moeda é de 1 pra 2, sendo assim, seja A o evento sair número
maior que 2 no lançamento do dado e B sair cara no lançamento da moeda,
temos que: \(P\left( A \right) = \frac{4}{6} = \frac{2}{3}\),
\(P\left( B \right) = \frac{1}{2}\) e a chance de ocorrer A e B é dada
pelo produto das probabilidade de cada evento:
\(P\left( A \right) \cdot P\left( B \right) = \frac{2}{3} \cdot \frac{1}{2} = \frac{1}{3}.\)

3. A \_\_\_\_\_\_\_\_\_\_\_\_\_\_\_ de um
\_\_\_\_\_\_\_\_\_\_\_\_\_\_\_\_\_\_\_ depende diretamente do número de
elementos do seu \_\_\_\_\_\_\_\_\_\_\_\_\_\_\_\_\_. O evento que ocorre
sem interferência de fatores externos é dito
\_\_\_\_\_\_\_\_\_\_\_\_\_\_\_\_\_\_\_\_\_\_\_\_\_\_\_\_\_.

Complete a frase acima:

Aleatório Espaço Amostral Evento Probabilidade

~BNCC: EF09MA19 SAEB: 9E2.4

A frase completa é:

A \textbf{probabilidade} de um \textbf{evento} depende diretamente do
número de elementos do seu \textbf{espaço amostral}. O evento que ocorre
sem interferência de fatores externos é dito \textbf{aleatório}.

4. Relacione as duas colunas

~

\begin{longtable}[]{@{}ll@{}}
\toprule\noalign{}
~(1) Evento & ( ) É o conjunto finito composto por todas as
possibilidades de ocorrência de um evento. \\
\midrule\noalign{}
\endhead
\bottomrule\noalign{}
\endlastfoot
(2) Espaço amostral & ( ) É um subconjunto do espaço amostral. \\
(3) Aleatório & ( ) Sempre é um número real positivo pertencente ao
intervalo {[}0,1{]}. \\
(4) Probabilidade & ( ) Não sofre influência de fatores externos ao
evento. \\
\end{longtable}

~~BNCC: EF09MA19 SAEB: 9E2.4

A correspondência em ordem é:

2 -- 1 -- 4 -- 3

5. Fernando viajou e levou na mochila apenas 5 camisetas distintas e 3
calças, sendo estas azul, preta e bege.

Sabendo que Fernando já escolheu a camiseta, mas está em dúvida sobre a
calça. Qual a probabilidade de que ele escolha a calça azul?

~~BNCC: EF09MA19 SAEB: 9E2.4

Dado que a camiseta foi escolhida não precisamos levar em consideração
esta escolha para determinar a probabilidade da calça a ser escolhida.
São 3 possibilidades e 1 escolha, ou seja,
\(P\left( A \right) = \frac{1}{3}\).

6. Analisando as informações sobre a qualidade de determinado produto de
iluminação foi gerada a tabela:

\begin{longtable}[]{@{}
  >{\raggedright\arraybackslash}p{(\columnwidth - 6\tabcolsep) * \real{0.1250}}
  >{\raggedright\arraybackslash}p{(\columnwidth - 6\tabcolsep) * \real{0.2361}}
  >{\raggedright\arraybackslash}p{(\columnwidth - 6\tabcolsep) * \real{0.2083}}
  >{\raggedright\arraybackslash}p{(\columnwidth - 6\tabcolsep) * \real{0.2361}}@{}}
\toprule\noalign{}
\begin{minipage}[b]{\linewidth}\raggedright
\end{minipage} & \begin{minipage}[b]{\linewidth}\raggedright
\end{minipage} & \begin{minipage}[b]{\linewidth}\raggedright
INTENSIDADE
\end{minipage} & \begin{minipage}[b]{\linewidth}\raggedright
\end{minipage} \\
\midrule\noalign{}
\endhead
\bottomrule\noalign{}
\endlastfoot
& & SATISFATÓRIA & INSATISFATÓRIA \\
\begin{minipage}[t]{\linewidth}\raggedright
\begin{quote}
VIDA

ÚTIL
\end{quote}
\end{minipage} & SATISFATÓRIA & & \\
& INSATISFATÓRIA & & \\
\end{longtable}

~Ao escolher um produto ao acaso, qual a probabilidade dele ser
reprovado na Intensidade e na Vida útil?

\begin{figure}
\centering
\includegraphics[width=1.86667in,height=1.86667in]{./_SAEB_9_MAT/media/image239.jpeg}
\caption{Balcão de vidro Descrição gerada automaticamente com confiança
média}
\end{figure}

https://labs.openai.com/s/MH4cirXDETwaVt7t4vCK3esf

BNCC: EF09MA19 SAEB: 9E2.4

De acordo com a tabela temos um total de 117 + 8 + 3 + 2 = 130 produtos
avaliados, deste apenas 2 estão na coluna INSATISFATÓRIA dos dois
fatores, sendo assim temos que a probabilidade será:
\(P\left( A \right) = \frac{2}{130}\).

7. Em um estudo laboratorial envolvendo cobaias para estudar os fatores
genéticos que determinavam cada tipo de cor de pelos, os estudantes
tomaram nota destes dados:

\begin{longtable}[]{@{}llllll@{}}
\toprule\noalign{}
\textbf{Números de descendentes} & & & & & \\
\midrule\noalign{}
\endhead
\bottomrule\noalign{}
\endlastfoot
Ninhada & Preto & Marrom & Creme & Albino & Total \\
1ª & 5 & 3 & 0 & 2 & 10 \\
2ª & 0 & 4 & 2 & 2 & 8 \\
3ª & 0 & 5 & 0 & 4 & 9 \\
Total & 5 & 12 & 2 & 8 & 27 \\
\end{longtable}

Sabendo que as cobaias estão todas juntas, ao escolher ao acaso um
descendente Albino, qual a probabilidade que ele seja da 1ª ninhada?

Deixar 3 linhas

BNCC: EF09MA19 SAEB: 9E2.4

As cobaias albinas são em seu total 8, como na 1ª ninhada temos 2
cobaias a probabilidade será: \(P\left( A \right) = \frac{2}{8} = 25\%\)

8. Em uma escola a escolha do representante de sala é feita por sorteio.
Em uma turma em que temos 12 meninas e 20 meninos, qual a probabilidade
de o representante ser do sexo feminino?

Deixar 2 linhas

BNCC: EF09MA19 SAEB: 9E2.4

Para determinar esta probabilidade devemos estabelecer a seguinte razão:

\(P\left( A \right) = \frac{12}{32} = 37,5\%\).

~9. Uma empresa premia seu funcionário quando atingem a meta de
trabalho, mas devido a crise foi necessário limitar a premiação. Este
ano será apenas 1 smartphones para ser entregue, ficou combina que caso
mais funcionários atinjam a meta, o smartphone será sorteado.

~Ao terminar a apuração foi verificado que 5 funcionários atingiram a
meta, entre eles, João Pedro. Ao realizar o sorteio, qual a chance de
João Pedro não ser um o sorteado?

BNCC: EF09MA19 SAEB: 9E2.4

João Pedro tem 1 chance em 5 possíveis, ou seja, a chance de João ganhar
é de \(P\left( A \right) = \frac{1}{5} = 20\%\), sendo assim, a chance
de não ganhar é de 80\%.

10. Os tipos sanguíneos são 4, como consta na tabela abaixo, sendo ainda
que existem ainda dois fatores o Rh+ e Rh-. As pessoas do tipo O com
Rh-- são consideradas doadoras universais e as do tipo AB com Rh+ são
receptoras universais.

\begin{longtable}[]{@{}lllll@{}}
\toprule\noalign{}
& \textbf{A} & \textbf{B} & \textbf{AB} & \textbf{O} \\
\midrule\noalign{}
\endhead
\bottomrule\noalign{}
\endlastfoot
\textbf{Rh+} & 37 & 44 & 33 & 85 \\
\textbf{Rh--} & 15 & 13 & 13 & 60 \\
\end{longtable}

~300 pessoas foram testadas e compuseram a tabela. Escolhendo uma pessoa
de grupo ao acaso, qual a chance de escolher uma doadora universal?

BNCC: EF09MA19 SAEB: 9E2.4

Segundo o enunciado o doador universal é quem é do tipo O com Rh-. Sendo
assim temos que:

\[P\left( A \right) = \frac{60}{300} = \frac{1}{5}.\]

\colorsec{Treino}

\begin{enumerate}
\def\labelenumi{\arabic{enumi}.}
\tightlist
\item
  Numa cidade, 56\% dos habitantes são mulheres. Destas, 2,8\% têm olhos
  azuis e 2,2\% dos homens, olhos da mesma cor. A probabilidade de uma
  pessoa nessa cidade, escolhida ao acaso, ter olhos azuis é cerca de:
\end{enumerate}

\begin{enumerate}

\item
  0,6\%
\item
  1,4\%
\item
  2,0\%
\item
  2,5\%
\end{enumerate}

SAEB: 9E2.4 - Resolver problemas que envolvam a probabilidade de
ocorrência de um resultado em eventos aleatórios equiprováveis
independentes ou dependentes. BNCC: EF09MA20 - Reconhecer, em
experimentos aleatórios, eventos independentes e dependentes e calcular
a probabilidade de sua ocorrência, nos dois casos.

Calculando as probabilidades de ser mulher e olhos azuis ou ser homem
com olhos azuis temos:

\(P\left( A \right) = \frac{56}{100} \cdot \frac{28}{1000} + \frac{44}{100} \cdot \frac{22}{1000} = 2,5\%\).

Alternativa d.

\begin{enumerate}
\def\labelenumi{\arabic{enumi}.}
\setcounter{enumi}{1}
\tightlist
\item
  Um caixa eletrônico de certo banco dispõe apenas de cédulas de 20 e 50
  reais. No caso de um saque de 400 reais, a probabilidade do número de
  cédulas entregues ser ímpar é igual a:
\end{enumerate}

\begin{enumerate}

\item
  25\%
\item
  40\%
\item
  66\%
\item
  60\%
\end{enumerate}

SAEB: 9E2.4 - Resolver problemas que envolvam a probabilidade de
ocorrência de um resultado em eventos aleatórios equiprováveis
independentes ou dependentes. BNCC: EF09MA20 - Reconhecer, em
experimentos aleatórios, eventos independentes e dependentes e calcular
a probabilidade de sua ocorrência, nos dois casos. Há 5 maneiras de
sacar R\$ 400,00 em notas de 20 ou 50, sendo que apenas em 2 temos
quantidade ímpar de notas, veja a tabela:

\begin{longtable}[]{@{}lll@{}}
\toprule\noalign{}
\textbf{Nota de 20} & \textbf{Nota de 50} & \textbf{Total} \\
\midrule\noalign{}
\endhead
\bottomrule\noalign{}
\endlastfoot
0 & 8 & 8 \\
5 & 6 & 11 \\
10 & 4 & 14 \\
15 & 2 & 17 \\
20 & 0 & 20 \\
\end{longtable}

Desta forma a probabilidade é de 2/5 ou 40\%. Alternativa b.

\begin{enumerate}
\def\labelenumi{\arabic{enumi}.}
\setcounter{enumi}{2}
\tightlist
\item
  Em uma urna são depositadas 5 bolas vermelhas, 6 bolas azuis e 4 bolas
  amarelas, todas com mesmo formato e tamanho. Se duas bolas forem
  retiradas sucessivamente, sem reposição, a probabilidade de que elas
  sejam de mesma cor é mais próxima de:
\end{enumerate}

\begin{enumerate}

\item
  10\%
\item
  15\%
\item
  30\%
\item
  45\%
\end{enumerate}

SAEB: 9E2.4 - Resolver problemas que envolvam a probabilidade de
ocorrência de um resultado em eventos aleatórios equiprováveis
independentes ou dependentes. BNCC: EF09MA20 - Reconhecer, em
experimentos aleatórios, eventos independentes e dependentes e calcular
a probabilidade de sua ocorrência, nos dois casos.

Para atender à solicitação temos que ter duas bolas vermelhas ou ter
duas bolas amarelas ou duas bolas azuis.

\[\frac{5}{15} \cdot \frac{4}{14} + \frac{6}{15} \cdot \frac{5}{14} + \frac{4}{15} \cdot \frac{3}{14}´ = \frac{20 + 30 + 12}{210} = \frac{62}{210} = \frac{31}{105} = 29,5\%.\]

Alternativa c.

\chapter{Simulado 1}
\markboth{Simulado 1}{}

\begin{enumerate}
\def\labelenumi{\arabic{enumi}.}
\tightlist
\item
  Os números distribuídos abaixo pertencem a dois conjuntos.
\end{enumerate}

A = \(\sqrt{2}\) B = \(\frac{3}{5}\) C = \(0,454545\ldots\) D =
\(\sqrt{5}\)

A distribuição dos conjuntos pode ser feita:

\begin{enumerate}

\item
  A e B pertencem aos Naturais, C e D pertencem aos Racionais.
\item
  A e D pertencem aos Irracionais, B e C pertencem aos Racionais.
\item
  A e C, pertencem aos Irracionais, B e D pertencem aos Racionais.
\item
  A e D pertencem aos Racionais, B e C pertencem aos Irracionais.
\end{enumerate}

SAEB - 9N1.3 - números racionais ou irracionais.

EF09MA02 - Reconhecer um número irracional como um número real cuja
representação decimal é infinita e não periódica, e estimar a
localização de alguns deles na reta numérica.

Alternativa b. As raízes não exatas são irracionais, dízimas periódicas
e frações são racionais.

\begin{enumerate}
\def\labelenumi{\arabic{enumi}.}
\setcounter{enumi}{1}
\tightlist
\item
  A distância média entre a Terra e o Sol é de aproximadamente 149.6
  milhões de quilômetros. Essa distância é fundamental para a vida em
  nosso planeta, pois determina a quantidade de energia solar que
  recebemos. A Terra orbita ao redor do Sol em uma trajetória elíptica,
  o que significa que a distância entre eles varia ao longo do ano. No
  ponto mais próximo da Terra ao Sol, chamado de perigeu, essa distância
  pode ser de cerca de 147 milhões de quilômetros, enquanto no ponto
  mais distante, chamado de apogeu, pode chegar a cerca de 152 milhões
  de quilômetros. Essas variações na distância não são significativas o
  suficiente para afetar drasticamente a vida na Terra, mas podem ter
  efeitos sutis no clima e nas estações do ano.
\end{enumerate}

A distância da Terra ao Sol no apogeu pode ser representado por:

\begin{enumerate}

\item
  \(1496 \cdot 10^{6}\)km
\item
  \(14,96 \cdot 10^{6}\)km
\item
  \(1,52 \cdot 10^{8}\)km
\item
  \(152\ 000\ km\)
\end{enumerate}

SAEB -- 9N2.1 - Resolver problemas de adição, subtração, multiplicação,

divisão, potenciação ou radiciação envolvendo números reais,

EF09MA03 - Efetuar cálculos com números reais, inclusive potências com
expoentes fracionários. Segundo o texto a distância entre Terra e Sol no
apogeu era de 152 milhões de quilômetro, escrevendo em Notação
Científica \(1,52 \cdot 10^{8}\)km.

\begin{enumerate}
\def\labelenumi{\arabic{enumi}.}
\setcounter{enumi}{2}
\tightlist
\item
  Observando a malha:
\end{enumerate}

\begin{figure}
\centering
\includegraphics[width=2.23809in,height=1.77753in]{./_SAEB_9_MAT/media/image240.png}
\caption{Forma Descrição gerada automaticamente}
\end{figure}

Montar figura semelhante

Qual a fração em relação ao total da malha está pintada?

\begin{enumerate}

\item
  \(\frac{1}{5}\)
\item
  \(\frac{2}{5}\)
\item
  \(\frac{3}{5}\)
\item
  \(\frac{4}{5}\)
\end{enumerate}

SAEB 9N1.7 - Representar frações menores ou maiores que a unidade por
meio de representações pictóricas OU associar frações a representações
pictóricas.
\(\frac{\text{Quadrados\ pintados}}{\text{Total\ de\ quadrados}} = \frac{32}{80} = \frac{2}{5}\)
, com isso temos a alternativa b.

\begin{enumerate}
\def\labelenumi{\arabic{enumi}.}
\setcounter{enumi}{3}
\tightlist
\item
  Pedro tem uma dívida com o banco no valor de R\$ 6 000,00. Neste mês
  recebeu um bônus por desempenho e pagará 20\% desta dívida.
\end{enumerate}

Qual o valor que pagará ao banco?

\begin{enumerate}

\item
  R\$ 120,00
\item
  R\$ 1.000,00
\item
  R\$ 1.200,00
\item
  R\$ 2.400,00
\end{enumerate}

SAEB: 9N2.3 - Resolver problemas que envolvam porcentagens, incluindo os
que lidam com acréscimos e decréscimos simples, aplicação de percentuais
sucessivos e determinação das taxas percentuais. BNCC: EF09MA05 -
Resolver e elaborar problemas que envolvam porcentagens, com a ideia de
aplicação de percentuais sucessivos e a determinação das taxas
percentuais, preferencialmente com o uso de tecnologias digitais, no
contexto da educação financeira. Calculando 20\% de 6.000 → 0,2 · 6000 =
1200. Desta forma Pedro irá pagar R\$1.200,00 ao banco. Alternativa c.

\begin{enumerate}
\def\labelenumi{\arabic{enumi}.}
\setcounter{enumi}{4}
\tightlist
\item
  Aluísio olhou sua carteira e decidiu dar um terço do dinheiro que
  tinha nela para sua neta mais velha. Posteriormente ele deu mais 10
  reais a ela e ficou com 20 reais na carteira.
\end{enumerate}

Qual equação permite encontrar o valor que o avô Aluísio tinha na sua
carteira?

\begin{enumerate}

\item
  \(\frac{x}{3} - 10 = 20\)
\item
  \(x - \frac{x}{3} - 10 = 20\)
\item
  \(x + 10 = \frac{x}{3} - 20\)
\item
  \(\frac{x}{3} - \frac{10}{3} = 20\)
\end{enumerate}

SAEB: 9A1.2 - Inferir uma equação, inequação polinomial de 1º grau ou um

sistema de equações de 1º grau com duas incógnitas que

modela um problema.

Seja x o valor que Aluísio tinha em sua carteira, então
\(\frac{x}{3}\ \)será um terço do valor da carteira, então temos:

\[x - \frac{x}{3} - 10 = 20.\]

Alternativa b.

\begin{enumerate}
\def\labelenumi{\arabic{enumi}.}
\setcounter{enumi}{5}
\tightlist
\item
  Observe a imagem:
\end{enumerate}

\begin{figure}
\centering
\includegraphics[width=3.5in,height=0.875in]{./_SAEB_9_MAT/media/image241.png}
\caption{Diagrama Descrição gerada automaticamente}
\end{figure}

Montar imagem semelhante

Seguindo este padrão, quantos palitos estarão na figura de n = 8.

\begin{enumerate}

\item
  10
\item
  12
\item
  16
\item
  17
\end{enumerate}

SAEB: - Identificar uma representação algébrica para o padrão ou a
regularidade de uma sequência de números racionais OU representar
algebricamente o padrão ou a regularidade de uma sequência de números
racionais.

Procurando a regularidade podemos observar que:

Para n = 1 temos 3

Para n = 2 temos 5

Para n = 3 temos 7

Para n = x temos 2x + 1.

Sendo assim para n = 8 teremos 2·8 + 1 = 17 palitos.

Alternativa d.

\begin{enumerate}
\def\labelenumi{\arabic{enumi}.}
\setcounter{enumi}{6}
\tightlist
\item
  Um objeto é lançado obliquamente. Sua trajetória é descrita pela
  \textgreater{} função \(h\left( t \right) = - t^{2} + 5t\), sendo
  representa a \textgreater{} altura, em metros, e \emph{t} o tempo em
  segundos. Quantos metros de \textgreater{} altura estará o objeto após
  3 segundos do lançamento?
\end{enumerate}

a) 1 b) 2 c) 4 d) 6 SAEB: 9A2.4 - Resolver problemas que possam ser
representados por equações polinomiais de 2º grau. Para achar a altura,
basta substituir t = 3 na expressão: h(3) = - 3² + 5 · 3 = - 9 + 15 = 6,
ou seja, a altura será de 6 metros. Alternativa d.

\begin{enumerate}
\def\labelenumi{\arabic{enumi}.}
\setcounter{enumi}{7}
\tightlist
\item
  Rodrigo por ser prudente sempre controla a velocidade em suas
  \textgreater{} viagens. Em uma viagem recente entre Porto Feliz e
  Cidade Alegre \textgreater{} ele fez a viagem em uma velocidade média
  de 80 km e com isso \textgreater{} gastou 2,5 horas.
\end{enumerate}

Para voltar se Rodrigo fizer a viagem com velocidade média de 100 km/h
deverá gastar quanto tempo? a) 1h 30min b) 2h c) 2h 40min d) 3h

BNCC: EF09MA07 - Resolver problemas que envolvam a razão entre duas
grandezas de espécies diferentes, como velocidade e densidade
demográfica.

SAEB: 9A2.1 - Resolver problemas que envolvam variação de
proporcionalidade direta ou inversa entre duas ou mais grandezas,
inclusive escalas, divisões proporcionais e taxa de variação.

Rodrigo na velocidade média de 80km/h demorou 2,5 horas para completar o
percurso sendo assim ao multiplicarmos 80 por 2,5 teremos a distância
percorrida, em km.

80 · 2,5 = 200 km.

Fazendo o mesmo percurso a uma velocidade média de 100 km/h teremos:

\(t = \frac{200}{100} = 2\) horas.

Alternativa b.

9. Apesar dos carros por aplicativos ter emplacado na maioria das
cidades, ainda temos cidades que ainda tem apenas taxis convencionais.
Em um destas cidades uma corrida de táxi, é cobrado um valor inicial
fixo, chamado de bandeirada, mais uma quantia proporcional aos
quilômetros percorridos. Se por uma corrida de 10 km paga-se R\$ 34,50 e
por uma corrida de 4 km paga-se R\$ 16,50, então o valor da bandeirada é

a) R\$ 7,50.

b) R\$ 6,50.

c) R\$ 5,50.

d) R\$ 4,50.

SAEB: 9A2.5 - Resolver problemas que envolvam função afim.

BNCC: EF09MA06 - Compreender as funções como relações de dependência
unívoca entre duas variáveis e suas representações numérica, algébrica e
gráfica e utilizar esse conceito para analisar situações que envolvam
relações funcionais entre duas variáveis.

A diferença entre o valor das duas corridas é de R\$ 18,00 e a diferença
da distância percorrida entre as duas corridas é de 6 km, sendo assim
podemos dizer que o custo por km percorrido é de R\$ 3,00. Como a
corrida de 4 km custou R\$16,50 e de deslocamento gastou-se 4 · 3 = 12,
a diferença 16,50 -- 12,00 = 4,50 é o valor da bandeirada.

Alternativa d.

10. A figura a seguir mostra uma circunferência, em que os arcos ADC e
AEB são congruentes e medem 150º cada um.

\includegraphics[width=1.57812in,height=1.68157in]{./_SAEB_9_MAT/media/image243.wmf}

Refazer a figura

Qual o valor de x?

a)

b)

c)

d)

SAEB: 9G1.8 - Reconhecer circunferência/círculo como lugares
geométricos, seus elementos (centro, raio, diâmetro, corda, arco, ângulo
central, ângulo inscrito).

BNCC: EF09MA11 - Resolver problemas por meio do estabelecimento de
relações entre arcos, ângulos centrais e ângulos inscritos na
circunferência, fazendo uso, inclusive, de softwares de geometria
dinâmica.

Como arcos ADC e AEB medem 150º cada um, juntos correspondem a 300º,
sobrando um último arco de 60º que é correspondente do ângulo inscrito
de medida x. Como x é um ângulo inscrito, sua medida é metade da medida
do arco, ou seja, 30º.

Alternativa c.

11. Na imagem a seguir, as retas u, r e s são paralelas e cortadas por
uma reta t transversal.

\begin{figure}
\centering
\includegraphics[width=2.19271in,height=1.32816in]{./_SAEB_9_MAT/media/image248.jpeg}
\caption{Retas u, r e s paralelas e interceptadas por uma reta t
transversal}
\end{figure}

Qual o valor de x e y respectivamente?

\begin{enumerate}

\item
  50º e 130º.
\item
  130º e 50º.
\item
  30º e 150º.
\item
  150º e 30º.
\end{enumerate}

SAEB: 9G1.10 - Identificar relações entre ângulos formados por retas
paralelas cortadas por uma transversal.

BNCC: EF09MA10 - Demonstrar relações simples entre os ângulos formados
por retas paralelas cortadas por uma transversal.

Para determinar os valores podemos ir completando os ângulos
correspondentes.

\begin{figure}
\centering
\includegraphics[width=2.44792in,height=1.64428in]{./_SAEB_9_MAT/media/image249.png}
\caption{Imagem de vídeo game Descrição gerada automaticamente com
confiança média}
\end{figure}

Refazer a figura

Alternativa a.

12. O coordenador de uma escola de Ensino Médio fez uma pesquisa para
conhecer as carreiras que os alunos pretendem prestar no vestibular.

Após tabular os dados dos 170 estudantes entrevistados, chegou-se a
seguinte tabela:

\begin{longtable}[]{@{}lll@{}}
\toprule\noalign{}
Carreira & Masculino & Feminino \\
\midrule\noalign{}
\endhead
\bottomrule\noalign{}
\endlastfoot
Medicina & 17 & 20 \\
Direito & 8 & 16 \\
Administração & 12 & 22 \\
Fisioterapia & 8 & 16 \\
Outras & 25 & 26 \\
\end{longtable}

Um desses estudantes é escolhido ao acaso e sabe-se que ele é do sexo
feminino. A probabilidade de este estudante ter escolhido Administração
é:

a) 7\%.

b) 12,9\%.

c) 16\%.

d) 22\%.

SAEB: 9E2.4 - Resolver problemas que envolvam a probabilidade de
ocorrência de um resultado em eventos aleatórios equiprováveis
independentes ou dependentes.

BNCC: EF09MA20 - Reconhecer, em experimentos aleatórios, eventos
independentes e dependentes e calcular a probabilidade de sua
ocorrência, nos dois casos.

Como o enunciado nos orienta que o estudante é do sexo feminino para
determinar a probabilidade de ser do curso de Administração temos
primeiramente que olhar o nosso espaço amostral, pois não serão mais os
170 estudantes entrevistados, mas sim apenas os de sexo feminino.

E = 20 + 16 + 22 +16 + 26 = 100 entre estas 22 escolheram Administração,
sendo assim a probabilidade de ser de a aluna ser de Administração é
22\%.

Alternativa d.

\textbf{13. Em um concurso público, as notas finais dos candidatos foram
as seguintes:}

\begin{longtable}[]{@{}ll@{}}
\toprule\noalign{}
Número de Candidatos & Nota Final \\
\midrule\noalign{}
\endhead
\bottomrule\noalign{}
\endlastfoot
7 & 6,0 \\
3 & 7,0 \\
4 & 9,0 \\
\end{longtable}

Com base na tabela anterior, a mediana das notas finais foi:

a) 6

b) 6,5

c) 7

d) 9

SAEB: 9E1.5 - Calcular os valores de medidas de tendência central de uma
pesquisa

estatística (média aritmética simples, moda ou mediana).

Como temos 14 elementos a mediana é dada pela média aritmética do 7º e
8º termo, ou seja, entre 6 e 7. Sendo assim a mediana será 6,5.

Alternativa b.

14. Em uma piscina retangular de 9,0 m x 12,0 m com água até a altura de
1,4 m. Um produto químico em pó deve ser misturado à água à razão de um
pacote para cada 3 000 litros. O número de pacotes a serem comprados é:

a) 30

b) 36

c) 51

d) 60

SAEB: 9A2.1 - Resolver problemas que envolvam variação de
proporcionalidade direta ou inversa entre duas ou mais grandezas,
inclusive escalas, divisões proporcionais e taxa de variação.

9M2.4 - Resolver problemas que envolvam volume de prismas retos ou
cilindros

retos.BNCC: EF09MA07 - Resolver problemas que envolvam a razão entre
duas grandezas de espécies diferentes, como velocidade e densidade
demográfica.

EF09MA19 - Resolver e elaborar problemas que envolvam medidas de volumes
de prismas e de cilindros retos, inclusive com uso de expressões de
cálculo, em situações cotidianas.

Inicialmente iremos calcular o volume de água presente na piscina:

9 · 12 · 1,4 = 151,2 m³ → 151 200 litros. Calculando agora quantos
pacotes precisaremos utilizar. 151 200 ÷ 3 000 = 50,4 pacotes, ou seja,
preciso comprar 51 pacotes.

Alternativa c.

15. Numa urna, foram colocados 10 cartões numerados de 1 a 10. Serão
sorteados, sem reposição, dois cartões. Qual a probabilidade,
aproximada, de os números presentes nos cartões sorteados serem pares?

a) 30\%

b) 22\%

c) 50\%

d) 60\%

SAEB: 9E2.4 - Resolver problemas que envolvam a probabilidade de
ocorrência de um resultado em eventos aleatórios equiprováveis
independentes ou dependentes.

BNCC: EF09MA20 - Reconhecer, em experimentos aleatórios, eventos
independentes e dependentes e calcular a probabilidade de sua
ocorrência, nos dois casos.

Os números pares disponíveis são: 2, 4, 6, 8 e 10. A probabilidade para
o 1º cartão ser par é: \(P\left( A \right) = \frac{5}{10}\). Já a
probabilidade do segundo também ser par é:
\(P\left( B \right) = \frac{4}{9}\), para sair par no 1º e par no 2º
temos: \(\frac{5}{10} \cdot \frac{4}{9} = \frac{20}{90} = \frac{2}{9}\),
ou seja, aproximadamente 22\%.

Alternativa c.

\chapter{Simulado 2}
\markboth{Simulado 2}{}

\begin{enumerate}
\def\labelenumi{\arabic{enumi}.}
\tightlist
\item
  Observe a reta numérica abaixo:
\end{enumerate}

\begin{figure}
\centering
\includegraphics[width=3.60725in,height=0.70833in]{./_SAEB_9_MAT/media/image250.png}
\caption{Linha do tempo Descrição gerada automaticamente}
\end{figure}

Refazer a imagem

Qual dos racionais abaixo é um candidato para a posição da seta?

\begin{enumerate}

\item
  \(\sqrt{3}\)
\item
  \(\sqrt{5}\)
\item
  \(\sqrt{6}\)
\item
  \(\sqrt{10}\)
\end{enumerate}

SAEB - 9N1.3 - números racionais ou irracionais. EF09MA02 - Reconhecer
um número irracional como um número real cuja representação decimal é
infinita e não periódica, e estimar a localização de alguns deles na
reta numérica. Alternativa d, pois \(3 < \sqrt{10} < 4\)

\begin{enumerate}
\def\labelenumi{\arabic{enumi}.}
\setcounter{enumi}{1}
\tightlist
\item
  A idade estimada do universo é de cerca de 13.8 bilhões de anos. Essa
  estimativa é baseada em dados observacionais e teóricos, incluindo a
  radiação cósmica de fundo e a taxa de expansão do universo.
  Acredita-se que o universo tenha se originado a partir de uma grande
  explosão, conhecida como Big Bang, que ocorreu há cerca de 13.8
  bilhões de anos. Desde então, o universo tem continuado a expandir-se
  e a evoluir, dando origem a galáxias, estrelas e planetas, incluindo a
  Terra. O estudo da idade do universo é fundamental para entendermos a
  história e a evolução do cosmos.
\end{enumerate}

A estimativa de idade do universo pode ser representada por:

\begin{enumerate}

\item
  \(138 \cdot 10^{6}\)
\item
  \(13,8 \cdot 10^{7}\)
\item
  \(1,38 \cdot 10^{10}\)
\item
  \(138\ 000\ 000\)
\end{enumerate}

SAEB -- 9N2.1 - Resolver problemas de adição, subtração, multiplicação,

divisão, potenciação ou radiciação envolvendo números reais,

EF09MA03 - Efetuar cálculos com números reais, inclusive potências com
expoentes fracionários. Alternativa c.~Segundo o texto a idade do
universo é estimada em 13,8 bilhões, escrevendo em Notação Científica
\(1,38 \cdot 10^{10}\).

\begin{enumerate}
\def\labelenumi{\arabic{enumi}.}
\setcounter{enumi}{2}
\tightlist
\item
  Quatro colegas pintaram uma parede. Jorge pintou \(\frac{1}{6}\) dessa
  parede, Gabriel pintou \(\frac{3}{18}\), Mário pintou \(\frac{2}{8}\)
  e Lucas \(\frac{3}{8}\).
\end{enumerate}

Eles ainda não terminaram a pintura da parede, mas quais colegas
pintaram a mesma quantidade?

\begin{enumerate}

\item
  Jorge e Mário
\item
  Gabriel e Lucas
\item
  Mário e Lucas
\item
  Jorge e Gabriel
\end{enumerate}

SAEB 9N1.8 -- Identificar frações equivalentes.

Para saber quais pintaram a mesma quantidade temos que comparar as
frações, uma maneira prática é reduzir todas aos mesmos denominadores:

\(\frac{1}{6}\) , \(\frac{3}{18} = \frac{1}{6}\) com isso já definimos
que Jorge e Gabriel pintaram a mesma quantidade.

Alternativa d.

\begin{enumerate}
\def\labelenumi{\arabic{enumi}.}
\setcounter{enumi}{3}
\tightlist
\item
  Em uma loja de artigos esportivos é comum em período de grande procura
  por determinada camisa de time o valor subir, por exemplo, em finais
  de campeonato, Copa da Mundo etc. Um determinado artigo teve seu preço
  reajustado com um aumento de 10\%, mas depois de certo período
  lançou-se a promoção que daria 10\% de desconto se fosse realizado o
  pagamento por PIX.
\end{enumerate}

Se o pagamento for realizado por PIX, durante o período promocional,
podemos afirmar que:

\begin{enumerate}

\item
  O preço será o mesmo que antes do aumento.
\item
  O preço ficará 1\% maior do que antes do aumento.
\item
  O preço ficará 1\% menor do que antes do aumento.
\item
  O preço ficará 99\% menor em comparação com o preço depois do aumento.
\end{enumerate}

SAEB: 9N2.3 - Resolver problemas que envolvam porcentagens, incluindo os
que lidam com acréscimos e decréscimos simples, aplicação de percentuais
sucessivos e determinação das taxas percentuais. BNCC: EF09MA05 -
Resolver e elaborar problemas que envolvam porcentagens, com a ideia de
aplicação de percentuais sucessivos e a determinação das taxas
percentuais, preferencialmente com o uso de tecnologias digitais, no
contexto da educação financeira. Quando um produto tem 10\% de aumento o
preço p é corrigido pelo fato (1+0,1) e ao ser descontado em 10\% é
corrigido pelo fator (1-0,1), como neste caso ocorreram os dois podemos
calcular assim: Seja p o valor antes do aumento, após o aumento será
p·(1+0,1) = 1,1p. Como depois temos o desconto, aplicaremos o fator de
correção: 1,1p·(1-0,1) = 1,1p·0,9 = 0,99p, ou seja, o produto custará
99\% do valor antes do aumento o que implica em um desconto de 1\%.
Alternativa c.

\begin{enumerate}
\def\labelenumi{\arabic{enumi}.}
\setcounter{enumi}{4}
\tightlist
\item
  O estacionamento de um shopping, cobra pela primeira hora, R\$ 8,00 e,
  em cada hora seguinte, ou fração da hora, R\$ 3,00.
\end{enumerate}

Uma pessoa que pagou 23 reais, logo, permaneceu com seu veículo no
estacionamento, por até:

a) 5 horas, porque 23 = 8 + 3x.

b) 3 horas, porque 23 = 8x -- 3.

c) 6 horas, porque 23 = 8 + (x -- 1) · 3.

d) 5 horas, porque 23 = 3 + (x -- 1) · 3.

SAEB: 9A1.1 - Resolver uma equação polinomial de 1º grau.

Como temos a 1ª hora do estacionamento por 8,00 + uma quantidade horas
no valor de 3,00 cada. Para formatar uma equação para este fim é
necessário calcular que ocorreu 1 hora de 8 mais (x -- 1) de 3, então
podemos definir que:

23 = 8 + (x -- 1) · 3. Resolvendo teremos:

23 -- 8 = 3x -- 3 → 18 = 3x → x = 6, ou seja, ficou estacionado por 6
horas.

Alternativa c.

\begin{enumerate}
\def\labelenumi{\arabic{enumi}.}
\setcounter{enumi}{5}
\tightlist
\item
  Analisando a sequência (1, 4, 9, 16, 25, \ldots{} ), Carlos e Lucas
  ficaram muito curiosos em estabelecer como continuar a sequência.
\end{enumerate}

Após pensar muito Carlos e Lucas estabeleceram que os três os próximos
números são:

a) 35, 46 e 55.

b) 36, 49 e 64.

c) 30, 41 e 54.

d) 41, 50, 59.

SAEB: 9A1.3 -- Identificar uma representação algébrica para o padrão ou
a regularidade de uma sequência de números racionais OU representar
algebricamente o padrão ou a regularidade de uma sequência de números
racionais.

Para encontrar os três próximos números da sequência, é importante
encontrar uma regularidade na sequência. Note que, do primeiro termo
para o segundo termo, somamos 3; do segundo para o terceiro termo,
somamos 5; do terceiro para o quarto termo e do quarto para o quinto
termo, somamos, respectivamente, 7 e 9, logo a soma aumenta duas
unidades a cada termo da sequência, ou seja, no próximo, somaremos 11,
depois 13, depois 15, depois 17 e assim sucessivamente. Para encontrar o
sucessor do 25, somaremos 11, depois 13 e pra concluir 15.

Alternativa b.

\begin{enumerate}
\def\labelenumi{\arabic{enumi}.}
\setcounter{enumi}{6}
\tightlist
\item
  Uma empresa em suas análises, estimou que o custo de produção, em
  milhares de reais, de n produtos de sua linha esportiva é calculado
  pela expressão \textbf{C(n)= n² -- n + 10.}
\end{enumerate}

Se o custo foi de 100 mil reais, então, o número de produtos produzidos
foi

a) 6. b) 7. c) 8. d) 10.

8. Mário gosta muito de montar e desmontar coisas, recentemente ganhou
um brinquedo de robótica e estava estudando as engrenagens e seus
movimentos. Ele percebeu que, enquanto a menor dá uma volta completa, a
maior gira só meia-volta.

\begin{figure}
\centering
\includegraphics[width=1.30208in,height=0.92708in]{./_SAEB_9_MAT/media/image251.png}
\caption{Roda de carro Descrição gerada automaticamente com confiança
média}
\end{figure}

Construir uma imagem semelhante. É importante o número de dentes de cada
engrenagem.

Quantas voltas dará a engrenagem grande, considerando que a engrenagem
pequena dará 20 voltas completas?

a) 20 voltas.

b) 5 voltas.

c) 10 voltas.

d) 15 voltas.

SAEB: 9A2.1 - Resolver problemas que envolvam variação de
proporcionalidade direta ou inversa entre duas ou mais grandezas,
inclusive escalas, divisões proporcionais e taxa de variação.

BNCC: EF09MA08 - Resolver e elaborar problemas que envolvam relações de
proporcionalidade direta e inversa entre duas ou mais grandezas,
inclusive escalas, divisão em partes proporcionais e taxa de variação,
em contextos socioculturais, ambientais e de outras áreas.

Como ficou perceptível para Mário que a engrenagem grande metade de uma
volta a cada volta da engrenagem pequena, basta então dividir pela
metade o número de voltas da engrenagem pequena.

20 ÷ 2 = 10.

Alternativa c.

9. Observando dois reservatórios A e B percebeu-se que os volumes, em
litros, deles variam em função do tempo t, medido em minutos, de acordo
com as seguintes relações:

\(V_{A} = 400 + 4t\) \emph{e} \(V_{B} = 6000 - 4t\)

Em que instante em que os reservatórios estarão com o mesmo volume?

a) t = 500 minutos

b) t = 550 minutos

c) t = 700 minutos

d) t = 1 500 minutos

SAEB: 9A2.5 - Resolver problemas que envolvam função afim.

BNCC: EF09MA06 - Compreender as funções como relações de dependência
unívoca entre duas variáveis e suas representações numérica, algébrica e
gráfica e utilizar esse conceito para analisar situações que envolvam
relações funcionais entre duas variáveis.

Estamos procurando valor de t, tal qual V\textsubscript{A} =
V\textsubscript{B}, ou seja,
\(400 + 4t = 6000 - 4t \rightarrow 8t = 5600 \rightarrow t = 700\)
\emph{minutos, pois ambos os reservatórios terão 3 200 litros.}

\emph{Alternativa c.}

\emph{10.} Em um círculo de raio 12 está inscrito um quadrilátero ABCD.
Sobre a soma dos ângulos opostos BÂD e BD, podemos afirmar que vale:

a) 12 x 180\textsuperscript{°}.

b) 360.

c) 90º.

d) 180\textsuperscript{°}.

SAEB: 9G1.8 - Reconhecer circunferência/círculo como lugares
geométricos, seus elementos (centro, raio, diâmetro, corda, arco, ângulo
central, ângulo inscrito)

BNCC: EF09MA11 - Resolver problemas por meio do estabelecimento de
relações entre arcos, ângulos centrais e ângulos inscritos na
circunferência, fazendo uso, inclusive, de softwares de geometria
dinâmica.

Pela propriedade de quadrilátero inscrito em circunferências, podemos
afirmar que ângulos opostos terão sua soma igual a 180º. A justificativa
é porque cada um dos ângulos terá um arco que completa 360º quando
somado ao arco referente ao ângulo oposto.

\begin{figure}
\centering
\includegraphics[width=1.22917in,height=1.15573in]{./_SAEB_9_MAT/media/image254.png}
\caption{Diagrama Descrição gerada automaticamente}
\end{figure}

Refazer a imagem

Alternativa d.

11. Em uma experiência em uma aula de Matemática, o professor André
desafiou os alunos a descobrirem a altura do mastro da bandeira. O aluno
que descobriu a altura fincou, paralelamente a esse mastro, um bastão de
1m. Medindo-se as sombras projetadas no chão pelo bastão e pelo mastro,
encontrando, respectivamente, 250 cm e 1 250 cm. Portanto, a altura do
mastro, em metros, é

a) 5,0.

b) 5,5.

c) 6,0.

d) 6,5.

SAEB: 9G1.7 - Reconhecer polígonos semelhantes ou as relações existentes
entre ângulos e lados correspondentes nesses tipos de polígonos.

BNCC: EF09MA12) Reconhecer as condições necessárias e suficientes para
que dois triângulos sejam semelhantes.

Para facilitar o entendimento, podemos fazer uma figura:

\begin{figure}
\centering
\includegraphics[width=3.06771in,height=1.1397in]{./_SAEB_9_MAT/media/image257.png}
\caption{Gráfico, Gráfico de linhas Descrição gerada automaticamente}
\end{figure}

Refazer a imagem

Dado que o momento foi o mesmo, podemos afirmar que os triângulos são
semelhantes e com isso podemos fazer a razão entre as sombras e alturas.
Escrevendo todas as medidas em metros, temos:

\(\frac{1}{2,5} = \frac{x}{12,5} \rightarrow x = \frac{12,5}{2,5} = 5\)
m.

Alternativa a.

12. \textbf{O gráfico a seguir mostram o número de alunos que utilizaram
carros por aplicativo para ir a universidade, durante uma determinada
semana, de segunda a sexta-feira.}

\begin{figure}
\centering
\includegraphics[width=3.79687in,height=1.84263in]{./_SAEB_9_MAT/media/image258.png}
\caption{Gráfico, Gráfico de barras Descrição gerada automaticamente}
\end{figure}

Dado que era a semana do consumidor e uma determinada empresa propôs
valores fixos independentes dos deslocamentos, apenas variando que no
período da manhã era R\$ 10,00 e a tarde era R\$ 15,00. Qual foi a
receita bruta do dia em que houve maior volume de atendimento?

a) R\$ 4.500,00

b) R\$ 9.000,00

c) R\$ 13.500,00

d) R\$ 15.000.00

SAEB: 9A1.8 - Associar uma das representações de uma função afim ou
quadrática a outra de suas representações (tabular, algébrica, gráfica)
ou associar uma situação que envolva função afim ou quadrática a uma das
suas representações (tabular, algébrica, gráfica).

EF09MA06 - Compreender as funções como relações de dependência unívoca
entre duas variáveis e suas representações numérica, algébrica e gráfica
e utilizar esse conceito para analisar situações que envolvam relações
funcionais entre duas variáveis.

O dia com maior movimento naquela semana foi na quinta-feira. Sendo que
ocorreram 450 utilizações de manhã e 300 no período da tarde.

Calculando a receita, temos: 450 · 10 + 300 · 15 = 4 500 + 4 500 = 9
000. A receita bruta foi de R\$ 9.000,00.

Alternativa b.

13. Uma empresa de importação de cosméticos possui 30 funcionários com a
seguinte distribuição salarial em reais.

\begin{longtable}[]{@{}ll@{}}
\toprule\noalign{}
Nº de funcionários & Salário em R\$ \\
\midrule\noalign{}
\endhead
\bottomrule\noalign{}
\endlastfoot
10 & 2.000,00 \\
12 & 3.600,00 \\
5 & 4.000,00 \\
3 & 6.000,00 \\
\end{longtable}

O diretor financeiro tem como objetivo descer o custo da folha de
pagamento, por isso propôs ao conselho que fossem realizadas demissões
na faixa salarial de R\$ 3.600,00 e com isso pretende chegar a uma
mediana de R\$ 2.800,00.

Quantos funcionários devem ser demitidos?

a) 8

b) 11

c) 9

d) 10

SAEB: 9E1.5 - Calcular os valores de medidas de tendência central de uma
pesquisa estatística (média aritmética simples, moda ou mediana).

EF09MA21) Analisar e identificar, em gráficos divulgados pela mídia, os
elementos que podem induzir, às vezes propositadamente, erros de
leitura, como escalas inapropriadas, legendas não explicitadas
corretamente, omissão de informações importantes (fontes e datas), entre
outros.

Para atingir o objetivo do diretor financeiro será necessário que a
mediana seja dada pela média aritmética entre 2000 e 3600, mas para que
isso ocorra, considerando demissões apenas de funcionário que recebem
3600, o total de funcionários deve ser igual a 22.

Logo será necessário demitir 8 pessoas.

Alternativa a.

14. Em uma distribuidora de produtos os funcionários ``juntam'' as
caixas de tal forma que se possa colocar sobre paletes e passar o
plástico filme em volta.

Veja o modelo que o funcionário da empilhadeira está organizando.
Precisa pegar os cubos A e fazer a pilha de caixa conforme o bloco B.

Quantas caixas A são necessárias para montar a B?

\begin{figure}
\centering
\includegraphics[width=2.41146in,height=1.62619in]{./_SAEB_9_MAT/media/image259.png}
\caption{Uma imagem contendo Diagrama Descrição gerada automaticamente}
\end{figure}

a) 60

b) 47

c) 94

d) 48

SAEB: 9G1.2 - Resolver problemas que envolvam volume de prismas retos ou
cilindros retos.

BNCC: EF09MA19 - Resolver e elaborar problemas que envolvam medidas de
volumes de prismas e de cilindros retos, inclusive com uso de expressões
de cálculo, em situações cotidianas.

Podemos perceber que a placa da base possuí 5 caixas de frente e 3
laterais, ou seja, são 15 vagas. Como temos 4 placas, uma sobre a outra,
ou seja, precisamos de 60 blocos de A para formar 1 do B.

Alternativa a.

15. Seja um triângulo retângulo, cujos catetos medem 6 e 8.

\begin{figure}
\centering
\includegraphics[width=2.26562in,height=1.15183in]{./_SAEB_9_MAT/media/image260.png}
\caption{Uma imagem contendo Diagrama Descrição gerada automaticamente}
\end{figure}

Qual a altura relativa à hipotenusa deste triângulo?

a) 3 cm

b) 4 cm

c) 4,8 cm

d) 10 cm

SAEB: 9G2.4 - Resolver problemas que envolvam relações métricas do
triângulo

retângulo, incluindo o teorema de Pitágoras.

BNCC: EF09MA13 - Demonstrar relações métricas do triângulo retângulo,
entre elas o teorema de Pitágoras, utilizando, inclusive, a semelhança
de triângulos.

Podemos utilizar as relações métricas no triângulo retângulo para
determinar a altura relativa.

Pelo Teorema de Pitágoras sabemos que a hipotenusa vale 10, sendo assim
podemos complementar a figura:

\begin{figure}
\centering
\includegraphics[width=1.71875in,height=1.10572in]{./_SAEB_9_MAT/media/image261.png}
\caption{Gráfico Descrição gerada automaticamente}
\end{figure}

Vamos usar a relação entre hipotenusa e altura. 10 · h = 6 · 8 → h = 4,8
cm.

Alternativa c.

\chapter{Simulado 3}
\markboth{Simulado 3}{}

\begin{enumerate}
\def\labelenumi{\arabic{enumi}.}
\tightlist
\item
  O número \(\sqrt{8} + \sqrt{18}\) pode ser escrito como:
\end{enumerate}

\begin{enumerate}

\item
  \(\sqrt{26}\)
\item
  \(2\sqrt{13}\)
\item
  \(5\sqrt{2}\)
\item
  \(\sqrt{10}\)
\end{enumerate}

SAEB - 9N1.3 - números racionais ou irracionais. EF09MA02 - Reconhecer
um número irracional como um número real cuja representação decimal é
infinita e não periódica, e estimar a localização de alguns deles na
reta numérica. Alternativa c.
\(\sqrt{8} + \sqrt{18} = \ \sqrt{2^{2} \cdot 2} + \sqrt{3^{2} \cdot 2} = 2\sqrt{2} + 3\sqrt{2} = 5\sqrt{2}.\)

\begin{enumerate}
\def\labelenumi{\arabic{enumi}.}
\setcounter{enumi}{1}
\tightlist
\item
  Em uma distribuidora de material para escritório, existem 100 pilhas
  de resmas de papel A4, com cada pilha tendo 1 metro de altura. Cada
  resma contém 500 folhas de papel. O fornecedor informa na embalagem
  que a espessura de uma única folha é de 0,4 milímetros.
\end{enumerate}

Quantas resmas tem na distribuidora?

\begin{enumerate}

\item
  2500
\item
  2000
\item
  1000
\item
  500
\end{enumerate}

SAEB -- 9N2.1 - Resolver problemas de adição, subtração, multiplicação,

divisão, potenciação ou radiciação envolvendo números reais,

EF09MA03 - Efetuar cálculos com números reais, inclusive potências com
expoentes fracionários.

Alternativa d.~Cada pilha tinha 1 m = 1 000 mm. O número de folhas em
uma pilha é calculado pela divisão entre 1 000 por 0,4 mm.
\(\frac{1000}{0,4} = 2500\).

Calcule 100 · 2 500 = 250 000 folhas, como cada resma tem 500 folhas
basta dividir 250 000 folhas por 500 folhas.
\(\frac{250000}{500} = 500\) resmas.

\begin{enumerate}
\def\labelenumi{\arabic{enumi}.}
\setcounter{enumi}{2}
\tightlist
\item
  As operações entre dízimas periódicas podem ser realizadas através das
  frações que as representa. Seja x = 0,272727... y = 0,5555...
\end{enumerate}

Qual o valor de x + y ?

\begin{enumerate}

\item
  \(\frac{27}{99}\)
\item
  \(\frac{5}{9}\)
\item
  \(\frac{82}{99}\)
\item
  \(\frac{32}{108}\)
\end{enumerate}

SAEB: 9N1.10 - Determinar uma fração geratriz para uma dízima periódica.

Primeiramente vamos escrever a fração geratriz de cada número:

\(0,27272727\ldots = \frac{27}{99}\) e \(0,5555\ldots = \ \frac{5}{9}\)

Agora podemos fazer a soma
\(\frac{27}{99} + \frac{5}{9} = \frac{27 + 55}{99} = \frac{82}{99}\).
Alternativa c.

\begin{enumerate}
\def\labelenumi{\arabic{enumi}.}
\setcounter{enumi}{3}
\tightlist
\item
  Foi anunciado em uma loja de móveis usados a seguinte mesa de
  escritório.
\end{enumerate}

\begin{figure}
\centering
\includegraphics[width=1.66071in,height=1.8286in]{./_SAEB_9_MAT/media/image262.png}
\caption{Diagrama, Desenho técnico Descrição gerada automaticamente}
\end{figure}

Montar figura semelhante

Qual será o valor da mesa pagamento à vista?

\begin{enumerate}

\item
  R\$500,00
\item
  R\$350,00
\item
  R\$300,00
\item
  R\$100,00
\end{enumerate}

SAEB: 9N2.3 - Resolver problemas que envolvam porcentagens, incluindo os
que lidam com acréscimos e decréscimos simples, aplicação de percentuais
sucessivos e determinação das taxas percentuais.

BNCC: EF09MA05 - Resolver e elaborar problemas que envolvam
porcentagens, com a ideia de aplicação de percentuais sucessivos e a
determinação das taxas percentuais, preferencialmente com o uso de
tecnologias digitais, no contexto da educação financeira.

Para determinar o resultado, precisamos apenas calcular 25\% de 400.
Sendo assim: 0,25 · 400 = 100. Como o desconto será de R\$100,00 o valor
a ser pago será de R\$ 300,00.

Alternativa c.

\begin{enumerate}
\def\labelenumi{\arabic{enumi}.}
\setcounter{enumi}{4}
\tightlist
\item
  Observe a pesagem na balança abaixo, que está em equilíbrio. As caixas
  de mesmo tamanho tem a mesma massa.
\end{enumerate}

\begin{figure}
\centering
\includegraphics[width=2.82812in,height=0.97757in]{./_SAEB_9_MAT/media/image263.png}
\caption{Desenho preto e branco Descrição gerada automaticamente com
confiança baixa}
\end{figure}

Montar o desenho com o dê cima

A massa da caixa pequena é

a) 50 g. b) 100 g. c) 150 g. d) 300 g.

SAEB: 9A1.1 - Resolver uma equação polinomial de 1º grau

Seja x a massa da caixa, temos que

300 + 3x = 600

3x = 300

x = 100. Cada caixa pequena tem 100 gramas.

Alternativa b.

\begin{enumerate}
\def\labelenumi{\arabic{enumi}.}
\setcounter{enumi}{5}
\tightlist
\item
  As variáveis x e y assumem valores conforme mostra o quadro abaixo:
\end{enumerate}

\begin{longtable}[]{@{}lllllll@{}}
\toprule\noalign{}
x & 5 & 6 & 7 & 8 & 9 & 10 \\
\midrule\noalign{}
\endhead
\bottomrule\noalign{}
\endlastfoot
y & 17 & 21 & 25 & 29 & 33 & 37 \\
\end{longtable}

A relação entre y e x é dada pela expressão:

a) y = 4x + 1. b) y = 2x + 2 c) y = 4x -- 3 d) y = 3x + 2

SAEB: 9A1.3 -- Identificar uma representação algébrica para o padrão ou
a regularidade de uma sequência de números racionais OU representar
algebricamente o padrão ou a regularidade de uma sequência de números
racionais.

Observando os valore de y, pode-se perceber que o aumento de y a cada
unidade aumentada em x é 4, sendo assim podemos estabelecer que a
expressão será 4x com algum ajuste. Fazendo 4·5 obtemos 20, mas o número
é 17 (3 a menos), fazendo 4·6 obtemos 24, mas o número é 21 (3 a menos),
logo podemos afirmar que y = 4x -- 3.

Alternativa c.

7. Os valores referentes às duas raízes da equação \textbf{x² + 2x -- 24
= 0} estão no intervalo

a) de 4 até 6. b) de -- 7 até 5. c) de -- 5 até 7. d) de -- 6 até 2.

SAEB: 9A1.7 - Resolver uma equação polinomial de 2º grau

Para resolver essa equação podemos utilizar da fórmula de Bhaskara.

\[\mathrm{\Delta} = 2^{2} - 4 \cdot 1 \cdot \left( 24 \right) = 4 + 96 = 100\]

\[x = \frac{- 2 \pm \sqrt{100}}{2 \cdot 1} = \frac{- 2 \pm 10}{2} = \left\{ \begin{matrix}
\frac{- 2 + 10}{2} = 4 \\
\frac{- 2 - 10}{2} = - 6 \\
\end{matrix} \right.\ \]

Dado que as raízes desta equação são -- 6 e 4 e considerando os
intervalos apresentados nas alternativas, temos que a alternativa
correta é b.

8. Josias é zelador em um condomínio, por isso contratou um jardineiro
para cortar o gramado do condomínio. Para fazer o serviço em uma região
de 40 m\textsuperscript{2} foram necessários 80 minutos. O gramado está
representado pela figura, indicando as regiões onde o trabalho já foi
realizado e onde o gramado ainda deve ser aparado.

\includegraphics[width=1.95312in,height=1.01528in]{./_SAEB_9_MAT/media/image264.wmf}

Considerando que o jardineiro mantenha o mesmo ritmo de trabalho no
restante do gramado.

Qual o tempo, em minuto, previsto para que o jardineiro conclua o
serviço de corte do gramado?

a) 320

b) 200

c) 120

d) 100

SAEB: 9A2.1 - Resolver problemas que envolvam variação de
proporcionalidade direta ou inversa entre duas ou mais grandezas,
inclusive escalas, divisões proporcionais e taxa de variação.

BNCC: EF09MA08 - Resolver e elaborar problemas que envolvam relações de
proporcionalidade direta e inversa entre duas ou mais grandezas,
inclusive escalas, divisão em partes proporcionais e taxa de variação,
em contextos socioculturais, ambientais e de outras áreas.

9. Jorge fez um estudo em laboratório acompanhando a população de um
determinado vírus. Ele montou a seguinte tabela:

\begin{longtable}[]{@{}ll@{}}
\toprule\noalign{}
Tempo em minutos & Quantidade \\
\midrule\noalign{}
\endhead
\bottomrule\noalign{}
\endlastfoot
1 & 1 \\
2 & 5 \\
3 & 9 \\
4 & 13 \\
5 & 17 \\
\end{longtable}

Supondo-se que o ritmo de crescimento dessa população tenha continuado a
obedecer a essa mesma lei, o número de vírus, ao final de 30 minutos,
era:

a) 102

b) 117

c) 197

d) 200

SAEB: 9A2.5 - Resolver problemas que envolvam função afim.

BNCC: EF09MA06 - Compreender as funções como relações de dependência
unívoca entre duas variáveis e suas representações numérica, algébrica e
gráfica e utilizar esse conceito para analisar situações que envolvam
relações funcionais entre duas variáveis.

Dada a regularidade de aumento de 4 vírus a cada 1 minuto, temos que a
quantidade tem a expressão do tipo q = 4n + k, como para 3 minutos temos
9 vírus podemos determinar k, pois 9 = 4·3 + k → 9 -- 12 = k → k = -- 3.

Dado que q = 4n -- 3 → q = 4·30 -- 3 = 120 -- 3 = 117.

Alternativa b.

10. Na figura abaixo, temos um quadrado AEDF e e .

\begin{figure}
\centering
\includegraphics[width=1.46354in,height=1.13788in]{./_SAEB_9_MAT/media/image267.png}
\caption{Forma Descrição gerada automaticamente}
\end{figure}

Qual é área do quadrado?

a) 5,76

b) 4,8

c) 20

d) 23,04

SAEB: 9G2.5 - Resolver problemas que envolvam polígonos semelhantes.

9M2.3 - Resolver problemas que envolvam área de figuras planas.

BNCC: EF09MA12 - Reconhecer as condições necessárias e suficientes para
que dois triângulos sejam semelhantes.

\(\frac{8 - x}{x} = \frac{x}{12 - x} \rightarrow x^{2} = 96 - 8x - 12x + x^{2} \rightarrow 20x = 96 \rightarrow x = 4,8\).

A área é igual a 4,8² = 23,04 unidade de medida quadrada.

\begin{figure}
\centering
\includegraphics[width=2.07292in,height=1.48748in]{./_SAEB_9_MAT/media/image268.png}
\caption{Gráfico, Gráfico de dispersão Descrição gerada automaticamente}
\end{figure}

Alternativa d.

11. Um triângulo retângulo tem a medida da hipotenusa é 13 cm e a de um
dos catetos é 5 cm.

Qual a medida do outro cateto?

a) 12

b) 11

c) 10

d) 8

SAEB: 9G2.4 - Resolver problemas que envolvam relações métricas do
triângulo

retângulo, incluindo o teorema de Pitágoras.

BNCC: EF09MA13 - Demonstrar relações métricas do triângulo retângulo,
entre elas o teorema de Pitágoras, utilizando, inclusive, a semelhança
de triângulos.

Calculando o valor de do cateto utilizando o Teorema de Pitágoras temos:

13² = 5² + x² → 169 -- 25 = x² → x = 12 cm.

12. Os gráficos mostram, em milhões de reais, o total do valor das
vendas que uma empresa realizou em cada mês, nos anos de 2021 e 2022.

\begin{figure}
\centering
\includegraphics[width=4.73437in,height=2.00699in]{./_SAEB_9_MAT/media/image269.png}
\caption{Gráfico, Gráfico de linhas Descrição gerada automaticamente}
\end{figure}

As vendas em 2022 foram bem melhores que as de 2021, mas olhando o
gráfico podemos afirmar que:

a) nas vendas de 2021 em vários meses tivemos quedas em relação ao mês
anterior.

b) houve uma queda brusca em 2022 de julho para agosto.

c) houve uma queda no mês de setembro para outubro em 2022.

d) o maior aumento foi de novembro para dezembro.

SAEB: 9E2.2 - Argumentar ou analisar argumentações/conclusões com base
nos dados apresentados em tabelas (simples ou de dupla entrada) ou
gráficos (barras simples ou agrupadas, colunas simples ou agrupadas,
pictóricos, de linhas, de setores ou em histograma).

BNCC: EF09MA21 - Analisar e identificar, em gráficos divulgados pela
mídia, os elementos que podem induzir, às vezes propositadamente, erros
de leitura, como escalas inapropriadas, legendas não explicitadas
corretamente, omissão de informações importantes (fontes e datas), entre
outros.

\begin{enumerate}

\item
  Incorreta. Em 2021 ocorreu crescimento em todos os meses.
\item
  Incorreta. Em 2022 de julho para agosto as vendas cresceram.
\item
  Correta.
\item
  Incorreta. Novembro para dezembro as vendas estavam estáveis.
\end{enumerate}

Alternativa c.

13. Um dado foi lançado 100 vezes. A tabela a seguir mostra os seis
resultados possíveis e as suas respectivas frequências de ocorrências:

\begin{longtable}[]{@{}lllllll@{}}
\toprule\noalign{}
\textbf{Resultado} & \textbf{1} & \textbf{2} & \textbf{3} & \textbf{4} &
\textbf{5} & \textbf{6} \\
\midrule\noalign{}
\endhead
\bottomrule\noalign{}
\endlastfoot
\textbf{Frequência} & 14 & 21 & 15 & 12 & 18 & 20 \\
\end{longtable}

Qual a moda dos resultados?

\begin{enumerate}

\item
  2
\item
  3
\item
  4
\item
  5
\end{enumerate}

SAEB: 9E1.5 - Calcular os valores de medidas de tendência central de uma
pesquisa estatística (média aritmética simples, moda ou mediana).

BNCC: EF09MA21 - Analisar e identificar, em gráficos divulgados pela
mídia, os elementos que podem induzir, às vezes propositadamente, erros
de leitura, como escalas inapropriadas, legendas não explicitadas
corretamente, omissão de informações importantes (fontes e datas), entre
outros.

Alternativa a.

14. Marcos comprou uma TV no valor de R\$ 4.500,00. Ele optou por dar
30\% de entrada e o restante foi negociado em 5 parcelas sem juros.

Qual o valor da parcela?

a) R\$ 135,00

b) R\$ 315,00

c) R\$ 530,00

d) R\$ 630,00

SAEB: 9N2.3 - Resolver problemas que envolvam porcentagens, incluindo os
que lidam com acréscimos e decréscimos simples, aplicação de percentuais
sucessivos e determinação de taxas percentuais.

BNCC: EF09MA05 - Resolver e elaborar problemas que envolvam
porcentagens, com a ideia de aplicação de percentuais sucessivos e a
determinação das taxas percentuais, preferencialmente com o uso de
tecnologias digitais, no contexto da educação financeira.

Calculando 30\% de 4.500 encontramos o valor de 1.350,00. O saldo que
ficou é 3.150,00. Sendo assim a parcela será igual 3150 ÷ 5 = 630. Serão
5 parcelas de R\$ 630,00.

Alternativa d.

15\textbf{.} Considere o lançamento simultâneo de dois dados
distinguíveis e não viciados, isto é, em cada dado, a chance de se obter
qualquer um dos resultados (1, 2, 3, 4, 5, 6) é a mesma. A probabilidade
de que a soma dos resultados seja 7 é:

a) \(\frac{5}{36}\)

b) \(\frac{1}{2}\)

c) \(\frac{1}{3}\)

d) \(\frac{1}{6}\)

SAEB: 9E2.4 - Resolver problemas que envolvam a probabilidade de
ocorrência de um resultado em eventos aleatórios equiprováveis
independentes ou dependentes.

BNCC: EF09MA20 - Reconhecer, em experimentos aleatórios, eventos
independentes e dependentes e calcular a probabilidade de sua
ocorrência, nos dois casos.

Para facilitar a contagem podemos montar uma tabela com as 36
possibilidades de resultados do lançamento de 2 dados:

\begin{longtable}[]{@{}lllllll@{}}
\toprule\noalign{}
\textbf{~} & \textbf{1} & \textbf{2} & \textbf{3} & \textbf{4} &
\textbf{5} & \textbf{6} \\
\midrule\noalign{}
\endhead
\bottomrule\noalign{}
\endlastfoot
\textbf{1} & \textbf{~} & \textbf{~} & \textbf{~} & \textbf{~} &
\textbf{~} & \textbf{~} \\
\textbf{2} & \textbf{~} & \textbf{~} & \textbf{~} & \textbf{~} &
\textbf{~} & \textbf{~} \\
\textbf{3} & \textbf{~} & \textbf{~} & \textbf{~} & \textbf{~} &
\textbf{~} & \textbf{~} \\
\textbf{4} & \textbf{~} & \textbf{~} & \textbf{~} & \textbf{~} &
\textbf{~} & \textbf{~} \\
\textbf{5} & \textbf{~} & \textbf{~} & \textbf{~} & \textbf{~} &
\textbf{~} & \textbf{~} \\
\textbf{6} & \textbf{~} & \textbf{~} & \textbf{~} & \textbf{~} &
\textbf{~} & \textbf{~} \\
\end{longtable}

Marcando as possibilidades de soma igual a 7 encontramos 6 situações.
Então a probabilidade de encontrar a soma 7 temos:

\(P\left( A \right) = \frac{6}{36} = \frac{1}{6}\)\emph{.}

\emph{Alternativa d.}

\chapter{Simulado 4}
\markboth{Simulado 4}{}

\begin{enumerate}
\def\labelenumi{\arabic{enumi}.}
\tightlist
\item
  O número \(\sqrt{12}\) é um número:
\end{enumerate}

\begin{enumerate}

\item
  Natural
\item
  Inteiro
\item
  Racional
\item
  Irracional
\end{enumerate}

SAEB - 9N1.3 - números racionais ou irracionais.

EF09MA02 - Reconhecer um número irracional como um número real cuja
representação decimal é infinita e não periódica, e estimar a
localização de alguns deles na reta numérica.

\(\sqrt{12} = 2\sqrt{3}\) é um número irracional, pois não é possível
escrever este número em forma de fração.

\begin{enumerate}
\def\labelenumi{\arabic{enumi}.}
\setcounter{enumi}{1}
\tightlist
\item
  A luz viaja 9,45·10\textsuperscript{15} metros por ano. O ano tem
  3,15·10\textsuperscript{7} segundos.
\end{enumerate}

Para determinar a velocidade calculamos a razão entre a distância e o
tempo.

Qual a velocidade da luz, em m/s?

\begin{enumerate}

\item
  \(3 \cdot 10^{8}\)
\item
  \(9,45 \cdot 10^{15}\)
\item
  \(3,15 \cdot 10^{7}\)
\item
  \(6,3 \cdot 10^{8}\)
\end{enumerate}

SAEB -- 9N2.1 - Resolver problemas de adição, subtração, multiplicação,

divisão, potenciação ou radiciação envolvendo números reais,

EF09MA03 - Efetuar cálculos com números reais, inclusive potências com
expoentes fracionários.

Alternativa

\(V = \frac{9,45 \cdot 10^{15}}{3,15 \cdot 10^{7}} = 3 \cdot 10^{8}m/s\)\emph{.}

\begin{enumerate}
\def\labelenumi{\arabic{enumi}.}
\setcounter{enumi}{2}
\tightlist
\item
  Comparando duas reportagens um leitor ficou confuso, pois parte do
  texto traziam números diferentes, veja este trecho:
\end{enumerate}

Jornal A: Segundo pesquisa realizada 3 a cada 8 pessoas já passaram por
tentativa de golpe ...

Jornal B: 37,5\% das pessoas já passaram por tentativa de golpe, segundo
pesquisa realiza. {[}...{]}

Observando os dados podemos afirmar que:

\begin{enumerate}

\item
  Ambos os jornais trazem os mesmos valores, mas representados em formas
  diferentes.
\item
  O Jornal A traz um número muito maior de pessoas na sua fração
  \(\frac{3}{8}\) quando comparado com o percentual de B.
\item
  O Jornal B traz um percentual que supera em muito a fração dada pelo
  Jornal A.
\item
  Não é possível executar comparação.
\end{enumerate}

SAEB 9N1.8 -- Identificar frações equivalentes. Escrevendo 37,5\% em
forma de fração irredutível, temos:
\(37,5\% = \frac{37,5}{100} = \frac{375}{1000} = \frac{3}{8}\), ou seja,
temos exatamente o mesmo valor representado. Alternativa a.

\begin{enumerate}
\def\labelenumi{\arabic{enumi}.}
\setcounter{enumi}{3}
\tightlist
\item
  Um comerciante precisou aumentar o preço de uma mercadoria de acordo
  com a inflação, pois quando não repassa o aumento ao consumidor acaba
  ficando com prejuízo. Em janeiro ajustou o preço de uma determinada
  mercadoria em 8\%, posteriormente em junho precisou aplicar novo
  aumento de 12\%.
\end{enumerate}

O preço da mercadoria após os dois aumentos é R\$ 60,48, qual era o
valor antes do aumento?

\begin{enumerate}

\item
  R\$ 50,00
\item
  R\$ 48,96
\item
  R\$ 53,22
\item
  R\$ 48,38
\end{enumerate}

SAEB: 9N2.3 - Resolver problemas que envolvam porcentagens, incluindo os
que lidam com acréscimos e decréscimos simples, aplicação de percentuais
sucessivos e determinação das taxas percentuais.

Podemos determinar o valor antes do aumento verificando qual foi o
percentual total aplicado. Para isso usaremos o fator de correção (1+i)
= (1 + 0,08) e (1+0,12). Foram dois aumentos sucessivos, sendo assim:
\(\left( 1 + 0,08 \right) \cdot \left( 1 + 0,12 \right) = 1,2026\), ou
seja, o valor de 60,48 representa 120,96\% do preço inicial.

Preço \%

X 100

60,48 120,96

Resolvendo a regra de três, temos: \(\frac{6048}{120,96} = 50\).

Ou seja, antes do aumento a mercadoria custava R\$ 50,00. Alternativa a.

\begin{enumerate}
\def\labelenumi{\arabic{enumi}.}
\setcounter{enumi}{4}
\tightlist
\item
  Tia Lúcia sempre distribui balas os seus sobrinhos, que são muitos.
  Quando Lúcia distribui 8 balas a cada aluno, sobram-lhe 44 balas; se
  ela der 10 balas a cada sobrinho, faltam-lhe 12 balas. Dado o
  apresentado, quantos sobrinhos tem a Lúcia.
\end{enumerate}

a) 22

b) 23

c) 24

d) 28

SAEB: 9A1.1 - Resolver uma equação polinomial de 1º grau.

Seja x o número de sobrinhos, sendo assim temos: 8x + 44 = 10x -- 12.

8x -- 10x = -- 12 -- 44 → -- 2x = -- 56 → x = 28.

Alternativa d.

\begin{enumerate}
\def\labelenumi{\arabic{enumi}.}
\setcounter{enumi}{5}
\tightlist
\item
  Observando a sequência de azulejos pintados:
\end{enumerate}

\begin{figure}
\centering
\includegraphics[width=3in,height=0.96526in]{./_SAEB_9_MAT/media/image270.png}
\caption{Desenho com traços pretos em fundo branco Descrição gerada
automaticamente com confiança média}
\end{figure}

Fazer figura semelhante

Ficou perceptível que a quantidade de azulejos pintados era dada pela
seguinte expressão q = n · (n + 1), sendo \textbf{n} o número do termo e
\textbf{q} a quantidade de azulejos. Qual será a quantidade de azulejos
pintados no 8º termo?

\begin{enumerate}

\item
  44
\item
  49
\item
  56
\item
  72
\end{enumerate}

SAEB: 9A2.2 - Resolver problemas que envolvam cálculo do valor numérico

de expressões algébricas.

Fazendo a substituição temos q = 8 · (8 + 1) = 8 · 9 = 72.

Alternativa d.

\begin{enumerate}
\def\labelenumi{\arabic{enumi}.}
\setcounter{enumi}{6}
\tightlist
\item
  A expressão determina o número total de diagonais de um polígono
  convexo, em que D representa o número total de diagonais do polígono,
  e \emph{n} o número de lados.
\end{enumerate}

Qual é o número de lados de um polígono que tem 14 diagonais?

a) 35 b) 32 c) 10 d) 7 SAEB: 9A1.7 - Resolver uma equação polinomial de
2º grau. Para determinar a quantidade de lados, precisamos resolver a
equação: \(14 = \frac{n^{2} - 3n}{2} \rightarrow n^{2} - 3n - 28 = 0\),
por se tratar de uma equação do 2º grau podemos aplicar Bhaskara.

\[\mathrm{\Delta} = \left( - 3 \right)^{2} - 4 \cdot 1 \cdot \left( - 28 \right) = 9 + 112 = 121\]

\(n = \frac{- ( - 3) \pm \sqrt{121}}{2} = \frac{3 \pm 11}{2} = \left\{ \begin{matrix} \frac{3 + 11}{2} = 7\ \ \ \ \ \ \ \ \ \ \ \ \ \ \ \ \ \ \ \ \ \ \ \ \ \ \ \ \ \  \\ \frac{3 - 11}{2} = - 4\ Não\ convém \\ \end{matrix} \right.\ \).
Por se tratar de quantidade de lados de um polígono desprezamos o valor
negativo. Alternativa d.

\begin{enumerate}
\def\labelenumi{\arabic{enumi}.}
\setcounter{enumi}{7}
\tightlist
\item
  Uma das grandes preocupações de hoje é a quantidade de açúcar
  ingerido, por isso as pessoas têm ficado mais atentar aos dados da
  embalagem.
\end{enumerate}

Em uma embalagem de um chocolate, consta que, em cada 100 gramas de
chocolate, contém 24 gramas de açúcar. Fernando comprou uma barra de 450
gramas desse chocolate.

Quantos gramas de açúcar contém essa barra que Fernando comprou?

a) 48 b) 96 c) 108 d) 160

SAEB: 9A2.1 - Resolver problemas que envolvam variação de
proporcionalidade direta ou inversa entre duas ou mais grandezas,
inclusive escalas, divisões proporcionais e taxa de variação.

BNCC: EF09MA08 - Resolver e elaborar problemas que envolvam relações de
proporcionalidade direta e inversa entre duas ou mais grandezas,
inclusive escalas, divisão em partes proporcionais e taxa de variação,
em contextos socioculturais, ambientais e de outras áreas.

Proporcionalmente se a cada 100 gramas obtemos 24 gramas de açúcar, em
400 gramas teremos 96 gramas (4 · 24 = 96). Podemos ainda concluir que
em 50 gramas teremos 12 gramas de açúcar, pois é metade da massa
descrita na etiqueta, sendo assim em 450 gramas teremos 96 + 12 = 108
gramas de açúcar.

Alternativa c.

9. Avaliando o consumo de bolas de sorvete e a temperatura foi anotado
os dados na tabela:

\begin{longtable}[]{@{}llllllll@{}}
\toprule\noalign{}
\emph{temperatura média mensal (ºC)} & \textbf{23} & \textbf{24} &
\textbf{25} & \textbf{27} & \textbf{28} & \textbf{29} & \textbf{30} \\
\midrule\noalign{}
\endhead
\bottomrule\noalign{}
\endlastfoot
\emph{bolas de sorvete} & 860 & 880 & 900 & 940 & 960 & 980 & 1000 \\
\end{longtable}

A regularidade se mantendo, qual seria a quantidade de bolas de sorvete
em um dia de temperatura de 34 º

\begin{enumerate}
\def\labelenumi{\Alph{enumi}.}
\setcounter{enumi}{2}
\tightlist
\item
\end{enumerate}

a) 1010

b) 1020

c) 1040

d) 1060

SAEB: 9A1.3 - Identificar uma representação algébrica para o padrão ou a
regularidade de uma sequência de números racionais ou representar
algebricamente o padrão ou a regularidade de uma sequência de números
racionais.

10. Observe a figura.

\begin{figure}
\centering
\includegraphics[width=1.85417in,height=1.11601in]{./_SAEB_9_MAT/media/image272.png}
\caption{Forma Descrição gerada automaticamente}
\end{figure}

Qual a medida do lado do quadrado?

a) 10.

b) 8.

c) 6.

d) 4.

SAEB: 9G2.5 - Resolver problemas que envolvam polígonos semelhantes.

BNCC: EF09MA12 - Reconhecer as condições necessárias e suficientes para
que dois triângulos sejam semelhantes.

\begin{figure}
\centering
\includegraphics[width=1.75521in,height=1.18424in]{./_SAEB_9_MAT/media/image273.png}
\caption{Forma Descrição gerada automaticamente}
\end{figure}

\(\frac{x}{12} = \frac{3}{x} \rightarrow x^{2} = 36 \rightarrow x = 6\).

Alternativa c.

11. Numa pesquisa de opinião, feita para verificar o nível de aprovação
de um produto, foram entrevistadas 1000 pessoas, que responderam sobre a
qualidade de uma marca de sabão em pó, escolhendo-se apenas uma
resposta.

O gráfico abaixo mostra o resultado da pesquisa.

\begin{figure}
\centering
\includegraphics[width=3.25in,height=2.18333in]{./_SAEB_9_MAT/media/image279.png}
\caption{Diagrama Descrição gerada automaticamente}
\end{figure}

Refazer o gráfico

De acordo com o gráfico, pode-se afirmar que o percentual de pessoas que
consideram a marca de sabão em pó boa ou regular é de:

a) 28\%.

b) 65\%.

c) 71\%.

d) 84\%.

SAEB: 9E2.2 - Argumentar ou analisar argumentações/conclusões com base
nos dados apresentados em tabelas (simples ou de dupla entrada) ou
gráficos (barras simples ou agrupadas, colunas simples ou agrupadas,
pictóricos, de linhas, de setores ou em histograma).

BNCC: EF09MA22 - Escolher e construir o gráfico mais adequado (colunas,
setores, linhas), com ou sem uso de planilhas eletrônicas, para
apresentar um determinado conjunto de dados, destacando aspectos como as
medidas de tendência central.

12. Uma empresa está analisando se coloca seu produto em um tipo de
cilindro de 6 cm de raio ou em um mais alto com 3 cm de raio.

\begin{figure}
\centering
\includegraphics[width=1.77016in,height=1.22917in]{./_SAEB_9_MAT/media/image280.png}
\caption{Diagrama Descrição gerada automaticamente}
\end{figure}

Sabendo que os volumes são iguais, qual o valor de x?

\begin{enumerate}

\item
  8
\item
  12
\item
  16
\item
  20
\end{enumerate}

SAEB: 9M2.4 - Resolver problemas que envolvam volume de prismas retos ou
cilindros

retos.

BNCC: EF09MA19 - Resolver e elaborar problemas que envolvam medidas de
volumes de prismas e de cilindros retos, inclusive com uso de expressões
de cálculo, em situações cotidianas.

\(V_{1} = \pi \cdot 6^{2} \cdot 4 = 144\pi\) →
\(V_{2} = \pi \cdot 3^{2} \cdot x\) →
\(\pi \cdot 3^{2} \cdot x = 144\pi \rightarrow x = 16.\)

Alternativa c.

13. As notas de um grupo de alunos em suas provas de Matemática foram:
8,4; 9,1; 7,2; 6,8; 8,7 e 7,2.

Qual a média e a mediana respectivamente?

a) 7,9; 7,8

b) 7,2; 7,8

c) 7,8; 7,8

d) 7,8; 7,9

SAEB: 9E1.5 - Calcular os valores de medidas de tendência central de uma
pesquisa estatística (média aritmética simples, moda ou mediana).

Para determinar a mediana precisamos colocar os número em ordem:

6,8 7,2 7,2 8,4 8,7 9,1

A mediana é calculada como média aritmética entre 7,2 e 8,4:
\(\frac{7,2 + 8,4}{2} = \frac{15,6}{2} = 7,8\). A média é :
\(\frac{6,8 + 7,2 + 7,2 + 8,4 + 8,7 + 9,1}{6} = 7,9\).

Alternativa a.

14. Um reservatório é formado por uma caixa de forma cúbica com 1 metro
de lado e está acoplada a um cano cilíndrico com 4 cm de diâmetro e 50 m
de comprimento.

Qual o volume total do sistema quando está completamente cheio?

a) 1 000 litros.

b) 1015 litros.

c) 1 062,8 litros.

d) 1600 litros.

SAEB: 9M2.4 - Resolver problemas que envolvam volume de prismas retos ou
cilindros retos.

BNCC: EF09MA19 - Resolver e elaborar problemas que envolvam medidas de
volumes de prismas e de cilindros retos, inclusive com uso de expressões
de cálculo, em situações cotidianas.

Volume da caixa cúbica: V\textsubscript{Caixa} = 1m³

Volume do cano: V\textsubscript{Cano} = (0,02)² · 3,14 · 50 = 0,0628 m³

Volume total: 1,0628 m³ = 1 062,8 litros.

Alternativa c.

15. O dominó é um jogo de tabuleiro que tem origem incerta, mas é
geralmente associado aos países europeus e latino-americanos. Há várias
teorias sobre a sua origem, mas nenhuma delas é totalmente comprovada.

Uma das teorias é que o dominó teria sido criado na China, por volta do
século XII, durante a Dinastia Song. No entanto, essa teoria é
controversa e não há evidências históricas que comprovem a sua
veracidade.

Outra teoria é que o dominó teria sido criado na Europa, mais
precisamente na Itália ou na França, durante o século XVIII. Nessa
época, o jogo era conhecido como "dominoes" e era jogado com peças de
madeira.

Independentemente da sua origem, o dominó se tornou um jogo muito
popular em todo o mundo, especialmente em países como México, Brasil e
Cuba, onde é considerado um símbolo da cultura local.

\includegraphics[width=2.40625in,height=1.82895in]{./_SAEB_9_MAT/media/image281.wmf}

Procurar outra imagem para ilustrar

Qual a probabilidade de ao escolher uma peça ao acaso e ela tenha dois
números diferentes entre si?

\begin{enumerate}

\item
  25\%.
\item
  75\%.
\item
  35\%.
\item
  60\%.
\end{enumerate}

SAEB: 9E2.4 - Resolver problemas que envolvam a probabilidade de
ocorrência de um resultado em eventos aleatórios equiprováveis
independentes ou dependentes.

BNCC: EF09MA20 - Reconhecer, em experimentos aleatórios, eventos
independentes e dependentes e calcular a probabilidade de sua
ocorrência, nos dois casos.

A probabilidade de que os números sejam diferentes é igual a
\(\frac{21}{28} = 0,75 = 75\%.\)
