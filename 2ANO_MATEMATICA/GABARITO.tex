\chapter{Respostas}
\pagestyle{plain}
\footnotesize

\pagecolor{gray!40}

\colorsec{Matemática – Módulo 1 – Treino}

\begin{enumerate}
\item
a) Correta. A senha do cartão identifica o dono do cartão para que não haja fraudes.
b) Incorreta. A senha do cartão de crédito não mede nenhuma grandeza.
c) Incorreta. A senha do cartão de crédito não quantifica nenhum grupo de objetos.
d) Incorreta. A senha do cartão de crédito não segue uma ordem lógica e sequencial.

\item
a) Incorreta. O aluno não entendeu que teria que diferenciar entre pares e ímpares.
b) Incorreta. O aluno indicou o 20° número ao invés do 30°.
c) Correta. Como temos a metade de números pares, basta multiplicar o 30 por 2.
d) Incorreta. O aluno indicou o 45° número ao invés do 30°.

\item
a) Correta. A ordem das dezenas está representeada pelo número 3.
b) Incorreta. A ordem das dezenas está representeada pelo número 8.
c) Incorreta. A ordem das dezenas está representeada pelo número 4.
d) Incorreta. A ordem das dezenas está representeada pelo número 2.
\end{enumerate}

\colorsec{Matemática – Módulo 2 – Treino}

\begin{enumerate}
\item
a) Incorreta. O aluno subtraiu os números.
b) Incorreta. O aluno errou a adição da ordem das unidades.
c) Correta. A soma dos dois números é igual a 236 + 132 = 368.
d) Incorreta. O aluno errou a adição da ordem das dezenas.

\item
a) Correta. A soma dos números resulta em 452.
b) Incorreta. A soma dos números resulta em 462.
c) Incorreta. A soma dos números resulta em 472.
d) Incorreta. A soma dos números resulta em 482.

\item
a) Incorreta. O aluno adicionou somente os municípios do Paraná e de Santa Catarina.
b) Incorreta. O aluno adicionou somente os municípios de Santa Catarina e os do Rio Grande do Sul.
c) Incorreta. O aluno adicionou somente os municípios de Paraná e os do Rio Grande do Sul.
d) Correta. O aluno adicionou corretamente, obtendo: 399 + 295 + 497 = 1191
\end{enumerate}

\colorsec{Matemática – Módulo 3 – Treino}

\begin{enumerate}
\item
a) Incorreta. Os braços da menina não podem calcular o espaço ocupado pela lousa.
b) Correta. Os braços da menina medem o comprimento entre a borda da lousa e o ponto marcado.
c) Incorreta. Os braços da menina não podem medir a massa da lousa
d) Incorreta. Os braços da menina não podem calcular o tempo.

\item
a) Incorreta. O marcador do tubo marca bem mais de 10 mL.
b) Incorreta. O marcador do tubo marca mais de 20 mL.
c) Correta. O marcador marca um pouquinho mais do que 40 mL; logo, mesmo que baixasse um pouco o nível com a retirada do conta-gotas, ele subiria novamente acima de 30 mL.
d) Incorreta. É impossível fazer uma estimativa tão precisa nesse caso, apenas com a observação.

\item
a) Incorreta. O aluno inverteu A e B, além de adicionar somente os lados desconhecidos.
b) Incorreta. O aluno adicionou os lados, mas esqueceu de adicionar os valores de A e B.
c) Correta. Os lados A e B são iguais aos seus correspondentes e a adição dos lados é igual a 2 + 2 + 5 + 2 + 2 + 5 = 18.
d) Incorreta. O aluno inverteu os valores de A e B.
\end{enumerate}

\colorsec{Matemática – Módulo 4 – Treino}

\begin{enumerate}
\item
a) Incorreta. O aluno contou dois dias a menos.
b) Incorreta. O aluno contou dois dias a menos
c) Correta. O aluno contou corretamente a quantidade de dias.
d) Incorreta. O aluno contou um dia a mais.

\item
a) Incorreta. Segunda-feira será dia 17.
b) Incorreta. Terça-feira será dia 18.
c) Incorreta. Quarta-feira será dia 19.
d) Correta. Quinta-feira será dia 20.

\item
a) Incorreta. O aluno contou 15 minutos a menos.
b) Correta. O aluno contou corretamente 30 minutos até fazer 10 horas, mais 15 minutos do número 3.
c) Incorreta. O aluno contou 5 minutos a mais.
d) Incorreta. O aluno contou 15 minutos a mais.
\end{enumerate}

\colorsec{Matemática – Módulo 5 – Treino}

\begin{enumerate}
\item
a) Incorreta. O aluno adicionou o troco à cédula de R\$ 200,00.
b) Incorreta. O aluno desconsiderou o valor do troco.
c) Correta. O aluno subtraiu o valor do troco da cédula dada por Claudia: 200 -- 87 = 113.
d) Incorreta. O aluno somente adicionou as cédulas de troco, desconsiderando a cédula dada.

\item
a) Incorreta. O aluno esqueceu de acrescentar uma das cédulas de 100.
b) Incorreta. O aluno achou que as moedas de 5 centavos valiam 5 reais, além de esquecer uma das cédulas de 100,00.
c) Correta. O aluno adicionou corretamente: 100 + 100 + 1 + 0,05 + 0,05 = 201,10
d) Incorreta. O aluno achou que as moedas de 5 centavos valiam 5 reais.

\item
a) Incorreta. A adição perfaz 350,00.
b) Incorreta. A adição perfaz 226,00.
c) Incorreta. A adição perfaz 226,00
d) Correta. A adição perfaz 326,00.
\end{enumerate}

\colorsec{Matemática – Módulo 6 – Treino}

\begin{enumerate}
\item
a) Incorreta. Cair de um lugar alto, caso não se tenha segurança, é certeza, em virtude da gravidade.
b) Incorreta. Um raio em dia de chuva é muito provável de acontecer.
c) Incorreta. Dormir todos os dias é certeza, ainda que ao longo da vida possamos passar algumas poucas noites acordados.
d) Correta. Em contato com o fogo a queimadura é certa; logo, não se queimar nesse caso é impossível.

\item
a) Incorreta. Situação pouco provável. Quanto mais treinamos, melhores somos em determinada atividade.
b) Incorreta. Situação impossível. Sorrir quando feliz é uma reação
automática do corpo.
c) Correta. É muito provável que alguém se queime ao manipular fogo de
forma descuidada.
d) Incorreta. Situação pouco provável. Em dias frios só suaremos se
praticarmos exercícios físicos intensos.

\item
a) Correta. Situação pouco provável. Geralmente fazemos isso todos os
dias; algumas pessoas podem ficar alguns dias sem fazer.
b) Incorreta. Impossível. Por menor que seja a dor, a pele ralada sempre gera incômodo.
c) Incorreta. Impossível. Ao praticar esportes o atrito aumenta a
temperatura corporal e o suor é ativado automaticamente para regular.
d) Incorreta. Impossível. Não tem como ficar seco estando em contato direto com a água.
\end{enumerate}

\colorsec{Matemática – Módulo 7 – Treino}

\begin{enumerate}
\item
a) Correta. A nota mais alta de Jéssica foi em Artes.
b) Incorreta. Em ciências, Jéssica tirou a terceira nota mais alta.
c) Incorreta. Em Geografia, assim como em História, Jéssica tirou a nota mais baixa.
d) Incorreta. Em matemática, Jéssica tirou a segunda nota mais alta.

\item
a) Incorreta. O chocolate foi o mais vendido.
b) Incorreta. O pudim foi o terceiro mais vendido.
c) Correta. O quindim foi o menos vendido.
d) Incorreta. O sorvete foi o segundo mais vendido.

\item
a) Incorreta. O aluno adicionou os preços mais altos.
b) Incorreta. O aluno adicionou os preços da loja 2.
c) Incorreta. O aluno adicionou os preços da loja 1.
d) Correta. O aluno adicionou os preços mais baixos: 125 + 70 + 650 + 86 + 23 = R\$ 954,00.
\end{enumerate}

\colorsec{Matemática – Módulo 8 – Treino}

\begin{enumerate}
\item
a) Incorreta. O aluno dividiu os pares pela metade.
b) Incorreta. O aluno entendeu pares de meia e pés de meia como sendo a
mesma coisa.
c) Correta. O aluno multiplicou os pares por 2, entendendo que cada par
tem dois pés de meia.
d) Incorreta. O aluno multiplicou os pares de meia por três.

\item
a) Incorreta. O aluno não multiplicou o valor da prateleira por nenhum fator.
b) Incorreta. O aluno dobrou o valor da prateleira.
c) Incorreta. O aluno triplicou o valor da prateleira.
d) Correta. O aluno quadruplicou o valor da prateleira.

\item
a) Correta. Mário percorreu 250 km a mais, que é exatamente a metade do percurso original.
b) Incorreta. O aluno entendeu que, adicionando a metade do valor original, resultaria no dobro.
c) Incorreta. O aluno entendeu que, adicionando a metade do valor original, resultaria no triplo.
d) Incorreta. O aluno entendeu que, adicionando a metade do valor original, resultaria no quádruplo.
\end{enumerate}

\colorsec{Simulado 1}

\begin{enumerate}
\item

\item

\item

\item

\item

\item

\item

\item

\item

\item

\item

\item

\item

\item

\item
\end{enumerate}

\colorsec{Simulado 2}

\begin{enumerate}
\item

\item

\item

\item

\item

\item

\item

\item

\item

\item

\item

\item

\item

\item

\item
\end{enumerate}

\colorsec{Simulado 3}

\begin{enumerate}
\item

\item

\item

\item

\item

\item

\item

\item

\item

\item

\item

\item

\item

\item

\item
\end{enumerate}

\colorsec{Simulado 4}

\begin{enumerate}
\item

\item

\item

\item

\item

\item

\item

\item

\item

\item

\item

\item

\item

\item

\item
\end{enumerate}