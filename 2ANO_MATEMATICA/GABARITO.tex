\chapter{Respostas}
\pagestyle{plain}
\footnotesize

\pagecolor{gray!40}

\colorsec{Matemática – Módulo 1 – Treino}

\begin{enumerate}
\item
a) Correta. A senha do cartão identifica o dono do cartão para que não haja fraudes.
b) Incorreta. A senha do cartão de crédito não mede nenhuma grandeza.
c) Incorreta. A senha do cartão de crédito não quantifica nenhum grupo de objetos.
d) Incorreta. A senha do cartão de crédito não segue uma ordem lógica e sequencial.

\item
a) Incorreta. O aluno não entendeu que teria que diferenciar entre pares e ímpares.
b) Incorreta. O aluno indicou o 20° número ao invés do 30°.
c) Correta. Como temos a metade de números pares, basta multiplicar o 30 por 2.
d) Incorreta. O aluno indicou o 45° número ao invés do 30°.

\item
a) Correta. A ordem das dezenas está representeada pelo número 3.
b) Incorreta. A ordem das dezenas está representeada pelo número 8.
c) Incorreta. A ordem das dezenas está representeada pelo número 4.
d) Incorreta. A ordem das dezenas está representeada pelo número 2.
\end{enumerate}

\colorsec{Matemática – Módulo 2 – Treino}

\begin{enumerate}
\item
a) Incorreta. O aluno subtraiu os números.
b) Incorreta. O aluno errou a adição da ordem das unidades.
c) Correta. A soma dos dois números é igual a 236 + 132 = 368.
d) Incorreta. O aluno errou a adição da ordem das dezenas.

\item
a) Correta. A soma dos números resulta em 452.
b) Incorreta. A soma dos números resulta em 462.
c) Incorreta. A soma dos números resulta em 472.
d) Incorreta. A soma dos números resulta em 482.

\item
a) Incorreta. O aluno adicionou somente os municípios do Paraná e de Santa Catarina.
b) Incorreta. O aluno adicionou somente os municípios de Santa Catarina e os do Rio Grande do Sul.
c) Incorreta. O aluno adicionou somente os municípios de Paraná e os do Rio Grande do Sul.
d) Correta. O aluno adicionou corretamente, obtendo: 399 + 295 + 497 = 1191
\end{enumerate}

\colorsec{Matemática – Módulo 3 – Treino}

\begin{enumerate}
\item
a) Incorreta. Os braços da menina não podem calcular o espaço ocupado pela lousa.
b) Correta. Os braços da menina medem o comprimento entre a borda da lousa e o ponto marcado.
c) Incorreta. Os braços da menina não podem medir a massa da lousa
d) Incorreta. Os braços da menina não podem calcular o tempo.

\item
a) Incorreta. O marcador do tubo marca bem mais de 10 mL.
b) Incorreta. O marcador do tubo marca mais de 20 mL.
c) Correta. O marcador marca um pouquinho mais do que 40 mL; logo, mesmo que baixasse um pouco o nível com a retirada do conta-gotas, ele subiria novamente acima de 30 mL.
d) Incorreta. É impossível fazer uma estimativa tão precisa nesse caso, apenas com a observação.

\item
a) Incorreta. O aluno inverteu A e B, além de adicionar somente os lados desconhecidos.
b) Incorreta. O aluno adicionou os lados, mas esqueceu de adicionar os valores de A e B.
c) Correta. Os lados A e B são iguais aos seus correspondentes e a adição dos lados é igual a 2 + 2 + 5 + 2 + 2 + 5 = 18.
d) Incorreta. O aluno inverteu os valores de A e B.
\end{enumerate}

\colorsec{Matemática – Módulo 4 – Treino}

\begin{enumerate}
\item
a) Incorreta. O aluno contou dois dias a menos.
b) Incorreta. O aluno contou dois dias a menos
c) Correta. O aluno contou corretamente a quantidade de dias.
d) Incorreta. O aluno contou um dia a mais.

\item
a) Incorreta. Segunda-feira será dia 17.
b) Incorreta. Terça-feira será dia 18.
c) Incorreta. Quarta-feira será dia 19.
d) Correta. Quinta-feira será dia 20.

\item
a) Incorreta. O aluno contou 15 minutos a menos.
b) Correta. O aluno contou corretamente 30 minutos até fazer 10 horas, mais 15 minutos do número 3.
c) Incorreta. O aluno contou 5 minutos a mais.
d) Incorreta. O aluno contou 15 minutos a mais.
\end{enumerate}

\colorsec{Matemática – Módulo 5 – Treino}

\begin{enumerate}
\item

\item

\item
\end{enumerate}

\colorsec{Matemática – Módulo 6 – Treino}

\begin{enumerate}
\item

\item

\item
\end{enumerate}

\colorsec{Matemática – Módulo 7 – Treino}

\begin{enumerate}
\item

\item

\item
\end{enumerate}

\colorsec{Matemática – Módulo 8 – Treino}

\begin{enumerate}
\item

\item

\item
\end{enumerate}

\colorsec{Simulado 1}

\begin{enumerate}
\item

\item

\item

\item

\item

\item

\item

\item

\item

\item

\item

\item

\item

\item

\item
\end{enumerate}

\colorsec{Simulado 2}

\begin{enumerate}
\item

\item

\item

\item

\item

\item

\item

\item

\item

\item

\item

\item

\item

\item

\item
\end{enumerate}

\colorsec{Simulado 3}

\begin{enumerate}
\item

\item

\item

\item

\item

\item

\item

\item

\item

\item

\item

\item

\item

\item

\item
\end{enumerate}

\colorsec{Simulado 4}

\begin{enumerate}
\item

\item

\item

\item

\item

\item

\item

\item

\item

\item

\item

\item

\item

\item

\item
\end{enumerate}