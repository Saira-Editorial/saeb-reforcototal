\chapter{Respostas}

\colorsec{Treino Português: Módulo 1}

\colorsec{Treino Português: Módulo 2}

\colorsec{Treino Português: Módulo 3}

\colorsec{Treino Português: Módulo 4}

\colorsec{Treino Português: Módulo 5}

\colorsec{Treino Português: Módulo 6}

\colorsec{Treino Português: Módulo 7}

\colorsec{Treino Português: Módulo 8}

\colorsec{Treino Português: Módulo 9}

\colorsec{Treino Português: Módulo 10}

\colorsec{Treino Arte: Módulo 1}

\colorsec{Treino Arte: Módulo 2}

\colorsec{Treino Arte: Módulo 3}

\colorsec{Treino Ciências humanas: Módulo 1}

\colorsec{Treino Ciências humanas: Módulo 2}

\colorsec{Treino Ciências humanas: Módulo 3}

\colorsec{Treino Ciências humanas: Módulo 4}

\colorsec{Treino Ciências humanas: Módulo 5}

\colorsec{Treino Ciências humanas: Módulo 6}

\colorsec{Simulado 1}

\begin{enumerate}

\item
a) Incorreta. O texto menciona somente o plástico.
b) Incorreta. O texto não trata da praticidade do processo.
c) Correta. Afirma-se explicitamente que o novo procedimento
aumenta a eficiência da reciclagem.
d) Incorreta. O texto parte de um pressuposto oposto.

\item
a) Correta. Ao longo do texto encontramos esses três personagens.
b) Incorreta. Falta mencionar o homem misterioro.
c) Incorreta. A vaca não pode ser considerada um personagem.
d) Incorreta. Os feijões não podem ser considerados personagens.

\item
a) Incorreta. Não há imagem associada à notícia.
b) Correta. O primeiro parágrafo do trecho configura o lide, que resume as principais informações da notícia.
c) Incorreta. Há um subtítulo, mas ele reforça uma informação que existe no corpo da notícia.
d) Incorreta. O título da notícia esclarece bem o assunto do texto.

\item
a) Correta. O texto expõe formas para manutenção da natureza brasileira.
b) Incorreta. O texto não trabalha com a possibilidade de expansão.
c) Incorreta. O texto não procura convencer o leitor dessa característica.
d) Incorreta. O texto apenas constata essa diminuição.

\item
a) Incorreta. O terceiro e o quarto verso não rimam entre si.
b) Incorreta. As rimas não são alternadas.
c) Incorreta. O primeiro e o segundo verso rimam entre si.
d) Correta. O primeiro verso rima com o segundo; o terceiro rima com o sexto; o quarto e o quinto rimam entre si.

\item
a) Incorreta. A palavra “cachola” não tem relação com “cachoeira”, apesar de haver semelhança fonética.
b) Correta. As palavras mencionadas são variações linguísticas de “cabeça” e “enrascada”, respectivamente.
c) Incorreta. As palavras “rascada” e “risco” não têm relação semântica.
d) Incorreta. As palavras mencionadas se assemelham apenas foneticamente às variações no texto.

\item
a) Incorreta. O ponto de exclamação é que é usado para destacar ideias.
b) Incorreta. A vírgula é que é usada para dividir duas frases.
c) Incorreta. O ponto final é que é usado para marcar o final de uma frase.
d) Correta. O ponto de interrogação é usado para marcar perguntas.

\item
a)  Incorreta. A pintura rupestre apresenta duas dimensões: altura e largura.
b)  Incorreta. O relevo é uma característica de obras tridimensionais.
c)  Incorreta. A arte rupestre é um exemplo de arte bidimensional.
d)  Correta. Por se tratar de uma arte dimensional, apresenta como
  característica a superfície plana.

\item
a)  Incorreta. Na dança, o responsável pela ação é o dançarino.
b)  Incorreta. Na escultura, o responsável pela ação é o artista plástico.
c)  Correta. No teatro, o responsável pela ação é o ator.
d)  Incorreta. No artesanato, o responsável pela ação é o artesão.

\item
a) Incorreta. Uma partitura não é uma obra de arte visual.
b) Correta. Está representada uma partitura, que é uma forma de notação musical.
c) Incorreta. Não se trata de um texto dramático, inclusive porque não há texto verbal.
d) Incorreta. A partitura é uma linguagem não verbal, mas não contém linguagem verbal.

\item
a) Incorreta. A bússola nos ajuda na localização geográfica.
b) Correta. A ampulheta serve para medir o tempo.
c) Incorreta. O termômetro serve para medir a temperatura.
d) Incorreta. A luneta serve para observar em grandes distâncias.

\item
a) Incorreta. O desmatamento das florestas pode piorar a poluição nos
rios.
b) Incorreta. O aumento da mineração pode aumentar os níveis de metais
nos rios.
c) Incorreta. A exploração do solo pode agravar a poluição dos rios a
partir do uso de agrotóxicos.
d) Correta. A reciclagem do lixo pode de fato contribuir para diminuir a
poluição dos rios.

\item
a) Incorreto. A prática não deve necessariamente ter retorno financeiro
para ser considerada patrimônio.
b) Correto. Como o maracatu representa um aspecto da cultura
afro-brasileira, para ser nomeado Patrimônio Imaterial o proponente deve
também representar a cultura de seu grupo.
c) Incorreto. A prática não precisa ser necessariamente dentro da
família, mas pode ser de um grupo maior de pessoas.
d) Incorreto. Existem práticas muito antigas que podem ser consideradas
Patrimônio Imaterial e devem ser preservadas.

\item
a) Incorreta. O ensino de história da África não garante a melhoria das
merendas na escola.
b) Correta. A obrigatoriedade do ensino de história da África contribui
para a diversidade do ensino da cultura de múltiplas etnias e raças no
Brasil.
c) Incorreta. Não há impacto sobre as crianças em situação de rua.
d) Incorreta. Já foi instituído o fim da escravidão no Brasil, o impacto
é na memória sobre este fenômeno.

\item
a) Correta. Trabalhar em casa só é possível porque o trabalhador
consegue se comunicar com seus colegas pela tecnologia.
b) Incorreta. O aumento de empregos em indústrias de produção não
influencia sobre o home office.
c) Incorreta. Não é mencionada uma dificuldade em encontrar empregos
presenciais.
d) Incorreta. Não é mencionada a criação de novas tarefas domésticas.
\end{enumerate}

\colorsec{Simulado 2}

\begin{enumerate}
\item
a) Incorreta. O trecho faz parte de um diálogo.
b) Incorreta. O trecho é uma fala da personagem.
c) Correta. O trecho contém instruções de cena específicas para o ator.
d) Incorreta. O trecho faz parte da fala de D. Carlota.

\item
a) Correta. O verbo “perguntar” está relacionado a uma fala do personagem.
b) Incorreta. O verbo “ficar” não é um verbo de enunciação.
c) Incorreta. O verbo “ver” não faz referência a uma fala.
d) Incorreta. O verbo “poder” não remete a um diálogo.

\item
a) Correta. A enfermeira disse algo ao médico, mas não foi uma pergunta.
b) Incorreta. A enfermeira não fez uma pergunta, mas, de qualquer modo, o médido fez um comentário com ela relacionado ao que ela tinha dito.
c) Incorreta. Não há indícios de que o médico tenha ficado nervoso, pois o verbo “responder” não dá essa indicação.
d) Incorreta. De fato, o médico dirigia-se à enfermeira, fazendo um comentário ao que ela tinha dito, e isso se indica por meio do verbo “responder”.

\item
a) Incorreta. Ambos os tipos de texto apresentam personagens.
b) Incorreta. Os dois tipos de texto podem apresentar a mesma temática.
c) Correta. O texto narrativo, ao contrário do dramático, apresenta um narrador.
d) Incorreta. Os registros linguísticos podem ser os mesmos nos dois tipos de texto.

\item
a) Incorreta. O adjetivo revela uma visão clara a respeito do assunto.
b) Incorreta. Uma visão neutra a respeito do assunto não utilizaria um adjetivo positivo.
c) Correta. O adjetivo “competitiva” revela um atributo positivo.
d) Incorreta. O adjetivo atribui um sentido oposto.

\item
a) Incorreta. O 1º ano é o segundo colocado na ordem crescente.
b) Correta. O número de matriculados no 2º ano, 17.850, é o menor que aparece na tabela.
c) Incorreta. O 3º ano é o terceiro colocado na ordem crescente.
d) Incorreta. O 4º ano é o segundo colocado na ordem decrescente.

\item
a) Correta. De fato, essa é uma opinião em que acredita o especialista citado no texto.
b) Incorreta. O Ingenuity, de fato, realizou o voo mais longo na superfície de Marte; trata-se de um fato.
c) Incorreta. Segundo o texto, o Ingenuity é mesmo um mini-helicóptero da Nasa.
d) Incorreta. É um fato expresso no texto que o robô Perseverance pousou em Marte no ano de 2021.

\item
a) Correta. O afoxé é um instrumento musical classificado como idiofone,
ou seja, que produz som por meio da vibração de seu próprio corpo.
b) Incorreta. Os instrumentos musicais que produzem som por meio da
vibração das cordas são os cordofones, como o violão.
c) Incorreta. Os instrumentos musicais que produzem som por meio de uma
membrana são os membrafones, como o tamborim.
d) Incorreta. Os instrumentos que produzem som por meio da vibração do ar
são os aerofones, como o trombone.

\item
a) Incorreta. A máscara dos yakas, apesar de fazer parte do acervo do
Museu Afro-Brasil, apresenta estética de matriz africana, mas faz parte
do contexto sociocultural do negro na África.
b) Correta. O maracatu é uma manifestação cultural afro-brasileira que
apresenta elementos estéticos de matriz africana e que surgiu no estado
de Pernambuco, no século XVIII, no contexto sociocultural do período
colonial brasileiro.
c) Incorreta. O painel de porta do povo iorubá apresenta estética de
matriz africana, mas não faz parte do contexto sociocultural do negro no
Brasil.
d) Incorreta. A escultura yumbe apresenta estética de matriz africana,
mas faz parte do contexto sociocultural do negro na África.

\item
a) Incorreta. Trata-se de um movimento corporal característico da dança do frevo.
b) Incorreta. Trata-se de um movimento corporal característico da dança do tango.
c) Correta. Trata-se de um movimento corporal característico da estética da dança \textit{hip
hop}: pernas para cima, cabeça para baixo, como apoio as mãos, cabeça ou ombros.
d) Incorreta. Trata-se de movimento característico da dança balé.

\item
a) Correta. A medida é importante para avaliar o impacto dos transportes
no ar e no meio ambiente do município.
b) Incorreta. Não há um interesse de diminuição da circulação de
produtos na região.
c) Incorreta. Não existe uma política ambiental sendo avaliada, mas sim
um impacto do uso de transportes.
d) Incorreta. Não há impactos sobre a qualidade da educação nas escolas.

\item
A alternativa correta é a d), cuja ordem corresponde às regiões.

\item
a) Incorreta. O texto fala sobre transporte público e não particular.
b) Incorreta. O texto não fala sobre as condições dos motoristas.
c) Correta. O texto mostra que o movimento queria a gratuidade do
transporte público.
d) Incorreta. O texto não fala sobre a qualidade das vias.

\item
a) Correta. O texto fala sobre sua boa localização - limites do
município - e sobre o interesse de empresários em seu potencial de
crescimento.
b) Incorreta. Não há menção à educação nem à beleza da paisagem.
c) Incorreta. Não há menção ao turismo ou ao passado do bairro.
d) Incorreta. Não há menção a riquezas minerais nem a um polo
gastronômico.

\item
a) Correta. Precisamos da pecuária para as carnes e da agricultura para
os demais ingredientes.
b) Incorreta. Não precisamos do turismo.
c) Incorreta. Não precisamos da mineração.
d) Incorreta. Não precisamos do turismo.
\end{enumerate}

\colorsec{Simulado 3}

\begin{enumerate}
\item
a) Incorreta. Apenas com base no título, essa informação não pode ser analisada.
b) Incorreta. No título, não se faz menção a esse aspecto do acontecimento.
c) Correta. Segundo o título, o Acre precisou ser socorrido; portanto fica entendido que o estado estava com dificuldades no momento de publicação da notícia.
d) Incorreta. Está claro, por meio de “mais de” que o valor expresso não é o exato.

\item
a) Correta. Entre parênteses, apresenta-se uma informação acessária relacionada à Estação Espacial Internacional, que ajuda o leitor a reconhecer a sigla pela qual ela é normalmente identificada.
b) Incorreta. O termo “Estação Espacial Internacional”, a que o trecho entre parênteses se relaciona, não é explicado por ele, mas tem revelado um aspecto acessório.
c) Incorreta. A expressão não é traduzida, apesar de aparecer a sigla em inglês.
d) Incorreta. A informação não tinha sido dada até aquele ponto.

\item
a) Incorreta. Não se trata de um advérbio de lugar.
b) Incorreta. Não se trata de um advérbio de negação.
c) Incorreta. Não se trata de um advérbio de modo.
d) Correta. De fato, trata-se de um advérbio de tempo.

\item
a) Incorreta. Não se trata de poemas de amor, mas orações.
b) Incorreta. Não há elementos para se afirmar que se trata de lamentações.
c) Incorreta. O tom da estrofe não é de alegria; portanto não há comemoração.
d) Correta. O eu lírico dirige-se a um “querubim” (provavelmente, sua amada), pedindo-lhe que reze por si.

\item
a) Incorreta. A locução “no entanto” não é consecutiva.
b) Incorreta. A locução “no entanto” não é conclusiva.
c) Correta. A locução “no entanto” é de oposição.
d) Incorreta. A locução “no entanto” não é aditiva.

\item
a) Incorreta. A imagem reforça o texto verbal.
b) Correta. O artifício visual reforça o ponto de vista construído no texto.
c) Incorreta. O cartaz cumpre a função de complementar, e não substituir o texto verbal.
d) Incorreta. O cartaz, como um todo, defende o ponto de vista oposto.

\item
a) Incorreta. O texto não menciona a expansão desses programas.
b) Correta. O texto cita explicitamente esse argumento, e ele é bem forte em relação ao ponto de vista que se quer defender.
c) Incorreta. O texto não trata do processo de fabricação das vacinas.
d) Incorreta. O texto menciona o calendário, mas não afirma que podemos nos vacinar apenas durante sua vigência.

\item
a) Correta. É um patrimônio imaterial, pois diz respeito a práticas e
domínios da vida social dos povos indígenas do Xingu.
b) Incorreta. É um patrimônio imaterial.
c) Incorreta. É transmitido de geração a geração, mas é um patrimônio
imaterial.
d) Incorreta. É um patrimônio cultural imaterial, mas pertence ao
patrimônio de povos indígenas do centro-oeste brasileiro (Mato Grosso do
Sul).

\item
a) Incorreta. A dança é uma linguagem artística representativa do forró,
mas a linguagem das artes visuais não se faz presente no forró.
b) Incorreta. A música é uma linguagem artística presente no forró,
porém a linguagem das artes visuais não se encontra representada nessa
manifestação artística.
c) Correta. A dança e a música são linguagens artísticas presentes no
forró.
d) Incorreta. A música é uma linguagem representativa do forró, mas a
linguagem do teatro não é característica dessa manifestação cultural.

\item
a) Incorreta. O maracá é de matriz indígena.
b) Correta. De fato, o maracá é um instrumento de origem indígena.
c) Incorreta. O maracá não foi trazido para o Brasil pelos escravizados.
d) Incorreta. A origem do maracá é bem conhecida.

\item
a) Incorreta. O texto não fala que os objetos ameaçam o meio ambiente.
b) Incorreta. O texto não trata como lixo, mas como objetos de memória.
c) Correta. Os objetos são tratados como objetos de memória, ou seja,
documentos sobre a história de Hugo.
d) Incorreta. Não é falado sobre o valor material dos objetos.

\item
a) Incorreta. A imagem não mostra o corte de árvores.
b) Incorreta. A imagem não mostra reciclagem de papéis, além dela ser
benéfica ao meio ambiente.
c) Incorreta. A imagem não mostra campos queimados.
d) Correta. A imagem mostra lixos jogados no chão da escola, local
inadequado.

\item
a) Incorreta. A escrita não foi inventada pelos fenícios.
b) Incorreta. A escola não foi inventada pelos fenícios.
c) Correta. O texto fala sobre a utilização pelos gregos do alfabeto
criado pelos fenícios.
d) Incorreta. Não foram os fenícios que criaram o livro.

\item
a) Incorreta. Não é uma organização proibida, mas sim a base da
sociedade.
b) Incorreta. Não é uma lenda, mas a organização principal das aldeias.
c) Correta. É uma junção de vários grupos familiares por diversos
interesses.
d) Incorreta. Não pode ser utilizada para denominar os povos nativos.

\item
a) Incorreta. A secretária não trabalha no ambiente doméstico.
b) Incorreta. A padeira não trabalha no ambiente doméstico, mas na
padaria.
c) Incorreta. A vendedora trabalha na loja e não na casa.
d) Correta. A faxineira é uma trabalhadora doméstica responsável pela
limpeza da casa.
\end{enumerate}

\colorsec{Simulado 4}

\begin{enumerate}

\item
a) Incorreta. Ambos os textos foram publicados em portais bastante conhecidos e respeitados.
b) Incorreta. Ambos os textos tratam de um assunto sério.
c) Correta. Apresentar a fala de um especialista é uma vantagem importante para um texto desse gênero.
d) Incorreta. A extensão do texto também não implica informações mais confiáveis, necessariamente.

\item
a) Incorreta. O substantivo “mão” não é retomado no texto.
b) Correta. A pronome “ele” retoma a palavra “dono”.
c) Incorreta. A conjunção “mas” não poderia ser retomada pelo pronome “ele”.
d) Incorreta. O verbo “era” não poderia ser retomado pelo pronome “ele”.

\item
a) Correta. O texto cita esse fato como a causa do fenômeno.
b) Incorreta. O texto menciona o fenômeno sendo observado somente no Brasil.
c) Incorreta. O texto apenas afirma que muitas pessoas registraram o fenômeno.
d) Incorreta. O aumento da população de esturjão é responsável pelo nome “superlua de esturjão”, mas não com a ocorrência da superlua.

\item
a) Correta. Esse sufixo é utilizado para indicar algo diminuto.
b) Incorreta. O sufixo \textbf{-ão}, por exemplo, é utilizado para indicar algo grande.
c) Incorreta. Esse sufixo não representa um juízo de valor.
d) Incorreta. Não há um afixo para demonstrar irrelevância.

\item
a) Incorreta. O verbo “vir” não é retomado ao final do diálogo.
b) Incorreta. A palavra “comigo” não reaparece posteriormente.
c) Incorreta. O verbo “arranjar” não é relacionada à palavra mencionada.
d) Correta. A sequência do diálogo permite que o leitor infira o sentido da palavra, relacionado ao sentido de “dinheiro”.

\item
a) Incorreta. O texto apenas descreve uma iniciativa relacionada às abelhas.
b) Incorreta. O texto menciona somente uma startup.
c) Incorreta. O texto não menciona a agressividade das abelhas.
d) Correta. O trecho “uma \textit{startup} {[}...{]} voltada a melhorar a precisão da polinização do café por abelhas” expõe o tema do texto.

\item
a) Incorreta. Podemos inferir que o instituto já existia.
b) Incorreta. Segundo o texto, a pesquisa já existia.
c) Correta. O texto menciona essa informação explicitamente.
d) Correta. Na realidade, afirma-se que os furacões se intensificam com o aquecimento do planeta.

\item
a) Incorreta. O balé clássico é um gênero de dança que possui técnicas e
  vocabulário próprios. Na dança contemporânea, não existem técnicas nem
  vocabulário predefinidos.
b) Correta. A dança contemporânea não se define em movimentos específicos,
  e o bailarino/intérprete tem autonomia para construir sua coreografia.
c) Incorreta. Na dança contemporânea, não há limitações para roupas e
  acessórios, inexistindo um padrão.
d) Incorreta. A postura reta e verticalizada é uma característica do balé
  clássico. A dança contemporânea tem como uma de suas características a
  maior mobilidade da coluna.

\item
a) Incorreta. O pandeiro, apesar de fazer parte dos instrumentos
característicos da roda de capoeira, foi introduzido no Brasil pelos
portugueses.
b) Correta. O berimbau foi introduzido no Brasil pelos escravizados
africanos e é um dos instrumentos característicos da roda de capoeira.
c) Incorreta. O cavaquinho é de origem portuguesa e marca presença em
ritmos como chorinho, samba e pagode.
d) Incorreta. A cuíca é outro instrumento trazido pelos escravizados para o
Brasil, mas é utilizada no samba.

\item
a) Incorreta. O Centro Histórico de Ouro Preto é um conjunto arquitetônico e
urbanístico e, por isso, constitui um exemplo de patrimônio material.
b) Incorreta. As ruínas das missões jesuíticas constituem um
conjunto arquitetônico e, por isso, de natureza material.
c) Incorreta. A Praça São Francisco é um conjunto arquitetônico,
composto de edifícios construídos no período durante o qual as coroas de
Portugal e Espanha estiveram unidas, entre 1580 e 1640; é, portanto, de
natureza material.
d) Correta. As festas são exemplos de patrimônio cultural imaterial.

\item
a) Incorreta. Não é um relógio.
b) Incorreta. Não é uma ampulheta.
c) Incorreta. Não é um diário.
d) Correta. É um calendário.

\item
a) Incorreta. O texto não fala sobre a ocupação de áreas agrícolas, mas
de biomas preservados.
b) Incorreta. O texto não fala que a melhora na vida dos humanos piora a
vida animal.
c) Correta. A expansão urbana invade biomas e áreas preservadas e
destinadas à vida animal.
d) Incorreta. O aumento das cidades não baixa a destruição das
florestas, pelo contrário, aumenta.

\item
a) Incorreta. O texto mostra que existem muitas denúncias.
b) Incorreta. A diversidade ensinada nas escolas melhoraria o problema,
não seria o motivo dele.
c) Incorreta. O texto não fala que essas comunidades são violentas, mas
sim violentadas.
d) Correta. O texto fala que ainda há resquícios de racismo em nossa
sociedade.

\item
a) Incorreta. Um condomínio residencial não é considerado patrimônio
histórico e artístico pelo IPHAN.
b) Incorreta. Uma plantação de arroz não é considerada patrimônio
histórico e artístico pelo IPHAN.
c) Correta. Uma igreja muito antiga pode fazer parte do patrimônio
protegido pelo IPHAN.
d) Incorreta. Uma floresta não é considerada patrimônio histórico e
artístico pelo IPHAN mas sim patrimônio da natureza.

\item
a) Incorreta. O texto não fala sobre o ferimento do direito de proteção
contra a tortura.
b) Incorreta. O texto não fala que as pessoas não têm direito à moradia.
c) Incorreta. O texto não fala que existem pessoas sem nacionalidade no
Catar.
d) Correta. O texto fala que as mulheres são discriminadas e que certas
comunidades não têm direitos iguais a outras.
\end{enumerate}