\chapter{Simulado 2}

\num{1} Leia o trecho de um texto.

\begin{quote}
{[}...{]}

D. CARLOTA --- Não sei nada; sou uma tagarela, que o senhor obrigou a
dar por paus e por pedras; mas, como é a última vez que nos vemos, não
importa. Agora, passe bem.

CAVALCANTE: Adeus, D. Carlota!

D. CARLOTA: Adeus, doutor!

CAVALCANTE: Adeus. (Dá um passo para a porta do fundo.) Talvez eu vá a
Atenas; não fuja se me vir vestido de frade.

{[}...{]}.

\fonte{Joaquim Maria Machado de Assis. \emph{Não consultes médico}.
Disponível em:
\emph{https://machado.mec.gov.br/obra-completa-lista/item/download/66\_390921fb4791464b4885563dc04a042c}.
Acesso em: 26 fev. 2023.}
\end{quote}

Nesse trecho de texto dramático, a ação a ser desempenhada pelo ator é descrita
no trecho

\begin{minipage}{.5\textwidth}
\begin{escolha}
\item “Adeus, doutor!”

\item “Talvez eu vá a Atenas;”

\item “(Dá um passo para a porta do fundo.)”

\item “Agora, passe bem.”
\end{escolha}
\end{minipage}
\sidetext{SAEB: Identificar as marcas de organização de textos dramáticos. BNCC:
EF35LP24 -- Identificar funções do texto dramático (escrito para ser
encenado) e sua organização por meio de diálogos entre personagens e
marcadores das falas das personagens e de cena.}

\num{2} Leia o trecho de um texto.

\begin{quote}
{[}...{]} Em casa, ficaram querendo bem a Escobar; a mesma prima
Justina achou que era um moço muito apreciável, apesar...

--- Apesar de quê? perguntou-lhe José Dias, vendo que ela não acabava a frase.

Não teve resposta, nem podia tê-la; prima Justina provavelmente não viu
defeito claro ou importante no nosso hóspede; {[}...{]}.

\fonte{Joaquim Maria Machado de Assis. \emph{Dom Casmurro.}
Disponível em:
\emph{https://machado.mec.gov.br/obra-completa-lista/item/download/13\_7101e1a36cda79f6c97341757dcc4d04}.
Acesso em: 26 fev. 2023.}
\end{quote}

O trecho reproduzido acima apresenta o verbo de enunciação

\begin{minipage}{.5\textwidth}
\begin{escolha}
\item “perguntar”.

\item “ficar”.

\item “ver”.

\item “poder”.
\end{escolha}
\end{minipage}
\sidetext{SAEB: Analisar os efeitos de sentido de verbos de enunciação. BNCC:
EF05LP10 -- Ler e compreender, com autonomia, anedotas, piadas e cartuns,
dentre outros gêneros do campo da vida cotidiana, de acordo com as
convenções do gênero e considerando a situação comunicativa e a
finalidade do texto.}

\num{3} Leia o texto.

\begin{quote}
A enfermeira diz ao médico:
--- Tem um homem invisível na sala de espera.
O médico responde:
--- Diga a ele que não posso vê-lo agora.

\fonte{Domínio público.}
\end{quote}

Os verbos de enunciação presentes nesse texto relacionam-se de modo que o segundo indica que

\begin{escolha}
\item a enfermeira havia feito uma pergunta.

\item a enfermeira ficou sem resposta.

\item o médico ficou nervoso com a fala da enfermeira.

\item o médico referia-se ao que lhe foi dito.
\end{escolha}

\coment{SAEB: Analisar os efeitos de sentido de verbos de enunciação. BNCC:
EF05LP10 -- Ler e compreender, com autonomia, anedotas, piadas e cartuns,
dentre outros gêneros do campo da vida cotidiana, de acordo com as
convenções do gênero e considerando a situação comunicativa e a
finalidade do texto.}

\num{4} Leia os trechos.

\begin{quote}
{[}...{]} Quis tapar-lhe a boca. José Dias viu no meu rosto algum
sinal diferente da expressão habitual, e perguntou-me com interesse:

--- Que é, Bentinho?

{[}...{]}

\fonte{Joaquim Maria Machado de Assis. \emph{Dom Casmurro.}
Disponível em:
\emph{https://machado.mec.gov.br/obra-completa-lista/item/download/13\_7101e1a36cda79f6c97341757dcc4d04}.
Acesso em: 26 fev. 2023.}
\end{quote}


\begin{quote}
José Dias (olhando com interesse): Que é, Bentinho?

\fonte{Texto adaptado para este material.}
\end{quote}

A partir da comparação entre os dois trechos, pode-se dizer que a
principal diferença entre o texto narrativo e o texto dramático reside

\begin{minipage}{.5\textwidth}
\begin{escolha}
\item no uso de personagens.

\item na temática explorada.

\item na presença de um narrador.

\item no tipo de linguagem explorado.
\end{escolha}
\end{minipage}
\sidetext{SAEB: Reconhecer diferentes gêneros textuais. BNCC: EF35LP29 -
Identificar, em narrativas, cenário, personagem central, conflito
gerador, resolução e o ponto de vista com base no qual histórias são
narradas, diferenciando narrativas em primeira e terceira pessoas.}

\num{5} Leia o texto.

\begin{quote}
\textbf{Superlaboratório Sirius atrai atenção de cientistas da
Argentina, Grã-Bretanha, Alemanha e EUA}

Acelerador de partículas em Campinas (SP) recebeu propostas de
pesquisa de diversas instituições brasileiras e internacionais.
Trabalhos selecionados serão agendados a partir de março.

{[}...{]} o Sirius surge como uma alternativa, já que a máquina é
mundialmente \textbf{competitiva}, possibilita diferentes tipos de
análise com qualidade e com rapidez” {[}...{]}.

\fonte{Fernando Evans. G1. Superlaboratório Sirius atrai atenção de cientistas da Argentina, Grã-Bretanha, Alemanha e EUA. Disponível em:
\emph{https://g1.globo.com/sp/campinas-regiao/noticia/2023/02/17/superlaboratorio-sirius-atrai-atencao-de-cientistas-da-argentina-gra-bretanha-alemanha-e-eua.ghtml}.
Acesso em: 26 fev. 2023.}
\end{quote}

O adjetivo destacado revela, acerca do laboratório, uma visão

\begin{minipage}{.5\textwidth}
\begin{escolha}
\item duvidosa.

\item neutra.

\item positiva.

\item negativa.
\end{escolha}
\end{minipage}
\sidetext{SAEB: Analisar os efeitos de sentido decorrentes do uso dos adjetivos.
Não há correspondência com a BNCC do quinto ano.}

\num{6} Analise esta tabela.


\begin{tabular}{|ll|}
\hline
\multicolumn{2}{|c|}{\begin{tabular}[c]{@{}c@{}}Número de alunos matriculados nos Anos Iniciais\\ Município de Fortaleza (CE)\end{tabular}} \\ \hline
\multicolumn{1}{|l|}{1º ano} & 18.221 \\ \hline
\multicolumn{1}{|l|}{2º ano} & 17.850 \\ \hline
\multicolumn{1}{|l|}{3º ano} & 18.435 \\ \hline
\multicolumn{1}{|l|}{4º ano} & 19.513 \\ \hline
\multicolumn{1}{|l|}{5º ano} & 20.103 \\ \hline
\end{tabular}

\fonte{Fonte de pesquisa: QEdu, 2021. Disponível em: \emph{https://qedu.org.br/municipio/2304400-fortaleza/censo-escolar}. Acesso em: 28 mar. 2023.}

Segundo a tabela, o ano escolar, entre os dos Anos Iniciais, com menos matriculados é o

\begin{minipage}{.5\textwidth}
\begin{escolha}
\item 1º ano.

\item 2º ano.

\item 3º ano.

\item 4º ano.
\end{escolha}
\end{minipage}
\sidetext{SAEB: Analisar informações apresentadas em gráficos, infográficos ou
tabelas. BNCC: EF05LP23 -- Comparar informações apresentadas em gráficos
ou tabelas.}

\num{7} Leia o texto a seguir e responda à pergunta:

\begin{quote}
O mini-helicóptero Ingenuity, da Nasa, entrou para o “Guinness World
Records” com o voo mais longo realizado na superfície marciana.

{[}...{]}

O Ingenuity, que se assemelha a um drone, pesa 1,8kg e chegou a Marte
dobrado e acoplado à parte inferior do Perseverance, robô da Nasa que
pousou no planeta em fevereiro de 2021.

{[}...{]}

“Temos confiança de que podemos contar com o Perseverance para trazer
as amostras de volta e adicionamos os helicópteros como uma espécie de
plano B”, disse Gramling.

{[}...{]}.

\fonte{G1. Helicóptero da Nasa entra para o “Guinness World Records” com o voo mais longo em Marte. Disponível em: \emph{https://g1.globo.com/ciencia/noticia/2023/02/14/helicoptero-da-nasa-entra-para-o-guinness-world-records-com-o-voo-mais-longo-em-marte.ghtml}. Acesso em: 26 fev. 2023.}
\end{quote}

No texto, é uma opinião que

\begin{escolha}
\item o Perseverance vai trazer para a Terra amostras de Marte.

\item o Ingenuity realizou o voo mais longo na superfície de Marte.

\item o Ingenuity é um mini-helicóptero da Nasa.

\item o robô Perseverance pousou em Marte no ano de 2021.
\end{escolha}

\coment{SAEB: Distinguir fatos de opiniões em textos. BNCC: EF05LP16 -- Comparar
informações sobre um mesmo fato veiculadas em diferentes mídias e
concluir sobre qual é mais confiável e por quê.}

\num{8}  Leia o texto.

O instrumento musical afoxé é composto de uma cabaça coberta por uma
rede formada por sementes, miçangas ou contas.

%Paulo: inserir esta imagem: https://commons.wikimedia.org/wiki/Category:Afox%C3%A9#/media/File:Abe_agbe_afoxe.jpg

O som desse instrumento é
produzido quando se gira a rede em um sentido e a base do instrumento (a
cabaça) no sentido oposto.

Assinale a alternativa que corresponde à forma como o som do afoxé é
produzido.

\begin{escolha}
\item
  Produz som pela vibração do instrumento.
\item
  Produz som pela vibração de cordas.
\item
  Produz som pela vibração de uma membrana esticada em algum suporte.
\item
  Produz som pela vibração do ar no seu interior.
\end{escolha}

\coment{SAEB: Identificar as características de instrumentos musicais
variados, bem como o potencial musical do corpo humano.}

\num{9}  Leia o texto.

Arte afro-brasileira é uma manifestação que retoma a estética e religiosidade africanas tradicionais, além dos contextos socioculturais do negro no Brasil.

Assinale a alternativa que contém um exemplo de arte afro-brasileira.

%\begin{longtable}[]{@{}l@{}}
%\toprule
%\begin{minipage}[b]{0.97\columnwidth}\raggedright\strut
%\begin{escolha}
%\item
%\end{escolha}
%
%\includegraphics[width=1.37789in,height=1.37500in]{media/image4.png}
%
%Máscara dos yakas, grupo étnico de Angola. Museu Afro Brasil.
%
%\emph{https://commons.wikimedia.org/wiki/File:M\%C3\%A1scara_(povo_yaka)_(01).jpg}\strut
%\end{minipage}\tabularnewline
%\midrule
%\endhead
%\begin{minipage}[t]{0.97\columnwidth}\raggedright\strut
%b)
%
%\includegraphics[width=2.73958in,height=1.83446in]{media/image5.png}
%
%Cortejo de maracatu. Recife, Brasil.
%
%\emph{https://commons.wikimedia.org/wiki/File:Encontro_estadual_de_maracatus.jpg}\strut
%\end{minipage}\tabularnewline
%\begin{minipage}[t]{0.97\columnwidth}\raggedright\strut
%c)
%
%Painel de porta (1910-1914), povo iorubá, Nigéria, África.
%
%\emph{https://commons.wikimedia.org/wiki/File:Pair_of_door_panels_and_lintel_Yoruba_BM.jpg?uselang=pt}\strut
%\end{minipage}\tabularnewline
%\begin{minipage}[t]{0.97\columnwidth}\raggedright\strut
%d)
%
%Escultura yumbe, da República Democrática do Congo, África.
%
%\emph{https://commons.wikimedia.org/wiki/File:African_Art,_Yombe_sculpture,_Louvre.jpg}\strut
%\end{minipage}\tabularnewline
%\bottomrule
%\end{longtable}

\coment{SAEB: Reconhecer a influência de distintas matrizes estéticas e
culturais nas manifestações das artes visuais, dança, música e teatro na
cultura brasileira.
BNCC: EF15AR03 -- Reconhecer e analisar a influência de distintas
matrizes estéticas e culturais das artes visuais nas manifestações
artísticas das culturas locais, regionais e nacionais.}

\num{10} Assinale a alternativa que apresenta movimento corporal próprio das danças do hip-hop.

%\begin{longtable}[]{@{}l@{}}
%\toprule
%\begin{minipage}[b]{0.97\columnwidth}\raggedright\strut
%\begin{escolha}
%\item
%\end{escolha}
%
%Editora, borrar as referências de marcas na imagem.
%
%\emph{https://live.staticflickr.com/4069/4438178315_7d9c4cdea2_b.jpg}\strut
%\end{minipage}\tabularnewline
%\midrule
%\endhead
%\begin{minipage}[t]{0.97\columnwidth}\raggedright\strut
%b)
%
%Editora, borrar o texto da faixa.
%
%\emph{https://live.staticflickr.com/2800/4345161183_4cf633fd9b_b.jpg}\strut
%\end{minipage}\tabularnewline
%\begin{minipage}[t]{0.97\columnwidth}\raggedright\strut
%c)
%
%Editora, borrar as referências de marcas na imagem.
%
%\emph{https://commons.wikimedia.org/wiki/File:Thai_Breakdancers.jpg}\strut
%\end{minipage}\tabularnewline
%\begin{minipage}[t]{0.97\columnwidth}\raggedright\strut
%d)
%
%\emph{https://commons.wikimedia.org/wiki/File:Chica_haciendo_posiciones_de_ballet.jpg?uselang=pt}\strut
%\end{minipage}\tabularnewline
%\bottomrule
%\end{longtable}

\coment{SAEB: Analisar relações entre as partes corporais e seu todo na
estética da dança.}


%\begin{figure}[htpb!]
%\vspace*{-3cm}
%\hspace*{-2.5cm}\includegraphics[scale=1]{../watermarks/2simulado5ano.pdf}
%\end{figure}
%
%\newwatermark[pagex={61,63}]{\vspace{2.5cm}\hspace*{8cm}\includegraphics[scale=1]{../watermarks/%bgsim5anoimpar.pdf}}
%\newwatermark[pagex={62}]{\vspace{2.5cm}\hspace*{7.8cm}\includegraphics[scale=1]{../watermarks/%bgsim5anopar.pdf}}
%
%\pagebreak
%\movetooddpage
%\markboth{Simulado 2}{}

\num{11} A Cetesb iniciou nesta quinta-feira, dia 16/04, a operação da nova
estação de monitoramento da qualidade do ar, instalada na escola de
educação infantil Prof. Zeferino Vaz, em Campinas. A unidade foi
adquirida com recursos de compensação ambiental do Aeroporto
Internacional de Viracopos, em função das obras de ampliação. {[}\ldots{}{]} A estação
vai avaliar a qualidade do ar da cidade, levando em consideração os
possíveis impactos de Viracopos, das rodovias Anhanguera, Bandeirantes e
Santos Dumont, bem como o conhecimento dos processos de transporte de
poluentes oriundos de outras regiões. {[}\ldots{}{]}

\fonte{Governo do estado de São Paulo. Secretaria de Meio Ambiente, Infraestrutura e Logística. Disponível em:
\emph{https://www.infraestruturameioambiente.sp.gov.br/2015/04/cetesb-inicia-hoje-operacao-da-nova-estacao-de-monitoramento-do-ar-de-campinas/}.
Acesso em: 23 fev. 2023.}

A medida no município de Campinas, em São Paulo, é
importante para

\begin{escolha}
\item entender o impacto dos transportes na cidade.

\item diminuir a circulação de produtos na região.

\item avaliar os resultados de políticas ambientais.

\item melhorar a qualidade da educação nas escolas.
\end{escolha}

\coment{BNCC: EF05GE12 - Identificar órgãos do poder público e canais de participação
social responsáveis por buscar soluções para a melhoria da qualidade de
vida (em áreas como meio ambiente, mobilidade, moradia e direito à
cidade) e discutir as propostas implementadas por esses órgãos que
afetam a comunidade em que vive.}

\num{12} O Brasil é dividido em cinco regiões. Identifique cada uma conforme sua numeração no mapa abaixo:


\begin{escolha}
\item 1 - Nordeste, 2 - Centro-Oeste, 3 - Sudeste, 4 - Sul, 5 - Norte.

\item 1 - Norte, 2 - Centro-Oeste, 3 - Sul, 4 - Sudeste, 5 - Nordeste.

\item 1 - Nordeste, 2 - Sul, 3 - Norte, 4 - Centro-Oeste, 5 - Sudeste.

\item 1 - Norte, 2 - Sul, 3 - Centro-Oeste, 4 Sudeste, 5 - Nordeste.
\end{escolha}

\begin{figure}[htpb!]
\includegraphics[width=.5\textwidth]{./imgs/img67.png}
\end{figure}
%Para o ilustrador: produzir um mapa do Brasil dividido pelas 5 regiões(1 - Norte, 5 - Nordeste, 3 - Centro-Oeste, 4 - Sudeste e 2 - Sul) e numerar dentro de cada região com o número indicado anteriormente.

\coment{BNCC: EF05HI02 - Identificar os mecanismos de organização do
poder político com vistas à compreensão da ideia de Estado e/ou de
outras formas de ordenação social.}

\pagebreak
\num{13}

\begin{quote}
Entre 2003 e 2013 diversas cidades brasileiras viram protestos sobre a
gratuidade do transporte coletivo. {[}\ldots{}{]} Em 2013 o aumento da passagem em São
Paulo levou milhares de manifestantes às ruas. O Movimento Passe Livre
foi um dos grupos responsáveis pela divulgação e pela extensão que os
protestos ganharam. {[}\ldots{}{]} Atualmente, no Brasil, diversas cidades adotam algum
tipo de isenção ou redução da tarifa principalmente para idosos,
deficientes ou estudantes. Algumas até aderiram à gratuidade integral,
sendo Volta Redonda (RJ), com 257.803 habitantes, e Maricá (RJ) 143.111
habitantes, as representantes mais populosas. {[}\ldots{}{]}

\fonte{Estadão. Passe Livre: conheça a história do movimento. Disponível em:
\emph{https://summitmobilidade.estadao.com.br/compartilhando-o-caminho/passe-livre-conheca-a-historia-do-movimento/}.
Acesso em: 23 fev. 2023.}
\end{quote}

\noindent{}O Movimento Passe Livre, segundo o trecho, reivindicava

\begin{escolha}
\item ajudas do governo na compra de carros particulares.

\item melhorias das condições de trabalho dos motoristas.

\item utilização do transporte público de maneira gratuita.

\item reformas dos asfaltos das grandes metrópoles do país.
\end{escolha}

\coment{BNCC: EF05HI05 - Associar o conceito de cidadania à conquista
de direitos dos povos e das sociedades, compreendendo-o como conquista
histórica.}

\pagebreak
\num{14}

\begin{quote}
Campo Grande é, historicamente, a região do Rio de Janeiro com maior
potencial de crescimento, por várias razões. Como está situado em região de limite do município, o lugar teve sua planície atravessada desde muito tempo. Além disso, lá há muitas reservas de água e muitas pessoas com interesses empreendedores. Atualmente, a economia local é composta por cerca de 3.700 estabelecimentos. 
\end{quote}

Segundo o texto, o bairro Campo Grande, no Rio de Janeiro, tem potencial
de crescimento por causa de sua

\begin{escolha}
\item boa localização e interesse econômico.

\item qualidade educacional e beleza natural.

\item atração turística e importância histórica.

\item riqueza mineral e produção gastronômica.
\end{escolha}

\coment{BNCC: EF05HI0 - Identificar os processos de
formação das culturas e dos povos, relacionando-os com o espaço
geográfico ocupado.}

\num{15}

\begin{quote}
\textbf{Ingredientes para receita de Hambúrguer:}\\
200 gramas de carne moída\\
1 tomate\\
100 gramas de alface\\
1 cebola\\
1 pão
\end{quote}

\noindent{}De quais setores de trabalho precisamos para conseguir os ingredientes
para preparar uma receita de hambúrguer em casa?

\begin{minipage}{0.5\textwidth}
\begin{escolha}
\item agricultura e pecuária.

\item agricultura e turismo.

\item pecuária e mineração.

\item pecuária e turismo.
\end{escolha}
\end{minipage}
\sidetext{BNCC: EF05GE05 - Identificar e comparar as mudanças dos tipos de trabalho
e desenvolvimento tecnológico na agropecuária, na indústria, no comércio
e nos serviços.}

\colorsec{Respostas}

\begin{enumerate}
\item
a) Incorreta. O trecho faz parte de um diálogo.
b) Incorreta. O trecho é uma fala da personagem.
c) Correta. O trecho contém instruções de cena específicas para o ator.
d) Incorreta. O trecho faz parte da fala de D. Carlota.

\item
a) Correta. O verbo “perguntar” está relacionado a uma fala do personagem.
b) Incorreta. O verbo “ficar” não é um verbo de enunciação.
c) Incorreta. O verbo “ver” não faz referência a uma fala.
d) Incorreta. O verbo “poder” não remete a um diálogo.

\item
a) Correta. A enfermeira disse algo ao médico, mas não foi uma pergunta.
b) Incorreta. A enfermeira não fez uma pergunta, mas, de qualquer modo, o médido fez um comentário com ela relacionado ao que ela tinha dito.
c) Incorreta. Não há indícios de que o médico tenha ficado nervoso, pois o verbo “responder” não dá essa indicação.
d) Incorreta. De fato, o médico dirigia-se à enfermeira, fazendo um comentário ao que ela tinha dito, e isso se indica por meio do verbo “responder”.

\item
a) Incorreta. Ambos os tipos de texto apresentam personagens.
b) Incorreta. Os dois tipos de texto podem apresentar a mesma temática.
c) Correta. O texto narrativo, ao contrário do dramático, apresenta um narrador.
d) Incorreta. Os registros linguísticos podem ser os mesmos nos dois tipos de texto.

\item
a) Incorreta. O adjetivo revela uma visão clara a respeito do assunto.
b) Incorreta. Uma visão neutra a respeito do assunto não utilizaria um adjetivo positivo.
c) Correta. O adjetivo “competitiva” revela um atributo positivo.
d) Incorreta. O adjetivo atribui um sentido oposto.

\item
a) Incorreta. O 1º ano é o segundo colocado na ordem crescente.
b) Correta. O número de matriculados no 2º ano, 17.850, é o menor que aparece na tabela.
c) Incorreta. O 3º ano é o terceiro colocado na ordem crescente.
d) Incorreta. O 4º ano é o segundo colocado na ordem decrescente.

\item
a) Correta. De fato, essa é uma opinião em que acredita o especialista citado no texto.
b) Incorreta. O Ingenuity, de fato, realizou o voo mais longo na superfície de Marte; trata-se de um fato.
c) Incorreta. Segundo o texto, o Ingenuity é mesmo um mini-helicóptero da Nasa.
d) Incorreta. É um fato expresso no texto que o robô Perseverance pousou em Marte no ano de 2021.

\item
a) Correta. O afoxé é um instrumento musical classificado como idiofone,
ou seja, que produz som por meio da vibração de seu próprio corpo.
b) Incorreta. Os instrumentos musicais que produzem som por meio da
vibração das cordas são os cordofones, como o violão.
c) Incorreta. Os instrumentos musicais que produzem som por meio de uma
membrana são os membrafones, como o tamborim.
d) Incorreta. Os instrumentos que produzem som por meio da vibração do ar
são os aerofones, como o trombone.

\item
a) Incorreta. A máscara dos yakas, apesar de fazer parte do acervo do
Museu Afro-Brasil, apresenta estética de matriz africana, mas faz parte
do contexto sociocultural do negro na África.
b) Correta. O maracatu é uma manifestação cultural afro-brasileira que
apresenta elementos estéticos de matriz africana e que surgiu no estado
de Pernambuco, no século XVIII, no contexto sociocultural do período
colonial brasileiro.
c) Incorreta. O painel de porta do povo iorubá apresenta estética de
matriz africana, mas não faz parte do contexto sociocultural do negro no
Brasil.
d) Incorreta. A escultura yumbe apresenta estética de matriz africana,
mas faz parte do contexto sociocultural do negro na África.

\item
a) Incorreta. Trata-se de um movimento corporal característico da dança do frevo.
b) Incorreta. Trata-se de um movimento corporal característico da dança do tango.
c) Correta. Trata-se de um movimento corporal característico da estética da dança \textit{hip
hop}: pernas para cima, cabeça para baixo, como apoio as mãos, cabeça ou ombros.
d) Incorreta. Trata-se de movimento característico da dança balé.

\item
a) Correta. A medida é importante para avaliar o impacto dos transportes
no ar e no meio ambiente do município.
b) Incorreta. Não há um interesse de diminuição da circulação de
produtos na região.
c) Incorreta. Não existe uma política ambiental sendo avaliada, mas sim
um impacto do uso de transportes.
d) Incorreta. Não há impactos sobre a qualidade da educação nas escolas.

\item
A alternativa correta é a d), cuja ordem corresponde às regiões.

\item
a) Incorreta. O texto fala sobre transporte público e não particular.
b) Incorreta. O texto não fala sobre as condições dos motoristas.
c) Correta. O texto mostra que o movimento queria a gratuidade do
transporte público.
d) Incorreta. O texto não fala sobre a qualidade das vias.

\item
a) Correta. O texto fala sobre sua boa localização - limites do
município - e sobre o interesse de empresários em seu potencial de
crescimento.
b) Incorreta. Não há menção à educação nem à beleza da paisagem.
c) Incorreta. Não há menção ao turismo ou ao passado do bairro.
d) Incorreta. Não há menção a riquezas minerais nem a um polo
gastronômico.

\item
a) Correta. Precisamos da pecuária para as carnes e da agricultura para
os demais ingredientes.
b) Incorreta. Não precisamos do turismo.
c) Incorreta. Não precisamos da mineração.
d) Incorreta. Não precisamos do turismo.
\end{enumerate}

