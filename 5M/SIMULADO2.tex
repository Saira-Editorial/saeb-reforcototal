\setcounter{chapter}{0}
\chapter[Simulado 2]{Simulado}

\num{1} Utilizando um ábaco Miguel representou o seguinte número:

%Pergunta: os alunos conhecem o ábaco? A ilustração é adequada e suficientemente autoexplicativa? Tenho para mim que a resposta a todas essas perguntas é NÃO. 

\begin{figure}[htpb!]
\includegraphics[width=.5\textwidth]{./imgs/mat20.png}
\end{figure}

Qual foi o número que Miguel representou?

\begin{minipage}{.5\textwidth}
\begin{escolha}
\item
  1.314
\item
  4.131
\item
  10.314
\item
  41.301
\end{escolha}
\end{minipage}


\num{2} Isabeli quer colocar o número 380 em uma reta numérica igual à
representada abaixo:

\begin{figure}[htpb!]
\includegraphics[width=\textwidth]{./imgs/mat21.png}
\end{figure}

Entre quais números da reta Isabeli deverá colocar o
número?

%\begin{minipage}{.5\textwidth}
\begin{escolha}
\item
  Entre 150 e 200.
\item
  Entre 250 e 300.
\item
  Entre 350 e 400.
\item
  Entre 450 e 500. 
\end{escolha}
%\end{minipage}


\num{3} Um lote de 26.104 lápis será embalado em caixas contendo 13
unidades de lápis em cada. Essas caixas serão distribuídas uma para cada
escola estadual que existe na região em que Lucas mora. Quantas escolas
receberão 1 caixa contendo lápis?

\begin{minipage}{.5\textwidth}
\begin{escolha}
\item
  26
\item
  28
\item
  208
\item
  2.008
\end{escolha}
\end{minipage}




\num{4} Analisando a sequência abaixo pode-se afirmar que o próximo número
será:

(240; 120; 60; 30; ...)

\begin{minipage}{.5\textwidth}
\begin{escolha}
\item
  20
\item
  15
\item
  10
\item
  5
\end{escolha}
\end{minipage}




\num{5} Raquel completará 11 anos daqui 5 semanas e 2 dias. Quantos dias
faltam para ela completar 12 anos?

\begin{minipage}{.5\textwidth}
\begin{escolha}
\item
  37
\item
  27
\item
  17
\item
  7
\end{escolha}
\end{minipage}


\pagebreak
\num{6} Um marceneiro quer medir a tábua abaixo, mas esqueceu sua trena.
Dessa forma resolveu usar seu próprio palmo, que mede aproximadamente 
21 cm, como unidade de medida.

\begin{figure}[htpb!]
\includegraphics[width=\textwidth]{../ilustracoes/MAT5/SAEB_5ANO_MAT_figura117.png}
\end{figure}

Sabendo-se que ele chegou à conclusão de que a tábua possui o comprimento
de 7 palmos, podemos afirmar que a tábua terá uma medida aproximada
de:

\begin{minipage}{.5\textwidth}
\begin{escolha}
\item
  1,10 m
\item
  1,40 m
\item
  1,50 m
\item
  1,60 m
\end{escolha}
\end{minipage}



\num{7} Jonas está marcando, com uma fita, no chão, a letra inicial
do nome de sua mãe.

\begin{figure}[htpb!]
\centering
\includegraphics[width=.5\textwidth]{../ilustracoes/MAT5/SAEB_5ANO_MAT_figura118.png}
\end{figure}

\pagebreak
Sabendo-se que cada lado do quadrado que forma o piso mede 1,2 m de
comprimento, quantos metros de fita Jonas precisará para concluir seu
trabalho?

\begin{minipage}{.5\textwidth}
\begin{escolha}
\item
  18
\item
  12
\item
  10
\item
  9
\end{escolha}
\end{minipage}


\num{8} Vanessa foi à loja de material escolar e comprou os seguintes
itens pelo respectivo preço indicado na figura abaixo:

\begin{figure}[htpb!]
\centering
\includegraphics[width=.5\textwidth]{../ilustracoes/MAT5/SAEB_5ANO_MAT_figura119.png}
\end{figure}

Qual foi o valor da compra realizada por Vanessa?

\begin{minipage}{.5\textwidth}
\begin{escolha}
\item
  R\$ 92,80
\item
  R\$ 101,80
\item
  R\$ 132,80
\item
  R\$ 173,80
\end{escolha}
\end{minipage}


\pagebreak
\num{9} Amanda acaba de jogar um dado, honesto, de 6 faces, em que cada
face contém um número natural distinto de 1 a 6. Qual a probabilidade de,
na face voltada para cima, sair um número menor ou igual a 6?

\begin{minipage}{.5\textwidth}
\begin{escolha}
\item
  0\%
\item
  25\%
\item
  50\%
\item
  100\%
\end{escolha}
\end{minipage}


\num{10} Três alunos realizaram 5 provas cada um e as notas obtidas por
eles se encontram na tabela abaixo:

\begin{center}
\begin{tabular}{c|ccccc}
\hline
\multicolumn{1}{l|}{\textbf{Aluno}} & \multicolumn{1}{l}{\textbf{1ª prova}} & \multicolumn{1}{l}{\textbf{2ª prova}} & \multicolumn{1}{l}{\textbf{3ª prova}} & \multicolumn{1}{l}{\textbf{4ª prova}} & \multicolumn{1}{l}{\textbf{5ª prova}} \\ \hline
W & 3 & 4 & 5 & 8 & 7 \\ \hline
X & 5 & 5 & 5 & 10 & 6 \\ \hline
Y & 4 & 9 & 3 & 9 & 5 \\ \hline
Z & 5 & 5 & 8 & 5 & 6 \\ \hline
\end{tabular}
\end{center}

O aluno que será classificado será aquele que tiver a maior
soma de todas as notas, pode-se afirmar que o aluno classificado será o
aluno:

\begin{minipage}{.5\textwidth}
\begin{escolha}
\item
  W
\item
  X
\item
  Y
\item
  Z
\end{escolha}
\end{minipage}


\pagebreak
\num{11} Gabriel ganhou de sua avó uma barra de chocolate conforme a 
figura abaixo:

\begin{figure}[htpb!]
\centering
\includegraphics[width=.2\textwidth]{../ilustracoes/MAT5/SAEB_5ANO_MAT_figura120.png}
\end{figure}

O número de quadradinhos que ele deverá comer para consumir 2/3 do total
da barra de chocolate?

\begin{minipage}{.5\textwidth}
\begin{escolha}
\item
  3
\item
  9
\item
  12
\item
  15
\end{escolha}
\end{minipage}



\num{12} Maria é especialista em fazer um café delicioso. Na receita 
que ela utiliza são utilizadas uma colher de sopa de pó de café para cada
250 ml de água. Se você, utilizando a receita de Maria, pretende
utilizar 750 ml de água, quantas colheres de sopa de pó de café você
deverá utilizar para seguir a receita de Maria?

\begin{minipage}{.5\textwidth}
\begin{escolha}
\item
  2
\item
  3
\item
  4
\item
  5
\end{escolha}
\end{minipage}


\pagebreak
\num{13} Para uma competição de xadrez foram inscritos 10 jogadores.
Quantas são as possibilidades de se formar o pódio com o resultado 
final, ou seja, primeiro, segundo e terceiro lugares?
%Acho que a formulação desse enunciado não é clara, mas não consigo redigir melhor. Além disso, sugiro justificativa mais extensa nos gabaritos. Será que a leitura crítica pode resoolver esse problema? (Rogério, 4/3/23, 14h12)

\begin{minipage}{.5\textwidth}
\begin{escolha}
\item
  90
\item
  360
\item
  720
\item
  1 000
\end{escolha}
\end{minipage}


\num{14} Pela manhã, Ricardo abasteceu seu carro, pois o tanque estava
totalmente vazio. Ele gastou R\$ 191,88 para encher o tanque
completamente.

Sabendo-se que o preço do litro do combustível utilizado por Ricardo
custa R\$ 3,69, quantos litros de combustível couberam no carro de
Ricardo?

\begin{minipage}{.5\textwidth}
\begin{escolha}
\item
  25
\item
  34
\item
  46
\item
  52
\end{escolha}
\end{minipage}



\pagebreak
\num{15} Leia um trecho de artigo científico.
\begin{quote}
  {[}\ldots{}{]} no desenvolvimento da dança são encontrados vários
  descaminhos, entre eles estão os fatores que apontam para a exclusão
  da dança nos planejamentos de educação física {[}\ldots{}{]}

{[}\ldots{}{]} perguntamos se acham que exista algum preconceito dos alunos a
respeito do conteúdo dança e {[}\ldots{}{]} pedimos para dizer quais os
preconceitos encontrados, e 100\% deles responderam que o maior
preconceito está ligado ao gênero por parte dos meninos.

\fonte{Vinicius Giacomini de Castro, Diogo Santos Silva e Marli das Graças Júlio. EFDEPORTES. O preconceito da dança nas escolas.
Revista Digital. Buenos Aires, Año 15, Nº 150, Noviembre de 2010.
Disponível em: \emph{
https://www.efdeportes.com/efd150/o-preconceito-da-danca-nas-escolas.htm}.
Acesso em: 14 fev. 2023.}
\end{quote}

\noindent{}Com base no texto, o pensamento estereotipado na dança é achar que é um(a)

\begin{escolha}
\item modalidade desconhecida por parte dos alunos.

\item esporte evitado na escola.

\item atividade de que os homens não podem participar.

\item prática corporal voltada para mulheres.
\end{escolha}



\num{16} Leia o texto.
\begin{quote}\enlargethispage{\baselineskip}
  Acontece na próxima sexta-feira, {[}\ldots{}{]} no Ginásio Municipal de
  Esportes Domingos Angelino Régis, no Centro de Navegantes, um evento
  direcionado aos alunos das 8ª Séries da Rede Municipal de Ensino, que
  tem por objetivo despertar nos estudantes a importância do esporte
  como mecanismo de motivação, superação e combate ao preconceito. O
  evento também vai contar com a participação de atletas do paradesporto
  e da Apae de Navegantes.

{[}\ldots{}{]} no local haverá uma apresentação das equipes de Basquete e
Handebol do Clube Roda Solta {[}\ldots{}{]}.
\end{quote}

\fonte{Prefeitura de Navegantes. Alunos participam de evento sobre motivação e superação através do esporte. Disponível em: \emph{https://bit.ly/3MfXCUl}.
Acesso em: 15 fev. 2023.}

\noindent{}O evento citado, para os alunos, serviu para que eles

\begin{escolha}
\item praticassem novos esportes.

\item promovessem a conscientização dos paratletas.

\item entendessem os benefícios dos esportes.

\item ajudassem na organização do evento.
\end{escolha}



\num{17} Leia sobre as cantigas de roda.
\begin{quote}
  {[}\ldots{}{]} caso das cantigas de roda que, historicamente fazem parte
  das tradicionais brincadeiras infantis {[}\ldots{}{]}

{[}\ldots{}{]} Ficou claro que a cantiga de roda é inserida em sala de aula
para promover o lúdico para a criança. {[}\ldots{}{]} Nem todas as crianças
sabem cantar muitas músicas que são tidas como tradicionais. Isso porque
o envolvimento das mesmas com tecnologias pode as estar afastando de
tradições ricas e importantes como são as cantigas de roda. {[}\ldots{}{]}

\fonte{UFG. Patrimônio, direitos culturais e cidadania. As cantigas de roda como manifestações do patrimônio cultural: o papel
da escola na perpetuação dessa cultura. Disponível em: \emph{
https://publica.ciar.ufg.br/ebooks/eipdcc-propostas-pratica-acoesdialogicas/artigos/artigo34.html}.
Acesso em: 15 fev. 2023.}
\end{quote}

\noindent{}Com base no texto, podemos perceber que a brincadeira tradicional citada

\begin{escolha}
\item está sendo esquecida por parte dos alunos.

\item vem ganhando popularidade por causa da tecnologia.

\item apresenta algumas desvantagens para os estudantes.

\item aparece como uma atividade pouco usada na escola.
\end{escolha}




\num{18} O coração, sangue e vasos sanguíneos compõem a estrutura do
sistema circulatório. Em conjunção com outros sistemas do organismo, o
sistema circulatório participa da nutrição das células do corpo humano,
transportando nutrientes para músculos e órgãos.

Esse sistema também tem por função a

\begin{minipage}{.5\textwidth}
\begin{escolha}
\item produção de gases.

\item eliminação de toxinas.

\item maturação de nutrientes.

\item digestão de alimentos.
\end{escolha}
\end{minipage}


\num{19} Um relatório de especialistas descreveu que, às margens de
um rio brasileiro, há pouca concentração vegetal, e uma extensa
deterioração do solo, tornando-o improdutivo. Além disso, o acúmulo de
lixo na região é muito grande. Eles concluíram que a área foi afetada
pela atividade humana e entregaram uma série de medidas para as
autoridades, visando evitar maiores problemas.

\begin{longtable}[]{@{}ll@{}}
\toprule
\textbf{Medida} & \textbf{Justificativa}\tabularnewline
Cultivo de plantas leguminosas. & Adubar o solo
empobrecido.\tabularnewline
Instalação de barreiras de contenção. & Impedir o acúmulo de detritos no
rio.\tabularnewline
Remoção do lixo acumulado na região. & Prevenir o transporte do lixo até
o rio.\tabularnewline
Plantio de árvores no solo recuperado. & Reflorestar o entorno do
rio.\tabularnewline
\bottomrule
\end{longtable}

Tais medidas buscam prevenir o processo de

\begin{minipage}{.5\textwidth}
\begin{escolha}
\item ressecamento de nascentes.

\item cruzamento de correntes oceânicas.

\item escoamento das águas.

\item assoreamento do rio.
\end{escolha}
\end{minipage}


\num{20} Em 21/12/2022, o Brasil teve o dia mais longo e a noite
mais curta do ano. Esse período marcou a passagem da primavera para o
verão no país, graças à maior incidência de luz do sol no Hemisfério Sul
e à inclinação do planeta Terra. Esse fenômeno é conhecido como
solstício de verão.

No Hemisfério Norte, qual estação do ano se iniciou no mesmo dia?

\begin{minipage}{.5\textwidth}
\begin{escolha}
\item Primavera.

\item Outono.

\item Verão.

\item Inverno.
\end{escolha}
\end{minipage}


\pagebreak