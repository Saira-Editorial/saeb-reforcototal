\setcounter{chapter}{0}
\chapter[Simulado 1]{Simulado}


\num{1} Gabriel durante sua aula estava aprendendo a montar números
utilizando o material dourado e montou a seguinte número?

\begin{figure}[htpb!]
\includegraphics[width=\textwidth]{../ilustracoes/MAT5/SAEB_5ANO_MAT_figura112.png}
\end{figure}

Qual é o número representado pelo material dourado na figura acima?

\begin{escolha}
\item
  623
\item
  423
\item
  503
\item
  523
\end{escolha}


\num{2} Alex estava jogando dardos em um alvo que possuía áreas de
pontuação e durante uma rodada notou que a expressão 2 x 1.000 + 3 x 100
+ 1 x 10, quando resolvida, gerava exatamente o número de pontos que ele
havia feito naquela rodada. A pontuação de Alex naquela rodado foi:

\begin{multicols}{2}
\begin{escolha}
\item
  231 pontos
\item
  2.031 pontos
\item
  2.301 pontos
\item
  2.310 pontos
\end{escolha}
\end{multicols}

\pagebreak
\num{3} A escola em que Jack estuda está promovendo uma conscientização de
preservação do meio ambiente através do plantio de mudas de árvores
nativas. Sabe-se que já foram plantadas 359 mudas e ainda serão
plantadas 246. Quantas mudas ao todo serão plantadas durante esse evento
da escola de Jack?

\begin{multicols}{2}
\begin{escolha}
\item
  513
\item
  523
\item
  605
\item
  705
\end{escolha}
\end{multicols}


\num{4} Na escola em que André estuda há 4.054 alunos. Já na escola 
em que Pedro estuda estão matriculados 2.843 alunos. Se, no próximo ano,
300 alunos se matricularem em cada uma das escolas, qual será a diferença
entre a quantidade de alunos das duas escolas?

\begin{multicols}{2}
\begin{escolha}
\item
  2.416
\item
  1.211
\item
  1.883
\item
  1.463
\end{escolha}
\end{multicols}



\num{5} A numeração das salas de um novo prédio comercial de 5 andares
segue a numeração dos andares e a orientação da árvore que se
localiza ao lado do prédio. Assim, em cada andar a numeração começa pela
sala mais próxima da árvore da seguinte maneira: 101, 102, 103; 201, 202, 
203 --- e assim sucessivamente, até o quinto andar.

\begin{figure}[htpb!]
\centering
\includegraphics[width=.6\textwidth]{../ilustracoes/MAT5/SAEB_5ANO_MAT_figura113.png}
\end{figure}

\pagebreak
\noindent{}Qual a numeração da sala do 3º andar que está com a janela fechada?

\begin{escolha}
\item
  301
\item
  302
\item
  303
\item
  304
\end{escolha}


\num{6} Ernesto comprou, para a festa de aniversário de sua filha, 
8 litros de refrigerante e copos descartáveis com capacidade de 200
mililitros cada. Quantos copos, com a capacidade máxima tomada por
refrigerante, poderão ser servidos nessa festa considerando que Ernesto 
não comprará mais refrigerante?

\begin{multicols}{2}
\begin{escolha}
\item
  16
\item
  20
\item
  32
\item
  40
\end{escolha}
\end{multicols}


\num{7} Observe a figura abaixo:

\begin{figure}[htpb!]
\includegraphics[width=.8\textwidth]{./imgs/mat19.png}
\end{figure}

Considerando tudo que está pintado, juntando duas metades de quadradinhos 
para formar um completo, quantos quadradinhos estão pintados?

\begin{multicols}{2}
\begin{escolha}
\item
  24
\item
  26
\item
  29
\item
  34
\end{escolha}
\end{multicols}


\pagebreak
\num{8} Júlio César e Fabrício juntaram todo dinheiro que ganharam de seus
pais no último mês. As quantias estão representadas na figura abaixo:

\begin{figure}[htpb!]
\includegraphics[width=.5\textwidth]{../ilustracoes/MAT5/SAEB_5ANO_MAT_figura114a.png}
\includegraphics[width=.5\textwidth]{../ilustracoes/MAT5/SAEB_5ANO_MAT_figura114b.png}
\end{figure}
%Colocar Ana Beatriz no lugar de Júlio Cesar e Camila no lugar de Fabrício

Somando-se os dois valores, qual o valor total que os dois conseguiram
juntar?

\begin{multicols}{2}
\begin{escolha}
\item
  R\$ 36,40
\item
  R\$ 37,70
\item
  R\$ 74,10
\item
  R\$ 85,20
\end{escolha}
\end{multicols}


\num{9} A mãe de Isabeli está esperando um bebê e hoje será o dia de
descobrir se será menina ou menino. Qual a probabilidade de Isabeli ter
uma irmã?

\begin{multicols}{2}
\begin{escolha}
\item
  0\%
\item
  25\%
\item
  50\%
\item
  100\%
\end{escolha}
\end{multicols}


\num{10} O gráfico abaixo mostra a taxa de desemprego de uma grande cidade
brasileira:

%Não gosto desse gráfico: iniciado em março, ele induz ao erro. O próprio autor da questão caiu nessa armadilha. Tive de refazer o gabarito. 

\begin{figure}[htpb!]
\centering
\includegraphics[width=.8\textwidth]{../ilustracoes/MAT5/SAEB_5ANO_MAT_figura115.png}
\end{figure}

Por meio da análise do gráfico, o mês que apresentou menor índice de
desemprego foi:

\begin{multicols}{2}
\begin{escolha}
\item
  Março
\item
  Julho
\item
  Outubro
\item
  Dezembro
\end{escolha}
\end{multicols}


\num{11} Em uma rua de comprimento AB, seis árvores serão plantadas de 
forma equidistante, em reta numérica, conforme a figura:

\begin{figure}[htpb!]
\includegraphics[width=\textwidth]{../ilustracoes/MAT5/SAEB_5ANO_MAT_figura116.png}
\end{figure}

Qual a fração representada pela distância entre a segunda e a terceira 
árvores em relação ao tamanho total?

\begin{multicols}{2}
\begin{escolha}
\item
  1/4
\item
  2/3
\item
  1/3
\item
  1/5
\end{escolha}
\end{multicols}


\num{12} Juca, a uma velocidade de 80 km/h costuma gastar 1 hora e 30
minutos para ir da cidade em que mora até a cidade em que sua avó mora.
Se ele, em certo dia, reduziu a velocidade para 60 km/h, o tempo que
gastou para ir da casa em que mora até a casa em que sua avó reside foi
de:

\begin{multicols}{2}
\begin{escolha}
\item
  1 horas
\item
  1 horas e 7 minutos
\item
  2 horas
\item
  2 horas e 15 minutos
\end{escolha}
\end{multicols}



\num{13}De quantas maneiras diferentes, uma pessoa pode se vestir tendo à
disposição 10 camisetas e 5 bermudas?

\begin{multicols}{2}
\begin{escolha}
\item
  5
\item
  10
\item
  15
\item
  50
\end{escolha}
\end{multicols}

\pagebreak
\num{14} 141,1 litros de suco de laranja dever ser  
colocados, igualmente, em 17 tambores. Quantos litros de suco de
laranja serão colocados em cada tambor?

\begin{escolha}
\item
  5,3 litros
\item
  6,3 litros
\item
  7,3 litros
\item
  8,3 litros
\end{escolha}


\num{15} Observe a imagem.
  \begin{figure}[htpb!]
  \centering
\includegraphics[width=\textwidth]{./imgs/img14.jpg}
\end{figure}
%Disponível em: https://br.freepik.com/fotos-gratis/lutadores-de-karate-no-campeonato-de-luta-de-tatami\_30182444.htm\#query=jud\%C3\%B4\&position=42\&from\_view=search\&track=sph. Acesso em: 14 fev. 2023.

\noindent{}Observando os dois competidores, podemos perceber que eles estão
demostrando respeito um ao outro. O motivo é que eles estão

\begin{escolha}
\item realizando uma saudação antes de lutar.

\item usando um kimono branco.

\item praticando a luta em uma competição.

\item evitando usar golpes específicos das lutas.
\end{escolha}

\pagebreak
\num{16} Leia sobre os jogos de oposição.
\begin{myquote}
  Os Jogos de Oposição {[}\ldots{}{]} têm como característica o ato de
  confrontação que acontece entre duplas, trios ou até mesmo em grupos.
  Seus objetivos são vencer o adversário, impor-se fisicamente ao outro,
  respeitar as regras e acima de tudo assegurar a segurança do colega
  durante as atividades.

Durante a aplicação dos Jogos de Oposição, precisamos levar em
consideração alguns critérios de segurança para que não ocorram
acidentes. {[}\ldots{}{]}

\fonte{Paraná. Secretaria da Educação. Jogos de Oposição. Disponível em: \emph{
http://www.educacaofisica.seed.pr.gov.br/modules/conteudo/conteudo.php?conteudo=413}.
Acesso em: 14 fev. 2023.}
\end{myquote}

\noindent{}Com base no texto, entende-se que as atividades práticas citadas são voltadas para
lutas, considerando que os participantes devem

\begin{escolha}
\item tentar para ganhar de qualquer maneira.

\item tomar os devidos cuidados para ninguém se machucar.

\item realizar a atividade individualmente.

\item modificar as regras do jogo.
\end{escolha}



\num{17} Leia uma notícia sobre um projeto de lei.
\begin{myquote}
  A Câmara analisa o Projeto de Lei 6933/10, {[}\ldots{}{]} que regulamenta a
  profissão de instrutor de artes marciais. A proposta inclui na
  categoria os profissionais {[}\ldots{}{]} que possuírem certificado de
  instrutor, monitor, professor ou 1° dan (graduação de arte marcial)
  emitido por uma federação ou associação registrada.

O certificado será concedido a quem comprovar a prática do esporte por
pelo menos dois anos e meio. Segundo o projeto, as federações e
associações criarão o código de ética dos profissionais e fiscalizarão o
cumprimento do período mínimo para obtenção do certificado. {[}\ldots{}{]}

\fonte{Câmara dos deputados. Educação, cultura e esportes. Proposta regulamenta profissão de instrutor de artes marciais. Disponível em: \emph{
https://www.camara.leg.br/noticias/143647-PROPOSTA-REGULAMENTA-PROFISSAO-DE-INSTRUTOR-DE-ARTES-MARCIAIS}.
Acesso em: 14 fev. 2023.}
\end{myquote}

\pagebreak
\noindent{}Após a leitura do texto, fica claro que o projeto de lei tem o objetivo de

\begin{escolha}
\item formar novos instrutores de lutas.

\item criar novas federações esportivas de lutas.

\item regulamentar a profissão de professores de lutas.

\item incentivar a prática de lutas.
\end{escolha}



\num{18} O Polígono das Secas é uma região que compreende os estados
brasileiros de Alagoas, Bahia, Ceará, Minas Gerais, Paraíba, Pernambuco,
Piauí, Rio Grande do Norte e Sergipe. Nessas localidades, os
trabalhadores rurais relatam que são comuns os longos períodos de seca.
Sem rios abundantes, as águas locais são encontradas em temperatura
menor que o normal, o que dificulta a evaporação. Toda a produção
agrícola é adaptada para a época de chuva, mais comum em fevereiro.

Uma das explicações para a falta de chuvas da região pode ser

\begin{escolha}
\item o acúmulo de erros de gestão pública.

\item o uso indiscriminado da água.

\item a evaporação acelerada das águas locais.

\item a baixa umidade do ar.
\end{escolha}


\num{19} Nos pulmões, acontece o processo conhecido como hematose. O
gás oxigênio inspirado é incorporado à corrente sanguínea pelos
alvéolos, enquanto o gás carbônico é removido e eliminado na expiração.
Esse processo é fundamental para a produção de energia no organismo,
pois o oxigênio participa de etapas da nutrição do organismo, como na
quebra de nutrientes para geração de energia.

O processo que acontece nos pulmões também pode ser chamado de

\begin{escolha}
\item nutrição.

\item troca gasosa.

\item respiração celular.

\item filtração pulmonar.
\end{escolha}


\pagebreak
\num{20} Uma criança, ao olhar para a Lua pela janela do quarto,
observou a forte iluminação do corpo celeste. Ela concluiu, então, que
as noites são iluminadas pela luz irradiada pela Lua, mas foi
repreendida ao dizer isso para a mãe, que a corrigiu e prometeu
observar, junto a ela, as fases da Lua a cada semana.

A criança estava errada em sua observação, pois

\begin{escolha}
\item as noites sem a Lua seriam mais claras.

\item a Lua não possui iluminação própria.

\item o Sol é um satélite natural mais poderoso.

\item a visualização da Lua sem telescópio é enganosa.
\end{escolha}



\pagebreak