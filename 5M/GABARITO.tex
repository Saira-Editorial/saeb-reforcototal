\pagebreak
\pagestyle{plain}
\scriptsize

\pagecolor{gray!40}

\section*{Matemática -- Módulo 1 -- Treino}

\begin{enumerate}
\item 
SAEB: - Compor ou decompor números naturais de até 6 ordens na forma aditiva, ou em suas ordens, ou em adições e multiplicações.
BNCC: EF05MA01 - Ler, escrever e ordenar números naturais até a ordem das centenas de milhar com
compreensão das principais características do sistema de numeração decimal.
a) Correta. 600 000 + 50 000 + 700 + 30 + 4.
b) Incorreta. Não se trata de 5.000, mas de 50.000; não se trata de 70, mas de 700; não se trata de 3, mas de 30.
c) Incorreta. Não se trata de 500, mas de 50.000.
d) Incorreta. Não se trata de 60.000, nem de 70, nem de 300.

\item
SAEB: - Identificar a ordem ocupada por um algarismo ou seu valor posicional (ou valor relativo) em um número natural de até 6 ordens.
BNCC: EF05MA10 - Concluir, por meio de investigações, que a relação de igualdade existente
entre dois membros permanece ao adicionar, subtrair, multiplicar ou dividir cada um desses
membros por um mesmo número, para construir a noção de equivalência.
a) Incorreta. 570 não pode estar no ponto Q, onde estaria o número 550.
b) Incorreta. No ponto R estaria o número 560, 20 unidades além de 540.
c) Correta. O 570 estará no ponto S, pois, como no ponto P está o 540 e cada
repartição é de 10 unidades, ele deve estar no terceiro ponto após o P
sem contar o ponto S.
d) Incorreta. No ponto T estaria o número 580, 30 unidades além de 540.

\item
SAEB: - Compor ou decompor números naturais de até 6 ordens na forma aditiva, ou em suas ordens, ou em adições e multiplicações.
BNCC: EF05MA10 - Concluir, por meio de investigações, que a relação de igualdade existente
entre dois membros permanece ao adicionar, subtrair, multiplicar ou dividir cada um desses
membros por um mesmo número, para construir a noção de equivalência.
a) Incorreta. Na posição correta seria a representação de 500, não de 50.
b) Incorreta. No número errado da casa de José, o 5 está na casa das unidades.
c) Correta. Como o número apresentado no enunciado está com o primeiro e o último
algarismos trocados, conclui-se que o número correto seria 521. Na placa,
o último algarismo é o 5 e tem valor relativo de 5 unidades, mas no
número correto ele estaria na centena comum, apresentando, então, um valor
relativo de 500 (quinhentos), ou cinco centenas.
d) Incorreta. No número errado se trata de 5 e, no número correto, de 500.

\end{enumerate}

\section*{Matemática -- Módulo 2 -- Treino}

\begin{enumerate}
\item
SAEB: Calcular o resultado de adições ou subtrações envolvendo números naturais de até 6 ordens.
Não há correspondência com a BNCC do quinto ano.
a) Incorreta. Se  o número escondido fosse 128, o resultado seria 289.
b) Correta. 417 -- 105 = 312.
c) Incorreta. Se  o número escondido fosse 158, o resultado seria 259.
d) Incorreta. Se  o número escondido fosse 256, o resultado seria 161.

\item
SAEB: Calcular o resultado de multiplicações ou divisões envolvendo números naturais de até 6 ordens.
Não há correspondência com a BNCC do quinto ano.
a) Incorreta. Para restarem 72 peças, teriam de ter sido vendidas 128.
b) Incorreta. Para restarem 94 peças, teriam de ter sido vendidas 106.
c) Incorreta. Para restarem 126 peças, teriam de ter sido vendidas 74.
d) 200 -- 2 x (1 x 5 + 2 x 7) = 200 -- 38 = 162 peças.

\item
SAEB: Resolver problemas de adição ou de subtração, envolvendo números naturais de até 6 ordens, com os significados de juntar, acrescentar, separar, retirar, comparar ou completar.
Não há correspondência com a BNCC do quinto ano.
a) Incorreta. Para restarem 268 pessoas na fila, deveriam estar aguardando 928 pessoas.
b) Correta.
1.200 -- 540 = 660.
932 -- 660 = 272 pessoas não conseguirão assistir a essa sessão.
c) Incorreta. Para restarem 294 pessoas na fila, deveriam estar aguardando 954 pessoas.
d) Incorreta. Para restarem 1.440 pessoas na fila, deveriam estar aguardando 2.100 pessoas.
\end{enumerate}

\section*{Matemática -- Módulo 3 -- Treino}

\begin{enumerate}
\item
SAEB: Inferir os elementos ausentes em uma sequência de números naturais ordenados, objetos ou figuras.
Não há correspondência com a BNCC do quinto ano.
a) Incorreta. 25 não é o resultado de 6, multiplicado por ele mesmo, mais ele mesmo.
b) Incorreta. 30 será o número de elementos da figura 5.
c) Incorreta. 35 não faz parte da sequência, em nenhuma posição.
d) Correta. (2; 6; 12; 20; 30; 42).

Professor, a sequência é dada por n² + n =
6² + 6 = 42 (sendo n o número da figura. 1² + 1 = 2; 2² + 2 = 6; 3² + 2 = 12; 4² + 4 = 20...)

\item
SAEB: Inferir ou descrever atributos ou propriedades comuns que os elementos que constituem uma sequência recursiva de números naturais apresentam.
Não há correspondência com a BNCC do quinto ano.
a) Incorreta. A sequência é decrescente e finita, ao contrário do que se afirma.
b) Incorreta. A sequência é realmente finita, mas decrescente.
c) Correta. Pela análise da sequência dada, percebemos que ela é finita e
decrescente.
d) Incorreta. A sequência é realmente decrescente, mas finita.

\item
SAEB: Inferir o padrão ou a regularidade de uma sequência de números naturais ordenados, objetos ou figuras.
Não há correspondência com a BNCC do quinto ano.
a) Incorreta. 6.079 é o antecessor do antecessor de 6.081.
b) Incorreta. 6.080 é o antecessor de 6.081.
c) Incorreta. 6.082 é o sucessor de 6.081.
d) Correta. Sucessor do sucessor de 6.081 = 6.081 +1 +1 = 6.083.
\end{enumerate}


\section*{Matemática -- Módulo 4 -- Treino}

\begin{enumerate}
\item
SAEB: Determinar o horário de início, o horário de término ou a duração de um acontecimento.
BNCC: EF05MA19 – Resolver e elaborar problemas envolvendo medidas das grandezas comprimento,
área, massa, tempo, temperatura e capacidade, recorrendo a transformações entre as unidades
mais usuais em contextos socioculturais.
a) Incorreta. Saindo a esse horário ele completará 3h de trabalho.
b) Incorreta. Saindo a esse horário ele completará 3h30min de trabalho.
c) Incorreta. Saindo a esse horário ele completará 4h de trabalho.
d) Correta. Como pela manhã ele entra às 8:00 e deve cumprir nesse período 4 horas e
meia de trabalho antes de sair para o almoço, conclui-se que ele sairá
para o almoço às 12:30.

\item
SAEB: Reconhecer a unidade de medida ou o instrumento mais apropriado para medições de comprimento, área, massa, tempo, capacidade ou temperatura.
BNCC: EF05MA19 – Resolver e elaborar problemas envolvendo medidas das grandezas comprimento,
área, massa, tempo, temperatura e capacidade, recorrendo a transformações entre as unidades
mais usuais em contextos socioculturais.
a) Incorreta. Um frasco contém 100 ml; menos do que a quantidade de que ela precisa.
b) Incorreta. Com a compra de 2 frascos faltará uma dose; ou seja, 10 ml.
c) Correta. 
3 x 10 x 7 = 210 ml. Como cada frasco contém 100 ml, ela terá que
comprar 3 frascos e haverá uma sobra de xarope.
d) Incorreta. Na compra de 4 frascos, um restará fechado e inutilizado; não há essa necessidade.

\item
SAEB: Determinar o horário de início, o horário de término ou a duração de um acontecimento.
BNCC: EF05MA19 – Resolver e elaborar problemas envolvendo medidas das grandezas comprimento,
área, massa, tempo, temperatura e capacidade, recorrendo a transformações entre as unidades
mais usuais em contextos socioculturais.
a) Correta. 
Saída: 10 horas e 42 minutos;
Chegada: 14 horas e 8 minutos;
Tempo de voo: 3 horas e 26 minutos = 206 minutos = 12.300 segundos.
b) Incorreta. Trata-se de mais do que esse tempo.
c) Incorreta. Trata-se de mais do que o dobro desse tempo.
d) Incorreta. Trata-se de algo próximo de cinco vezes esse tempo mencionado.
\end{enumerate}

\section*{Matemática -- Módulo 5 -- Treino}

\begin{enumerate}
\item
SAEB: Medir ou comparar perímetro ou área de figuras planas desenhadas em malha quadriculada.
Não há correspondência com a BNCC do quinto ano.
a) Incorreta. Nesse caso, seriam 4 lados de quadrado.
b) Correta. Ele deverá andar 5 lados de quadrado. Como cada lado de quadrado possui
medida igual a 2 m, ele deverá andar 10 metros.
c) Incorreta. Nesse caso, seriam 6 lados de quadrado.
d) Incorreta. Nesse caso, seriam 7 lados de quadrado.

\item
SAEB: Medir ou comparar perímetro ou área de figuras planas desenhadas em malha quadriculada.
Não há correspondência com a BNCC do quinto ano.
a) Correta. Como cada lado será ampliado em duas vezes, as medidas dos lados do novo
triângulo deverão ser dobradas, ou seja, multiplicadas por 2.
b) Incorreta. Nesse caso, o triângulo seria reduzido.
c) Incorreta. Nese caso, o triângulo sofreria diminuição.
d) Incorreta. Nesse caso, a redução no tamanho do triângulo seria muito significativa.

\item
SAEB: Identificar horas em relógios analógicos ou associar horas em relógios analógicos e digitais. Não há correspondência com a BNCC do quinto ano.
a) Incorreta. Nesse caso, ela teria gastado apenas 15 minutos para se arrumar.
b) Incorreta. Nesse caso, ela teria gastado 25 minutos para se arrumar.
c) Incorreta. Nesse caso, Maria teria se arrumado em 30 minutos.
d) Correta. O relógio está marcando 11 horas e 35 minutos; se acrescentarmos a esse
horário 35 minutos, teremos no relógio 12 horas e 10 minutos.
\end{enumerate}

\section*{Matemática -- Módulo 6 -- Treino}

\begin{enumerate}
\item
SAEB: Resolver problemas que envolvam moedas e/ou cédulas do sistema monetário brasileiro.
Não há correspondência com a BNCC do quinto ano.
a) Correta.
12 x 0,50 + 8 x 0,25 = 6 + 2 = R\$ 8,00. Portanto, 4 notas de 2 reais.
b) Incorreta. 6 notas de 2 reais totalizariam R\$ 12,00.
c) Incorreta. 8 notas de 2 reais totalizariam R\$ 16,00.
d) Incorreta. 20 notas de 2 reais totalizariam R\$ 40,00.

\item
SAEB: Relacionar valores de moedas e/ou cédulas do sistema monetário brasileiro, com base nas imagens desses objetos.
Não há correspondência com a BNCC do quinto ano.
a) Incorreta. R\$ 9,00 é o valor somente das células.
b) Incorreta. R\$ 9,90 seria o valor das células mais R\$ 0,90 em moedas.
c) Incorreta. R\$ 10,10 seria um valor R\$ 0,05 menos que o total encontrado.
d) Correta. 
R\$ 9,00 em cédulas e R\$ 1,15 em moedas. Portanto, no total, ela
encontrou em sua bolsa R\$ 10,15.

\item
SAEB: Resolver problemas que envolvam moedas e/ou cédulas do sistema monetário brasileiro.
Não há correspondência com a BNCC do quinto ano.
a) Incorreta. R\$ 18,00 seriam R\$ 10,00 a menos do que o valor pago.
b) Incorreta. R\$ 22,00 seriam R\$ 6,00 a menos do que o valor pago.
c) Correta. 2 x 5,00 + 1 x 6,00 + 1 x 12,00 = 10,00 + 6,00 + 12,00 = R\$ 28,00.
d) Incorreta. R\$ 30,00 seriam R\$ 2,00 a mais do que o valor pago.
Professor, você pode utilizar este exercício e estimular os alunos a
realizarem outras combinações conforme a preferência de cada um e
encontrarem qual o valor que eles pagariam nessa lanchonete pelo respectivo pedido.
\end{enumerate}

\section*{Matemática -- Módulo 7 -- Treino}

\begin{enumerate}
\item
SAEB: Determinar a probabilidade de ocorrência de um resultado em eventos aleatórios, quando todos os resultados possíveis têm a mesma chance de ocorrer (equiprováveis).
BNCC: EF05MA23 – Determinar a probabilidade de ocorrência de um resultado em eventos aleatórios,
quando todos os resultados possíveis têm a mesma chance de ocorrer (equiprováveis).
a) Incorreta. Nesse caso, trata-se da chance de escolha de apenas um dia da semana.
b) Correta. 
Dias da semana: 7.
Escolhas determinadas: 2.
Probabilidade: 2/7.
c) Incorreta. Essa seria a chance de uma escolha dentre 5 possibilidades.
d) Incorreta. Nesse caso, as chances são maiores do que as reais.

\item
SAEB: Determinar a probabilidade de ocorrência de um resultado em eventos aleatórios, quando todos os resultados possíveis têm a mesma chance de ocorrer (equiprováveis).
BNCC: EF05MA22 – Apresentar todos os possíveis resultados de um experimento aleatório,
estimando se esses resultados são igualmente prováveis ou não.
a) Correta. 
Total de pessoas: 28 + 7 = 35.
Número de garçons: 7.
Probabilidade: 7/35 = 1/5 = 0,2 = 20\%.
b) Incorreta. Trata-de de mais que o dobro das chances reais.
c) Incorreta. Trata-se de 7 chances em 10.
d) Incorreta. Trata-se de totais chances de ocorrência, como se todas as pessoas presentes no restaurantes fossem garçons.

\item
SAEB: Determinar a probabilidade de ocorrência de um resultado em eventos aleatórios, quando todos os resultados possíveis têm a mesma chance de ocorrer (equiprováveis).
BNCC: EF05MA23 - Determinar a probabilidade de ocorrência de um resultado em eventos aleatórios,
quando todos os resultados possíveis têm a mesma chance de ocorrer (equiprováveis).
a) Incorreta. Trata-se de apenas um quarto das chances reais.
b) Incorreta. Trata-se de apenas metade das chances reais.
c) Incorreta. Trata-se de aproximadamente uma chance em três possibilidades.
d) Correta. Como Carol participará apenas do sorteio final, ela pode ser sorteada ou
não e isso no leva a concluir que ela terá 50\% de chance de iniciar a
disputa.
\end{enumerate}

\section*{Matemática -- Módulo 8 -- Treino}

\begin{enumerate}
\item
SAEB: Ler/identificar ou comparar dados estatísticos expressos em gráficos (barras simples ou agrupadas, colunas simples ou agrupadas, pictóricos ou de linhas).
BNCC: EF05MA24 – Interpretar dados estatísticos apresentados em textos, tabelas e gráficos (colunas
ou linhas), referentes a outras áreas do conhecimento ou a outros contextos, como saúde e
trânsito, e produzir textos com o objetivo de sintetizar conclusões.
a) Incorreta. No dia 05/03 foram vendidos apenas 18 sanduíches.
b) Incorreta. No dia 06/03 foram vendidos apenas 16 sanduíches.
c) Correta. O dia com a maior quantidade vendas foi o dia 07/03, com 25 produtos
vendidos.
d) Incorreta. No dia 08/03 foram vendidos apenas 14 sanduíches, tendo sido o dia com as menores vendas.

\item
SAEB: Ler/identificar ou comparar dados estatísticos expressos em tabelas (simples ou de dupla entrada).
BNCC: EF05MA24 - Interpretar dados estatísticos apresentados em textos, tabelas e gráficos (colunas
ou linhas), referentes a outras áreas do conhecimento ou a outros contextos, como saúde e
trânsito, e produzir textos com o objetivo de sintetizar conclusões.
a) Incorreta. Trata-se do mês com as menores vendas.
b) Incorreta. 45 não é o triplo das vendas de nenhum dos outros meses.
c) Incorreta. 62 não é o triplo das vendas de nenhum dos outros meses.
d) Correta. Em julho, com uma venda de 72 bolas, que é o triplo das 24 unidades vendidas em abril.

\item
SAEB: Resolver problemas que envolvam dados apresentados tabelas (simples ou de dupla entrada) ou gráficos estatísticos (barras simples ou agrupadas, colunas simples ou agrupadas, pictóricos ou de linhas).
BNCC: EF05MA24 – Interpretar dados estatísticos apresentados em textos, tabelas e gráficos (colunas
ou linhas), referentes a outras áreas do conhecimento ou a outros contextos, como saúde e
trânsito, e produzir textos com o objetivo de sintetizar conclusões.
a) Incorreta. Trata-se do número de crianças de 4 a 6 anos.
b) Incorreta. Trata-se do número de crianças de 7 a 9 anos, apenas.
c) Incorreta. Trata-se da soma entre as quantidades de crianças de 4 a 6 anos e de 10 a 12 anos.
d) Correta. Segundo o gráfico apresentado, 12 + 9 = 21 crianças de 7 a 12 anos
visitaram a loja.
\end{enumerate}

\section*{Matemática -- Módulo 9 -- Treino}

\begin{enumerate}
\item
SAEB: Resolver problemas que envolvam fração como resultado de uma divisão (quociente).
BNCC: EF05MA03 - Identificar e representar frações (menores e maiores que a unidade),
associando-as ao resultado de uma divisão ou à ideia de parte de um todo, utilizando a reta
numérica como recurso.
a) Incorreta. 3/3 representa todos os bombons.
b) Incorreta. 2/5 é menos do que a metade.
c) Correta. Chocolate branco/ chocolate ao leite = 4/8 = ½.
d) Incorreta. 4/6 é mais do que a metade.

Professor, reforce com os alunos a diferença entre a razão de componentes e a razão com relação ao total.

\item
SAEB: Representar frações menores ou maiores que a unidade (por meio de representações pictóricas) ou associar frações a representações pictóricas.
BNCC: EF05MA03 – Identificar e representar frações (menores e maiores que a unidade),
associando-as ao resultado de uma divisão ou à ideia de parte de um todo, utilizando a reta
numérica como recurso.
a) Incorreta. As partes não são iguais.
b) Incorreta. As partes não são iguais entre si.
c) Correta. A única que contém divisões iguais e que condizem com a divisão é a figura que aparece nesta alternativa.
d) Incorreta. Há, de fato, divisão em três partes, mas elas não são iguais.
Professor, reforce bastante com os alunos que as partes devem ser de
mesmo tamanho para representarem uma parte do todo.

\item
SAEB: Resolver problemas que envolvam 10\%, 25\%, 50\%, 75\% e 100\% associando essas representações, respectivamente, à décima parte, quarta parte, metade, três quartos e um inteiro.
BNCC: EF05MA06 – Associar as representações 10\%, 25\%, 50\%, 75\% e 100\% respectivamente à
décima parte, quarta parte, metade, três quartos e um inteiro, para calcular porcentagens,
utilizando estratégias pessoais, cálculo mental e calculadora, em contextos de educação
financeira, entre outros.
a) Incorreta. 8 não representam 25\% dos alunos da turma.
b) Correta. 25\% de 36 = ¼ x36 = 9 alunos por sessão.
c) Incorreta. 10 representam mais do que 25\% dos alunos.
d) Incorreta. 11 representam mais do que 25\% dos alunos.
\end{enumerate}

\section*{Matemática -- Módulo 10 -- Treino}

\begin{enumerate}
\item
SAEB: Resolver problemas que envolvam variação de proporcionalidade direta entre duas grandezas.
BNCC: EF05MA12 - Resolver problemas que envolvam variação de proporcionalidade direta entre
duas grandezas, para associar a quantidade de um produto ao valor a pagar, alterar as
quantidades de ingredientes de receitas, ampliar ou reduzir escala em mapas, entre outros.
a) Incorreta. Esseé o valor de uma única pizza.
b) Incorreta. Trata-se do valor, já mencionado, de duas pizzas.
c) Incorreta. Esse é o valor aproximado de 3 pizzas.
d) Correta. Valor de cada pizza: R\$ 81,60/2 = R\$ 40,80. Valor de 6 pizzas: 6 x 40,80 = R\$ 244,80.

\item
SAEB: Resolver problemas que envolvam variação de proporcionalidade direta entre duas grandezas.
BNCC: EF05MA12 - Resolver problemas que envolvam variação de proporcionalidade direta entre
duas grandezas, para associar a quantidade de um produto ao valor a pagar, alterar as
quantidades de ingredientes de receitas, ampliar ou reduzir escala em mapas, entre outros.
a) Incorreta. 9 colheres fazem 3 receitas, ou seja, 24 cafezinhos.
b) Correta. Para 48 cafezinhos, ela terá que fazer 6 receitas. Sendo assim, basta multiplicar a quantidade de colheres de pó de café para 8 cafezinhos também por
6. 3 x 6 = 18 colheres de sopa de pó de café.
c) 24 colheres fariam 64 cafés.
d) 48 colheres fariam 128 cafés.

\item
SAEB: Resolver problemas que envolvam variação de proporcionalidade direta entre duas grandezas.
BNCC: EF05MA12 - Resolver problemas que envolvam variação de proporcionalidade direta entre
duas grandezas, para associar a quantidade de um produto ao valor a pagar, alterar as
quantidades de ingredientes de receitas, ampliar ou reduzir escala em mapas, entre outros.
a) Incorreta. Em 1 minuto imprimem-se 100 folhas.
b) Incorreta. Em 15 minutos imprimem-se 1500 folhas.
c) Correta. Quantidade de folhas para 700 jornais: 5 x 700 = 3.500 folhas. Tempo gasto para a produção de 3.500 folhas = 3.500/100 = 35 minutos.
d) Incorreta. Em 55 minutos imprimem-se 5500 folhas.
\end{enumerate}

\section*{Matemática -- Módulo 11 -- Treino}

\begin{enumerate}
\item
SAEB: Resolver problemas simples de contagem (combinatória).
BNCC: EF05MA09 - Resolver e elaborar problemas simples de contagem envolvendo o princípio
multiplicativo, como a determinação do número de agrupamentos possíveis ao se combinar
cada elemento de uma coleção com todos os elementos de outra coleção, por meio de
diagramas de árvore ou por tabelas.
a) Incorreta. Isso aconteceria com uso de apenas uma cor, com 8 opções de cores.
b) Incorreta. Trata-se de um número muito menor de opções do que as possibilidades reais.
c) Correta. 8 x 7 = 56 combinações diferentes de cores para a bandeira.
d) Incorreta. Não chega a haver tantas combinações assim.

\item
SAEB: Resolver problemas simples de contagem (combinatória).
BNCC: EF05MA09 - Resolver e elaborar problemas simples de contagem envolvendo o princípio
multiplicativo, como a determinação do número de agrupamentos possíveis ao se combinar
cada elemento de uma coleção com todos os elementos de outra coleção, por meio de
diagramas de árvore ou por tabelas.
a) Correta. 6 x 18 = 108 possibilidades.
b) Incorreta. Trata-se de um número muito menor de opções do que as possibilidades reais.
c) Incorreta. Trata-se de quase um décimos das possibilidades reais.
d) Incorreta. Não chega a haver tantas combinações assim.

\item
SAEB: Resolver problemas simples de contagem (combinatória).
BNCC: EF05MA09 - Resolver e elaborar problemas simples de contagem envolvendo o princípio
multiplicativo, como a determinação do número de agrupamentos possíveis ao se combinar
cada elemento de uma coleção com todos os elementos de outra coleção, por meio de
diagramas de árvore ou por tabelas.
a) Incorreta. Trata-se de um número muito ptóximo daquele das possibilidades reais, mas ainda um pouco menor.
b) Correta. 
Sair de X e passar por S antes de chegar a Z: 3 x 2 = 6;
Sair de X passar por S e Y antes de chegar a Z: 3 x 2 x 2 = 12;
Sair de X passar por Y antes de chegar a Z: 1 x 2 = 2;
Sair de X passar por R antes de chegar a Z: 3 x 1 = 3;
Sair de X passar por R e Y antes chegar a Z: 3 x 3 x 2 = 18;
Total: 6 + 12 + 2 + 3 + 18 = 41 caminhos diferentes.
c) Incorreta. É preciso analizar o diagrama para se chegar à conclusão.
d) Incorreta. Não chega a haver tantas combinações assim.
\end{enumerate}

\section*{Matemática -- Módulo 12 -- Treino}

\begin{enumerate}
\item
SAEB: Resolver problemas de adição ou de subtração, envolvendo números racionais apenas na sua representação decimal finita até a ordem dos milésimos, com os significados de juntar, acrescentar, separar, retirar, comparar ou completar.
BNCC: EF05MA07 - Resolver e elaborar problemas de adição e subtração com números naturais e
com números racionais, cuja representação decimal seja finita, utilizando estratégias diversas,
como cálculo por estimativa, cálculo mental e algoritmos.
a) Correta. 100 -- 38,25 -- 21,55 = R\$ 40,20.
b) Incorreta. Para sobrar esse valor de troco, a compra deveria totalizar R\$ 47,60.
c) Incorreta. Para sobrar esse valor de troco, a compra deveria totalizar R\$ 41,40.
d) Correta. Seria impossível sobrar um troco maior do que o valor pago, de cem reais.

\item
SAEB: Resolver problemas de multiplicação ou de divisão, envolvendo números racionais apenas na representação decimal finita até a ordem dos milésimos, com os significados de formação de grupos iguais (incluindo repartição equitativa de medida), proporcionalidade ou disposição retangular.
BNCC: EF05MA08 - Resolver e elaborar problemas de multiplicação e divisão com números naturais e
com números racionais cuja representação decimal é finita (com multiplicador natural e divisor
natural e diferente de zero), utilizando estratégias diversas, como cálculo por estimativa,
cálculo mental e algoritmos.
a) Incorreta. Nesse caso o aluno terá multiplicado o valor de R\$ 3,75 por 5, sem considerar o valor diferenciado da primeira hora.
b) Correta. 1 x 8 + (5 -- 1) x 3,75 = R\$ 23,00.
c) Incorreta. Nesse caso o aluno terá considerado o valor da primeira hora, mais 5 horas com o valor de R\$ 3,75.
d) Incorreta. Nesse caso o aluno terá multiplicado o valor da primeira hora pelas 5 horas.

\item
SAEB: Resolver problemas de multiplicação ou de divisão, envolvendo números racionais apenas na representação decimal finita até a ordem dos milésimos, com os significados de formação de grupos iguais (incluindo repartição equitativa de medida), proporcionalidade ou disposição retangular.
BNCC: EF05MA08 – Resolver e elaborar problemas de multiplicação e divisão com números naturais e
com números racionais cuja representação decimal é finita (com multiplicador natural e divisor
natural e diferente de zero), utilizando estratégias diversas, como cálculo por estimativa,
cálculo mental e algoritmos.
a) Incorreta. Nesse caso o aluno terá considerado a quantidade de um quarto da produção.
b) Correta. Metade do suco produzido: 248,40/2 = 124,20 = 124 200 ml.
c) Incorreta. Nesse caso o aluno terá multiplicado a produção por 2, em vez de dividir.
d) Incorreta. Nesse caso o aluno terá multiplicado a produção por 4, em vez de dividir por 2.
\end{enumerate}

\section*{Matemática -- Módulo 13 -- Treino}

\begin{enumerate}
\item
SAEB: Identificar/inferir a equação que modela um problema
envolvendo adição, subtração, multiplicação ou divisão.
a) Incorreta: é necessário multiplicar o número de fileiras
pelo número de cadeiras, isto é: 25 X 8 e 6 X 6. 
b) Incorreta: é necessário multiplicar o número de fileiras
pelo número de cadeiras, isto é: 25 X 8 e 6 X 6.  
c) Correta: o número de fileiras foi adequadamente multiplicado 
pelo número de cadeiras, isto é: 25 X 8 e 6 X 6. 
d) Incorreta: é necessário multiplicar o número de fileiras
pelo número de cadeiras, isto é: 25 X 8 e 6 X 6.  
25 x 8 + 6 x 6.

\item
SAEB: Identificar/inferir a equação que modela um problema
envolvendo adição, subtração, multiplicação ou divisão.
Para alcançar o valor total arrecadado com a 
venda de ingressos, é preciso somar a arrecadação de entradas inteiras
(87 X R\$ 26,00 = R\$ 2.262,00) com a de meias-entradas 
(65 X R\$ 13,00 = R\$ 845,00). O total é de R\$ 3.107,00. 
a) Incorreta: R\$ 2.262,00 corresponde apenas à arrecadação de entradas 
inteiras.  
b) Incorreta: R\$ 845,00 corresponde apenas à arrecadação de 
meias-entradas. 
c) Incorreta: R\$ 1.417,00 não corresponde à soma da arrecadação de 
entradas inteiras e meias-entradas.
d) Correta: R\$ 3.107,00 corresponde à soma da arrecadação de 
entradas inteiras e meias-entradas.

\item
SAEB: Identificar/inferir a equação que modela um problema
envolvendo adição, subtração, multiplicação ou divisão.
a) Correta: (37 -- 5)/4 + [(2 x 5) + 6]/2 = (32)/4 + 16/2 = 8 + 8 = 16.
b) Incorreta: o valor pensado por Marcel é 16.
c) Incorreta: o valor pensado por Marcel é 16.
d) Incorreta: o valor pensado por Marcel é 16.
\end{enumerate}

\section*{Matemática -- Módulo 14 -- Treino}

\begin{enumerate}
\item
SAEB:
– Identificar os indivíduos (universo ou população-alvo da pesquisa),
as variáveis ou os tipos de variáveis (quantitativas ou categóricas) em 
um conjunto de dados.
– Representar ou associar os dados de uma pesquisa estatística ou
de um levantamento em listas, tabelas (simples ou de dupla entrada) ou
gráficos (barras simples ou agrupadas, colunas simples ou agrupadas,
pictóricos ou de linhas).
No enunciado, afirma-se que a empresa tem 350 funcionários; esse 
é, portanto, o número que representa a população. A amostra, por sua vez, 
se refere apenas àqueles que participaram da pesquisa, isto é, 75
funcionários.
a) Incorreta. A população é de 350 funcionários (não 10); a amostra é de
75 (não de 350).
b) Incorreta. A amostra correta é de 75 funcionários, não de 10.
c) Incorreta. Nessa alternativa, os números de população e de amostra
estão invertidos.
d) Correta: A população é de 350 funcionários; a amostra corresponde ao
número de candidatos que participaram da pesquisa: 75.

\item
SAEB:
– Identificar os indivíduos (universo ou população-alvo da pesquisa),
as variáveis ou os tipos de variáveis (quantitativas ou categóricas) em 
um conjunto de dados.
– Representar ou associar os dados de uma pesquisa estatística ou
de um levantamento em listas, tabelas (simples ou de dupla entrada) ou
gráficos (barras simples ou agrupadas, colunas simples ou agrupadas,
pictóricos ou de linhas).
a) Incorreta. As variáveis qualitativas não são representadas por 
meio de quantidades. O número do irmãos não pode, assim, ser uma variável 
qualitativa.
b) Correta. As variáveis qualitativas discretas só podem ser medidas
com números inteiros, sem frações. É o caso do número de irmãos. 
c) Incorreta. As variáveis qualitativas contínuas são fracionáveis,
divisíveis. Não é, claramente, o caso do número de irmãos. 
d) Incorretas. As definições de variáveis quantitativas e qualitativas
excluem a possibilidade de haver quaisquer variáveis que participem das
duas categorias ao mesmo tempo.

\item
SAEB:
– Identificar os indivíduos (universo ou população-alvo da pesquisa),
as variáveis ou os tipos de variáveis (quantitativas ou categóricas) em 
um conjunto de dados.
– Representar ou associar os dados de uma pesquisa estatística ou
de um levantamento em listas, tabelas (simples ou de dupla entrada) ou
gráficos (barras simples ou agrupadas, colunas simples ou agrupadas,
pictóricos ou de linhas).
a) Incorreta. Existem 10 notas maiores ou iguais a 7 (sete) e 10 notas menores do que 7 (sete).
b) Correta. Entre as notas apresentadas, temos 10 notas maiores ou iguais
a 7 (sete).
c) Incorreta. Incorreta. Existem 10 notas maiores ou iguais a 7 (sete) e 10 notas menores do que 7 (sete).
d) Incorreta. Incorreta. Existem 10 notas maiores ou iguais a 7 (sete) e 10 notas menores do que 7 (sete).
\end{enumerate}

\section*{Matemática -- Módulo 15 -- Treino}

\begin{enumerate}
\item
SAEB:
– Inferir a finalidade de realização de uma pesquisa estatística ou de
um levantamento, dada uma tabela (simples ou de dupla entrada) ou
gráfico (barras simples ou agrupadas, colunas simples ou agrupadas,
pictóricos ou de linhas) com os dados dessa pesquisa.
– Argumentar ou analisar argumentações/conclusões com base em dados
apresentados em tabelas (simples ou de dupla entrada) ou gráficos
(barras simples ou agrupadas, colunas simples ou agrupadas, pictóricos
ou de linhas).
a) Correta. O candidato A cumpre os dois critérios exigidos pelas
regras do processo seletivo: suas notas são todas superiores a 30, e três
delas são iguais (as de Português, Matemática e Direito).
b) Incorreta. Embora as notas do candidato B sejam superiores a 30, 
cumprindo o primeiro dos critérios do processo seletivo, apenas duas 
delas são idênticas (as de Português e Direito), de modo que esse 
B perde para A, que teve três notas iguais. 
c) Incorreta. O candidato C não pode ser aprovado, porque teve uma nota
inferior a 30.
d) Incorreta. O candidato D não pode ser aprovado, porque teve duas notas
inferiores a 30.

\item
SAEB:
– Inferir a finalidade de realização de uma pesquisa estatística ou de
um levantamento, dada uma tabela (simples ou de dupla entrada) ou
gráfico (barras simples ou agrupadas, colunas simples ou agrupadas,
pictóricos ou de linhas) com os dados dessa pesquisa.
– Argumentar ou analisar argumentações/conclusões com base em dados
apresentados em tabelas (simples ou de dupla entrada) ou gráficos
(barras simples ou agrupadas, colunas simples ou agrupadas, pictóricos
ou de linhas).
a) Incorreta. O número de alunos do 4º ano é 86, não 60.
b) Correta. O número de alunos do 4º ano é igual à soma do número de
alunos das turmas A, B e C: 32 + 29 + 25 = 86.
c) Incorreta. O número de alunos do 4º ano é 86, não 91.
d) Incorreta. O número de alunos do 4º ano é 86, não 150.

\item
SAEB:
– Inferir a finalidade de realização de uma pesquisa estatística ou de
um levantamento, dada uma tabela (simples ou de dupla entrada) ou
gráfico (barras simples ou agrupadas, colunas simples ou agrupadas,
pictóricos ou de linhas) com os dados dessa pesquisa.
– Argumentar ou analisar argumentações/conclusões com base em dados
apresentados em tabelas (simples ou de dupla entrada) ou gráficos
(barras simples ou agrupadas, colunas simples ou agrupadas, pictóricos
ou de linhas).
A razão do número do livros retirados em abril (205) e junho 
(210) é expressa da seguinte maneira: 205/210.  
a) Incorreta. 
b) Incorreta.
c) Correta. 205/210 = 51/52.
b) Incorreta.
\end{enumerate}


\section*{Educação Física -- Módulo 1 -- Treino}

\begin{enumerate}
\item
SAEB: Analisar os esportes e as lutas nas suas manifestações
profissionais e de lazer.
BNCC: EF35EF03 – Descrever, por meio de múltiplas linguagens (corporal,
oral, escrita, audiovisual), as brincadeiras e os jogos populares do
Brasil e de matriz indígena e africana, explicando suas características
e a importância desse patrimônio histórico cultural na preservação das
diferentes culturas.
a) Correta. O texto fala de uma prática corporal voltada ao lazer, ou
seja, uma brincadeira que tem regras a adaptações para a diversão.
b) Incorreta. Por mais que o nome da brincadeira seja “golpear com as
mãos”, tobdaé é uma brincadeira para o lazer e não uma luta.
c) Incorreta. O texto mostra duas variações (regras) que podem ser
modificadas para brincar.
d) Incorreta. O texto mostra como a peteca é feita, mas não são
materiais oficiais semelhantes aos dos esportes e, sim, adaptações.

\item
SAEB: Identificar a importância do respeito ao oponente e às normas de
segurança na vivência das práticas corporais (jogos, lutas, ginásticas,
esportes e dança).
BNCC: EF35EF15 – Identificar as características das lutas do contexto
comunitário e regional e lutas de matriz indígena e africana,
reconhecendo as diferenças entre lutas e brigas e entre lutas e as
demais práticas corporais.
a) Incorreta. A luta huka-huka tem regras e objetivos que não podem ser
modificados, ou seja, as regras se mantêm as mesmas.
b) Incorreta. O texto mostra que os lutadores ganham respeito e
reconhecimento, não prêmios.
c) Incorreta. A luta huka-hula é uma manifestação corporal que não
promove a briga entres os praticantes, e sim a cultura indígena e o
respeito.
d) Correta. Por meio do texto é possível analisar que a luta não tem
juiz e os próprios lutadores reconhecem a vitória do outro. Portanto, é
um sinal de demostrar respeito com outro.

\item
SAEB: Analisar os esportes e as lutas nas suas manifestações
profissional e de lazer.
BNCC: EF35EF15 – Identificar as características das lutas do contexto
comunitário e regional e lutas de matriz indígena e africana,
reconhecendo as diferenças entre lutas e brigas e entre lutas e as
demais práticas corporais.
a) Incorreta. O fato de diferentes etnias indígenas as realizarem não faz com que as
lutas se tornem um esporte, apenas mostra como essas práticas corporais
são importantes para essa cultura.
b) Correta. Uma das principais característica de uma prática corporal
ser um esporte é que ela deve estar presente em uma competição esportiva
oficial, como os Jogos dos Povos Indígenas.
c) Incorreta. O fato de a luta ter diferentes formas de começar (em pé ou
ajoelhado) não é uma definição do esporte, só mostra algumas versões dela.
d) Incorreta. A presença de pinturas corporais é uma característica da própria
cultura indígena, não dos esportes.
\end{enumerate}

\section*{Educação Física -- Módulo 2 -- Treino}

\begin{enumerate}
\item
SAEB: Analisar o protagonismo do trabalho coletivo na vivência dos jogos
populares e dos esportes.
BNCC: EF35EF06 – Diferenciar os conceitos de jogo e esporte, identificando as características que os constituem na contemporaneidade
e suas manifestações (profissional e comunitária/lazer).
a) Incorreta. O texto mostra que o praticante deve usar um cinto com
velcro com bandeirinha, mas o tag-rugby serve para popularizar o rugby e
não criar novos equipamentos.
b) Incorreta. Por mais que o tag-rugby evite o contato físico, esse jogo
pré-desportivo tem o propósito de incentivar a pratica do rugby e não
acabar com os conflitos nos outros esportes.
c) Incorreta. O tag-rugby não é um esporte e sim uma brincadeira do
rugby.
d) Correta. O texto mostra algumas variações do rugby para torná-lo mais
lúdico para as pessoas, com o propósito de popularizar esse esporte.

\item
SAEB: Valorizar o patrimônio histórico representado pelas brincadeiras e
jogos, com ênfase naqueles de origem indígena e africana.
BNCC: EF35EF01 – Experimentar e fruir brincadeiras e jogos populares do
Brasil e do mundo, incluindo aqueles de matriz indígena e africana, e
recriá-los, valorizando a importância desse patrimônio histórico-cultural.
a) Incorreta. São as regras do jogo mancala que podem ser
alteradas, não a cultura africana.
b) Incorreta. Por mais que o texto mostre o significado da palavra
“mancala”, o jogo de tabuleiro não ensina novas palavras, e sim
apresenta um costume da cultura africana.
c) Correta. Por meio do texto podemos perceber que o mancala é um jogo
que representa a força da cultura africana; ou seja, por meio do jogo podemos conhecer diferentes tradições de outras culturas.
d) Incorreta. O texto mostra que o mancala é usado na escola, mas não
para que os alunos estudem um conteúdo relacionado à cultura local
(Acre), e sim sobre a cultura africana.

\item
SAEB: Identificar as brincadeiras e os jogos populares como patrimônio
histórico-cultural.
BNCC: EF35EF01 – Experimentar e fruir brincadeiras e jogos populares do
Brasil e do mundo, incluindo aqueles de matriz indígena e africana, e
recriá-los, valorizando a importância desse patrimônio histórico
cultural.
a) Incorreta. O texto fala que os locais específicos para brincar são para
incentivar a pratica de algumas brincadeiras, não para restringir as
brincadeiras tradicionais.
b) Incorreta. O objetivo é preservar a cultura local por meio das
brincadeiras, não de criar novas regras.
c) Incorreta. O texto não cita que os locais voltados para as
brincadeiras vão incentivar o comércio, e sim incentivar as pessoas a
realizarem algumas brincadeiras tradicionais.
d) Correta. O texto mostra que algumas brincadeiras se tornarem um
patrimônio cultural e vai haver locais para brincar com o objetivo de
as pessoas continuarem praticando essas brincadeiras e preservando a cultura
local.
\end{enumerate}

\section*{Educação Física -- Módulo 3 -- Treino}

\begin{enumerate}
\item
SAEB: Comparar os elementos constitutivos de danças populares do Brasil
e do mundo com aqueles de danças de matrizes indígena e africana.
BNCC: EF35EF10 – Comparar e identificar os elementos constitutivos
comuns e diferentes (ritmo, espaço, gestos) em danças populares do
Brasil e do mundo e danças de matriz indígena e africana.
a) Incorreta. O fato de o toré ser realizado por diferentes povos não quer
dizer que é uma dança, e sim uma prática difundida na cultura indígena.
b) Correta. O instrumento musical (maracá) serve para marcar o ritmo na música e na dança; ou seja, é um elemento constitutivo da dança.
c) Incorreta. Porque a dança ser realizado ao ar livre não é uma
característica própria das danças.
d) Incorreta. Qualquer atividade promove a interação entre as
pessoas e não somente as danças.

\item
SAEB: Valorizar o patrimônio histórico representado pelas danças
populares, com ênfase naquelas de matriz indígena e africana.
BNCC: EF35EF09 – Experimentar, recriar e fruir danças populares
do Brasil e do mundo e danças de matriz indígena e africana, valorizando
e respeitando os diferentes sentidos e significados dessas danças em
suas culturas de origem.
a) Correta. O texto mostra que os negros escravizados, do Congo e
de Angola, desenvolveram o jongo e, por isso, ele tem elementos
culturais africanos.
b) Incorreta. O jongo tem influência da cultura africana do Congo
e de Angola, não da cultura brasileira.
c) Incorreta. Eram os negros escravizados que trabalhavam nas
fazendas que realizavam a dança do jongo.
d) Incorreta. Por mais que a dança fosse praticada em eventos religiosos,
isso foi criado no Brasil e não nos países africanos. Além disso, o
evento religioso não definia que a dança é de origem africana.

\item
SAEB: Valorizar o patrimônio histórico representado pelas danças
populares, com ênfase naquelas de matriz indígena e africana.
BNCC: EF35EF11 – Formular e utilizar estratégias para a execução de
elementos constitutivos das danças populares do Brasil e do mundo, e das
danças de matriz indígena e africana.
a) Incorreta. O samba é voltado para a dança e não para praticar a capoeira (luta africana).
b) Incorreta. O samba é de origem africana e não europeia. Apenas
alguns instrumentos portugueses são usados, mas isso não faz com que o
praticante conheça a cultura alimentar europeia.
c) Incorreta. O samba não tem o objetivo de o praticante entender
como a Unesco reconhece uma atividade como patrimônio cultural.
d) Correta. O texto mostra alguns elementos culturais presentes
no samba e, por conta disso, o praticante dessa dança vai poder conhecer
alguns costumes e tradições da cultura africana.
\end{enumerate}

\section*{Ciências da Natureza -- Módulo 1 -- Treino}

\begin{enumerate}
\item
BNCC: EF05CI02 - Aplicar os conhecimentos sobre as mudanças de estado
físico da água para explicar o ciclo hidrológico e analisar suas
implicações na agricultura, no clima, na geração de energia elétrica, no
provimento de água potável e no equilíbrio dos ecossistemas regionais
(ou locais).
a) Incorreta. A ebulição é a passagem rápida de uma substância do estado
líquido para o estado gasoso em determinada temperatura. No caso
mencionado, há um fenômeno específico realizado em condições próprias de
solo e vegetação, como na floresta amazônica, chamado de
evapotranspiração. 
b) Correta. A evapotranspiração é um fenômeno que combina a evaporação
de líquidos com a transpiração de folhas. No caso da floresta amazônica,
a umidade se eleva, pois as árvores funcionam como “bombas” de água,
participando também da regulação do regime de chuvas de toda a região.
c) Incorreta. A condensação ocorre quando há agregação de substâncias
gasosas, de modo que as partículas se unam e formem um líquido. Esse
fenômeno pode acontecer nas nuvens, numa das etapas do ciclo hidrológico
da água, mas não é descrito no texto.
d) Incorreta. A precipitação ocorre quando há quantidade suficiente de
água no estado líquido nas nuvens. Apesar de a chuva ser
mencionada no texto, não há descrição desse fenômeno, e sim da
evapotranspiração, processo em que a combinação de evaporação da água de
solos e transpiração das folhas acontece.

\item
Saeb: Eixo cognitivo B.
BNCC: (EF05CI03) Selecionar argumentos que justifiquem a importância da
cobertura vegetal para a manutenção do ciclo da água, a conservação dos
solos, dos cursos de água e da qualidade do ar atmosférico.
a) Incorreta. A perda de vegetação acentua a desregulação do regime de
chuvas, pois desequilibra o ciclo hidrológico da água.
b) Incorreta. O problema mencionado é a diminuição da cobertura vegetal,
logo não há relação com a eliminação de poluentes das florestas, que
pode se dar a partir de iniciativas para localizar os focos de poluição
e realizar forças-tarefa para preservar o ambiente.
c) Incorreta. A redução da cobertura vegetal pode acarretar o
desequilíbrio do ciclo hidrológico da água, e favorece o acontecimento
de fenômenos como o assoreamento, que altera os cursos d'água e eleva o
leito de rios e lagos.
d) Correta. A perda de vegetação pode acarretar um extenso processo de
erosão dos solos que tem como uma das principais consequências a
inundação de rios, causada por fenômenos como o assoreamento, quando a
falta de vegetação, aliada com a perda da qualidade do solo, faz com que
detritos sólidos sejam arrastados para o fundo dos rios, elevando o
leito.

\item
Saeb: Eixo cognitivo C.
BNCC: (EF05CI04) Identificar os principais usos da água e de outros
materiais nas atividades cotidianas para discutir e propor formas
sustentáveis de utilização desses recursos.
a) Incorreta. A água residual, desde que não tenha componentes
corrosivos, não deve ser contaminante das superfícies metálicas a serem
lavadas. Em muitas situações, ela pode ser reaproveitada em atividades
como lavagens, irrigação de plantas e descarga de bacias sanitárias.
b) Incorreta. A água residual não tratada pode contaminar os mananciais
de água limpa, sem passar pelo tratamento adequado. Caso se deseje
devolver a água para os mananciais, deve-se adotar um regime de
tratamento avançado para eliminar microrganismos e componentes tóxicos
da água residual.
c) Correta. Caso se deseje reutilizar água residual para fins potáveis,
deve-se adotar um esquema avançado de tratamento a fim de filtrar-lhe as
impurezas e eliminar microrganismos, além de propor uma
avaliação da qualidade para o consumo humano.
d) Incorreta. O reaproveitamento da água traz grandes benefícios em
termos de economia de recursos e do uso sustentável de um bem natural.
Ele pode ser realizado em escala doméstica, reaproveitando água para
procedimentos como lavagens de calçadas e carros, ou em escala
industrial, com a reutilização de água residual em processos de lavagem
e produção de energia.
\end{enumerate}

\section*{Ciências da Natureza -- Módulo 2 -- Treino}

\begin{enumerate}
\item
BNCC: (EF05CI06) Selecionar argumentos que justifiquem por que os
sistemas digestório e respiratório são considerados corresponsáveis pelo
processo de nutrição do organismo, com base na identificação das funções
desses sistemas.
a) Correta. No sistema circulatório, há o transporte de oxigênio e dos
nutrientes para todas as extremidades do corpo.
b) Incorreta. O sistema digestório é responsável por digerir os
alimentos, separar os nutrientes para o aproveitamento em diferentes
órgãos e descartar as fezes.
c) Incorreta. No sistema respiratório, há a troca gasosa e o uso do
oxigênio para alguns processos no organismo humano, como a divisão da
glicose em porções menores para o transporte. Esse sistema, em si, não
consegue irrigar todas as extremidades do corpo humano.
d) Incorreta. O sistema reprodutivo é responsável por processos relacionados
à formação e ao desenvolvimento da vida, que não dizem respeito, por sua 
vez, ao transporte de nutrientes para as extremidades do corpo.

\item
BNCC: EF05CI07 - Justificar a relação entre o funcionamento do sistema
circulatório, a distribuição dos nutrientes pelo organismo e a
eliminação dos resíduos produzidos.
a) Incorreta. O corpo humano pode produzir células de defesa sozinho, mas
precisa de imunização prévia para determinadas doenças, que são muito agressivas.
A vacinação se faz, portanto, essencial.
b) Incorreta. O conteúdo da vacina não participa da regulação dos sistemas do
organismo, e sim da estimulação da produção de células de defesa pelo sistema
imunológico.
c) Correta. Sem a vacinação, as pessoas ficam vulneráveis ao retorno de
doenças por conta da fragilidade da defesa do organismo, que necessita
de anticorpos de memória para combater algumas doenças, e, assim, prevenir
casos graves e mortes.
d) Incorreta. O sangue não leva o conteúdo da vacina para combater
infecções. As vacinas atuam simulando uma infecção mais fraca, para
fazer com que o organismo produza defesas que ficarão na memória
imunológica. Assim, quando a infecção real acontecer, o organismo terá
estruturas para combatê-la.

\item
BNCC: (EF05CI08) Organizar um cardápio equilibrado com base nas
características dos grupos alimentares (nutrientes e calorias) e nas
necessidades individuais (atividades realizadas, idade, sexo etc.) para
a manutenção da saúde do organismo.
a) Incorreta. As informações dos rótulos nutricionais estimam a
quantidade de nutrientes do alimento com base no consumo calórico
recomendado diariamente, além de também considerar o consumo médio de
calorias necessárias para o corpo humano, sendo, assim, estimativas
confiáveis.
b) Incorreta. Apesar das necessidades calóricas variarem de corpo a
corpo, o assunto em questão é a contagem de calorias, que não deve ser o
único quesito de avaliação para uma alimentação saudável, visto que dois
alimentos podem apresentar a mesma quantidade calórica numa porção, mas
quantidades diferentes de nutrientes como carboidratos, gorduras e
proteínas.
c) Correta. Demais nutrientes, como carboidratos, gorduras e proteínas
são importantes numa alimentação balanceada. Deve-e evitar o exagero no
consumo de alimentos ricos em gorduras, como os ultraprocessados, pois
eles representam uma fonte pobre de calorias.
d) Incorreta. Os alimentos ultraprocessados representam calorias vazias,
ricas em gorduras e com pouco aproveitamento para o organismo além do
acúmulo de gordura. O consumo em excesso pode, inclusive, ser maléfico à
saúde, causando doenças.
\end{enumerate}

\section*{Ciências da Natureza -- Módulo 3 -- Treino}

\begin{enumerate}
\item
Habilidade da BNCC: EF05CI11 - Associar o movimento diário
do Sol e das demais estrelas no céu ao movimento de rotação da Terra.
a) Correta. Como a irradiação de luz solar não é uniforme em todas as
regiões da Terra, devido aos movimentos de rotação e translação,
observam-se nos exemplos diferentes fusos horários e distintas estações do
ano. Enquanto o Brasil tem um dia quente e ensolarado, é madrugada 
no Japão, o dia começa na Austrália e faz frio nos Estados Unidos.
b) Incorreta. A existência do dia e da noite em dois polos distintos da
Terra ao mesmo tempo não é explicada pela ocorrência ou falta de camadas
de proteção solar no planeta.
c) Incorreta. Embora seja preocupante, a elevação dos níveis d'água nos
hemisférios não explica ocorrências milenares como a existência do dia
e da noite em pontos distintos do planeta ao mesmo tempo.
d) Incorreta. As mudanças climáticas não têm relação direta com o
fenômeno descrito, no qual, em pontos distintos do planeta Terra, ao mesmo
tempo, observa-se um dia quente, uma madrugada, um alvorecer e um dia
frio. Essas ocorrências são explicadas pela irradiação solar no planeta,
que muda conforme o movimento de Rotação.

\item
Habilidade da BNCC: EF05CI12 - Concluir sobre a periodicidade
das fases da Lua, com base na observação e no registro das formas 
aparentes da Lua no céu ao longo de, pelo menos, dois meses.
a) Incorreta. A Terra não realiza Revolução, e o texto descreve, além
dos movimentos citados, a Translação terrestre, movimento de giro da
Terra em torno do sol durante 365 dias, e a Revolução lunar, movimento
de giro da Lua em torno da Terra durante aproximadamente 28 dias.
b) Incorreta. A Terra não realiza Revolução, e os movimentos de Rotação
e Translação terrestre descritos no texto não foram citados.
c) Correta. Os movimentos realizados, na ordem de descrição do texto,
são Rotação terrestre, Translação terrestre, Rotação lunar, Translação
lunar e Revolução lunar.
d) Incorreta. Nenhum movimento do Sol foi mencionado no texto, e, além
disso, não se cita o movimento da Terra em torno do seu próprio eixo, a
Rotação terrestre, e nem o movimento de giro da Lua em torno do sol, a
Translação lunar.

\item
Habilidade BNCC: EF05CI11 - Associar o movimento diário do Sol e das
demais estrelas no céu ao movimento de rotação da Terra.
a) Incorreta. O equinócio de primavera representa o período de fim do
inverno e começo da primavera, ou seja, nessas datas, a incidência da
luz solar é maior na região equatorial da Terra, fazendo com que os dias
e as noites tenham durações iguais.
b) Correta. No solstício de verão, há maior incidência da luz solar em
um dos hemisférios da Terra --- norte, em junho, e sul, em dezembro.
Nesse período, inicia-se o verão, em que os dias duram mais do que as
noites.
c) Incorreta. No solstício de inverno, há menor incidência da luz solar
em um dos hemisférios da terra --- norte, em dezembro, e sul, em junho.
Nesse período, inicia-se o inverno, quando as noites duram mais do que
os dias.
d) Incorreta. O equinócio de outono representa o período de fim do verão
e começo do outono, ou seja, nessas datas, a incidência da luz solar é
maior na região equatorial da Terra, fazendo com que os dias e as noites
tenham durações iguais.
\end{enumerate}

\section*{Simulado 1}

\begin{enumerate}
\item
SAEB: Compor ou decompor números naturais de até 6
ordens na forma aditiva, ou em suas ordens, ou em adições e 
multiplicações.
BNCC: EF05MA01 - Ler, escrever e ordenar números naturais até a ordem das 
centenas de milhar com compreensão das principais características do 
sistema de numeração decimal. 
EF05MA10 - Concluir, por meio de investigações, que a relação de 
igualdade existente entre dois membros permanece ao adicionar, subtrair, 
multiplicar ou dividir cada um desses membros por um mesmo número, para 
construir a noção de equivalência.
EF05MA11 - Resolver e elaborar problemas cuja conversão em sentença 
matemática seja uma igualdade com uma operação em que um dos termos é 
desconhecido.
Para encontrar a resposta, é preciso verificar que cada um dos 
cinco grandes cubos é composto por cem unidades. À direita, cada uma das 
duas colunas tem dez unidades. Finalmente, cada cubinho pequeno 
corresponde a apenas uma unidade. Dessa maneira, somam-se as cem unidades 
dos grandes cubos, com as vinte das colunas e as três dos cubinhos: 5 x 
100 + 2 x 10 + 3 x 1 = 523.  
a) Incorreta. O número correto é 523, não 623.
b) Incorreta. O número correto é 523, não 423.
c) Incorreta. O número correto é 523, não 503.
d) Correta. 5 x 100 + 2 x 10 + 3 x 1 = 523.

\item
SAEB: Compor ou decompor números naturais de até 6 ordens na
forma aditiva, ou em suas ordens, ou em adições e multiplicações.
BNCC: EF05MA01 - Ler, escrever e ordenar números naturais até a ordem das 
centenas de milhar com compreensão das principais características do 
sistema de numeração decimal. 
EF05MA10 - Concluir, por meio de investigações, que a relação de 
igualdade existente entre dois membros permanece ao adicionar, subtrair, 
multiplicar ou dividir cada um desses membros por um mesmo número, para 
construir a noção de equivalência.
EF05MA11 - Resolver e elaborar problemas cuja conversão em sentença 
matemática seja uma igualdade com uma operação em que um dos termos é 
desconhecido.
Para obter a resposta correta, basta solucionar a expressão numérica.
a) Incorreta. (2 x 1.000) + (3 x 100) + (1 x 10) = 2.310.
b) Incorreta. (2 x 1.000) + (3 x 100) + (1 x 10) = 2.310.
c) Incorreta. (2 x 1.000) + (3 x 100) + (1 x 10) = 2.310.
d) Correta. (2 x 1.000) + (3 x 100) + (1 x 10) = 2.310.

\item
SAEB: Determinar o número desconhecido que torna verdadeira
uma igualdade que envolve as operações fundamentais com números naturais
de até 6 ordens.
BNCC: EF05MA01 - Ler, escrever e ordenar números naturais até a ordem das 
centenas de milhar com compreensão das principais características do 
sistema de numeração decimal. 
EF05MA10 - Concluir, por meio de investigações, que a relação de 
igualdade existente entre dois membros permanece ao adicionar, subtrair, 
multiplicar ou dividir cada um desses membros por um mesmo número, para 
construir a noção de equivalência.
EF05MA11 - Resolver e elaborar problemas cuja conversão em sentença 
matemática seja uma igualdade com uma operação em que um dos termos é 
desconhecido.
a) Incorreta. 359 + 246 = 605, não 513.
b) Incorreta. 359 + 246 = 605, não 523.
c) Correta. 359 + 246 = 605.
d) Incorreta. 359 + 246 = 605, não 705.

\item
SAEB: Determinar o número desconhecido que torna verdadeira
uma igualdade que envolve as operações fundamentais com números naturais
de até 6 ordens.
BNCC: EF05MA01 - Ler, escrever e ordenar números naturais até a ordem das 
centenas de milhar com compreensão das principais características do 
sistema de numeração decimal. 
EF05MA10 - Concluir, por meio de investigações, que a relação de 
igualdade existente entre dois membros permanece ao adicionar, subtrair, 
multiplicar ou dividir cada um desses membros por um mesmo número, para 
construir a noção de equivalência.
EF05MA11 - Resolver e elaborar problemas cuja conversão em sentença 
matemática seja uma igualdade com uma operação em que um dos termos é 
desconhecido.
a) Incorreta. 4.054 -- 2.843 = 1.211, não 2.416.
b) Correta. 4.054 -- 2.843 = 1.211. Como o aumento foi o mesmo
nos dois números, não precisamos somar 300 aos números
antigos já que a diferença entre eles se manterá a mesma.
c) Incorreta. 4.054 -- 2.843 = 1.211, não 1.883.
d) Incorreta. 4.054 -- 2.843 = 1.211, não 1.463.

\item
SAEB: Inferir o padrão ou a regularidade de uma sequência de
números naturais ordenados, objetos ou figuras.
a) Incorreta. 301 é o número da sala mais próxima da árvore.
b) Incorreta. 302 é o número da sala intermediária do terceiro andar.
c) Correta. De acordo com as orientações do enunciado, a sala mais
próxima da árvore será indicada com o final 1. A mais distante, com o 
final 3. Dessa maneira, o número da sala de janela fechada do terceiro 
andar --- iniciado com o número 3, seguido de zero --- é 303.
d) Incorreta. Existem apenas 3 salas por andar, de modo que não pode
haver sala de número 304.

\item
SAEB: Resolver problemas de multiplicação ou de divisão,
envolvendo números naturais de até 6 ordens, com os significados de
formação de grupos iguais (incluindo repartição equitativa e medida),
proporcionalidade ou disposição retangular.
a) Incorreta. O número máximo de copos que poderão ser servidos é 40, não
16.
b) Incorreta. O número máximo de copos que poderão ser servidos é 40, não 20.
c) Incorreta. O número máximo de copos que poderão ser servidos é 40, não 32.
d) Correta. 8 litros correspondem a 8.000 ml de refrigerante que foram
comprados. É possível obter o número máximo de copos que poderão ser
servidos dividindo a quantidade de litros pela capacidade máxima de 
cada copo: 8.000/200 = 40.

\item
SAEB: Medir ou comparar perímetro ou área de figuras planas
desenhadas em malha quadriculada.
a) Incorreta. O número total de quadradinhos é 29, não 24.
b) Incorreta. O número total de quadradinhos é 29, não 26.
c) Incorreta. Primeiramente, é preciso contar os quadradinhos completos, 
que são 24. Depois, contar os quadradinhos que estão pela metade (10) e 
dividir esse número por 2: 10/2 = 5. Vinte e quatro quadradinhos 
completos + 10 metades (que correspondem a 5 completos) = 29 quadradinhos 
no total. 
d) Incorreta. O número total de quadradinhos é 29, não 34.

\item
SAEB: Relacionar valores de moedas e/ou cédulas do sistema
monetário brasileiro, com base nas imagens desses objetos.
a) Incorreta. O valor total é de R\$ 74,10.
b) Incorreta. O valor total é de R\$ 74,10.  
c) Correta. Ana Beatriz tem, em cédulas, R\$ 20,00 + 10,00 + 5,00; em
moedas R\$ 2,70, em um total de R\$ 37,70. Camila tem R\$ 20,00 + 10,00 
em cédulas e R\$ 6,40 em moedas: R\$ 36,40. Juntas, elas têm R\$ 37,70 + 
36,40 = R\$ 74,10.   
b) Incorreta. O valor total é de R\$ 74,10.

\item
SAEB: Determinar a probabilidade de ocorrência de um
resultado em eventos aleatórios, quando todos os resultados possíveis
têm a mesma chance de ocorrer (equiprováveis).
BNCC: EF05MA22 - Apresentar todos os possíveis resultados de um experimento aleatório, estimando se esses resultados são igualmente prováveis ou não. 
EF05MA23 - Determinar a probabilidade de ocorrência de um resultado em eventos aleatórios, quando todos os resultados possíveis têm a mesma chance de ocorrer (equiprováveis).
a) Incorreta. A probabilidade é de 50\%.
b) Incorreta. A probabilidade é de 50\%. 
c) Correta. Como nesse evento só existem duas possibilidades (menino ou
menina), a probabilidade de Isabeli ter uma irmã é de 50\%.
d) Incorreta. A probabilidade é de 50\%.

\item
SAEB: Resolver problemas que envolvam dados apresentados
tabelas (simples ou de dupla entrada) ou gráficos estatísticos (barras
simples ou agrupadas, colunas simples ou agrupadas, pictóricos ou de
linhas).
BNCC: EF05MA24 - Interpretar dados estatísticos apresentados em textos, tabelas e gráficos (colunas ou linhas), referentes a outras áreas do conhecimento ou a outros contextos, como saúde e trânsito, e produzir textos com o objetivo de sintetizar conclusões.
a) Incorreta. Em março o índice foi de 8,6\%, maior do que os 6,8\% de dezembro. 
b) Incorreta. Em julho o índice foi de 8,1\%, maior do que os 6,8\% de dezembro.
c) Incorreta. Em outubro o índice foi de 7,5\%, maior do que os 6,8\% de dezembro.
d) Correta. Analisando o gráfico, verifica-se que a menor taxa de 
desemprego ocorreu em dezembro, com 6,8\%.

\item
SAEB: Representar frações menores ou maiores que a unidade
(por meio de representações pictóricas) ou associar frações a
representações pictóricas.
BNCC: EF05MA03 - Identificar e representar frações (menores e maiores que a unidade), associando-as ao resultado de uma divisão ou à ideia de parte de um todo, utilizando a reta numérica como recurso.
EF05MA04 - Identificar frações equivalentes.
EF05MA06 - Associar as representações 10\%, 25\%, 50\%, 75\% e 100\% respectivamente à décima parte, quarta parte, metade, três quartos e um inteiro, para calcular porcentagens, utilizando estratégias pessoais, cálculo mental e calculadora, em contextos de educação financeira, entre outros.
a) Incorreta. Como a distância total entre as seis árvores está dividida 
em 5 partes iguais, a fração será 1/5 (não 1/4).
b) Incorreta.  Como a distância total entre as seis árvores está dividida 
em 5 partes iguais, a fração será 1/5 (não 2/3)
c) Incorreta.  Como a distância total entre as seis árvores está dividida 
em 5 partes iguais, a fração será 1/5 (não 1/3)
d) Correta. Como a distância total entre as seis árvores está dividida em 
5 partes iguais, a fração será 1/5.

\item
SAEB: Resolver problemas que envolvam variação de
proporcionalidade direta entre duas grandezas.
BNCC: EF05MA12 - Resolver problemas que envolvam variação de 
proporcionalidade direta entre duas grandezas, para associar a quantidade 
de um produto ao valor a pagar, alterar as quantidades de ingredientes de 
receitas, ampliar ou reduzir escala em mapas, entre outros.
a) Incorreta. Com a redução de velocidade, o percurso terá a duração de 2 horas.
b) Incorreta. Com a redução de velocidade, o percurso terá a duração de 2 horas.
c) Correta. Como, na situação inicial, Juca percorre 80 km em uma hora,
conclui-se que em uma hora e meia ele percorre 120 km. Dessa maneira,
reduzindo a velocidade para 60 km/h, ele percorrerá 60 km em uma hora; 
mantendo-se a proporção, os 120 km serão percorridos em 2 horas.
d) Incorreta. Com a redução de velocidade, o percurso terá a duração de 2 horas.

\item
SAEB: Resolver problemas simples de contagem (combinatória).
BNCC: EF05MA09 - Resolver e elaborar problemas simples de contagem 
envolvendo o princípio multiplicativo, como a determinação do número de 
agrupamentos possíveis ao se combinar cada elemento de uma coleção com 
todos os elementos de outra coleção, por meio de diagramas de árvore ou 
por tabelas.
a) Incorreta. Existema 50 formas diferentes de se vestir, não 5.
b) Incorreta. Existema 50 formas diferentes de se vestir, não 10.
c) Incorreta. Existema 50 formas diferentes de se vestir, não 15.
d) Correta. Para obter o número de maneiras diferentes de se vestir, 
basta multiplicar o número de camisetas pelo de bermudas: 10 x 5 = 50.

\item
SAEB: Resolver problemas de multiplicação ou de divisão,
envolvendo números racionais apenas na representação decimal finita até
a ordem dos milésimos, com os significados de formação de grupos iguais
(incluindo repartição equitativa de medida), proporcionalidade ou
disposição retangular.
BNCC: EF05MA07 - Resolver e elaborar problemas de adição e subtração com números naturais e com números racionais, cuja representação decimal seja finita, utilizando estratégias diversas, como cálculo por estimativa, cálculo mental e algoritmos.
EF05MA08 - Resolver e elaborar problemas de multiplicação e divisão com números naturais e com números racionais cuja representação decimal é finita (com multiplicador natural e divisor natural e diferente de zero), utilizando estratégias diversas, como cálculo por estimativa, cálculo mental e algoritmos.
a) Incorreta. Cada tambor conterá 8,3 litros, não 5,3.
b) Incorreta. Cada tambor conterá 8,3 litros, não 6,3.
c) Incorreta. Cada tambor conterá 8,3 litros, não 7,3.
d) Correta. Para calcular a quantidade de litros de suco de laranja
que será colocada em cada tambor, basta dividir o número de litros pelo
de tambores: 141,1/17 = 8,3 litros.

\item
SAEB: Identificar elementos constitutivos dos esportes, da ginástica e
das lutas.
BNCC: EF35EF06 -- Diferenciar os conceitos de jogo e esporte,
identificando as características que os constituem na contemporaneidade
e suas manifestações (profissional e comunitária/lazer).
a) Correta. A saudação nas artes marciais ocidentais consiste em
inclinar o tronco para a frente e mostrar o devido respeito ao adversário
antes de lutar.
b) Incorreta. A vestimenta utilizada não é voltada para o
respeito e para simbolizar a paz.
c) Incorreta. A competição não é um evento que promove o
respeito, e sim a competição.
d) Incorreta. Em competições, os atletas devem usar técnicas da
luta para competir.

\item
SAEB: Identificar a importância do respeito ao oponente e às normas de
segurança na vivência das práticas corporais (jogos, lutas, ginásticas,
esportes e dança).
BNCC: EF35EF01 -- Experimentar e fruir brincadeiras e jogos populares do
Brasil e do mundo, incluindo aqueles de matriz indígena e africana, e
recriá-los, valorizando a importância desse patrimônio histórico
cultural.
a) Incorreta. O participante pode tentar a vitória, desde que
respeite as regras e as normas de segurança.
b) Correta. O próprio texto cita que os participantes devem
seguir as regras e normas de segurança para preservar a integridade
física do outro.
c) Incorreta. O texto mostra que os jogos de oposição são
realizados em duplas ou em grupos.
d) Incorreta. Os participantes devem respeitar as regras, não modificá-las.

\item
SAEB: Analisar os esportes e as lutas nas suas manifestações
profissional e de lazer.
BNCC: EF35EF06 -- Diferenciar os conceitos de jogo e esporte,
identificando as características que os constituem na contemporaneidade
e suas manifestações (profissional e comunitária/lazer).
a) Incorreta. O projeto de lei é para regulamentar a profissão dos
instrutores, não ter novos profissionais.
b) Incorreta. O objetivo é regulamentar os professores de lutas, não criar novas entidades esportivas.
c) Correta. No trecho “\ldots{} regulamenta a profissão de instrutor
de artes marciais\ldots{}”, é possível analisar que o projeto de lei é
profissionalizar e regulamentar os instrutores de lutas.
d) Incorreta. O objetivo é regulamentar os instrutores, não
fazer com que mais pessoas pratiquem lutas.

\item
BNCC: EF05CI02 - Aplicar os conhecimentos sobre as mudanças
de estado físico da água para explicar o ciclo hidrológico e analisar
suas implicações na agricultura, no clima, na geração de energia
elétrica, no provimento de água potável e no equilíbrio dos ecossistemas
regionais (ou locais).
a) Incorreta. Apesar de a gestão pública poder lidar com o problema, de
maneira a mitigar os potenciais danos, as características do Polígono das
Secas são naturais, decorrentes do ciclo hidrológico da água nesta
região.
b) Incorreta. O uso indiscriminado da água, neste caso, não explica a
falta de chuvas em toda a região de estados com diferentes populações e
costumes. 
c) Incorreta. A evaporação das águas locais é lenta, pois estas tem por
característica uma temperatura menor que as de outras localidades. Sendo
assim, a evaporação se torna mais difícil.
d) Correta. A umidade do ar é muito baixa na região, graças à ausência
de rios abundantes e da característica das águas locais de possuir
temperatura mais baixa. Dessa maneira, há menor disponibilidade de água
evaporada para o ciclo hidrológico da água, e, portanto, longos períodos
de seca são registrados.

\item
BNCC: EF05CI06 - Selecionar argumentos que justifiquem por
que os sistemas digestório e respiratório são considerados
corresponsáveis pelo processo de nutrição do organismo, com base na
identificação das funções desses sistemas.
a) Incorreta. A nutrição diz respeito às escolhas de fontes de
nutrientes, geralmente relacionada com a alimentação, e não explica o
processo de troca de gases que acontece nos pulmões.
b) Correta. A hematose é um processo de troca gasosa, em que há infusão
de gás oxigênio inspirado no sangue, ao passo em que o gás carbônico é
eliminado do corpo por meio da expiração. Os gases são, assim, trocados
o tempo todo durante o processo respiratório, que atua de maneira
conjunta com o processo digestório.
c) Incorreta. A respiração celular é um processo de obtenção de energia
baseado na oxidação, mas não ocorre nos pulmões e nem pode ser chamada
de hematose. É uma etapa distinta na produção de energia.
d) Incorreta. A filtração pulmonar não pode ser descrita pela troca
entre gases no pulmão. É um conceito genérico, que não leva em
consideração o processo inspiratório e expiratório.

\item
BNCC: EF05CI12 - Concluir sobre a periodicidade das fases da
Lua, com base na observação e no registro das formas aparentes da Lua no
céu ao longo de, pelo menos, dois meses.
a) Incorreta. Sem a Lua, não se pode prever como seriam os dias ou as
noites, pois haveria implicações na posição e iluminação da
Terra.
b) Correta. A luz da Lua é apenas reflexo da luz solar.
c) Incorreta. O Sol é uma estrela, não um satélite natural, ou seja, não
tem a órbita atrelada a um planeta.
d) Incorreta. A Lua pode ser vista da superfície terrestre sem um
telescópio em suas quatro fases, embora a visualização com telescópio
forneça detalhes mais próximos que a observação a olho nu.
\end{enumerate}

\section*{Simulado 2}

\begin{enumerate}
\item
SAEB: Compor ou decompor números naturais de até 6 ordens na
forma aditiva, ou em suas ordens, ou em adições e multiplicações.
BNCC: EF05MA01 - Ler, escrever e ordenar números naturais até a ordem das 
centenas de milhar com compreensão das principais características do 
sistema de numeração decimal. 
EF05MA10 - Concluir, por meio de investigações, que a relação de 
igualdade existente entre dois membros permanece ao adicionar, subtrair, 
multiplicar ou dividir cada um desses membros por um mesmo número, para 
construir a noção de equivalência.
EF05MA11 - Resolver e elaborar problemas cuja conversão em sentença 
matemática seja uma igualdade com uma operação em que um dos termos é 
desconhecido.
a) Incorreta. O total correto é 10.314.
b) Incorreta. O total correto é 10.314.
c) Correta. Para alcançar a resposta correta, basta somar as quantidades: 
1 x 10.000 + 3 x 100 + 1 x 10 + 4 x 1 = 10.314.
d)  Incorreta. O total correto é 10.314.

\item
SAEB: Comparar ou ordenar números racionais (naturais de até
6 ordens, representação fracionária ou decimal finita até a ordem dos
milésimos), com ou sem suporte da reta numérica.
BNCC: EF05MA01 - Ler, escrever e ordenar números naturais até a ordem das 
centenas de milhar com compreensão das principais características do 
sistema de numeração decimal. 
EF05MA10 - Concluir, por meio de investigações, que a relação de 
igualdade existente entre dois membros permanece ao adicionar, subtrair, 
multiplicar ou dividir cada um desses membros por um mesmo número, para 
construir a noção de equivalência.
EF05MA11 - Resolver e elaborar problemas cuja conversão em sentença 
matemática seja uma igualdade com uma operação em que um dos termos é 
desconhecido.
a) Incorreta. O número 380 deverá ser colocado entre o 350 e o 400.
b) Incorreta. O número 380 deverá ser colocado entre o 350 e o 400.
c) Correta. Seguindo a sequência da reta numérica conclui-se que o 
número 380 deverá ser colocado entre o 350 e o 400.
d) Incorreta. O número 380 deverá ser colocado entre o 350 e o 400.

\item
SAEB: Resolver problemas de multiplicação ou de divisão,
envolvendo números naturais de até 6 ordens, com os significados de
formação de grupos iguais (incluindo repartição equitativa e medida),
proporcionalidade ou disposição retangular.
a) Incorreta. O número de escolas é 2.008, não 26.
b) Incorreta. O número de escolas é 2.008, não 28.
c) Incorreta.  O número de escolas é 2.008, não 208.
d) Correta. O número de caixas que serão produzidas é numericamente igual
ao número de escolas que receberão as caixas. Sendo assim, 26.104/13 = 
2.008 escolas.

\item
SAEB: Inferir o padrão ou a regularidade de uma sequência de
números naturais ordenados, objetos ou figuras.
a) Incorreta. O próximo número da sequência deve ser 15, não 20.
b) Correta. Em todos os casos, o número antecedente corresponde ao 
dobro do número seguinte: 240 é o dobro de 120, que é o dobro de 60, 
que é o dobro de 30. O próximo número da sequência será, portanto, 15.
c) Incorreta. O próximo número da sequência deve ser 15, não 10.
d) Incorreta. O próximo número da sequência deve ser 15, não 5.

\item
SAEB: Determinar o horário de início, o horário de término ou
a duração de um acontecimento.
a) Correta. Como cada semana tem sete dias, então: 5 x 7 + 2 = 37 dias.
b) Incorreta. Raquel completará 12 anos em 37 dias.
c) Incorreta. Raquel completará 12 anos em 37 dias.
d) Incorreta. Raquel completará 12 anos em 37 dias.

\item
SAEB: Estimar/inferir medida de comprimento, capacidade ou
massa de objetos, utilizando unidades de medida convencionais ou não ou
medir comprimento, capacidade ou massa de objetos.
BNCC: EF05MA19 - Resolver e elaborar problemas envolvendo medidas das 
grandezas comprimento, área, massa, tempo, temperatura e capacidade, 
recorrendo a transformações entre as unidades mais usuais em contextos 
socioculturais.
a) Incorreta. A tábua tem aproximadadamente 1,5 m.
b) Incorreta. A tábua tem aproximadadamente 1,5 m.
c) Correta. 7 palmos x 21 cm = 147 cm, isto é, aproximadamente 1,5 m.
d) Incorreta. A tábua tem aproximadadamente 1,5 m.

\item
SAEB: Medir ou comparar perímetro ou área de figuras planas
desenhadas em malha quadriculada.
a) Incorreta. São necessários 12 metros de fita para Jonas concluir 
o trabalho.
b) Correta. Para escrever a letra, são necessários 10 lados de quadrados.
Como cada lado do piso mede 1,20 m, então Jonas precisará de 
1,2 m x 10 = 12 metros de fita.
c) Incorreta. São necessários 12 metros de fita para Jonas concluir 
o trabalho.
d) Incorreta. São necessários 12 metros de fita para Jonas concluir 
o trabalho.

\item
SAEB: Resolver problemas que envolvam moedas e/ou cédulas do
sistema monetário brasileiro.
a) Incorreta. O valor total da compra foi de R\$ 173,80.
b) Incorreta. O valor total da compra foi de R\$ 173,80.
c) Incorreta. O valor total da compra foi de R\$ 173,80.
d) Correta. R\$ 123,90 + 49,90 = R\$ 173,80.

\item
SAEB: Determinar a probabilidade de ocorrência de um
resultado em eventos aleatórios, quando todos os resultados possíveis
têm a mesma chance de ocorrer (equiprováveis).
BNCC: EF05MA22 - Apresentar todos os possíveis resultados de um
experimento aleatório, estimando se esses resultados são igualmente 
prováveis ou não.
EF05MA23 - Determinar a probabilidade de ocorrência de um resultado em 
eventos aleatórios, quando todos os resultados possíveis têm a mesma 
chance de ocorrer (equiprováveis).
a) Incorreta. Todos os números escritos são menores ou igual a 6, de modo
que a probabilidade de um deles sair é de 100\%.
b) Incorreta. Todos os números escritos são menores ou igual a 6, de modo
que a probabilidade de um deles sair é de 100\%.
c) Incorreta. Todos os números escritos são menores ou igual a 6, de modo
que a probabilidade de um deles sair é de 100\%.
d) Correta. Todos os números escritos são menores ou igual a 6, de modo
que a probabilidade de um deles sair é de 100\%.

\item
SAEB: Resolver problemas que envolvam dados apresentados
tabelas (simples ou de dupla entrada) ou gráficos estatísticos (barras
simples ou agrupadas, colunas simples ou agrupadas, pictóricos ou de
linhas).
BNCC: EF05MA24 - Interpretar dados estatísticos apresentados em textos, 
tabelas e gráficos (colunas ou linhas), referentes a outras áreas do 
conhecimento ou a outros contextos, como saúde e trânsito, e produzir 
textos com o objetivo de sintetizar conclusões.
a) Incorreta. A soma das notas do aluno W é 3 + 4 + 5 + 8 + 7 = 27.
b) Correta. A soma das notas do aluno X é 5 + 5 + 5 + 10 + 6 = 31.
c) Incorreta. A soma das notas do aluno Y é 4 + 9 + 3 + 9 + 5 = 30.
d) Incorreta. A soma das notas do aluno Z é 5 + 5 + 8 + 5 + 6 = 29.

\item
SAEB: Representar frações menores ou maiores que a unidade
(por meio de representações pictóricas) ou associar frações a
representações pictóricas.
BNCC: EF05MA03 - Identificar e representar frações (menores e maiores que 
a unidade), associando-as ao resultado de uma divisão ou à ideia de parte 
de um todo, utilizando a reta numérica como recurso.
EF05MA04 - Identificar frações equivalentes.
EF05MA06 - Associar as representações 10\%, 25\%, 50\%, 75\% e 100\% 
respectivamente à décima parte, quarta parte, metade, três quartos e um 
inteiro, para calcular porcentagens, utilizando estratégias pessoais, 
cálculo mental e calculadora, em contextos de educação financeira, entre 
outros.
a) Incorreta. 2/3 de 18 quadradinhos correspondem a 12 quadradinhos. 
b) Incorreta. 2/3 de 18 quadradinhos correspondem a 12 quadradinhos.
c) Correta. Para consumir 2/3 da barra ele terá que consumir 2/3 de 18
quadradinhos, isto é: 12 quadradinhos.
d) Incorreta. 2/3 de 18 quadradinhos correspondem a 12 quadradinhos.

\item
SAEB: Resolver problemas que envolvam variação de
proporcionalidade direta entre duas grandezas.
BNCC: EF05MA12 - Resolver problemas que envolvam variação de 
proporcionalidade direta entre duas grandezas, para associar a quantidade 
de um produto ao valor a pagar, alterar as quantidades de ingredientes de 
receitas, ampliar ou reduzir escala em mapas, entre outros.
a) Incorreta. São necessárias 3 colheres de sopa de pó de café para
seguir a receita de Maria.
b) Correta. Pela proporção, se quisermos usar 750 ml de água, é
necessário usar 3 colheres de sopa de pó de café, porque 750 ml 
correspondem ao triplo de 250 ml. 
c) Incorreta. São necessárias 3 colheres de sopa de pó de café para
seguir a receita de Maria.
d) Incorreta. São necessárias 3 colheres de sopa de pó de café para
seguir a receita de Maria.

\item
SAEB: Resolver problemas simples de contagem (combinatória).
BNCC: EF05MA09 - Resolver e elaborar problemas simples de contagem 
envolvendo o princípio multiplicativo, como a determinação do número de 
agrupamentos possíveis ao se combinar cada elemento de uma coleção com 
todos os elementos de outra coleção, por meio de diagramas de árvore ou 
por tabelas.
a) Incorreta. Existem 720 possibilidades distintas de formação do pódio.
b) Incorreta. Existem 720 possibilidades distintas de formação do pódio.
c) Correta. 10 x 9 x 8 = 720.
d) Incorreta. Existem 720 possibilidades distintas de formação do pódio.

\item
SAEB: Resolver problemas de multiplicação ou de divisão,
envolvendo números racionais apenas na representação decimal finita até
a ordem dos milésimos, com os significados de formação de grupos iguais
(incluindo repartição equitativa de medida), proporcionalidade ou
disposição retangular.
BNCC: EF05MA07 - Resolver e elaborar problemas de adição e subtração com 
números naturais e com números racionais, cuja representação decimal seja 
finita, utilizando estratégias diversas, como cálculo por estimativa, 
cálculo mental e algoritmos.
EF05MA08 - Resolver e elaborar problemas de multiplicação e divisão com 
números naturais e com números racionais cuja representação decimal é 
finita (com multiplicador natural e divisor natural e diferente de zero), 
utilizando estratégias diversas, como cálculo por estimativa, cálculo 
mental e algoritmos.
a) Incorreta. Couberam 52 litros de combustível no carro de Ricardo.
b) Incorreta. Couberam 52 litros de combustível no carro de Ricardo.
c) Incorreta. Couberam 52 litros de combustível no carro de Ricardo. 
d) Correta. Para encontrar a quantidade de litros que Ricardo colocou 
em seu carro, basta dividir o valor total gasto pelo valor do litro:
R\$ 191,88/3,69 = 52 litros.

\item
SAEB: Avaliar situações de preconceito no contexto das práticas
corporais.
BNCC: EF35EF09 -- Experimentar, recriar e fruir danças populares do
Brasil e do mundo e danças de matriz indígena e africana, valorizando e
respeitando os diferentes sentidos e significados dessas danças em suas
culturas de origem.
a) Incorreta. Os alunos conhecem algumas danças, mas alguns preferem não praticar essa modalidade.
b) Incorreta. A dança pode ser, sim, ensinada na escola, mas é uma
prática que tem alguns preconceitos.
c) Incorreta. Os homens podem participar, mas existem alguns
pensamentos equivocados de que a dança é exclusiva para as mulheres.
d) Correta. Com no trecho “\ldots{} 100\% deles responderam que o maior
preconceito está ligado ao gênero por parte dos meninos\ldots{}”, é possível
analisar que os meninos acreditam que as danças são para apenas um
gênero, ou seja, para o gênero feminino.

\item
SAEB: Avaliar meios para superar situações de preconceito no contexto
das práticas corporais.
BNCC: EF35EF06 -- Diferenciar os conceitos de jogo e esporte,
identificando as características que os constituem na contemporaneidade
e suas manifestações (profissional e comunitária/lazer).
a) Incorreta. O evento serviu para combater o preconceito, não
para apresentar novos esportes.
b) Incorreta. Foram os paratletas que deram palestras no vento
para os alunos falando sobre a inclusão no esporte.
c) Correta. No trecho “\ldots{} esporte como mecanismo de motivação,
superação e combate ao preconceito\ldots{}”, é possível analisar as vantagens
e benefícios que os esportes podem proporcionar.
d) Incorreta. Não foram os alunos que organizaram o evento e sim
os paratletas e a prefeitura local.

\item
SAEB: Identificar as brincadeiras e os jogos populares como patrimônio
histórico-cultural.
BNCC: EF35EF01 -- Experimentar e fruir brincadeiras e jogos populares do
Brasil e do mundo, incluindo aqueles de matriz indígena e africana, e
recriá-los, valorizando a importância desse patrimônio histórico
cultural.
a) Correta. O texto mostra que algumas crianças não sabiam cantar
algumas cantigas tradicionais.
b) Incorreta. É a tecnologia que está afastando as crianças das
cantigas populares.
c) Incorreta. As cantigas trazem muitas vantagens e benefícios
aos alunos por serem algo lúdico.
d) Incorreta. As cantigas são atividades sempre usadas no ambiente escolar.

\item
BNCC: EF05CI06 - Selecionar argumentos que justifiquem por
que os sistemas digestório e respiratório são considerados
corresponsáveis pelo processo de nutrição do organismo, com base na
identificação das funções desses sistemas.
a) Incorreta. O sistema circulatório não tem por função a produção
gasosa, e sim o transporte do gás oxigênio incorporado à corrente
sanguínea para as estruturas do corpo humano.
b) Correta. Além de transportar nutrientes e gás oxigênio para os
tecidos do corpo humano, o sistema circulatório encaminha as substâncias
tóxicas, como as toxinas, para a eliminação.
c) Incorreta. A maturação de nutrientes não acontece no sistema
circulatório.
d) Incorreta. A digestão de alimentos é função do sistema digestório,
capaz de quebrá-los em diversos órgãos para a absorção de diferentes
partes de um alimento. O sistema circulatório atua no transporte dessas
células concentradas em nutrientes para as extremidades do corpo humano.

\item
BNCC: EF05CI03 - Selecionar argumentos que justifiquem a
importância da cobertura vegetal para a manutenção do ciclo da água, a
conservação dos solos, dos cursos de água e da qualidade do ar
atmosférico.
a) Incorreta. O ressecamento de nascentes pode ocorrer por falta de água
nos lençóis freáticos. Os problemas apontados e as medidas de correção
não buscam combater esse problema.
b) Incorreta. Dentre os problemas apontados e soluções pretendidas, não
há relação com o cruzamento de correntes oceânicas distintas, que ocorre
com a movimentação de grandes quantidades de água em oceanos e mares.
c) Incorreta. O escoamento regular da água de um rio não representa um
problema a ser solucionado, sendo influenciado por fatores como o regime
de chuvas e o ciclo hidrológico de determinada região.
d) Correta. Ao tratar o solo erodido, replantar uma estrutura vegetal,
reflorestar o entorno do rio e recolher o lixo depositado, os
especialistas buscam prevenir a intensificação do assoreamento do rio
pela atividade humana, processo que pode atrapalhar os cursos d'água,
causar inundações e morte de espécies marinhas.

\item
BNCC: EF05CI11 - Associar o movimento diário do Sol e das
demais estrelas no céu ao movimento de rotação da Terra.
a) Incorreta. Se a primavera estava terminando no Hemisfério Sul, o
oposto acontecia no Hemisfério Norte. Portanto, terminava o outono para
se iniciar o inverno.
b) Incorreta. No Hemisfério Norte, o outono terminava para dar lugar ao
inverno.
c) Incorreta. O verão foi iniciado no Hemisfério Sul; não poderia, portanto, também ter começado no hemisfério oposto.
d) Correta. No Hemisfério Norte, no mesmo período do ano, ocorre o início
do inverno, graças ao movimento de Rotação da terra e da inclinação do
planeta, fazendo com que haja menor incidência da luz solar na região
oposta ao Hemisfério Sul.
\end{enumerate}

\section*{Simulado 3}

\begin{enumerate}
\item
SAEB: Compor ou decompor números naturais de até 6 ordens na
forma aditiva, ou em suas ordens, ou em adições e multiplicações.
BNCC: EF05MA01 - Ler, escrever e ordenar números naturais até a ordem das 
centenas de milhar com compreensão das principais características do 
sistema de numeração decimal. 
EF05MA10 - Concluir, por meio de investigações, que a relação de 
igualdade existente entre dois membros permanece ao adicionar, subtrair, 
multiplicar ou dividir cada um desses membros por um mesmo número, para 
construir a noção de equivalência.
EF05MA11 - Resolver e elaborar problemas cuja conversão em sentença 
matemática seja uma igualdade com uma operação em que um dos termos é 
desconhecido.
a) Incorreta. O total é de 4.335. 
b) Correta. 4 x 1.000 + 3 x 100 + 3 x 10 + 5 x 1 = 
4.000 + 300 + 30 + 5 = 4.335.
c) Incorreta. O total é de 4.335. 
d) Incorreta. O total é de 4.335. 

\item
SAEB: Escrever números racionais (naturais de até 6 ordens,
representação fracionária ou decimal finita até a ordem dos milésimos)
em sua representação por algarismos ou em língua materna ou associar o
registro numérico ao registro em língua materna.
BNCC: EF05MA01 - Ler, escrever e ordenar números naturais até a ordem das 
centenas de milhar com compreensão das principais características do 
sistema de numeração decimal. 
EF05MA10 - Concluir, por meio de investigações, que a relação de 
igualdade existente entre dois membros permanece ao adicionar, subtrair, 
multiplicar ou dividir cada um desses membros por um mesmo número, para 
construir a noção de equivalência.
EF05MA11 - Resolver e elaborar problemas cuja conversão em sentença 
matemática seja uma igualdade com uma operação em que um dos termos é 
desconhecido.
a) Incorreta. O número correto é 75.4271.
b) Correta. Basta colocar os algarismos dados em ordem decrescente para
formar o número desejado.
c) Incorreta. O número correto é 75.4271.
d) Incorreta. O número correto é 75.4271.

\item
SAEB: Calcular o resultado de multiplicações ou divisões
envolvendo números naturais de até 6 ordens.
a) Incorreta. O número que completa adequadamente os quadradinhos é 8.
b) Incorreta. O número que completa adequadamente os quadradinhos é 8.
c) Incorreta. O número que completa adequadamente os quadradinhos é 8.
d) Correta. Basta fazer as contas para identificar que o número que
completa adequadamente os quadradinhos é o 8.

\item
SAEB: Inferir o padrão ou a regularidade de uma sequência de
números naturais ordenados, objetos ou figuras.
a) Incorreta. Observando a sequência apresentava, verifica-se que,
para descobrir um termo, basta multiplicar por 3 seu antecessor. 
b) Incorreta. Observando a sequência apresentava, verifica-se que,
para descobrir um termo, basta multiplicar por 3 seu antecessor. 
c) Correta. 3 X 3 = 9; 9 X 3 = 27; 27 X 3 = 81; 81 X 3 = 243; 
243 X 3 = 729. 
d) Incorreta. Observando a sequência apresentava, verifica-se que,
para descobrir um termo, basta multiplicar por 3 seu antecessor.

\item
SAEB: Determinar o horário de início, o horário de término ou
a duração de um acontecimento.
a) Incorreta. O horário de encerramento das atividades é 17h30.
b) Correta. 9 + 8,5 = 17,5 = 17 horas e 30 minutos.
c) Incorreta. O horário de encerramento das atividades é 17h30.
d) Incorreta. O horário de encerramento das atividades é 17h30.

\item
SAEB: Determinar o horário de início, o horário de término ou
a duração de um acontecimento.
a) Correta. A diferença entre 15h34 e 14h55 é de 39 minutos.
b) Incorreta. O programa tem duração de 39 minutos.
c) Incorreta. O programa tem duração de 39 minutos.
d) Incorreta. O programa tem duração de 39 minutos.

\item
SAEB: Resolver problemas que envolvam área de figuras planas.
a) Incorreta. O total é de 18 metros quadrados.
b) Incorreta. O total é de 18 metros quadrados.
c) Correta. Contando o número de quadradinhos que representa o tapete
chega-se a 18 (17 quadradinhos completos somados a duas metades), que
correspondem a 18 metros quadrados de carpete.
d) Incorreta. O total é de 18 metros quadrados.

\item
SAEB: Determinar a probabilidade de ocorrência de um
resultado em eventos aleatórios, quando todos os resultados possíveis
têm a mesma chance de ocorrer (equiprováveis).
BNCC: EF05MA22, EF05MA23.
a) Incorreta. A probabilidade é de 38\%.
b) Correta. O total de números é 50. Os números de interesse são 19, 
pois o número 20 e o 40 não entram na contagem. Dessa forma, calcula-se
a probabilidade da seguinte maneira: 19/50 = 38/100 = 38\%.
c) Incorreta. A probabilidade é de 38\%.
d) Incorreta. A probabilidade é de 38\%.

\item
SAEB: Argumentar ou analisar argumentações/conclusões com
base em dados apresentados em tabelas (simples ou de dupla entrada) ou
gráficos (barras simples ou agrupadas, colunas simples ou agrupadas,
pictóricos ou de linhas).
BNCC: EF05MA24 - Interpretar dados estatísticos apresentados em textos, 
tabelas e gráficos (colunas ou linhas), referentes a outras áreas do 
conhecimento ou a outros contextos, como saúde e trânsito, e produzir 
textos com o objetivo de sintetizar conclusões.
a) Incorreta. Na disciplina II, o universitário obteve média 8,0,
suficiente para aprovação. 
b) Incorreta. Na disciplina III, o universitário obteve média 6,0,
suficiente para aprovação. 
c) Correta. A única disciplina em que o universitário apresentou 
nota inferior à média foi a IV, com nota 5,00.
d) Incorreta. Na disciplina V, o universitário obteve média 7,5,
suficiente para aprovação.

\item
SAEB: Argumentar ou analisar
argumentações/conclusões com base em dados apresentados em tabelas
(simples ou de dupla entrada) ou gráficos (barras simples ou agrupadas,
colunas simples ou agrupadas, pictóricos ou de linhas). 
BNCC: EF05MA24 - Interpretar dados estatísticos apresentados em textos, 
tabelas e gráficos (colunas ou linhas), referentes a outras áreas do 
conhecimento ou a outros contextos, como saúde e trânsito, e produzir 
textos com o objetivo de sintetizar conclusões.
a) Incorreta. Os nadadores das raias 1 e 8 foram mais lentos do que
os das raias 3, 5 e 6. 
b) Correta. Os nadadores mais velozes são os que fizeram a prova em
menos tempo, isto é, os que nadaram nas raias 3 (20,50), 5 (20,60) 
e 6 (20,60).
c) Incorreta. Os nadadores das raias 1, 7 e 8 foram mais lentos do que
os das raias 3, 5 e 6.
d) Incorreta. Os nadadores das raias 5 e 7 foram mais lentos do que
os das raias 3, 5 e 6.

\item
SAEB: Resolver problemas que envolvam fração como resultado
de uma divisão (quociente).
BNCC: EF05MA03 - Identificar e representar frações (menores e maiores que 
a unidade), associando-as ao resultado de uma divisão ou à ideia de parte 
de um todo, utilizando a reta numérica como recurso.
EF05MA04 - Identificar frações equivalentes.
EF05MA06 - Associar as representações 10\%, 25\%, 50\%, 75\% e 100\% 
respectivamente à décima parte, quarta parte, metade, três quartos e um 
inteiro, para calcular porcentagens, utilizando estratégias pessoais, 
cálculo mental e calculadora, em contextos de educação financeira, entre 
outros.
a) Incorreta. R\$ 300,00 corresponde a metade do valor do prêmio. Esse é
o valor ganho pelo primeiro colocado.
b) Incorreta. R\$ 200,00 corresponde a um terço do valor do prêmio. Esse é
o valor ganho pelo segundo colocado.
c) Correta. O primeiro colocado receberá metade de R\$ 600,00, 
o que corresponde a R\$ 300,00. O segundo ganhará um terço de
R\$ 600,00, isto é, R\$ 200,00. O restante, que será destinado ao
terceiro, é igual a R\$ 600,00 - (R\$ 300,00 + R\$ 200,00) = R\$ 100,00.
d) Incorreta. Nenhum dos três primeiros colocados ganhou apenas 
R\$ 50,00.

\item
SAEB: Identificar frações equivalentes.
BNCC: EF05MA03 - Identificar e representar frações (menores e maiores que 
a unidade), associando-as ao resultado de uma divisão ou à ideia de parte 
de um todo, utilizando a reta numérica como recurso.
EF05MA04 - Identificar frações equivalentes.
EF05MA06 - Associar as representações 10\%, 25\%, 50\%, 75\% e 100\% 
respectivamente à décima parte, quarta parte, metade, três quartos e um 
inteiro, para calcular porcentagens, utilizando estratégias pessoais, 
cálculo mental e calculadora, em contextos de educação financeira, entre 
outros.
a) Incorreta. Camilo perdeu 6 pênaltis, logo a razão é de 6/14, que 
é igual a 3/7.
b) Correta. Se ele acertou 8 de 14, então errou 6 pênaltis.
Portanto a razão será: 6/14 = 3/7.
c) Incorreta. Camilo perdeu 6 pênaltis, logo a razão é de 6/14, que 
é igual a 3/7.
d) Incorreta. Camilo perdeu 6 pênaltis, logo a razão é de 6/14, que 
é igual a 3/7.

\item
SAEB: Resolver problemas simples de contagem (combinatória).
EF05MA09 - Resolver e elaborar problemas simples de contagem envolvendo o 
princípio multiplicativo, como a determinação do número de agrupamentos 
possíveis ao se combinar cada elemento de uma coleção com todos os 
elementos de outra coleção, por meio de diagramas de árvore ou por 
tabelas.
a) Incorreta. Existem 15 maneiras diferentes de posicionar essa cadeira. 
b) Correta. Se temos 3 opções para o assento e 5 opções para o encosto, 
multiplicamos uma pela outra para obter o total de maneiras diferentes 
de se posicionar essa cadeira. 3 x 5 = 15.
c) Incorreta. Existem 15 maneiras diferentes de posicionar essa cadeira.
d) Incorreta. Existem 15 maneiras diferentes de posicionar essa cadeira.

\item
SAEB: Resolver problemas de multiplicação ou de divisão,
envolvendo números racionais apenas na representação decimal finita até
a ordem dos milésimos, com os significados de formação de grupos iguais
(incluindo repartição equitativa de medida), proporcionalidade ou
disposição retangular.
BNCC: EF05MA07 - Resolver e elaborar problemas de adição e subtração com 
números naturais e com números racionais, cuja representação decimal seja 
finita, utilizando estratégias diversas, como cálculo por estimativa, 
cálculo mental e algoritmos.
EF05MA08 - Resolver e elaborar problemas de multiplicação e divisão com 
números naturais e com números racionais cuja representação decimal é 
finita (com multiplicador natural e divisor natural e diferente de zero), 
utilizando estratégias diversas, como cálculo por estimativa, cálculo 
mental e algoritmos.
c) Correta. Dividindo o fio de cobre em pedaços de tamanho desejado, 
obtemos o total de 50 pedaços, pois 7 metros /0,14 metros = 50.
Como ele já possui 8 pedaços, ele precisará de mais 42, pois 50 - 8 = 42.

\item
SAEB: Valorizar o patrimônio histórico representado pelas brincadeiras e
jogos, com ênfase naqueles de origem indígena e africana.
BNCC: EF35EF01 -- Experimentar e fruir brincadeiras e jogos
populares do Brasil e do mundo, incluindo aqueles de matriz indígena e
africana, e recriá-los, valorizando a importância desse patrimônio
histórico cultural.
a) Correta. As brincadeiras citadas (peteca e perna de pau) são tradicionais de origem indígenas que muitas pessoas
conhecem.
b) Incorreta. As brincadeiras de origem indígena também são
realizadas por outros povos e culturas.
c) Incorreta. No trecho “\ldots{}Existem muitos jeitos de
brincar\ldots{}”, é possível analisar que existem variações nas brincadeiras.
d) Incorreta. No trecho “\ldots{}o objetivo é sempre desfrutar o
momento e a companhia dos amigos\ldots{}”, podemos compreender que as
brincadeiras são realizadas em grupo para promover a socialização.

\item
SAEB: Analisar o protagonismo do trabalho coletivo na vivência dos jogos
populares e dos esportes.
BNCC: EF35EF06 -- Diferenciar os conceitos de jogo e esporte,
identificando as características que os constituem na contemporaneidade
e suas manifestações (profissional e comunitária/lazer).
a) Correta. Os jogos são atividades voltadas para a diversão e a socialização.
b) Incorreta. Os esportes competitivos visam apenas a competições e vitórias.
c) Incorreta. Assim como os esportes, as modalidades olímpicas
visam ao alto rendimento e às competições.
d) Incorreta. O texto fala sobre jogos pré-depsortivos e não
sobre atividades escolares.

\item
SAEB: Comparar os elementos constitutivos de danças populares do Brasil
e do mundo com aqueles de danças de matrizes indígena e africana.
BNCC: EF35EF10 -- Comparar e identificar os elementos constitutivos
comuns e diferentes (ritmo, espaço, gestos) em danças populares do
Brasil e do mundo e danças de matriz indígena e africana.
a) Incorreta. As danças podem ser realizadas em espaço aberto ou
fechado e isso não define se uma atividade é uma dança oficial ou não.
b) Incorreta. A dança pode ser realizada individualmente ou em
duplas. O fato de a dança ser em grupo não define que uma prática seja
considerada uma dança.
c) Incorreta. As vestimentas e pinturas não são próprias da
dança, já que em esportes, ginásticas e lutas podem aparecer esses elementos.
d) Correta. A pessoa se movimentado na batida da música
(instrumento musical) vai estar realizando o elemento constitutivo do
ritmo e do gesto da dança.

\item
BNCC: EF05CI06 - Selecionar argumentos que justifiquem por
que os sistemas digestório e respiratório são considerados
corresponsáveis pelo processo de nutrição do organismo, com base na
identificação das funções desses sistemas.
a) Correta. A produção do quimo e do quilo
é parte do processo de digestão.
b) Incorreta. A alimentação vem antes da digestão e é, costumeiramente,
uma ação mecânica que acontece a partir da boca. É dela que advêm os
nutrientes a serem digeridos e processados no organismo. 
c) Incorreta. O processo de hidratação do organismo ocorre com a
ingestão de água; não é, portanto, descrito nas etapas de produção de
quimo e quilo.
d) Incorreta. A evacuação é a etapa de eliminação das fezes do
organismo, que são o produto final restante de todas as etapas da
digestão. Trata-se, portanto, de apenas mais uma etapa do processo
de digestão.

\item
BNCC: EF05CI08 - Organizar um cardápio equilibrado com base
nas características dos grupos alimentares (nutrientes e calorias) e nas
necessidades individuais (atividades realizadas, idade, sexo etc.) para
a manutenção da saúde do organismo.
a) Incorreta. Alimentos ricos em gorduras e açúcares representam,
geralmente, maior teor calórico que outros, embora somente essa
informação não seja usada para classificação em calorias vazias ou
cheias.
b) Incorreta. A alimentação com fontes pobres de calorias não é
essencial para as dietas dos seres humanos, sejam elas quais forem, e
deve ser realizada de maneira consciente e sem excessos.
c) Correta. Como são compostas majoritariamente por gorduras e açúcares,
as calorias vazias são pobres em demais nutrientes como vitaminas,
fibras e proteínas.
d) Incorreta. Calorias vazias podem, quando ingeridas em excesso, causar
doenças como o câncer e a diabetes. Não são, portanto, menos ofensivas
ao organismo que as calorias cheias.

\item
BNCC: EF05CI12 - Concluir sobre a periodicidade das fases da
Lua, com base na observação e no registro das formas aparentes da Lua no
céu ao longo de, pelo menos, dois meses.
a) Incorreta. A lua cheia acontece na semana 4; portanto, num ciclo
lunar sinódico, em meados da semana 5, ocorrerá a fase seguinte, que é a
lua minguante.
b) Incorreta. Em um ciclo lunar sinódico, a lua nova não pode ocorrer
logo após a lua cheia sem a passagem pela lua minguante.
c) Incorreta. A lua cheia não pode dar lugar à lua crescente, que é a
fase da lua que a antecede. Isso só poderia ocorrer caso a fase anterior
fosse a lua nova.
d) Correta. Como se trata de um ciclo lunar sinódico, ou seja, completo,
a fase seguinte que se espera observar no meio da semana é a lua
minguante.
\end{enumerate}

\section*{Simulado 4}

\begin{enumerate}
\item
SAEB: Identificar a ordem ocupada por um algarismo ou seu
valor posicional (ou valor relativo) em um número natural de até 6
ordens.
BNCC: EF05MA01 - Ler, escrever e ordenar números naturais até a ordem das 
centenas de milhar com compreensão das principais características do 
sistema de numeração decimal. 
EF05MA10 - Concluir, por meio de investigações, que a relação de 
igualdade existente entre dois membros permanece ao adicionar, subtrair, 
multiplicar ou dividir cada um desses membros por um mesmo número, para 
construir a noção de equivalência.
EF05MA11 - Resolver e elaborar problemas cuja conversão em sentença 
matemática seja uma igualdade com uma operação em que um dos termos é 
desconhecido.
a) Correta. No número 3.756, o algarismo que ocupa o valor posicional
de centena é o 7.
b) Incorreta. No número 3.756, o algarismo que ocupa o valor posicional
de centenas é o 7. 6 ocupa o valor da unidade. 
c) Incorreta. No número 3.756, o algarismo que ocupa o valor posicional
de centenas é o 7. 5 ocupa o valor da dezena.
d) Incorreta. No número 3.756, o algarismo que ocupa o valor posicional
de centenas é o 7. 3 ocupa o valor do milhar.

\item
SAEB: Compor ou decompor números naturais de até 6 ordens na
forma aditiva, ou em suas ordens, ou em adições e multiplicações.
BNCC: EF05MA01 - Ler, escrever e ordenar números naturais até a ordem das 
centenas de milhar com compreensão das principais características do 
sistema de numeração decimal. 
EF05MA10 - Concluir, por meio de investigações, que a relação de 
igualdade existente entre dois membros permanece ao adicionar, subtrair, 
multiplicar ou dividir cada um desses membros por um mesmo número, para 
construir a noção de equivalência.
EF05MA11 - Resolver e elaborar problemas cuja conversão em sentença 
matemática seja uma igualdade com uma operação em que um dos termos é 
desconhecido.
a) Incorreta. Jorge colheu 255 produtos. 
b) Correta. 10 dezenas de pés de rúcula = 10 X 10 = 100 produtos.
1 centena de espigas de milho = 100 produtos. 5 dezenas de tomate
= 50 produtos. 2 cebolas e 3 pepinos = 5 produtos. Somando todos os
produtos teremos: 100 + 100 + 50 + 5 = 255.
c) Incorreta. Jorge colheu 255 produtos.
d) Incorreta. Jorge colheu 255 produtos.

\item
SAEB: Resolver problemas de adição ou de subtração,
envolvendo números naturais de até 6 ordens, com os significados de
juntar, acrescentar, separar, retirar, comparar ou completar.
Resolver problemas de multiplicação ou de divisão, envolvendo números
naturais de até 6 ordens, com os significados de formação de grupos
iguais (incluindo repartição equitativa e medida), proporcionalidade ou
disposição retangular.
BNCC: EF05MA01 - Ler, escrever e ordenar números naturais até a ordem das 
centenas de milhar com compreensão das principais características do 
sistema de numeração decimal. 
EF05MA10 - Concluir, por meio de investigações, que a relação de 
igualdade existente entre dois membros permanece ao adicionar, subtrair, 
multiplicar ou dividir cada um desses membros por um mesmo número, para 
construir a noção de equivalência.
EF05MA11 - Resolver e elaborar problemas cuja conversão em sentença 
matemática seja uma igualdade com uma operação em que um dos termos é 
desconhecido.
a) Incorreta.
b) Incorreta.
c) Resposta: C. 4 + 2 x 7 + 8 / 2 = 4 + 14 + 4 = 22.
d) Incorreta.

\item
SAEB: Inferir os elementos ausentes em uma sequência de
números naturais ordenados, objetos ou figuras.
a) Incorreta. O número que falta na sequência é 302.
b) Correta. O número que está faltando na sequência é 302 (antecessor 
de 303), pois, na sequência, somam-se 100 unidades ao número seguinte: 
2, 102, 202, 302, 402 e 502.
c) Incorreta. O número que falta na sequência é 302.
d) Incorreta. O número que falta na sequência é 302.

\item
SAEB: Estimar/inferir medida de comprimento, capacidade ou
massa de objetos, utilizando unidades de medida convencionais ou não ou
medir comprimento, capacidade ou massa de objetos.
a) Incorreta. 
b) Incorreta.
c) Correta. Pela análise figura podemos estimar que o garrafão
será o recipiente que pode conter exetamente 3 litos de água.
d) Incorreta.

\item
SAEB: Medir ou comparar perímetro ou área de figuras planas
desenhadas em malha quadriculada.
a) Incorreta. O perímetro foi duplicado.
b) Incorreta. O perímetro foi duplicado.
c) Correta. Observando as figuras, percebe-se que a base do triângulo foi
duplicada (ela tinha dois quadradinhos na Figura I e tem quatro na Figura 
II), de modo que o perímetro também será duplicado.
d) Incorreta. O perímetro foi duplicado.

\item
SAEB: Resolver problemas que envolvam moedas e/ou cédulas do
sistema monetário brasileiro.
a) Correta. 1 cédula de R\$ 10,00 será somado aos R\$ 25,00 das outras 
cinco cédulas e aos R\$ 3,00 das moedas. R\$ 10,00 + R\$ 25,00 + R\$ 3,00
= R\$ 38,00. 
b) Incorreta. O valor gasto por Rafael é R\$ 38,00, e o total proposto nesta alternativa é de R\$ 23,00.
c) Incorreta. O valor gasto por Rafael é R\$ 38,00, e o total proposto nesta alternativa é de R\$ 28,00.
d) Incorreta. O valor gasto por Rafael é R\$ 38,00, e o total proposto nesta alternativa é de R\$ 32,00.

\item
SAEB: Relacionar valores de moedas e/ou cédulas do sistema
monetário brasileiro, com base nas imagens desses objetos.
a) Incorreta. O total no cofrinho de Alana é de R\$ 10,00.
b) Incorreta. O total no cofrinho de Alana é de R\$ 10,00.
c) Correta. 10 x 0,05 + 5 x 0,50 + 70 x 0,10 = 0,50 + 2,50 + 7 = R\$ 10,00. Portanto, Alana tem R\$ 10,00 no cofrinho.
d) Incorreta. O total no cofrinho de Alana é de R\$ 10,00.

\item
SAEB: Determinar a probabilidade de ocorrência de um
resultado em eventos aleatórios, quando todos os resultados possíveis
têm a mesma chance de ocorrer (equiprováveis).
BNCC: EF05MA22 - Apresentar todos os possíveis resultados de um experimento aleatório, estimando se esses resultados são igualmente prováveis ou não.
EF05MA23 - Determinar a probabilidade de ocorrência de um resultado em eventos aleatórios, quando todos os resultados possíveis têm a mesma chance de ocorrer (equiprováveis).
a) Incorreta. A probabilidade de escolha de um número par é de 50\%.
b) Incorreta. A probabilidade de escolha de um número par é de 50\%. 
c) Correta. No conjunto de números apresentados, a metade é par, 
de modo que a probabilidade de se escolher um número par é de 50\%.
d) Incorreta. A probabilidade de escolha de um número par é de 50\%.

\item
SAEB: Argumentar ou analisar argumentações/conclusões com
base em dados apresentados em tabelas (simples ou de dupla entrada) ou
gráficos (barras simples ou agrupadas, colunas simples ou agrupadas,
pictóricos ou de linhas).
a) Incorreta. O atleta A teve um total de 18 pontos (6 + 6 + 6), um a menos do que o atleta D.
b) Incorreta. O atleta B teve um total de 18 pontos (7 + 3 + 8), um a menos do que o atleta D.
c) Incorreta. O atleta C teve um total de 18 pontos (5 + 7 + 6), um a menos do que o atleta D.
d) O atleta D é o vencedor, porque obteve 19 pontos (5 + 6 + 8) -- um a mais do que os outros três atletas.

\item
SAEB: Resolver problemas que envolvam fração como resultado
de uma divisão (quociente).
BNCC: EF05MA03 - Identificar e representar frações (menores e maiores que a unidade), associando-as ao resultado de uma divisão ou à ideia de parte de um todo, utilizando a reta numérica como recurso.
EF05MA04 - Identificar frações equivalentes.
EF05MA06 - Associar as representações 10\%, 25\%, 50\%, 75\% e 100\% respectivamente à décima parte, quarta parte, metade, três quartos e um inteiro, para calcular porcentagens, utilizando estratégias pessoais, cálculo mental e calculadora, em contextos de educação financeira, entre outros.
a) Incorreta. O piloto abandonou a prova depois de 22 voltas.
b) Correta. 2/7 x 77 = 22 voltas.
c) Incorreta. O piloto abandonou a prova depois de 22 voltas.
d) Incorreta. O piloto abandonou a prova depois de 22 voltas.

\item
SAEB: Resolver problemas que envolvam variação de
proporcionalidade direta entre duas grandezas.
BNCC: EF05MA12 - Resolver problemas que envolvam variação de proporcionalidade direta entre duas grandezas, para associar a quantidade de um produto ao valor a pagar, alterar as quantidades de ingredientes de receitas, ampliar ou reduzir escala em mapas, entre outros.
a) Incorreta. 16/2000 = 1/125.
b) Incorreta. 16/2000 = 1/125.
c) Correta. 16/2000 = 1/125.
d) Incorreta. 16/2000 = 1/125.

\item
SAEB: Resolver problemas simples de contagem (combinatória).
BNCC: EF05MA09 - Resolver e elaborar problemas simples de contagem envolvendo o princípio multiplicativo, como a determinação do número de agrupamentos possíveis ao se combinar cada elemento de uma coleção com todos os elementos de outra coleção, por meio de diagramas de árvore ou por tabelas.
a) Incorreta. É possível formar 72 algarismos nas condições propostas
no enunciado. 
b) Correta. Para a escolha do primeiro algarismo, existem 9
opções de algarismos; para a escolha do segundo, há 8 opção, já que
não é permitido repetir algarismos. Dessa maneira, é possível formar 72
(9 X 8) números com essas condições.
c) Incorreta. É possível formar 72 algarismos nas condições propostas
no enunciado.
d) Incorreta. É possível formar 72 algarismos nas condições propostas
no enunciado.

\item
SAEB: Resolver problemas de multiplicação ou de divisão,
envolvendo números racionais apenas na representação decimal finita até
a ordem dos milésimos, com os significados de formação de grupos iguais
(incluindo repartição equitativa de medida), proporcionalidade ou
disposição retangular.
BNCC: EF05MA07 - Resolver e elaborar problemas de adição e subtração com números naturais e com números racionais, cuja representação decimal seja finita, utilizando estratégias diversas, como cálculo por estimativa, cálculo mental e algoritmos.
EF05MA08 - Resolver e elaborar problemas de multiplicação e divisão com números naturais e com números racionais cuja representação decimal é finita (com multiplicador natural e divisor natural e diferente de zero), utilizando estratégias diversas, como cálculo por estimativa, cálculo mental e algoritmos.
a) Incorreta. O valor da parcela é de R\$ 73,10.
b) Incorreta. O valor da parcela é de R\$ 73,10. 
c) Correta. O valor que Alexandre pagará pelo tênis é R\$ 324,80 -- 32,40
= R\$ 292,40. Dividindo esse valor em 3 vezes, teremos: R\$ 292,40/4 = 
R\$ 73,10.
d) Incorreta. O valor da parcela é de R\$ 73,10.

\item
SAEB: Valorizar o patrimônio histórico representado pelas danças
populares, com ênfase naquelas de matriz indígena e africana
BNCC: EF35EF11 -- Formular e utilizar estratégias para a execução de
elementos constitutivos das danças populares do Brasil e do mundo, e das
danças de matriz indígena e africana.
a) Incorreta. Apenas o samba, entre ambas, é realizado no carnaval.
b) Correta. O samba se originou do semba, ou seja, as duas
surgiram com base nas influências culturais da África.
c) Incorreta. As duas danças não são voltadas para as
competições.
d) Incorreta. O samba e o semba não são danças religiosas.

\item
SAEB: Identificar a importância
do respeito ao oponente e às normas de segurança na vivência das
práticas corporais (jogos, lutas, ginásticas, esportes e dança).
BNCC: EF35EF15 -- Identificar as características das lutas do contexto
comunitário e regional e lutas de matriz indígena e africana,
reconhecendo as diferenças entre lutas e brigas e entre lutas e as
demais práticas corporais.
a) Correta. O trecho “\ldots{} As artes marciais ensinam a filosofia
do respeito\ldots{}” mostra que as lutas ensinam o valor ético de respeitar
o outro.
b) Incorreta. O objetivo das lutas é ensinar valores éticos e
morais, não criar novas competições.
c) Incorreta. O projeto apresentado serve para transformar os alunos
em cidadãos do bem.
d) Incorreta. É justamente o oposto: as lutas
evitam e amenizam as brigas entre as pessoas.

\item
SAEB: Analisar os esportes e as
lutas nas suas manifestações profissional e de lazer.
BNCC: EF35EF06 -- Diferenciar os conceitos de jogo e esporte,
identificando as características que os constituem na contemporaneidade
e suas manifestações (profissional e comunitária/lazer).
a) Correta. No começo as pessoas adaptaram alguns materiais para
simular as rebatidas na bola realizadas no tênis de campo.
b) Incorreta. O texto cita que antes o tênis de mesa era uma
brincadeira.
c) Incorreta. Não era um treinamento, e sim uma brincadeira.
d) Incorreta. O tênis de mesa antigamente era voltado para o
lazer.

\item
BNCC: EF05CI02 - Aplicar os conhecimentos sobre as mudanças
de estado físico da água para explicar o ciclo hidrológico e analisar
suas implicações na agricultura, no clima, na geração de energia
elétrica, no provimento de água potável e no equilíbrio dos ecossistemas
regionais (ou locais).
a) Correta. Em dias mais quentes, há a intensificação de fenômenos
essenciais para o aumento do regime de chuvas: a evaporação de águas de
oceanos, rios e solos e a evapotranspiração de plantas. Com isso, o ar
se torna mais úmido e as chuvas caem com maior frequência.
b) Incorreta. A luz solar mais intensa pode aquecer mais as fontes de
água a ser vaporizada, mas não consegue impedir a frequência de chuvas,
que é relatada no texto como mais frequente no verão.
c) Incorreta. Em dias quentes, é comum que se observe um considerável
aumento na precipitação das nuvens.
d) Incorreta. O sol não participa da formação de gotículas de chuva
diretamente, e sim da vaporização da água que, ao chegar nas nuvens, se
agrega para formar gotículas.

\item
BNCC: EF05CI04 - Identificar os principais usos da água e de
outros materiais nas atividades cotidianas para discutir e propor formas
sustentáveis de utilização desses recursos.
a) Incorreta. As plantações não sobrevivem com o uso de resíduos de água
com toxinas e contaminantes, por isso nem todo tipo de água pode ser
reutilizada.
b) Incorreta. Apesar de pequenos agricultores lidarem com um volume de
plantações muito menor que o de grandes empresas, não é esse o fato que
justifica a reutilização da água, que também deve ser feita em escala
industrial.
c) Correta. Atividades como a irrigação consomem a água para fins
não potáveis. Nessas situações, águas residuais podem ser reutilizadas.
d) Incorreta. A água de reúso não desempenha papel fundamental e
decisivo no crescimento de vegetais, pois não apresenta componentes que
possam acelerar esse processo.

\item
BNCC: EF05CI06 - Selecionar argumentos que justifiquem por
que os sistemas digestório e respiratório são considerados
corresponsáveis pelo processo de nutrição do organismo, com base na
identificação das funções desses sistemas. 
EF05CI07 - Justificar a
relação entre o funcionamento do sistema circulatório, a distribuição
dos nutrientes pelo organismo e a eliminação dos resíduos produzidos.
a) Incorreta. Todos os três sistemas possuem papéis importantes e
fundamentais no processo de nutrição do organismo, sendo indispensáveis
para a produção de energia e manutenção da vida.
b) Correta. A ação conjunta dos sistemas garante a digestão, seleção de
nutrientes, produção de energia na divisão dos nutrientes, transporte e
fixação nos tecidos do corpo humano.
c) Incorreta. O sistema digestório é composto por diversos órgãos, e a
etapa da produção de energia precisa do oxigênio inspirado no sistema
respiratório e transportado no sangue pelo sistema circulatório.
d) Incorreta. Apesar de possuir reservas energéticas e de gordura, a
ação dos sistemas demonstra que eles funcionam em conjunto, mas não são
capazes de manter a sobrevivência saudável de um ser humano sem a
alimentação.
\end{enumerate}

\blankpage