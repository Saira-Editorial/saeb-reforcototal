\chapter{Respostas}
\pagestyle{plain}
\footnotesize

\pagecolor{gray!40}

\colorsec{Módulo 1 – Treino}

	\begin{enumerate}

	\item

	SAEB: Compor ou decompor números racionais positivos (representação
decimal finita) na forma aditiva, ou em suas ordens, ou em adições e
multiplicações.

BNCC: EF06MA01 -- Comparar, ordenar, ler e escrever números naturais e
números racionais cuja representação decimal é finita, fazendo uso da
reta numérica.

Alternativa A: incorreta, pois o aluno pode ter uma mal interpretação e
contar as classes ao invés das ordens.

Alternativa B: incorreta, pois o aluno pode ter uma mal interpretação e
considerar que as ordens são um conjunto de 3 números após o ponto.

Alternativa C: correta, pois são 4 ordens ao total.

Alternativa D: incorreta, pois o aluno pode compreender que ordens são a
soma de todos os números descritos.

	\item

	SAEB: Converter uma representação de um número racional positivo para
outra representação.

BNCC: EF06MA05 -- Classificar números naturais em primos e compostos,
estabelecer relações entre números, expressas pelos termos ``é múltiplo
de'', ``é divisor de'', ``é fator de'', e estabelecer, por meio de
investigações, critérios de divisibilidade por 2, 3, 4, 5, 6, 8, 9, 10,
100 e 1000.

Alternativa A: incorreta, o aluno pode esquecer de somar um ``X''.

Alternativa B: correta, pois essa é a representação em numerais romanos.

Alternativa C: incorreta, o aluno pode compreender que IX é 11 ao invés
de 9.

Alternativa D: incorreta, o aluno pode esquecer de somar a idade do
irmão mais novo.

	\item

	SAEB: Converter uma representação de um número racional positivo para
outra representação.

BNCC: EF06MA01 -- Comparar, ordenar, ler e escrever números naturais e
números racionais cuja representação decimal é finita, fazendo uso da
reta numérica.

Alternativa A: incorreta, pois o aluno pode considerar que a letra ``D''
representa ``Dezena'', logo o valor seria esse.

Alternativa B: correta, pois essa é a representação dos números.

Alternativa C: incorreta, pois o aluno pode confundir e contar um ``I''
a mais e considerar que o valor correto é esse.

Alternativa D: incorreta, pois o aluno pode considerar que a letra D
signifique Duzentos, logo o resultado seria esse.

	\end{enumerate}

\colorsec{Módulo 2 – Treino}

	\begin{enumerate}

		\item SAEB: Resolver problemas que envolvam as ideias de múltiplo, divisor,
máximo divisor comum ou mínimo múltiplo comum.

BNCC: EF06MA06 -- Resolver e elaborar problemas que envolvam as ideias
de múltiplo e de divisor.

Alternativa A: incorreta, pois o aluno pode realizar a somar a soma ao
invés de calcular o m.m.c.

Alternativa B: incorreta, pois o aluno pode considerar que o valor do
m.m.c. em si já é a resposta.

Alternativa C: incorreta, pois o aluno pode confundir m.m.c. com m.d.c.
nos cálculos e chegar a esse resultado.

Alternativa D: correta, pois somando o resultado do m.m.c. com o ano de
1984 obtemos este valor.

		\item SAEB: Resolver problemas que envolvam as ideias de múltiplo, divisor,
máximo divisor comum ou mínimo múltiplo comum.

BNCC: EF06MA07 -- Compreender, comparar e ordenar frações associadas às
ideias de partes de inteiros e resultado de divisão, identificando
frações equivalentes.

Alternativa A: incorreta, pois O aluno pode confundir o resultado do
m.d.c. dos valores como resposta.

Alternativa B: incorreta, pois o aluno pode calcular incorretamente o
m.d.c. esquecendo do valor 5 no final, onde o resultado seria esse.

Alternativa C: incorreta, pois o aluno pode esquecer de contar um número
``2'' no cálculo do m.d.c.

Alternativa D: correta, pois, calculando o m.d.c., obtemos 24,
realizando a operação 1080: 24 obtemos 45.

		\item SAEB: Resolver problemas que envolvam as ideias de múltiplo, divisor,
máximo divisor comum ou mínimo múltiplo comum.

BNCC: EF06MA06 -- Resolver e elaborar problemas que envolvam as ideias
de múltiplo e de divisor.

Alternativa A: incorreta, pois o aluno pode considerar correta essa
alternativa caso ele considere que se encontram no mesmo dia de todo
mês.

Alternativa B: incorreta, pois o aluno pode chegar a esse resultado se
considerar que outubro tenha 30 dias.

Alternativa C: correta, pois realizando o m.m.c., temos 42 dias. 20 dias
depois de 20 de setembro cairá no dia 1 de novembro, lembrando que
outubro tem 31 dias.

Alternativa D: incorreta, pois o aluno pode considerar que setembro
tenha 31 dias.

	\end{enumerate}

\colorsec{Módulo 3 – Treino}

	\begin{enumerate}

		\item SAEB: Representar frações menores ou maiores que a unidade por meio de
representações pictóricas ou associar frações a representações
pictóricas.

BNCC: EF06MA09 -- Resolver e elaborar problemas que envolvam o cálculo
da fração de uma quantidade e cujo resultado seja um número natural, com
e sem uso de calculadora.

Alternativa A: incorreta, pois o aluno provavelmente efetuou a operação
de maneira incorreta.

Alternativa B: correta, pois 2/4 = (2 x 2) / (4 x 2) = 4/8
\textgreater{} 3/8; logo, Maria comeu mais torta.

Alternativa C: incorreta, pois a operação demonstra que Maria e João
comeram porções diferentes.

Alternativa D: incorreta, pois o aluno deve saber comparar frações
diferentes.

		\item SAEB: Identificar frações equivalentes.

BNCC: EF06MA09 -- Resolver e elaborar problemas que envolvam o cálculo
da fração de uma quantidade e cujo resultado seja um número natural, com
e sem uso de calculadora.

Alternativa A: correta, pois:

Para comparar frações elas devem possuir os denominadores iguais. Para
isso, calculamos o MMC entre 5, 4, 3 e 9, que são os denominadores das
frações sorteadas.

\includegraphics[width=5.01042in,height=1.44792in]{./imgSAEB_6_MAT/media/image36.png}

Para encontrar as frações equivalentes, dividimos 180 pelos
denominadores das frações sorteadas e, multiplicamos o resultado pelos
numeradores.

Para 3/5

180 / 5 = 36, como 36 x 3 = 108, a fração equivalente será 108 / 180.

Para 1/4

180/4 = 45, como 45 x 1 = 45, a fração equivalente será 45/180

Para 2/3

180/3 = 60, como 60 x 2 = 120, a fração equivalente será 120/180

Para 5/9

180/9 = 20, como 20 x 5 = 100. A fração equivalente será 100/180

Com as frações equivalentes, basta ordenar pelos numeradores em ordem
crescente e associar com as frações sorteadas. Logo 1/4, 5/9, 3/5, 2/3.

Alternativa B: incorreta, O aluno pode considerar que quanto maior o
denominador, maior o valor fracionário, assim 1/4 seria uma fração maior
que 2/3.

Alternativa C: incorreta, o aluno pode se confundir na forma de calcular
o m.m.c. e colocar erroneamente as frações de forma incorreta.

Alternativa D: incorreta, o aluno pode se confundir e colocar as frações
em forma decrescente ao invés de crescente.

		\item SAEB: Representar frações menores ou maiores que a unidade por meio de
representações pictóricas ou associar frações a representações
pictóricas

BNCC: EF06MA09 -- Resolver e elaborar problemas que envolvam o cálculo
da fração de uma quantidade e cujo resultado seja um número natural, com
e sem uso de calculadora.

Alternativa A: incorreta, pois o aluno pode erroneamente considerar que
ambos os potes de sorvete foram divididos em 3 partes.

Alternativa B: incorreta, pois o aluno erroneamente pode considerar que
somando as partes de chocolates de ambos os potes sem calcular o m.m.c.
pode se tornar uma resposta correta.

Alternativa C: correta, pois: o primeiro pote continha 3 sabores em
iguais quantidades: 1/3 de chocolate, 1/3 de baunilha e 1/3 de morango.
No segundo pote, havia 1/2 de chocolate e 1/2 de baunilha. Considerando
os dois potes de sorvete, dividimos os dois potes em partes iguais.
Fazendo então o m.m.c. de (2,3), obtemos que cada pote foi dividido em 6
partes iguais. Portanto nos dois potes temos 12 partes iguais. Sendo que
destas, 5 partes correspondem ao sabor chocolate.

Alternativa D: incorreta, o aluno pode considerar dividir os potes em 3
partes iguais e somar sem calcular o m.m.c., que chegará a esse
resultado erroneamente.

	\end{enumerate}

\colorsec{Módulo 4 – Treino}

	\begin{enumerate}

		\item SAEB: Resolver problemas que envolvam porcentagens, incluindo os que
lidam com acréscimos e decréscimos simples, aplicação de percentuais
sucessivos e determinação de taxas percentuais.

BNCC: EF06MA13 -- Resolver e elaborar problemas que envolvam
porcentagens, com base na ideia de proporcionalidade, sem fazer uso da
``regra de três'', utilizando estratégias pessoais, cálculo mental e
calculadora, em contextos de educação financeira, entre outros.

Alternativa A: incorreta, pois o aluno pode considerar que aumentar 25\%
signifique aumentar 25 centavos.

Alternativa B: incorreta, pois o aluno pode calcular erroneamente 1,20 x
0,025, chegando a esse resultado equivocado.

Alternativa C: incorreta, poiso aluno pode considerar que 25\% tenha
relação com o valor R\$1,25, pela semelhança.

Alternativa D: correta, pois R\$ 1,20 x 0,25 = 0,3, logo somando R\$1,20
+ R\$0,30 temos 1,50

		\item SAEB: Resolver problemas que envolvam porcentagens, incluindo os que
lidam com acréscimos e decréscimos simples, aplicação de percentuais
sucessivos e determinação de taxas percentuais.

BNCC: EF06MA13 -- Resolver e elaborar problemas que envolvam
porcentagens, com base na ideia de proporcionalidade, sem fazer uso da
``regra de três'', utilizando estratégias pessoais, cálculo mental e
calculadora, em contextos de educação financeira, entre outros.

Alternativa A: incorreta, pois aluno pode deduzir que ao multiplicar o
valor por 0,68 que o valor logo terá 68 \% de desconto.

Alternativa B: incorreta, pois o aluno pode deduzir que, ao multiplicar
o valor por 0,68, os livros terão 6,8 \% devido à semelhança dos termos.

Alternativa C: incorreta, pois o cálculo pode ser feito corretamente mas
a semelhança de 3,2\% para 32\% pode confundir o aluno na hora de
decisão de assinalar a resposta correta.

Alternativa D: correta, pois, ao multiplicar qualquer valor de livro por
68\%, obtém-se um desconto de 32\%.

		\item SAEB: Resolver problemas que envolvam porcentagens, incluindo os que
lidam com acréscimos e decréscimos simples, aplicação de percentuais
sucessivos e determinação de taxas percentuais.

BNCC: EF06MA13 -- Resolver e elaborar problemas que envolvam
porcentagens, com base na ideia de proporcionalidade, sem fazer uso da
``regra de três'', utilizando estratégias pessoais, cálculo mental e
calculadora, em contextos de educação financeira, entre outros.

Alternativa A: incorreta, pois o aluno pode realizar o cálculo
corretamente, mas confundir os valores próximos de R\$1.555, 00 com
R\$1.545, 00 devido à semelhança.

Alternativa B: incorreta, pois o aluno, por meio de dedução, pode
considerar que, somando 3\% ao valor inicial e subtraindo 3\%, o valor
inicial fique inerte.

Alternativa C: correta, pois:

Cálculo do acréscimo

1500 · 0,03 = 45

1.550 + 45 = 1.545

Cálculo do desconto

1.545 · 0,03 = 46,35

1.545 - 46,35 = 1.498,65

Alternativa D: incorreta, pois o aluno, por meio de dedução, pode
considerar que, somando 3\% ao valor inicial e subtraindo 3\%, o valor
inicial fique inerte.
	
	\end{enumerate}

\colorsec{Módulo 5 – Treino}

	\begin{enumerate}

		\item SAEB: Resolver problemas que possam ser representados por sistema de
equações de 1º grau com duas incógnitas.

BNCC: EF06MA14 -- Reconhecer que a relação de igualdade matemática não
se altera ao adicionar, subtrair, multiplicar ou dividir os seus dois
membros por um mesmo número e utilizar essa noção para determinar
valores desconhecidos na resolução de problemas.

Alternativa A: incorreta, pois o aluno pode chegar à conclusão de que o
número de bolas vermelhas é 72, dividindo por 4 ao tentar encontrar o
número de bolas amarelas.

Alternativa B: incorreta, pois o aluno pode considerar que o enunciado
pede o número de bolas amarelas.

Alternativa C: incorreta, pois o aluno pode realizar a operação 360:4,
obtendo um resultado incorreto.

Alternativa D: correta, pois, realizando o sistema, temos que:

X + Y = 360

X = 4y

Inserindo o valor de X na primeira equação, temos que:

4y + y = 360

5y = 360

y = 72

Realizando 360 - 72 = 288, temos o valor correto de bolas vermelhas.

		\item SAEB: Resolver problemas que possam ser representados por sistema de
equações de 1º grau com duas incógnitas.

BNCC: EF06MA14 -- Reconhecer que a relação de igualdade matemática não
se altera ao adicionar, subtrair, multiplicar ou dividir os seus dois
membros por um mesmo número e utilizar essa noção para determinar
valores desconhecidos na resolução de problemas.

Alternativa A: incorreta, pois o aluno durante a resolução pode
confundir e ao invés de dividir 8 por 0,05, realizar a multiplicação.

Alternativa B: incorreta, pois o aluno pode resolver a equação
erroneamente, calculando 0,05 : 8.

Alternativa C: correta, pois, ao substituir t por 8, temos:

8 = 0,05 . x

8/0,05 = x

x = 160

Alternativa D: incorreta, pois o aluno pode erroneamente colocar o valor
8 na incógnita x.

		\item SAEB: Resolver problemas que possam ser representados por sistema de
equações de 1º grau com duas incógnitas.

BNCC: EF06MA14 -- Reconhecer que a relação de igualdade matemática não
se altera ao adicionar, subtrair, multiplicar ou dividir os seus dois
membros por um mesmo número e utilizar essa noção para determinar
valores desconhecidos na resolução de problemas.

Alternativa A: correta, pois, considerando:

x = quantidade do produto A em gramas

y = quantidade do produto B em gramas

x + y = 100 (I)

x·A + y·B = 3,60 (II)

De (I), deduzimos:

y = 100 - x

Que aplicamos em (II):

x·A + (100-x)·B = 3,60

Substituindo A e B pelos seus custos em reais:

x·0,03 + (100-x)y·0,05 = 3,60

Multiplicando toda a equação acima por 100, a fim de tornar inteiros
seus coeficientes:

x·3 + (100-x)·5 = 360

3x + 500 - 5x = 360

-2x = 360 - 500

-2x = -140

x = -140/-2

x = 70 gramas

Alternativa B: incorreta, pois o aluno pode simplesmente retirar a
quantidade de gramas do enunciado e considerar como resposta correta.

Alternativa C: incorreta, pois o aluno pode considerar o preço final do
produto como resposta correta.

Alternativa D: incorreta, pois o aluno pode esquecer de dividir a
equação final por 2, chegando a esse resultado.

	\end{enumerate}

\colorsec{Módulo 6 – Treino}

	\begin{enumerate}

		\item SAEB: Resolver problemas que envolvam variação de proporcionalidade
direta ou inversa entre duas ou mais grandezas, inclusive escalas,
divisões proporcionais e taxa de variação.

Alternativa A: correta, pois, ao realizar a regra de 3 simples, obtemos
o valor de 2 máquinas.

Alternativa B: incorreta, pois o aluno pode esquecer de realizar a
conversão de uma hora para meia hora, chegando a esse resultado
erroneamente.

Alternativa C: incorreta, pois o aluno pode, ao invés de realizar o
cruzamento na regra de três, multiplicar linearmente, chegando a esse
resultado.

Alternativa D: O aluno pode realizar a conversão corretamente, mas errar
o cruzamento no cálculo de regra de três, chegando a esse valor
erroneamente.

		\item SAEB: Resolver problemas que envolvam variação de proporcionalidade
direta ou inversa entre duas ou mais grandezas, inclusive escalas,
divisões proporcionais e taxa de variação.

Alternativa A: incorreta, pois o aluno pode realizar o cruzamento de
dados erroneamente na regra de 3 e chegar a esse valor.

Alternativa B: correta, pois, realizando a regra de 3 composta, obtemos
o valor 35.

Alternativa C: incorreta, pois, ao esquecer que as apostilas não tem
mais 27 folhas e sim 35, o aluno chega a esse resultado erroneamente.

Alternativa D: incorreta, pois, caso o aluno esqueça de ler todo o
enunciado, ele não compreenderá que os minutos de funcionamento da
impressora 2 diminuem, chegando a essa resposta.

		\item SAEB: Resolver problemas que envolvam variação de proporcionalidade
direta ou inversa entre duas ou mais grandezas, inclusive escalas,
divisões proporcionais e taxa de variação.

Alternativa A: incorreta, pois, caso o aluno realize a multiplicação da
regra de 3 sem cruzamentos, chegará a esse valor.

Alternativa B: incorreta, pois, ao calcular o cruzamento da regra de 3
erroneamente, o alunochegará a esse valor.

Alternativa C: incorreta, pois, ao confundir o resultado em horas com
minutos, o aluno acabará assinalando essa alternativa erroneamente.

Alternativa D: correta, pois realizando a regra de três composta,
obtemos esse valor.

	\end{enumerate}

\colorsec{Módulo 7 – Treino}

	\begin{enumerate}

		\item SAEB: Resolver problemas que envolvam relações entre os elementos de uma
circunferência/círculo (raio, diâmetro, corda, arco, ângulo central,
ângulo inscrito).

BNCC: EF06MA18 -- Reconhecer, nomear e comparar polígonos, considerando
lados, vértices e ângulos, e classificá-los em regulares e não
regulares, tanto em suas representações no plano como em faces de
poliedros.

Alternativa A: incorreta, pois o aluno pode esquecer que o valor do raio
é a metade do diâmetro, chegando nesse valor.

Alternativa B: incorreta, pois ao confundir a fórmula do perímetro da
circunferência com a fórmula da área da circunferência chegará a esse
valor.

Alternativa C: incorreta, pois, ao realizar uma soma ao invés de uma
multiplicação na fórmula, obterá esse valor.

Alternativa D: correta, pois ao considerar pi = 3, temos que 2.3.6 =
36cm

		\item SAEB: Resolver problemas que envolvam relações entre os elementos de uma
circunferência/círculo (raio, diâmetro, corda, arco, ângulo central,
ângulo inscrito).

BNCC: EF06MA18 -- Reconhecer, nomear e comparar polígonos, considerando
lados, vértices e ângulos, e classificá-los em regulares e não
regulares, tanto em suas representações no plano como em faces de
poliedros.

Alternativa A: correta, pois, ao calcular a fórmula da área do círculo,
temos que A= 3 . 10² = 300m²

Alternativa B: incorreta, pois, ao realizar o cálculo de perímetro da
circunferência, ao invés do cálculo da área chegaremos a esse valor.

Alternativa C: incorreta, pois o aluno pode esquecer de trocar o
diâmetro pelo raio na fórmula e chegará a esse valor.

Alternativa D: incorreta, pois o aluno pode esquecer de verificar que o
valor do enunciado se trata de m² e não cm².

		\item SAEB: Relacionar o número de vértices, faces ou arestas de prismas ou
pirâmides, em função do seu polígono da base.

BNCC: EF06MA17 -- Quantificar e estabelecer relações entre o número de
vértices, faces e arestas de prismas e pirâmides, em função do seu
polígono da base, para resolver problemas e desenvolver a percepção
espacial.

Alternativa A: correta, pois

V + F = A + 2

8 + 6 = A + 2

14 = A + 2

A = 14 - 2

A = 12 arestas

Alternativa B: incorreta, pois o aluno pode somar todos os números do
poliedro e chegar a essa conclusão equivocada.

Alternativa C: incorreta pois o aluno pode realizar uma subtração ao
invés de utilizar a fórmula.

Alternativa D: incorreta o aluno pode realizar uma multiplicação ao
invés de utilizar a fórmula.

	\end{enumerate}

\colorsec{Módulo 8 – Treino}

	\begin{enumerate}

		\item SAEB: Identificar relações entre ângulos formados por retas paralelas
cortadas por uma transversal.

BNCC: EF06MA19 -- Identificar características dos triângulos e
classificá-los em relação às medidas dos lados e dos ângulos.

Alternativa A: incorreta, pois o aluno pode chegar a essa conclusão
somando os valores literais e dividindo em seguida pelos números
restantes.

Alternativa B: incorreta, pois o aluno pode considerar multiplicar os
valores e dividir após o resultado para chegar a esse valor.

Alternativa c: correta, pois, Utilizando os conhecimentos sobre
bissetriz obtemos que as medidas dos Ângulos BÔP e PÔA são iguais, logo

2x+8=3x-10

2x-3x= -10 - 8

-x = -18

x = 18

Alternativa d: incorreta, pois o aluno, pela falta de conhecimento sobre
bissetriz, pode relembrar que 45 seja o valor da bissetriz do ângulo
reto e chegar a essa conclusão mesmo não se tratando de um ângulo reto.

		\item SAEB: Identificar relações entre ângulos formados por retas paralelas
cortadas por uma transversal.

BNCC: EF06MA19 -- Identificar características dos triângulos e
classificá-los em relação às medidas dos lados e dos ângulos.

Alternativa A: incorreta, pois o aluno errou na soma dos ângulos.

Alternativa B: incorreta, pois o aluno não soube aplicar a soma dos
ângulos internos.

Alternativa C: incorreta, pois o aluno não somou 80º para encontrar o
resultado correto.

Alternativa D: correta, pois a soma dos ângulos internos de um triângulo
é sempre igual a 180 graus. Se um dos ângulos mede 90 graus, a soma dos
outros dois ângulos deve ser igual a 180 - 90 = 90 graus.

		\item SAEB: Identificar relações entre ângulos formados por retas paralelas
cortadas por uma transversal.

BNCC: EF06MA19 -- Identificar características dos triângulos e
classificá-los em relação às medidas dos lados e dos ângulos.

Alternativa A: incorreta, pois o aluno pode considerar cortar mais peças
em relação àquilo que o enunciado recomenda.

Alternativa B: incorreta, pois o aluno pode considerar cortar mais peças
em relação àquilo que o enunciado recomenda.

Alternativa C: correta, pois, para cortar apenas 2 peças de madeira, o
brinquedo deverá ser um triangulo de lado 50 cm. Como será um triangulo
equilátero, terá 3 lados iguais.

Alternativa D: incorreta, pois o aluno pode considerar cortar mais peças
em relação àquilo que o enunciado recomenda.

	\end{enumerate}

\colorsec{Módulo 9 – Treino}

	\begin{enumerate}

		\item SAEB: Descrever ou esboçar deslocamento de pessoas e/ou de objetos em
representações bidimensionais (mapas, croquis etc.), plantas de
ambientes ou vistas, de acordo com condições dadas

BNCC: EF06MA21 -- Construir figuras planas semelhantes em situações de
ampliação e de redução, com o uso de malhas quadriculadas, plano
cartesiano ou tecnologias digitais.

Alternativa A: incorreta, pois o aluno pode esquecer de somar um
quadrinho e chegar a esse número.

Alternativa B: correta: pois, é utilizada a rota 20 + 10 + 30 + 20 + 10
= 90.

Alternativa C: incorreta, pois o aluno pode considerar seguir por uma
rota que não seja a mais vantajosa.

Alternativa D: incorreta, pois o aluno pode considerar seguir por uma
rota que não seja a mais vantajosa.

		\item 
SAEB: Descrever ou esboçar deslocamento de pessoas e/ou de objetos em
representações bidimensionais (mapas, croquis etc.), plantas de
ambientes ou vistas, de acordo com condições dadas

BNCC: EF06MA21 -- Construir figuras planas semelhantes em situações de
ampliação e de redução, com o uso de malhas quadriculadas, plano
cartesiano ou tecnologias digitais.

Alternativa A: correta, pois as coordenadas indicam essa localização.

Alternativa B: incorreta, pois o aluno pode se perder na condução do
mapa e chegar a essa conclusão erroneamente.

Alternativa C: incorreta, pois o aluno pode se perder na condução do
mapa e chegar a essa conclusão erroneamente.

Alternativa D: incorreta, pois o aluno pode se perder na condução do
mapa e chegar a essa conclusão erroneamente.

		\item SAEB: Descrever ou esboçar deslocamento de pessoas e/ou de objetos em
representações bidimensionais (mapas, croquis etc.), plantas de
ambientes ou vistas, de acordo com condições dadas

BNCC: EF06MA21 -- Construir figuras planas semelhantes em situações de
ampliação e de redução, com o uso de malhas quadriculadas, plano
cartesiano ou tecnologias digitais.

Alternativa A: correta, pois essa definição está correta.

Alternativa B: incorreta, pois as primeiras definições estão incorretas.

Alternativa C: incorreta, pois a definição de planta está incorreta.

Alternativa D: incorreta, pois há diferenças entre esses conceitos.

	\end{enumerate}

\colorsec{Módulo 10 – Treino}

	\begin{enumerate}

		\item SAEB: Representar ou associar os dados de uma pesquisa estatística ou de
um levantamento em listas, tabelas (simples ou de dupla entrada) ou
gráficos (barras simples ou agrupadas, colunas simples ou agrupadas,
pictóricos, de linhas, de setores, ou em histograma).

BNCC: EF06MA31 -- Identificar as variáveis e suas frequências e os
elementos constitutivos (título, eixos, legendas, fontes e datas) em
diferentes tipos de gráfico.

Alternativa A: correta, pois 25 \% de 8 é igual a 2.

Alternativa B: incorreta, pois o aluno pode contar vagões a mais que não
estão coloridos e chegar a essa conclusão.

Alternativa C: incorreta, pois o aluno pode considerar realizar a conta
sobre o represente de vagões que não estão coloridos.

Alternativa D: incorreta, pois o aluno pode não compreender o conceito e
frações e porcentagem e chegar a esse valor erroneamente.

		\item SAEB: Interpretar o significado das medidas de tendência central (média
aritmética simples, moda e mediana) ou da amplitude.

BNCC: EF06MA32 -- Interpretar e resolver situações que envolvam dados de
pesquisas sobre contextos ambientais, sustentabilidade, trânsito,
consumo responsável, entre outros, apresentadas pela mídia em tabelas e
em diferentes tipos de gráficos e redigir textos escritos com o objetivo
de sintetizar conclusões.

Alternativa A: incorreta, pois o aluno pode realizar o cálculo da
mediana ao invés da media aritmética e chegar a essa conclusão.

Alternativa b: incorreta, pois o aluno pode conspirar que o tempo esteja
sendo modificado por uma P.A, de constante de mesmo valor subindo e
descendo o tempo.

Alternativa C: correta, pois utilizando a média aritmética 18 + 16 + 14
+ 20 = 68

68/4 = 17

Alternativa D: incorreta, pois o aluno pode somar todos os termos e
esquecer de dividir chegando a esse valor.

		\item SAEB: Interpretar o significado das medidas de tendência central (média
aritmética simples, moda e mediana) ou da amplitude.

BNCC: EF06MA32 -- Interpretar e resolver situações que envolvam dados de
pesquisas sobre contextos ambientais, sustentabilidade, trânsito,
consumo responsável, entre outros, apresentadas pela mídia em tabelas e
em diferentes tipos de gráficos e redigir textos escritos com o objetivo
de sintetizar conclusões.

Alternativa A: incorreta, pois o aluno que não compreender corretamente
o enunciado pode calcular 2 notas a menos e chegar a esse resultado.

Alternativa b: correta, pois o resultado final da média é 48 alunos.

Alternativa c: incorreta, pois o aluno que não compreender corretamente
o enunciado pode calcular 2 notas a mais e chegar a esse valor.

Alternativa d: incorreta, pois o aluno que não compreender corretamente
o enunciado pode calcular 4 notas a mais e chegar a esse valor.
erroneamente

	\end{enumerate}

\colorsec{Módulo 11 – Treino}


	\begin{enumerate}

		\item SAEB: Resolver problemas que envolvam medidas de grandezas (comprimento,
massa, tempo, temperatura, capacidade ou volume) em que haja conversões
entre unidades mais usuais.

BNCC: EF06MA24 -- Resolver e elaborar problemas que envolvam as
grandezas comprimento, massa, tempo, temperatura, área (triângulos e
retângulos), capacidade e volume (sólidos formados por blocos
retangulares), sem uso de fórmulas, inseridos, sempre que possível, em
contextos oriundos de situações reais e/ou relacionadas às outras áreas
do conhecimento.

Alternativa A: correta, pois realizamos o cálculo de 36.000 : 120 = 300

Alternativa B: incorreta, pois o aluno pode converter erroneamente
toneladas para kg e chegar a esse valor.

Alternativa C: incorreta, pois o aluno pode converter erroneamente
toneladas para kg e chegar a esse valor.

Alternativa D: incorreta, pois o aluno pode converter erroneamente
toneladas para kg e chegar a esse valor.

		\item SAEB: Resolver problemas que envolvam medidas de grandezas (comprimento,
massa, tempo, temperatura, capacidade ou volume) em que haja conversões
entre unidades mais usuais.

BNCC: EF06MA24 -- Resolver e elaborar problemas que envolvam as
grandezas comprimento, massa, tempo, temperatura, área (triângulos e
retângulos), capacidade e volume (sólidos formados por blocos
retangulares), sem uso de fórmulas, inseridos, sempre que possível, em
contextos oriundos de situações reais e/ou relacionadas às outras áreas
do conhecimento.

Alternativa A: incorreta, pois o aluno pode realizar a subtração entre
valores e chegar a esse valor.

Alternativa B: incorreta, pois o aluno pode simplesmente dividir o valor
325 por 2 e chegar a essa conclusão.

Alternativa C: correta, pois a operação chega ao valor 35g.

Alternativa D: incorreta, pois o aluno pode simplesmente ler o enunciado
e colocar esse valor como correto.

		\item SAEB: Resolver problemas que envolvam medidas de grandezas (comprimento,
massa, tempo, temperatura, capacidade ou volume) em que haja conversões
entre unidades mais usuais.

BNCC: EF06MA24 -- Resolver e elaborar problemas que envolvam as
grandezas comprimento, massa, tempo, temperatura, área (triângulos e
retângulos), capacidade e volume (sólidos formados por blocos
retangulares), sem uso de fórmulas, inseridos, sempre que possível, em
contextos oriundos de situações reais e/ou relacionadas às outras áreas
do conhecimento.

Alternativa A: correta, pois a operação matemática a partir da fórmula
da área chega ao valor 16cm².

Alternativa B: incorreta, pois o aluno pode considerar que o perímetro e
a área sejam iguais e chegar a essa conclusão.

Alternativa C: incorreta, pois o aluno pode erroneamente dividir o
retângulo em 2, descobrindo essa área incorretamente.

Alternativa D: incorreta, pois o aluno pode considerar a área final do
retângulo ao invés da área de cada quadrado.

	\end{enumerate}

\colorsec{Módulo 12 – Treino}

	\begin{enumerate}

		\item SAEB: Resolver problemas que envolvam a probabilidade de ocorrência de
um resultado em eventos aleatórios equiprováveis independentes ou
dependentes.

BNCC: EF06MA30 -- Calcular a probabilidade de um evento aleatório,
expressando-a por número racional (forma fracionária, decimal e
percentual) e comparar esse número com a probabilidade obtida por meio
de experimentos sucessivos.

Alternativa A: correta, pois 4 x 5 = 20 maneiras.

Alternativa B: incorreta pois, o aluno pode realizar a soma ao invés da
multiplicação.

Alternativa C: incorreta, pois o aluno pode realizar a potenciação ao
invés da multiplicação.

Alternativa D: incorreta, pois o aluno pode realizar a subtração ao
invés da multiplicação.

		\item SAEB: Resolver problemas que envolvam a probabilidade de ocorrência de
um resultado em eventos aleatórios equiprováveis independentes ou
dependentes.

BNCC: EF06MA30 -- Calcular a probabilidade de um evento aleatório,
expressando-a por número racional (forma fracionária, decimal e
percentual) e comparar esse número com a probabilidade obtida por meio
de experimentos sucessivos.

Alternativa A: incorreta, pois o aluno pode realizar a soma ao invés da
multiplicação e chegar a esse resultado.

Alternativa b: incorreta, pois o aluno pode realizar uma potenciação ao
invés da multiplicação,

Alternativa C: correta, pois são possíveis 36 combinações.

Alternativa D: incorreta, pois o aluno pode realizar uma soma ao invés
da multiplicação.

		\item BNCC: EF06MA30 -- Calcular a probabilidade de um evento aleatório,
expressando-a por número racional (forma fracionária, decimal e
percentual) e comparar esse número com a probabilidade obtida por meio
de experimentos sucessivos.

SAEB: Resolver problemas que envolvam a probabilidade de ocorrência de
um resultado em eventos aleatórios equiprováveis independentes ou
dependentes.

Alternativa A: incorreta, pois o aluno pode ter considerado apenas dois
números.

Alternativa B: correta, pois há cinco números pares entre um total de
10.

Alternativa C: incorreta, incorreta, pois o aluno pode ter considerado
seis números.

alternativa D: incorreta, pois incorreta, pois o aluno pode ter
considerado oito números.

