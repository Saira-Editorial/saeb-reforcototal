\chapter{APRESENTAÇÃO}

O \textbf{SISTEMA DE AVALIAÇÃO DA EDUCAÇÃO BÁSICA (SAEB)} É UM CONJUNTO
DE AVALIAÇÕES EXTERNAS EM LARGA ESCALA QUE PERMITE AO INSTITUTO NACIONAL
DE ESTUDOS E PESQUISAS EDUCACIONAIS ANÍSIO TEIXEIRA (INEP) REALIZAR UM
DIAGNÓSTICO DA EDUCAÇÃO BÁSICA BRASILEIRA E DE FATORES QUE PODEM
INTERFERIR NO DESEMPENHO DOS ESTUDANTES.

POR MEIO DE TESTES E QUESTIONÁRIOS, APLICADOS A CADA DOIS ANOS NA REDE
PÚBLICA E EM UMA AMOSTRA DA REDE PRIVADA, O SAEB REFLETE OS NÍVEIS DE
APRENDIZAGEM DEMONSTRADOS PELOS ESTUDANTES AVALIADOS, EXPLICANDO ESSES
RESULTADOS A PARTIR DE UMA SÉRIE DE INFORMAÇÕES CONTEXTUAIS.

O SAEB PERMITE QUE AS ESCOLAS E AS REDES MUNICIPAIS E ESTADUAIS DE
ENSINO AVALIEM A QUALIDADE DA EDUCAÇÃO OFERECIDA AOS ESTUDANTES. O
RESULTADO DA AVALIAÇÃO É UM INDICATIVO DA QUALIDADE DO ENSINO BRASILEIRO
E OFERECE SUBSÍDIOS PARA ELABORAÇÃO, MONITORAMENTO E APRIMORAMENTO DE
POLÍTICAS EDUCACIONAIS, SEMPRE COM BASE EM EVIDÊNCIAS.

AS MÉDIAS DE DESEMPENHO DOS ESTUDANTES, APURADAS NO SAEB, JUNTAMENTE COM
AS TAXAS DE APROVAÇÃO, REPROVAÇÃO E ABANDONO, APURADAS NO CENSO ESCOLAR,
COMPÕEM O ÍNDICE DE DESENVOLVIMENTO DA EDUCAÇÃO BÁSICA (IDEB).

REALIZADO DESDE 1990, O SAEB PASSOU POR UMA SÉRIE DE APRIMORAMENTOS
TEÓRICO-METODOLÓGICOS AO LONGO DAS EDIÇÕES. A EDIÇÃO DE 2019 MARCA O
INÍCIO DE UM PERÍODO DE TRANSIÇÃO ENTRE AS MATRIZES DE REFERÊNCIA
UTILIZADAS DESDE 2001 E AS NOVAS MATRIZES ELABORADAS EM CONFORMIDADE COM
A BASE NACIONAL COMUM CURRICULAR (BNCC).

\pagebreak
\colorsec{REVISA SAEB: REFORÇO ESCOLAR}

COMO O PRÓPRIO NOME DIZ, A EDUCAÇÃO BÁSICA É AQUELA EM QUE SE PROMOVE A
FORMAÇÃO MAIS ESSENCIAL DOS ALUNOS. O SISTEMA DE AVALIAÇÃO DA EDUCAÇÃO
BÁSICA (SAEB) AJUDA NA DETECÇÃO DOS PONTOS FORTES E FRACOS NA FORMAÇÃO
DOS ALUNOS EM ESTADOS, EM MUNICÍPIOS, EM ESCOLAS, FUNCIONANDO COMO UM
PARÂMETRO PARA QUE PROBLEMAS SEJAM SOLUCIONADOS E PARA QUE, ANO APÓS
ANO, ESSA FORMAÇÃO EVOLUA E AJUDE NO CRESCIMENTO DESSAS ESCOLAS, DESSES
MUNICÍPIOS E DESSES ESTADOS.

A PREPARAÇÃO ADEQUADA PARA AVALIAÇÕES EM LARGA ESCALA, COMO A DO SAEB, É
IMPORTANTE PARA QUE, NO MOMENTO DA PROVA, OS ALUNOS POSSAM ESTAR ATENTOS
E TRANQUILOS PARA DAREM O MELHOR POSSÍVEL DE SEU POTENCIAL. ASSIM, UM
MATERIAL DIDÁTICO DE APOIO QUE, DE FATO, PROMOVA ESSA PREPARAÇÃO É O
MAIOR DOS ALIADOS PARA PROFESSORES E GESTORES. ESTE MATERIAL TEM
EXATAMENTE ESTA INTENÇÃO: GARANTIR UM MELHOR APROVEITAMENTO DE CADA
ESTUDANTE NA RESOLUÇÃO DE ATIVIDADES, PARA QUE CONSIGAM RESULTADOS
EXCELENTES NESSA TRAJETÓRIA DE AVALIAÇÃO.

O \textbf{REVISA SAEB} ESTÁ DIVIDIDO EM 18 VOLUMES, DISTRIBUÍDOS, AO
LONGO DO ENSINO FUNDAMENTAL, DA SEGUINTE MANEIRA:

\begin{itemize}
\item
  NOS ANOS INICIAIS, PARA O \textbf{PRIMEIRO}, O \textbf{SEGUNDO}, O
  \textbf{TERCEIRO} E O \textbf{QUARTO ANO}, E NOS ANOS FINAIS, PARA O
  \textbf{SEXTO}, O \textbf{SÉTIMO} E O \textbf{OITAVO ANO}, EXISTE UM
  VOLUME POR ANO DE LÍNGUA PORTUGUESA E EXISTE UM VOLUME POR ANO DE
  MATEMÁTICA.
\item
  O \textbf{QUINTO ANO} APRESENTA UMA ESTRUTURA ESPECIAL. EM UM VOLUME,
  APARECEM ESTES COMPONENTES: LÍNGUA PORTUGUESA, ARTE E CIÊNCIAS
  HUMANAS. EM OUTRO VOLUME, APARECEM ESTES COMPONENTES: MATEMÁTICA,
  EDUCAÇÃO FÍSICA E CIÊNCIAS DA NATUREZA.
\item
  O \textbf{NONO ANO} TAMBÉM APRESENTA UMA ESTRUTURA ESPECIAL. SÃO DOIS
  VOLUMES: UM CONTÉM LÍNGUA PORTUGUESA, ARTE, LÍNGUA INGLESA E CIÊNCIAS
  HUMANAS; O OUTRO CONTÉM MATEMÁTICA, EDUCAÇÃO FÍSICA E CIÊNCIAS DA
  NATUREZA.
\end{itemize}

CADA VOLUME, EM CADA COMPONENTE, ESTÁ DIVIDIDO EM MÓDULOS TEMÁTICOS (DE
UMA OU DUAS AULAS), E CADA MÓDULO CONTA COM A ESTRUTURA DESCRITA A
SEGUIR.

\begin{itemize}
\item
  A \textbf{ABERTURA DO MÓDULO} APRESENTA UM RESUMO TEÓRICO DE
  CONTEXTUALIZAÇÃO, VINCULADO, PRINCIPALMENTE, A HABILIDADES DAS
  MATRIZES DO SAEB, MAS TAMBÉM A ALGUMAS HABILIDADES DA BNCC QUE TÊM
  RELAÇÃO COM ESSAS PRIMEIRAS HABILIDADES.
\item
  NA SEQUÊNCIA, ABRE-SE UMA SEÇÃO DE \textbf{ATIVIDADES}, QUE CONTÉM
  EXERCÍCIOS DE MODELOS VARIADOS PARA RETOMADA E FIXAÇÃO
  DO CONTEÚDO TRAZIDO PELO MÓDULO.
\item
  NO FECHAMENTO DE CADA MÓDULO, SURGE UMA SEÇÃO CHAMADA \textbf{TREINO},
  QUE CONTÉM TRÊS ITENS (QUESTÕES NO MODELO DO INEP, QUE É O MESMO
  UTILIZADO NOS TESTES COGNITIVOS DO SAEB). CADA ITEM CHECA O
  DESENVOLVIMENTO DE UMA HABILIDADE DAS MATRIZES DO SAEB E, SEMPRE QUE
  HÁ CONEXÃO, ESTÁ VINCULADO A UMA HABILIDADE DA BNCC DO ANO
  CORRESPONDENTE.
\item
  OS VOLUMES SE ENCERRAM COM QUATRO \textbf{SIMULADOS}, PARA SEREM
  APLICADOS BIMESTRALMENTE. AO LONGO DOS QUATROS SIMULADOS, TODAS AS
  HABILIDADES DAS MATRIZES DO SAEB, EM CADA ANO, SÃO TRABALHADAS, DESDE
  QUE VINCULADAS A CONTEÚDOS PREVISTOS PELA BNCC PARA O ANO EM
  ESPECÍFICO. IGUALMENTE COMPOSTOS DE ITENS, EM NÚMEROS VARIADOS, OS
  SIMULADOS TAMBÉM APRESENTAM, SEMPRE QUE HÁ CONEXÃO, HABILIDADES DA
  BNCC.
\end{itemize}
