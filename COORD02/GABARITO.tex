\chapter{Respostas}
\pagestyle{plain}
\footnotesize

\pagecolor{gray!40}

\section*{Matemática – Módulo 1 – Treino}

\begin{enumerate}
\item
(A) Incorreta. O número apresentado não representa quantidade ou contagem.
(B) Incorreta. O número apresentado não apresenta necessariamente a ordem de matriculas feitas na escola.
(C) Incorreta. O número não representa nenhuma unidade como metro ou grama.
(D) Correta. Esse número é um código de identificação da aluna Júlia nos registros da escola.
SAEB: Reconhecer o que os números naturais indicam em diferentes
situações: quantidade, ordem, medida ou código de identificação.
BNCC: EF01MA01 -- Utilizar números naturais como indicador de quantidade
ou de ordem em diferentes situações cotidianas e reconhecer situações em
que os números não indicam contagem nem ordem, mas sim código de
identificação.

\item
(A) Correta. Carlos tem a mesma quantidade de bolinhas que alguns amigos, porém todas são amarelas.
(B) Incorreta. Cristiano só tem bolinhas verdes, porém tem menos do que Carlos.
(C) Incorreta. Júlio tem bastante bolinhas azuis, porém em menor quantidade do que as amarelas de Carlos.
(D) Incorreta. Ricardo tem poucas bolinhas de cada cor, apesar de ter o mesmo número total de Carlos.
SAEB: Comparar ou ordenar quantidade de objetos (até 2 ordens).
BNCC: EF01MA05 -- Comparar números naturais de até duas ordens em
situações cotidianas, com e sem suporte da reta numérica.

\item
(A)  Incorreta. O aluno pode ter considerado somente o zero, entendendo que não haveria outras ordens preenchidas.
(B)  Incorreta. O aluno pode ter se esquecido de considerar o zero e o cem.
(C)  Incorreta. O aluno pode ter se esquecido de considerar ou o zero ou o cem.
(D)  Correta. O aluno contou estes números: 0, 10, 20, 30, 40, 50, 60, 70, 80, 90 e 100.
SAEB: Identificar a ordem ocupada por um algarismo OU seu valor
posicional (ou valor relativo) em um número natural de até 3 ordens.
BNCC: EF01MA03 -- Estimar e comparar quantidades de objetos de dois
conjuntos (em torno de 20 elementos), por estimativa e/ou por
correspondência (um a um, dois a dois) para indicar ``tem mais'', ``tem
menos'' ou ``tem a mesma quantidade''.
\end{enumerate}

\section*{Matemática – Módulo 2 – Treino}

\begin{enumerate}
\item
(A)  Incorreta. O aluno pode ter subtraído ao invés de somar.
(B)  Incorreta. O aluno considerou somente as figurinhas de Benício.
(C)  Incorreta. O aluno considerou somente as figurinhas de Bernardo.
(D)  Correta. O aluno somou corretamente os dois valores: 25 + 32 = 57.
SAEB: Calcular o resultado de adições e subtrações, envolvendo
número naturais de até 3 ordens.
BNCC: EF01MA08 -- Resolver e elaborar problemas de adição e de subtração,
envolvendo números de até dois algarismos, com os significados de juntar, acrescentar, separar e retirar, com o suporte de imagens e/ou material manipulável, utilizando estratégias e formas de registro pessoais.

\item
(A) Incorreta. Adicionando-se 16 a 15, compõe-se o número 31, que é maior que o número de cestas da partida (30).
(B) Correta. Adicionando-se 16 a 14, compõe-se a exata quantidade de cestas do jogo, ou seja, 30.
(C) Incorreta. Adicionando-se 15 a 14, compõe-se o número 29, que é menor que o número de cestas da partida (30).
(D) Incorreta. Adicionando-se 15 a 13, compõe-se o número 28, que é menor que o número de cestas da partida (30).
SAEB: Compor ou decompor números naturais de até 3 ordens por
meio de diferentes adições.
BNCC: EF01MA07 -- Compor e decompor número de até duas ordens, por meio
de diferentes adições, com o suporte de material manipulável,
contribuindo para a compreensão de características do sistema de
numeração decimal e o desenvolvimento de estratégias de cálculo.

\item
(A)  Incorreta. O aluno pode ter confundido o valor do lanche com o valor
  que Vinícius tinha.
(B)  Correta. O aluno acrescentou corretamente o valor do dinheiro que Vinícius tinha
  aos valores que ganhou e, então, retirou o valor do sorvete para
  descobrir o valor do lanche: ([(15 + 12 + 10) - 5] = 32).
(C)  Incorreta. O aluno pode ter se esquecido de retirar o valor do sorvete.
(D)  Incorreta. O aluno acrescentou o valor do sorvete ao invés de retirar.
SAEB: Resolver problemas de adição ou de subtração, envolvendo
números naturais de até 3 ordens, com os significados de juntar,
acrescentar, separar ou retirar.
BNCC: EF01MA08 -- Resolver e elaborar problemas de adição e de subtração,
envolvendo números de até dois algarismos, com os significados de
juntar, acrescentar, separar e retirar, com o suporte de imagens e/ou
material manipulável, utilizando estratégias e formas de registro
pessoais.
\end{enumerate}

\section*{Matemática – Módulo 3 – Treino}

\begin{enumerate}
\item
(A) Incorreta. A menina de saia amarela é a mais baixa.
(B) Correta. A menina mascando goma tem a maior altura.
(C) Incorreta. O menino de boné é o terceiro mais alto (ou o segundo mais
baixo).
(D) Incorreta. O menino de tênis azul é o segundo mais alto.
SAEB: Comparar comprimentos, capacidades ou massas ou ordenar
imagens de objetos com base na comparação visual de seus comprimentos,
capacidades ou massas.
BNCC: EF01MA15 -- Comparar comprimentos, capacidades ou massas,
utilizando termos como mais alto, mais baixo, mais comprido, mais curto,
mais grosso, mais fino, mais largo, mais pesado, mais leve, cabe mais,
cabe menos, entre outros, para ordenar objetos de uso cotidiano.

\item
(A) Incorreta. A balança é utilizada para medir massa em quilogramas.
(B) Incorreta. O copo de medição é utilizado para medir volume em litros ou mililitros.
(C) Correta. A régua pode ser usada para medir a distância linear entre o tampo da mesa e o chão.
(D) Incorreta. A panela só pode ser utilizada para medir volume.
SAEB: Reconhecer unidades de medida e/ou instrumentos utilizados
para medir comprimento, tempo, massa ou capacidade.
BNCC: EF01MA15 -- Comparar comprimentos, capacidades ou massas,
utilizando termos como mais alto, mais baixo, mais comprido, mais curto,
mais grosso, mais fino, mais largo, mais pesado, mais leve, cabe mais,
cabe menos, entre outros, para ordenar objetos de uso cotidiano.

\item
(A) Incorreta. O aluno pode ter invertido o sentido de leitura do visor.
(B) Incorreta. O aluno entendeu que o ponteiro estava sobre o número 70.
(C) Correta. O aluno percebeu que o ponteiro está deslocado uma unidade
além do 70.
(D) Incorreta. O aluno deve ter pensado que cada marcação vale 2.
SAEB: Identificar a medida de comprimento, da capacidade ou da
massa de objetos, dada a imagem de um instrumento de medida.
BNCC: EF01MA15 -- Comparar comprimentos, capacidades ou massas,
utilizando termos como mais alto, mais baixo, mais comprido, mais curto,
mais grosso, mais fino, mais largo, mais pesado, mais leve, cabe mais,
cabe menos, entre outros, para ordenar objetos de uso cotidiano.
\end{enumerate}

\section*{Matemática – Módulo 4 – Treino}

\begin{enumerate}
\item
(A) Incorreta. O aluno pode ter se confundido com a primeira segunda-feira.
(B) Incorreta. O aluno pode ter se confundido com o domingo, entendendo que
a segunda-feira é o primeiro dia a aparecer no calendário.
(C) Correta. Seguindo-se a coluna da segunda-feira (a segunda da esquerda para a direita), a última linha dessa coluna é a que contém o número 27.
(D) Incorreta. O aluno pode ter confundido a segunda-feira com o último
dia do mês.
SAEB: Identificar datas, dias da semana ou meses do ano em
calendário ou escrever uma data, apresentando o dia, o mês e o ano.
BNCC: EF01MA17 -- Reconhecer e relacionar períodos do dia, dias da semana
e meses do ano, utilizando calendário, quando necessário.

\item
a) Incorreta. O aluno entendeu que deveria seguir a sequência das letras.
b) Incorreta. O aluno não considerou o amanhecer e foi direto ao dia.
c) Incorreta. O aluno pode ter confundido o entardecer com o amanhecer.
d) Correta. A sequência mostra o amanhecer, seguido do dia claro, seguido
do entardecer e seguido da noite estrelada.
SAEB: Identificar sequência de acontecimentos relativos a um dia.
BNCC: EF01MA16 -- Relatar em linguagem verbal ou não verbal sequência de
acontecimentos relativos a um dia, utilizando, quando possível, os
horários dos eventos.

\item
a) Incorreta. O aluno só considerou os 15 minutos antes de
completar 7:00 horas.
b) Incorreta. O aluno esqueceu-se de considerar os 15 minutos antes de
completar 7:00 horas.
c) Correta. O aluno percebeu que o balão subiu no mesmo minuto em
que desceu, porém na hora posterior; logo o voo durou uma hora.
d) Incorreta. O aluno somou a hora aos 45 minutos.
SAEB: Determinar o horário de início, o horário de término ou a
duração de um acontecimento.
BNCC: EF01MA16 -- Relatar em linguagem verbal ou não verbal sequência de
acontecimentos relativos a um dia, utilizando, quando possível, os
horários dos eventos.
\end{enumerate}

\section*{Matemática – Módulo 5 – Treino}

\begin{enumerate}
\item
a) Incorreta. O aluno confundiu a cédula de 10 reais com uma moeda de 10 centavos.
b) Incorreta. O aluno confundiu a cédula de 10 reais com uma moeda de 1 real.
c) Correta. A cédula representada é a de 10 reais.
d) Incorreta. O aluno confundiu a cédula de 10 reais com outra cédula, a de 100 reais.
SAEB: Relacionar valores de moedas e/ou cédulas do sistema
monetário brasileiro, com base nas imagens desses objetos.
BNCC: EF01MA19 -- Reconhecer e relacionar valores de moedas e cédulas do
sistema monetário brasileiro para resolver situações simples do
cotidiano do estudante.

\item
a) Correta. A composição resulta em 100 reais, valor suficiente para pagar pelo chapéu.
b) Incorreta. Apesar de a composição resultar em 75 reais, o valor do chapéu, não existe cédula de 25 reais.
c) Incorreta. A composição resulta em R\$ 30,00, valor insuficiente para pagar pelo chapéu.
d) Incorreta. A composição resulta em R\$ 15,00, valor insuficiente para pagar pelo chapéu.
SAEB: Resolver problemas que envolvam moedas e/ou cédulas do
sistema monetário brasileiro.
BNCC: EF01MA19 -- Reconhecer e relacionar valores de moedas e cédulas do
sistema monetário brasileiro para resolver situações simples do
cotidiano do estudante.

\item
a) Incorreta. Ana tem R\$ 180,00, valor que não é suficiente para pagar a fatura de R\$ 182,00.
b) Incorreta. Não existe uma moeda de R\$ 2,00 para completar o valor necessário para pagar a fatura.
c) Correta. Ana tem R\$ 180,00 e, com mais uma cédula de R\$ 2,00, completaria o valor para pagar a fatura.
d) Incorreta. Com mais uma cédula de R\$ 5,00, Ana teria R\$ 185,00, valor mais que suficiente para pagar a fatura de R\$ 182,00.
SAEB: Resolver problemas que envolvam moedas e/ou cédulas do
sistema monetário brasileiro.
BNCC: EF01MA19 -- Reconhecer e relacionar valores de moedas e cédulas do
sistema monetário brasileiro para resolver situações simples do
cotidiano do estudante.
\end{enumerate}

\section*{Matemática – Módulo 6 – Treino}

\begin{enumerate}
\item
a) Incorreta. Uma caneta é feita para escrever.
b) Incorreta. Gatos também dormem.
c) Incorreta. Todo navio foi feito para flutuar.
d) Correta. Peixes não podem cantar.
SAEB: Classificar resultados de eventos cotidianos aleatórios como
``pouco prováveis'', ``muito prováveis'', ``certos'' ou ``impossíveis''.
BNCC: EF01MA20 -- Classificar eventos envolvendo o acaso, tais como
``acontecerá com certeza'', ``talvez aconteça'' e ``é impossível
acontecer'', em situações do cotidiano.

\item
a) Incorreta. É muito provável que faça calor no verão.
b) Correta. É pouco provável que faça frio no verão, ainda que seja
possível.
c) Incorreta. É muito provável que neve caia no inverno, principalmente
no hemisfério Norte do planeta.
d) Incorreta. É certeza que as árvores florescem na primavera.
SAEB: Classificar resultados de eventos cotidianos aleatórios como
``pouco prováveis'', ``muito prováveis'', ``certos'' ou ``impossíveis''.
BNCC: EF01MA20 -- Classificar eventos envolvendo o acaso, tais como
``acontecerá com certeza'', ``talvez aconteça'' e ``é impossível
acontecer'', em situações do cotidiano.

\item
a) Incorreta. É sempre possível que chova em alguma lugar.
b) Incorreta. Sempre existe a chance de não chover, apesar do tempo fechado.
c) Incorreta. Como o tempo está bem fechado na cena retratada, é muito provável que chova.
d) Correta. O tempo bem fechado indica que a probabilidade de chuva é bem grande.
SAEB: Classificar
resultados de eventos cotidianos aleatórios como ``pouco prováveis'',
``muito prováveis'', ``certos'' ou ``impossíveis''.
BNCC: EF01MA20 -- Classificar eventos envolvendo o acaso, tais como
``acontecerá com certeza'', ``talvez aconteça'' e ``é impossível
acontecer'', em situações do cotidiano.
\end{enumerate}

\section*{Matemática – Módulo 7 – Treino}

\begin{enumerate}
\item
a) Incorreta. A banana foi a segunda mais escolhida.
b) Incorreta. A maçã foi a segunda mais escolhida.
c) Correta. O morango foi escolhido 5 vezes; logo foi o mais escolhido.
d) Incorreta. A pera foi a fruta menos escolhida.
SAEB: Ler/identificar ou comparar dados estatísticos expressos
em gráficos (barras simples, colunas simples ou pictóricos).
BNCC: EF01MA21 -- Ler dados expressos em tabelas e em gráficos de colunas
simples.

\item
a) Correta. Apenas um aluno da escola ouve música clássica.
b) Incorreta. O funk é pouco ouvido, porém mais do que a música clássica.
c) Incorreta. O rock é tão ouvido quanto o funk.
d) Incorreta. O sertanejo é o ritmo mais ouvido.
SAEB: Ler/identificar ou comparar dados estatísticos ou informações, expressos
em tabelas (simples ou de dupla entrada).
BNCC: EF01MA21 -- Ler dados expressos em tabelas e em gráficos de colunas
simples.

\item
a) Incorreta. O basquete foi o segundo menos escolhido, com 4 alunos.
b) Incorreta. O futebol foi o mais escolhido, com 10 alunos.
c) Incorreta. O judô foi o segundo mais escolhido, com 5 alunos.
d) Incorreta. O tênis de mesa foi o menos escolhido, com 3 alunos.
SAEB:
Ler/identificar ou comparar dados estatísticos expressos em gráficos
(barras simples, colunas simples ou pictóricos).
BNCC: EF01MA21 -- Ler dados expressos em tabelas e em gráficos de colunas
simples.
\end{enumerate}

\section*{Simulado 1}

\begin{enumerate}
\item
a) Incorreta. Não se trata de um número de identificação, porque ele não varia em diferentes pacotes.
b) Correta. O número 1 seguido da unidade quilograma indica a medida de
massa do pacote de açúcar.
c) Incorreta. Ainda que esse número indique uma quantidade de açúcar,
essa quantificação traz uma unidade de medida, e não uma
quantificação exata de determinado objeto.
d) Incorreta. Esse número não expressa a posição do pacote em nenhum
lugar.
SAEB: Reconhecer o que os números naturais indicam em diferentes
situações: quantidade, ordem, medida ou código de identificação.
BNCC: EF01MA01 -- Utilizar números naturais como indicador de quantidade
ou de ordem em diferentes situações cotidianas e reconhecer situações em
que os números não indicam contagem nem ordem, mas sim código de
identificação.

\item
a) Incorreta. A quinta letra é a letra E.
b) Incorreta. A sexta letra é a letra F.
c) Correta. A sétima letra é a letra G.
d) Incorreta. A oitava letra é a letra H.
SAEB: Identificar a posição ordinal de um objeto ou termo em uma
sequência (1º, 2º etc.)
BNCC: EF01MA01 -- Utilizar números naturais como indicador de quantidade
ou de ordem em diferentes situações cotidianas e reconhecer situações em
que os números não indicam contagem nem ordem, mas sim código de
identificação.

\item
a) Incorreta. O aluno pode ter subtraído os números ao invés de
somar.
b) Incorreta. O aluno esqueceu-se de adicionar as unidades.
c) Incorreta. O aluno inverteu a ordem das unidades e das dezenas.
d) Correta. O aluno somou corretamente os termos e encontrou o resultado 110.
SAEB: Calcular o resultado de adições e subtrações, envolvendo
número naturais de até 3 ordens.
BNCC: EF01MA08 -- Resolver e elaborar problemas de adição e de subtração,
envolvendo números de até dois algarismos, com os significados de
juntar, acrescentar, separar e retirar, com o suporte de imagens e/ou
material manipulável, utilizando estratégias e formas de registro
pessoais.

\item
a) Incorreta. Essa adição resulta em 51.
b) Incorreta. Essa adição resulta em 60.
c) Correta. Essa subtração resulta em 50.
d) Incorreta. Essa subtração resulta em 55.
SAEB: Compor ou decompor números naturais de até 3 ordens por
meio de diferentes adições.
BNCC: EF01MA07 -- Compor e decompor número de até duas ordens, por meio
de diferentes adições, com o suporte de material manipulável,
contribuindo para a compreensão de características do sistema de
numeração decimal e o desenvolvimento de estratégias de cálculo.

\item
a) Incorreta. Os lápis têm comprimentos diferentes.
b) Incorreta. O lápis de baixo é o maior lápis.
c) Correta. O lápis de cima é o menor lápis.
d) Incorreta. Os comprimentos podem ser comparados.
SAEB: Comparar comprimentos, capacidades ou massas ou ordenar
imagens de objetos com base na comparação visual de seus comprimentos,
capacidades ou massas.
BNCC: EF01MA15 -- Comparar comprimentos, capacidades ou massas,
utilizando termos como mais alto, mais baixo, mais comprido, mais curto,
mais grosso, mais fino, mais largo, mais pesado, mais leve, cabe mais,
cabe menos, entre outros, para ordenar objetos de uso cotidiano.

\item
a) Correta. O copo graduado usa sua própria capacidade (volume) para
medir o espaço ocupado por líquidos.
b) Incorreta. O comprimento é melhor medido com réguas ou trenas, por exemplo.
c) Incorreta. A massa é melhor medida com balanças.
d) Incorreta. Tamanhos (ou comprimentos) são melhor medidos com instrumentos como a régua.
SAEB: Reconhecer unidades de medida e/ou instrumentos utilizados
para medir comprimento, tempo, massa ou capacidade.
BNCC: EF01MA15 -- Comparar comprimentos, capacidades ou massas,
utilizando termos como mais alto, mais baixo, mais comprido, mais curto,
mais grosso, mais fino, mais largo, mais pesado, mais leve, cabe mais,
cabe menos, entre outros, para ordenar objetos de uso cotidiano.

\item
a) Incorreta. Essa era a última tarefa do dia.
b) Incorreta. O aluno deve ter esquecido alguma das tarefas e, por isso,
selecionou a quarta tarefa de carlos.
c) Incorreta. O aluno escolheu a primeira tarefa.
d) Incorreta. O terceiro horário da rotina é o das 09:30, na sequência programada por carlos.
SAEB: Identificar sequência de acontecimentos relativos a um
dia.
BNCC: EF01MA16 -- Relatar em linguagem verbal ou não verbal sequência de
acontecimentos relativos a um dia, utilizando, quando possível, os
horários dos eventos.

\item
a) Incorreta. O aluno confundiu a terça de carnaval com o início do feriado prolongado.
b) Incorreta. O aluno não percebeu que o carnaval é só um dia, e não todo
o fim de semana.
c) Incorreta. O dia de carnaval mesmo cai na terça-feira.
d) Correta. O aluno percebeu que o dia destacado de azul estava na coluna da terça-feira.
SAEB: Identificar datas, dias da semana ou meses do ano em
calendário ou escrever uma data, apresentando o dia, o mês e o ano.
BNCC: EF01MA17 -- Reconhecer e relacionar períodos do dia, dias da semana
e meses do ano, utilizando calendário, quando necessário.

\item
a) Incorreta. O aluno esqueceu-se de adicionar uma das cédulas de 20
reais.
b) Incorreta. Além de o aluno ter se esquecido de adicionar uma das
cédulas de 20 reais, também confundiu a cédula de 2 reais com
uma cédula de 5 reais.
c) Correta. O aluno adicionou as cédulas corretamente: 20 + 20 + 2 = 42
reais.
d) Incorreta. O aluno confundiu a cédula de 2 reais com uma de 5
reais.
SAEB: Relacionar valores de moedas e/ou cédulas do sistema
monetário brasileiro, com base nas imagens desses objetos.
BNCC: EF01MA19 -- Reconhecer e relacionar valores de moedas e cédulas do
sistema monetário brasileiro para resolver situações simples do
cotidiano do estudante.

\item
a) Correta. Duas cédulas de 5 reais somam o valor exato para pagar pelo produto.
b) Incorreta. Bastaria uma cédula de 10 reais.
c) Incorreta. Bastaria uma cédula de 20 reais, e haveria troco de 10 reais.
d) Incorreta. Bastaria uma cédula de 50 reais, e haveria troco de 40 reais.
SAEB: Resolver problemas que envolvam moedas e/ou cédulas do
sistema monetário brasileiro.
BNCC: EF01MA19 -- Reconhecer e relacionar valores de moedas e cédulas do
sistema monetário brasileiro para resolver situações simples do
cotidiano do estudante.

\item
a) Correta. Somente aranhas podem tecer teias.
b) Incorreta. As gaivotas nadam predominantemente sobra as águas do mar.
c) Incorreta. É uma característica típica de um macaco pular de galho em
galho.
d) Incorreta. Todo rinoceronte adulto é muito pesado.
SAEB: Classificar resultados de eventos cotidianos aleatórios como
``pouco prováveis'', ``muito prováveis'', ``certos'' ou ``impossíveis''.
BNCC: EF01MA20 -- Classificar eventos envolvendo o acaso, tais como
``acontecerá com certeza'', ``talvez aconteça'' e ``é impossível
acontecer'', em situações do cotidiano.

\item
a) Incorreta. Infelizmente, uma pessoa pode passar por um dia sem sentir alegria.
b) Correta. Uma pessoa certamente sentirá fome ao longo de um dia.
c) Incorreta. Felizmente, uma pessoa pode passar por um dia sem sentir dor.
d) Incorreta. Felizmente, uma pessoa pode passar por um dia sem sentir tristeza.
SAEB: Classificar resultados de eventos cotidianos aleatórios como
``pouco prováveis'', ``muito prováveis'', ``certos'' ou ``impossíveis''.
BNCC: EF01MA20 -- Classificar eventos envolvendo o acaso, tais como
``acontecerá com certeza'', ``talvez aconteça'' e ``é impossível
acontecer'', em situações do cotidiano.

\item
a) Incorreta. A bala não é um dos doces favoritos, porque recebeu apenas 13 votos.
b) Correta. O chocolate foi o mais votado, com 34 votos, seguido do sorvete, com 28 votos.
c) Incorreta. O pirulito obteve apenas 17 votos, e o bala apenas 22 votos.
d) Incorreta. O sorvete foi bem votado, mas o pirulito está entre os menos votados.
SAEB: Ler/identificar ou comparar dados estatísticos expressos
em gráficos (barras simples, colunas simples ou pictóricos).
BNCC: EF01MA21 -- Ler dados expressos em tabelas e em gráficos de colunas
simples.

\item
a) Incorreta. Barra do Turvo teve a segunda menor temperatura.
b) Correta. Campos do Jordão teve a menor temperatura da tabela.
c) Incorreta. Ituverava teve a maior temperatura da tabela.
d) Incorreta. Taubaté teve a segunda maior temperatura da tabela.
SAEB: Ler/identificar ou comparar dados estatísticos ou
informações, expressos em tabelas (simples ou de dupla entrada).
BNCC: EF01MA21 -- Ler dados expressos em tabelas e em gráficos de colunas
simples.

\item
a) Incorreta. O amaciante foi o produto mais caro.
b) Correta. O detergente foi o produto mais barato
c) Incorreta. O desinfetante foi o segundo produto mais caro.
d) Incorreta. O sabão em pó foi o terceiro produto mais caro.
SAEB: Ler/identificar ou comparar dados estatísticos ou
informações, expressos em tabelas (simples ou de dupla entrada).
BNCC: EF01MA21 -- Ler dados expressos em tabelas e em gráficos de colunas
simples.
\end{enumerate}

\section*{Simulado 2}

\begin{enumerate}
\item
a) Incorreta. O aluno inverteu a ordem das centenas e das dezenas.
b) Incorreta. O aluno inverteu a ordem das unidades e das centenas.
c) Correta. O aluno identificou corretamente as ordens dos números
por meio do número escrito por extenso.
d) Incorreta. O aluno inverteu a ordem das dezenas com a das centenas.
SAEB: Escrever números naturais de até 3 ordens em sua
representação por algarismos ou em língua materna ou associar o registro
numérico de números naturais de até 3 ordens ao registro em língua
materna.
BNCC: EF01MA01 -- Utilizar números naturais como indicador de quantidade
ou de ordem em diferentes situações cotidianas e reconhecer situações em
que os números não indicam contagem nem ordem, mas sim código de
identificação.

\item
a) Correta. Antônio tem a maior idade; logo nasceu primeiro.
b) Incorreta. Cláudio foi o terceiro a nascer nessa relação de pessoas.
c) Incorreta. Henrique é o mais novo; logo foi o último a nascer entre as pessoas da lista.
d) Incorreta. Marcos foi o segundo a nascer nessa relação de pessoas.
SAEB: Comparar ou ordenar quantidades de objetos (até 2
ordens).
BNCC: EF01MA05 -- Comparar números naturais de até duas ordens em
situações cotidianas, com e sem suporte da reta numérica.

\item
a) Correta. Juntas, as duas coleções somam 20 moedas; então, para que os dois amigos tenham o mesmo número de moedas, cada um deve ter metade da soma, ou seja, 10 moedas. Se José der 6 moedas a João, ambos ficam com 10.
b) Incorreta. Esse é o número de moedas que cada um deve ter.
c) Incorreta. O aluno pode ter pensado em completar a quantidade de
moedas de João para ficar igual à de José.
d) Incorreta. O aluno pode ter adicionado as moedas somente.
SAEB: Resolver problemas de adição ou de subtração, envolvendo
números naturais de até 3 ordens, com os significados de juntar,
acrescentar, separar ou retirar.
BNCC: EF01MA08 -- Resolver e elaborar problemas de adição e de subtração,
envolvendo números de até dois algarismos, com os significados de
juntar, acrescentar, separar e retirar, com o suporte de imagens e/ou
material manipulável, utilizando estratégias e formas de registro
pessoais.

\item
a) Incorreta. A adição resulta em 60.
b) Correta. A adição resulta em 64.
c) Incorreta. A adição resulta em 68.
d) Incorreta. A adição resulta em 72.
SAEB: Calcular o resultado de adições e subtrações, envolvendo
número naturais de até 3 ordens.
BNCC: EF01MA08 -- Resolver e elaborar problemas de adição e de subtração,
envolvendo números de até dois algarismos, com os significados de
juntar, acrescentar, separar e retirar, com o suporte de imagens e/ou
material manipulável, utilizando estratégias e formas de registro
pessoais.

\item
a) Incorreta. O aluno entendeu que a jarra tinha 1 litro de capacidade.
b) Correta. O aluno contou duas marcações faltantes e considerou 1
litro para cada.
c) Incorreta. O aluno contou a quantidade de marcações já preenchidas
por líquido.
d) Incorreta. O aluno considerou a capacidade máxima contando todas as marcações.
SAEB: Estimar/inferir medida de comprimento, capacidade ou massa
de objetos, utilizando unidades de medida convencionais ou não ou medir
comprimento, capacidade ou massa de objetos.
BNCC: EF01MA15 -- Comparar comprimentos, capacidades ou massas,
utilizando termos como mais alto, mais baixo, mais comprido, mais curto,
mais grosso, mais fino, mais largo, mais pesado, mais leve, cabe mais,
cabe menos, entre outros, para ordenar objetos de uso cotidiano.

\item
a) Incorreta. O centímetro é utilizado em réguas, por exemplo.
b) Incorreta. O litro é utilizado em copos medidores, por exemplo.
c) Incorreta. O metro é utilizado em trenas, por exemplo.
d) Correta. A balança mede em quilogramas.
SAEB: Identificar a medida de comprimento, da capacidade ou da
massa de objetos, dada a imagem de um instrumento de medida.
BNCC: EF01MA15 -- Comparar comprimentos, capacidades ou massas,
utilizando termos como mais alto, mais baixo, mais comprido, mais curto,
mais grosso, mais fino, mais largo, mais pesado, mais leve, cabe mais,
cabe menos, entre outros, para ordenar objetos de uso cotidiano.

\item
a) Incorreta. O aluno não percebeu que a volta seria em outro mês.
b) Incorreta. O aluno não percebeu que a volta seria em outro mês, além
de contabilizar o dia 2.
c) Correta. Até dia 31 de março, são 29 dias, além dos 30 dias de abril: obtemos
59 dias.
d) Incorreta. O aluno considerou o mês da sequência, porém contabilizou
o dia 2.
SAEB: Determinar a data de início, a data de término ou a
duração de um acontecimento entre duas datas.
BNCC: EF01MA17 -- Reconhecer e relacionar períodos do dia, dias da semana
e meses do ano, utilizando calendário, quando necessário.

\item
a) Incorreta. O aluno confundiu 18 horas com 6 horas, diferentes nas notações AM e PM.
b) Correta. 18:00 -- 10:30 = 07:30.
c) Incorreta. O aluno se esqueceu de subtrair os 30 minutos que estavam no horário de partida.
d) Incorreta. O aluno contou meia hora a mais.
SAEB: Determinar o horário de início, o horário de término ou a
duração de um acontecimento.
BNCC: EF01MA16 -- Relatar em linguagem verbal ou não verbal sequência de
acontecimentos relativos a um dia, utilizando, quando possível, os
horários dos eventos.

\item
a) Incorreta. O aluno não adicionou os valores das moedas.
b) Incorreta. O aluno adicionou somente duas moedas.
c) Incorreta. O aluno adicionou uma moeda a menos.
d) Incorreta. O aluno adicionou corretamente os valores das moedas:
R\$ 1,00 + R\$ 1,00 + R\$ 1,00 + R\$ 1,00 = R\$ 4,00.
SAEB: Relacionar valores de moedas e/ou cédulas do sistema
monetário brasileiro, com base nas imagens desses objetos.
BNCC: EF01MA19 -- Reconhecer e relacionar valores de moedas e cédulas do
sistema monetário brasileiro para resolver situações simples do
cotidiano do estudante.

\item
a) Incorreta. Juca tem R\$ 50,00, valor insuficiente para comprar um brinquedo de R\$ 65,00.
b) Correta. Juca tem R\$ 50,00, valor suficiente para comprar um brinquedo de R\$ 35,00.
c) Incorreta. Juca tem R\$ 50,00, valor insuficiente para comprar um brinquedo de R\$ 52,00.
d) Incorreta. Juca tem R\$ 50,00, valor insuficiente para comprar um brinquedo de R\$ 85,00.
SAEB: Resolver problemas que envolvam moedas e/ou cédulas do
sistema monetário brasileiro.
BNCC: EF01MA19 -- Reconhecer e relacionar valores de moedas e cédulas do
sistema monetário brasileiro para resolver situações simples do
cotidiano do estudante.

\item
a) Incorreta. É muito provável que a lua apareça no céu à noite.
b) Incorreta. O sol nasce no céu de manhã todos os dias; é um acontecimento certo.
c) Incorreta. É certo que passará um carro por uma rua de uma cidade.
d) Correta. Existe a possibilidade uma lagartixa subir pelas costas de alguém, mas é pouco provável.
SAEB: Classificar resultados de eventos cotidianos aleatórios como
``pouco prováveis'', ``muito prováveis'', ``certos'' ou ``impossíveis''.
BNCC: EF01MA20 -- Classificar eventos envolvendo o acaso, tais como
``acontecerá com certeza'', ``talvez aconteça'' e ``é impossível
acontecer'', em situações do cotidiano.

\item
a) Incorreta. Dinossauros não usariam saias.
b) Correta. Leões podem ser treinados para realizar número de circo.
c) Incorreta. Tubarões não podem viver fora da água.
d) Incorreta. Sereias são seres mitológicos, ou seja, não existem na
realidade.
SAEB: Classificar resultados de eventos cotidianos aleatórios como
``pouco prováveis'', ``muito prováveis'', ``certos'' ou ``impossíveis''.
BNCC: EF01MA20 -- Classificar eventos envolvendo o acaso, tais como
``acontecerá com certeza'', ``talvez aconteça'' e ``é impossível
acontecer'', em situações do cotidiano.

\item
a) Correta. A quantidade de bolas é igual à de bonecos (5 de cada).
b) Incorreta. João tem 5 bolas e 6 carrinhos.
c) Incorreta. João tem 6 carrinhos e 3 peões.
d) Incorreta. João tem 3 peões e 5 bonecos.
SAEB: Ler/identificar ou comparar dados estatísticos expressos
em gráficos (barras simples, colunas simples ou pictóricos).
BNCC: EF01MA21 -- Ler dados expressos em tabelas e em gráficos de colunas
simples.

\item
a) Incorreta. Na segunda-feira, o dia estará ensolarado.
b) Correta. Na terça-feira, o dia estará chuvoso.
c) Incorreta. Na quarta-feira, o dia estará nublado, mas não chuvoso.
d) Incorreta. Na quinta-feira, o dia estará nublado, mas não chuvoso.
SAEB: Ler/identificar ou comparar dados estatísticos expressos
em gráficos (barras simples, colunas simples ou pictóricos).
BNCC: EF01MA21  -- Ler dados expressos em tabelas e em gráficos de colunas
simples.

\item
a) Incorreta. Bianca parece estar com medo.
b) Correta. Eduardo está visivelmente feliz.
c) Incorreta. José parece estar com raiva.
d) Incorreta. Maristela parece estar triste.
SAEB: Ler/identificar ou comparar dados estatísticos expressos
em gráficos (barras simples, colunas simples ou pictóricos).
BNCC: EF01MA21 -- Ler dados expressos em tabelas e em gráficos de colunas
simples.
\end{enumerate}

\section*{Simulado 3}

\begin{enumerate}
\item
a) Incorreta. Em ordem crescente, a Bahia seria o terceiro estado da lista.
b) Correta. Em ordem crescente, o primeiro estado da lista seria o Amazonas.
c) Incorreta. Em ordem crescente, o Ceará seria o segundo estado da lista.
d) Incorreta. Em ordem crescente, Minas Gerais continuaria a ser o último estado da lista.
SAEB: Comparar ou ordenar números naturais de até 3 ordens com
ou sem suporte da reta numérica.
BNCC: EF01MA01 -- Utilizar números naturais como indicador de quantidade
ou de ordem em diferentes situações cotidianas e reconhecer situações em
que os números não indicam contagem nem ordem, mas sim código de
identificação.

\item
a) Incorreta. O número 999 tem ocupada a ordem das centenas.
b) Correta. O aluno identificou que, na ordem das centenas, aparece o
algarismo 9.
c) Incorreta. O número de dezenas é que é pelo menos 90.
d) Incorreta. O número de unidades é que é pelo menos 900.
SAEB: Identificar a ordem ocupada por um algarismo ou seu valor
posicional (ou valor relativo) em um número natural de até 3 ordens.
BNCC: EF01MA01 -- Utilizar números naturais como indicador de quantidade
ou de ordem em diferentes situações cotidianas e reconhecer situações em
que os números não indicam contagem nem ordem, mas sim código de
identificação.

\item
a) Incorreta. Apesar de a adição dos números de gols compor o número 15, o time A foi o vencedor nessa configuração.
b) Correta. Nessa configuração, o time B é o vencedor, e a adição dos números de gols compõe o número 15.
c) Incorreta. A adição dos números de gols compõe o número 16.
d) Incorreta. A adição dos números de gols compõe o número 14.
SAEB: Compor ou decompor números naturais de até 3 ordens por
meio de diferentes adições.
BNCC: EF01MA07 -- Compor e decompor número de até duas ordens, por meio
de diferentes adições, com o suporte de material manipulável,
contribuindo para a compreensão de características do sistema de
numeração decimal e o desenvolvimento de estratégias de cálculo.

\item
a) Incorreta. Essa composição leva a R\$ 45,00.
b) Correta. Essa composição leva a R\$ 35,00.
c) Incorreta. Essa composição leva a R\$ 30,00.
d) Incorreta. Essa composição leva a R\$ 25,00.
SAEB: Compor ou decompor números naturais de até 3 ordens por meio
de diferentes adições.
BNCC: EF01MA07 -- Compor e decompor número de até duas ordens, por meio
de diferentes adições, com o suporte de material manipulável,
contribuindo para a compreensão de características do sistema de
numeração decimal e o desenvolvimento de estratégias de cálculo.

\item
a) Correta. Se visto de cima ou de lado, o carro parece maior, uma vez
que seu comprimento é maior que sua largura.
b) Incorreta. Se visto de frente, vemos a largura, que é menor que o
comprimento.
c) Incorreta. Existe diferença entre as vistas que mostram a largura e o
comprimento.
d) Incorreta. Se visto de trás, vemos a largura, que é menor que o
comprimento.
SAEB: Comparar comprimentos, capacidades ou massas ou ordenar
imagens de objetos com base na comparação visual de seus comprimentos,
capacidades ou massas.
BNCC: EF01MA15 -- Comparar comprimentos, capacidades ou massas,
utilizando termos como mais alto, mais baixo, mais comprido, mais curto,
mais grosso, mais fino, mais largo, mais pesado, mais leve, cabe mais,
cabe menos, entre outros, para ordenar objetos de uso cotidiano.

\item
a) Incorreta. O elefante é o animal mais pesado entre os quatro representados.
b) Incorreta. Não há como saber se a massa do tigre é maior do que a do
jacaré, porém é maior que a do tatu.
c) Incorreta. Da mesma forma, não há como saber se a massa do jacaré é
menor do que a do tigre, mas é menor que a do tatu.
d) Correta. O tatu é, sem dúvidas, o animal de menor massa entre os
quatro representados.
SAEB: Estimar/inferir medida de comprimento, capacidade ou massa
de objetos, utilizando unidades de medida convencionais ou não ou medir
comprimento, capacidade ou massa de objetos.
BNCC: EF01MA15 -- Comparar comprimentos, capacidades ou massas,
utilizando termos como mais alto, mais baixo, mais comprido, mais curto,
mais grosso, mais fino, mais largo, mais pesado, mais leve, cabe mais,
cabe menos, entre outros, para ordenar objetos de uso cotidiano.

\item
a) Incorreta. Até a outra segunda-feira, seriam 7 dias.
b) Incorreta. Até a outra terça-feira, seriam 8 dias.
c) Incorreta. Até a outra quarta-feira, seriam 9 dias.
d) Correta. Até a outra quinta-feira, seriam 10 dias.
SAEB: Identificar datas, dias da semana ou meses do ano em
calendário ou escrever uma data, apresentando o dia, o mês e o ano.
BNCC: EF01MA17 -- Reconhecer e relacionar períodos do dia, dias da semana
e meses do ano, utilizando calendário, quando necessário.

\item
a) Incorreta. O aluno contou somente 1 dia.
b) Incorreta. O aluno contou somente 2 dias.
c) Correta. O aluno contou corretamente 3 dias.
d) Incorreta. O aluno contou 4 dias.
SAEB: Determinar a data de início, a data de término ou a
duração de um acontecimento entre duas datas.
BNCC: EF01MA17 -- Reconhecer e relacionar períodos do dia, dias da semana
e meses do ano, utilizando calendário, quando necessário.

\item
a) Incorreta. O aluno confundiu a moeda de 25 centavos com uma moeda de
10 centavos.
b) Correta. O aluno identificou corretamente a moeda de 25 centavos.
c) Incorreta. O aluno confundiu a moeda de 25 centavos com uma moeda de
50 centavos.
d) Incorreta. O aluno confundiu a moeda de 25 centavos com uma moeda de
1 real.
SAEB: Relacionar valores de moedas e/ou cédulas do sistema
monetário brasileiro, com base nas imagens desses objetos.
BNCC: EF01MA19 -- Reconhecer e relacionar valores de moedas e cédulas do
sistema monetário brasileiro para resolver situações simples do
cotidiano do estudante.

\item
a) Incorreta. A camisa de 110 reais é 30 reais mais cara do que o valor que
roberto pode pagar.
b) Incorreta. A calça de 100 reais é 20 reais mais cara do que o valor que
roberto pode pagar.
c) Incorreta. A calça de 90 reais é 10 reais mais cara do que o valor que
roberto pode pagar.
d) Correta. Roberto tem 20 + 20 + 20 + 20 = 80; logo Roberto tem 20
reais a mais do que o preço da bermuda e pode comprá-la.
SAEB: Resolver problemas que envolvam moedas e/ou cédulas do
sistema monetário brasileiro.
BNCC: EF01MA19 -- Reconhecer e relacionar valores de moedas e cédulas do
sistema monetário brasileiro para resolver situações simples do
cotidiano do estudante.

\item
a) Correta. O Batman é um super herói rico que cria acessórios
tecnológicos possíveis. Não tem superpoderes.
b) Incorreta. O homem aranha tem habilidades possíveis somente às
aranhas.
c) Incorreta. O homem formiga pode encolher-se ao tamanho de um inseto,
Habilidade impossível para um ser humano.
d) Incorreta. O super-homem tem uma série de habilidades como voar e
soltar raios pelos olhos, que são impossíveis para um ser humano.
SAEB: Classificar resultados de eventos cotidianos aleatórios como
``pouco prováveis'', ``muito prováveis'', ``certos'' ou ``impossíveis''.
BNCC: EF01MA20 -- Classificar eventos envolvendo o acaso, tais como
``acontecerá com certeza'', ``talvez aconteça'' e ``é impossível
acontecer'', em situações do cotidiano.

\item
a) Incorreta. O cabelo sempre volta a crescer depois de cortado.
b) Correta. Uma pessoa pode sofrer uma queda sem se machucar.
c) Incorreta. Depois de correr muito, todos nós ficamos suados.
d) Incorreta. A unha sempre volta a crescer depois de cortada.
SAEB: Classificar resultados de eventos cotidianos aleatórios como
``pouco prováveis'', ``muito prováveis'', ``certos'' ou ``impossíveis''.
BNCC: EF01MA20 -- Classificar eventos envolvendo o acaso, tais como
``acontecerá com certeza'', ``talvez aconteça'' e ``é impossível
acontecer'', em situações do cotidiano.

\item
a) Incorreta. A lua pode estar encoberta por nuvens ou estar na fase nova.
b) Incorreta. As estrelas podem ser encobertas por nuvens ou ofuscadas pelas luzes da cidade.
c) Correta. O Sol nunca pode ser vista à noite.
d) Incorreta. Nem sempre chove à noite. A chuva não está ligada ao
momento do dia.
SAEB: Classificar resultados de eventos cotidianos aleatórios como
``pouco prováveis'', ``muito prováveis'', ``certos'' ou ``impossíveis''.
BNCC: EF01MA20 -- Classificar eventos envolvendo o acaso, tais como
``acontecerá com certeza'', ``talvez aconteça'' e ``é impossível
acontecer'', em situações do cotidiano.

\item
a) Incorreta. Correr é a segunda brincadeira mais praticada.
b) Incorreta. Jogar bola é uma das brincadeiras mais praticadas.
c) Incorreta. Jogar \textit{videogame} é uma das brincadeiras mais praticadas.
d) Correta. Pular corda é a brincadeira menos praticada.
SAEB: Ler/identificar ou comparar dados estatísticos expressos
em gráficos (barras simples, colunas simples ou pictóricos).
BNCC: EF01MA21 -- Ler dados expressos em tabelas e em gráficos de colunas
simples.

\item
a) Incorreta. Esse é o número de livros em maior quantidade (álbum de figurinhas).
b) Incorreta. Essa é a soma do número de livros de super-heróis com o número de livros de anime.
c) Incorreta. Essa soma não inclui o caça-palavra.
d) Correta. O aluno adicionou corretamente: (2 + 2 + 1 + 3 = 8).
SAEB: Ler/identificar ou comparar dados estatísticos ou
informações, expressos em tabelas (simples ou de dupla entrada).
BNCC: EF01MA21 -- Ler dados expressos em tabelas e em gráficos de colunas
simples.
\end{enumerate}

\section*{Simulado 4}

\begin{enumerate}
\item
a) Incorreta. O aluno se squeceu de contar um dos itens salgados.
b) Correta. Temos um sanduíche e dois hambúrgueres; logo temos três
itens salgados.
c) Incorreta. O aluno contou um item salgado a mais.
d) Incorreta. O aluno contou dois itens salgados a mais.
SAEB: Comparar ou ordenar quantidades de objetos (até 2 ordens).
BNCC: EF01MA03 -- Estimar e comparar quantidades de objetos de dois
conjuntos (em torno de 20 elementos), por estimativa e/ou por
correspondência (um a um, dois a dois) para indicar ``tem mais'', ``tem
menos'' ou ``tem a mesma quantidade''.

\item
a) Incorreta. Evandro tem o maior tempo; logo não pode ser o primeiro.
b) Incorreta. Além de Evandro ter o maior tempo, Jonas tem o segundo
melhor tempo; logo também não pode ser o último.
c) Incorreta. Jonas não tem o tempo menor; logo não pode ser o
primeiro.
d) correta. Se ordenarmos os tempos de forma crescente, teremos: 25 (Maurício em primeiro),
30 (Jonas em segudo) e 45 (Evandro em terceiro).
SAEB: Identificar a posição ordinal de um objeto ou termo em
uma sequência (1º, 2º etc.).
BNCC: EF01MA01 -- Utilizar números naturais como indicador de quantidade
ou de ordem em diferentes situações cotidianas e reconhecer situações em
que os números não indicam contagem nem ordem, mas sim código de
identificação.

\item
a) Incorreta. Com isso, a caixa \textbf{B} ficará com um kiwi a mais que a caixa \textbf{A}.
b) Incorreta. Com isso, a caixa \textbf{C} ficará sem melancia.
c) Correta. Com isso, haverá dois morangos em cada caixa.
d) Incorreta. Com isso, a caixa \textbf{A} ficará com uma melancia a menos que a
caixa \textbf{B}.
SAEB: Resolver problemas de adição ou de subtração, envolvendo
números naturais de até 3 ordens, com os significados de juntar,
acrescentar, separar ou retirar.
BNCC: EF01MA08 -- Resolver e elaborar problemas de adição e de subtração,
envolvendo números de até dois algarismos, com os significados de
juntar, acrescentar, separar e retirar, com o suporte de imagens e/ou
material manipulável, utilizando estratégias e formas de registro
pessoais.

\item
a) Incorreta. O aluno somou apenas os valores dos dois primeiros itens: 23 + 19 = 42.
b) Incorreta. O aluno somou apenas os valores dos três primeiros itens: 23 + 19 + 8 = 50.
c) Incorreta. O aluno somou apenas os valores dos últimos itens: 8 + 43 = 51.
d) Correta. O aluno somou corretamente os valores de todos os itens da lista: 23 + 19 + 8 + 43 = 93.
SAEB: Calcular o resultado de adições e subtrações, envolvendo
número naturais de até 3 ordens.
BNCC: EF01MA08 -- Resolver e elaborar problemas de adição e de subtração,
envolvendo números de até dois algarismos, com os significados de
juntar, acrescentar, separar e retirar, com o suporte de imagens e/ou
material manipulável, utilizando estratégias e formas de registro
pessoais.

\item
a) Incorreta. A balança é indicada para a medição de massa, e não de volume.
b) Correta. Usamos a capacidade do copo para medir o espaço ocupado pelo
líquido do suco de laranjas.
c) Incorreta. A régua é usada para medir comprimento, e não volume.
d) Incorreta. As mão não são uma unidade de medida padrão, apesar de poderem ser
usadas para medir comprimentos, por exemplo.
SAEB: Reconhecer unidades de medida e/ou instrumentos utilizados
para medir comprimento, tempo, massa ou capacidade.
BNCC: EF01MA15 -- Comparar comprimentos, capacidades ou massas,
utilizando termos como mais alto, mais baixo, mais comprido, mais curto,
mais grosso, mais fino, mais largo, mais pesado, mais leve, cabe mais,
cabe menos, entre outros, para ordenar objetos de uso cotidiano.

\item
a) Incorreta. Essa é a massa de uma sacola de supermercados
cheia, por exemplo.
b) Correta. Essa pode ser a massa estimada de um ônibus.
c) Incorreta. Um carro poderia ter essa massa, porém um ônibus terá sempre
mais do que isso.
d) Incorreta. Essa é a massa de dois sacos de cimento, por exemplo.
SAEB: Estimar/inferir medida de comprimento, capacidade ou massa
de objetos, utilizando unidades de medida convencionais ou não ou medir
comprimento, capacidade ou massa de objetos.
BNCC: EF01MA15 -- Comparar comprimentos, capacidades ou massas,
utilizando termos como mais alto, mais baixo, mais comprido, mais curto,
mais grosso, mais fino, mais largo, mais pesado, mais leve, cabe mais,
cabe menos, entre outros, para ordenar objetos de uso cotidiano.

\item
a) Incorreta. O aluno se deixou influenciar por marcarem 10 horas os
dois relógios, além de se esquecer dos 40 minutos.
b) Incorreta. O aluno se deixou influenciar por marcarem 10 horas os
dois relógios.
c) Incorreta. O aluno acertou as horas, mas esqueceu os 40 minutos.
d) Correta. O aluno acertou as 12 horas de diferença, além dos 40
minutos.
SAEB: Determinar o horário de início, o horário de término ou a
duração de um acontecimento.
BNCC: EF01MA16 -- Relatar em linguagem verbal ou não verbal sequência de
acontecimentos relativos a um dia, utilizando, quando possível, os
horários dos eventos.

\item
a) Incorreta. O aluno confundiu a duração de uma hora, de 60 minutos, com
90 minutos.
b) Incorreta. O aluno retirou 10 minutos da contagem correta.
c) Correta. O tempo de 90 minutos pode ser decomposto em 60 + 30, o que corresponde a 1 hora (60 minutos) e 30 minutos.
d) Incorreta. O aluno adicionou 10 minutos à contagem correta.
SAEB: Determinar o horário de início, o horário de término ou
a duração de um acontecimento.
BNCC: EF01MA16 -- Relatar em linguagem verbal ou não verbal sequência de
acontecimentos relativos a um dia, utilizando, quando possível, os
horários dos eventos.

\item
a) Correta. A arara-vermelha aparece nas cédulas de 10 reais.
b) Incorreta. Apesar de ser um animal do Pantanal, o jacaré não aparece em uma cédula de real.
c) Incorreta. O aluno pode ter confundido com o mico-leão-dourado
presente nas notas de 20 reais.
d) Incorreta. O aluno pode ter confundido com a onça-pintada, presente
nas cédulas de 50 reais.
SAEB: Relacionar valores de moedas e/ou cédulas do sistema
monetário brasileiro, com base nas imagens desses objetos.
BNCC: EF01MA19 -- Reconhecer e relacionar valores de moedas e cédulas do
sistema monetário brasileiro para resolver situações simples do
cotidiano do estudante.

\item
a) Incorreta. Na cédula de 5 reais, aparece a garça.
b) Incorreta. Na cédula de 10 reais, aparece a arara-vermelha.
c) Incorreta. Na cédula de 20 reais, aparece o mico-leão-dourado.
d) Correta. Na cédula de 50 reais, aparece a onça-pintada.
SAEB: Relacionar valores de moedas e/ou cédulas do sistema
monetário brasileiro, com base nas imagens desses objetos.
BNCC: EF01MA19 -- Reconhecer e relacionar valores de moedas e cédulas do
sistema monetário brasileiro para resolver situações simples do
cotidiano do estudante.

\item
a) Incorreta. Baleias vivem nos mares e não voam.
b) Incorreta. Cavalos são animais terrestres que não voam.
c) Correta. Morcegos voam, apesar de não serem aves.
d) Incorreta. Tamanduás são animais terrestres que não voam.
SAEB: Relacionar valores de moedas e/ou cédulas do sistema
monetário brasileiro, com base nas imagens desses objetos.
BNCC: EF01MA19 -- Reconhecer e relacionar valores de moedas e cédulas do
sistema monetário brasileiro para resolver situações simples do
cotidiano do estudante.

\item
a) Incorreta. Cada dado tem 6 números; logo, se Marcos tirar dois 6,
obterá 12.
b) Incorreta. Cada dado tem 6 números; logo, se Marcos tirar dois 5,
obterá 10 -- ou, então, somando um 4 e um 6.
c) Incorreta. Cada dado tem 6 números; logo, se Marcos tirar um 3 e um
2 ou um 4 e um 1, obterá 5.
d) Correta. Com dois dados, o menor número possível é 2 (somando-se o 1 de um dado com 1 do outro dado).
SAEB: Classificar resultados de eventos cotidianos aleatórios como
``pouco prováveis'', ``muito prováveis'', ``certos'' ou ``impossíveis''.
BNCC: EF01MA20 -- Classificar eventos envolvendo o acaso, tais como
``acontecerá com certeza'', ``talvez aconteça'' e ``é impossível
acontecer'', em situações do cotidiano.

\item
a) Correta. Como são 5 jogadas e duas já foram, só há mais
três oportunidades de sair com coroa.
b) Incorreta. Se duas jogadas já foram gastas, existe a possibilidade de que, nas três restantes, saia coroa.
c) Incorreta. Em duas das três jogadas restantes, pode sair coroa.
d) Incorreta. Em pelo menos uma das três jogadas restantes, pode sair coroa.
SAEB: Relacionar valores de moedas e/ou cédulas do sistema
monetário brasileiro, com base nas imagens desses objetos.
BNCC: EF01MA19 -- Reconhecer e relacionar valores de moedas e cédulas do
sistema monetário brasileiro para resolver situações simples do
cotidiano do estudante.

\item
a) Incorreta. O aluno fez a conta usando o número de pessoas que escolheu churrasco e o número de pessoas que escolheu feijoada.
b) Incorreta. O aluno fez a conta usando o número de pessoas que escolheu churrasco e o número de pessoas que escolheu estrogonofe.
c) Correta. Churrasco (10) foi a comida mais votada, enquanto salada (4) foi a menos votada; portanto 10 -- 4 = 6.
d) Incorreta. Erroneamente, o aluno subtraiu duas unidades do número de pessoas que escolheu churrasco.
SAEB: Ler/identificar ou comparar dados estatísticos expressos
em gráficos (barras simples, colunas simples ou pictóricos).
BNCC: EF01MA21 -- Ler dados expressos em tabelas e em gráficos de colunas
simples.

\item
a) Incorreta. Nesse número, falta um item.
b) Correta. São seis linhas na tabela depois do cabeçalho; logo são seis itens diferentes.
c) Incorreta. Esse é o número de folhas de papel colorido.
d) Incorreta. Essa é a somatória de itens no total.
SAEB: Ler/identificar ou comparar dados estatísticos ou
informações, expressos em tabelas (simples ou de dupla entrada).
BNCC: EF01MA21 -- Ler dados expressos em tabelas e em gráficos de colunas
simples.
\end{enumerate}