\chapter{Respostas}
\pagestyle{plain}
\footnotesize

\pagecolor{gray!40}

\colorsec{Língua Portuguesa – Módulo 1 – Treino}

\begin{enumerate}
\item
BNCC: EF89LP04 - Identificar e avaliar teses/opiniões/posicionamentos explícitos e implícitos, argumentos e contra-argumentos em textos argumentativos do campo (carta de leitor, comentário, artigo de opinião, resenha crítica etc.), posicionando-se frente à questão controversa de forma sustentada. a) Incorreta. Não é uma opinião, mas um fato apresentado pelo psiquiatra. b) Incorreta. Nesse trecho, ele apresenta quantas solicitações de atendimento de casos novos ele atendia por ano, o que não se caracteriza como uma opinião. c) Correta. Ao falar sobre a quantidade da procura de pessoas dependentes de jogos, o psiquiatra cita que, por ser uma população com alto risco, esse crescimento no número de procuras o deixava preocupado. d) Incorreta. O psiquiatra apenas conclui sua fala ao dizer que tudo o que foi falado faz parte de sua experiência ao longo de 20 anos de trabalho.

\item
BNCC: EF89LP04 - Identificar e avaliar teses/opiniões/posicionamentos explícitos e implícitos, argumentos e contra-argumentos em textos argumentativos do campo (carta de leitor, comentário, artigo de opinião, resenha crítica etc.), posicionando-se frente à questão controversa de forma sustentada. (A) Incorreta. O desenvolvimento cognitivo e o socioemocional são dois componentes importantes para a formação do indivíduo, segundo o diretor do colégio. (B) Incorreta. Refere-se à opinião do diretor Willmann, o qual diz que a Seeduc não lhes dá condição melhor, mas que valoriza a ideia de a escola se apropriar do que recebe e acreditar naquilo que está fazendo. (C) Incorreta. São as parcerias com a escola que não dão nada de material e oferecem a mesma formação que é dada para outras escolas. (D) Correta. O estudante diz que eles têm uma integração com o professor, o que não acontece em outras escolas. Foi a única escola que realmente mudou a vida dele, lugar onde gosta de estar e que ajudou a melhorar seu relacionamento com sua família.

\item
BNCC: EF89LP04 - Identificar e avaliar teses/opiniões/posicionamentos
explícitos e implícitos, argumentos e contra-argumentos em textos
argumentativos do campo (carta de leitor, comentário, artigo de opinião,
resenha crítica etc.), posicionando-se frente à questão controversa de
forma sustentada. (A) Incorreta. O trecho apresenta dados de pesquisa, sem apresentar opinião do articulista. (B) Incorreta. O trecho é um dado, e não a opinião do articulista. (C) Correta. O artigo de opinião apresenta dados de pesquisas para afirmar que a situação de vida das pessoas depende do acaso, de onde elas nasceram e quem as educou. (D) Incorreta. O trecho é um exemplo que vai auxiliar a tese de que a condição de vida das pessoas depende do acaso.
\end{enumerate}

\colorsec{Língua Portuguesa – Módulo 2 – Treino}

\begin{enumerate}
\item
BNCC: EF69LP02 - Analisar e comparar peças publicitárias variadas (cartazes, folhetos, outdoor, anúncios e propagandas em diferentes mídias, spots, jingle, vídeos etc.), de forma a perceber a articulação entre elas em campanhas, as especificidades das várias semioses e mídias, a adequação dessas peças ao público-alvo, aos objetivos do anunciante e/ou da campanha e à construção composicional e estilo dos gêneros em questão, como forma de ampliar suas possibilidades de compreensão (e produção) de textos pertencentes a esses gêneros. a) Incorreta. Mesmo que a ``atenção'' produza segurança no trânsito, o objetivo dessa campanha não é apenas mostrar essa informação, mas conscientizar a população a tomar essa atitude. b) Incorreta. Mesmo que se estabeleça uma relação entre a ``atenção'' e a ``segurança'', essa relação não é a finalidade do texto, o qual pretende levar as pessoas a adotar posturas mais amplas de segurança no trânsito. c) Correta. A finalidade dessa campanha é conscientizar a população de forma geral a ter mais atenção em suas ações no trânsito, o que o torna mais seguro para todos ``Onde o cidadão é mais atento e respeitoso, o trânsito respnde com menos mortes e mais organização''. d) Incorreta. A finalidade da campanha não é conscientizar a população sobre o trânsito apenas no mês de maio, mas sim em todos os meses.

\item
BNCC: EF69LP20 - Identificar, tendo em vista o contexto de produção, a
forma de organização dos textos normativos e legais, a lógica de
hierarquização de seus itens e subitens e suas partes: parte inicial
(título -- nome e data -- e ementa), blocos de artigos (parte, livro,
capítulo, seção, subseção), artigos (caput e parágrafos e incisos) e
parte final (disposições pertinentes à sua implementação) e analisar
efeitos de sentido causados pelo uso de vocabulário técnico, pelo uso do
imperativo, de palavras e expressões que indicam circunstâncias, como
advérbios e locuções adverbiais, de palavras que indicam generalidade,
como alguns pronomes indefinidos, de forma a poder compreender o caráter
imperativo, coercitivo e generalista das leis e de outras formas de
regulamentação. (A) Incorreta. O texto não mostra como a Associação fará a preservação e os meios para tal, mas as diretrizes dela nos planos para a preservação do meio ambiente. (B) Incorreta. O texto é em relação à Associação e promovido por ela, por isso não se pode afirmar que seja uma apresentação dos planos governamentais. (C) Incorreta. Os planos são da Associação à preservação ambiental, e não do governo para o desenvolvimento da Associação. (D) Correta. A finalidade é apresentar ao leitor os planos da Associação (APREMAVI) na preservação ambiental, o que pode ser visto em ``tem por objetivos'' e nos planos descritos nas alíneas, como em ``Promover, estimular e apoiar ações e trabalhos {[}\ldots{}{]}''.

\item
BNCC: EF69LP27 - Analisar a forma composicional de textos pertencentes a
gêneros normativos/ jurídicos e a gêneros da esfera política, tais como
propostas, programas políticos (posicionamento quanto a diferentes ações
a serem propostas, objetivos, ações previstas etc.), propaganda política
(propostas e sua sustentação, posicionamento quanto a temas em
discussão) e textos reivindicatórios: cartas de reclamação, petição
(proposta, suas justificativas e ações a serem adotadas) e suas marcas
linguísticas, de forma a incrementar a compreensão de textos
pertencentes a esses gêneros e a possibilitar a produção de textos mais
adequados e/ou fundamentados quando isso for requerido. a) Incorreta. O edital não tem o objetivo de promover o espaço, mas oferecer um programa de residência para alguns pesquisadores. b) Incorreta. Não aparece esse tipo de instrução no texto. c) Incorreta. Não há dicas ou informações sobre um modo mais fácil de ser selecionado. d) Correta. O trecho diz respeito a quais serão os projetos contemplados pela instituição, já que mostra que três serão relativos à temática ``3 vezes 22'', outros três sobre pesquisas relativas ao acervo constante, um a respeito do restauro da biblioteca e, finalmente, outro dedicado à pesquisa de todo o material existente.
\end{enumerate}

\colorsec{Língua Portuguesa – Módulo 3 – Treino}

\begin{enumerate}
\item
BNCC: EF69LP47 - Analisar, em textos narrativos ficcionais, as
diferentes formas de composição próprias de cada gênero, os recursos
coesivos que constroem a passagem do tempo e articulam suas partes, a
escolha lexical típica de cada gênero para a caracterização dos cenários
e dos personagens e os efeitos de sentido decorrentes dos tempos
verbais, dos tipos de discurso, dos verbos de enunciação e das
variedades linguísticas (no discurso direto, se houver) empregados,
identificando o enredo e o foco narrativo e percebendo como se estrutura
a narrativa nos diferentes gêneros e os efeitos de sentido decorrentes
do foco narrativo típico de cada gênero, da caracterização dos espaços
físico e psicológico e dos tempos cronológico e psicológico, das
diferentes vozes no texto (do narrador, de personagens em discurso
direto e indireto), do uso de pontuação expressiva, palavras e
expressões conotativas e processos figurativos e do uso de recursos
linguístico-gramaticais próprios a cada gênero narrativo.
 a) Incorreta. Não se trata de um texto longo, além de não viverem, os dois personagens, na cabana. b) Correta. Trata-se de um texto curto, com conflito resumido e simples e poucos personagens. c) Incorreta. Além de o texto não ser longo, Gabriel é o filho, não o narrador-personagem. d) Incorreta. Além de os personagens não serem exatamente amigos, o
eremita aprendeu a valorizar o silêncio antes do outro homem.

\item
BNCC: EF69LP44 - Inferir a presença de valores sociais, culturais e
humanos e de diferentes visões de mundo, em textos literários,
reconhecendo nesses textos formas de estabelecer múltiplos olhares sobre
as identidades, sociedades e culturas e considerando a autoria e o
contexto social e histórico de sua produção.
 a) Incorreta. Pelo contrário, afirma-se que houve aumento nos casos, não diminuição ou arrefecimento. b) Incorreta. Não se qualifica o ECA como texto frágil - o que se afirma é que, simplesmente, ele, juntamente com outros textos legais, não basta para resolver o problema. c) Incorreta. O que se diz no texto é que, muitas vezes, as crianças são consideradas como fardos ou pesos por seus pais. d) Correta. Ao final do texto, afirma-se que ``É hora de agir, de nos unirmos para garantir um futuro melhor para as nossas crianças.''.

\item
BNCC: EF69LP44 - Inferir a presença de valores sociais, culturais e
humanos e de diferentes visões de mundo, em textos literários,
reconhecendo nesses textos formas de estabelecer múltiplos olhares sobre
as identidades, sociedades e culturas e considerando a autoria e o
contexto social e histórico de sua produção. a) Incorreta. O texto não defende explicitamente a importância da profissão dos médicos, mas a importância de haver solidariedade entre os homens. b) Correta. Pode-se identificar que Sylvio Saraiva defendia o altruísmo e a solidariedade entre os homens. c) Incorreta. O texto usa ``lei'' em sentido figurado, argumentando que a solidariedade deve ser tão importante quanto uma lei. d) Incorreta. Não existe a explicitação dessa ideia no discurso de
formatura.
\end{enumerate}

\colorsec{Língua Portuguesa – Módulo 4 – Treino}

\begin{enumerate}
\item
BNCC: EF69LP16 - Analisar e utilizar as formas de composição dos gêneros
jornalísticos da ordem do relatar, tais como notícias (pirâmide
invertida no impresso X blocos noticiosos hipertextuais e
hipermidiáticos no digital, que também pode contar com imagens de vários
tipos, vídeos, gravações de áudio etc.), da ordem do argumentar, tais
como artigos de opinião e editorial (contextualização, defesa de
tese/opinião e uso de argumentos) e das entrevistas: apresentação e
contextualização do entrevistado e do tema, estrutura pergunta e
resposta etc. a) Incorreta. O marcador temporal da quarta-feira corresponde ao início do Campeonato Paulista, não a quando se deu o fato noticiado. b) Incorreta. O assunto principal da notícia é a criação da campanha ``\#ElasNoEstádio''. c) Correta. O ``quem'' da notícia corresponde à Federação Paulista de Futebol, que criou uma campanha de incentivo à presença das mulheres no estádio. d) Incorreta. O ``onde'' da notícia corresponde à sede da entidade, informação que não aparece no texto.

\item
BNCC: EF69LP43 - Identificar e utilizar os modos de introdução de outras
vozes no texto -- citação literal e sua formatação e paráfrase --, as
pistas linguísticas responsáveis por introduzir no texto a posição do
autor e dos outros autores citados (``Segundo X; De acordo com Y; De
minha/nossa parte, penso/amos que''...) e os elementos de normatização
(tais como as regras de inclusão e formatação de citações e paráfrases,
de organização de referências bibliográficas) em textos científicos,
desenvolvendo reflexão sobre o modo como a intertextualidade e a
retextualização ocorrem nesses textos. a) Incorreta. Os ratos não eram superativados, mas sim as proteínas presentes no corpo dos animais. b) Incorreta. Os testes não foram feitos para investigar a força e a massa muscular superativada nos ratos, mas para observar como a ativação de uma proteína consegue evitar a perda de massa muscular. c) Correta. Em ``A força e a massa muscular podem ser conservadas por meio da ativação de uma proteína, o que foi observado por cientistas em testes com ratos'', observa-se que as informações presentes no trecho original foram mantidas, embora tenham sido comunicadas de modo diferente. d) Incorreta. O estudo não é para investigar a força dos cientistas, mas para compreender como a massa muscular reage a uma proteína superativada.

\item
BNCC: EF89LP05 - Analisar o efeito de sentido produzido pelo uso, em
textos, de recurso a formas de apropriação textual (paráfrases,
citações, discurso direto, indireto ou indireto livre). a) Incorreta. A fala do morador relata o que está acontecendo no condomínio, não apresentando uma opinião específica que possa ser diferente da opinião das autoras. b) Incorreta. A fala do morador mostra o que o condomínio dele está fazendo para conter a pandemia. c) Correta. Ao apresentar a fala literal do morador, as autoras da notícia mostram imparcialidade d) Incorreta. A fala literal é objetiva, já que mostra o que foi dito sem a interferência das autoras do texto. 
\end{enumerate}

\colorsec{Língua Portuguesa – Módulo 5 – Treino}

\begin{enumerate}
\item
BNCC: EF67LP04 - Distinguir, em segmentos descontínuos de textos, fato
da opinião enunciada em relação a esse mesmo fato. a) Incorreta. O trecho corresponde a uma informação sobre o enredo do filme; não apresenta, portanto, opiniões do resenhista sobre ele, demonstrando que o aluno considerou um trecho descritivo como uma opinião em vez de um fato. b) Correta. O trecho ``A fotografia traz cenas belas e bem montadas''mostra qual a opinião do resenhista sobre um aspecto técnico do filme, já que exibe um juízo de valor sobre a fotografia, de modo a caracterizar as cenas como ``belas'' e ``bem montadas''c) Incorreta. O trecho não contém opinião, visto que representa apenas o modo como o filme é estruturado. d) Incorreta. O trecho corresponde a uma informação do enredo do filme,
trazendo uma informação técnica da sinopse.

\item
a) Incorreta. O autor expõe sua opinião ao utilizar a palavra
``ceticismo'', dizendo que o mundo estava descrente sobre os
acontecimentos. b) Correta. O autor utiliza como exemplo o fato sobre a epidemia do ebola que aconteceu no continente africano para justificar a descrença da população a respeito da nova pandemia do coronavírus c) Incorreta. No trecho, o autor indica que, quando surgiram os primeiros infectados, o mundo se viu em meio a uma situação incomum. d) Incorreta. Trata-se da opinião do autor a respeito de como as coisas são resolvidas no Brasil.

\item
a) Correta. Somente a participação na enquete popular não iria garantir
o resultado oficial, visto que para esse resultado era necessária a
participação na enquete oficial das mascotes. b) Incorreta. Não há garantias de que quem participou da votação popular também votou na oficial, fato que pode ter ocorrido ou não. c) Incorreta. Não há como saber, pelos textos, que os autores da enquete não tinham conhecimento da homenagem às personalidades. d) Incorreta. Os nomes Oba e Eba também participaram da enquete oficial, apesar de não serem citados no texto II, isso porque, na enquete popular o jornal de esportes colocou as opções que estavam para ser escolhidas oficialmente.
\end{enumerate}

\colorsec{Língua Portuguesa – Módulo 6 – Treino}

\begin{enumerate}
\item
BNCC: EF69LP05 - Inferir e justificar, em textos multissemióticos --
tirinhas, charges, memes, gifs etc. --, o efeito de humor, ironia e/ou
crítica pelo uso ambíguo de palavras, expressões ou imagens ambíguas, de
clichês, de recursos iconográficos, de pontuação etc.
 a) Incorreta. O colorido não gera humor. b) Incorreta. Pratos de vegetais combinam entre si. c) Incorreta. Vegetarianos comem ovos. d) Correta. O humor da charge se encontra no fato de que em apenas um dos dias da semana não se come carne pelo meio ambiente e pelo planeta; nos demais, o impedimento é o preço da carne.

\item
BNCC: EF69LP03 - Identificar, em notícias, o fato central, suas
principais circunstâncias e eventuais decorrências; em reportagens e
fotorreportagens o fato ou a temática retratada e a perspectiva de
abordagem, em entrevistas os principais temas/subtemas abordados,
explicações dadas ou teses defendidas em relação a esses subtemas; em
tirinhas, memes, charge, a crítica, ironia ou humor presente. (A) Incorreta. O tema principal da entrevista gira em torno da evolução das animações. (B) Incorreta. O filme é apenas citado. (C) Incorreta. O festival não é o tema principal. (D) Correta. O tema principal é a evolução da animação no Brasil, dadas
as informações sobre a mudança ocorrida nos últimos 12 anos

\item
BNCC: EF69LP05 - Inferir e justificar, em textos multissemióticos --
tirinhas, charges, memes, gifs etc. --, o efeito de humor, ironia e/ou
crítica pelo uso ambíguo de palavras, expressões ou imagens ambíguas, de
clichês, de recursos iconográficos, de pontuação etc.
 a) Incorreta. Não há tom crítico na charge, mas um tom humorístico. b) Incorreta. A charge não tem caráter atemporal, visto que depende da compreensão dos personagens da Família Addams, que podem não ser atuais, além da situação do coronavírus, que é um acontecimento novo. c) Incorreta. Não há elementos que indiquem que a família recebeu a personagem Mãozinha. d) Correta. Mediante o conhecimento prévio do leitor sobre as personagens da série \emph{A Família Addams}, o efeito de humor ocorre pela situação de o coronavírus ter afetado mais fortemente uma personagem que é apenas uma mão, a qual tem de lidar com as restrições e os procedimentos higiênicos específicos da pandemia relacionada ao coronavírus.
\end{enumerate}

\colorsec{Língua Portuguesa – Módulo 7 – Treino}

\begin{enumerate}
\item
Vale a pena explicar aos alunos, neste ponto, a diferença básica entre os conceitos de "tema" e "assunto". O assunto é algo mais amplo, que geralmente pode ser resumido em uma palavra ou expressão muito curta, como "ecologia" ou "agenda 2030". Já o tema é um recorte do assunto, sendo mais específico. De um único assunto podem surgir vários temas. Do assunto "Agenda 2030", por exemplo, extraiu-se o tema "Perspectivas brasileiras para cumprimento dos objetivos da Agenda 2030".
(a) Incorreta. Os textos não se complementam, mas sim se contrapõem.
(b) Correta. Os textos tratam de um tema comum, o planejamento para o
cumprimento da agenda 2030 da ONU, mediante perspectivas diferentes,
contrapondo-se, tendo em vista que o primeiro afirma que a erradicação
da pobreza e da fome é possível, enquanto o segundo afirma que é
impossível.
(c) Incorreta. Os textos não tratam do assunto pela mesma perspectiva.
(d) Incorreta. Os textos tratam do mesmo tema.

\item
a) Incorreta. A notícia expõe os países que jogarão antes da copa de
2022, além de haver imparcialidade do autor.
b) Incorreta. A notícia não informa sobre possíveis acontecimentos para
a Copa, além de o autor ser imparcial.
c) Incorreta. O autor mostra imparcialidade, não expondo a sua opinião
em relação ao assunto.
d) Correta. A notícia mostra imparcialidade do autor, que não expõe sua
opinião sobre o assunto.

\item
a) Incorreta. O texto não é voltado para um público menos escolarizado.
 b) Incorreta. O texto traz informações precisas a respeito da produção petrolífera dos Estados Unidos e cita que entrou em crise oito dias atrás, um fato histórico. c) Incorreta. O texto foi escrito majoritariamente em terceira pessoa, o que evidencia a intenção do jornalista de ser imparcial e objetivo. d) Correta. As informações são apresentadas de modo impessoal e objetivo.
\end{enumerate}

\colorsec{Língua Portuguesa – Módulo 8 – Treino}

\begin{enumerate}
\item
BNCC: EF89LP31 -- Analisar e utilizar modalização epistêmica, isto é,
modos de indicar uma avaliação sobre o valor de verdade e as condições
de verdade de uma proposição, tais como os asseverativos -- quando se
concorda com (``realmente, evidentemente, naturalmente, efetivamente,
claro, certo, lógico, sem dúvida'' etc.) ou discorda de (``de jeito
nenhum, de forma alguma'') uma ideia; e os quase-asseverativos, que
indicam que se considera o conteúdo como quase certo (``talvez, assim,
possivelmente, provavelmente, eventualmente'').
 a) Incorreta. Inexistem modalizadores nos dois textos, uma vez que no texto 2 a autora contextualiza a Lei Seca e seu objetivo, mas sem emitir opinião ou mostrar posição em relação ao assunto. b) Incorreta. O texto 2 apresenta o contexto da Lei Seca e seu objetivo, mas isso não é um modalizador. c) Correta. Há um modalizador apenas no texto I, o que pode ser visto pelo uso do advérbio ``principalmente'', para evidenciar a importância da Lei Seca. d) Incorreta. Não existem modalizadores em ambos os textos.

\item
BNCC: EF69LP04 -- Identificar e analisar os efeitos de sentido que
 fortalecem a persuasão nos textos publicitários, relacionando as estratégias de persuasão e apelo ao consumo com os recursos linguístico-discursivos utilizados, como imagens, tempo verbal, jogos de palavras, figuras de linguagem etc., com vistas a fomentar práticas de consumo conscientes. a) Incorreta. Essas são recomendações gerais para o dia da doação de sangue. b) Incorreta. A campanha pede aos doadores de sangue que evitem a ingestão de alimentos gordurosos. c) Incorreta. Essa informação trata do grupo de pessoas que não pode doar sangue. d) Correta. Mulheres grávidas ou que estejam amamentando estão no grupo de pessoas que não podem doar sangue.

\item
BNCC: EF89LP16 -- Analisar a modalização realizada em textos noticiosos
e argumentativos, por meio das modalidades apreciativas, viabilizadas
por classes e estruturas gramaticais como adjetivos, locuções adjetivas,
advérbios, locuções adverbiais, orações adjetivas e adverbiais, orações
relativas restritivas e explicativas etc., de maneira a perceber a
apreciação ideológica sobre os fatos noticiados ou as posições
implícitas ou assumidas. a) Incorreta. O trecho não traz exemplos. b) Correta. No trecho ``Isso deve ocorrer por meio da promoção de palestras'', a candidata consegue amarrar a exposição feita anteriormente e, mediante modalizador ``deve'', dá sua opinião e propõe uma intervenção pelo Estado. c) Incorreta. Nesse trecho, não há críticas. d) Incorreta. Ao usar a expressão ``dessa forma'', ela deixou em evidência a tentativa de retomar e reunir os argumentos anteriores, finalizando o texto.
\end{enumerate}

\colorsec{Língua Portuguesa – Módulo 9 – Treino}

\begin{enumerate}
\item
a) Correta. O uso ``só'' mostra o ponto de vista do veículo de
comunicação de que esse tipo de destruição era observado somente após
fenômenos naturais. b) Incorreta. A palavra ``só'' indica uma restrição, não quantidade. c) Incorreta. A comparação é feita pela interpretação da frase, não pela utilização de ``só''. d) Incorreta. O termo ``só'' indica que essa destruição era vista após fenômenos naturais, não que nunca foi vista.

\item
BNCC: EF89LP37 -- Analisar os efeitos de sentido do uso de figuras de
linguagem como ironia, eufemismo, antítese, aliteração, assonância,
dentre outras. a) Correta. A repetição das mesmas palavras e estruturas de frases no começo de versos denomina-se anáfora e, nesse poema, essa figura de linguagem está originando musicalidade. b) Incorreta. A gradação é uma figura de linguagem na qual os sentidos das expressões se intensificam. c) Incorreta. A alegoria é uma figura de linguagem que compara dois elementos. d) Incorreta. A metáfora é uma figura de linguagem que compara dois elementos sem utilizar conjunção.

\item
BNCC: EF89LP37 -- Analisar os efeitos de sentido do uso de figuras de linguagem como ironia, eufemismo, antítese, aliteração, assonância, dentre outras. a) Incorreta. A anáfora caracteriza-se pela repetição de uma mesma expressão em diferentes versos. b) Correta. A frase em destaque compara o riso a um instrumento musical, sem utilizar palavras comparativas, caracterizando-se como uma metáfora c) Incorreta. A comparação difere-se da metáfora por utilizar palavras comparativas; por exemplo, ``como''. d) Incorreta. A gradação consiste em construir uma série na qual o sentido das expressões vai se intensificando até um clímax ou se
reduzindo até um anticlímax.
\end{enumerate}

\colorsec{Língua Portuguesa – Módulo 10 – Treino}

\begin{enumerate}
\item
BNCC: EF09LP11 -- Inferir efeitos de sentido decorrentes do uso de recursos de coesão sequencial (conjunções e articuladores textuais). a) Incorreta. Uma vez que a informação é complementar à outra, não há exclusão de informação. b) Correta. A conjunção ``e'' usada no período tem papel aditivo, ligando duas orações independentes, no chamado período composto por coordenação. c) Incorreta. Não há sentido de explicação no trecho, porque a primeira oração apresenta um sentido completo. d) Incorreta. Embora a conjunção introduza uma oração que complementa a primeira, esta não apresenta função de dar fundamentações à primeira.

\item
BNCC: EF09LP11 -- Inferir efeitos de sentido decorrentes do uso de
recursos de coesão sequencial (conjunções e articuladores textuais). a) Incorreta. A mídia deve ser parceira do governo nessa questão, segundo a autora, não havendo exclusão do meio digital, mas sim de inserção de pensamento crítico na população. b) Incorreta. O texto, nesse momento, não apresenta exemplos dos problemas que a mídia pode causar, mas sim apresenta propostas de solução para a tese. c) Incorreta. O texto não contrasta a mídia com o Ministério da Educação, mas sim propõe uma articulação entre as duas esferas. d) Correta. As expressões apresentam os pronomes ``isso'', ``disso'' e ``dessa'', os quais retomam as ideias dispostas no texto, que estão relacionadas diretamente com a resolução da tese retomada nessa conclusão

\item
BNCC: EF09LP08 - Identificar, em textos lidos e em produções próprias, a relação que conjunções (e locuções conjuntivas) coordenativas e subordinativas estabelecem entre as orações que conectam. a) Incorreta. É uma oração subordinada adverbial consecutiva. b) Incorreta. É uma oração subordinada adverbial condicional. c) Incorreta. É uma oração subordinada adverbial causal. d) Correta. ``Assim que cheguei em casa'' é uma oração subordinada adverbial temporal, assim como a oração destacada no texto.
\end{enumerate}

\colorsec{Língua Portuguesa – Módulo 11 – Treino}

\begin{enumerate}
\item
BNCC: EF69LP55 -- Reconhecer as variedades da língua falada, o conceito de norma-padrão e o de preconceito linguístico. a) Incorreta. A expressão não apresenta traços da língua falada. b) Incorreta. A expressão está em conformidade com a norma-padrão. c) Correta. Verifica-se a variedade falada da língua no verso ``tava tudo bem nublado'', pois ``tava'' é uma forma reduzida de ``estava''. d) Incorreta. A expressão está em conformidade com a norma-padrão.

\item
BNCC: EF69LP56: Fazer uso consciente e reflexivo de regras e normas da norma-padrão em situações de fala e escrita nas quais ela deve ser usada. a) Incorreta. Esse não é a razão de a norma-padrão ser usada na carta. b) Incorreta. O trecho não mostra a chegada dos portugueses, além de que esse não é o motivo de a norma-padrão ser usada. c) Incorreta. Nesse trecho, não há mais dados para afirmar o tempo dos acontecimentos, além de esse não ser o motivo de a norma padrão ser usada. d) Correta. Por ser uma dirigida a uma autoridade, a carta usa a norma-padrão da língua utilizada no período.

\item
BNCC: EF09LP07 - Comparar o uso de regência verbal e regência nominal na norma-padrão com seu uso no português brasileiro coloquial oral. a) Correta. Os verbos que não precisam de complemento são os intransitivos. b) Incorreta. Os verbos transitivos diretos são os que necessitam complemento e se ligam a ele de forma direta. c) Incorreta. Os verbos transitivos indiretos são os que necessitam complemento e se ligam a ele indiretamente. d) Incorreta. O objeto indireto é o termo que completa o sentido de um
verbo transitivo indireto.
\end{enumerate}

\colorsec{Arte – Módulo 1 –  Treino}

\begin{enumerate}
\item
SAEB: Analisar formas, gêneros e estilos distintos de artes visuais e
dança, em diferentes contextos, por meio de seus elementos
constitutivos.
BNCC: EF69AR01 -- Pesquisar, apreciar e analisar formas distintas das artes visuais tradicionais e
contemporâneas, em obras de artistas brasileiros e estrangeiros de diferentes épocas e em
diferentes matrizes estéticas e culturais, de modo a ampliar a experiência com diferentes
contextos e práticas artístico-visuais e cultivar a percepção, o imaginário, a capacidade de
simbolizar e o repertório imagético.
a) Correta. Na instalação, há o diálogo entre diferentes linguagens e a
interação com o público, que percorre o espaço percebendo a obra de
vários ângulos e com os diferentes sentidos.
b) Incorreta. A instalação pode ter um caráter efêmero, só existindo no
período determinado para a exposição, ou pode ser desmontada e recriada
em outro local.
c) Incorreta. Não se relaciona com a arte moderna, pois é uma linguagem
artística que surgiu a partir dos anos 1960, ou seja, é arte
contemporânea.
d. Incorreta. O espectador participa ativamente da obra, e não se
comporta somente como apreciador.

\item
SAEB: Analisar a função do tema como projeto integrador das diferentes
linguagens artísticas.
BNCC: EF69AR03 -- Analisar situações nas quais as linguagens das artes visuais se integram às
linguagens audiovisuais (cinema, animações, vídeos etc.), gráficas (capas de livros, ilustrações
de textos diversos etc.), cenográficas, coreográficas, musicais etc.
a) Incorreta. Apesar de, no cinema, o Impressionismo ser do mesmo período temporal que o Expressionismo,
a estética é diferenciada. No impressionismo, o foco da filmagem é no
exterior, no ângulo subjetivo (olhar do personagem). Esse movimento teve
início na França.
b) Correta. O filme \emph{O gabinete do dr. Caligari} é do movimento
expressionista alemão e possui como características o foco da filmagem
no interior, a morbidez temática e a interpretação teatral.
c) Incorreta. O movimento surrealista do cinema também é do mesmo
período temporal (1920-1930) e teve início na França.
d. Incorreta. O neorrealismo italiano surgiu na década de 1940 e buscava
descrever de forma fiel a vida nos anos de guerra. As filmagens eram
feitas nas ruas.

\item
SAEB: Reconhecer artistas que contribuíram para o desenvolvimento e a
disseminação de diferentes gêneros e estilos nas artes visuais, dança,
música e teatro.
BNCC: EF69AR04 -- Analisar os elementos constitutivos das artes visuais (ponto, linha, forma, direção,
cor, tom, escala, dimensão, espaço, movimento etc.) na apreciação de diferentes produções
artísticas.
a) Incorreta. Leonardo da Vinci é representante da arte renascentista.
Suas obras são regidas pela figuração e pela imitação do mundo.
b) Incorreta. As obras de Frida Kahlo eram regidas pela figuração
estilizada.
c) Correta. Piet Mondrian (1872-1944), pintor holandês, é um dos maiores
representantes do abstracionismo geométrico.
d) Incorreta. Vincent Van Gogh utilizava em suas obras arte figurativa
estilizada.
\end{enumerate}

\colorsec{Arte – Módulo 2 –  Treino}

\begin{enumerate}
\item
SAEB: Analisar formas, gêneros e estilos distintos de música e teatro
em diferentes contextos, por meio de seus elementos constitutivos.
BNCC: EF69AR21 -- Explorar e analisar fontes e materiais sonoros em práticas de composição/criação,
execução e apreciação musical, reconhecendo timbres e características de instrumentos
musicais diversos.
a) Incorreta. Flauta, trompete e trombone são exemplos de aerofones.
b) Incorreta. Violino, guitarra e harpa são exemplos de cordofones.
c) Correta. Triângulos, pratos e sinos produzem som por meio da vibração 
de corpos sólidos, sem estar submetidos a tensão.
d) Incorreta. Surdo, tamborim e pandeiro são exemplos de membranofones.

\item
SAEB: Analisar o papel dos profissionais e a utilização dos
equipamentos culturais no sistema de produção e circulação das artes
visuais, dança, música e teatro.
BNCC: EF69AR09 -- Pesquisar e analisar diferentes formas de expressão, representação e encenação
da dança, reconhecendo e apreciando composições de dança de artistas e grupos brasileiros e
estrangeiros de diferentes épocas.
a) Incorreta. O balé surgiu nas cortes italianas no início do século XVI.
b) Correta. O balé surgiu na Itália, no início do século XVI, no período renascentista.
c) Incorreta. O balé é anterior à Idade Contemporânea.
d) Incorreta. O balé surgiu antes do arte moderna.

\item
SAEB: Analisar formas, gêneros e estilos distintos de música e teatro
em diferentes contextos, por meio de seus elementos constitutivos.
EF69AR24 -- Reconhecer e apreciar artistas e grupos de teatro brasileiros e estrangeiros de
diferentes épocas, investigando os modos de criação, produção, divulgação, circulação e
organização da atuação profissional em teatro.
a)  Incorreta. Não se trata de um gênero teatral.
b)  Incorreta. Não se trata de um movimento teatral.
c)  Incorreta. Cenário realista e recriação detalhada são características
  do teatro tradicional.
d) Correta. Trata-se de uma expressão cunhada pelo crítico húngaro Martin Esslin
  para definir os traços estilísticos e os temas das peças teatrais no
  pós-Segunda Guerra Mundial.
\end{enumerate}

\colorsec{Arte – Módulo 3 –  Treino}

\begin{enumerate}
\item
SAEB: Avaliar produções que inter-relacionam diferentes linguagens
artísticas.
BNCC: EF69AR34 -- Analisar e valorizar o patrimônio cultural, material e imaterial, de culturas
diversas, em especial a brasileira, incluindo suas matrizes indígenas, africanas e europeias,
de diferentes épocas, e favorecendo a construção de vocabulário e repertório relativos às
diferentes linguagens artísticas.
a) Incorreta. O carimbó expressa o canto, o toque dos instrumentos e a
  dança, com influências culturais indígena, africana e europeia.
b) Incorreta. O frevo é uma forma de expressão musical, coreográfica e
  poética, originária do Pernambuco.
c) Incorreta. O samba de roda do recôncavo baiano é uma expressão
  musical, coreográfica, poética e festiva, com matrizes africana e
  europeia.
d) Correta. Todos os elementos da descrição referem-se à roda de capoeira.

\item
SAEB: Avaliar o papel das diversas linguagens artísticas no
questionamento de estereótipos e preconceitos.
BNCC: EF69AR34 -- Analisar e valorizar o patrimônio cultural, material e imaterial, de culturas
diversas, em especial a brasileira, incluindo suas matrizes indígenas, africanas e europeias,
de diferentes épocas, e favorecendo a construção de vocabulário e repertório relativos às
diferentes linguagens artísticas.
a) Incorreta. Escultura é um patrimônio cultural imaterial.
b) Incorreta. Monumentos fazem parte do patrimônio cultural material.
c) Correta. Cultos e religiões fazem parte do patrimônio cultural imaterial.
d) Incorreta. Espaços físicos do município são patrimônios culturais materiais.

\item
SAEB: Avaliar nas linguagens artísticas a diversidade do patrimônio
cultural da humanidade (material e imaterial), em especial o brasileiro,
a partir de suas diferentes matrizes.
BNCC: EF69AR34 -- Analisar e valorizar o patrimônio cultural, material e imaterial, de culturas
diversas, em especial a brasileira, incluindo suas matrizes indígenas, africanas e europeias,
de diferentes épocas, e favorecendo a construção de vocabulário e repertório relativos às
diferentes linguagens artísticas.
a) Incorreta. Fazem parte do patrimônio arqueológico os bens relacionados
  a vestígios da ocupação pré-histórica.
b) Correta. Fazem parte do patrimônio paisagístico tanto as áreas
  naturais quanto lugares criados pelo homem aos quais é atribuído
  valor por sua configuração paisagística e que se destaquem por sua
  relação com o território onde se encontram.
c) Incorreta. Fazem parte do patrimônio etnográfico bens de valor
  etnográfico e de referência para grupos sociais.
d) Incorreta. Fazem parte do patrimônio das artes aplicadas os bens que
  têm seu valor artístico associado à função utilitária, muito comum na
  arquitetura, no \textit{design} e nas artes gráficas.
\end{enumerate}

\colorsec{Inglês – Módulo 1 –  Treino}

\begin{enumerate}
\item
SAEB: Identificar a finalidade de um texto em língua inglesa, com base
em sua estrutura, organização textual, pistas gráficas e/ou aspectos
linguísticos.
BNCC: EF06LI07 -- Formular hipóteses sobre a finalidade de um texto em
língua inglesa, com base em sua estrutura, organização textual e pistas
gráficas.
a) Incorreta. Esse é o intuito de um texto instrucional.
b) Incorreta. Esse é o intuito de uma carta.
c) Incorreta. Esse é o intuito de um texto literário.
d) Correta. Esse é um dos intuitos de uma notícia, como no caso do
texto.

\item
SAEB: Identificar o assunto de um texto, a partir de sua organização, de
palavras cognatas e/ou de palavras formas por afixação.
BNCC: EF06LI06 -- Antecipar o sentido global de textos em língua inglesa
por inferências, com base em leitura rápida, observando títulos,
primeiras e últimas frases de parágrafos e palavras-chave repetidas.
a) Incorreta. O texto não menciona a fauna da região.
b) Incorreta. O texto apenas menciona os ventos.
c) Incorreta. O texto não traz medidas para reverter a situação.
d) Correta. O texto fala sobre os níveis baixos de gelo observados na
Antártida.

\item
SAEB: Localizar informações específicas, a partir de diferentes
objetivos de leitura, em textos em língua inglesa.
BNCC: EF06LI08 -- Identificar o assunto de um texto, reconhecendo sua
organização textual e palavras cognatas.
a) Incorreta. O texto menciona descobertas feitas em Marte.
b) Incorreta. O texto menciona que o lago existiu há muito tempo em
Marte.
c) Correta. Essa informação é mencionada explicitamente.
d) Incorreta. O texto afirma que essa descoberta ocorreu na subida do
monte.
\end{enumerate}

\colorsec{Inglês – Módulo 2 –  Treino}

\begin{enumerate}
\item
SAEB: Identificar os recursos verbais e/ou não verbais que contribuem
para a construção da argumentação em textos em língua inglesa.
BNCC: EF09LI07 -- Identificar argumentos principais e as
evidências/exemplos que os sustentam.
a) Incorreta. O texto apenas menciona a ocorrência de incêndios.
b) Incorreta. O texto afirma que essa coexistência é difícil.
c) Incorreta. O texto apenas afirma que a temperatura tem aumentado.
d) Correta. Essa afirmação pode ser encontrada no trecho ```Recognizing
that climate is an important driver can help us better predict when
they'll occur''.

\item
SAEB: Identificar os recursos verbais e/ou não verbais que contribuem
para a construção da argumentação em textos em língua inglesa.
BNCC: EF09LI07 -- Identificar argumentos principais e as
evidências/exemplos que os sustentam.
a) Incorreta. O texto expõe um argumento oposto.
b) Incorreta. O texto afirma que foram descobertos novos objetos na
órbita.
c) Correta. Essa afirmação aparece no trecho ``because
any objects around Jupiter would be moving at the same rate as the gas
giant''.
d) Incorreta. A dificuldade reside na observação dos objetos que
circundam Júpiter, mas não do planeta em si.

\item
SAEB: Identificar os recursos verbais e/ou não verbais que contribuem
para a construção da argumentação em textos em língua inglesa.
BNCC: EF09LI07 -- Identificar argumentos principais e as
evidências/exemplos que os sustentam.
a) Correta. Essa informação pode ser encontrada no trecho ``it is very important to keep on wearing masks since we the pandemic is not over yet''.
b) Incorreta. De acordo com o texto, a pandemia ainda não terminou.
c) Incorreta. O especialista defende a continuidade do uso.
d) Incorreta. O texto menciona políticos que fazem o contrário.
\end{enumerate}

\colorsec{Inglês – Módulo 3 –  Treino}

\begin{enumerate}
\item
SAEB: Contrapor perspectivas sobre um mesmo assunto em textos em língua
inglesa.
BNCC: EF09LI06 -- Distinguir fatos de opiniões em textos argumentativos
da esfera jornalística.
a) Incorreta. O texto menciona uma grande descoberta de um mosaico.
b) Correta. A falta de espaço constitui o tema da notícia.
c) Incorreta. O texto não menciona o interesse do público.
d) Incorreta. O texto apenas menciona o conselho.

\item
SAEB: Avaliar a qualidade e a validade das informações veiculadas em
textos de língua inglesa, incluindo textos provenientes de ambientes
virtuais.
BNCC: EF09LI06 -- Distinguir fatos de opiniões em textos argumentativos
da esfera jornalística.
a) Incorreta. O texto não faz referência à idade dos achados.
b) Incorreta. O texto afirma exatamente o contrário, destacando a
relevância dos achados.
c) Correta. A especialista destaca o uso de arcos para lançar as
flechas.
d) Incorreta. O texto não afirma isso.

\item
SAEB: Distinguir fatos de opiniões em textos em língua inglesa.
BNCC: EF09LI06 -- Distinguir fatos de opiniões em textos argumentativos
da esfera jornalística.
a) Incorreta. O texto menciona justamente a observação do céu.
b) Correta. Essa aproximação observada durante algumas noites é o tema
do texto.
c) Incorreta. O texto não menciona, especificamente, o brilho dos planetas.
d) Incorreta. O texto menciona que a aproximação pode ser observada
já há algum tempo.
\end{enumerate}

\colorsec{Ciências Humanas – Módulo 1 – Treino}

\begin{enumerate}
\item
BNCC: EF09GE06 -- Associar o critério de divisão do mundo em Ocidente e
Oriente com o Sistema Colonial implantado pelas potências europeias.
a) Incorreta. A porção colonizada pela Europa também
  integra o mundo ocidental. b) Incorreta: O Islamismo é uma religião presente majoritariamente no
  mundo oriental. c) Incorreta. Essa concepção considera apenas a posição dos países no
  globo, mas a concepção de mundo ocidental considera principalmente o
  aspecto cultural. d) Correta. O mundo ocidental é entendido como o conjunto dos países e
  das sociedades que foram condicionados em alguma medida pela sociedade
  europeia ocidental, e isso inclui a própria Europa e todo o conjunto de
  países colonizados pelos europeus, como o continente americano, a
  Austrália e a Nova Zelândia.

\item
BNCC: EF09GE05 -- Analisar fatos e situações para compreender a
integração mundial (econômica, política e cultural), comparando as
diferentes interpretações: globalização e mundialização.
a) Incorreta. A astronomia é uma ciência que se desenvolve desde o Egito
  e a Grécia antiga. b) Incorreta. Os primeiros mapas não dependiam de satélites para sua
  confecção e baseavam-se em dados de navegação, esquemas matemáticos e
  idealizações do mundo. c) Correta. O mapa T-O representava apenas a porção do mundo conhecida
  por seus elaboradores; assim todo o continente americano ficava de
  fora dessa representação, pois apenas o mundo conhecido era
  representado. d) Incorreta. Apesar da existência de lendas sobre a existência de seres
  míticos para além do mundo conhecido, essas lendas só apareceram mais
  fortemente durante as grandes navegações.

\item
BNCC: EF09GE05 -- Analisar fatos e situações para compreender a
integração mundial (econômica, política e cultural), comparando as
diferentes interpretações: globalização e mundialização.
a) Incorreta. Não existe no texto a contraposição à ideia apresentada
  pela alternativa. b) Incorreta. A menção à busca pelo desenvolvimento
  humano e a crítica ao neoliberalismo evidenciam que o Fórum critica a
  falta dos aspectos apresentados na alternativa. c) Incorreta. O fato de se reivindicar maior desenvolvimento humano é uma
  evidência que atesta o não desenvolvimento dos países
  subdesenvolvidos. d) Correta. No texto, critica-se a política neoliberal, ou seja, aquela
  que coloca nos mercados e nas empresas financeiras o centro das políticas
  econômicas, permitindo que o capital financeiro de países estrangeiros
  possa agir sem regulamentação em todo o mundo.
\end{enumerate}

\colorsec{Ciências Humanas – Módulo 2 – Treino}

\begin{enumerate}
\item
BNCC: EF09GE18 -- Identificar e analisar as cadeias industriais e de
inovação e as consequências dos usos de recursos naturais e das
diferentes fontes de energia (tais como termoelétrica, hidrelétrica,
eólica e nuclear) em diferentes países.
a) Incorreta. Se for considerado apenas o aumento da produção de painéis,
haveria um impacto potencialmente negativo em razão da exploração dos
recursos naturais.
b) Incorreta. No texto, cita-se que existem 49.000 painéis
solares domésticos, o que é insuficiente para suprir toda a demanda
energética do país, que tem mais de 200 milhões de habitantes.
c) Correta. A instalação de painéis solares permite o aproveitamento do
potencial energético local, já que são instalados em cima de casas para
poder captar a energia do Sol.
d) Incorreta. Os painéis não geram diminuição do consumo de energia; apenas
se tornam uma fonte a mais de eletricidade disponível.

\item
BNCC: EF09GE18 -- Identificar e analisar as cadeias industriais e de
inovação e as consequências dos usos de recursos naturais e das diferentes fontes de energia (tais
como termoelétrica, hidrelétrica, eólica e nuclear) em diferentes países.
a) Correta. A matriz brasileira apresenta alguns avanços em relação à matriz mundial, mas a tendência nas duas é a mesma.
b) Incorreta. Há algumas melhorias, do ponto de vista da sustentabilidade, da matriz brasileira em relação à mundial.
c) Incorreta. A energia nuclear está menos presente no Brasil do que no mundo.
d) Incorreta. Há fontes renováveis tanto na matriz brasileira quanto na mundial.

\item
BNCC: EF09GE18 -- Identificar e analisar as cadeias industriais e de
inovação e as consequências dos usos de recursos naturais e das diferentes fontes de energia (tais
como termoelétrica, hidrelétrica, eólica e nuclear) em diferentes países.
a) Incorreta. O fato de as emissões humanas estarem interferindo no clima
global mostra que há uma integração da sociedade com o ciclo climático
global.
b) Incorreta. Os gases do efeito estufa não têm capacidade de gerar
resfriamento atmosférico.
c) Incorreta. Apesar de poderem favorecer crescimento florestal, as
emissões de gases do efeito estufa têm seu maior impacto no aquecimento
atmosférico, estando também relacionadas à própria destruição florestal
existente, já que o desmatamento promove a emissão de gás carbônico.
d) Correta. As emissões de gases do efeito estufa se integraram à dinâmica
climática, pois o maior aquecimento atmosférico está alterando os ciclos
dos sistemas atmosféricos e terrestres como um todo.
\end{enumerate}

\colorsec{Ciências Humanas – Módulo 3 – Treino}

\begin{enumerate}
\item

\item

\item
\end{enumerate}

\colorsec{Ciências Humanas – Módulo 4 – Treino}

\begin{enumerate}
\item

\item

\item
\end{enumerate}

\colorsec{Ciências Humanas – Módulo 5 – Treino}

\begin{enumerate}
\item

\item

\item
\end{enumerate}

\colorsec{Ciências Humanas – Módulo 6 – Treino}

\begin{enumerate}
\item

\item

\item
\end{enumerate}

\colorsec{Simulado 1}

\begin{enumerate}
\item
\item
\item
\item
\item
\item
\item
\item
\item
\item
\item
\item
\item
\item
\item
\end{enumerate}

\colorsec{Simulado 2}

\begin{enumerate}
\item
\item
\item
\item
\item
\item
\item
\item
\item
\item
\item
\item
\item
\item
\item
\end{enumerate}

\colorsec{Simulado 3}

\begin{enumerate}
\item
\item
\item
\item
\item
\item
\item
\item
\item
\item
\item
\item
\item
\item
\item
\end{enumerate}

\colorsec{Simulado 4}

\begin{enumerate}
\item
\item
\item
\item
\item
\item
\item
\item
\item
\item
\item
\item
\item
\item
\item
\end{enumerate}