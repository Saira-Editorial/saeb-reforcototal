\chapter{Respostas}
\pagestyle{plain}
\footnotesize

\pagecolor{gray!40}

\colorsec{Língua Portuguesa – Módulo 1 – Treino}

\begin{enumerate}
\item
BNCC: EF89LP04 - Identificar e avaliar teses/opiniões/posicionamentos explícitos e implícitos, argumentos e contra-argumentos em textos argumentativos do campo (carta de leitor, comentário, artigo de opinião, resenha crítica etc.), posicionando-se frente à questão controversa de forma sustentada. a) Incorreta. Não é uma opinião, mas um fato apresentado pelo psiquiatra. b) Incorreta. Nesse trecho, ele apresenta quantas solicitações de atendimento de casos novos ele atendia por ano, o que não se caracteriza como uma opinião. c) Correta. Ao falar sobre a quantidade da procura de pessoas dependentes de jogos, o psiquiatra cita que, por ser uma população com alto risco, esse crescimento no número de procuras o deixava preocupado. d) Incorreta. O psiquiatra apenas conclui sua fala ao dizer que tudo o que foi falado faz parte de sua experiência ao longo de 20 anos de trabalho.

\item
BNCC: EF89LP04 - Identificar e avaliar teses/opiniões/posicionamentos explícitos e implícitos, argumentos e contra-argumentos em textos argumentativos do campo (carta de leitor, comentário, artigo de opinião, resenha crítica etc.), posicionando-se frente à questão controversa de forma sustentada. (A) Incorreta. O desenvolvimento cognitivo e o socioemocional são dois componentes importantes para a formação do indivíduo, segundo o diretor do colégio. (B) Incorreta. Refere-se à opinião do diretor Willmann, o qual diz que a Seeduc não lhes dá condição melhor, mas que valoriza a ideia de a escola se apropriar do que recebe e acreditar naquilo que está fazendo. (C) Incorreta. São as parcerias com a escola que não dão nada de material e oferecem a mesma formação que é dada para outras escolas. (D) Correta. O estudante diz que eles têm uma integração com o professor, o que não acontece em outras escolas. Foi a única escola que realmente mudou a vida dele, lugar onde gosta de estar e que ajudou a melhorar seu relacionamento com sua família.

\item
BNCC: EF89LP04 - Identificar e avaliar teses/opiniões/posicionamentos
explícitos e implícitos, argumentos e contra-argumentos em textos
argumentativos do campo (carta de leitor, comentário, artigo de opinião,
resenha crítica etc.), posicionando-se frente à questão controversa de
forma sustentada. (A) Incorreta. O trecho apresenta dados de pesquisa, sem apresentar opinião do articulista. (B) Incorreta. O trecho é um dado, e não a opinião do articulista. (C) Correta. O artigo de opinião apresenta dados de pesquisas para afirmar que a situação de vida das pessoas depende do acaso, de onde elas nasceram e quem as educou. (D) Incorreta. O trecho é um exemplo que vai auxiliar a tese de que a condição de vida das pessoas depende do acaso.
\end{enumerate}

\colorsec{Língua Portuguesa – Módulo 2 – Treino}

\begin{enumerate}
\item
BNCC: EF69LP02 - Analisar e comparar peças publicitárias variadas (cartazes, folhetos, outdoor, anúncios e propagandas em diferentes mídias, spots, jingle, vídeos etc.), de forma a perceber a articulação entre elas em campanhas, as especificidades das várias semioses e mídias, a adequação dessas peças ao público-alvo, aos objetivos do anunciante e/ou da campanha e à construção composicional e estilo dos gêneros em questão, como forma de ampliar suas possibilidades de compreensão (e produção) de textos pertencentes a esses gêneros. a) Incorreta. Mesmo que a ``atenção'' produza segurança no trânsito, o objetivo dessa campanha não é apenas mostrar essa informação, mas conscientizar a população a tomar essa atitude. b) Incorreta. Mesmo que se estabeleça uma relação entre a ``atenção'' e a ``segurança'', essa relação não é a finalidade do texto, o qual pretende levar as pessoas a adotar posturas mais amplas de segurança no trânsito. c) Correta. A finalidade dessa campanha é conscientizar a população de forma geral a ter mais atenção em suas ações no trânsito, o que o torna mais seguro para todos ``Onde o cidadão é mais atento e respeitoso, o trânsito respnde com menos mortes e mais organização''. d) Incorreta. A finalidade da campanha não é conscientizar a população sobre o trânsito apenas no mês de maio, mas sim em todos os meses.

\item
BNCC: EF69LP20 - Identificar, tendo em vista o contexto de produção, a
forma de organização dos textos normativos e legais, a lógica de
hierarquização de seus itens e subitens e suas partes: parte inicial
(título -- nome e data -- e ementa), blocos de artigos (parte, livro,
capítulo, seção, subseção), artigos (caput e parágrafos e incisos) e
parte final (disposições pertinentes à sua implementação) e analisar
efeitos de sentido causados pelo uso de vocabulário técnico, pelo uso do
imperativo, de palavras e expressões que indicam circunstâncias, como
advérbios e locuções adverbiais, de palavras que indicam generalidade,
como alguns pronomes indefinidos, de forma a poder compreender o caráter
imperativo, coercitivo e generalista das leis e de outras formas de
regulamentação. (A) Incorreta. O texto não mostra como a Associação fará a preservação e os meios para tal, mas as diretrizes dela nos planos para a preservação do meio ambiente. (B) Incorreta. O texto é em relação à Associação e promovido por ela, por isso não se pode afirmar que seja uma apresentação dos planos governamentais. (C) Incorreta. Os planos são da Associação à preservação ambiental, e não do governo para o desenvolvimento da Associação. (D) Correta. A finalidade é apresentar ao leitor os planos da Associação (APREMAVI) na preservação ambiental, o que pode ser visto em ``tem por objetivos'' e nos planos descritos nas alíneas, como em ``Promover, estimular e apoiar ações e trabalhos {[}\ldots{}{]}''.

\item
BNCC: EF69LP27 - Analisar a forma composicional de textos pertencentes a
gêneros normativos/ jurídicos e a gêneros da esfera política, tais como
propostas, programas políticos (posicionamento quanto a diferentes ações
a serem propostas, objetivos, ações previstas etc.), propaganda política
(propostas e sua sustentação, posicionamento quanto a temas em
discussão) e textos reivindicatórios: cartas de reclamação, petição
(proposta, suas justificativas e ações a serem adotadas) e suas marcas
linguísticas, de forma a incrementar a compreensão de textos
pertencentes a esses gêneros e a possibilitar a produção de textos mais
adequados e/ou fundamentados quando isso for requerido. a) Incorreta. O edital não tem o objetivo de promover o espaço, mas oferecer um programa de residência para alguns pesquisadores. b) Incorreta. Não aparece esse tipo de instrução no texto. c) Incorreta. Não há dicas ou informações sobre um modo mais fácil de ser selecionado. d) Correta. O trecho diz respeito a quais serão os projetos contemplados pela instituição, já que mostra que três serão relativos à temática ``3 vezes 22'', outros três sobre pesquisas relativas ao acervo constante, um a respeito do restauro da biblioteca e, finalmente, outro dedicado à pesquisa de todo o material existente.
\end{enumerate}

\colorsec{Língua Portuguesa – Módulo 3 – Treino}

\begin{enumerate}
\item
BNCC: EF69LP47 - Analisar, em textos narrativos ficcionais, as
diferentes formas de composição próprias de cada gênero, os recursos
coesivos que constroem a passagem do tempo e articulam suas partes, a
escolha lexical típica de cada gênero para a caracterização dos cenários
e dos personagens e os efeitos de sentido decorrentes dos tempos
verbais, dos tipos de discurso, dos verbos de enunciação e das
variedades linguísticas (no discurso direto, se houver) empregados,
identificando o enredo e o foco narrativo e percebendo como se estrutura
a narrativa nos diferentes gêneros e os efeitos de sentido decorrentes
do foco narrativo típico de cada gênero, da caracterização dos espaços
físico e psicológico e dos tempos cronológico e psicológico, das
diferentes vozes no texto (do narrador, de personagens em discurso
direto e indireto), do uso de pontuação expressiva, palavras e
expressões conotativas e processos figurativos e do uso de recursos
linguístico-gramaticais próprios a cada gênero narrativo.
 a) Incorreta. Não se trata de um texto longo, além de não viverem, os dois personagens, na cabana. b) Correta. Trata-se de um texto curto, com conflito resumido e simples e poucos personagens. c) Incorreta. Além de o texto não ser longo, Gabriel é o filho, não o narrador-personagem. d) Incorreta. Além de os personagens não serem exatamente amigos, o
eremita aprendeu a valorizar o silêncio antes do outro homem.

\item
BNCC: EF69LP44 - Inferir a presença de valores sociais, culturais e
humanos e de diferentes visões de mundo, em textos literários,
reconhecendo nesses textos formas de estabelecer múltiplos olhares sobre
as identidades, sociedades e culturas e considerando a autoria e o
contexto social e histórico de sua produção.
 a) Incorreta. Pelo contrário, afirma-se que houve aumento nos casos, não diminuição ou arrefecimento. b) Incorreta. Não se qualifica o ECA como texto frágil - o que se afirma é que, simplesmente, ele, juntamente com outros textos legais, não basta para resolver o problema. c) Incorreta. O que se diz no texto é que, muitas vezes, as crianças são consideradas como fardos ou pesos por seus pais. d) Correta. Ao final do texto, afirma-se que ``É hora de agir, de nos unirmos para garantir um futuro melhor para as nossas crianças.''.

\item
BNCC: EF69LP44 - Inferir a presença de valores sociais, culturais e
humanos e de diferentes visões de mundo, em textos literários,
reconhecendo nesses textos formas de estabelecer múltiplos olhares sobre
as identidades, sociedades e culturas e considerando a autoria e o
contexto social e histórico de sua produção. a) Incorreta. O texto não defende explicitamente a importância da profissão dos médicos, mas a importância de haver solidariedade entre os homens. b) Correta. Pode-se identificar que Sylvio Saraiva defendia o altruísmo e a solidariedade entre os homens. c) Incorreta. O texto usa ``lei'' em sentido figurado, argumentando que a solidariedade deve ser tão importante quanto uma lei. d) Incorreta. Não existe a explicitação dessa ideia no discurso de
formatura.
\end{enumerate}

\colorsec{Língua Portuguesa – Módulo 4 – Treino}

\begin{enumerate}
\item

\item

\item
\end{enumerate}

\colorsec{Língua Portuguesa – Módulo 5 – Treino}

\begin{enumerate}
\item

\item

\item
\end{enumerate}

\colorsec{Língua Portuguesa – Módulo 6 – Treino}

\begin{enumerate}
\item

\item

\item
\end{enumerate}

\colorsec{Língua Portuguesa – Módulo 7 – Treino}

\begin{enumerate}
\item

\item

\item
\end{enumerate}

\colorsec{Língua Portuguesa – Módulo 8 – Treino}

\begin{enumerate}
\item

\item

\item
\end{enumerate}

\colorsec{Língua Portuguesa – Módulo 9 – Treino}

\begin{enumerate}
\item

\item

\item
\end{enumerate}

\colorsec{Língua Portuguesa – Módulo 10 – Treino}

\begin{enumerate}
\item

\item

\item
\end{enumerate}

\colorsec{Língua Portuguesa – Módulo 11 – Treino}

\begin{enumerate}
\item

\item

\item
\end{enumerate}

\colorsec{Arte – Módulo 1 –  Treino}

\begin{enumerate}
\item

\item

\item
\end{enumerate}

\colorsec{Arte – Módulo 2 –  Treino}

\begin{enumerate}
\item

\item

\item
\end{enumerate}

\colorsec{Arte – Módulo 3 –  Treino}

\begin{enumerate}
\item

\item

\item
\end{enumerate}

\colorsec{Inglês – Módulo 1 –  Treino}

\begin{enumerate}
\item

\item

\item
\end{enumerate}

\colorsec{Inglês – Módulo 2 –  Treino}

\begin{enumerate}
\item

\item

\item
\end{enumerate}

\colorsec{Inglês – Módulo 3 –  Treino}

\begin{enumerate}
\item

\item

\item
\end{enumerate}

\colorsec{Ciências Humanas – Módulo 1 – Treino}

\begin{enumerate}
\item

\item

\item
\end{enumerate}

\colorsec{Ciências Humanas – Módulo 2 – Treino}

\begin{enumerate}
\item

\item

\item
\end{enumerate}

\colorsec{Ciências Humanas – Módulo 3 – Treino}

\begin{enumerate}
\item

\item

\item
\end{enumerate}

\colorsec{Ciências Humanas – Módulo 4 – Treino}

\begin{enumerate}
\item

\item

\item
\end{enumerate}

\colorsec{Ciências Humanas – Módulo 5 – Treino}

\begin{enumerate}
\item

\item

\item
\end{enumerate}

\colorsec{Ciências Humanas – Módulo 6 – Treino}

\begin{enumerate}
\item

\item

\item
\end{enumerate}

\colorsec{Simulado 1}

\begin{enumerate}
\item
\item
\item
\item
\item
\item
\item
\item
\item
\item
\item
\item
\item
\item
\item
\end{enumerate}

\colorsec{Simulado 2}

\begin{enumerate}
\item
\item
\item
\item
\item
\item
\item
\item
\item
\item
\item
\item
\item
\item
\item
\end{enumerate}

\colorsec{Simulado 3}

\begin{enumerate}
\item
\item
\item
\item
\item
\item
\item
\item
\item
\item
\item
\item
\item
\item
\item
\end{enumerate}

\colorsec{Simulado 4}

\begin{enumerate}
\item
\item
\item
\item
\item
\item
\item
\item
\item
\item
\item
\item
\item
\item
\item
\end{enumerate}