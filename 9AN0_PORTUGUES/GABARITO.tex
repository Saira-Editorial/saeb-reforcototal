\chapter{Respostas}
\pagestyle{plain}
\footnotesize

\pagecolor{gray!40}

\colorsec{Língua Portuguesa – Módulo 1 – Treino}

\begin{enumerate}
\item
BNCC: EF89LP04 - Identificar e avaliar teses/opiniões/posicionamentos explícitos e implícitos, argumentos e contra-argumentos em textos argumentativos do campo (carta de leitor, comentário, artigo de opinião, resenha crítica etc.), posicionando-se frente à questão controversa de forma sustentada. a) Incorreta. Não é uma opinião, mas um fato apresentado pelo psiquiatra. b) Incorreta. Nesse trecho, ele apresenta quantas solicitações de atendimento de casos novos ele atendia por ano, o que não se caracteriza como uma opinião. c) Correta. Ao falar sobre a quantidade da procura de pessoas dependentes de jogos, o psiquiatra cita que, por ser uma população com alto risco, esse crescimento no número de procuras o deixava preocupado. d) Incorreta. O psiquiatra apenas conclui sua fala ao dizer que tudo o que foi falado faz parte de sua experiência ao longo de 20 anos de trabalho.

\item
BNCC: EF89LP04 - Identificar e avaliar teses/opiniões/posicionamentos explícitos e implícitos, argumentos e contra-argumentos em textos argumentativos do campo (carta de leitor, comentário, artigo de opinião, resenha crítica etc.), posicionando-se frente à questão controversa de forma sustentada. (A) Incorreta. O desenvolvimento cognitivo e o socioemocional são dois componentes importantes para a formação do indivíduo, segundo o diretor do colégio. (B) Incorreta. Refere-se à opinião do diretor Willmann, o qual diz que a Seeduc não lhes dá condição melhor, mas que valoriza a ideia de a escola se apropriar do que recebe e acreditar naquilo que está fazendo. (C) Incorreta. São as parcerias com a escola que não dão nada de material e oferecem a mesma formação que é dada para outras escolas. (D) Correta. O estudante diz que eles têm uma integração com o professor, o que não acontece em outras escolas. Foi a única escola que realmente mudou a vida dele, lugar onde gosta de estar e que ajudou a melhorar seu relacionamento com sua família.

\item
BNCC: EF89LP04 - Identificar e avaliar teses/opiniões/posicionamentos
explícitos e implícitos, argumentos e contra-argumentos em textos
argumentativos do campo (carta de leitor, comentário, artigo de opinião,
resenha crítica etc.), posicionando-se frente à questão controversa de
forma sustentada. (A) Incorreta. O trecho apresenta dados de pesquisa, sem apresentar opinião do articulista. (B) Incorreta. O trecho é um dado, e não a opinião do articulista. (C) Correta. O artigo de opinião apresenta dados de pesquisas para afirmar que a situação de vida das pessoas depende do acaso, de onde elas nasceram e quem as educou. (D) Incorreta. O trecho é um exemplo que vai auxiliar a tese de que a condição de vida das pessoas depende do acaso.
\end{enumerate}

\colorsec{Língua Portuguesa – Módulo 2 – Treino}

\begin{enumerate}
\item

\item

\item
\end{enumerate}

\colorsec{Língua Portuguesa – Módulo 3 – Treino}

\begin{enumerate}
\item

\item

\item
\end{enumerate}

\colorsec{Língua Portuguesa – Módulo 4 – Treino}

\begin{enumerate}
\item

\item

\item
\end{enumerate}

\colorsec{Língua Portuguesa – Módulo 5 – Treino}

\begin{enumerate}
\item

\item

\item
\end{enumerate}

\colorsec{Língua Portuguesa – Módulo 6 – Treino}

\begin{enumerate}
\item

\item

\item
\end{enumerate}

\colorsec{Língua Portuguesa – Módulo 7 – Treino}

\begin{enumerate}
\item

\item

\item
\end{enumerate}

\colorsec{Língua Portuguesa – Módulo 8 – Treino}

\begin{enumerate}
\item

\item

\item
\end{enumerate}

\colorsec{Língua Portuguesa – Módulo 9 – Treino}

\begin{enumerate}
\item

\item

\item
\end{enumerate}

\colorsec{Língua Portuguesa – Módulo 10 – Treino}

\begin{enumerate}
\item

\item

\item
\end{enumerate}

\colorsec{Língua Portuguesa – Módulo 11 – Treino}

\begin{enumerate}
\item

\item

\item
\end{enumerate}

\colorsec{Arte – Módulo 1 –  Treino}

\begin{enumerate}
\item

\item

\item
\end{enumerate}

\colorsec{Arte – Módulo 2 –  Treino}

\begin{enumerate}
\item

\item

\item
\end{enumerate}

\colorsec{Arte – Módulo 3 –  Treino}

\begin{enumerate}
\item

\item

\item
\end{enumerate}

\colorsec{Inglês – Módulo 1 –  Treino}

\begin{enumerate}
\item

\item

\item
\end{enumerate}

\colorsec{Inglês – Módulo 2 –  Treino}

\begin{enumerate}
\item

\item

\item
\end{enumerate}

\colorsec{Inglês – Módulo 3 –  Treino}

\begin{enumerate}
\item

\item

\item
\end{enumerate}

\colorsec{Ciências Humanas – Módulo 1 – Treino}

\begin{enumerate}
\item

\item

\item
\end{enumerate}

\colorsec{Ciências Humanas – Módulo 2 – Treino}

\begin{enumerate}
\item

\item

\item
\end{enumerate}

\colorsec{Ciências Humanas – Módulo 3 – Treino}

\begin{enumerate}
\item

\item

\item
\end{enumerate}

\colorsec{Ciências Humanas – Módulo 4 – Treino}

\begin{enumerate}
\item

\item

\item
\end{enumerate}

\colorsec{Ciências Humanas – Módulo 5 – Treino}

\begin{enumerate}
\item

\item

\item
\end{enumerate}

\colorsec{Ciências Humanas – Módulo 6 – Treino}

\begin{enumerate}
\item

\item

\item
\end{enumerate}

\colorsec{Simulado 1}

\begin{enumerate}
\item
\item
\item
\item
\item
\item
\item
\item
\item
\item
\item
\item
\item
\item
\item
\end{enumerate}

\colorsec{Simulado 2}

\begin{enumerate}
\item
\item
\item
\item
\item
\item
\item
\item
\item
\item
\item
\item
\item
\item
\item
\end{enumerate}

\colorsec{Simulado 3}

\begin{enumerate}
\item
\item
\item
\item
\item
\item
\item
\item
\item
\item
\item
\item
\item
\item
\item
\end{enumerate}

\colorsec{Simulado 4}

\begin{enumerate}
\item
\item
\item
\item
\item
\item
\item
\item
\item
\item
\item
\item
\item
\item
\item
\end{enumerate}