\chapter{Respostas}
\pagestyle{plain}
\footnotesize

\pagecolor{gray!40}

\colorsec{Língua Portuguesa – Módulo 1 – Treino}

\begin{enumerate}
\item
BNCC: EF89LP04 - Identificar e avaliar teses/opiniões/posicionamentos explícitos e implícitos, argumentos e contra-argumentos em textos argumentativos do campo (carta de leitor, comentário, artigo de opinião, resenha crítica etc.), posicionando-se frente à questão controversa de forma sustentada. a) Incorreta. Não é uma opinião, mas um fato apresentado pelo psiquiatra. b) Incorreta. Nesse trecho, ele apresenta quantas solicitações de atendimento de casos novos ele atendia por ano, o que não se caracteriza como uma opinião. c) Correta. Ao falar sobre a quantidade da procura de pessoas dependentes de jogos, o psiquiatra cita que, por ser uma população com alto risco, esse crescimento no número de procuras o deixava preocupado. d) Incorreta. O psiquiatra apenas conclui sua fala ao dizer que tudo o que foi falado faz parte de sua experiência ao longo de 20 anos de trabalho.

\item
BNCC: EF89LP04 - Identificar e avaliar teses/opiniões/posicionamentos explícitos e implícitos, argumentos e contra-argumentos em textos argumentativos do campo (carta de leitor, comentário, artigo de opinião, resenha crítica etc.), posicionando-se frente à questão controversa de forma sustentada. (A) Incorreta. O desenvolvimento cognitivo e o socioemocional são dois componentes importantes para a formação do indivíduo, segundo o diretor do colégio. (B) Incorreta. Refere-se à opinião do diretor Willmann, o qual diz que a Seeduc não lhes dá condição melhor, mas que valoriza a ideia de a escola se apropriar do que recebe e acreditar naquilo que está fazendo. (C) Incorreta. São as parcerias com a escola que não dão nada de material e oferecem a mesma formação que é dada para outras escolas. (D) Correta. O estudante diz que eles têm uma integração com o professor, o que não acontece em outras escolas. Foi a única escola que realmente mudou a vida dele, lugar onde gosta de estar e que ajudou a melhorar seu relacionamento com sua família.

\item
BNCC: EF89LP04 - Identificar e avaliar teses/opiniões/posicionamentos
explícitos e implícitos, argumentos e contra-argumentos em textos
argumentativos do campo (carta de leitor, comentário, artigo de opinião,
resenha crítica etc.), posicionando-se frente à questão controversa de
forma sustentada. (A) Incorreta. O trecho apresenta dados de pesquisa, sem apresentar opinião do articulista. (B) Incorreta. O trecho é um dado, e não a opinião do articulista. (C) Correta. O artigo de opinião apresenta dados de pesquisas para afirmar que a situação de vida das pessoas depende do acaso, de onde elas nasceram e quem as educou. (D) Incorreta. O trecho é um exemplo que vai auxiliar a tese de que a condição de vida das pessoas depende do acaso.
\end{enumerate}

\colorsec{Língua Portuguesa – Módulo 2 – Treino}

\begin{enumerate}
\item
BNCC: EF69LP02 - Analisar e comparar peças publicitárias variadas (cartazes, folhetos, outdoor, anúncios e propagandas em diferentes mídias, spots, jingle, vídeos etc.), de forma a perceber a articulação entre elas em campanhas, as especificidades das várias semioses e mídias, a adequação dessas peças ao público-alvo, aos objetivos do anunciante e/ou da campanha e à construção composicional e estilo dos gêneros em questão, como forma de ampliar suas possibilidades de compreensão (e produção) de textos pertencentes a esses gêneros. a) Incorreta. Mesmo que a ``atenção'' produza segurança no trânsito, o objetivo dessa campanha não é apenas mostrar essa informação, mas conscientizar a população a tomar essa atitude. b) Incorreta. Mesmo que se estabeleça uma relação entre a ``atenção'' e a ``segurança'', essa relação não é a finalidade do texto, o qual pretende levar as pessoas a adotar posturas mais amplas de segurança no trânsito. c) Correta. A finalidade dessa campanha é conscientizar a população de forma geral a ter mais atenção em suas ações no trânsito, o que o torna mais seguro para todos ``Onde o cidadão é mais atento e respeitoso, o trânsito respnde com menos mortes e mais organização''. d) Incorreta. A finalidade da campanha não é conscientizar a população sobre o trânsito apenas no mês de maio, mas sim em todos os meses.

\item
BNCC: EF69LP20 - Identificar, tendo em vista o contexto de produção, a
forma de organização dos textos normativos e legais, a lógica de
hierarquização de seus itens e subitens e suas partes: parte inicial
(título -- nome e data -- e ementa), blocos de artigos (parte, livro,
capítulo, seção, subseção), artigos (caput e parágrafos e incisos) e
parte final (disposições pertinentes à sua implementação) e analisar
efeitos de sentido causados pelo uso de vocabulário técnico, pelo uso do
imperativo, de palavras e expressões que indicam circunstâncias, como
advérbios e locuções adverbiais, de palavras que indicam generalidade,
como alguns pronomes indefinidos, de forma a poder compreender o caráter
imperativo, coercitivo e generalista das leis e de outras formas de
regulamentação. (A) Incorreta. O texto não mostra como a Associação fará a preservação e os meios para tal, mas as diretrizes dela nos planos para a preservação do meio ambiente. (B) Incorreta. O texto é em relação à Associação e promovido por ela, por isso não se pode afirmar que seja uma apresentação dos planos governamentais. (C) Incorreta. Os planos são da Associação à preservação ambiental, e não do governo para o desenvolvimento da Associação. (D) Correta. A finalidade é apresentar ao leitor os planos da Associação (APREMAVI) na preservação ambiental, o que pode ser visto em ``tem por objetivos'' e nos planos descritos nas alíneas, como em ``Promover, estimular e apoiar ações e trabalhos {[}\ldots{}{]}''.

\item
BNCC: EF69LP27 - Analisar a forma composicional de textos pertencentes a
gêneros normativos/ jurídicos e a gêneros da esfera política, tais como
propostas, programas políticos (posicionamento quanto a diferentes ações
a serem propostas, objetivos, ações previstas etc.), propaganda política
(propostas e sua sustentação, posicionamento quanto a temas em
discussão) e textos reivindicatórios: cartas de reclamação, petição
(proposta, suas justificativas e ações a serem adotadas) e suas marcas
linguísticas, de forma a incrementar a compreensão de textos
pertencentes a esses gêneros e a possibilitar a produção de textos mais
adequados e/ou fundamentados quando isso for requerido. a) Incorreta. O edital não tem o objetivo de promover o espaço, mas oferecer um programa de residência para alguns pesquisadores. b) Incorreta. Não aparece esse tipo de instrução no texto. c) Incorreta. Não há dicas ou informações sobre um modo mais fácil de ser selecionado. d) Correta. O trecho diz respeito a quais serão os projetos contemplados pela instituição, já que mostra que três serão relativos à temática ``3 vezes 22'', outros três sobre pesquisas relativas ao acervo constante, um a respeito do restauro da biblioteca e, finalmente, outro dedicado à pesquisa de todo o material existente.
\end{enumerate}

\colorsec{Língua Portuguesa – Módulo 3 – Treino}

\begin{enumerate}
\item
BNCC: EF69LP47 - Analisar, em textos narrativos ficcionais, as
diferentes formas de composição próprias de cada gênero, os recursos
coesivos que constroem a passagem do tempo e articulam suas partes, a
escolha lexical típica de cada gênero para a caracterização dos cenários
e dos personagens e os efeitos de sentido decorrentes dos tempos
verbais, dos tipos de discurso, dos verbos de enunciação e das
variedades linguísticas (no discurso direto, se houver) empregados,
identificando o enredo e o foco narrativo e percebendo como se estrutura
a narrativa nos diferentes gêneros e os efeitos de sentido decorrentes
do foco narrativo típico de cada gênero, da caracterização dos espaços
físico e psicológico e dos tempos cronológico e psicológico, das
diferentes vozes no texto (do narrador, de personagens em discurso
direto e indireto), do uso de pontuação expressiva, palavras e
expressões conotativas e processos figurativos e do uso de recursos
linguístico-gramaticais próprios a cada gênero narrativo.
 a) Incorreta. Não se trata de um texto longo, além de não viverem, os dois personagens, na cabana. b) Correta. Trata-se de um texto curto, com conflito resumido e simples e poucos personagens. c) Incorreta. Além de o texto não ser longo, Gabriel é o filho, não o narrador-personagem. d) Incorreta. Além de os personagens não serem exatamente amigos, o
eremita aprendeu a valorizar o silêncio antes do outro homem.

\item
BNCC: EF69LP44 - Inferir a presença de valores sociais, culturais e
humanos e de diferentes visões de mundo, em textos literários,
reconhecendo nesses textos formas de estabelecer múltiplos olhares sobre
as identidades, sociedades e culturas e considerando a autoria e o
contexto social e histórico de sua produção.
 a) Incorreta. Pelo contrário, afirma-se que houve aumento nos casos, não diminuição ou arrefecimento. b) Incorreta. Não se qualifica o ECA como texto frágil - o que se afirma é que, simplesmente, ele, juntamente com outros textos legais, não basta para resolver o problema. c) Incorreta. O que se diz no texto é que, muitas vezes, as crianças são consideradas como fardos ou pesos por seus pais. d) Correta. Ao final do texto, afirma-se que ``É hora de agir, de nos unirmos para garantir um futuro melhor para as nossas crianças.''.

\item
BNCC: EF69LP44 - Inferir a presença de valores sociais, culturais e
humanos e de diferentes visões de mundo, em textos literários,
reconhecendo nesses textos formas de estabelecer múltiplos olhares sobre
as identidades, sociedades e culturas e considerando a autoria e o
contexto social e histórico de sua produção. a) Incorreta. O texto não defende explicitamente a importância da profissão dos médicos, mas a importância de haver solidariedade entre os homens. b) Correta. Pode-se identificar que Sylvio Saraiva defendia o altruísmo e a solidariedade entre os homens. c) Incorreta. O texto usa ``lei'' em sentido figurado, argumentando que a solidariedade deve ser tão importante quanto uma lei. d) Incorreta. Não existe a explicitação dessa ideia no discurso de
formatura.
\end{enumerate}

\colorsec{Língua Portuguesa – Módulo 4 – Treino}

\begin{enumerate}
\item
BNCC: EF69LP16 - Analisar e utilizar as formas de composição dos gêneros
jornalísticos da ordem do relatar, tais como notícias (pirâmide
invertida no impresso X blocos noticiosos hipertextuais e
hipermidiáticos no digital, que também pode contar com imagens de vários
tipos, vídeos, gravações de áudio etc.), da ordem do argumentar, tais
como artigos de opinião e editorial (contextualização, defesa de
tese/opinião e uso de argumentos) e das entrevistas: apresentação e
contextualização do entrevistado e do tema, estrutura pergunta e
resposta etc. a) Incorreta. O marcador temporal da quarta-feira corresponde ao início do Campeonato Paulista, não a quando se deu o fato noticiado. b) Incorreta. O assunto principal da notícia é a criação da campanha ``\#ElasNoEstádio''. c) Correta. O ``quem'' da notícia corresponde à Federação Paulista de Futebol, que criou uma campanha de incentivo à presença das mulheres no estádio. d) Incorreta. O ``onde'' da notícia corresponde à sede da entidade, informação que não aparece no texto.

\item
BNCC: EF69LP43 - Identificar e utilizar os modos de introdução de outras
vozes no texto -- citação literal e sua formatação e paráfrase --, as
pistas linguísticas responsáveis por introduzir no texto a posição do
autor e dos outros autores citados (``Segundo X; De acordo com Y; De
minha/nossa parte, penso/amos que''...) e os elementos de normatização
(tais como as regras de inclusão e formatação de citações e paráfrases,
de organização de referências bibliográficas) em textos científicos,
desenvolvendo reflexão sobre o modo como a intertextualidade e a
retextualização ocorrem nesses textos. a) Incorreta. Os ratos não eram superativados, mas sim as proteínas presentes no corpo dos animais. b) Incorreta. Os testes não foram feitos para investigar a força e a massa muscular superativada nos ratos, mas para observar como a ativação de uma proteína consegue evitar a perda de massa muscular. c) Correta. Em ``A força e a massa muscular podem ser conservadas por meio da ativação de uma proteína, o que foi observado por cientistas em testes com ratos'', observa-se que as informações presentes no trecho original foram mantidas, embora tenham sido comunicadas de modo diferente. d) Incorreta. O estudo não é para investigar a força dos cientistas, mas para compreender como a massa muscular reage a uma proteína superativada.

\item
BNCC: EF89LP05 - Analisar o efeito de sentido produzido pelo uso, em
textos, de recurso a formas de apropriação textual (paráfrases,
citações, discurso direto, indireto ou indireto livre). a) Incorreta. A fala do morador relata o que está acontecendo no condomínio, não apresentando uma opinião específica que possa ser diferente da opinião das autoras. b) Incorreta. A fala do morador mostra o que o condomínio dele está fazendo para conter a pandemia. c) Correta. Ao apresentar a fala literal do morador, as autoras da notícia mostram imparcialidade d) Incorreta. A fala literal é objetiva, já que mostra o que foi dito sem a interferência das autoras do texto. 
\end{enumerate}

\colorsec{Língua Portuguesa – Módulo 5 – Treino}

\begin{enumerate}
\item
BNCC: EF67LP04 - Distinguir, em segmentos descontínuos de textos, fato
da opinião enunciada em relação a esse mesmo fato. a) Incorreta. O trecho corresponde a uma informação sobre o enredo do filme; não apresenta, portanto, opiniões do resenhista sobre ele, demonstrando que o aluno considerou um trecho descritivo como uma opinião em vez de um fato. b) Correta. O trecho ``A fotografia traz cenas belas e bem montadas''mostra qual a opinião do resenhista sobre um aspecto técnico do filme, já que exibe um juízo de valor sobre a fotografia, de modo a caracterizar as cenas como ``belas'' e ``bem montadas''c) Incorreta. O trecho não contém opinião, visto que representa apenas o modo como o filme é estruturado. d) Incorreta. O trecho corresponde a uma informação do enredo do filme,
trazendo uma informação técnica da sinopse.

\item
a) Incorreta. O autor expõe sua opinião ao utilizar a palavra
``ceticismo'', dizendo que o mundo estava descrente sobre os
acontecimentos. b) Correta. O autor utiliza como exemplo o fato sobre a epidemia do ebola que aconteceu no continente africano para justificar a descrença da população a respeito da nova pandemia do coronavírus c) Incorreta. No trecho, o autor indica que, quando surgiram os primeiros infectados, o mundo se viu em meio a uma situação incomum. d) Incorreta. Trata-se da opinião do autor a respeito de como as coisas são resolvidas no Brasil.

\item
a) Correta. Somente a participação na enquete popular não iria garantir
o resultado oficial, visto que para esse resultado era necessária a
participação na enquete oficial das mascotes. b) Incorreta. Não há garantias de que quem participou da votação popular também votou na oficial, fato que pode ter ocorrido ou não. c) Incorreta. Não há como saber, pelos textos, que os autores da enquete não tinham conhecimento da homenagem às personalidades. d) Incorreta. Os nomes Oba e Eba também participaram da enquete oficial, apesar de não serem citados no texto II, isso porque, na enquete popular o jornal de esportes colocou as opções que estavam para ser escolhidas oficialmente.
\end{enumerate}

\colorsec{Língua Portuguesa – Módulo 6 – Treino}

\begin{enumerate}
\item
BNCC: EF69LP05 - Inferir e justificar, em textos multissemióticos --
tirinhas, charges, memes, gifs etc. --, o efeito de humor, ironia e/ou
crítica pelo uso ambíguo de palavras, expressões ou imagens ambíguas, de
clichês, de recursos iconográficos, de pontuação etc.
 a) Incorreta. O colorido não gera humor. b) Incorreta. Pratos de vegetais combinam entre si. c) Incorreta. Vegetarianos comem ovos. d) Correta. O humor da charge se encontra no fato de que em apenas um dos dias da semana não se come carne pelo meio ambiente e pelo planeta; nos demais, o impedimento é o preço da carne.

\item
BNCC: EF69LP03 - Identificar, em notícias, o fato central, suas
principais circunstâncias e eventuais decorrências; em reportagens e
fotorreportagens o fato ou a temática retratada e a perspectiva de
abordagem, em entrevistas os principais temas/subtemas abordados,
explicações dadas ou teses defendidas em relação a esses subtemas; em
tirinhas, memes, charge, a crítica, ironia ou humor presente. (A) Incorreta. O tema principal da entrevista gira em torno da evolução das animações. (B) Incorreta. O filme é apenas citado. (C) Incorreta. O festival não é o tema principal. (D) Correta. O tema principal é a evolução da animação no Brasil, dadas
as informações sobre a mudança ocorrida nos últimos 12 anos

\item
BNCC: EF69LP05 - Inferir e justificar, em textos multissemióticos --
tirinhas, charges, memes, gifs etc. --, o efeito de humor, ironia e/ou
crítica pelo uso ambíguo de palavras, expressões ou imagens ambíguas, de
clichês, de recursos iconográficos, de pontuação etc.
 a) Incorreta. Não há tom crítico na charge, mas um tom humorístico. b) Incorreta. A charge não tem caráter atemporal, visto que depende da compreensão dos personagens da Família Addams, que podem não ser atuais, além da situação do coronavírus, que é um acontecimento novo. c) Incorreta. Não há elementos que indiquem que a família recebeu a personagem Mãozinha. d) Correta. Mediante o conhecimento prévio do leitor sobre as personagens da série \emph{A Família Addams}, o efeito de humor ocorre pela situação de o coronavírus ter afetado mais fortemente uma personagem que é apenas uma mão, a qual tem de lidar com as restrições e os procedimentos higiênicos específicos da pandemia relacionada ao coronavírus.
\end{enumerate}

\colorsec{Língua Portuguesa – Módulo 7 – Treino}

\begin{enumerate}
\item
Vale a pena explicar aos alunos, neste ponto, a diferença básica entre os conceitos de "tema" e "assunto". O assunto é algo mais amplo, que geralmente pode ser resumido em uma palavra ou expressão muito curta, como "ecologia" ou "agenda 2030". Já o tema é um recorte do assunto, sendo mais específico. De um único assunto podem surgir vários temas. Do assunto "Agenda 2030", por exemplo, extraiu-se o tema "Perspectivas brasileiras para cumprimento dos objetivos da Agenda 2030".
(a) Incorreta. Os textos não se complementam, mas sim se contrapõem.
(b) Correta. Os textos tratam de um tema comum, o planejamento para o
cumprimento da agenda 2030 da ONU, mediante perspectivas diferentes,
contrapondo-se, tendo em vista que o primeiro afirma que a erradicação
da pobreza e da fome é possível, enquanto o segundo afirma que é
impossível.
(c) Incorreta. Os textos não tratam do assunto pela mesma perspectiva.
(d) Incorreta. Os textos tratam do mesmo tema.

\item
a) Incorreta. A notícia expõe os países que jogarão antes da copa de
2022, além de haver imparcialidade do autor.
b) Incorreta. A notícia não informa sobre possíveis acontecimentos para
a Copa, além de o autor ser imparcial.
c) Incorreta. O autor mostra imparcialidade, não expondo a sua opinião
em relação ao assunto.
d) Correta. A notícia mostra imparcialidade do autor, que não expõe sua
opinião sobre o assunto.

\item
a) Incorreta. O texto não é voltado para um público menos escolarizado.
 b) Incorreta. O texto traz informações precisas a respeito da produção petrolífera dos Estados Unidos e cita que entrou em crise oito dias atrás, um fato histórico. c) Incorreta. O texto foi escrito majoritariamente em terceira pessoa, o que evidencia a intenção do jornalista de ser imparcial e objetivo. d) Correta. As informações são apresentadas de modo impessoal e objetivo.
\end{enumerate}

\colorsec{Língua Portuguesa – Módulo 8 – Treino}

\begin{enumerate}
\item
BNCC: EF89LP31 -- Analisar e utilizar modalização epistêmica, isto é,
modos de indicar uma avaliação sobre o valor de verdade e as condições
de verdade de uma proposição, tais como os asseverativos -- quando se
concorda com (``realmente, evidentemente, naturalmente, efetivamente,
claro, certo, lógico, sem dúvida'' etc.) ou discorda de (``de jeito
nenhum, de forma alguma'') uma ideia; e os quase-asseverativos, que
indicam que se considera o conteúdo como quase certo (``talvez, assim,
possivelmente, provavelmente, eventualmente'').
 a) Incorreta. Inexistem modalizadores nos dois textos, uma vez que no texto 2 a autora contextualiza a Lei Seca e seu objetivo, mas sem emitir opinião ou mostrar posição em relação ao assunto. b) Incorreta. O texto 2 apresenta o contexto da Lei Seca e seu objetivo, mas isso não é um modalizador. c) Correta. Há um modalizador apenas no texto I, o que pode ser visto pelo uso do advérbio ``principalmente'', para evidenciar a importância da Lei Seca. d) Incorreta. Não existem modalizadores em ambos os textos.

\item
BNCC: EF69LP04 -- Identificar e analisar os efeitos de sentido que
 fortalecem a persuasão nos textos publicitários, relacionando as estratégias de persuasão e apelo ao consumo com os recursos linguístico-discursivos utilizados, como imagens, tempo verbal, jogos de palavras, figuras de linguagem etc., com vistas a fomentar práticas de consumo conscientes. a) Incorreta. Essas são recomendações gerais para o dia da doação de sangue. b) Incorreta. A campanha pede aos doadores de sangue que evitem a ingestão de alimentos gordurosos. c) Incorreta. Essa informação trata do grupo de pessoas que não pode doar sangue. d) Correta. Mulheres grávidas ou que estejam amamentando estão no grupo de pessoas que não podem doar sangue.

\item
BNCC: EF89LP16 -- Analisar a modalização realizada em textos noticiosos
e argumentativos, por meio das modalidades apreciativas, viabilizadas
por classes e estruturas gramaticais como adjetivos, locuções adjetivas,
advérbios, locuções adverbiais, orações adjetivas e adverbiais, orações
relativas restritivas e explicativas etc., de maneira a perceber a
apreciação ideológica sobre os fatos noticiados ou as posições
implícitas ou assumidas. a) Incorreta. O trecho não traz exemplos. b) Correta. No trecho ``Isso deve ocorrer por meio da promoção de palestras'', a candidata consegue amarrar a exposição feita anteriormente e, mediante modalizador ``deve'', dá sua opinião e propõe uma intervenção pelo Estado. c) Incorreta. Nesse trecho, não há críticas. d) Incorreta. Ao usar a expressão ``dessa forma'', ela deixou em evidência a tentativa de retomar e reunir os argumentos anteriores, finalizando o texto.
\end{enumerate}

\colorsec{Língua Portuguesa – Módulo 9 – Treino}

\begin{enumerate}
\item
a) Correta. O uso ``só'' mostra o ponto de vista do veículo de
comunicação de que esse tipo de destruição era observado somente após
fenômenos naturais. b) Incorreta. A palavra ``só'' indica uma restrição, não quantidade. c) Incorreta. A comparação é feita pela interpretação da frase, não pela utilização de ``só''. d) Incorreta. O termo ``só'' indica que essa destruição era vista após fenômenos naturais, não que nunca foi vista.

\item
BNCC: EF89LP37 -- Analisar os efeitos de sentido do uso de figuras de
linguagem como ironia, eufemismo, antítese, aliteração, assonância,
dentre outras. a) Correta. A repetição das mesmas palavras e estruturas de frases no começo de versos denomina-se anáfora e, nesse poema, essa figura de linguagem está originando musicalidade. b) Incorreta. A gradação é uma figura de linguagem na qual os sentidos das expressões se intensificam. c) Incorreta. A alegoria é uma figura de linguagem que compara dois elementos. d) Incorreta. A metáfora é uma figura de linguagem que compara dois elementos sem utilizar conjunção.

\item
BNCC: EF89LP37 -- Analisar os efeitos de sentido do uso de figuras de linguagem como ironia, eufemismo, antítese, aliteração, assonância, dentre outras. a) Incorreta. A anáfora caracteriza-se pela repetição de uma mesma expressão em diferentes versos. b) Correta. A frase em destaque compara o riso a um instrumento musical, sem utilizar palavras comparativas, caracterizando-se como uma metáfora c) Incorreta. A comparação difere-se da metáfora por utilizar palavras comparativas; por exemplo, ``como''. d) Incorreta. A gradação consiste em construir uma série na qual o sentido das expressões vai se intensificando até um clímax ou se
reduzindo até um anticlímax.
\end{enumerate}

\colorsec{Língua Portuguesa – Módulo 10 – Treino}

\begin{enumerate}
\item
BNCC: EF09LP11 -- Inferir efeitos de sentido decorrentes do uso de recursos de coesão sequencial (conjunções e articuladores textuais). a) Incorreta. Uma vez que a informação é complementar à outra, não há exclusão de informação. b) Correta. A conjunção ``e'' usada no período tem papel aditivo, ligando duas orações independentes, no chamado período composto por coordenação. c) Incorreta. Não há sentido de explicação no trecho, porque a primeira oração apresenta um sentido completo. d) Incorreta. Embora a conjunção introduza uma oração que complementa a primeira, esta não apresenta função de dar fundamentações à primeira.

\item
BNCC: EF09LP11 -- Inferir efeitos de sentido decorrentes do uso de
recursos de coesão sequencial (conjunções e articuladores textuais). a) Incorreta. A mídia deve ser parceira do governo nessa questão, segundo a autora, não havendo exclusão do meio digital, mas sim de inserção de pensamento crítico na população. b) Incorreta. O texto, nesse momento, não apresenta exemplos dos problemas que a mídia pode causar, mas sim apresenta propostas de solução para a tese. c) Incorreta. O texto não contrasta a mídia com o Ministério da Educação, mas sim propõe uma articulação entre as duas esferas. d) Correta. As expressões apresentam os pronomes ``isso'', ``disso'' e ``dessa'', os quais retomam as ideias dispostas no texto, que estão relacionadas diretamente com a resolução da tese retomada nessa conclusão

\item
BNCC: EF09LP08 - Identificar, em textos lidos e em produções próprias, a relação que conjunções (e locuções conjuntivas) coordenativas e subordinativas estabelecem entre as orações que conectam. a) Incorreta. É uma oração subordinada adverbial consecutiva. b) Incorreta. É uma oração subordinada adverbial condicional. c) Incorreta. É uma oração subordinada adverbial causal. d) Correta. ``Assim que cheguei em casa'' é uma oração subordinada adverbial temporal, assim como a oração destacada no texto.
\end{enumerate}

\colorsec{Língua Portuguesa – Módulo 11 – Treino}

\begin{enumerate}
\item
BNCC: EF69LP55 -- Reconhecer as variedades da língua falada, o conceito de norma-padrão e o de preconceito linguístico. a) Incorreta. A expressão não apresenta traços da língua falada. b) Incorreta. A expressão está em conformidade com a norma-padrão. c) Correta. Verifica-se a variedade falada da língua no verso ``tava tudo bem nublado'', pois ``tava'' é uma forma reduzida de ``estava''. d) Incorreta. A expressão está em conformidade com a norma-padrão.

\item
BNCC: EF69LP56: Fazer uso consciente e reflexivo de regras e normas da norma-padrão em situações de fala e escrita nas quais ela deve ser usada. a) Incorreta. Esse não é a razão de a norma-padrão ser usada na carta. b) Incorreta. O trecho não mostra a chegada dos portugueses, além de que esse não é o motivo de a norma-padrão ser usada. c) Incorreta. Nesse trecho, não há mais dados para afirmar o tempo dos acontecimentos, além de esse não ser o motivo de a norma padrão ser usada. d) Correta. Por ser uma dirigida a uma autoridade, a carta usa a norma-padrão da língua utilizada no período.

\item
BNCC: EF09LP07 - Comparar o uso de regência verbal e regência nominal na norma-padrão com seu uso no português brasileiro coloquial oral. a) Correta. Os verbos que não precisam de complemento são os intransitivos. b) Incorreta. Os verbos transitivos diretos são os que necessitam complemento e se ligam a ele de forma direta. c) Incorreta. Os verbos transitivos indiretos são os que necessitam complemento e se ligam a ele indiretamente. d) Incorreta. O objeto indireto é o termo que completa o sentido de um
verbo transitivo indireto.
\end{enumerate}

\colorsec{Arte – Módulo 1 –  Treino}

\begin{enumerate}
\item
SAEB: Analisar formas, gêneros e estilos distintos de artes visuais e
dança, em diferentes contextos, por meio de seus elementos
constitutivos.
BNCC: EF69AR01 -- Pesquisar, apreciar e analisar formas distintas das artes visuais tradicionais e
contemporâneas, em obras de artistas brasileiros e estrangeiros de diferentes épocas e em
diferentes matrizes estéticas e culturais, de modo a ampliar a experiência com diferentes
contextos e práticas artístico-visuais e cultivar a percepção, o imaginário, a capacidade de
simbolizar e o repertório imagético.
a) Correta. Na instalação, há o diálogo entre diferentes linguagens e a
interação com o público, que percorre o espaço percebendo a obra de
vários ângulos e com os diferentes sentidos.
b) Incorreta. A instalação pode ter um caráter efêmero, só existindo no
período determinado para a exposição, ou pode ser desmontada e recriada
em outro local.
c) Incorreta. Não se relaciona com a arte moderna, pois é uma linguagem
artística que surgiu a partir dos anos 1960, ou seja, é arte
contemporânea.
d. Incorreta. O espectador participa ativamente da obra, e não se
comporta somente como apreciador.

\item
SAEB: Analisar a função do tema como projeto integrador das diferentes
linguagens artísticas.
BNCC: EF69AR03 -- Analisar situações nas quais as linguagens das artes visuais se integram às
linguagens audiovisuais (cinema, animações, vídeos etc.), gráficas (capas de livros, ilustrações
de textos diversos etc.), cenográficas, coreográficas, musicais etc.
a) Incorreta. Apesar de, no cinema, o Impressionismo ser do mesmo período temporal que o Expressionismo,
a estética é diferenciada. No impressionismo, o foco da filmagem é no
exterior, no ângulo subjetivo (olhar do personagem). Esse movimento teve
início na França.
b) Correta. O filme \emph{O gabinete do dr. Caligari} é do movimento
expressionista alemão e possui como características o foco da filmagem
no interior, a morbidez temática e a interpretação teatral.
c) Incorreta. O movimento surrealista do cinema também é do mesmo
período temporal (1920-1930) e teve início na França.
d. Incorreta. O neorrealismo italiano surgiu na década de 1940 e buscava
descrever de forma fiel a vida nos anos de guerra. As filmagens eram
feitas nas ruas.

\item
SAEB: Reconhecer artistas que contribuíram para o desenvolvimento e a
disseminação de diferentes gêneros e estilos nas artes visuais, dança,
música e teatro.
BNCC: EF69AR04 -- Analisar os elementos constitutivos das artes visuais (ponto, linha, forma, direção,
cor, tom, escala, dimensão, espaço, movimento etc.) na apreciação de diferentes produções
artísticas.
a) Incorreta. Leonardo da Vinci é representante da arte renascentista.
Suas obras são regidas pela figuração e pela imitação do mundo.
b) Incorreta. As obras de Frida Kahlo eram regidas pela figuração
estilizada.
c) Correta. Piet Mondrian (1872-1944), pintor holandês, é um dos maiores
representantes do abstracionismo geométrico.
d) Incorreta. Vincent Van Gogh utilizava em suas obras arte figurativa
estilizada.
\end{enumerate}

\colorsec{Arte – Módulo 2 –  Treino}

\begin{enumerate}
\item
SAEB: Analisar formas, gêneros e estilos distintos de música e teatro
em diferentes contextos, por meio de seus elementos constitutivos.
BNCC: EF69AR21 -- Explorar e analisar fontes e materiais sonoros em práticas de composição/criação,
execução e apreciação musical, reconhecendo timbres e características de instrumentos
musicais diversos.
a) Incorreta. Flauta, trompete e trombone são exemplos de aerofones.
b) Incorreta. Violino, guitarra e harpa são exemplos de cordofones.
c) Correta. Triângulos, pratos e sinos produzem som por meio da vibração 
de corpos sólidos, sem estar submetidos a tensão.
d) Incorreta. Surdo, tamborim e pandeiro são exemplos de membranofones.

\item
SAEB: Analisar o papel dos profissionais e a utilização dos
equipamentos culturais no sistema de produção e circulação das artes
visuais, dança, música e teatro.
BNCC: EF69AR09 -- Pesquisar e analisar diferentes formas de expressão, representação e encenação
da dança, reconhecendo e apreciando composições de dança de artistas e grupos brasileiros e
estrangeiros de diferentes épocas.
a) Incorreta. O balé surgiu nas cortes italianas no início do século XVI.
b) Correta. O balé surgiu na Itália, no início do século XVI, no período renascentista.
c) Incorreta. O balé é anterior à Idade Contemporânea.
d) Incorreta. O balé surgiu antes do arte moderna.

\item
SAEB: Analisar formas, gêneros e estilos distintos de música e teatro
em diferentes contextos, por meio de seus elementos constitutivos.
EF69AR24 -- Reconhecer e apreciar artistas e grupos de teatro brasileiros e estrangeiros de
diferentes épocas, investigando os modos de criação, produção, divulgação, circulação e
organização da atuação profissional em teatro.
a)  Incorreta. Não se trata de um gênero teatral.
b)  Incorreta. Não se trata de um movimento teatral.
c)  Incorreta. Cenário realista e recriação detalhada são características
  do teatro tradicional.
d) Correta. Trata-se de uma expressão cunhada pelo crítico húngaro Martin Esslin
  para definir os traços estilísticos e os temas das peças teatrais no
  pós-Segunda Guerra Mundial.
\end{enumerate}

\colorsec{Arte – Módulo 3 –  Treino}

\begin{enumerate}
\item
SAEB: Avaliar produções que inter-relacionam diferentes linguagens
artísticas.
BNCC: EF69AR34 -- Analisar e valorizar o patrimônio cultural, material e imaterial, de culturas
diversas, em especial a brasileira, incluindo suas matrizes indígenas, africanas e europeias,
de diferentes épocas, e favorecendo a construção de vocabulário e repertório relativos às
diferentes linguagens artísticas.
a) Incorreta. O carimbó expressa o canto, o toque dos instrumentos e a
  dança, com influências culturais indígena, africana e europeia.
b) Incorreta. O frevo é uma forma de expressão musical, coreográfica e
  poética, originária do Pernambuco.
c) Incorreta. O samba de roda do recôncavo baiano é uma expressão
  musical, coreográfica, poética e festiva, com matrizes africana e
  europeia.
d) Correta. Todos os elementos da descrição referem-se à roda de capoeira.

\item
SAEB: Avaliar o papel das diversas linguagens artísticas no
questionamento de estereótipos e preconceitos.
BNCC: EF69AR34 -- Analisar e valorizar o patrimônio cultural, material e imaterial, de culturas
diversas, em especial a brasileira, incluindo suas matrizes indígenas, africanas e europeias,
de diferentes épocas, e favorecendo a construção de vocabulário e repertório relativos às
diferentes linguagens artísticas.
a) Incorreta. Escultura é um patrimônio cultural imaterial.
b) Incorreta. Monumentos fazem parte do patrimônio cultural material.
c) Correta. Cultos e religiões fazem parte do patrimônio cultural imaterial.
d) Incorreta. Espaços físicos do município são patrimônios culturais materiais.

\item
SAEB: Avaliar nas linguagens artísticas a diversidade do patrimônio
cultural da humanidade (material e imaterial), em especial o brasileiro,
a partir de suas diferentes matrizes.
BNCC: EF69AR34 -- Analisar e valorizar o patrimônio cultural, material e imaterial, de culturas
diversas, em especial a brasileira, incluindo suas matrizes indígenas, africanas e europeias,
de diferentes épocas, e favorecendo a construção de vocabulário e repertório relativos às
diferentes linguagens artísticas.
a) Incorreta. Fazem parte do patrimônio arqueológico os bens relacionados
  a vestígios da ocupação pré-histórica.
b) Correta. Fazem parte do patrimônio paisagístico tanto as áreas
  naturais quanto lugares criados pelo homem aos quais é atribuído
  valor por sua configuração paisagística e que se destaquem por sua
  relação com o território onde se encontram.
c) Incorreta. Fazem parte do patrimônio etnográfico bens de valor
  etnográfico e de referência para grupos sociais.
d) Incorreta. Fazem parte do patrimônio das artes aplicadas os bens que
  têm seu valor artístico associado à função utilitária, muito comum na
  arquitetura, no \textit{design} e nas artes gráficas.
\end{enumerate}

\colorsec{Língua Inglesa – Módulo 1 –  Treino}

\begin{enumerate}
\item
SAEB: Identificar a finalidade de um texto em língua inglesa, com base
em sua estrutura, organização textual, pistas gráficas e/ou aspectos
linguísticos.
BNCC: EF06LI07 -- Formular hipóteses sobre a finalidade de um texto em
língua inglesa, com base em sua estrutura, organização textual e pistas
gráficas.
a) Incorreta. Esse é o intuito de um texto instrucional.
b) Incorreta. Esse é o intuito de uma carta.
c) Incorreta. Esse é o intuito de um texto literário.
d) Correta. Esse é um dos intuitos de uma notícia, como no caso do
texto.

\item
SAEB: Identificar o assunto de um texto, a partir de sua organização, de
palavras cognatas e/ou de palavras formas por afixação.
BNCC: EF06LI06 -- Antecipar o sentido global de textos em língua inglesa
por inferências, com base em leitura rápida, observando títulos,
primeiras e últimas frases de parágrafos e palavras-chave repetidas.
a) Incorreta. O texto não menciona a fauna da região.
b) Incorreta. O texto apenas menciona os ventos.
c) Incorreta. O texto não traz medidas para reverter a situação.
d) Correta. O texto fala sobre os níveis baixos de gelo observados na
Antártida.

\item
SAEB: Localizar informações específicas, a partir de diferentes
objetivos de leitura, em textos em língua inglesa.
BNCC: EF06LI08 -- Identificar o assunto de um texto, reconhecendo sua
organização textual e palavras cognatas.
a) Incorreta. O texto menciona descobertas feitas em Marte.
b) Incorreta. O texto menciona que o lago existiu há muito tempo em
Marte.
c) Correta. Essa informação é mencionada explicitamente.
d) Incorreta. O texto afirma que essa descoberta ocorreu na subida do
monte.
\end{enumerate}

\colorsec{Língua Inglesa – Módulo 2 –  Treino}

\begin{enumerate}
\item
SAEB: Identificar os recursos verbais e/ou não verbais que contribuem
para a construção da argumentação em textos em língua inglesa.
BNCC: EF09LI07 -- Identificar argumentos principais e as
evidências/exemplos que os sustentam.
a) Incorreta. O texto apenas menciona a ocorrência de incêndios.
b) Incorreta. O texto afirma que essa coexistência é difícil.
c) Incorreta. O texto apenas afirma que a temperatura tem aumentado.
d) Correta. Essa afirmação pode ser encontrada no trecho ```Recognizing
that climate is an important driver can help us better predict when
they'll occur''.

\item
SAEB: Identificar os recursos verbais e/ou não verbais que contribuem
para a construção da argumentação em textos em língua inglesa.
BNCC: EF09LI07 -- Identificar argumentos principais e as
evidências/exemplos que os sustentam.
a) Incorreta. O texto expõe um argumento oposto.
b) Incorreta. O texto afirma que foram descobertos novos objetos na
órbita.
c) Correta. Essa afirmação aparece no trecho ``because
any objects around Jupiter would be moving at the same rate as the gas
giant''.
d) Incorreta. A dificuldade reside na observação dos objetos que
circundam Júpiter, mas não do planeta em si.

\item
SAEB: Identificar os recursos verbais e/ou não verbais que contribuem
para a construção da argumentação em textos em língua inglesa.
BNCC: EF09LI07 -- Identificar argumentos principais e as
evidências/exemplos que os sustentam.
a) Correta. Essa informação pode ser encontrada no trecho ``it is very important to keep on wearing masks since we the pandemic is not over yet''.
b) Incorreta. De acordo com o texto, a pandemia ainda não terminou.
c) Incorreta. O especialista defende a continuidade do uso.
d) Incorreta. O texto menciona políticos que fazem o contrário.
\end{enumerate}

\colorsec{Língua Inglesa – Módulo 3 –  Treino}

\begin{enumerate}
\item
SAEB: Contrapor perspectivas sobre um mesmo assunto em textos em língua
inglesa.
BNCC: EF09LI06 -- Distinguir fatos de opiniões em textos argumentativos
da esfera jornalística.
a) Incorreta. O texto menciona uma grande descoberta de um mosaico.
b) Correta. A falta de espaço constitui o tema da notícia.
c) Incorreta. O texto não menciona o interesse do público.
d) Incorreta. O texto apenas menciona o conselho.

\item
SAEB: Avaliar a qualidade e a validade das informações veiculadas em
textos de língua inglesa, incluindo textos provenientes de ambientes
virtuais.
BNCC: EF09LI06 -- Distinguir fatos de opiniões em textos argumentativos
da esfera jornalística.
a) Incorreta. O texto não faz referência à idade dos achados.
b) Incorreta. O texto afirma exatamente o contrário, destacando a
relevância dos achados.
c) Correta. A especialista destaca o uso de arcos para lançar as
flechas.
d) Incorreta. O texto não afirma isso.

\item
SAEB: Distinguir fatos de opiniões em textos em língua inglesa.
BNCC: EF09LI06 -- Distinguir fatos de opiniões em textos argumentativos
da esfera jornalística.
a) Incorreta. O texto menciona justamente a observação do céu.
b) Correta. Essa aproximação observada durante algumas noites é o tema
do texto.
c) Incorreta. O texto não menciona, especificamente, o brilho dos planetas.
d) Incorreta. O texto menciona que a aproximação pode ser observada
já há algum tempo.
\end{enumerate}

\colorsec{Ciências Humanas – Módulo 1 – Treino}

\begin{enumerate}
\item
BNCC: EF09GE06 -- Associar o critério de divisão do mundo em Ocidente e
Oriente com o Sistema Colonial implantado pelas potências europeias.
a) Incorreta. A porção colonizada pela Europa também
  integra o mundo ocidental. b) Incorreta: O Islamismo é uma religião presente majoritariamente no
  mundo oriental. c) Incorreta. Essa concepção considera apenas a posição dos países no
  globo, mas a concepção de mundo ocidental considera principalmente o
  aspecto cultural. d) Correta. O mundo ocidental é entendido como o conjunto dos países e
  das sociedades que foram condicionados em alguma medida pela sociedade
  europeia ocidental, e isso inclui a própria Europa e todo o conjunto de
  países colonizados pelos europeus, como o continente americano, a
  Austrália e a Nova Zelândia.

\item
BNCC: EF09GE05 -- Analisar fatos e situações para compreender a
integração mundial (econômica, política e cultural), comparando as
diferentes interpretações: globalização e mundialização.
a) Incorreta. A astronomia é uma ciência que se desenvolve desde o Egito
  e a Grécia antiga. b) Incorreta. Os primeiros mapas não dependiam de satélites para sua
  confecção e baseavam-se em dados de navegação, esquemas matemáticos e
  idealizações do mundo. c) Correta. O mapa T-O representava apenas a porção do mundo conhecida
  por seus elaboradores; assim todo o continente americano ficava de
  fora dessa representação, pois apenas o mundo conhecido era
  representado. d) Incorreta. Apesar da existência de lendas sobre a existência de seres
  míticos para além do mundo conhecido, essas lendas só apareceram mais
  fortemente durante as grandes navegações.

\item
BNCC: EF09GE05 -- Analisar fatos e situações para compreender a
integração mundial (econômica, política e cultural), comparando as
diferentes interpretações: globalização e mundialização.
a) Incorreta. Não existe no texto a contraposição à ideia apresentada
  pela alternativa. b) Incorreta. A menção à busca pelo desenvolvimento
  humano e a crítica ao neoliberalismo evidenciam que o Fórum critica a
  falta dos aspectos apresentados na alternativa. c) Incorreta. O fato de se reivindicar maior desenvolvimento humano é uma
  evidência que atesta o não desenvolvimento dos países
  subdesenvolvidos. d) Correta. No texto, critica-se a política neoliberal, ou seja, aquela
  que coloca nos mercados e nas empresas financeiras o centro das políticas
  econômicas, permitindo que o capital financeiro de países estrangeiros
  possa agir sem regulamentação em todo o mundo.
\end{enumerate}

\colorsec{Ciências Humanas – Módulo 2 – Treino}

\begin{enumerate}
\item
BNCC: EF09GE18 -- Identificar e analisar as cadeias industriais e de
inovação e as consequências dos usos de recursos naturais e das
diferentes fontes de energia (tais como termoelétrica, hidrelétrica,
eólica e nuclear) em diferentes países.
a) Incorreta. Se for considerado apenas o aumento da produção de painéis,
haveria um impacto potencialmente negativo em razão da exploração dos
recursos naturais.
b) Incorreta. No texto, cita-se que existem 49.000 painéis
solares domésticos, o que é insuficiente para suprir toda a demanda
energética do país, que tem mais de 200 milhões de habitantes.
c) Correta. A instalação de painéis solares permite o aproveitamento do
potencial energético local, já que são instalados em cima de casas para
poder captar a energia do Sol.
d) Incorreta. Os painéis não geram diminuição do consumo de energia; apenas
se tornam uma fonte a mais de eletricidade disponível.

\item
BNCC: EF09GE18 -- Identificar e analisar as cadeias industriais e de
inovação e as consequências dos usos de recursos naturais e das diferentes fontes de energia (tais
como termoelétrica, hidrelétrica, eólica e nuclear) em diferentes países.
a) Correta. A matriz brasileira apresenta alguns avanços em relação à matriz mundial, mas a tendência nas duas é a mesma.
b) Incorreta. Há algumas melhorias, do ponto de vista da sustentabilidade, da matriz brasileira em relação à mundial.
c) Incorreta. A energia nuclear está menos presente no Brasil do que no mundo.
d) Incorreta. Há fontes renováveis tanto na matriz brasileira quanto na mundial.

\item
BNCC: EF09GE18 -- Identificar e analisar as cadeias industriais e de
inovação e as consequências dos usos de recursos naturais e das diferentes fontes de energia (tais
como termoelétrica, hidrelétrica, eólica e nuclear) em diferentes países.
a) Incorreta. O fato de as emissões humanas estarem interferindo no clima
global mostra que há uma integração da sociedade com o ciclo climático
global.
b) Incorreta. Os gases do efeito estufa não têm capacidade de gerar
resfriamento atmosférico.
c) Incorreta. Apesar de poderem favorecer crescimento florestal, as
emissões de gases do efeito estufa têm seu maior impacto no aquecimento
atmosférico, estando também relacionadas à própria destruição florestal
existente, já que o desmatamento promove a emissão de gás carbônico.
d) Correta. As emissões de gases do efeito estufa se integraram à dinâmica
climática, pois o maior aquecimento atmosférico está alterando os ciclos
dos sistemas atmosféricos e terrestres como um todo.
\end{enumerate}

\colorsec{Ciências Humanas – Módulo 3 – Treino}

\begin{enumerate}
\item
BNCC: EF09GE02 -- Analisar a atuação das corporações internacionais e
das organizações econômicas mundiais na vida da população em relação ao
consumo, à cultura e à mobilidade. (A) Correta. Um patrimônio cultural imaterial corresponde a uma expressão representativa da diversidade sociocultural existente. Normalmente tem origem histórica longínqua e é praticado por grupos sociais coesos, como a população negra baiana, no caso da capoeira. (B) Incorreta. Apesar de poder ter essa função, não é o fator que gerou reconhecimento da capoeira como patrimônio imaterial. (C) Incorreta. Expressões religiosas podem ser reconhecidas como patrimônio cultural, porém a capoeira é uma expressão sem ligação obrigatória com a religião. (D) Incorreta. Trata-se de uma expressão sociocultural que, a princípio, não tem aspecto que a conecte a práticas ambientalistas.

\item
BNCC: EF09HI26 -- Discutir e analisar as causas da violência contra
populações marginalizadas (negros, indígenas, mulheres, homossexuais,
camponeses, pobres etc.) com vistas à tomada de consciência e à
construção de uma cultura de paz, empatia e respeito às pessoas. (A) Incorreta. O esquema trata da proporção, ou seja, das características apenas daqueles que foram vítimas de assassinato. (B) Correta. A proporção com maioria de negros mostra que essa população está inserida em contextos que a torna vítima preferencial de crimes de assassinato. (C) Incorreta. Inexiste qualquer informação no esquema que relacione o posicionamento econômico aos crimes de assassinato. (D) Incorreta. Inexiste qualquer informação sobre esse dado no esquema apresentado.

\item
BNCC: EF09HI26 -- Discutir e analisar as causas da violência contra
populações marginalizadas (negros, indígenas, mulheres, homossexuais,
camponeses, pobres etc.) com vistas à tomada de consciência e à
construção de uma cultura de paz, empatia e respeito às pessoas. (A) Incorreta. Tal proposta não resolve dois dos problemas apresentados, que são a baixa ocupação de cargos gerenciais por pessoas negras e o menor rendimento médio dessa população. (B) Incorreta. A alternativa não é viável, pois o baixo número de pessoas negras em cargos mais elevados não permitiria que apenas o mérito promovesse uma maior integração econômica e profissional dessa população. (C) Correta. Tal alternativa tem potencial de atacar todos os problemas apresentados, pois a maior qualificação profissional ampliaria as oportunidades para essa população, que também seria beneficiada por programas de reserva de vagas em cargos mais elevados, o que contribuiria também para o aumento do rendimento médio e a melhor utilização do potencial profissional dessa população. (D) Incorreta. Tal alternativa não ataca nenhum dos problemas apresentados, além de ser potencialmente danosa à população negra, que já enfrenta grandes dificuldades para a composição do orçamento doméstico.
\end{enumerate}

\colorsec{Ciências Humanas – Módulo 4 – Treino}

\begin{enumerate}
\item
BNCC: EF09GE08 -- Analisar transformações territoriais, considerando o
movimento de fronteiras, tensões, conflitos e múltiplas regionalidades
na Europa, na Ásia e na Oceania. (A) Correta. Ao aprovar o referendo em prol da independência da região, os catalães demonstram sua territorialidade, ou seja, seu sentimento de pertencimento à Catalunha, e não ao Estado espanhol. (B) Incorreta. A nação catalã, ou seja, o grupo de pessoas com traços étnicos e culturais em comum, existe independentemente do referendo. (C) Incorreta. O referendo compreende a consulta popular sobre determinado tema, mas que não envolve a escolha por partidos políticos ou o seu rechaço em relação a eles. (D) Incorreta. O referendo é uma das etapas dos catalães em busca de sua legitimidade frente ao Estado espanhol, ou seja, a conquista de sua independência. Sua realização e aprovação não resultou, diretamente, na independência da Catalunha, tanto é que não foi considerado legítimo pelo governo espanhol.

\item
(A) Incorreta. A importância de se vacinar é reforçada nas campanhas de
vacinação, além do fato de não se escolher os imunizantes
disponibilizados. Nesse caso, a escolha dos \emph{influencers} não visa
remediar descontentamento com resultados de testes.
(B) Correta. \emph{Influencers} são reconhecidos pela sociedade, ou por
grupos mais específicos, e suas ações muitas vezes servem como estímulos
para pessoas não famosas.
(C) Incorreta. O texto não destaca a repulsa da população pelas campanhas
vacinais.
(D) Incorreta. O texto não destaca o rechaço da população às figuras
públicas.

\item
(A) Incorreta. O Estado brasileiro não está transferindo o monopólio da
força para as mãos do crime organizado.
(B) Incorreta. Na situação descrita, não há reconhecimento da
legitimidade do crime organizado pelo Estado. No caso, houve a imposição
da força pelo crime organizado, que resultou no domínio das atividades
comerciais.
(C) Incorreta. O texto não destaca a aceitação populacional das ações do
crime organizado. Na realidade, o cumprimento das imposições feitas
pelos criminosos se dá para a manutenção da segurança e da integridade
próprias.
(D) Correta. O texto destaca que o crime organizado controla o mercado de
gás em áreas de comunidade, impondo seus preços e locais de compra, à
revelia das determinações do Estado e de sua regulação de preços. A
``obediência'' às regras impostas pelo crime organizado é explicada pela
manutenção da segurança e da integridade dos populares.
\end{enumerate}

\colorsec{Ciências Humanas – Módulo 5 – Treino}

\begin{enumerate}
\item
BNCC: EF09HI23 -- Identificar direitos civis, políticos e sociais
expressos na Constituição de 1988 e relacioná-los à noção de cidadania e
ao pacto da sociedade brasileira de combate a diversas formas de
preconceito, como o racismo. (A) Incorreta. A diminuição da qualidade do ensino público frente ao ensino privado desrespeitaria a igualdade de condições estabelecida pela Constituição. (B) Incorreta. Tal ação desrespeitaria a regra que estabelece que o dinheiro deve ser voltado para o investimento nos serviços públicos. (C) Incorreta. Tal ação contraria a previsão de fornecimento de atendimento em creches para crianças de 0 a 6 anos. (D) Correta. A garantia de atendimento educacional especializado é prevista em lei, sendo portanto uma ação garantida pela Constituição.

\item
BNCC: EF09HI09 -- Relacionar as conquistas de direitos políticos,
sociais e civis à atuação de movimentos sociais. (A) Correta. Em razão da desproporcionalidade entre a população negra total e a população negra matrículada na universidade, é preciso criar medidas que favoreçam a entrada dessa população na universidade, como o apoio financeiro para se manter na universidade. (B) Incorreta. Não se menciona no texto o possível papel da religiosidade negra na desigualdade de acesso à universidade pública, não tendo, portanto, relação com o questionamento apresentado. (C) Incorreta. Universidades públicas são geridas pelos governos, ou seja, pelo Estado, e não por entes privados como empresários. Dessa maneira, a ação proposta pela alternativa é inexequível. (D) Incorreta. O fim das cotas raciais dificultaria ainda mais o acesso da população negra à universidade, pois a população negra não teria nenhum benefício compensatório para fazer frente às desigualdade de condições existente entre a população negra e a população branca. 

\item
BNCC: EF09HI21 -- Identificar e relacionar as demandas indígenas e
quilombolas como forma de contestação ao modelo desenvolvimentista da
ditadura. (A) Correta. As populações indígenas costumam conviver em maior harmonia com o ambiente natural, sendo tidas como populações praticantes de modos de vida sustentáveis. No contexto apresentado, os poderes autoritários do regime militar promoveram o massacre contra a população indígena, de modo que existiu uma oposição entre o modo de vida indígena e a expansão do domínio territorial na ditadura militar. (B) Incorreta. Não se menciona no texto a instalação de indústrias, mas principalmente a abertura de estradas. (C) Incorreta. Como se menciona no texto, o regime militar promoveu massacres contra a população indígena, desrespeitando o direito de essas populações viverem em paz em suas terras. (D) Incorreta. As populações indígenas não são classificados como populações negras.
\end{enumerate}

\colorsec{Ciências Humanas – Módulo 6 – Treino}

\begin{enumerate}
\item
BNCC: EF09GE11 -- Relacionar as mudanças técnicas e científicas
decorrentes do processo de industrialização com as transformações no
trabalho em diferentes regiões do mundo e suas consequências no Brasil. (A) Incorreta. No texto, não se destaca a possibilidade de renovação da mão de obra por conta da introdução de tecnologias no ambiente de trabalho. (B) Incorreta. No texto, não se fala de desemprego estrutural. (C) Correta. Os especialistas mencionados no texto acreditam no potencial que a inteligência artificial tem de reduzir a sobrecarga humana com ações rotineiras, o que abriria espaço para que o ser humano desenvolva outras potencialidades. (D) Incorreta. Pelo contrário, a introdução de tecnologias no ambiente de trabalho tende a reduzir postos de trabalho tidos como manuais.

\item
BNCC: EF09GE11 -- Relacionar as mudanças técnicas e científicas
decorrentes do processo de industrialização com as transformações no
trabalho em diferentes regiões do mundo e suas consequências no Brasil. (A) Correta. Concomitantemente à ampliação da mecanização do campo, percebe-se que ocorre a redução dos postos de trabalho, ou seja, as máquinas têm reduzido a necessidade da mão de obra no campo. (B) Incorreta: No texto, ressalta-se o aumento da produtividade, mas isso não está associado aos trabalhadores em si. (C) Incorreta. Pelo contrário, percebe-se que a mecanização tem sido um dos motivos da diminuição dos postos de trabalho no campo. (D) Incorreta. Pelo contrário, a perda de postos de trabalho tem sido um dos fatores de expulsão da população do meio rural brasileiro.

\item
BNCC: EF09GE12 -- Relacionar o processo de urbanização às transformações
da produção agropecuária, à expansão do desemprego estrutural e ao papel
crescente do capital financeiro em diferentes países, com destaque para
o Brasil.
(A) Correta. O êxodo rural brasileiro foi impulsionado pela mecanização
do campo, que foi retirando postos de trabalho no meio rural, e pela
histórica concentração das terras, que limita as possibilidades de
permanência no campo.
(B) Incorreta. No período, não houve redução da área urbana no país, além
de não ter havido uma política de cercamento agrário.
(C) Incorreta. Embora tenha havido o estímulo da produção de
\emph{commodities} agrícolas no período, especialmente nos anos 1980, a
luta pela reforma agrária compreende uma forma de permanência no espaço
rural, e não de saída.
(D) Incorreta. Embora tenha havido a internacionalização da produção do
campo, não houve o desincentivo à industrialização nacional.
\end{enumerate}

\colorsec{Simulado 1}

\begin{enumerate}
\item
BNCC: EF69LP02 -- Analisar e comparar peças publicitárias variadas
(cartazes, folhetos, outdoor, anúncios e propagandas em diferentes
mídias, spots, jingle, vídeos etc.), de forma a perceber a articulação
entre elas em campanhas, as especificidades das várias semioses e
mídias, a adequação dessas peças ao público-alvo, aos objetivos do
anunciante e/ou da campanha e à construção composicional e estilo dos
gêneros em questão, como forma de ampliar suas possibilidades de
compreensão (e produção) de textos pertencentes a esses gêneros. a) Incorreta. O texto é destinado a quem pode e quer praticar a adoção. b) Correta. O público-alvo é toda pessoa que deseja participar do processo de adoção. c) Incorreta. O texto não é para pessoas que querem dar um filho para a adoção, mas às pessoas que querem adotar uma criança. d) Incorreta. A campanha se destina a quem tem o desejo de adotar uma
criança. O propósito é convencer o leitor a realizar esse ato.

\item
BNCC: EF69LP02 -- Analisar e comparar peças publicitárias variadas
(cartazes, folhetos, outdoor, anúncios e propagandas em diferentes
mídias, spots, jingle, vídeos etc.), de forma a perceber a articulação
entre elas em campanhas, as especificidades das várias semioses e
mídias, a adequação dessas peças ao público-alvo, aos objetivos do
anunciante e/ou da campanha e à construção composicional e estilo dos
gêneros em questão, como forma de ampliar suas possibilidades de
compreensão (e produção) de textos pertencentes a esses gêneros. a) Incorreta. Embora também possa haver referência a isso, a imagem aponta para a adoção de crianças mais velha, menos comum do que a de bebês. b) Incorreta. Não há qualquer relação possível de se estabelecer com a ideia de adoção de crianças ainda não nascidas. c) Correta. As imagens mostram crianças crescidas; além disso, sabe-se que esses processos de adoção são menos comuns. d) Incorreta. Não se menciona a adoção de irmãos, nem ela pode ser lida nas entrelinhas.

\item
BNCC: EF69LP02 -- Analisar e comparar peças publicitárias variadas
(cartazes, folhetos, outdoor, anúncios e propagandas em diferentes
mídias, spots, jingle, vídeos etc.), de forma a perceber a articulação
entre elas em campanhas, as especificidades das várias semioses e
mídias, a adequação dessas peças ao público-alvo, aos objetivos do
anunciante e/ou da campanha e à construção composicional e estilo dos
gêneros em questão, como forma de ampliar suas possibilidades de
compreensão (e produção) de textos pertencentes a esses gêneros. a) Incorreta. Não há incoerência entre texto escrito e imagem, pelo contrário. b) Incorreta. Não se constrói uma contradição entre texto escrito e imagem. c) Incorreta. Há relação próxima entre texto escrito e imagem; as fotos complementam o que o texto escrito diz. d) Correta. Há, de fato, uma complementação. As fotos mostram famílias felizes com seus filhos, mostrando as ideias de laço e de amor mencionadas no texto verbal.

\item
BNCC: EF69LP02 -- Analisar e comparar peças publicitárias variadas
(cartazes, folhetos, outdoor, anúncios e propagandas em diferentes
mídias, spots, jingle, vídeos etc.), de forma a perceber a articulação
entre elas em campanhas, as especificidades das várias semioses e
mídias, a adequação dessas peças ao público-alvo, aos objetivos do
anunciante e/ou da campanha e à construção composicional e estilo dos
gêneros em questão, como forma de ampliar suas possibilidades de
compreensão (e produção) de textos pertencentes a esses gêneros. a) Correta. É somente na leitura desse texto em letras menores que se pode compreender que se trata de Santa Catarina. b) Incorreta. Não há qualquer característica geográfica apresentada no anúncio, nem de Santa Catarina, nem de outro estado. c) Incorreta. O slogan é geral e não faz referência a Santa Catarina. d) Incorreta. O texto em letras brancas é geral e não faz referência a Santa Catarina.

\item
BNCC: EF69LP02 -- Analisar e comparar peças publicitárias variadas
(cartazes, folhetos, outdoor, anúncios e propagandas em diferentes
mídias, spots, jingle, vídeos etc.), de forma a perceber a articulação
entre elas em campanhas, as especificidades das várias semioses e
mídias, a adequação dessas peças ao público-alvo, aos objetivos do
anunciante e/ou da campanha e à construção composicional e estilo dos
gêneros em questão, como forma de ampliar suas possibilidades de
compreensão (e produção) de textos pertencentes a esses gêneros.. a) Incorreta. Os símbolos são de instituições, não são brasões de famílias. b) Correta. Trata-se, de fato, de órgãos e instituições variados que apoiam a campanha. c) Incorreta. As instituições representadas não acolhem crianças. d) Incorreta. Não há esse tipo de financiamento.

\item
BNCC: EF69LP44 -- Inferir a presença de valores sociais, culturais e
humanos e de diferentes visões de mundo, em textos literários,
reconhecendo nesses textos formas de estabelecer múltiplos olhares sobre
as identidades, sociedades e culturas e considerando a autoria e o
contexto social e histórico de sua produção.
 a) Incorreta. Os jogos teatrais e a improvisação em si são encenados de modos diferentes, não tendo uma única forma de serem realizados. b) Incorreta. Inexiste uma tentativa de se estabelecer uma forma correta, mas de ajudae na preparação da peça. c) Correta. É possível afirmar que a função dos jogos teatrais e da improvisação é auxiliar a preparação da peça, pois ajudam na composição dos elementos do espetáculo teatral. d) Incorreta. Nessas situações, pode não haver a presença de plateia.

\item
BNCC: EF69LP44 -- Inferir a presença de valores sociais, culturais e
humanos e de diferentes visões de mundo, em textos literários,
reconhecendo nesses textos formas de estabelecer múltiplos olhares sobre
as identidades, sociedades e culturas e considerando a autoria e o
contexto social e histórico de sua produção. a) Incorreta. Há, no conteúdo do texto, respostas para a questão proposta. b) Incorreta. Não há passo a passo, nem no título, nem no corpo do texto. Há, apenas, diretrizes. c) Incorreta. Não há dados estatísticos no texto, nem isso se apresenta na pergunta do título. d) Correta. De fato, ao se referir diretamente ao leitor, a questão o envolve e o faz querer ler o texto.

\item
SAEB: Analisar formas, gêneros e estilos distintos de artes visuais e
dança, em diferentes contextos, por meio de seus elementos
constitutivos.
BNCC: EF69AR01 -- Pesquisar, apreciar e analisar formas distintas das
artes visuais tradicionais e contemporâneas, em obras de artistas
brasileiros e estrangeiros de diferentes épocas e em diferentes matrizes
estéticas e culturais, de modo a ampliar a experiência com diferentes
contextos e práticas artístico-visuais e cultivar a percepção, o
imaginário, a capacidade de simbolizar e o repertório imagético.
a) Incorreta. O texto não descreve Vincent van Gogh, pintor pós-impressionista neerlandês.
b) Incorreta. O texto não descreve Ernst Kirchner, pintor expressionista alemão.
c) Correta. O texto descreve a obra de Edvard Munch, artista que pintou ``O grito''.
d) Incorreta. O texto não descreve Wassily Kandinsky, pintor expressionista russo.

\item
SAEB: Analisar formas, gêneros e estilos distintos de artes visuais e
dança, em diferentes contextos, por meio de seus elementos
constitutivos.
BNCC: EF69AR01 -- Pesquisar, apreciar e analisar formas distintas das
artes visuais tradicionais e contemporâneas, em obras de artistas
brasileiros e estrangeiros de diferentes épocas e em diferentes matrizes
estéticas e culturais, de modo a ampliar a experiência com diferentes
contextos e práticas artístico-visuais e cultivar a percepção, o
imaginário, a capacidade de simbolizar e o repertório imagético.
a) Correta. O pintor Vincent Van Gogh retratou a si mesmo nessa obra.
b) Incorreta. O gênero natureza-morta refere-se à representação de
  objetos inanimados.
c) Incorreta. O gênero paisagem refere-se à representação de um local.
d) Incorreta. O gênero retrato refere-se à representação de uma figura
  individual ou de um grupo feita por uma pessoa não retratada.

\item
SAEB: Analisar a função do tema como projeto integrador das diferentes
linguagens artísticas.
BNCC: EF69AR03 -- Analisar situações nas quais as linguagens das artes
visuais se integram às linguagens audiovisuais (cinema, animações,
vídeos etc.), gráficas (capas de livros, ilustrações de textos diversos
etc.), cenográficas, coreográficas, musicais etc.
a) Incorreta. Segundo a descrição do texto, a artista não utilizou o
  gesto.
b) Incorreta. Segundo a descrição do texto, a artista não utilizou
  instrumento musical.
c) Correta. Luz e som são materialidades presentes na obra.
d) Incorreta. Segundo a descrição do texto, a artista não utilizou
  pincel.

\item
SAEB: Identificar o assunto de um texto, a partir de sua organização, de
palavras cognatas e/ou de palavras formas por afixação. BNCC: EF06LI06
-- Antecipar o sentido global de textos em língua inglesa por
inferências, com base em leitura rápida, observando títulos, primeiras e
últimas frases de parágrafos e palavras-chave repetidas.
a) Incorreta. O texto trata da Antártida. b) Incorreta. O texto não
apresenta formas de impedirmos o processo mencionado. c) Correta. O tema
do texto pode ser encontrado com base no trecho ``The results showed the
glacier was more in danger of becoming unstable that previously
thought''. d) Incorreta. O texto não cita animais da região.

\item
SAEB: Localizar informações específicas, a partir de diferentes
objetivos de leitura, em textos em língua inglesa. BNCC: EF06LI08 --
Identificar o assunto de um texto, reconhecendo sua organização textual
e palavras cognatas.
a) Incorreta. O texto menciona tempestades em outros lugares do mundo.
b) Incorreta. O texto menciona efeitos contrários. c) Incorreta. O texto
menciona Barcelona para comparar as duas cidades. d) Correta. Podemos
encontrar a resposta no trecho ``London could be facing severe
drought''.

\item
SAEB: Identificar a finalidade de um texto em língua inglesa, com base
em sua estrutura, organização textual, pistas gráficas e/ou aspectos
linguísticos. BNCC: EF06LI07 -- Formular hipóteses sobre a finalidade de
um texto em língua inglesa, com base em sua estrutura, organização
textual e pistas gráficas.
a) Correta. O texto apresenta personagens e diálogos. b) Incorreta. O
texto não apresenta argumentos. c) Incorreta. O texto não expõe
informações recentes ao leitor. d) Incorreta. O texto não apresenta
instruções para o leitor.

\item
BNCC: EF09GE18 -- Identificar e analisar as cadeias industriais e de
inovação e as consequências dos usos de recursos naturais e das diferentes fontes de energia (tais como termoelétrica, hidrelétrica, eólica e nuclear) em diferentes países.
a) Incorreta. A extração do carvão não produz massiva emissão de fumaça.
b) Incorreta. As ilhas de calor são fenômenos associados a grandes
  cidades.
c) Incorreta. Ao contrário da mineração de metais, não se utiliza água
  para a extração de carvão mineral.
d) Correta. Para extrair o carvão, toda a superfície acima da mina precisa
  ser removida.

\item
BNCC: EF09GE18 -- Identificar e analisar as cadeias industriais e de
inovação e as consequências dos usos de recursos naturais e das diferentes fontes de energia (tais como termoelétrica, hidrelétrica, eólica e nuclear) em diferentes países.
a) Incorreta. O escoamento é orientado pela força da gravidade, e não pela
  energia solar.
b) Incorreta. A precipitação é o efeito da perda de calor por parte das
  nuvens, pois, ao perder calor, o vapor de água condensa-se em pequenas
  gotículas formando nuvens e posteriormente chuva.
c) Incorreta. Ao perder calor, o vapor de água condensa-se em pequenas
  gotículas formando nuvens e posteriormente chuva;
d) Correta. A evaporação é o efeito do ganho de calor (por meio da absorção da energia solar) por parte da água
  líquida, o que gera sua evaporação.

\item
BNCC: EF09GE18 -- Identificar e analisar as cadeias industriais e de
inovação e as consequências dos usos de recursos naturais e das
diferentes fontes de energia (tais como termoelétrica, hidrelétrica,
eólica e nuclear) em diferentes países.
a) Correta. O texto demonstra o aumento da produção de energia eólica, que é de uma fonte renovável.
b) Incorreta. O texto demonstra que o Brasil tem muito potencial para energia eólica.
c) Incorreta. O texto demonstra que a energia eólica ainda corresponde a uma parcela pequena da matriz energética brasileira.
d) Incorreta. O texto demonstra o aumento de produção de energia de uma fonte renovável.

\item
BNCC: EF09HI03 -- Identificar os mecanismos de inserção dos
negros na sociedade brasileira pós-abolição e avaliar os seus
resultados. a)  Incorreta. As políticas de cotas não versam sobre a população branca; apenas beneficiam a população negra e mais pobre. b)  Incorreta. Ainda que fosse uma política de privilégios, a população foco não é minoritária, pois os negros compõem maioria demográfica. c)  Correta. As leis de cotas buscam reparar injustiças históricas cometidas contra a população negra, que, por exemplo, sequer teve direito a trabalho livre após o fim da escravidão, o que causou inúmeras consequências relacionadas à exclusão econômica dessa população. d)  Incorreta. Essa alternativa possui uma abordagem racista ao afirmar que a população negra é menos capaz que a população branca.

\item
BNCC: EF09HI04 -- Discutir a importância da participação da
população negra na formação econômica, política e social do Brasil. a) Incorreta. Atualmente existe extensa estrutura jurídica que embasa a organização dos governos, não sendo este o fator causador da baixa expressão política da população periférica. b) Correta. Por serem as mais afetadas e, ao mesmo tempo, as menos representadas, as pautas da população negra são pouco discutidas no debate público; assim, a criação de canais de expressão e participação popular daria mais evidência às pautas que afetam a população negra e periférica. c) Incorreta. Os mecanismos de desastres ambientais já são bem conhecidos, e o problema repousa nas atitudes que deveriam ser, mas não são tomadas para evitá-los, o que se explica em parte por racismo institucional. d) Incorreta. A questão pede uma contribuição para a solução do conjunto dos problemas ambientais que afetam a população negra, e não intervenções pontuais.
\end{enumerate}

\colorsec{Simulado 2}

\begin{enumerate}
\item
BNCC: EF69LP16 -- Analisar e utilizar as formas de composição dos
gêneros jornalísticos da ordem do relatar, tais como notícias (pirâmide
invertida no impresso X blocos noticiosos hipertextuais e
hipermidiáticos no digital, que também pode contar com imagens de vários
tipos, vídeos, gravações de áudio etc.), da ordem do argumentar, tais
como artigos de opinião e editorial (contextualização, defesa de
tese/opinião e uso de argumentos) e das entrevistas: apresentação e
contextualização do entrevistado e do tema, estrutura pergunta e
resposta etc. a) Incorreta. As perguntas e as respostas são introduzidas diretamente no texto principal, o que não ocorre no texto de apresentação. b) Incorreta. Os argumentos da entrevistada aparecem nas próprias respostas. c) Correta. O gênero textual entrevista normalmente apresenta uma breve apresentação do entrevistado no primeiro parágrafo, destinado ao leitor. d) Incorreta. O texto apresenta apenas a história da entrevistada, não mostrando os propósitos da entrevista.

\item
BNCC: EF69LP16 -- Analisar e utilizar as formas de composição dos
gêneros jornalísticos da ordem do relatar, tais como notícias (pirâmide
invertida no impresso X blocos noticiosos hipertextuais e
hipermidiáticos no digital, que também pode contar com imagens de vários
tipos, vídeos, gravações de áudio etc.), da ordem do argumentar, tais
como artigos de opinião e editorial (contextualização, defesa de
tese/opinião e uso de argumentos) e das entrevistas: apresentação e
contextualização do entrevistado e do tema, estrutura pergunta e
resposta etc.
 a) Incorreta. Não há baterias separadas de perguntas e de respostas. b) Incorreta. Não aparece, no texto, a estratégia da diferenciação por cores. c) Correta. Há intercalação entre perguntas e respostas e elas são marcadas com as letras P e R. d) Incorreta. As respostas são, naturalmente, dadas pela entrevistada.

\item
BNCC: EF69LP16 -- Analisar e utilizar as formas de composição dos
gêneros jornalísticos da ordem do relatar, tais como notícias (pirâmide
invertida no impresso X blocos noticiosos hipertextuais e
hipermidiáticos no digital, que também pode contar com imagens de vários
tipos, vídeos, gravações de áudio etc.), da ordem do argumentar, tais
como artigos de opinião e editorial (contextualização, defesa de
tese/opinião e uso de argumentos) e das entrevistas: apresentação e
contextualização do entrevistado e do tema, estrutura pergunta e
resposta etc. a) Incorreta. Não aparecem fatos concretos do mundo nessa argumentação. b) Correta. O personagem D. Quixote serve, de fato, como exemplo, na literatura, de presonagem que foi além do romance a que pertencia. c) Incorreta. D. Quixote é usado como personagem de exemplo, simplesmente, não como autoridade no assunto. d) Incorreta. A citação do personagem não funciona como contra-argumentação.

\item
BNCC: EF69LP43 -- Identificar e utilizar os modos de introdução de
outras vozes no texto -- citação literal e sua formatação e paráfrase
--, as pistas linguísticas responsáveis por introduzir no texto a
posição do autor e dos outros autores citados (``Segundo X; De acordo
com Y; De minha/nossa parte, penso/amos que''...) e os elementos de
normatização (tais como as regras de inclusão e formatação de citações e
paráfrases, de organização de referências bibliográficas) em textos
científicos, desenvolvendo reflexão sobre o modo como a
intertextualidade e a retextualização ocorrem nesses textos.
 a) Incorreta. Paráfrases também podem apresentar verbos de elocução, portanto a presença do verbo não garante que seja uma citação direta. b) Correta. Acontece a citação direta quando a fala é transcrita de forma literal, estando entre aspas e com a identificação de quem deu a declaração, como ocorre no trecho ``'No primeiro ano após um infarto, um a cada cinco pacientes pode sofrer novo infarto ou, mesmo, morte súbita. Muitas tentativas já foram feitas para identificar quem é esse paciente que corre mais risco'', diz o cardiologista.``. c) Incorreta. As paráfrases que são compostas pelas palavras do autor do texto. d) Incorreta. A citação direta é a transcrição da fala literal de alguém.

\item
BNCC: EF69LP43 -- Identificar e utilizar os modos de introdução de
outras vozes no texto -- citação literal e sua formatação e paráfrase
--, as pistas linguísticas responsáveis por introduzir no texto a
posição do autor e dos outros autores citados (``Segundo X; De acordo
com Y; De minha/nossa parte, penso/amos que''...) e os elementos de
normatização (tais como as regras de inclusão e formatação de citações e
paráfrases, de organização de referências bibliográficas) em textos
científicos, desenvolvendo reflexão sobre o modo como a
intertextualidade e a retextualização ocorrem nesses textos.
 a) Incorreta. Há, sim, logo no início do texto citado, a menção a ganhos econômicos. b) Incorreta. A ideia é evitar que esses eventos ocorram no ano que já sucede um, sem que se fale em evitar o já ocorrido. c) Incorreta. Não há menção a dados numéricos no texto. d) Correta. A menção a uma potencial economia de gastos aparece bem no início do texto, em destaque.

\item
BNCC: EF89LP05 -- Analisar o efeito de sentido produzido pelo uso, em
textos, de recurso a formas de apropriação textual (paráfrases,
citações, discurso direto, indireto ou indireto livre).
a) Incorreta. O trecho não expõe as ideias do especialista de forma
indireta, mas sim de forma direta.
b) Incorreta. O trecho, além de não apontar o distanciamento do pastor
como a forma adequada de isolamento, não produz a comparação com o ano
de 2020, apenas apresenta que o distanciamento social foi ``um modo de
reduzir os contágios'' ``em grande parte das epidemias da história da
humanidade''.
c) Correta. A transcrição da fala do especialista ``\,`Às vezes acontece
isso', explica José Prieto, catedrático de microbiologia da Universidade
Complutense de Madri'' dá o respaldo necessário às informações presentes
na reportagem -- ``Quando um determinado número de pessoas já superou a
doença e está imune a ela, o contágio fica mais difícil, e a enfermidade
míngua'' -- dando credibilidade ao que foi informado.
d) Incorreta. Não há a exposição de dados históricos, mas de um fato que
aconteceu no passado e sua relação no combate das epidemias na história
da humanidade.

\item
BNCC: EF89LP05 -- Analisar o efeito de sentido produzido pelo uso, em
textos, de recurso a formas de apropriação textual (paráfrases,
citações, discurso direto, indireto ou indireto livre).
a) Incorreta. Embora personagens sejam citados de forma inespecífica,
seus nomes não aparecem.
b) Correta. Trata-se de um paralelo entre a gripe de 1918 e a recente
pandemia de COVID-19.
c) Incorreta. Não há depoimentos de pessoas que vivenciaram a recente
pandemia.
d) Incorreta. Não se selecionaram, de forma proposital, palavras raras
da língua.

\item
SAEB: Analisar formas, gêneros e estilos distintos de música e teatro
em diferentes contextos, por meio de seus elementos constitutivos.
BNCC: EF69AR24 -- Reconhecer e apreciar artistas e grupos de teatro
brasileiros e estrangeiros de diferentes épocas, investigando os modos
de criação, produção, divulgação, circulação e organização da atuação
profissional em teatro.
a) Incorreta. As técnicas desenvolvidas ao longo da história do Teatro
do Oprimido encontram-se representadas nos ramos da árvore.
b) Incorreta. As técnicas consideradas como base da metodologia
encontram-se representadas no tronco da árvore.
c) Incorreta. As ações sociais concretas e continuadas encontram-se
representadas no topo da árvore.
d) Correta. No solo, em que ficam as raízes, encontra-se representado o conjunto de saberes,
composto da história, da filosofia, da política e da participação.

\item
SAEB: Identificar diferentes formas de registro das artes por meio de
notação ou procedimentos e técnicas de áudio e audiovisual.
BNCC: EF69AR22 -- Explorar e identificar diferentes formas de registro
musical (notação musical tradicional, partituras criativas e
procedimentos da música contemporânea), bem como procedimentos e
técnicas de registro em áudio e audiovisual.
a) Correta. A notação convencional se utiliza de cinco pautas, quatro espaços e notas musicais.
b) Incorreta. A notação musical textual não se assemelha à convencional partitura.
c) Incorreta. As tablaturas são exemplos não convencionais de notação musical.
d) Incorreta. As tablaturas são exemplos não convencionais de notação musical.

\item
SAEB: Analisar o papel dos profissionais e a utilização dos
equipamentos culturais no sistema de produção e circulação das artes
visuais, dança, música e teatro.
BNCC: EF69AR18 -- Reconhecer e apreciar o papel de músicos e grupos de
música brasileiros e estrangeiros que contribuíram para o
desenvolvimento de formas e gêneros musicais.
a) Incorreta. Apesar de o gênero musical samba ser muito tocado no Carnaval brasileiro, a banda Chiclete 
com banana se projetou nacionalmente com a introdução do axé em suas apresentações.
b) Incorreta. O sertanejo, gênero musical brasileiro, não faz parte da
  tradição do Carnaval e não é o estilo de música tocado pela banda
Chiclete com banana.
c) Incorreta. O gênero musical funk, apesar de muito tocado no Brasil,
 tem origens em comunidades afro-americanas, ou seja, não se trata de um gênero desenvolvido inicialmente no
  Brasil.
d) Correta. A banda Chiclete com banana iniciou sua trajetória tocando
  todo estilo de música, do rock ao forró, mas foi com o axé que ela se
  projetou nacionalmente.

\item
SAEB: Avaliar a presença, no mundo globalizado, da língua inglesa e/ou
de produtos culturais de países de língua inglesa. BNCC: EF07LI21 --
Analisar o alcance da língua inglesa e os seus contextos de uso no mundo
globalizado.
a) Incorreta. O texto trata da mídia latino-americana. b) Incorreta. No
texto, afirma-se exatamente o contrário. c) Incorreta. O texto não traz
afirmações sobre o ensino de inglês no Brasil. d) Correta. Essa
afirmação pode ser encontrada no trecho ``English is the global
communication language par excellence''.

\item
SAEB: Contrapor perspectivas sobre um mesmo assunto em textos em língua
inglesa. BNCC: EF09LI06 -- Distinguir fatos de opiniões em textos
argumentativos da esfera jornalística.
a) Incorreta. O texto menciona discordâncias a respeito do acordo. b)
Incorreta. O texto menciona o evento como um encontro internacional
importante. c) Incorreta. O texto não faz menção a líderes que não
compareceram ao evento. d) Correta. Essa informação está no trecho ``US
President Donald Trump withdrew from the agreement, but his successor
Joe Biden rejoined on his first day in office in January 2021''.

\item
SAEB: Distinguir fatos de opiniões em textos em língua inglesa. BNCC:
EF09LI06 -- Distinguir fatos de opiniões em textos argumentativos da
esfera jornalística.
a) Incorreta. Apesar de se falar em um estádio, o texto não menciona os
treinos de um time. b) Correta. Essa afirmação pode ser encontrada no
trecho ``is to be wrapped in a solar membrane to reduce carbon
emissions''. c) Incorreta. O texto não menciona essa possibilidade. d)
Incorreta. O texto apenas menciona a cidade de Londres, sem falar de seu
crescimento.

\item
BNCC: EF09GE05 -- Analisar fatos e situações para compreender a
integração mundial (econômica, política e cultural), comparando as
diferentes interpretações: globalização e mundialização.
a) Correta. O arado começou a ser utilizado para diminuir o tempo gasto com o preparo e a semeadura do solo; assim  constitui um instrumento que
otimiza o trabalho humano, isto é, a transformação de recursos naturais
em produtos com valor de uso. b) Incorreta. O arado não foi usado para domestificar animais, mas começou a ser empregado com o auxílio de animais domesticados.
c) Incorreta. O arado é um instrumento de preparação do solo para plantio.
d) Incorreta. O arado é empregado em terrenos já disponibilizados ao
plantio.

\item
BNCC: EF09GE05 -- Analisar fatos e situações para compreender a
integração mundial (econômica, política e cultural), comparando as
diferentes interpretações: globalização e mundialização.
a) Incorreta. O setor de serviços constitui destaque interno no Brasil.
b) Incorreta. A desindustrialização demonstra que o Brasil passa por um
  processo de perda de tecnologia, seja por perder empresas detentoras
  de tecnologias, seja por não investir na tecnologia, setor que anda
  lado a lado com o desenvolvimento industrial.
c) Correta. O recorde de exportações agrícolas revela o pleno
  desenvolvimento do setor no país, o que mostrar que o Brasil tem se
  especializado na produção de itens básicos classificados como
  matéria-prima (\textit{commodities}), e essa configuração econômica
  contribui para o país ficar dependente de tecnologia externa, já que o
  desenvolvimento tecnológico, mesmo para máquinas agrícolas, parte do
  setor industrial.
d) Incorreta. Em um dos textos, menciona-se a perda crônica de empresas
  industriais.

\item
BNCC: EF09GE06 -- Associar o critério de divisão do mundo em Ocidente e
Oriente com o Sistema Colonial implantado pelas potências europeias.
a) Incorreta. Não apenas a ausência da figura do monarca é criticada, mas
também a ausência da fé cristã e da lei europeia.
b) Incorreta. O centro da visão colonialista não estava na língua falada,
mas nas características culturais diferenciadas dos povos indígenas.
c) Incorreta. A fala do cronista do português não se caracteriza por um tom de cuidado com os indígenas.
d) Correta. A monarquia, a fé cristã como instituição de estado e
a legislação portuguesa são encaradas como base de uma civilização
correta na fala de Pero Magalhães, motivo pelo qual ele critica a ausência
desses elementos entre os indígenas.

\item
BNCC: EF09GE03 -- Identificar diferentes manifestações
culturais de minorias étnicas como forma de compreender a multiplicidade
cultural na escala mundial, defendendo o princípio do respeito às
diferenças. a)  Correta. A construção de projetos culturais de matrizes afro-brasileiras demonstra que as manifestações culturais promovidas pela associação têm como objetivo resgatar e consolidar uma identidade cultural própria da população negra. b)  Incorreta. Os projetos desenvolvidos não têm como foco o divertimento, e sim a construção de práticas que representam a socialização da população negra em uma matriz cultural de origem africana. c)  Incorreta. A associação busca a naturalização e a cotidianização das expressões culturais negras, o que promoveria maior integração dessas práticas com a cultura nacional. d)  Incorreta. O objetivo da associação é a valorização das expressões culturais negras, que não são utilizadas como uma forma de contestação da ordem, mas como instrumento de consolidação de práticas socioculturais de origem africana

\item
BNCC: EF09HI26 -- Discutir e analisar as causas da violência
contra populações marginalizadas (negros, indígenas, mulheres,
homossexuais, camponeses, pobres etc.) com vistas à tomada de
consciência e à construção de uma cultura de paz, empatia e respeito às
pessoas. a)  Incorreta. Não há menção no texto à invasão de terras públicas como causa de violência. b)  Correta. As mortes ocorridas em área previamente destinada ao extrativismo tradicional demonstram que disputas pelos recursos naturais são causas evidentes da violência rural. c)  Incorreta. Não há menção no texto à prática de roubo da produção agropecuária como causa de violência. d)  Incorreta. No texto, fica claro que a área onde ocorreram as mortes era destinada ao extrativismo, ou seja, existe regulamentação da atividade.
\end{enumerate}

\colorsec{Simulado 3}

\begin{enumerate}
\item
BNCC: EF69LP20 -- Identificar, tendo em vista o contexto de produção, a
forma de organização dos textos normativos e legais, a lógica de
hierarquização de seus itens e subitens e suas partes: parte inicial
(título -- nome e data -- e ementa), blocos de artigos (parte, livro,
capítulo, seção, subseção), artigos (caput e parágrafos e incisos) e
parte final (disposições pertinentes à sua implementação) e analisar
efeitos de sentido causados pelo uso de vocabulário técnico, pelo uso do
imperativo, de palavras e expressões que indicam circunstâncias, como
advérbios e locuções adverbiais, de palavras que indicam generalidade,
como alguns pronomes indefinidos, de forma a poder compreender o caráter
imperativo, coercitivo e generalista das leis e de outras formas de
regulamentação. a) Incorreta. As entidades populares fazem parte dos órgãos que reivindicam melhorias no Código Florestal. b) Incorreta. As mudanças exigidas são referentes à agricultura familiar e camponesa, não ao meio ambiente. c) Correta. O texto busca exigir melhorias para o Código Florestal, a fim de mostrar que mesmo que ele atenda às preocupações com o uso sustentável do meio ambiente, há espaços para melhorias relacionadas à agricultura familiar e camponesa. d) Incorreta. O modo de prover alimentos é o que o manifesto procura proteger.

\item
BNCC: EF69LP20 -- Identificar, tendo em vista o contexto de produção, a
forma de organização dos textos normativos e legais, a lógica de
hierarquização de seus itens e subitens e suas partes: parte inicial
(título -- nome e data -- e ementa), blocos de artigos (parte, livro,
capítulo, seção, subseção), artigos (caput e parágrafos e incisos) e
parte final (disposições pertinentes à sua implementação) e analisar
efeitos de sentido causados pelo uso de vocabulário técnico, pelo uso do
imperativo, de palavras e expressões que indicam circunstâncias, como
advérbios e locuções adverbiais, de palavras que indicam generalidade,
como alguns pronomes indefinidos, de forma a poder compreender o caráter
imperativo, coercitivo e generalista das leis e de outras formas de
regulamentação. a) Correta. As leis costumam, de fato, ser identificadas por um número e pela data de sua publicação. b) Incorreta. A notícia foi redigida, muito provavelmente, pouco tempo antes de sua publicação -- em um tempo distante do ano de 1965. c) Incorreta. Pelo contrário, a data de publicação mostra que a lei está vigente já há muitos anos. d) Incorreta.O número da lei é uma espécie de código, mas a data não faz parte disso.

\item
BNCC: EF69LP27 -- Analisar a forma composicional de textos pertencentes
a gêneros normativos/ jurídicos e a gêneros da esfera política, tais
como propostas, programas políticos (posicionamento quanto a diferentes
ações a serem propostas, objetivos, ações previstas etc.), propaganda
política (propostas e sua sustentação, posicionamento quanto a temas em
discussão) e textos reivindicatórios: cartas de reclamação, petição
(proposta, suas justificativas e ações a serem adotadas) e suas marcas
linguísticas, de forma a incrementar a compreensão de textos
pertencentes a esses gêneros e a possibilitar a produção de textos mais
adequados e/ou fundamentados quando isso for requerido.
a) Incorreta. O texto não é reflexivo.
b) Incorreta. A linguagem é impessoal.
c) Correta. A estrutura dos textos normativos presente na organização do
conteúdo, como exemplificado em ``atentamente o disposto nos itens 1. ao
12., e suas alíneas, deste Capítulo'' confere ao texto o caráter
normativo característico dos editais.
d) Incorreta. O texto é formal.

\item
BNCC: EF69LP27 -- Analisar a forma composicional de textos pertencentes
a gêneros normativos/ jurídicos e a gêneros da esfera política, tais
como propostas, programas políticos (posicionamento quanto a diferentes
ações a serem propostas, objetivos, ações previstas etc.), propaganda
política (propostas e sua sustentação, posicionamento quanto a temas em
discussão) e textos reivindicatórios: cartas de reclamação, petição
(proposta, suas justificativas e ações a serem adotadas) e suas marcas
linguísticas, de forma a incrementar a compreensão de textos
pertencentes a esses gêneros e a possibilitar a produção de textos mais
adequados e/ou fundamentados quando isso for requerido.
a) Incorreta. Não se fala sobre legibilidade e os familiares são
proibidos de levar documentos, não de estar no espaço.
b) Correta. Trata-se, de fato, dos dois pontos em destaque como
proibições no texto.
c) Incorreta. É natural que os candidatos não possam se atrasar, mas
isso não está em foco no texto. Além disso, os candidatos fazem, sim, as
duas provas.
d) Incorreta. Os candidatos só podem participar do processo se estiverem
no local de provas e a comunicação entre candidatos não é um foco do
texto.

\item
BNCC: EF69LP47 - Analisar, em textos narrativos ficcionais, as
diferentes formas de composição próprias de cada gênero, os recursos
coesivos que constroem a passagem do tempo e articulam suas partes, a
escolha lexical típica de cada gênero para a caracterização dos cenários
e dos personagens e os efeitos de sentido decorrentes dos tempos
verbais, dos tipos de discurso, dos verbos de enunciação e das
variedades linguísticas (no discurso direto, se houver) empregados,
identificando o enredo e o foco narrativo e percebendo como se estrutura
a narrativa nos diferentes gêneros e os efeitos de sentido decorrentes
do foco narrativo típico de cada gênero, da caracterização dos espaços
físico e psicológico e dos tempos cronológico e psicológico, das
diferentes vozes no texto (do narrador, de personagens em discurso
direto e indireto), do uso de pontuação expressiva, palavras e
expressões conotativas e processos figurativos e do uso de recursos
linguístico-gramaticais próprios a cada gênero narrativo.
a) Incorreta. O texto serve para um objetivo diferente de simpelsmente
informar. Trata-se de texto literário. b) Incorreta. O texto não tem o
objetivo de mostrar regras ou dar instruções. c) Correta. Os elementos
essenciais da narração -- como tempo, espaço e personagens -- estão
presentes. d) Incorreta. O texto não tem o objetivo de comprovar uma
tese por meio de argumentos.

\item
BNCC: EF69LP47 - Analisar, em textos narrativos ficcionais, as
diferentes formas de composição próprias de cada gênero, os recursos
coesivos que constroem a passagem do tempo e articulam suas partes, a
escolha lexical típica de cada gênero para a caracterização dos cenários
e dos personagens e os efeitos de sentido decorrentes dos tempos
verbais, dos tipos de discurso, dos verbos de enunciação e das
variedades linguísticas (no discurso direto, se houver) empregados,
identificando o enredo e o foco narrativo e percebendo como se estrutura
a narrativa nos diferentes gêneros e os efeitos de sentido decorrentes
do foco narrativo típico de cada gênero, da caracterização dos espaços
físico e psicológico e dos tempos cronológico e psicológico, das
diferentes vozes no texto (do narrador, de personagens em discurso
direto e indireto), do uso de pontuação expressiva, palavras e
expressões conotativas e processos figurativos e do uso de recursos
linguístico-gramaticais próprios a cada gênero narrativo.
a) Incorreta. Esse verso simplesmente compõe a descrição do ambiente. b)
Incorreta. Esse verso trata do passado dos amigos. c) Correta. Esse
verso atrela o valor das coisas ao percurso, não à chegada. d)
Incorreta. Esse verso fala de desejos futuros, que extrapolam,
inclusive, o tempo do poema.

\item
BNCC: EF69LP47 - Analisar, em textos narrativos ficcionais, as
diferentes formas de composição próprias de cada gênero, os recursos
coesivos que constroem a passagem do tempo e articulam suas partes, a
escolha lexical típica de cada gênero para a caracterização dos cenários
e dos personagens e os efeitos de sentido decorrentes dos tempos
verbais, dos tipos de discurso, dos verbos de enunciação e das
variedades linguísticas (no discurso direto, se houver) empregados,
identificando o enredo e o foco narrativo e percebendo como se estrutura
a narrativa nos diferentes gêneros e os efeitos de sentido decorrentes
do foco narrativo típico de cada gênero, da caracterização dos espaços
físico e psicológico e dos tempos cronológico e psicológico, das
diferentes vozes no texto (do narrador, de personagens em discurso
direto e indireto), do uso de pontuação expressiva, palavras e
expressões conotativas e processos figurativos e do uso de recursos
linguístico-gramaticais próprios a cada gênero narrativo.
a) Incorreta. Há pouca luz na descrição e aenergia não é um ponto em
destaque. b) Incorreta. Não se vê essa dinãmica e essa aceleração nos
versos. c) Incorreta. Não se lê, nos versos, a ideia de leveza e o
suspense criado é o oposto da despreocupação. d) Correta.Como um poema
de narrativa de mistério, o ambiente descrito é sombrio e obscuro, para
manter o clima de teror e suspense.

\item
SAEB: Avaliar produções que inter-relacionam diferentes linguagens
artísticas.
BNCC: EF69AR34 -- Analisar e valorizar o patrimônio cultural, material e
imaterial, de culturas diversas, em especial a brasileira, incluindo
suas matrizes indígenas, africanas e europeias, de diferentes épocas, e
favorecendo a construção de vocabulário e repertório relativos às
diferentes linguagens artísticas.
a) Incorreta. O Círio de Nossa Senhora de Nazaré é uma celebração
religiosa de Belém, no Pará.
b) Correta. Essa é uma das descrições possíveis para a Festa do Divino
Espírito Santo em Pirenópolis, em Goiás.
c) Incorreta. A Festa do Bumba Meu Boi, com predominância no Norte e no
Nordeste do Brasil, refere-se às manifestações culturais e religiosas em
torno da figura do boi.
d) Incorreta. A Festa do Senhor do Bonfim é típica da cultura e da vida
social em Salvador, na Bahia, e articula duas matrizes religiosas, a
católica e a afro-brasileira.

\item
SAEB: Avaliar o papel das diversas linguagens artísticas no
questionamento de estereótipos e preconceitos.
BNCC: EF69AR33 -- Analisar aspectos históricos, sociais e políticos da
produção artística, problematizando as narrativas eurocêntricas e as
diversas categorizações da arte (arte, artesanato, folclore, design
etc.).
a) Incorreta. O estereótipo cultural está associado a culturas, etnias e
  raças, e esse não é caso relatado no texto.
b) Correta. Trata-se de um estereótipo de gênero, pois diz respeito à
  generalização sobre atributos e características que homens e mulheres
  possuem ou deveriam possuir.
c) Incorreta. O estereótipo de beleza diz respeito ao modelo padrão de
  beleza incutido na mente sobre como devem ser os aspectos físicos das
  pessoas consideradas bonitas.
d) Incorreta. O estereótipo socioeconômico diz respeito,
  principalmente, à classe social a que uma pessoa pertence.

\item
SAEB: Avaliar nas linguagens artísticas a diversidade do patrimônio
cultural da humanidade (material e imaterial), em especial o brasileiro,
a partir de suas diferentes matrizes.
BNCC: EF69AR34 -- Analisar e valorizar o patrimônio cultural, material e
imaterial, de culturas diversas, em especial a brasileira, incluindo
suas matrizes indígenas, africanas e europeias, de diferentes épocas, e
favorecendo a construção de vocabulário e repertório relativos às
diferentes linguagens artísticas.
a) Incorreta. Ouro Preto é um sítio urbano e a primeira cidade brasileira
  a receber o título de patrimônio cultural mundial.
b) Correta. O Pantanal foi inscrito pela UNESCO na lista do patrimônio
  natural mundial.
c) Incorreta. Paraty é um patrimônio cultural arquitetônico e
  paisagístico.
d) Incorreta. O Cristo Redentor é uma estátua que faz parte da paisagem
  do Rio de Janeiro.

\item
SAEB: Identificar os recursos verbais e/ou não verbais que contribuem
para a construção da argumentação em textos em língua inglesa. BNCC:
EF09LI07 -- Identificar argumentos principais e as evidências/exemplos
que os sustentam.
a) Incorreta. No texto, afirma-se, genericamente, que parte do corpo em
geral podem ser afetadas. b) Incorreta. O texto menciona a contaminação
de superfícies. c) Correta. Essa afirmação pode ser encontrada no trecho
``COVID is a respiratory tract infection, although it can sometimes
affect other parts of the body''. d) Incorreta. O texto afirma que a
higienização deve ser realizada, além do uso de máscaras.

\item
SAEB: Avaliar a qualidade e a validade das informações veiculadas em
textos de língua inglesa, incluindo textos provenientes de ambientes
virtuais. BNCC: EF09LI06 -- Distinguir fatos de opiniões em textos
argumentativos da esfera jornalística.
a) Incorreta. Segundo o texto, devemos ser cuidadosos com textos
compartilhados por familiares. b) Correta. Checar a fonte das
informações e confiar em especialistas são duas ferarmentas importantes
contra as inverdades na internet. c) Incorreta. De acordo com o texto,
nem toda fonte da internet é confiável. d) Incorreta. O texto apenas
menciona o uso de redes sociais.

\item
SAEB: Avaliar o uso do léxico (tais como palavras polissêmicas ou
expressões metafóricas) em textos em língua inglesa. BNCC: EF06LI08 --
Identificar o assunto de um texto, reconhecendo sua organização textual
e palavras cognatas.
a) Incorreta. A palavra não é utilizada no sentido figurado de
``mácula''. b) Incorreta. A palavra não é utilizada na acepção de
``mancha''. c) Correta. O termo se refere ao local onde a espécie foi
encontrada. d) Incorreta. A palavra não é utilizada na acepção de
``pinta''.

\item
BNCC: EF09HI16 -- Relacionar a Carta dos Direitos Humanos ao processo de
afirmação dos direitos fundamentais e de defesa da dignidade humana,
valorizando as instituições voltadas para a defesa desses direitos e
para a identificação dos agentes responsáveis por sua violação.
a) Incorreta. O texto trata de uma prisão injustas, e não de fatos
análogos à escravidão.
b) Correta. A modelo mencionada no texto foi presa, sem ter praticado
nenhum crime, com base em supostas provas forjadas pelas forças policiais, o
que contraria um dos artigos da ``Declaração universal dos direitos
humanos'', de que ninguém será arbitrariamente preso ou detido ou
exilado.
c) Incorreta. Não há menção a tortura no texto.
d) Incorreta. Não se trata de privação da nacionalidade.

\item
a) Incorreta. Apesar de empréstimos comporem quase sempre o circuito da
agricultura, a prática agroecológica por si não precisa de empréstimos
para ser efetivada.
b) Incorreta. Como o contexto fala de expansão de cultivos, tal alternativa
não se impõe como uma necessidade.
c) Incorreta. Como se mencionam cultivos, as áreas totalmente florestadas
não seriam úteis.
d) Correta. A população Pataxó precisa de áreas para o plantio, já que,
conforme o texto, estão desenvolvendo cultivos agrícolas.

\item
BNCC: EF09GE10 -- Analisar os impactos do processo de industrialização na
produção e circulação de produtos e culturas na Europa, na Ásia e na
Oceania.
a) Correta. No texto, destaca-se que o avanço da mecanização foi
responsável por destruir a atividade laboral dos trabalhadores
especializados, que detinham o conhecimento da sua produção, como os
tecelões e artesões, uma vez que a indústria potencializou a capacidade
produtiva das mercadorias finais desses trabalhadores.
b) Incorreta. No texto, não se faz menção ao pleno emprego no setor
industrial.
c) Incorreta. Pelo contrário, por se tornarem apenas ``mãos'', os
trabalhadores não tinham mais a possibilidade de criatividade nas
atividades laborais.
d) Incorreta. O texto é enfático ao ressaltar as péssimas condições de
vida de grande parte dos trabalhadores urbanos.

\item
BNCC: EF09GE08 -- Analisar transformações territoriais, considerando o
movimento de fronteiras, tensões, conflitos e múltiplas regionalidades
na Europa, na Ásia e na Oceania.
a) Incorreta. As relações envolvendo Iraque e Irã não envolvem a
disputa entre judeus e árabes.
b) Incorreta. As relações envolvendo Síria e Líbano não envolvem a
disputa entre judeus e árabes.
c) Correta. Judeus e árabes têm sua disputa mais conhecida envolvendo
a formação do Estado de Israel, judeu, e o impedimento à formação do
Estado da Palestina, árabe. Não bastasse ambos estarem situados no mesmo
território, Israel vem incorporando áreas destinadas aos
palestinos.
d) Incorreta. As relações envolvendo Iêmen e Arábia Saudita não
envolvem a disputa entre judeus e árabes.

\item
BNCC: EF09GE08 -- Analisar transformações territoriais, considerando o
movimento de fronteiras, tensões, conflitos e múltiplas regionalidades
na Europa, na Ásia e na Oceania.
a)  Incorreta. O termo ``apátrida'' compreende as pessoas que não apresentam uma nacionalidade e não está diretamente relacionado aos aspectos culturais.
b)  Correta. Os curdos são apátridas por não terem o sentimento de pertencimento aos países que habitam, e isso evidencia que os curdos carecem de um território próprio para a formação do seu Estado.
c)  Incorreta. O termo ``apátrida'' compreende as pessoas que não apresentam uma nacionalidade e não está diretamente relacionado aos aspectos linguísticos.
d)  Incorreta. O termo ``apátrida'' compreende as pessoas que não apresentam uma nacionalidade e não está diretamente relacionado aos dialetos locais.
\end{enumerate}

\colorsec{Simulado 4}

\begin{enumerate}
\item
BNCC: EF89LP04 - Identificar e avaliar teses/opiniões/posicionamentos
explícitos e implícitos, argumentos e contra-argumentos em textos
argumentativos do campo (carta de leitor, comentário, artigo de opinião,
resenha crítica etc.), posicionando-se frente à questão controversa de
forma sustentada.
a) Incorreta. Trata-se de dois parágrafos argumentativos, mas não se
trata de todos eles. b) Incorreta. Trata-se de dois parágrafos
argumentativos, mas não se trata de todos eles. c) Correta. São, de
fato, os parágrafos dedicados ao desenvolvimento, onde se coloca a
argumentação. d) Incorreta. O primeiro parágrafo é introdutório e o
quinto parágrafo também faz parte da argumentação.

\item
BNCC: EF89LP04 - Identificar e avaliar teses/opiniões/posicionamentos
explícitos e implícitos, argumentos e contra-argumentos em textos
argumentativos do campo (carta de leitor, comentário, artigo de opinião,
resenha crítica etc.), posicionando-se frente à questão controversa de
forma sustentada.
a) Correta. Na tese, elaborada no primeiro parágrafo, diz-se que a
diferença de valorização está no tipo de produção de cada segmento. b)
Incorreta. Não se menciona a agilidade de produção no texto. c)
Incorreta. Não se estabalece qualquer relação com o tipo de mão de obra
utilizada. d) Incorreta. Não se estabelece relação com o público-alvo
desses segmentos.

\item
BNCC: EF89LP16 -- Analisar a modalização realizada em textos noticiosos
e argumentativos, por meio das modalidades apreciativas, viabilizadas
por classes e estruturas gramaticais como adjetivos, locuções adjetivas,
advérbios, locuções adverbiais, orações adjetivas e adverbiais, orações
relativas restritivas e explicativas etc., de maneira a perceber a
apreciação ideológica sobre os fatos noticiados ou as posições
implícitas ou assumidas.
a) Incorreta. A palavra não atua como modalizador. b) Incorreta. O verbo
na forma nominal não mostra nenhuma ideia do enunciador. c) Incorreta. O
substantivo não demonstra qualquer ideia prévia do enunciador. d)
Correta. O advérbio atua como modalizador, mostrando que o enunciador
considera equivocada e injusta a desvalorização da cultura diante da
indústria.

\item
BNCC: EF89LP16 -- Analisar a modalização realizada em textos noticiosos
e argumentativos, por meio das modalidades apreciativas, viabilizadas
por classes e estruturas gramaticais como adjetivos, locuções adjetivas,
advérbios, locuções adverbiais, orações adjetivas e adverbiais, orações
relativas restritivas e explicativas etc., de maneira a perceber a
apreciação ideológica sobre os fatos noticiados ou as posições
implícitas ou assumidas.
a) Incorreta. O texto aparece com muitos modalizadores apreciativos. b)
Incorreta. Há muito mais do que um modalizador apreciativo no trecho. c)
Incorreta. Há quase uma dezena de modalizadores apreciativos no trecho.
d) Correta. Surpreendente, única, emocionante, extasiada, espetacular,
impecável, cativante, inesquecível, excepcional.

\item
BNCC: EF69LP47 - Analisar, em textos narrativos ficcionais, as
diferentes formas de composição próprias de cada gênero, os recursos
coesivos que constroem a passagem do tempo e articulam suas partes, a
escolha lexical típica de cada gênero para a caracterização dos cenários
e dos personagens e os efeitos de sentido decorrentes dos tempos
verbais, dos tipos de discurso, dos verbos de enunciação e das
variedades linguísticas (no discurso direto, se houver) empregados,
identificando o enredo e o foco narrativo e percebendo como se estrutura
a narrativa nos diferentes gêneros e os efeitos de sentido decorrentes
do foco narrativo típico de cada gênero, da caracterização dos espaços
físico e psicológico e dos tempos cronológico e psicológico, das
diferentes vozes no texto (do narrador, de personagens em discurso
direto e indireto), do uso de pontuação expressiva, palavras e
expressões conotativas e processos figurativos e do uso de recursos
linguístico-gramaticais próprios a cada gênero narrativo.
a) Incorreta. A crônica tem um tom mais delimitado ao seu tempo,
trazendo ao enredo acontecimentos comuns do cotidiano. b) Correta.
Trata-se de uma narrativa do tipo conto, com um conflito único e poucos
personagens. c) Incorreta. A novela é um texto mais longo, com mais
personagens e mais conflitos do que o conto. d) Incorreta. A novela é um
texto bem mais longo, com muito mais personagens e mais conflitos do que
o conto.

\item
BNCC: EF69LP47 - Analisar, em textos narrativos ficcionais, as
diferentes formas de composição próprias de cada gênero, os recursos
coesivos que constroem a passagem do tempo e articulam suas partes, a
escolha lexical típica de cada gênero para a caracterização dos cenários
e dos personagens e os efeitos de sentido decorrentes dos tempos
verbais, dos tipos de discurso, dos verbos de enunciação e das
variedades linguísticas (no discurso direto, se houver) empregados,
identificando o enredo e o foco narrativo e percebendo como se estrutura
a narrativa nos diferentes gêneros e os efeitos de sentido decorrentes
do foco narrativo típico de cada gênero, da caracterização dos espaços
físico e psicológico e dos tempos cronológico e psicológico, das
diferentes vozes no texto (do narrador, de personagens em discurso
direto e indireto), do uso de pontuação expressiva, palavras e
expressões conotativas e processos figurativos e do uso de recursos
linguístico-gramaticais próprios a cada gênero narrativo.
a) Incorreta. Nota-se que, ao longo da narrativa, a personagem amadurece
e se descobre contadora de histórias. b) Incorreta. Isabela cresce e se
torna mulher ao longo da narrativa. c) Correta. Mostra-se, no conto, que
Isabela começa mais menina se torna mulher durante suas descobertas. d)
Incorreta. Isabela já é uma criança crescida quando a história começa e
não é ainda uma idosa quando ela termina.

\item
BNCC: EF69LP47 - Analisar, em textos narrativos ficcionais, as
diferentes formas de composição próprias de cada gênero, os recursos
coesivos que constroem a passagem do tempo e articulam suas partes, a
escolha lexical típica de cada gênero para a caracterização dos cenários
e dos personagens e os efeitos de sentido decorrentes dos tempos
verbais, dos tipos de discurso, dos verbos de enunciação e das
variedades linguísticas (no discurso direto, se houver) empregados,
identificando o enredo e o foco narrativo e percebendo como se estrutura
a narrativa nos diferentes gêneros e os efeitos de sentido decorrentes
do foco narrativo típico de cada gênero, da caracterização dos espaços
físico e psicológico e dos tempos cronológico e psicológico, das
diferentes vozes no texto (do narrador, de personagens em discurso
direto e indireto), do uso de pontuação expressiva, palavras e
expressões conotativas e processos figurativos e do uso de recursos
linguístico-gramaticais próprios a cada gênero narrativo.
a) Incorreta. Há memórias, que a menina descobre e começa a ler, no
diário que encontra. b) Incorreta. As histórias não são inventadas, são
reais do tempo da bisavó de Isabela. c) Incorreta. Isabela descobre
justamente que sua história e sua formação se mesclam às histórias de
seus antepassados. d) Correta. A menina, durante a leitura, descobre
como ler aquelas memórias é importante para sua própria formação.

\item
SAEB: Avaliar nas linguagens artísticas a diversidade do patrimônio
cultural da humanidade (material e imaterial), em especial o brasileiro,
a partir de suas diferentes matrizes.
BNCC: EF69AR34 -- Analisar e valorizar o patrimônio cultural, material e
imaterial, de culturas diversas, em especial a brasileira, incluindo
suas matrizes indígenas, africanas e europeias, de diferentes épocas, e
favorecendo a construção de vocabulário e repertório relativos às
diferentes linguagens artísticas.
a) Correta. Fazem parte do patrimônio arqueológico os bens relacionados a
  vestígios da ocupação pré-histórica.
b) Incorreta. Fazem parte do patrimônio paisagístico tanto as áreas
  naturais, quanto lugares criados pelo ser humano aos quais é atribuído
  valor por sua configuração paisagística e que se destaquem por sua
  relação com o território onde se encontram.
c) Incorreta. Fazem parte do patrimônio etnográfico bens de valor
  etnográfico e de referência para grupos sociais.
d) Incorreta. Fazem parte do patrimônio das artes aplicadas os bens que
  têm seu valor artístico associado à função utilitária, muito comum na
  arquitetura, no \textit{design} e nas artes gráficas.

\item
SAEB: Reconhecer artistas que contribuíram para o desenvolvimento e a
disseminação de diferentes gêneros e estilos nas artes visuais, dança,
música e teatro.
BNCC: EF69AR04 -- Analisar os elementos constitutivos das artes visuais
(ponto, linha, forma, direção, cor, tom, escala, dimensão, espaço,
movimento etc.) na apreciação de diferentes produções artísticas.
a) Incorreta. O maior expoente do cubismo foi o artista espanhol Pablo Picasso.
b) Incorreta. O maior expoente do impressionismo foi o pintor francês Claude Monet.
c) Incorreta. O maior expoente do expressionismo foi o pintor alemão Ernst Ludwig Kirchner.
d) Correta. O holandês Piet Mondrian foi o maior expoente do abstracionismo.

\item
SAEB: Identificar os usos de diferentes tecnologias e recursos
digitais na produção e circulação das linguagens artísticas.
BNCC: EF69AR21 -- Explorar e analisar fontes e materiais sonoros em
práticas de composição/criação, execução e apreciação musical,
reconhecendo timbres e características de instrumentos musicais
diversos.
a) Incorreta. É uma característica de instalações artísticas, mas, para
  que ela seja considerada instalação sonora, faz-se necessário que explore fontes e materiais sonoros em 
  sua composição.
b) Incorreta. Nem todos os recursos digitais utilizados em instalações
  exploram a materialidade sonora.
c) Incorreta. As instalações podem ocupar ruas, praças, entre outras
  possibilidades, mas, para que essa instalação em espaços
  alternativosseja considerada sonora, precisa lançar mão de fontes
  e materiais sonoros.
d) Correta. As instalações sonoras têm como
  característica determinante a exploração de fontes e materiais sonoros
  em sua criação e em sua composição.

\item
SAEB: Reconhecer elementos de forma e/ou conteúdo de textos de cunho
artístico-cultural (artes, literatura, música, dança, festividades,
entre outros) em língua inglesa. BNCC: EF07LI06 -- Antecipar o sentido
global de textos em língua inglesa por inferências, com base em leitura
rápida, observando títulos, primeiras e últimas frases de parágrafos e
palavras-chave repetidas.
a) Incorreta. ``Js'' refere-se às letras. b) Incorreta. ``Sunday''
significa ``domingo'' e ``J-O'' refere-se às letras. c) Incorreta. Pip é
um personagem, mas ``Prayer-Book'' significa ``livro de orações''. d)
Correta. Joe e Pip são os personagens que aparecem no diálogo.

\item
SAEB: Identificar o assunto de um texto, a partir de sua organização, de
palavras cognatas e/ou de palavras formas por afixação. BNCC: EF06LI06
-- Antecipar o sentido global de textos em língua inglesa por
inferências, com base em leitura rápida, observando títulos, primeiras e
últimas frases de parágrafos e palavras-chave repetidas.
a) Incorreta. O texto cita somente galáxias recentemente descobertas. b)
Correta. A resposta pode ser encontrada no trecho ``Scientists studying
the universe have recently detected what appears to be colossal galaxies
that existed approximately 600 million years after the Big Bang''. c)
Incorreta. O texto apenas menciona o telescópio, mas não explica seu
funcionamento. d) Incorreta. O texto não menciona informações sobre o
universo.

\item
SAEB: Localizar informações específicas, a partir de diferentes
objetivos de leitura, em textos em língua inglesa. BNCC: EF06LI08 --
Identificar o assunto de um texto, reconhecendo sua organização textual
e palavras cognatas.
a) Correta. A informação pode ser encontrada no trecho ``a dinosaur
devoured its final meal: a tiny mammal measuring the size of a mouse''.
b) Incorreta. O texto não menciona uma nova espécie. c) Incorreta. O
texto menciona um mamífero encontrado dentro das entranhas de um
dinossauro. d) Incorreta. O texto menciona somente o tamanho do fóssil.

\item
BNCC: EF09GE10 -- Analisar os impactos do processo de industrialização na
produção e circulação de produtos e culturas na Europa, na Ásia e na
Oceania.
a) Incorreta. O texto destaca que o avanço da produção inglesa
``inundou'' o mercado indiano de tecidos.
b) Incorreta. O texto destaca o oposto: a produção inglesa, mais
competitiva, acessou o mercado indiano, minando a produção local de
tecidos.
c) Correta. A ideia de competitividade compreende a
comparação da atividade produtiva de determinado segmento entre dois
locais, considerando-se qual oferece o produto mais barato com menor custo. No caso, a
indústria inglesa era mais competitiva frente à produção de tecidos
indiana, razão pela qual se notou a maior presença dos produtos
industriais.
d) Incorreta. Pelo contrário, o texto destaca a presença de tecidos
ingleses na Índia.

\item
BNCC: EF09GE12 -- Relacionar o processo de urbanização às transformações
da produção agropecuária, à expansão do desemprego estrutural e ao papel
crescente do capital financeiro em diferentes países, com destaque para
o Brasil.
a) Correta. Os trabalhadores foram realocados para outras
funções, dada a substituição do trabalho manual pelo uso de máquinas.
b) Incorreta. Nesse caso, não houve a perda de postos de trabalho, mas os
trabalhadores apenas foram realocados.
c) Incorreta. Pelo contrário, percebe-se que o trabalho manual foi
substituído pelo uso das máquinas.
d) Incorreta. O texto não destaca algum tipo de flexibilização
laboral.

\item
BNCC: EF09GE08 -- Analisar transformações territoriais, considerando o movimento de fronteiras,
tensões, conflitos e múltiplas regionalidades na Europa, na Ásia e na Oceania.
a) Correta. A atuação do crime organizado, no caso do oferecimento de
segurança por agentes do crime organizado, dá-se pelo vácuo de poder
deixado pelo Estado e cooptado pelos criminosos. Na ausência de um
agente regulador, os criminosos impõem sua lógica aos habitantes locais,
referendados pelo aparato bélico, que depois utilizam para combater o
próprio Estado e manter o controle territorial.
b) Incorreta. Na situação descrita, não há o envolvimento popular no
controle da segurança oferecida; pelo contrário, a lógica dos criminosos
é imposta, com os populares seguindo suas regras.
c) Incorreta. O crime organizado não tem ações e existência tidas
como legítimas, uma vez que o monopólio da violência pertence somente ao
Estado.
d) Incorreta. Nas áreas de controle do crime organizado, não há uma
opção ou alternativa paralela no oferecimento da segurança.

\item
BNCC: EF09GE08 -- Analisar transformações territoriais, considerando o movimento de fronteiras,
tensões, conflitos e múltiplas regionalidades na Europa, na Ásia e na Oceania.
a) Correta. A reforma agrária no Brasil visa corrigir um aspecto
histórico na estrutura fundiária brasileira, pautada historicamente pelo
latifúndio. Dessa forma, o MST tem como finalidade forçar o Estado
nacional a corrigir uma assimetria fomentada ao longo do tempo.
b) Incorreta. A luta pela reforma agrária no Brasil não visa prejudicar
o agronegócio nacional e, muito menos, promover insegurança
para os produtores.
c) Incorreta. De acordo com o texto, o MST não tem como finalidade
intermediar a relação entre pequenos e grandes produtores agrícolas.
d) Incorreta. De acordo com o texto, o MST não tem como finalidade
centralizar e direcionar a produção agrícola nacional.

\item
BNCC: EF09HI21 -- Identificar e relacionar as demandas indígenas e quilombolas
como forma de contestação ao modelo desenvolvimentista da ditadura.
a) Incorreta. As populações indígenas possuem um modo de vida não urbano;
assim impor a urbanização geraria desestruturação social.
b) Correta. A garantia de que a gestão, o controle e o uso do território
indígena seja responsabilidade direta e exclusiva da população indígena
evita que formas de exploração prejudiciais às populações indígenas
ocorram, preservando a organização e o costume desse povo.
c) Incorreta. Os povos indígenas possuem costumes e culturas próprias que
foram gestadas fora do capitalismo, baseando sua subsistência na
exploração dos recursos disponíveis sem visar à acumulação de capitais
(lucro), e integrar tais comunidades a circuitos econômicos dominantes
significaria a imposição a esses povos da exploração abusiva de recursos, visto que, por se tratar de uma área preservada, possuem muitos
recursos básicos cobiçados (terras, minérios, madeira, água).
d) Incorreta. As invasões aos territórios objetivam caça, pesca e
extrativismo; logo fornecer área para plantio não seria uma solução
plausível para as invasões.
\end{enumerate}