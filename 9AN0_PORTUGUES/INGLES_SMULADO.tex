\section{Simulado 1}\label{simulado-1}



\section{Simulado 2}\label{simulado-2}

\num{1} Leia o texto.

\begin{quote}
\textbf{Latin American Digital Media Outlets Turn to English to Reach
New Audiences}
\end{quote}

\begin{quote}
Despite Spanish being spoken by more than 500 million people worldwide,
English has become the dominant global language for communication in
entertainment, information, and business. In recent years, several
digital media outlets in Latin America, spanning from Mexico to Chile,
have opted to translate and produce content in English to broaden their
audience reach and ultimately boost profits. However, this task can
prove challenging at times, and success is not always guaranteed.
\end{quote}

\fonte{Fonte de pesquisa: Katherine Pennacchio. MediaTalks. Latin American media seek to influence public debate and engage audience by translating its journalism into English. Disponível em: *https://mediatalks.uol.com.br/en/2021/10/15/latin-american-media-seek-to-influence-public-debate-and-engage-audience-in-u-s-by-translating-their-journalism-to-english/*. Acesso em: 01 mar. 2023.}

A partir do texto, conclui-se que

a) a língua inglesa exerce influência na mídia europeia.

b) o inglês é irrelevante no mundo contemporâneo.

c) a demanda para o ensino de inglês no Brasil não existe.

d) o inglês é considerado a língua internacional da área midiática.

SAEB: Avaliar a presença, no mundo globalizado, da língua inglesa e/ou
de produtos culturais de países de língua inglesa. BNCC: EF07LI21 --
Analisar o alcance da língua inglesa e os seus contextos de uso no mundo
globalizado.

a) Incorreta. O texto trata da mídia latino-americana. b) Incorreta. No
texto, afirma-se exatamente o contrário. c) Incorreta. O texto não traz
afirmações sobre o ensino de inglês no Brasil. d) Correta. Essa
afirmação pode ser encontrada no trecho ``English is the global
communication language par excellence''.

\num{2} Leia o texto.

\begin{quote}
\textbf{World Leaders Gather at COP27 to Address Climate Change and
Build on Paris Agreement}
\end{quote}

\begin{quote}
The COP27 summit in Egypt will host representatives from various
nations, who will gather to negotiate extensive measures aimed at
combating climate change. The pressure is on for them to surpass the
significant pledges made in Paris seven years ago.
\end{quote}

\begin{quote}
The Paris Agreement was a momentous agreement, in which almost all of
the world's nations endorsed a unified approach to reducing greenhouse
gas emissions, which are responsible for global warming. Despite former
US President Donald Trump's decision to pull out of the agreement, his
successor Joe Biden reinstated it on his first day in office in January
2021.
\end{quote}

\fonte{Fonte de pesquisa: Helen Briggs e Esme Stallard. BBC. COP27: Why is the Paris climate agreement still important? Disponível em: *https://www.bbc.com/news/science-environment-35073297*. Acesso em: 01 mar. 2023.}

O que se compreende do texto como um todo?

a) Todos os líderes mundiais concordam a respeito do acordo de Paris.

b) O evento COP27 não é muito relevante.

c) Diversos líderes mundiais não compareceram ao encontro COP27.

d) Os últimos presidentes dos EUA discordam a respeito do acordo de
Paris.

SAEB: Contrapor perspectivas sobre um mesmo assunto em textos em língua
inglesa. BNCC: EF09LI06 -- Distinguir fatos de opiniões em textos
argumentativos da esfera jornalística.

a) Incorreta. O texto menciona discordâncias a respeito do acordo. b)
Incorreta. O texto menciona o evento como um encontro internacional
importante. c) Incorreta. O texto não faz menção a líderes que não
compareceram ao evento. d) Correta. Essa informação está no trecho ``US
President Donald Trump withdrew from the agreement, but his successor
Joe Biden rejoined on his first day in office in January 2021''.

\num{3} Leia o texto.

\begin{quote}
\textbf{West Ham United Stadium to be Wrapped in Solar Membrane in
Effort to Reduce Carbon Emissions}
\end{quote}

\begin{quote}
The West Ham United stadium, which was originally built for the 2012
Olympics, is set to undergo a green transformation with the installation
of a solar membrane aimed at reducing carbon emissions.
\end{quote}

\begin{quote}
The project is expected to cost around £4 million during the first two
years of implementation, but the investment is expected to be recouped
within five years. Planning documents suggest that work on the site
located in east London could commence later this year.
\end{quote}

\begin{quote}
The stadium's owner, the London Legacy Development Corporation (LLDC),
established to oversee the area's development around the Queen Elizabeth
Olympic Park in Stratford after the 2012 Games, states that the building
could start producing energy by the end of 2024.
\end{quote}

\textbackslash{}fonte\{Fonte de pesquisa: Noah Vickers. BBC. London
Stadium to be covered in solar panels to generate power. Disponível em:
\emph{https://www.bbc.com/news/uk-england-london-64758344}. Acesso em 01
mar. 2023.

De acordo com o texto, os painéis solares serão instalados no estádio
para

a) auxiliar os treinos do time.

b) reduzir as emissões de carbono.

c) evitar apagões.

d) estimular o crescimento da cidade de Londres.

SAEB: Distinguir fatos de opiniões em textos em língua inglesa. BNCC:
EF09LI06 -- Distinguir fatos de opiniões em textos argumentativos da
esfera jornalística.

a) Incorreta. Apesar de se falar em um estádio, o texto não menciona os
treinos de um time. b) Correta. Essa afirmação pode ser encontrada no
trecho ``is to be wrapped in a solar membrane to reduce carbon
emissions''. c) Incorreta. O texto não menciona essa possibilidade. d)
Incorreta. O texto apenas menciona a cidade de Londres, sem falar de seu
crescimento.

\section{Simulado 3}\label{simulado-3}

\num{1} Leia o texto.

\begin{quote}
\textbf{Coronavirus effect}
\end{quote}

\begin{quote}
COVID is a respiratory tract infection, although it can sometimes affect
other parts of the body. The coronavirus primarily enters the body
through the nostrils and mouth, where it then replicates in the
respiratory tract. The only way to prevent contracting the infection is
by limiting the virus's entry through the nose and mouth. Properly
wearing masks that cover the nose and mouth can help, as well as
regularly sanitizing hands, particularly after touching a contaminated
surface, to reduce the likelihood of the virus entering the body.
\end{quote}

\fonte{Fonte de pesquisa: Times of India. Coronavirus Effect: Here's Why You Should Wear Masks In Public Places Even When There Is No Mandate. Disponível em: *https://timesofindia.indiatimes.com/life-style/health-fitness/health-news/coronavirus-effect-heres-why-you-should-wear-masks-in-public-places-even-when-there-is-no-mandate/photostory/92539871.cms?picid=92540018*. Acesso em: 02 mar. 2023.}

De acordo com o texto, um dos argumentos para o uso de máscaras contra a
covid-19 refere-se ao fato de que

a) nem todos os órgãos do corpo podem ser afetados.

b) a medida impede a contaminação de superfícies.

c) a doença afeta o trato respiratório primeiro.

d) a higienização de mãos não é importante.

SAEB: Identificar os recursos verbais e/ou não verbais que contribuem
para a construção da argumentação em textos em língua inglesa. BNCC:
EF09LI07 -- Identificar argumentos principais e as evidências/exemplos
que os sustentam.

a) Incorreta. No texto, afirma-se, genericamente, que parte do corpo em
geral podem ser afetadas. b) Incorreta. O texto menciona a contaminação
de superfícies. c) Correta. Essa afirmação pode ser encontrada no trecho
``COVID is a respiratory tract infection, although it can sometimes
affect other parts of the body''. d) Incorreta. O texto afirma que a
higienização deve ser realizada, além do uso de máscaras.

\num{2} Leia o texto.

\begin{quote}
\textbf{Tips for Navigating Breaking News in the Age of Social Media}
\end{quote}

\begin{quote}
Breaking news can be accessed by anyone with an internet connection,
albeit sometimes in a modified version. Social media posts tend to
circulate at a rate that surpasses the capacity of most moderators and
fact-checkers, and their content can be a mixed bag of factual, false,
out-of-context, and even propagandistic information. How can you
differentiate between trustworthy and untrustworthy sources, and what
criteria should you employ when deciding what to share and report to
tech companies? Here are some fundamental guidelines that everyone
should utilize when consuming breaking news online: analyze who is
disseminating the information. If it's your acquaintances or relatives,
be cautious and do not accept their posts as accurate, unless they have
direct experience or are acknowledged experts on the subject. If it's a
stranger or organization, keep in mind that having a verified check mark
or being popular does not equate to reliability.
\end{quote}

\fonte{Fonte de pesquisa: Heather Kelly. The Washington Post. How to avoid falling for and spreading misinformation online. Disponível em: *https://www.washingtonpost.com/technology/2022/02/24/tips-avoid-misinformation-ukraine-2/*. Acesso em: 02 mar. 2023.}

Segundo o texto, como podemos identificar um argumento confiável?

a) Devemos confiar somente em membros de nossa família.

b) Devemos checar as fontes, dando prioridade a especialistas.

c) Devemos confiar somente nas informações encontradas na internet.

d) Devemos utilizar somente redes sociais confiáveis.

SAEB: Avaliar a qualidade e a validade das informações veiculadas em
textos de língua inglesa, incluindo textos provenientes de ambientes
virtuais. BNCC: EF09LI06 -- Distinguir fatos de opiniões em textos
argumentativos da esfera jornalística.

a) Incorreta. Segundo o texto, devemos ser cuidadosos com textos
compartilhados por familiares. b) Correta. Checar a fonte das
informações e confiar em especialistas são duas ferarmentas importantes
contra as inverdades na internet. c) Incorreta. De acordo com o texto,
nem toda fonte da internet é confiável. d) Incorreta. O texto apenas
menciona o uso de redes sociais.

\num{3} Leia o texto.

\begin{quote}
\textbf{Discovery of a Potential New Hummingbird Species in Peru's
Cordillera Azul National Park}
\end{quote}

\begin{quote}
When researchers spotted a hummingbird with shiny gold feathers on its
throat in Peru's Cordillera Azul National Park, they thought it might be
a newly discovered species. The park, located on a remote \textbf{spot}
of the Andes Mountains' eastern slopes, is an ideal place to spot
genetically distinct species.
\end{quote}

\fonte{Fonte de pesquisa:  Ashley Strickland. CNN. Newly discovered hummingbird looks like it's wearing a golden collar. Disponível em: *https://edition.cnn.com/2023/03/02/world/gold-throated-hummingbird-hybrid-scn/index.html*. Acesso em: 02 mar. 2023.}

No contexto apresentado no texto, a palavra ``spot'' significa

a) ``mácula''.

b) ``mancha''.

c) ``local''.

d) ``pinta''.

SAEB: Avaliar o uso do léxico (tais como palavras polissêmicas ou
expressões metafóricas) em textos em língua inglesa. BNCC: EF06LI08 --
Identificar o assunto de um texto, reconhecendo sua organização textual
e palavras cognatas.

a) Incorreta. A palavra não é utilizada no sentido figurado de
``mácula''. b) Incorreta. A palavra não é utilizada na acepção de
``mancha''. c) Correta. O termo se refere ao local onde a espécie foi
encontrada. d) Incorreta. A palavra não é utilizada na acepção de
``pinta''.

\section{Simulado 4}\label{simulado-4}

num\{1\} Leia o texto.

\begin{quote}
I had never heard Joe read aloud to any greater extent than this
monosyllable, and I had observed at church last Sunday, when I
accidentally held our Prayer-Book upside down, that it seemed to suit
his convenience quite as well as if it had been all right. Wishing to
embrace the present occasion of finding out whether in teaching Joe, I
should have to begin quite at the beginning, I said, ``Ah! But read the
rest, Jo.''
\end{quote}

``The rest, eh, Pip?'' said Joe, looking at it with a slow, searching
eye, ``One, two, three. Why, here's three Js, and three Os, and three
J-O, Joes in it, Pip!''

\fonte{Charles Dickens. _Great Expectations_. Disponível em:
*https://www.gutenberg.org/cache/epub/1400/pg1400-images.html*. Acesso
em: 02 mar. 2023.}

As duas personagens envolvidas no diálogo reproduzido acima são:

a) Os e Js.

b) J-O e Sunday.

c) Pip e Prayer-Book.

d) Joe e Pip.

SAEB: Reconhecer elementos de forma e/ou conteúdo de textos de cunho
artístico-cultural (artes, literatura, música, dança, festividades,
entre outros) em língua inglesa. BNCC: EF07LI06 -- Antecipar o sentido
global de textos em língua inglesa por inferências, com base em leitura
rápida, observando títulos, primeiras e últimas frases de parágrafos e
palavras-chave repetidas.

a) Incorreta. ``Js'' refere-se às letras. b) Incorreta. ``Sunday''
significa ``domingo'' e ``J-O'' refere-se às letras. c) Incorreta. Pip é
um personagem, mas ``Prayer-Book'' significa ``livro de orações''. d)
Correta. Joe e Pip são os personagens que aparecem no diálogo.

\num{2} Leia o texto.

\begin{quote}
\textbf{Astronomers Discover Massive Early Galaxies Dating Back to 600
Million Years After Big Bang}
\end{quote}

\begin{quote}
Scientists studying the universe have recently detected what appears to
be colossal galaxies that existed approximately 600 million years after
the Big Bang. These findings suggest that the early universe may have
undergone a rapid development that led to the formation of these
enormous galaxies. While the James Webb Space Telescope has identified
galaxies that are even older, dating back to a mere 300 million years
after the universe's beginning, it is the vast size and advanced age of
these six massive galaxies that have astonished astronomers. These
discoveries were reported on Wednesday.
\end{quote}

\textbackslash{}fonte\{Fonte de pesquisa: The Associate Press. NPR. The
Webb telescope finds surprisingly massive galaxies from the universe's
youth. Disponível em:
\emph{https://www.npr.org/2023/02/22/1158793897/webb-telescope-huge-early-galaxies-big-bang}.
Acesso em: 02 mar. 2023.

O texto trata

a) da natureza da galáxia que habitamos.

b) da descoberta de galáxias pelo telescópio James Webb.

c) do funcionamento do telescópio James Webb.

d) da idade de nosso universo.

SAEB: Identificar o assunto de um texto, a partir de sua organização, de
palavras cognatas e/ou de palavras formas por afixação. BNCC: EF06LI06
-- Antecipar o sentido global de textos em língua inglesa por
inferências, com base em leitura rápida, observando títulos, primeiras e
últimas frases de parágrafos e palavras-chave repetidas.

a) Incorreta. O texto cita somente galáxias recentemente descobertas. b)
Correta. A resposta pode ser encontrada no trecho ``Scientists studying
the universe have recently detected what appears to be colossal galaxies
that existed approximately 600 million years after the Big Bang''. c)
Incorreta. O texto apenas menciona o telescópio, mas não explica seu
funcionamento. d) Incorreta. O texto não menciona informações sobre o
universo.

\num{3} Leia o texto.

\begin{quote}
\textbf{Preserved Remains of Dinosaur's Last Meal Found in Fossilized
Microraptor}
\end{quote}

\begin{quote}
Around 120 million years ago, during the Cretaceous Period, a dinosaur
devoured its final meal: a tiny mammal measuring the size of a mouse.
Remarkably, the remains of the meal still exist to this day. An
observant researcher noticed the preserved foot of the mammal inside the
abdomen of a fossilized Microraptor zhaoianus, a feathered therapod that
measured less than one meter (or three feet) in length.
\end{quote}

\fonte{Fonte de pesquisa: Katie Hunt. CNN. Rare evidence that dinosaurs feasted on mammals uncovered. Disponível em: *https://edition.cnn.com/2022/12/26/world/dinosaur-mammal-last-meal-scn/index.html*. Acesso em: 02 mar. 2023.}

De acordo com o texto, os pesquisadores encontraram

a) evidências de que dinossauros consumiam mamíferos.

b) uma nova espécie de dinossauros.

c) novas espécies de mamíferos.

d) espécies de dinossauros minúsculas.

SAEB: Localizar informações específicas, a partir de diferentes
objetivos de leitura, em textos em língua inglesa. BNCC: EF06LI08 --
Identificar o assunto de um texto, reconhecendo sua organização textual
e palavras cognatas.

a) Correta. A informação pode ser encontrada no trecho ``a dinosaur
devoured its final meal: a tiny mammal measuring the size of a mouse''.
b) Incorreta. O texto não menciona uma nova espécie. c) Incorreta. O
texto menciona um mamífero encontrado dentro das entranhas de um
dinossauro. d) Incorreta. O texto menciona somente o tamanho do fóssil.
