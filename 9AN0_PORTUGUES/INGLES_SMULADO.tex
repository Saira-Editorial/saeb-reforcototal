\section{Simulado 4}\label{simulado-4}

num\{1\} Leia o texto.

\begin{quote}
I had never heard Joe read aloud to any greater extent than this
monosyllable, and I had observed at church last Sunday, when I
accidentally held our Prayer-Book upside down, that it seemed to suit
his convenience quite as well as if it had been all right. Wishing to
embrace the present occasion of finding out whether in teaching Joe, I
should have to begin quite at the beginning, I said, ``Ah! But read the
rest, Jo.''
\end{quote}

``The rest, eh, Pip?'' said Joe, looking at it with a slow, searching
eye, ``One, two, three. Why, here's three Js, and three Os, and three
J-O, Joes in it, Pip!''

\fonte{Charles Dickens. _Great Expectations_. Disponível em:
*https://www.gutenberg.org/cache/epub/1400/pg1400-images.html*. Acesso
em: 02 mar. 2023.}

As duas personagens envolvidas no diálogo reproduzido acima são:

\begin{escolha}
\item Os e Js.

\item J-O e Sunday.

\item Pip e Prayer-Book.

\item Joe e Pip.
\end{escolha}

SAEB: Reconhecer elementos de forma e/ou conteúdo de textos de cunho
artístico-cultural (artes, literatura, música, dança, festividades,
entre outros) em língua inglesa. BNCC: EF07LI06 -- Antecipar o sentido
global de textos em língua inglesa por inferências, com base em leitura
rápida, observando títulos, primeiras e últimas frases de parágrafos e
palavras-chave repetidas.

a) Incorreta. ``Js'' refere-se às letras. b) Incorreta. ``Sunday''
significa ``domingo'' e ``J-O'' refere-se às letras. c) Incorreta. Pip é
um personagem, mas ``Prayer-Book'' significa ``livro de orações''. d)
Correta. Joe e Pip são os personagens que aparecem no diálogo.

\num{2} Leia o texto.

\begin{quote}
\textbf{Astronomers Discover Massive Early Galaxies Dating Back to 600
Million Years After Big Bang}
\end{quote}

\begin{quote}
Scientists studying the universe have recently detected what appears to
be colossal galaxies that existed approximately 600 million years after
the Big Bang. These findings suggest that the early universe may have
undergone a rapid development that led to the formation of these
enormous galaxies. While the James Webb Space Telescope has identified
galaxies that are even older, dating back to a mere 300 million years
after the universe's beginning, it is the vast size and advanced age of
these six massive galaxies that have astonished astronomers. These
discoveries were reported on Wednesday.
\end{quote}

\textbackslash{}fonte\{Fonte de pesquisa: The Associate Press. NPR. The
Webb telescope finds surprisingly massive galaxies from the universe's
youth. Disponível em:
\emph{https://www.npr.org/2023/02/22/1158793897/webb-telescope-huge-early-galaxies-big-bang}.
Acesso em: 02 mar. 2023.

O texto trata

\begin{escolha}
\item da natureza da galáxia que habitamos.

\item da descoberta de galáxias pelo telescópio James Webb.

\item do funcionamento do telescópio James Webb.

\item da idade de nosso universo.
\end{escolha}

SAEB: Identificar o assunto de um texto, a partir de sua organização, de
palavras cognatas e/ou de palavras formas por afixação. BNCC: EF06LI06
-- Antecipar o sentido global de textos em língua inglesa por
inferências, com base em leitura rápida, observando títulos, primeiras e
últimas frases de parágrafos e palavras-chave repetidas.

a) Incorreta. O texto cita somente galáxias recentemente descobertas. b)
Correta. A resposta pode ser encontrada no trecho ``Scientists studying
the universe have recently detected what appears to be colossal galaxies
that existed approximately 600 million years after the Big Bang''. c)
Incorreta. O texto apenas menciona o telescópio, mas não explica seu
funcionamento. d) Incorreta. O texto não menciona informações sobre o
universo.

\num{3} Leia o texto.

\begin{quote}
\textbf{Preserved Remains of Dinosaur's Last Meal Found in Fossilized
Microraptor}
\end{quote}

\begin{quote}
Around 120 million years ago, during the Cretaceous Period, a dinosaur
devoured its final meal: a tiny mammal measuring the size of a mouse.
Remarkably, the remains of the meal still exist to this day. An
observant researcher noticed the preserved foot of the mammal inside the
abdomen of a fossilized Microraptor zhaoianus, a feathered therapod that
measured less than one meter (or three feet) in length.
\end{quote}

\fonte{Fonte de pesquisa: Katie Hunt. CNN. Rare evidence that dinosaurs feasted on mammals uncovered. Disponível em: *https://edition.cnn.com/2022/12/26/world/dinosaur-mammal-last-meal-scn/index.html*. Acesso em: 02 mar. 2023.}

De acordo com o texto, os pesquisadores encontraram

\begin{escolha}
\item evidências de que dinossauros consumiam mamíferos.

\item uma nova espécie de dinossauros.

\item novas espécies de mamíferos.

\item espécies de dinossauros minúsculas.
\end{escolha}

SAEB: Localizar informações específicas, a partir de diferentes
objetivos de leitura, em textos em língua inglesa. BNCC: EF06LI08 --
Identificar o assunto de um texto, reconhecendo sua organização textual
e palavras cognatas.

a) Correta. A informação pode ser encontrada no trecho ``a dinosaur
devoured its final meal: a tiny mammal measuring the size of a mouse''.
b) Incorreta. O texto não menciona uma nova espécie. c) Incorreta. O
texto menciona um mamífero encontrado dentro das entranhas de um
dinossauro. d) Incorreta. O texto menciona somente o tamanho do fóssil.
