\blankpage
\chapter{1. Texto e significado}

\colorsec{Habilidades do SAEB}

\begin{itemize}
\item Identificar a finalidade de um texto em língua inglesa, com base em
sua estrutura, organização textual, pistas gráficas e/ou aspectos
linguísticos.

\item Identificar o assunto de um texto, a partir de sua organização, de
palavras cognatas e/ou de palavras formas por afixação.

\item Localizar informações específicas, a partir de diferentes objetivos de
leitura, em textos em língua inglesa.

\item Reconhecer elementos de forma e/ou conteúdo de textos de cunho
artístico-cultural (artes, literatura, música, dança, festividades,
entre outros) em língua inglesa.

\item Avaliar o uso do léxico (tais como palavras polissêmicas ou expressões
metafóricas) em textos em língua inglesa.
\end{itemize}

\conteudo{Os textos escritos constituem uma forma de comunicação que utiliza a
linguagem escrita para transmitir informações, ideias e sentimentos.
Eles podem assumir diversas formas, como cartas, e-mails, artigos de
jornal, livros, entre outros.
\reversemarginpar
\marginnote{Neste módulo, os alunos retomam como reconhecer o gênero de um texto e
seu tema principal. Para isso, estudarão algumas estratégias de leitura
relacionadas à busca por palavras-chave e por informações explicitamente
citadas. Os alunos também entrarão em contato com algumas
características de textos literários.\\
\textbf{Habilidades da BNCC}: EF06LI07, EF06LI08 e EF07LI06.}

Para que um texto escrito seja efetivo, é importante suas ideias estejam
organizadas de algum modo que faça sentido para o emissor e para o
receptor e é importante que a linguagem utilizada seja precisa e
adequada ao contexto.

O aspecto mais importante a ser extraído de um texto é seu sentido por
meio da leitura. Para isso, devemos prestar atenção em algumas de suas
características centrais, utilizando estratégias que facilitem esse
processo cognitivo.

Em primeiro lugar, devemos ser capazes de identificar o tipo de texto
que estamos lendo: narrativo, dissertativo, expositivo, injuntivo ou
descritivo. A partir disso, também conseguimos definir o gênero, como
uma notícia ou um conto, por exemplo.}

\colorsec{Atividades}

\num{1} Como podemos definir um texto efetivo?

\linhas{4}

\coment{Um texto bem-sucedido deve ser construído a partir de trechos claros,
coesos e coerentes, isto é, a partir de uma comunicação direta, que
atenda às intenções do emissor e que, ao mesmo tempo, preencha as
expectativas do receptor, considerando-se o contexto em que se dá a
comunicação.}

\num{2} Complete a frase:\\
O traço mais importante a ser extraído de um texto é seu \preencher.

\coment{sentido}


\num{3} Assinale V para verdadeiro e F para falso.

\begin{boxlist}
\item Não é importante identificarmos o tipo de texto que estamos lendo. \coment{F}

\item Identificar palavras-chave é uma estratégia importante na
compreensão textual. \coment{V}

\item Textos literários não contam histórias. \coment{F}

\item Diálogos geralmente constituem uma parte importante de textos
literários. \coment{V}
\end{boxlist}

\num{4} Como se chamam os indivíduos que participam de uma narrativa?

\linhas{1}

\coment{São as personagens.}

\num{5} Qual é a principal função de uma notícia?

\linhas{2}

\coment{Uma notícia deve transmitir informações relevantes ao leitor.}

\num{6} Complete a frase a seguir:\\
Uma notícia deve levar em conta \preencher sobre um mesmo tema.

\coment{diferentes aspectos}
%Paulo, acrescentar opções: diferentes aspectos, um só aspecto
\colorsec{Leia o texto para responder às questões 7 e 8}

\begin{quote}
\textbf{UK space launch: Historic Cornwall rocket mission set to blast off}

The first ever orbital space launch from British soil is getting ready
to blast off.

Monday's mission will see a repurposed 747 jumbo jet release a rocket
over the Atlantic to take nine satellites high above the Earth.

Newquay Airport in Cornwall is the starting point for the operation, on
Monday evening after 2100 GMT.

If it succeeds, it will be a major milestone for UK space, marking the
birth of a home-grown launch industry.

{[}\ldots{}{]}.

\fonte{BBC. UK space launch: Historic Cornwall rocket mission set to blast off.\\
Disponível em: https://www.bbc.com/news/science-environment-64190848.
Acesso em: 28 fev. 2023.}
\end{quote}

\num{7} Qual é o tema explorado no texto reproduzido acima?

\linhas{2}

\coment{O texto trata do primeiro lançamento de um foguete orbital realizado no
Reino Unido.}

\num{8} Segundo o texto, o que esse lançamento representará para a indústria britânica?

\linhas{3}

\coment{De acordo com o texto, caso seja bem-sucedido, o lançamento será um
marco para o nascimento de uma indústria interna de lançamentos
espaciais.}

\colorsec{Leia o texto para responder às questões 9 e 10}

\begin{quote}
\textbf{Persuasion}

``If we can persuade your father to all this,'' said Lady Russell,
looking over her paper, ``much may be done. If he will adopt these
regulations, in seven years he will be clear; and I hope we may be able
to convince him and Elizabeth {[}\ldots{}{]}''.

\fonte{Jane Austen. \emph{Persuasion.} Disponível em:
https://www.gutenberg.org/cache/epub/105/pg105-images.html. Acesso em:
28 fev. 2023.}
\end{quote}

\num{9} Quais são as personagens envolvidas trecho de texto citado?

\linhas{2}

\coment{As personagens citadas são o pai, Lady Russell e Elizabeth.}

\num{10} Que marcas encontradas no texto demarcam os diálogos?

\linhas{4}

\coment{Os diálogos são marcados por aspas e pelo uso do trecho ``said Lady Russell''.}

\section{Abre seção Treino}

\num{1} Leia o texto.

\begin{quote}
\textbf{New type of salty ice may exist on extraterrestrial ocean moons}

The mysterious red streaks crisscrossing the surface of Jupiter's moon
Europa may be the result of a newly discovered kind of salty ice.

Europa has long intrigued scientists because the moon has a subsurface
ocean beneath a thick shell of ice. Plumes of water have been known to
erupt from cracks in the ice shell, releasing the contents of the moon's
alien ocean into space.

Ocean worlds like Europa are the best bet for finding evidence of life
outside of Earth, according to scientists.

The chemical signature of Europa's surface red streaks, thought to be a
frozen mix of water and salts, seemed unusual because it didn't match
any known substance on Earth.

{[}\ldots{}{]}

\fonte{Ashley Strickland. CNN. New type of salty ice may exist on
extraterrestrial ocean moons. Disponível em:
https://edition.cnn.com/2023/02/20/world/salty-ice-europa-ocean-moons-scn/index.html.
Acesso em: 03 mar. 2023.}
\end{quote}


O principal intuito do texto é

\begin{escolha}
\item ensinar o leitor a cumprir determinada tarefa.

\item transmitir uma mensagem entre duas pessoas.

\item narrar uma história ficcional.

\item informar o leitor sobre uma nova descoberta.
\end{escolha}

\coment{Saeb: Identificar a finalidade de um texto em língua inglesa, com base
em sua estrutura, organização textual, pistas gráficas e/ou aspectos
linguísticos.

BNCC: EF06LI07 - Formular hipóteses sobre a finalidade de um texto em
língua inglesa, com base em sua estrutura, organização textual e pistas
gráficas.

a) Incorreta. Esse é o intuito de um texto instrucional;
b) Incorreta. Esse é o intuito de uma carta;
c) Incorreta. Esse é o intuito de um texto literário;
d) Correta. Esse é um dos intuitos de uma notícia, como no caso do
texto.}

\num{2} Leia o texto.

\begin{quote}
\textbf{Antarctica sea-ice hits new record low}

There is now less sea-ice surrounding the Antarctic continent than at
any time since we began using satellites to measure it in the late
1970s.

It is the southern hemisphere summer, when you'd expect less sea-ice,
but this year is exceptional, according to the National Snow and Ice
Data Center.

Winds and warmer air and water reduced coverage to just 1.91 million
square km (737,000 sq miles) on 13 February.

What is more, the melt still has some way to go this summer.

Last year, the previous record-breaking minimum of 1.92 million sq km
(741,000 sq miles) wasn't reached until 25 February.

Three of the last record-breaking years for low sea-ice have happened in
the past seven years: 2017, 2022 and now 2023.

{[}\ldots{}{]}

\fonte{BBC. Antarctica sea-ice hits new record low. Disponível em:
https://www.bbc.com/news/science-environment-64649596. Acesso em: 28
fev. 2023.}
\end{quote}

O assunto tratado no texto refere-se

\begin{escolha}
\item à fauna encontrada na Antártida.

\item aos fortes ventos observados na Antártida.

\item às medidas tomadas para aumentar a quantidade de gelo ártico.

\item à diminuição da quantidade de gelo marinho na Antártida.
\end{escolha}

\coment{Saeb: Identificar o assunto de um texto, a partir de sua organização, de
palavras cognatas e/ou de palavras formas por afixação.

BNCC: EF06LI06 - Antecipar o sentido global de textos em língua inglesa
por inferências, com base em leitura rápida, observando títulos,
primeiras e últimas frases de parágrafos e palavras-chave repetidas.

a) Incorreta. O texto não menciona a fauna da região;
b) Incorreta. O texto apenas menciona os ventos;
c) Incorreta. O texto não traz medidas para reverter a situação;
d) Correta. O texto fala sobre os níveis baixos de gelo observados na
Antártida.}

\num{3} Leia o texto.

\begin{quote}
\textbf{NASA rover finds 'clearest evidence yet' of an ancient lake on Mars}

In the foothills of a Martian mountain, NASA's Curiosity rover found
stunning new evidence of an ancient lake in the form of rocks etched
with the ripples of waves --- and the telltale signs appeared in an
unlikely place.

{[}\ldots{}{]}

The wave-marked rocks were found about one-half mile (800 meters) into
Curiosity's ascent of Mount Sharp. As the rover climbed higher, it
traveled over rocks that would have formed more recently. That's why
researchers didn't expect to see such clear markers of a large body of
water.

{[}\ldots{}{]}

\fonte{Jackie Wattles. CNN. NASA rover finds 'clearest evidence yet' of an
ancient lake on Mars. Disponível em:
https://edition.cnn.com/2023/02/10/world/mars-nasa-curiosity-rover-ancient-waters-scn/index.html.
Acesso em: 28 fev. 2023.}
\end{quote}


De acordo com o texto, as rochas foram encontradas

\begin{escolha}
\item em um meteoro que caiu na Terra.

\item em um lago no centro de Marte.

\item a 800 metros de distância durante a subida do monte Sharp.

\item em uma região muito distante do monte Sharp.
\end{escolha}

\coment{Dificuldade: Difícil.

Saeb: Localizar informações específicas, a partir de diferentes
objetivos de leitura, em textos em língua inglesa.

BNCC: EF06LI08 - Identificar o assunto de um texto, reconhecendo sua
organização textual e palavras cognatas.

a) Incorreta. O texto menciona descobertas feitas em Marte;
b) Incorreta. O texto menciona que o lago existiu há muito tempo em
Marte;
c) Correta. Essa informação é mencionada explicitamente;
d) Incorreta. O texto afirma que essa descoberta ocorreu na subida do
monte.}

\chapter{2. A construção da argumentação}


\colorsec{Habilidades do SAEB}

\begin{itemize}
\item Identificar os recursos verbais e/ou não verbais que contribuem para a
construção da argumentação em textos em língua inglesa.
\end{itemize}

\conteudo{Textos argumentativos são textos que têm como objetivo convencer o
leitor de determinada posição ou ideia. Eles são usados em vários
contextos, desde ensaios acadêmicos até debates políticos e propagandas
publicitárias. A efetividade de um texto assim reside no uso correto de
seu elemento mais básico: o argumento.
\reversemarginpar\marginnote{Neste módulo, o professor deverá mostrar aos alunos como os argumentos
são construídos em um texto. Partindo de uma definição da estrutura
básica de um texto argumentativo, serão analisados componentes
necessários para a compreensão dos gêneros relacionados, como a
exposição de informações e exemplos.\\
Habilidade da BNCC: EF09LI07.}

Os argumentos são os pontos que o autor usa para defender sua tese. Eles
devem ser bem fundamentados e apresentar evidências que comprovem a
veracidade da afirmação, já que o objetivo do texto argumentativo é
persuadir o leitor a adotar a posição defendida. É importante ter em
mente que os argumentos devem ser apresentados de forma organizada e
lógica, para que o texto argumentativo seja coeso e coerente.

Em um texto argumentativo, utilizamos diversos recursos verbais para
desenvolver e defender nosso ponto de vista. Devemos prestar atenção
nessas estruturas para localizar a tese principal. Podemos, por exemplo,
empregar conectivos, isto é, palavras que conectam as ideias do texto,
estabelecendo relações lógicas entre elas. Exemplos de conectivos
incluem "therefore", "because" ou "so".

Outra maneira de construirmos argumentos convincentes se dá pelo uso de
exemplos relacionados ao ponto de vista defendido. São recursos verbais
que ajudam a ilustrar e dar suporte aos argumentos apresentados. Eles
podem ser dados em forma de histórias, situações hipotéticas, dados
estatísticos, entre outros.}

\colorsec{Atividades}

\num{1} Qual é o objetivo de um texto argumentativo?

\linhas{2}

\coment{Um texto argumentativo procura convencer o leitor a respeito de uma
posição ou uma ideia.}

\num{2} Complete a frase:\\
O elemento mais básico encontrado em um texto argumentativo é o \preencher.

\coment{Argumento}

\num{3} Quais qualidades um argumento deve apresentar para convencer o leitor?

\linhas{3}

\coment{Argumentos devem ser bem fundamentados e apresentar evidências que
comprovem a veracidade da afirmação.}

\num{4} Assinale V para verdadeiro e F para falso.

\begin{boxlist}
\item Um texto argumentativo lança mão de recursos verbais para
convencer o leitor. \coment{V}

\item Um texto argumentativo deve apresentar diferentes aspectos sobre
um assunto. \coment{V}

\item ``So'' não pode ser considerado um conectivo de argumentação. \coment{F}

\item Um texto argumentativo não apresenta exemplos. \coment{F}
\end{boxlist}

\num{5} Como podemos identificar a tese principal de um texto argumentativo?

\linhas{3}

\coment{Devemos prestar atenção nos recursos verbais utilizados para convencer o
leitor e para apresentar o ponto de vista a ser defendido.}

\num{6} Em que medida o uso de evidências e exemplos pode fortalecer uma argumentação?

\linhas{4}

\coment{O uso de exemplos ajuda a tornar o argumento mais concreto e tangível
para o leitor, permitindo que ele o associe a sua realidade.}

\colorsec{Leia o texto a seguir para responder às questões de 7 a 10}

\begin{quote}
\textbf{Ten reasons why you should get a COVID-19 vaccine}

At the brink of an unprecedented Covid-19 vaccination campaign, some may
still not be sure whether the benefits outweigh the risks. Here
Professor Martin Michaelis and Dr Mark Wass of the School of Biosciences
give the ten reasons why you should get vaccinated:

\textbf{The vaccines do not interact with DNA}

None of the vaccines interacts with our DNA. Hence, a manipulation of
DNA is technically impossible.

\textbf{Most side effects are caused by the desired immune response}

Most of the reported side effects to the vaccine in trials are caused by
the immune system response. Many infectious disease symptoms are caused
by the immune system, not pathogens. This is why it is often difficult
to tell apart infectious diseases because they cause overlapping
symptoms {[}\ldots{}{]}.

\fonte{Sam Wood. University of Kent. Ten reasons why you should get a COVID-19
vaccine. Disponível em:
https://www.kent.ac.uk/news/covid19/27376/ten-reasons-why-you-should-get-a-covid-19-vaccine.
Acesso em: 03 mar. 2023.}
\end{quote}

\num{7} Qual é o principal objetivo desse texto argumentativo?

\linhas{2}

\coment{O texto busca convencer o leitor a respeito da vacinação contra o
coronavírus.}

\num{8} Encontre no texto um argumento utilizado para defender a
imunização.

\linhas{2}

\coment{O aluno pode citar, por exemplo, o fato de a maioria dos efeitos
colaterais ser causada pela resposta imune desejada.}

\num{9} Por que os argumentos apresentados podem ser considerados
confiáveis?

\linhas{2}

\coment{O texto menciona argumentos oferecidos por dois especialistas no
assunto; logo, podem ser considerados confiáveis.}

\num{10} Complete a frase a seguir a partir da leitura do texto:\\
As vacinas podem ser consideradas seguras porque \preencher com nosso DNA.

\coment{não interagem}

\section{Abre seção Treino}

\num{1} Leia o texto.

\begin{quote}
\textbf{Climate change is fueling more conflict between humans and wildlife}

Wildfires pushing tigers towards Sumatran villages. Drought prodding
elephants into African cropland. Hotter ocean temperatures forcing
whales into shipping lanes.

Humans and wildlife have long struggled to harmoniously coexist. Climate
change is pitting both against each other more often, new research
finds, amplifying conflicts over habitat and resources.

``We should expect these kinds of conflicts to increase in the future,''
said lead researcher Briana Abrahms, a wildlife biologist at the
University of Washington. ``Recognizing that climate is an important
driver can help us better predict when they'll occur and help us
{[}intervene{]}.''

{[}\ldots{}{]}

\fonte{Nathan Rott. NPR. Climate change is fueling more conflict between humans
and wildlife. Disponível em:
https://www.npr.org/2023/03/02/1160471867/climate-change-is-fueling-more-conflict-between-humans-and-wildlife.
Acesso em: 28 fev. 2023.}
\end{quote}


De acordo com o texto, reconhecer a relevância do clima pode nos ajudar a

\begin{escolha}
\item prever a ocorrência de incêndios.

\item encontrar maneiras de coexistência entre humanos e animais.

\item diminuir a temperatura dos oceanos.

\item evitar conflitos entre a humanidade e a natureza.
\end{escolha}

\coment{Saeb: Identificar os recursos verbais e/ou não verbais que contribuem
para a construção da argumentação em textos em língua inglesa.

BNCC: EF09LI07 - Identificar argumentos principais e as
evidências/exemplos que os sustentam.

a) Incorreta. O texto apenas menciona a ocorrência de incêndios;
b) Incorreta. O texto afirma que essa coexistência é difícil;
c) Incorreta. O texto apenas afirma que a temperatura tem aumentado;
d) Correta. Essa afirmação pode ser encontrada no trecho ```Recognizing
that climate is an important driver can help us better predict when
they'll occur {[}\ldots{}{]}'.}

\num{2} Leia o texto.

\begin{quote}
\textbf{Jupiter now has 92 moons after new discovery}

Jupiter already reigns as king of the planets --- it's the largest one
in our solar system. And now, the gas giant has the most known moons,
too.

Astronomers have observed 12 additional moons orbiting Jupiter, bringing
its total number of confirmed moons to 92.

{[}\ldots{}{]}

The team could tell the difference between Jupiter and the objects
around it versus the distant solar system objects \textbf{because} any
objects around Jupiter would be moving at the same rate as the gas
giant. Distant solar system objects can't move as quickly as objects
moving with Jupiter.

{[}\ldots{}{]}

\fonte{Ashley Strickland. CNN. Jupiter now has 92 moons after new discovery.
Disponível em:
https://edition.cnn.com/2023/02/06/world/jupiter-new-moons-scn/index.html.
Acesso em: 28 fev. 2023.}
\end{quote}

Qual é a explicação introduzida pelo conectivo ``because''?

\begin{escolha}
\item O autor destaca a facilidade de observar objetos orbitando o planeta
Júpiter.

\item É mencionada a quantidade desprezível de objetos orbitando Júpiter.

\item É dito que os objetos que orbitam Júpiter se movem na mesma
velocidade que o planeta.

\item A dificuldade de localizar o planeta Júpiter é exposta.
\end{escolha}

\coment{Saeb: Identificar os recursos verbais e/ou não verbais que contribuem
para a construção da argumentação em textos em língua inglesa.

BNCC: EF09LI07 - Identificar argumentos principais e as
evidências/exemplos que os sustentam.

a) Incorreta. O texto expõe um argumento oposto;
b) Incorreta. O texto afirma que foram descobertos novos objetos na
órbita;
c) Correta. Essa afirmação aparece no trecho ``{[}\ldots{}{]} because
any objects around Jupiter would be moving at the same rate as the gas
giant'';
d) Incorreta. A dificuldade reside na observação dos objetos que
circundam Júpiter, mas não do planeta em si.}

\num{3} Leia o texto.


\begin{quote}
\textbf{Why you should wear a face mask even if your state doesn't
require it}

Health experts have long argued that face masks are critical to slowing
the spread of the coronavirus and ending the pandemic. But some elected
officials and their constituents still refuse to wear them.

{[}\ldots{}{]}

``We are not out of the woods. We haven't reached the end of the
pandemic,'' said CNN medical analyst Dr. Leana Wen. ``It's
counterproductive and truly infuriating these governors are treating
this as if the pandemic is over. It's not true.

{[}\ldots{}{]}''.

\fonte{Alaa Elassar. Why you should wear a face mask even if your state doesn't
require it. CNN. Disponível em:
https://edition.cnn.com/2021/03/08/health/face-mask-mandate-states-coronavirus-pandemic/index.html.
Acesso em: 28 fev. 2023.}
\end{quote}

Identifique, no texto, um argumento para o uso de máscaras:

\begin{escolha}
\item Segundo o especialista, ainda não chegamos ao fim da pandemia.

\item Os governadores defendem o uso contínuo das máscaras.

\item Segundo o especialista, isso não é mais necessário.

\item Muitos políticos ainda defendem seu uso.
\end{escolha}

\coment{Saeb: Identificar os recursos verbais e/ou não verbais que contribuem
para a construção da argumentação em textos em língua inglesa.

BNCC: EF09LI07 - Identificar argumentos principais e as
evidências/exemplos que os sustentam.

a) Correta. Essa informação pode ser encontrada no trecho ``We haven't
reached the end of the pandemic'';
b) Incorreta. O texto afirma exatamente o contrário;
c) Incorreta. O especialista defende a continuidade do uso;
d) Incorreta. O texto menciona políticos que fazem o contrário.}

\chapter{3. Diferentes perspectivas}


\colorsec{Habilidades do SAEB}

\begin{itemize}
\item Contrapor perspectivas sobre um mesmo assunto em textos em língua
inglesa.

\item Distinguir fatos de opiniões em textos em língua inglesa.

\item Avaliar a qualidade e a validade das informações veiculadas em textos
de língua inglesa, incluindo textos provenientes de ambientes virtuais.
\end{itemize}

\conteudo{Existem diferentes perspectivas sobre um mesmo assunto que podem ser
influenciadas por fatores como experiências pessoais, valores, crenças,
ideologias, cultura e contexto social. Essas perspectivas podem ser
opostas ou complementares e podem levar a diferentes interpretações e
conclusões. Ao lermos um texto, uma de nossas tarefas principais é a de
identificar visões diferentes e compará-las.
\reversemarginpar\marginnote{Neste módulo, os alunos aprenderão a diferenciar as possíveis visões
acerca de um mesmo tema que encontramos nos textos. Para isso, o
professor deverá diferenciar fato de opinião, mostrando também como
devemos buscar fontes confiáveis ao pesquisarmos um assunto.\\
Habilidades da BNCC: EF07LI21, EF09LI06 e EF09LI17.}

É de suma importância também identificarmos fatos mencionados nos
textos. Devemos, inclusive, distingui-los das opiniões para termos uma
compreensão clara de um assunto. Fatos se referem a informações
objetivas e verificáveis, enquanto opiniões se referem a juízos de valor
subjetivos e pessoais.

Os fatos são informações que podem ser comprovadas através de evidências
concretas, como estatísticas, registros históricos ou experimentos
científicos. Eles são verificáveis e independentes das crenças pessoais
ou dos sentimentos das pessoas. Já as opiniões são juízos de valor que
expressam as crenças ou os sentimentos pessoais de uma pessoa. Elas não
são necessariamente baseadas em fatos ou evidências concretas e podem
variar de acordo com a perspectiva pessoal de cada indivíduo.}

\colorsec{Atividades}

\num{1} Que fatores podem influenciar duas perspectivas diferentes sobre um
mesmo assunto?

\linhas{3}

\coment{Fatores como experiências pessoais, valores, crenças, ideologias,
cultura e contexto social influenciam o desenvolvimento de perspectivas.}

\num{2} O que devemos fazer ao encontrarmos duas opiniões diferentes em um texto?

\linhas{2}

\coment{Devemos compará-las para obtermos uma visão mais ampla.}

\num{3} Complete a frase:\\
Em um mesmo texto, podemos encontrar \preencher diferentes.

\coment{Perspectivas}

\num{4} Assinale V para verdadeiro e F para falso:

\begin{boxlist}
\item Fatos e opiniões são conceitos sinônimos. \coment{F}

\item É muito importante identificarmos diferentes perspectivas sobre um
tema. \coment{V}

\item Não é possível encontrarmos fatos e opiniões em um mesmo texto. \coment{F}

\item Devemos levar em conta a qualidade dos fatos apresentados em um
texto. \coment{V}
\end{boxlist}

\num{5} Como podemos verificar a validade das informações que encontramos em um texto?

\linhas{2}

\coment{Devemos consultar diferentes fontes para confirmar ou negar a validade
das informações que encontramos.}

\colorsec{Leia o texto a seguir para responder às questões de 6 a 10}


\begin{quote}
\textbf{Rewilding seas: Some waters off England to get full protection}

Three stretches of water off the English coast are to get the strictest
possible environmental protections, as part of new measures to restore
the health of the sea.

Fishing will be banned along with all activities that damage the sea
bed, such as mining and laying cables.

Proposals to bring in Highly Protected Marine Areas (HPMAs) at two other
sites have been dropped.

Critics say the plans lack ambition and progress is far too slow.

Prof Callum Roberts, of the University of Exeter, said the protected
areas would make a big difference to the sea-life within them but
covered only 0.5\% of English seas.

"At this rate of progress it'll take 260 years to get to the level of
protection that science says is needed - which is 30\% of the seas
highly protected," he said.

\fonte{BBC. Rewilding seas: Some waters off England to get full protection.
Disponível em: https://www.bbc.com/news/science-environment-64790153.
Acesso em: 03 mar. 2023.}
\end{quote}


\num{6} Que fato é discutido nesse texto?

\linhas{3}

\coment{O texto apresenta um fato no trecho ``Three stretches of water off the
English coast are to get the strictest possible environmental
protections''.}

\num{7} Encontre, no texto, uma perspectiva negativa a respeito do tema.

\linhas{2}

\coment{Uma visão negativa pode ser encontrada no trecho ``Critics say the plans
lack ambition and progress is far too slow''.}

\num{8} Identifique no texto uma informação confiável.

\linhas{3}

\coment{De acordo com o professor Callum Roberts, "At this rate of progress
it'll take 260 years to get to the level of protection that science says
is needed''.}

\num{9} Por que essa informação pode ser considerada confiável?

\linhas{2}

\coment{Essa informação é proveniente de um especialista, e é, portanto, mais
confiável.}

\num{10} Procure no texto um exemplo que justifique a medida mencionada.

\linhas{2}

\coment{O aluno pode mencionar, por exemplo, que a medida auxiliará na
restauração da saúde do oceano ``as part of new measures to restore the
health of the sea''.}

\section{Abre seção Treino}

\num{1} Leia o texto.

\begin{quote}
\textbf{England's archaeological history gathers dust as museums fill up}

{[}\ldots{}{]}

London's largest mosaic find in 50 years was unearthed during a
regeneration project near the Shard in Southwark and archaeologists
working on the route of the HS2 high-speed railway found a vast wealthy
Roman trading settlement.

But Historic England says that museums could soon run out of room for
such artefacts. A report commissioned by the public body and Arts
Council England shows that unless they acquire more storage space, the
amount of material coming out of the ground will soon be greater than
the space available to store it.

{[}\ldots{}{]}

\fonte{BBC. England's archaeological history gathers dust as museums fill up.
Disponível em: https://www.bbc.com/news/science-environment-64707488.
Acesso em: 03 mar. 2023.}
\end{quote}

De acordo com o texto, os museus ingleses não serão mais capazes de
estocar artefatos porque

\begin{escolha}
\item objetos do tipo não são mais encontrados no país.

\item eles estão ficando sem espaço físico.

\item o público não se interessa mais por arqueologia.

\item o conselho de artes será fechado.
\end{escolha}

\coment{Saeb: Contrapor perspectivas sobre um mesmo assunto em textos em língua
inglesa.

BNCC: EF09LI06 - Distinguir fatos de opiniões em textos argumentativos
da esfera jornalística.

a) Incorreta. O texto menciona uma grande descoberta de um mosaico;
b) Correta. A falta de espaço constitui o tema da notícia;
c) Incorreta. O texto não menciona o interesse do público;
d) Incorreta. O texto apenas menciona o conselho.}

\num{2} Leia o texto.

\begin{quote}
\textbf{Earliest evidence of bow and arrow use outside Africa unearthed in France}

{[}\ldots{}{]}

The arrowheads found in the cave were of different sizes. The largest
artifacts were 60 millimeters (2.4 inches) in length, while the smallest
were just 10 millimeters (0.4 inch). To understand exactly how the
points were used, co-lead study author Laure Metz, an archaeologist at
Aix-Marseille Université in France, and her colleagues undertook a
series of experiments with replica weapons.

{[}\ldots{}{]}

Fracture marks on the replica flints shot with a bow also closely
matched the pattern of wear on many of the points excavated from the
cave, revealing them to be the result of a ``ballistic technology''
{[}\ldots{}{]}.

``When you have these light weapons you need to correct this low kinetic
energy with mechanical propulsion. And the only way to make these
fractures on the really tiny arrows \ldots{} was with the bow,'' Metz
explained.

{[}\ldots{}{]}

\fonte{Katie Hunt. Earliest evidence of bow and arrow use outside Africa
unearthed in France. CNN. Disponível em:
https://edition.cnn.com/2023/02/23/world/france-cave-earliest-bow-arrow-use-outside-africa-scn/index.html.
Acesso em: 28 fev. 2023.}
\end{quote}

De acordo com Laure Metz, a especialista consultada, as marcas encontradas na caverna revelam que

\begin{escolha}
\item os achados são mais recentes do que se acreditava.

\item os achados não são relevantes.

\item arcos foram utilizados para lançar as flechas.

\item as fechas escavadas são falsas.
\end{escolha}

\coment{Saeb: Avaliar a qualidade e a validade das informações veiculadas em
textos de língua inglesa, incluindo textos provenientes de ambientes
virtuais.

BNCC: EF09LI06 - Distinguir fatos de opiniões em textos argumentativos
da esfera jornalística.

a) Incorreta. O texto não faz referência à idade dos achados;
b) Incorreta. O texto afirma exatamente o contrário, destacando a
relevância dos achados;
c) Correta. A especialista destaca o uso de arcos para lançar as
flechas;
d) Incorreta. O texto não afirma isso, mostrando como os achados são
importantes.}

\num{3} Leia o texto.

\begin{quote}
\textbf{Look up! Venus and Jupiter are going in for a nighttime kiss}

Last night, after dinner, I went outside to take care of our chickens.
And I literally gasped. Up in the sky were two dazzlingly bright objects
close to each other. It was a beautiful, extraordinary sight. I felt a
tingle of joy and a moment of calm. I felt what psychologists call awe
-- an emotion that can relieve stress and calm nerves. Who doesn't need
that?

And tonight is going to be an even better night to experience this awe.
So, after sunset I'm taking my entire family outside to feel this warm
and lovely feeling of awe. Because these two bright objects -- the
planets Venus and Jupiter -- will be even closer.

`They've been coming in closer and closer for a little nighttime kiss,'
says Jackie Faherty, who's an astronomer at the American Museum of
Natural History.

{[}\ldots{}{]}

\fonte{Michaeleen Doucleff. Look up! Venus and Jupiter are going in for a
nighttime kiss. NPR. Disponível em:
https://www.npr.org/2023/03/01/1160382060/look-up-venus-and-jupiter-are-going-in-for-a-nighttime-kiss.
Acesso em: 28 fev. 2023.}
\end{quote}

O fato explorado no texto diz respeito

\begin{escolha}
\item à dificuldade de observarmos o céu à noite.

\item à aproximação dos planetas Vênus e Júpiter.

\item ao brilho de uma estrela que surge somente à noite.

\item a uma ocorrência astronômica única.
\end{escolha}

\coment{Saeb: Distinguir fatos de opiniões em textos em língua inglesa.

BNCC: EF09LI06 - Distinguir fatos de opiniões em textos argumentativos
da esfera jornalística.

a) Incorreta. O texto menciona justamente a observação do céu;
b) Correta. Essa aproximação observada durante algumas noites é o tema
do texto;
c) Incorreta. Embora seja mencionado, esse não é o fato principal do
texto;
d) Incorreta. O texto menciona que a aproximação pode ser observada
durante algumas noites.}



