\blankpage
\chapter{1. Texto e significado}

\colorsec{Habilidades do SAEB}

\begin{itemize}
\item Identificar a finalidade de um texto em língua inglesa, com base em
sua estrutura, organização textual, pistas gráficas e/ou aspectos
linguísticos.

\item Identificar o assunto de um texto, a partir de sua organização, de
palavras cognatas e/ou de palavras formas por afixação.

\item Localizar informações específicas, a partir de diferentes objetivos de
leitura, em textos em língua inglesa.

\item Reconhecer elementos de forma e/ou conteúdo de textos de cunho
artístico-cultural (artes, literatura, música, dança, festividades,
entre outros) em língua inglesa.

\item Avaliar o uso do léxico (tais como palavras polissêmicas ou expressões
metafóricas) em textos em língua inglesa.
\end{itemize}

\colorsec{Habilidades da BNCC}

\begin{itemize}
\item EF06LI07, EF06LI08, EF07LI06.
\end{itemize}

\conteudo{Os textos escritos constituem uma forma de comunicação que utiliza a
linguagem escrita para transmitir informações, ideias e sentimentos.
Eles podem assumir diversas formas, como cartas, e-mails, artigos de
jornal, livros, entre outros.

Para que um texto escrito seja efetivo, é importante que suas ideias estejam
organizadas de algum modo que faça sentido para o emissor e para o
receptor e é importante que a linguagem utilizada seja precisa e
adequada ao contexto.

O aspecto mais importante a ser extraído de um texto é seu sentido por
meio da leitura. Para isso, devemos prestar atenção em algumas de suas
características centrais, utilizando estratégias que facilitem esse
processo cognitivo.

Em primeiro lugar, devemos ser capazes de identificar o tipo de texto
que estamos lendo: narrativo, dissertativo, expositivo, injuntivo ou
descritivo. A partir disso, também conseguimos definir o gênero, como
uma notícia ou um conto, por exemplo.}

\colorsec{Atividades}

\num{1} Como podemos definir um texto efetivo?

\linhas{4}

\coment{Um texto bem-sucedido deve ser construído a partir de trechos claros,
coesos e coerentes, isto é, a partir de uma comunicação direta, que
atenda às intenções do emissor e que, ao mesmo tempo, preencha as
expectativas do receptor, considerando-se o contexto em que se dá a
comunicação.}

\num{2} Complete a frase a seguir.\\
O traço mais importante a ser extraído de um texto é seu \preencher.

\coment{sentido}


\num{3} Assinale V para verdadeiro e F para falso.

\begin{boxlist}
\item Não é importante identificarmos o tipo de texto que estamos lendo. \coment{F}

\item Identificar palavras-chave é uma estratégia importante na
compreensão de um texto. \coment{V}

\item Textos literários não contam histórias. \coment{F}

\item Diálogos geralmente constituem uma parte importante de textos
literários. \coment{V}
\end{boxlist}

\num{4} Como se chamam os indivíduos que participam de uma narrativa?

\linhas{1}

\coment{São as personagens.}

\num{5} Qual é a principal função de uma notícia?

\linhas{2}

\coment{Uma notícia deve transmitir informações relevantes ao leitor.}

\num{6} Complete a frase com uma das opções a seguir.\\
Uma notícia deve levar em conta \preencher sobre um mesmo tema.

\begin{itemize}
	\item diferentes aspectos
	\item um só aspecto
\end{itemize}

\coment{diferentes aspectos}

Leia o texto para responder às questões 7 e 8.

\begin{quote}
\textbf{A historic day for the UK}

On Monday evening, the very first orbital space launch will take place in the United Kingdom.

The launch will depart from Newquay airport. Nine satellites will be taken to Earth’s orbit.

This represents a breakthrough to the British space industry.

{[}\ldots{}{]}.

\fonte{Fonte de pesquisa: Jonathan Amos e Rebecca Morelle. BBC. UK space launch: Historic Cornwall rocket mission set to blast off.
Disponível em: \emph{https://www.bbc.com/news/science-environment-64190848}.
Acesso em: 28 fev. 2023.}
\end{quote}

\num{7} Qual é o tema explorado nesse texto?

\linhas{2}

\coment{O texto trata do primeiro lançamento de um foguete orbital realizado no
Reino Unido.}

\num{8} Segundo o texto, o que esse lançamento representará para a indústria britânica?

\linhas{3}

\coment{De acordo com o texto, o lançamento será um marco para o nascimento de uma indústria interna de lançamentos espaciais.}

Leia o texto para responder às questões 9 e 10.

\begin{quote}
\textbf{Persuasion}

``If we can persuade your father to all this,'' said Lady Russell,
looking over her paper, ``much may be done. If he will adopt these
regulations, in seven years he will be clear; and I hope we may be able
to convince him and Elizabeth {[}\ldots{}{]}''.

\fonte{Jane Austen. \emph{Persuasion}. Disponível em:
\emph{https://www.gutenberg.org/cache/epub/105/pg105-images.html}. Acesso em:
28 fev. 2023.}
\end{quote}

\num{9} Quais são as personagens envolvidas nesse trecho de texto citado?

\linhas{2}

\coment{As personagens citadas são o pai, Lady Russell e Elizabeth.}

\num{10} Que marcas encontradas no texto demarcam os diálogos?

\linhas{4}

\coment{Os diálogos são marcados por aspas e pelo uso do trecho ``said Lady Russell''.}

\section{Treino}

\num{1} Leia o texto.

\begin{quote}
\textbf{Salty ice found on extraterrestrial ocean moon}

Scientists may have found an explanation for the mysterious red patches observed on the surface of Europa, one of Jupiter’s moons.

By analyzing the chemical nature of these strange structures, it was possible to determine that they were made from a mixture of water and salt. Its configuration was different from everything known on Earth.

{[}\ldots{}{]}

\fonte{Fonte de pesquisa: Ashley Strickland. CNN. New type of salty ice may exist on extraterrestrial ocean moons. Disponível em: \emph{https://edition.cnn.com/2023/02/20/world/salty-ice-europa-ocean-moons-scn/index.html}.
Acesso em: 03 mar. 2023.}
\end{quote}


O principal intuito do texto é

\begin{escolha}
\item ensinar o leitor a cumprir determinada tarefa.

\item transmitir uma mensagem entre duas pessoas.

\item narrar uma história ficcional.

\item informar o leitor sobre uma nova descoberta.
\end{escolha}

\coment{SAEB: Identificar a finalidade de um texto em língua inglesa, com base
em sua estrutura, organização textual, pistas gráficas e/ou aspectos
linguísticos.
BNCC: EF06LI07 -- Formular hipóteses sobre a finalidade de um texto em
língua inglesa, com base em sua estrutura, organização textual e pistas
gráficas.

a) Incorreta. Esse é o intuito de um texto instrucional.
b) Incorreta. Esse é o intuito de uma carta.
c) Incorreta. Esse é o intuito de um texto literário.
d) Correta. Esse é um dos intuitos de uma notícia, como no caso do
texto.}

\num{2} Leia o texto.

\begin{quote}
\textbf{Antarctic ice hits an extremely dangerous level}

Bad news for the planet. Antarctica is facing a new low on the quantity of ice found in its territories.

On February 13th satellites were able to determine that the coverage of ice has been greatly reduced by warmer winds and waters.

This is very worrisome since the melting is expected to continue until the end of the month.  

{[}\ldots{}{]}

\fonte{Fonte de pesquisa: Jonathan Amos e Erwan Rivault. BBC. Antarctica sea-ice hits new record low. Disponível em: \emph{https://www.bbc.com/news/science-environment-64649596}. Acesso em: 28 fev. 2023.}
\end{quote}

O assunto tratado no texto refere-se

\begin{escolha}
\item à fauna encontrada na Antártida.

\item aos fortes ventos observados na Antártida.

\item às medidas tomadas para aumentar a quantidade de gelo ártico.

\item à diminuição da quantidade de gelo na Antártida.
\end{escolha}

\coment{SAEB: Identificar o assunto de um texto, a partir de sua organização, de
palavras cognatas e/ou de palavras formas por afixação.
BNCC: EF06LI06 -- Antecipar o sentido global de textos em língua inglesa
por inferências, com base em leitura rápida, observando títulos,
primeiras e últimas frases de parágrafos e palavras-chave repetidas.

a) Incorreta. O texto não menciona a fauna da região.
b) Incorreta. O texto apenas menciona os ventos.
c) Incorreta. O texto não traz medidas para reverter a situação.
d) Correta. O texto fala sobre os níveis baixos de gelo observados na
Antártida.}

\num{3} Leia o texto.

\begin{quote}
\textbf{An ancient lake on Mars: Curiosity's latest findings}

Curiosity, Nasa’s latest Martian rover, was able find new clues about an ancient lake that once flowed through the red planet.

The robot managed to find rocks marked with distinctive wave-shaped imprints. This is usually a sign of water erosion.

These rocks were located 800 meters into the ascent of Mount Sharp.

{[}\ldots{}{]}

\fonte{Fonte de pesquisa: Jackie Wattles. CNN. NASA rover finds ``clearest evidence yet'' of an ancient lake on Mars. Disponível em: \emph{https://edition.cnn.com/2023/02/10/world/mars-nasa-curiosity-rover-ancient-waters-scn/index.html}.
Acesso em: 28 fev. 2023.}
\end{quote}

De acordo com o texto, as rochas foram encontradas

\begin{escolha}
\item em um meteoro que caiu na Terra.

\item em um lago no centro de Marte.

\item a 800 metros de distância durante a subida do monte Sharp.

\item em uma região muito distante.
\end{escolha}

\coment{SAEB: Localizar informações específicas, a partir de diferentes
objetivos de leitura, em textos em língua inglesa.
BNCC: EF06LI08 -- Identificar o assunto de um texto, reconhecendo sua
organização textual e palavras cognatas.

a) Incorreta. O texto menciona descobertas feitas em Marte.
b) Incorreta. O texto menciona que o lago existiu há muito tempo em
Marte.
c) Correta. Essa informação é mencionada explicitamente.
d) Incorreta. O texto afirma que essa descoberta ocorreu na subida do
monte.}

\chapter{2. A construção da argumentação}


\colorsec{Habilidade do SAEB}

\begin{itemize}
\item Identificar os recursos verbais e/ou não verbais que contribuem para a
construção da argumentação em textos em língua inglesa.
\end{itemize}

\colorsec{Habilidade da BNCC}

\begin{itemize}
	\item EF09LI07.
\end{itemize}

\conteudo{Textos argumentativos são textos que têm como objetivo convencer o
leitor de determinada posição ou ideia. Eles são usados em vários
contextos, desde ensaios acadêmicos até debates políticos e propagandas
publicitárias. A efetividade de um texto assim reside no uso correto de
seu elemento mais básico: o \textbf{argumento}.

Os argumentos são os pontos que o autor usa para defender sua \textbf{tese}, ou seja, seu posicionamento. Eles
devem ser bem fundamentados e apresentar evidências que comprovem a
veracidade da afirmação, já que o objetivo do texto argumentativo é
persuadir o leitor a adotar a posição defendida. É importante ter em
mente que os argumentos devem ser apresentados de forma organizada e
lógica, para que o texto argumentativo seja coeso e coerente.

Em um texto argumentativo, utilizamos diversos recursos verbais para
desenvolver e defender nosso ponto de vista. Devemos prestar atenção
nessas estruturas para localizar a tese principal. Podemos, por exemplo,
empregar conectivos, isto é, palavras que conectam as ideias do texto,
estabelecendo relações lógicas entre elas. Exemplos de conectivos
incluem ``therefore'', ``because'' ou ``so''.

Outra maneira de construirmos argumentos convincentes se dá pelo uso de
exemplos relacionados ao ponto de vista defendido. São recursos verbais
que ajudam a ilustrar e dar suporte aos argumentos apresentados. Eles
podem ser dados em forma de histórias, situações hipotéticas, dados
estatísticos, entre outros.}

\colorsec{Atividades}

\num{1} Qual é o objetivo de um texto argumentativo?

\linhas{2}

\coment{Um texto argumentativo procura convencer o leitor a respeito de uma
posição ou uma ideia.}

\num{2} Complete a frase a seguir.\\
O elemento mais básico encontrado em um texto argumentativo é o \preencher.

\coment{argumento}

\num{3} Quais qualidades um argumento deve apresentar para convencer o leitor?

\linhas{3}

\coment{Argumentos devem ser bem fundamentados e apresentar evidências que
comprovem a veracidade da afirmação.}

\num{4} Assinale V para verdadeiro e F para falso.

\begin{boxlist}
\item Um texto argumentativo lança mão de recursos verbais para
convencer o leitor. \coment{V}

\item Um texto argumentativo deve apresentar diferentes aspectos sobre
um assunto. \coment{V}

\item ``So'' não pode ser considerado um conectivo de argumentação. \coment{F}

\item Um texto argumentativo não apresenta exemplos. \coment{F}
\end{boxlist}

\num{5} Como podemos identificar a tese principal de um texto argumentativo?

\linhas{4}

\coment{Devemos prestar atenção nos recursos verbais utilizados para convencer o
leitor e para apresentar o ponto de vista a ser defendido.}

\num{6} Em que medida o uso de evidências e exemplos pode fortalecer uma argumentação?

\linhas{4}

\coment{O uso de exemplos ajuda a tornar o argumento mais concreto e tangível
para o leitor, permitindo que ele o associe a sua realidade.}

Leia o texto a seguir para responder às questões de 7 a 10.

\begin{quote}
\textbf{Why getting a COVID-19 vaccine is important}

Some people may still have some doubts about getting a COVID-19 vaccine. We talked with experts on the field. Here we will offer two reasons why you should get vaccinated:

\textit{Vaccines and DNA}
Do not fall for fake news! Vaccines do not alter our DNA. It is genetically safe to get vaccinated.

\textit{Side effects are a good sign}
Although bothersome, side effects are a positive outcome of vaccination. They usually indicate the desired immune response.

\fonte{Fonte de pesquisa: Sam Wood. University of Kent. Ten reasons why you should get a COVID-19 vaccine. Disponível em: \emph{https://www.kent.ac.uk/news/covid19/27376/ten-reasons-why-you-should-get-a-covid-19-vaccine}.
Acesso em: 03 mar. 2023.}
\end{quote}

\num{7} Qual é o principal objetivo desse texto argumentativo?

\linhas{2}

\coment{O texto busca convencer o leitor a respeito da vacinação contra o
coronavírus.}

\num{8} Encontre no texto um argumento utilizado para defender a
imunização.

\linhas{2}

\coment{O aluno pode citar, por exemplo, o fato de a maioria dos efeitos
colaterais ser causada pela resposta imune desejada.}

\num{9} Por que os argumentos apresentados podem ser considerados
confiáveis?

\linhas{2}

\coment{O texto menciona argumentos oferecidos por especialistas no
assunto; logo, podem ser considerados confiáveis.}

\num{10} Complete a frase a seguir a partir da leitura do texto.\\
As vacinas podem ser consideradas seguras porque \preencher com nosso DNA.

\coment{não interagem}

\section{Treino}

\num{1} Leia o texto.

\begin{quote}
\textbf{Humans and wildlife are victims of climate change}

According to a new research, the harmony between humans and animals has been deeply affected by climate change.

Tigers, whales and even elephants have all been displaced by these man-made transformations.

This is likely to cause very disturbing problems. According to Briana Abrahms, a biologist from the University of Washington, we must recognize the importance of climate in wildlife preservation so we can prevent these conflicts.  

\fonte{Fonte de pesquisa: Nathan Rott. NPR. Climate change is fueling more conflict between humans and wildlife. Disponível em: \emph{https://www.npr.org/2023/03/02/1160471867/climate-change-is-fueling-more-conflict-between-humans-and-wildlife}.
Acesso em: 28 fev. 2023.}
\end{quote}


De acordo com o texto, reconhecer a relevância do clima pode nos ajudar a

\begin{escolha}
\item prever a ocorrência de incêndios.

\item encontrar maneiras de coexistência entre humanos e animais.

\item diminuir a temperatura dos oceanos.

\item evitar conflitos entre a humanidade e a natureza.
\end{escolha}

\coment{SAEB: Identificar os recursos verbais e/ou não verbais que contribuem
para a construção da argumentação em textos em língua inglesa.
BNCC: EF09LI07 -- Identificar argumentos principais e as
evidências/exemplos que os sustentam.

a) Incorreta. O texto apenas menciona a ocorrência de incêndios.
b) Incorreta. O texto afirma que essa coexistência é difícil.
c) Incorreta. O texto apenas afirma que a temperatura tem aumentado.
d) Correta. Essa afirmação pode ser encontrada no trecho ```Recognizing
that climate is an important driver can help us better predict when
they'll occur''.}

\num{2} Leia o texto.

\begin{quote}
\textbf{Scientists discover new moons orbiting Jupiter}

Jupiter was already known as the biggest planet of our solar system; now, scientists have discovered that it also has the most moons – 92 in total.

Given the distance between Earth and Jupiter, finding these new celestial bodies was a little tricky. Scientists were able to identify them \textbf{because} they move at the same speed rate as the giant planet.

\fonte{Fonte de pesquisa: Ashley Strickland. CNN. Jupiter now has 92 moons after new discovery.
Disponível em: \emph{
https://edition.cnn.com/2023/02/06/world/jupiter-new-moons-scn/index.html}.
Acesso em: 28 fev. 2023.}
\end{quote}

Qual é a explicação introduzida pelo conectivo ``because''?

\begin{escolha}
\item O autor destaca a facilidade de observar objetos orbitando o planeta
Júpiter.

\item É mencionada a quantidade desprezível de objetos orbitando Júpiter.

\item É dito que os objetos que orbitam Júpiter se movem na mesma
velocidade que o planeta.

\item O autor expõe a dificuldade de localizar o planeta Júpiter.
\end{escolha}

\coment{SAEB: Identificar os recursos verbais e/ou não verbais que contribuem
para a construção da argumentação em textos em língua inglesa.
BNCC: EF09LI07 -- Identificar argumentos principais e as
evidências/exemplos que os sustentam.

a) Incorreta. O texto expõe um argumento oposto.
b) Incorreta. O texto afirma que foram descobertos novos objetos na
órbita.
c) Correta. Essa afirmação aparece no trecho ``because
any objects around Jupiter would be moving at the same rate as the gas
giant''.
d) Incorreta. A dificuldade reside na observação dos objetos que
circundam Júpiter, mas não do planeta em si.}

\num{3} Leia o texto.


\begin{quote}
\textbf{Reasons to continue wearing a mask}

Even though many doctors and experts still argue for the use of masks, some politicians not only refuse to wear them but also spread misinformation about the matter.

According to Dr. Leana Wen, it is very important to keep on wearing masks since we the pandemic is not over yet. 

\fonte{Texto escrito para este material.}
\end{quote}

Identifique, no texto, um argumento para o uso de máscaras.

\begin{escolha}
\item Segundo o especialista, ainda não chegamos ao fim da pandemia.

\item Segundo o especialista, estamos nos aproximando do fim da pandemia.

\item Segundo o especialista, esse uso não é mais necessário.

\item Muitos políticos ainda defendem seu uso.
\end{escolha}

\coment{SAEB: Identificar os recursos verbais e/ou não verbais que contribuem
para a construção da argumentação em textos em língua inglesa.
BNCC: EF09LI07 -- Identificar argumentos principais e as
evidências/exemplos que os sustentam.

a) Correta. Essa informação pode ser encontrada no trecho ``it is very important to keep on wearing masks since we the pandemic is not over yet''.
b) Incorreta. De acordo com o texto, a pandemia ainda não terminou.
c) Incorreta. O especialista defende a continuidade do uso.
d) Incorreta. O texto menciona políticos que fazem o contrário.}

\chapter{3. Diferentes perspectivas}


\colorsec{Habilidades do SAEB}

\begin{itemize}
\item Contrapor perspectivas sobre um mesmo assunto em textos em língua
inglesa.

\item Distinguir fatos de opiniões em textos em língua inglesa.

\item Avaliar a qualidade e a validade das informações veiculadas em textos
de língua inglesa, incluindo textos provenientes de ambientes virtuais.
\end{itemize}

\colorsec{Habilidades da BNCC}

\begin{itemize}
	\item EF07LI21, EF09LI06, EF09LI17.
\end{itemize}

\conteudo{Existem diferentes perspectivas sobre um mesmo assunto que podem ser
influenciadas por fatores como experiências pessoais, valores, crenças,
ideologias, cultura e contexto social. Essas perspectivas podem ser
opostas ou complementares e podem levar a diferentes interpretações e
conclusões. Ao lermos um texto, uma de nossas tarefas principais é identificar visões diferentes e compará-las.

É de suma importância também identificarmos fatos mencionados nos
textos. Devemos, inclusive, distingui-los das opiniões para termos uma
compreensão clara de um assunto. Os \textbf{fatos} referem-se a informações
objetivas e verificáveis, enquanto as \textbf{opiniões} referem-se a juízos de valor
subjetivos e pessoais.

Os fatos são informações que podem ser comprovadas por meio de evidências
concretas, como estatísticas, registros históricos ou experimentos
científicos. Eles são verificáveis e independentes das crenças pessoais
ou dos sentimentos das pessoas. Já as opiniões são juízos de valor que
expressam as crenças ou os sentimentos pessoais de uma pessoa. Elas não
são necessariamente baseadas em fatos ou evidências concretas e podem
variar de acordo com a perspectiva pessoal de cada indivíduo.}

\colorsec{Atividades}

\num{1} Que fatores podem influenciar duas perspectivas diferentes sobre um
mesmo assunto?

\linhas{4}

\coment{Fatores como experiências pessoais, valores, crenças, ideologias,
cultura e contexto social influenciam o desenvolvimento de perspectivas.}

\num{2} O que devemos fazer ao encontrarmos duas opiniões diferentes em um texto?

\linhas{2}

\coment{Devemos compará-las para obtermos uma visão mais ampla.}

\num{3} Complete a frase a seguir.\\
Em um mesmo texto, podemos encontrar \preencher diferentes.

\coment{perspectivas}

\num{4} Assinale V para verdadeiro e F para falso.

\begin{boxlist}
\item Fatos e opiniões nomeiam ideias que se relacionam com o mesmo conceito. \coment{F}

\item É muito importante identificarmos diferentes perspectivas sobre um
tema. \coment{V}

\item Não é possível encontrarmos fatos e opiniões em um mesmo texto. \coment{F}

\item Devemos levar em conta a qualidade dos fatos apresentados em um
texto. \coment{V}
\end{boxlist}

\num{5} Como podemos verificar a validade das informações que encontramos em um texto?

\linhas{2}

\coment{Devemos consultar diferentes fontes para confirmar ou negar a validade
das informações que encontramos.}

Leia o texto a seguir para responder às questões de 6 a 10.


\begin{quote}
\textbf{Waters off the coast of England will get extra protection}

Some parts of the coast of England will get further protection. The main goal of this measure is to improve health of the sea.

Activities such as mining and fishing will be altogether banned. Although necessary, some experts consider these bans to be insufficient.

For example, Prof. Callum Roberts, from the University of Exeter, says that, while this kind of measure is important, only a small amount of the English coast will be protected. 

\fonte{Fonte de pesquisa: Helen Briggs. BBC. Rewilding seas: Some waters off England to get full protection. Disponível em: \emph{https://www.bbc.com/news/science-environment-64790153}. Acesso em: 03 mar. 2023.}
\end{quote}


\num{6} Que fato é discutido nesse texto?

\linhas{4}

\coment{O texto apresenta um fato no trecho ``Some parts of the coast of England will get further protection''.}

\num{7} Encontre, no texto, uma perspectiva negativa a respeito do tema.

\linhas{4}

\coment{Uma visão negativa pode ser encontrada no trecho ``Although necessary, some experts consider these bans to be insufficient''.}

\num{8} Identifique no texto uma informação confiável.

\linhas{6}

\coment{De acordo com o professor Callum Roberts, ``while this kind of measure is important, only a small amount of the English coast will be protected''.}

\num{9} Por que essa informação pode ser considerada confiável? Explique.

\linhas{4}

\coment{Essa informação é proveniente de um especialista, e é, portanto, mais
confiável.}

\num{10} Procure no texto um exemplo que justifique a medida mencionada.

\linhas{4}

\coment{O aluno pode mencionar, por exemplo, que a medida auxiliará na
restauração da saúde do oceano: ``The main goal of this measure is to improve health of the sea''.}

\section{Treino}

\num{1} Leia o texto.

\begin{quote}
\textbf{England's museums running out of space}

Two exciting historical discoveries were recently made in England. A huge mosaic was found in Southwark during renovation work and a Roman settlement was uncovered by archaeologists near a railway.

However, England is also facing an urgent matter: its museums are simply running out of space. According to a recent report by the Arts
Council England, the number of historical artefacts will soon exceed the physical infrastructure to stock it. 

\fonte{Fonte de pesquisa: Patrick Hughes. BBC. England's archaeological history gathers dust as museums fill up. Disponível em: \emph{https://www.bbc.com/news/science-environment-64707488}.
Acesso em: 03 mar. 2023.}
\end{quote}

De acordo com o texto, os museus ingleses não serão mais capazes de
estocar artefatos, porque

\begin{escolha}
\item objetos do tipo não são mais encontrados no país.

\item eles estão ficando sem espaço físico.

\item o público não se interessa mais por arqueologia.

\item o conselho de artes será fechado.
\end{escolha}

\coment{SAEB: Contrapor perspectivas sobre um mesmo assunto em textos em língua
inglesa.
BNCC: EF09LI06 -- Distinguir fatos de opiniões em textos argumentativos
da esfera jornalística.

a) Incorreta. O texto menciona uma grande descoberta de um mosaico.
b) Correta. A falta de espaço constitui o tema da notícia.
c) Incorreta. O texto não menciona o interesse do público.
d) Incorreta. O texto apenas menciona o conselho.}

\num{2} Leia o texto.

\begin{quote}
\textbf{New evidence of bow and arrow use found in France}

According to Laure Metz, an archaeologist from Aix-Marseille Université, the markings etched on arrows recently found in a Cave in France indicate that humans were using bows to shoot these projectiles.

Metz and her colleagues were able to determine this after running a series of tests with replica weapons. The results were extremely similar to what they found on the cave. This was a very relevant discovery for the archeology community.   

\fonte{Fonte de pesquisa: Katie Hunt. CNN. Earliest evidence of bow and arrow use outside Africa unearthed in France. Disponível em: \emph{
https://edition.cnn.com/2023/02/23/world/france-cave-earliest-bow-arrow-use-outside-africa-scn/index.html}.
Acesso em: 28 fev. 2023.}
\end{quote}

De acordo com Laure Metz, a especialista consultada, as marcas encontradas na caverna revelam que

\begin{escolha}
\item os achados são mais recentes do que se acreditava.

\item os achados não são relevantes.

\item arcos foram utilizados para lançar as flechas.

\item as flechas escavadas são falsas.
\end{escolha}

\coment{SAEB: Avaliar a qualidade e a validade das informações veiculadas em
textos de língua inglesa, incluindo textos provenientes de ambientes
virtuais.
BNCC: EF09LI06 -- Distinguir fatos de opiniões em textos argumentativos
da esfera jornalística.

a) Incorreta. O texto não faz referência à idade dos achados.
b) Incorreta. O texto afirma exatamente o contrário, destacando a
relevância dos achados.
c) Correta. A especialista destaca o uso de arcos para lançar as
flechas.
d) Incorreta. O texto não afirma isso.}

\num{3} Leia o texto.

\begin{quote}
\textbf{Venus and Jupiter are coming closer and closer}

If you look at the night sky tonight, you will probably be able to notice a very interesting event. Venus and Jupiter are getting very close to each other. According to Jackie Faherty, an astronomer at the American Museum of Natural History, it looks as if the planets are kissing each other.

This event has been going on for some time. With each night, the planets come closer and closer. If you go outside tonight, you might able to catch a glimpse of this beautiful phenomenon. 

\fonte{Fonte de pesquisa: Michaeleen Doucleff. NPR Venus and Jupiter are going in for a nighttime kiss. NPR. Disponível em: \emph{
https://www.npr.org/2023/03/01/1160382060/look-up-venus-and-jupiter-are-going-in-for-a-nighttime-kiss}. Acesso em: 28 fev. 2023.}
\end{quote}

O fato explorado no texto diz respeito

\begin{escolha}
\item à dificuldade de observarmos o céu à noite.

\item à aproximação dos planetas Vênus e Júpiter.

\item ao brilho dos planetas Vênus e Júpiter.

\item a uma ocorrência astronômica que pode ser observada em um dia específico.
\end{escolha}

\coment{SAEB: Distinguir fatos de opiniões em textos em língua inglesa.
BNCC: EF09LI06 -- Distinguir fatos de opiniões em textos argumentativos
da esfera jornalística.

a) Incorreta. O texto menciona justamente a observação do céu.
b) Correta. Essa aproximação observada durante algumas noites é o tema
do texto.
c) Incorreta. O texto não menciona, especificamente, o brilho dos planetas.
d) Incorreta. O texto menciona que a aproximação pode ser observada
já há algum tempo.}



