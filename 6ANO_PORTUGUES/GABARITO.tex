\chapter{Respostas}
\pagestyle{plain}
\footnotesize

\pagecolor{gray!40}

\colorsec{Português – Módulo 1 – Treino}

\begin{enumerate}
\item (A) Incorreta. O texto afirma que as crianças aprendem a consumir de forma inconsequente, porém não é essa a principal razão do desequilíbrio global.
(B) Incorreta. Critérios e valores distorcidos de consumo são passados para as pessoas desde a infância, e não critérios e valores distorcidos de modo geral, como dá a entender a afirmativa.
(C) Correta. O fato de explorarmos de forma irresponsável o meio ambiente ao longo de décadas é o principal responsável pelo desequilíbrio global.
(D) Incorreta. Critérios e valores distorcidos com relação ao consumo (e não de modo geral) são de fato problemas de ordem ética, econômica e social, porém não são os responsáveis diretos pelo desequilíbrio global.

\item (A) Incorreta. Os emojis de fato estão presentes no dia a dia das pessoas
que utilizam ferramentas de comunicação digital, por exemplo, mas
aqueles que não estão familiarizados com esse tipo de linguagem poderiam
se confundir.
(B) Incorreta. Os textos traduzem as expressões dos emojis, mas somente
eles não tornam a pesquisa intuitiva.
(C) Incorreta. Os emojis estão representados em cores de sinalização
praticamente universal, que já fazem parte do nosso cotidiano, como o
semáforo: vermelho quando não se deve prosseguir, amarelo para alerta e
verde para prosseguir. Contudo, sozinhas as cores não fazem sentido na
pesquisa, elas precisam estar atreladas a uma imagem.
(D) Correta. A combinação de expressões faciais, texto de apoio e cores
torna a pesquisa intuitiva, ou seja, fácil de ser respondida por todos
os públicos, e inclusiva, isto é, acessível a todas as pessoas.

\item (A) Incorreta. A intenção de usar a palavra ``mulher'' no diminutivo não
é fazer chacota com a autora do livro: pelo contrário, é ironizar o fato
de mulheres serem subestimadas por supostamente não conseguirem fazer
certas coisas.
(B) Incorreta. Algumas vezes o diminutivo é usado de forma carinhosa,
porém, neste caso, a palavra ``mulherzinha'' é usada de forma irônica.
(C) Incorreta. A palavra no diminutivo de certa forma brinca com o fato
de a autora ser uma mulher pequena, porém ela tem outro efeito de
sentido além desse.
(C) Correta. Chamar uma mulher de ``mulherzinha'' em alguns contextos é
pejorativo, contudo, neste caso, pretende-se dar outro efeito de sentido
à palavra ironizando esse fato ao mesmo tempo em que a palavra é usada
em sentido literal, para representar a autora, que é uma mulher
realmente pequena.
\end{enumerate}

\colorsec{Português – Módulo 2 – Texto}

\begin{enumerate}
% número 8
\item (A) Incorreta. O texto orienta sobre direitos de animais selvagens,
mas não informa que prendê-los é crime.
(B) Incorreta. O trecho não menciona nada sobre poder manter animais
domésticos presos.
(C) Correta. A Declaração Universal dos Direitos dos Animais é um texto normativo que
tem como princípio garantir direitos de sobrevivência dos animais de
modo geral, a fim de orientar a convivência harmoniosa entre eles e os
seres humanos. Cabe a nós garantir esse direito, uma vez que os animais
são seres irracionais.
(D) Incorreta. O trecho menciona que restringir a liberdade de animais
selvagens, mesmo que para estudo, fere esse direito, porém não menciona
que é crime ou que a pessoa será punida por fazer isso.

% número 9
\item (A) Incorreta. Não se trata de tornar o convívio mais alegre, e sim
igualitário e inclusivo.
(B) Incorreta. O objetivo do texto normativo é regulamentar normas de
bom convívio, a fim de tornar o dia a dia das pessoas mais justos em
direitos.
(C) Incorreta. O direito a atendimento prioritário não faz da vida de
pessoas pertencentes aos grupos da imagem mais segura.
(D) Mais justo. O texto normativo apresentado trata do direito ao atendimento
prioritário dos grupos descritos nas imagens: idosos, gestantes, pessoas
com crianças de colo, pessoas com deficiência e autistas. Por seu
caráter reivindicatório, o objetivo é tornar o convívio entre as pessoas
mais justo.

% número 10
\item (A) Incorreta. O artigo de opinião é um gênero textual jornalístico que
tem como objetivo comentar e opinar a respeito de um tema.
(B) Correta. A carta de reclamação é um gênero textual que expressa uma indignação ou
uma insatisfação de determinado indivíduo diante de alguma relação
comercial, de serviço ou compra, por exemplo. Na postagem, o consumidor
reclama de ônibus de determinada linha, afirmando que é impossível
viajar neles.
(C) Incorreta. O texto informativo tem como objetivo informar o(a)
leitor(a) sobre determinado assunto, elucidando e esclarecendo-o(a)
sobre o tema em questão.
(D) Incorreta. A campanha publicitária tem como objetivo persuadir uma
pessoa a comprar um produto ou adotar um comportamento.
\end{enumerate}

\colorsec{Português – Módulo 2 – Treino}

\begin{enumerate}
\item (A) Correta. A campanha é um incentivo para que as pessoas se
conscientizem e reivindiquem a manutenção do SUS.
(B) Incorreta. A campanha não pretende convidar pessoas a utilizarem o
SUS.
(C) Incorreta. A campanha, embora defenda o SUS, não informa que de
alguma forma ele é atacado.
(D) Incorreta. Embora veiculada em redes sociais, a campanha busca
defender a manutenção do SUS e não divulgar seus serviços.

\item (A) Incorreta. Nenhum piso tátil tem função de evitar que qualquer pessoa
deixe de acessar um recurso público.
(B) Correta. O piso tátil em bolinha, conforme explica a imagem 1, é um
alerta para a pessoa com deficiência visual, ou sejam, serve para
orientá-la de que ali há um objeto; o piso com função de direcionamento
é aquele com padrão em faixas; cada símbolo tem um significado
diferente.
(C) Incorreta. formas diferentes, significados diferentes.
(D) Incorreta. A proteção é à pessoa, não ao ambiente público.

\item (A) Incorreto, O texto não compara termos técnicos.
(b) Embora seja um texto de divulgação científica, a linguagem torna-se
acessível quando as pessoas conseguem criar uma imagem do que está sendo
explicado.
(C) Incorreto. As dimensões são explicadas por meio de linguagem
coloquial.
(D) Incorreto. Não há uma metáfora ou comparação, e sim explicações sobre
os termos técnicos.
\end{enumerate}

\colorsec{Português — Módulo 3 — Texto}

\begin{enumerate}
% número 10
\item (A) Incorreta. Embora a expressão ``de mal a pior'' esteja presente em
algumas canções da MPB, ela é um ditado popular.
(B) Correta. A frase da imagem é um ditado popular: "de mal a pior",
usada quando se quer dizer que não há nada tão ruim que não possa
piorar.
(C) Incorreta. De mal a pior é um ditado, uma expressão popular, e não
uma figura de linguagem. Aqui, na imagem, seu uso pode ser associado a
uma ironia, para dar o efeito de sentido de humor, mas o texto em si é
um dito popular.
(D) ``De mal a pior'' não aparece em nenhuma cantiga popular.
\end{enumerate}

\colorsec{Português — Módulo 3 — Treino}

\begin{enumerate}
\item (A) Incorreta. O texto 1 é um cartum e o texto 2 é um texto de divulgação
científica.
(B) Os textos apresentam intertextualidade com relação ao tema, que é a
influência do Sol na Terra.
(C) Incorreta. Os textos apresentam a mesma temática geral.
(D) Incorreta. Apenas o texto 2 é de cunho informativo, pois transmite a
mensagem de maneira assertiva.

\item (A) Incorreta. Os elementos não verbais da imagem não se dissociam de
elementos cotidianos; eles fazem parte deles.
(B) Incorreta. A obra não pretende instruir como acionar o alerta sonoro.
(C) Correta. A imagem forma uma intertextualidade ao unir o sinal sonoro,
na parede, com um hidrante, no chão, e a pintura do garoto brincando com
um martelo que mede a força.
(D) Incorreta. Os elementos verbais, como os quadros na parece, não fazem
diferença na composição da imagem.

(A)Incorreta. Embora a letra fale sobre mata, caule e raiz, não se pode
afirmar que a imagem é de um desmatamento.
(B) Incorreta. A letra é uma canção de amor entre pessoas, e não sobre
natureza, portanto, não apresenta intertextualidade com a imagem.
(C) Incorreta. A letra fala apenas de sentimentos humanos, e não
apresenta intertextualidade com a imagem.
(D) Correta. A imagem apresenta um caule cortado em formato de coração,
comparando o sentimento com a mata.
\end{enumerate}

\colorsec{Português — Módulo 4 — Texto}

\begin{enumerate}
% número 6
\item (A) A resposta correta é "a". As aspas reproduzem a fala do especialista na forma de discurso indireto.
(B)  Incorreta. O discurso direto é representado por meio de travessão e
quebra de parágrafo.
(C)  Incorreta. A função das aspas, aqui, é demarcar o que foi
reproduzido da fala do especialista.
(D)  Incorreta. O fato de um especialista ter sido entrevistado apenas
não exige por si só que o trecho em que isso é mencionado seja usado
entre aspas.
\end{enumerate}

\colorsec{Português — Módulo 4 — Treino}

\begin{enumerate}
\item (A) Correta. O informe apresenta dicas de saúde mental que englobam
comportamentos que auxiliam na manutenção do físico e do psicológico.
(B) Incorreta. As informações do infográfico são dicas, sugestões, e não
necessariamente bastam para se ter saúde mental.
(C) Incorreta. Não se pode afirmar que não é possível ter saúde mental se
essas dicas não forem seguidas.
(D) Incorreta. A saúde mental depende de uma série de circunstâncias. O
infográfico apresenta apenas algumas dicas.

\item (A) Incorreta. A citação usada como argumento é indireta.
(B) Incorreta. Seria uma citação direta se o trecho estivesse
identificado entre aspas, por exemplo.
(C) Correta. O texto apresenta a tese de que os primeiros anos de vida
funcionam como base para as aquisições que o cérebro fará nos anos
seguintes, portanto, promover a saúde mental infantil é uma forma de
cuidar de nossos futuros jovens, adultos e idosos. Para argumentar, é
utilizado como recurso uma citação indireta de um dado da OMS.
(D) Incorreta. A paráfrase ocorre com mais frequência em textos que
demonstram que o enunciador reconta uma história, e não em artigos
científicos.

\item (A) Incorreta. O primeiro parágrafo é apenas uma introdução da matéria.
(B) Incorreta. Esta é uma citação direta, destacada entre aspas.
(C) Incorreta. Esta é uma introdução à citação direta subsequente.
(D) Correta. É possível identificar que esta é uma citação indireta a
partir de expressões modalizadoras que representam o posicionamento da
entrevistada; neste caso, o verbo ``dizer'' faz essa função.
\end{enumerate}

\colorsec{Português — Módulo 5 — Texto}

\begin{enumerate}
% número 7
\item (A) Incorreta. O texto não apresenta entrevista ou mesmo citação
indireta de qualquer especialista em Frida Kahlo.
(B) Corrreta. A frase é uma paráfrase.
(C) Incorreta. O destaque dado à frase, na verdade, é uma paráfrase em
que a autora reproduz falas comuns atribuídas a historiadores. Ela
utiliza suas próprias palavras, porém destacando que a frase é de outra
pessoa.
(D) Incorreto. A frase é um fato sobre Frida. Embora esteja escrito em
sentido figurado, é fato que a partir do acidente Frida passou a pintar
aquilo que sentia.
\end{enumerate}

\colorsec{Português — Módulo 5 — Treino}

\begin{enumerate}
\item (A) Incorreta. A resenhista considera o enredo simples, mas pode não ser
simples para outras pessoas.
(B) Incorreta. Querer ler mais é uma opinião da autora.
(C) Incorreta. É satisfatório para a autora ver que a personalidade das
personagens se mantém.
(D) Correta. O fato de as personagens apresentarem certo amadurecimento
não é uma opinião da resenhista, e sim uma constatação.

\item (A) Correta. A entrevistada considera o SUS imprescindível para o país.
(B) Incorreta. A jornalista não exprime sua opinião na matéria.
(C) Incorreta. A reportagem fala sobre a importância do SUS no ponto de
vista de uma especialista que o defende, o que pode não ser a opinião de
todas as pessoas.
(D) Incorreta. A jornalista reproduz, por citação direta e indireta, a
opinião da especialista.

\item (A) Incorreta. É objetivo da campanha conscientizar as novas gerações
sobre o bullying.
(B) Incorreta. A campanha visa conscientizar adolescentes e jovens, mas
não necessariamente o bullying é responsabilidade apenas desse grupo.
(C) Incorreta. Esta é uma opinião do especialista, e não objetivo da
campanha.
(D) Correta. Sobre a campanha especificamente, a informação trazida no
texto é de que foi criada pela Safernet em parceria com Instagram e
Facebook.
\end{enumerate}

\colorsec{Português — Módulo 6 — Texto}

\begin{enumerate}
% número 1
\item (A) Incorreta. Em nenhum momento a tirinha dá a impressão de que a
atendente não sabe o que é baguete.
(B) Correta. O humor está presente na confusão de interpretações que se fez.
(C) Incorreta. A atendente ouve a pergunta, porém entende-a de forma
ambígua.
(D) Incorreta. Mesmo que a atendente fizesse uma piada proposital, o
humor desta residiria no fato de a pergunta ter sido compreendida de forma equivocada.
\end{enumerate}

\colorsec{Português — Módulo 6 — Treino}

\begin{enumerate}
\item (A) Incorreta. A origem do termo é obscura, mas esse é um fato sobre a
palavra e não o que dá o humor do texto.
(B) Incorreta. O texto explica que é um eufemismo para a morte, mas não é
isso que provoca o humor.
(C) Incorreta. O fato de ser uma possível onomatopeia não provoca por si
só o efeito de humor ao texto.
(D) Correta. O efeito de humor é apresentado na linha fina do texto,
quando a pessoa informa que há décadas fica pensando no ``bendito do
beleléu''.

\item (A) Incorreta. Apesar de as feições do homem serem de fato engraçadas, o
humos não emerge disso no meme.
(B) Incorreta. Os dois animais não aparecem em comparação direta.
(C) Correta. A relação entre imagem e texto é o que faz emergir o humor
do meme.
(D) Incorreta. Essa relação não se estabelece no meme, nem pela imagem,
nem pelo texto verbal.

\item (A) Incorreta. Não há ironia no texto verbal.
(B) Incorreta. O verbo no imperativo, pelo contrário, faz uma referência
direta ao leitor, pedindo uma atitude realmente útil.
(C) Correta. O homem forte vestido de fada e a relação da imagem dele com
o texto verbal é o que gera humor.
(D) Incorreta. De fato, a cor amarela é ligada à educação no trânsito.
Mas não há equívoco proposital no uso da cor rosa nesse caso.
\end{enumerate}

\colorsec{Português — Módulo 7 — Treino}

\begin{enumerate}
\item (A) Incorreta. A resposta, dada no primeiro parágrafo, não é a opinião do
autor, e sim uma explicação técnica.
(B) Incorreta. A pergunta é assertiva é não tem o intuito de gerar
reflexão, uma vez que a resposta já é fornecida no primeiro parágrafo.
(C) Incorreta. A expressão está entre aspas porque ``pai de pet'' é um
termo comum apenas a pessoas que o conhecem.
(D) Correta. O título é uma pergunta pertinente e de senso comum, pois
muitas pessoas consideram pets como filhos.

\item (A) Incorreta. Apenas o Título 2 apresenta o ponto de vista do veículo.
(B) Correta. O Título 2 apresenta o ponto de vista do veículo ao
confirmar que pessoas são pais de pet, admitindo que há uma rotina para
essas pessoas.
(C) Incorreta. O Título 1 apresenta o tema da matéria por meio de uma
pergunta, sem se posicionar a respeito dela.
(D) Incorreta. O Título 2 apresenta sim um ponto de vista a respeito do
tema da matéria.

\item (A) Incorreta. O trecho apresenta uma possibilidade real de acontecer,
mediante um fato apresentado.
(B) Incorreta. O trecho apresenta um fato constatado após um a pesquisa.
(C) Incorreta. Trata-se de um fato que Guy Ryder é diretor-geral da OIT e
que ele destacou esse ponto sobre os dados.
(D) Correta. É um posicionamento do diretor-geral da OIT que não podemos
ficar parados diante dos dados apresentados sobre o trabalho infantil.
\end{enumerate}

\colorsec{Português — Módulo 8 — Treino}

\begin{enumerate}
\item (A) Incorreta. O advérbio de negação, aqui, não nega o posicionamento do
enunciador da campanha, pelo contrário, ele o afirma.
(B) Correta. Embora seja um advérbio de negação, enquanto modalizador do
discurso ele funciona como um reforço de certeza, afirmação, no
enunciado de que Bullying não é legal.
(C) Incorreta. Não existe uma condição para que o enunciador considere o
bullying legal ou não.
(D) Incorreta. Não é uma possibilidade o bullying ser ou não legal.

\item (A) Incorreta. O especialista coloca apenas um posicionamento: ele é a
favor da proibição em alguns casos.
(B) Incorreta. O especialista não se contradiz. Ele é contraditório à
proibição dos jogos.
(C) Correta. O objetivo do especialista é demarcar para o leitor que irá
começar a apresentar seu posicionamento dali adiante.
(D) Incorreta. O especialista fala apenas sobre um assunto: ele é contra
a proibição, e justifica seu posicionamento.

\item (A) Correta. Ao trocar o verbo ``deveria'', que exprime uma
obrigatoriedade, por ``poderia'', a frase toma o sentido de condição, ou
seja, poderiam ou não ser conscientizadas a depender de alguma
circunstância.
(B) Incorreta. O verbo ``poderia'' exprime possibilidade mediante
condição. O verbo ``dever'' já está representando obrigatoriedade.
(C) Incorreta. O verbo ``poderia'' não geraria contradição, haveria
apenas modificação de sentido.
(D) Incorreta. O verbo ``poderia'' não exprime necessidade ou obrigação,
se inserido nesse contexto.
\end{enumerate}

\colorsec{Português — Módulo 9 — Texto}

\begin{enumerate}
% número 9
\item (A) Incorreta. A figura de linguagem presente na frase é a gradação.
(B) Correta. Costuma-se utilizar o termo "batata da perna" para
referir-se à panturrilha por comparação.
(C) Incorreta. A figura de linguagem presente na frase é a sinestesia.
(D) Incorreta. A figura de linguagem presente na frase é a metonímia.
\end{enumerate}

\colorsec{Português — Módulo 9 — Treino}

\begin{enumerate}
\item (A) Incorreta. Onomatopeia é a representação gráfica dos sons.
(B) Correta. Aliteração é a repetição de sons consonantais idênticos ou
semelhantes para sugerir acusticamente algum elemento, ato, fenômeno.
(C) Incorreta. Pleonasmo é uma figura de linguagem que expressa a
redundância.
(D) Incorreta. Metáfora é uma comparação implícita entre duas coisas que
têm características em comum.

\item (A) Correta. A campanha argumenta comparando a proteção ao Meio Ambiente
a uma ideia, representada visualmente pela lâmpada.
(B) Incorreta. O eufemismo utiliza-se de expressões para atenuar uma
ideia tida como agressiva ou desagradável.
(C) Incorreta. A hipérbole é o exagero proposital para enfatizar uma
ideia ou emoção.
(D) Incorreta. A personificação consiste em atribuir características
humanas a seres e objetos inanimados.

\item (A) Correta. Metáfora é uma comparação sem uso de conector, como acontece
na passagem.
(B) Incorreta. A personificação consiste em atribuir características
humanas a seres e objetos inanimados.
(C) Incorreta. A catacrese ocorre quando uma palavra é usada de forma
aproximada, quando não se tem um termo mais específico.
(D) Incorreta. A criatura a que o texto se refere é o 14Bis, avião
construído por Santos Dummont. Não ocorre metonímia.
\end{enumerate}

\colorsec{Português — Módulo 10 — Treino}

\begin{enumerate}
\item (A) Incorreta. Não há referência à profissional, apenas ao sistema.
(B) Correta. Os pronomes referem-se ambos ao SUS.
(C) Incorreta. Os pronomes fazem referência à importância do SUS.
(D) Incorreta. Os pronomes fazem referência ao último substantivo do qual
se fala, no caso, o SUS; e concordam, inclusive, com ele.

\item (A) Incorreta. O pronome refere-se a alunos e professores.
(B) Incorreta. A fala de Ruth Rocha não faz qualquer menção a livros no
plural. O pronome acompanha o substantivo em número.
(C) Incorreta. O pronome refere-se a alunos e professores, todos têm que
gostar do livro.
(D) Correta. O pronome faz referência a alunos e professores, o que é
reforçado pelo fato de o verbo estar na 3ª pessoa do plural.

\item (A) Incorreta. Não se pode afirmar que no momento da entrevista havia um
prato feito próximo a ele.
(B) Incorreta. O texto não informa que a receita foi patenteada por
Ricardo ou pelo livro, pelo contrário, a ideia do livro é divulgá-la.
(C) Correta. Trata-se da construção de um processo de coesão referencial
catafórica.
(D) Incorreta. Caso o intuito de Ricardo fosse fazer com que o leitor se
sentisse próximo à receita, o pronome mais adequado seria ``essa''.
\end{enumerate}

\colorsec{Português — Módulo 11 — Texto}

\begin{enumerate}
% número 7
\item (A) Incorreta. Ao dizer que a jogada foi muito difícil, exprime-se um
julgamento de valor, que não vem ao caso, uma vez que a expressão faz
referência ao parágrafo anterior.
(B) Incorreta. Ao dizer que a jogada foi muito fácil, exprime-se um
julgamento de valor, que não vem ao caso, uma vez que a expressão faz
referência ao parágrafo anterior.
(C) Correta. Gaetaninho driblou a mãe com perfeição, conforme descrito
no parágrafo anterior.
(D) Incorreta. O fato de driblar a mãe com perfeição não significa que
Gaetaninho sabia jogar como um profissional.
\end{enumerate}

\colorsec{Português — Módulo 11 — Treino}

\begin{enumerate}
\item (A) Incorreta. A canção não faz menção à condição social do eleitor, até
mesmo porque uma campanha política sempre visa alcançar a todos.
(B) Incorreta. A canção não menciona nada sobre o contexto econômico do
leitor ou mesmo da época de modo geral.
(C) Correta. A letra faz uma alusão ao contexto histórico, ou seja, à
época, em que foi escrita. Isso pode ser percebido por ``terra do leite
grosso'', que era considerada Minas Gerais, ``bota cerca no caminho'',
quando a maioria das cidades ainda eram rurais, e ``colosso'', um termo
que praticamente caiu em desuso nos dias de hoje.
(D) Incorreta. Embora a letra mencione algumas regiões, o candidato era
para presidente, ou seja, alcançava todo o Brasil.

\item (A) Incorreta. As variações culturais ocorrem de acordo com a cultura dos
falantes, e muitas vezes estão relacionadas a aspectos geográficos.
(B) Incorreta. As variações históricas tratam das mudanças ocorridas na
língua com o decorrer do tempo e não de acordo com a faixa etária.
(C) Correta. As variações sociais são as diferenças de acordo com o grupo
social do falante, incluindo a faixa etária. No trecho, o humor é dado
principalmente pela inocência da criança, que responde de acordo com seu
repertório.
(D) Incorreta. As variações estilísticas remetem ao contexto que exige a
adaptação da fala ou ao estilo dela. Aqui, pai e filho conversam
informalmente apenas.

\item (A) Correta. O entrevistado é venezuelano; portanto, é esperado e
aceitável que não utilize a norma padrão da língua portuguesa. No
trecho, ele esquece de conjugar o verbo ``morava'' em ``Sempre cozinhei
este prato quando Venezuela\ldots{}''.
(B) Incorreta. O entrevistado ainda carrega resquícios de seu idioma
materno, e isso independe de seu contexto social atual.
(C) Incorreta. A faixa etária do entrevistado não importa para o tipo de
deslize ocorrido na frase.
(D) Incorreta. Não fica claro qual é a classe econômica do entrevistado e
isso não importa para o tipo de deslize ocorrido na frase.
\end{enumerate}

\colorsec{Simulado 1}

\begin{enumerate}
\item (A) A intenção é a de certeza.
(B) O modalizador em questão é o verbo decidi, que promove a intenção de
certeza do enunciador.
(C) A intenção de condição aprece no segundo quadrinho, por meio do
modalizador ``só'', quando o enunciador afirma que ``é só todo mundo
ficar em casa que isso passa logo''.
(D) Em nenhum dos quadrinho há qualquer intenção de negação por parte do
enunciador.

\item (A) Incorreta. O prato não é feito somente em celebrações. No texto mesmo
é informado que Ricardo faz o prato no Brasil.
(B) Incorreta. O prato é sinônimo de boas lembranças para Ricardo, não
necessariamente na Venezuela.
(C) Incorreta. Ricardo costumava fazer o prato em reuniões com seus
familiares e amigos, mas o texto não informa que é tipicamente feito
somente nessas ocasiões.
(D) Para Ricardo, o prato remete às boas memórias que tinha das reuniões
com amigos e familiares na Venezuela, com música e cerveja.

\item (A) Incorreta. O título não trata especificamente da história de Ricardo.
Ele é apenas um exemplo.
(B) Incorreta. O título não trata apenas das fronteiras entre um país e
outro, mas sim dos pratos de cada lugar que aparecem no livro.
(C) Incorreta. Não se pode afirmar que esse é o nome do livro.
(D) O título da matéria, ``Sabores sem fronteiras'', refere-se ao fato de
o livro trazer receitas do mundo todo.

\item (A) Correta. A pessoa utiliza uma hipérbole ao dizer que é insuportável
viajar nos ônibus daquela linha, pois são pequenos e sem ar.
(B) Incorreta. Metáfora é uma comparação, o que não ocorre no comentário.
(C) Incorreta. Catacrese é uma figura de palavra que ocorre quando não há
um termo específico para designar um conceito.
(D) Incorreta. Personificação é o ato de dar características humanas a
seres e objetos inanimados.

\item (A) Imprescindível quer dizer que o SUS é absolutamente necessário, ou
seja, indispensável.
(B) Incorreta. Irrecusável é a característica do que não se pode recusar,
o que não é o caso do SUS no contexto.
(C) Incorreta. Incontestável é a característica do que não se contesta, o
que não é o caso do SUS no contexto.
(D) Incorreta. Insustentável significa que o SUS não se manteria, não se
sustentaria, o que não é o caso do contexto.

\item (A) Incorreta. A canção não tem como objetivo incentivar as pessoas a
votarem, mas sim de divulgar um candidato.
(B) Correta. A canção ressalta as qualidades do candidato, a fim de
convencer as pessoas a votarem nele.
(C) Incorreta. Seu Toninho é um nome que representa o eleitor de modo
geral.
(D) Incorreta. Apenas o candidato é paulista e não os eleitores de modo
geral.

\item (A) Incorreta. O quadrinho não menciona o que poderia apagar o sol, e sim
o que aconteceria aos outros astros se ele se apagasse.
(B) Incorreta. Embora o fim do Sol seja uma hipótese de extinção da vida
na Terra, a temática do quadrinho são as consequências do apagão do Sol.
(C) Correta. O quadrinho explica, de forma lúdica, que sem o Sol nenhum
planeta se manteria vivo, pois estão todos os três girando em torno do
Sol.
(D) Incorreta. A extinção da vida na Terra poderia ser causada pelo
apagão do sol, mas não é a única hipótese.

\item (A) Incorreta. O texto pode ser consumido por adultos, mas não é o
público-alvo.
(B) O infográfico apresenta imagens com traços leves e lúdicos, bem como
uma fonte semelhante à letra cursiva. Esse tipo de imagem é mais comum
em meios infanto-juvenis. Além disso, os textos são claros e dão pistas
de que são destinados à pessoas mais vulneráveis, que precisam do apoio
de um adulto, por exemplo.
(C) Incorreta. Qualquer jovem ou criança é o público do texto, não
especificamente alunos de escolas públicas.
(D) Incorreta. A linguagem usada para alcançar adultos e idosos poderia
ser mais formal e apresentar outros tipos de imagem, mais densas, por
exemplo.

\item (A) Incorreta. O autor não apresenta em sua obra trechos da obra original
Chapeuzinho Vermelho.
(B) Incorreta. O autor não faz menção à obra original conhecida,
Chapeuzinho Vermelho.
(C) Correta. Chapeuzinho Amarelo é a história de Chapeuzinho Vermelho
recontada sob a perspectiva de Chico Buarque, portanto, pode-se dizer
que é uma paráfrase.
(D) Incorreta. A história Chapeuzinho Amarelo não cita de nenhuma forma a
obra original conhecida, Chapeuzinho Vermelho.

\item (A) Incorreta. O trecho representa uma verdade, segundo a especialista.
(B) Correta. Neste trecho fica implícito que o fato de o professor ser
importante é uma opinião da autora.
(C) Incorreta. O trecho apresenta um fato sobre a autora.
(D) Incorreta. O trecho apresenta um fato sobre a autora.

\item (A) Correta. O lobo que aparece neste trecho o um lobo presente na
memória coletiva, consagrado pela história da Chapeuzinho Vermelho, o
que se justifica pela descrição do animal.
(B) Incorreta. A ideia não era reforçar o animal em que ela estava
pensando, e sim especificá-lo.
(C) Incorreta. Embora ocorra aliteração, o intuito do trecho não é
provocar humor com essa figura de som.
(D) Incorreta. A ideia é fazer com que o leitor pense no lobo em que ela
estava pensando, ou seja, o da Chapeuzinho Vermelho.

\item (A) Incorreta. A metáfora é uma comparação implícita entre duas coisas
que têm características em comum.
(B) Correta. A campanha utiliza-se de um eufemismo, por meio da junção do
recurso visual da cova (sentido literal) com o verbo enterrar (sentido
conotativo), para dar o sentido de humor e argumentar contra o discurso
de ódio.
(C) Incorreta. A hipérbole é um exagero proposital para enfatizar uma
ideia ou emoção.
(D) Incorreta. A personificação é a atribuição de características humanas
a seres inanimados ou animais.

\item (A) Incorreta. As variações culturais ocorrem de acordo com a cultura dos
falantes, e muitas vezes estão relacionadas a aspectos geográficos.
(B) Correta. As variações históricas tratam das mudanças ocorridas na
língua com o decorrer do tempo. O texto nitidamente foi escrito décadas
atrás, o que se pode notar pelo uso de palavras já ultrapassadas, como
tasca, e a menção ao programa Roda Viva no presente, como se ainda
existisse.
(C) Incorreta. As variações sociais são as diferenças de acordo com o
grupo social do falante, como profissão, faixa etária, entre outros.
(D) Incorreta. As variações estilísticas remetem ao contexto que exige a
adaptação da fala ou ao estilo dela.

\item (A) Incorreta. Embora as crianças sejam os pacientes de pediatras, a
campanha deixa claro que não se referem a ``seus pacientes'', mas sim
``seus filhos''
(B) Incorreta. O público de modo geral não necessariamente possui filhos.
(C) Incorreta. Embora a vacina se destine às crianças, a campanha se
dirige aos seus pais.
(D) ``Seu'' é um pronome possessivo que se refere aos filhos das pessoas
às quais a campanha é dirigida

\item (A) Incorreta. O texto não deixa claro o conhecimento das pessoas em
geral sobre a aparência da família Real, porém, não é isso que a
pesquisadora usa para justificar a pesquisa.
(B) Incorreta. Em nenhum momento o texto menciona que pessoas não
acreditam na existência da família Real.
(C) Incorreta. A pesquisadora examinou os restos mortais dos membros da
família Real no intuito de desenvolver sua pesquisa, e não porque é
importante fazer isso.
(D) Correta. Segundo a pesquisadora, o principal legado da pesquisa é
fazer com que esses personagens históricos do século 19 sejam vistos
como pessoas como nós, o que realmente eram, e que tinham a vida deles
no passado.

\item (A) Correta. O infográfico traz dicas de consumo consciente de modo
geral, que impactam no bem-estar de todos no planeta.
(B) Incorreta. O infográfico traz dicas variadas de como zelar pelo
planeta, e não por florestas especificamente.
(C) Incorreta. Embora o consumo desenfreado e inconsequente ocasione
sérios riscos ao planeta, a temática tratada no infográfico é consumo
consciente.
(D) Incorreta. O infográfico não fala apenas de ecologia, e sim de
consumo consciente.
\end{enumerate}

\colorsec{Simulado 2}

\begin{enumerate}
\item (A) Incorreto. Como a campanha foi produzida em época de pandemia, a
hashtag \#fiqueemcasa incentivava as pessoas a não saírem, porém, este
não é o objetivo da campanha.
(B) Incorreta. A campanha orienta as pessoas as saírem apenas se
necessário, mas o objetivo é respeitar as leis de trânsito.
(C) Incorreta. A campanha incentiva as pessoas a respeitarem as leis de
trânsito. Como foi veiculada em época de pandemia, o texto menciona sair
de casa apenas se necessário.
(D) Correta. A campanha é uma iniciativa da Polícia Rodoviária Federal e
visa incentivar as pessoas a respeitarem as leis de trânsito.

\item (A) Incorreta. É parte do trabalho do Legislativo buscar a solução para
as demandas.
(B) A notícia trata da reclamação da população sobre manutenção do
asfalto de várias ruas e avenidas de Sorriso. Damiani, o vereador,
apenas é o porta-voz entrevistado.
(C) Incorreta. O vereador é o porta-voz dos pedidos da população.
(D) Incorreta. O vereador ficou encarregado de apresentar as demandas
para o Executivo.

\item (A) Incorreta. A mãe se emociona com o presente do filho.
(B) Incorreta. A mãe se emociona com o tipo de presente caro.
(C) O humor da tira é o fato de o presente ser um botijão de gás que, de
tão caro, emocionou a mãe.
(D) Incorreta. O presente é incomum, mas aqui especificamente o humor
está no fato de ser um botijão de gás, que custa muito caro.

\item (A) O trecho contextualiza o leitor com relação à personalidade de Mia: o
que ela pensa, como ela pensa e os dilemas pelos quais ela está
passando.
(B) Incorreta. O trecho não deixa claro o tempo cronológico, embora o
diário seja datado. Essa informação só é percebida no decorrer da obra.
(C) Incorreta. O trecho não apresenta personagens secundários, apenas Mia
falando sobre a mãe.
(D) Incorreta. O trecho não apresenta o local onde acontece a história.

\item (A) Incorreta. Embora o coração quase sempre esteja ligado à paixão,
neste contexto ele está relacionado à ansiedade.
(B) A expressão ``prepare seu coração'' está relacionada à sensação de
palpitação causada pela ansiedade.
(C) Incorreta. Como o próprio enunciado explica, a expressão é usada em
sentido conotativo, ou seja, não literal.
(D) Incorreta. Como o próprio enunciado explica, a expressão é usada em
sentido conotativo, ou seja, não literal.

\item (A) Incorreta. Embora os macacos aparentemente estivessem discutindo no
primeiro quadrinho, a linguagem é humorística e leve.
(B) Correta. O texto adota uma linguagem leve e humorística.
(C) Incorreta. O terceiro quadrinho propõe uma fala mais informativa da
personagem, mas a tira é de humor.
(D) Incorreta. Em nenhum momento o texto aparenta qualquer indício de
tristeza ou melancolia.

\item (A) Incorreta. O trecho e a forma de se expressar não tem relação com a
classe econômica dos personagens envolvidos, mas sim com o período
histórico em que o texto foi criado
(B) Correta. O texto original foi criado em meados de 1940, e por isso,
retrata a variação linguística histórica, com palavras mais rebuscadas.
Por exemplo: Obsequiava, que no contexto quer dizer presentear.
(C) Incorreta. O trecho e a forma de se expressar não tem relação com a
contexto social dos personagens envolvidos, mas sim com o período
histórico em que o texto foi criado
(D) Incorreta. O trecho e a forma de se expressar não tem relação com a
origem geográfica dos personagens envolvidos, mas sim com o período
histórico em que o texto foi criado

\item (A) Incorreta. A pesquisa é da arqueóloga que é sobre a família Real
brasileira.
(B) O trecho é um texto de divulgação científica cujo objetivo é
apresentar um trabalho de pesquisa e seus resultados.
(C) Incorreta. O texto tem a intenção de divulgar um trabalho de pesquisa
e seus resultados.
(D) Incorreta. O texto de divulgação científica é informativo, e não
opinativo.

\item (A) Incorreta. Os trechos não incentivam o leitor a buscar seus sonhos.
(B) Correta. Tanto no trecho de ``O pequeno príncipe'' quando no trecho
de ``Extraordinário'' a mensagem que se passa é de uma relação gentil
com as pessoas, a fim de conquistá-las.
(C) Incorreta. Os trechos falam de relações de afeto, e não de esperança.
(D) Incorreta. Os trechos não dão lição de moral relacionada a
aprendizado com os erros.

\item (A)
  Como a própria estudante informa, ``o novo ensino médio assustou
  bastante no começo, mas a coordenação da minha escola fez com que tudo
  ficasse mais leve''.
(B)
  Incorreta. Embora fossem muitas disciplinas, a escola auxiliou os
  estudantes e tudo ficou mais leve.
(C)
  Incorreta. A estudante achou confuso de início, mas gostou da
  proposta.
(D)
  Incorreta. A estudante menciona dificuldades no início, mas acabou
  gostando da proposta.

\item (A)
  Incorreta. Embora no trecho o objetivo seja apresentar um argumento
  para a tese de que é necessário o uso do drone, o pesquisador o
  compara a floresta ao organismo humano a fim de facilitar a
  compreensão do leitor.
(B)
  Correta. Para facilitar o entendimento do leitor quanto à importância
  do drone para o reflorestamento, o pesquisador usa uma metáfora,
  comparando a floresta ao nosso organismo.
(C)
  Incorreta. O intuito é facilitar a compreensão do leitor, e não apenas
  gerar uma imagem da floresta na mente dele.
(D)
  Incorreta. O texto, por ser uma artigo de divulgação científica,
  naturalmente é interessante para quem o lê pois o público-alvo com
  frequência são pessoas que têm afinidade com o assunto.

\item (A) Incorreta. O fato de o cliente ser incomum é subjetivo e não é
responsável pelo efeito de humor.
(B) Incorreta. O atendente utiliza uma linguagem adequada ao contexto.
(C) Incorreta. O cliente compreende a pergunta e responde de forma a
adianta-se quanto as opções de bebidas. O atendente que não compreendeu.
(D) Correta. O humor da tirinha se dá na resposta do atendente, que não
compreende a pergunta do cliente, no quadrinho anterior, e responde de
forma literal.

\item (A) Incorreta. A campanha não informa que em Natal é isso que as pessoas
ficam fazendo. Ela usa um momento turístico para persuadir as pessoas as
visitarem a cidade.
(B) O verbo no gerúndio dá a ideia de um presente que está acontecendo.
Na campanha, as pessoas no carro estão fazendo ola, o que gera uma
ambiguidade quando se lê a linha fina abaixo, ``Continue vivendo
experiências'', ou seja, orientando ao turista que continue viajando e
``fazendo ola'' por aí.
(C) Incorreta. ``Fazendo ola'' não é uma frase com o intuito de se fazer
uma descrição visual da imagem.
(D) Incorreta. A campanha tem o objetivo, como um todo, de convidar o
turista a visitar Natal por meio de um texto leve, e um dos recursos é a
escolha léxical.

\item (A) Correta. A carta de reclamação é um gênero textual que expressa uma
indignação ou uma insatisfação de determinado indivíduo. É o caminho
formal para que representantes de bairro, por exemplo, entrem em contato
e informem as demandas da região.
(B) Incorreta. Vídeos na internet podem alcançar muitas pessoas, porém,
não é certeza que os poderes responsáveis por resolver as demandas irão
assistir e responder.
(C) Incorreta. Postagens em redes sociais só alcançam órgãos responsáveis
por atender às demandas quando são direcionadas, porém nem sempre elas
são atendidas por não estarem formalizadas.
(D) Incorreta. O artigo de opinião é um texto veiculado em meios de
comunicação de grande circulação que exprime o ponto de vista do
jornalista a respeito de determinado assunto.

\item (A) Incorreta. O itálico é uma marca gráfica usada para pontuar um
pensamento da personagem. Observa-se que trechos em destaque no diário
são escritos em caixas altas, como em ``E, entre DOIS MILHÕES de caras,
ela foi namorar logo o Sr.~Gianini''.
(B) Correta. O texto é o diário da personagem. Enquanto escreve sobre o
fato de a mãe estar saindo com seu professor de álgebra, a menina pede a
Deus, em pensamento, que ninguém da escola descubra.
(C) Incorreta. O texto é escrito em primeira pessoa, ou seja, a narradora
é a própria personagem.
(D) Incorreta. O trecho não apresenta a personagem conversando com outra
pessoa.
\end{enumerate}

\colorsec{Simulado 3}

\begin{enumerate}
\item (A) Incorreta. O filho não homenageia a mãe, ele a presenteia com um
botijão de gás.
(B) Correta. A charge é uma ilustração que tem como premissa fazer uma
crítica social. Aqui, a crítica social é feita ao alto preço do botijão
de gás, que para a mãe é tão valioso quanto uma joia, por exemplo.
(C) Incorreta. A charge não critica filhos que não homenageiam as mães,
ele mostra uma mãe sendo presenteada.
(D) Incorreta. A charge é um gênero voltado à crítica social. Em nenhum
momento ela incentiva filhos a presentearem as mães.

\item (A) Incorreta. Os drones do artigo são usados em um trabalho de
reflorestamento.
(B) Incorreta. O artigo traz informações de uma pesquisa que utiliza
drones e inteligência artificial para o reflorestamento.
(C) O texto apresenta uma pesquisa sobre inteligência artificial e seus
benefícios para o reflorestamento.
(D) Incorreta. O artigo mostra os benefícios do uso da IA para vencer os
desafios do reflorestamento.

\item (A) Incorreta. O aplicativo é barato e mais simples que os métodos
tradicionais.
(B) Incorreta. O método facilita o trabalho para profissionais com pouco
treinamento.
(C) Correta. O método é mais barato e de fácil utilização por
profissionais com pouco treinamento, conforme explica o trecho: ``O
aplicativo e o método desenvolvidos na pesquisa poderão ser utilizados
gratuitamente por profissionais com pouco treinamento, bastando que
sigam o guia que será lançado em livro até o final do primeiro semestre
de 2023 pelo Pacto pela Restauração da Mata Atlântica em parceria com a
Aliança pela Restauração na Amazônia.''
(D)
  Incorreta. O pesquisador acredita que o método auxiliará o país como
  um todo.

\item (A) Incorreta. Essa alternativa apresenta dados científicos sobre as duas
espécies.
(B) Incorreta. Essa alternativa apresenta dados científicos sobre as duas
espécies.
(C) Correta. O texto da tirinha começa com informações científicas sobre
duas espécies de peixes que vivem em cooperação. NO terceiro quadrinho
da tira ocorre a personificação dos personagens, em que o peixe que vive
de restos se torna o dentista do peixe maior, que se torna o paciente.
(D) Incorreta. Embora contenha teor cômico, o fato que gera o teor de
humor na tirinha é a personificação do Bodião-limpador como dentista e
da Moreia como paciente.

\item (A) Correta. O texto tem como objetivo informar leis e regras a serem
seguidas.
(B) Incorreta. O texto informative não tem a intenção de persuadir o leitor a adotar comportamentos.
(C) Incorreta. O texto informativo não tem a intenção de divertir o leitor.
(D) Incorreta. O texto informativo não tem a intenção de questionar fatos.

\item (A) Correta. A mensagem em textos normativos é assertiva pois tem caráter
informativo.
(B) Incorreta. A mensagem é assertiva e o documento é incontestável.
(C) Incorreta. A mensagem não é específica, é assertiva.
(D) Incorreta. A mensagem é assertiva, portando, bem direcionada e
correta.

\item (A) Incorreta. O texto direciona-se a todas as pessoas.
(B) Correta. A Declaração Universal dos Direitos Humanos é para todas as
pessoas.
(C) Incorreta. O texto não se direciona a acadêmicos apenas, mas sim ao
  público em geral.
(D)  Incorreta. O texto não se direciona apenas a políticos, mas ao público
  em geral.

\item (A) Correta. O texto normativo apresenta uma linguagem formal,
respeitando a norma padrão da língua, para que alcance de forma
assertiva todas as pessoas.
(B) Incorreta. A linguagem é formal e respeita a norma padrão da língua.
(C) Incorreta. A linguagem é formal, mas não técnica. Por linguagem
técnica entende-se a variação linguística típica de alguns grupos,
especialmente relacionados a profissões, como exemplo, em manuais
técnicos.
(D) Incorreta. A linguagem poética é encontrada em textos literários e
geralmente segue a estrutura em estrofes e versos.

\item (A) Incorreta. 
(B) Incorretas. O menino não é irônico no texto.
(C) Correta. A fala da menina é irônica, pois quando ela afirma que é
isso que a preocupa, ela está justamente contrariando a orientação do
menino.
(D) Incorreta. O menino não é irônico no texto.

\item (A)
  Correta. O objetivo da notícia é informar as pessoas sobre a aprovação
  de um projeto de lei.
(B)
  Incorreta. O projeto de lei é sobre o incentivo à adoção de animais, e
  não a notícia incentiva a prática.
(C)
  Incorreta. A notícia não critica o novo projeto de lei.
(D)
  Incorreta. A notícia não informa sobre pet shops e clínicas
  veterinárias.

\item (B)
  Incorreta. O autor do projeto tem um ponto de vista que também se
  embasa no mesmo argumento usado na justificativa para o projeto.
(C)
  Incorreta. O governo de modo geral não se manifestou sobre o projeto.
(D)
  Incorreta. Nem toda a população pode estar de acordo com a
  justificativa do projeto.

\item (A) Correta. O criador do projeto de lei mostra-se apoiador da causa de
animais abandonados.
(B) Incorreta. O autor do projeto não critica a causa de animais
abandonados, ele apoia a causa.
(C) Incorreta. O argumento para o projeto de lei demonstra também a
opinião do autor do projeto com relação à causa.
(D) Incorreta. O autor do projeto deixa clara sua opinião sobre a causa.

\item (A) Incorreta. O infográfico não traz informações básicas de modo geral,
são informações específicas do ponto de vista científico.
(B) Incorreta. O infográfico tem caráter informativo. A linguagem
apelativa é típica de propagandas, por exemplo. O texto também pode ser
direcionado e compreendido para qualquer leitor.
(C) Incorreta. O infográfico não apresenta informações técnicas, e sim
científicas, e é direcionado a qualquer pessoa.
(D) O infográfico é produzido pela Faculdade de Medicina da Universidade
Federal de Minas Gerais e apresenta informações científicas com um texto
leve que pode ser compreendido por qualquer leitor.

\item (A) Incorreta. O infográfico é produzido pela UFMG, mas não se pode
afirmar que em uma matéria médicos afirmariam alguma coisa.
(B) Correta. O infográfico traz informações relacionadas à primeira
infância e pré-adolescência e fala sobre as consequências do uso precoce
e excessivo das telas. A opção que apresenta menção a esses três itens
abordados é a b.
(C) Incorreta. O infográfico fala especificamente sobre crianças.
(D) Incorreta. O infográfico não menciona apenas a primeira infância; ele
traz informações sobre crianças com menos de 2 anos até 12 anos.
\end{enumerate}

\colorsec{Simulado 4}

\begin{enumerate}
\item (A) Correta. Segundo o artigo, o contato dos bebês com tecnologia digital
deve ser zero, pois os efeitos da luz azul da tela podem implicar
redução de melatonina, que é essencial para o sono, que por sua vez é
importante para o desenvolvimento de crianças com idade inferior a 2
anos.
(B) Incorreta. O texto deixa claro o dano a bebês, mas não menciona que
uso da tecnologia digital é inofensiva a crianças.
(C) Incorreta. O texto orienta que a exposição após os dois anos deve ser
feita com cautela, e não que ela deve ser feita.
(D) Incorreta. Esta é a tese e não o argumento do artigo.

\item (A)
  Incorreta. Estudos realizados pela OMS, SBP, governos dos Estados
  Unidos e Canadá, e não apenas um desses estudos, embasam as
  recomendações dos especialistas.
(B) Correta. Estudos realizados pela OMS, SBP, governos dos Estados
Unidos e Canadá embasam as recomendações dos especialistas.
(C)
  Incorreta. Estudos realizados pela OMS, SBP, governos dos Estados
  Unidos e Canadá embasam as recomendações dos especialistas.
(D)
  Incorreta. As opiniões dos especialistas se embasam em estudos
  realizados pela OMS, SBP, governos dos Estados Unidos e Canadá.

\item (A) O artigo é claro em informar que a tecnologia digital é prejudicial,
especialmente em bebês (crianças com idade inferior a 2 anos).
(B) Incorreta: A mensagem principal não é que os pais sejam cautelosos,
mas sim que permitam zero contato até os 2 anos.
(C) Incorreta. A mensagem do artigo é exatamente o contrário.
(D) Incorreta. O Artigo deixa claro os efeitos prejudiciais no
desenvolvimento das crianças.

\item (A) O artigo não utiliza linguagem técnica para expor seu argumento.
(B) O artigo não utiliza não expõe opiniões próprias ou sentimentos para
expor seu argumento.
(C) Correta. embora o texto fale sobre pesquisas científicas e e assuntos
relacionados a saúde, o do artigo passa uma mensagem simples e direta
aos leitores.
(D) O artigo expor seu argumento de forma clara, sem possibilidade de
interpretação.

\item (A) Correta. O conceito de meme é ressignificar uma mensagem, geralmente,
em tom jocoso. No exemplo, temos um texto que se referia à construção de
estádios e infraestrutura para a copa do mundo em 2014, em que as
pessoas que não apoiavam o evento se utilizavam do jargão ``não vai ter
copa!'', e esse jargão foi utilizado como piada em um ``anúncio'' de
imóvel que não possui o cômodo ``copa''.
(B) Incorreta. Nos textos de referência, a notícia serve de inspiração
para o meme.
(C) Incorreta. Nos textos de referência, a notícia o meme estão
relacionados.
(D) Incorreta. Nos textos de referência, a notícia o meme estão
relacionados.

\item (A) Incorreta. O meme não gera um contexto adicional, mas sim pega uma
parte da notícia e a utiliza em outro contexto, com outra mensagem.
(B) Incorreta. O meme em si é feito para viralizar, porém o engajamento
ocorre em torno da piada, e não no texto original.
(C) Correta. Geralmente o meme surge como uma paródia ou crítica à uma
notícia ou fato do momento. O meme do texto 2 se relaciona à notícia
fazendo uma paródia entre ``copa do mundo'' e o cômodo ``copa''.
(D) Incorreta. O meme da coletânea não serve como complemento para a
notícia.

\item (A) Incorreta. O autor utiliza a terceira pessoa do singular para
referir-se ao ``rico pomposo''. Mesmo assim, o pronome é oculto.
(B) Incorreta. O autor utiliza a terceira pessoa do singular no feminino,
porém o pronome é oculto.
(C) Incorreta. O autor não utiliza terceira pessoa do plural.
(D) Correta. O autor se inclui em uma ação comum a todas as pessoas e
acaba referindo a si mesmo na primeira pessoa do plural: nós, em ``Vemos
um rico pomposo''.

\item (A) Correta. O ``se'' é um pronome reflexivo. Nessa construção, o sujeito
é ``tanta gente orgulhosa'' e comete a ação sobre si mesmo.
(B) Incorreta. Os pronomes demonstrativos são usados para apontar a
relação de distância (de lugar ou de tempo entre o nome ao qual se
referem e as pessoas do discurso.
(C) Incorreta. Os pronomes indefinidos são aqueles que se referem à
terceira pessoa do discurso de forma vaga, imprecisa e genérica.
(D) Incorreta. Os pronomes interrogativos são aqueles que utilizamos nas
construções de perguntas diretas ou indiretas: que, quem, qual, quanto.

\item (A) Incorreta. O cordel dá sentido às coisas e diz que de repente elas
podem perder o sentido.
(B) Incorreta. O autor não fala do propósito das coisas no texto, ele
menciona quando é que as coisas perdem o propósito.
(C) Correta. O cordel fala sobre a temporalidade das coisas, que de
repente se tem tudo o que é material, mas quando a pessoa morre, por
exemplo, tudo pode virar nada.
(D) Incorreta. O cordel fala justamente sobre a temporalidade das coisas.

\item (A) Incorreta. O excesso de flechas para acertar apenas 1 alvo, embora
possa ter um teor cômico, não é o fator principal de humor nessa
tirinha.
(B) Incorreta. O disfarce do alvo, embora possa ter um teor cômico, não é
o fator principal de humor nessa tirinha.
(C) Incorreta. A consequência do erro, na forma como é exposta pelo
segundo personagem, possui teor cômico, mas não é o fator principal de
humor nessa tirinha.
(D) Correta. O fato que causa o efeito de humor é justamente o momento
que o personagem dorme durante a contagem de cordeiros. Essa ``técnica''
é muito utilizada para induzir as crianças ao sono, e na tirinha temos
um personagem com status de herói, que adormece como uma criança.

\item (A) Incorreta. A variação histórica ocorre quando a sentença apresenta
termos que já caíram em desuso.
(B) Incorreta. A variação social é o tipo de linguagem utilizada por
determinado grupo social, que por preferências, atividades e ou nível
socioeconômico adota um linguajar próprio.
(C) Incorreta. A variação estilística ocorre quando o falante altera seu
estilo de linguagem a fim de se adequar a contextos distintos.
(D) Correta. No trecho, a variação linguística está relacionada à
localização geográfica em que a história se passa, ou seja, no Nordeste
do Brasil. A expressão ``deixe de agouro com o menino'' também pode
significar ``não deseje mal para o menino'', em outras regiões do
Brasil.

\item (A) Incorreta. João Grilo não tem uma boa relação com o patrão, o que se
confirma no trecho seguinte, em ``Muito pelo contrário, ainda hei de me
vingar do que ele e a mulher me fizeram quando estive doente.''
(B) Incorreta. Estar ``pegado'' significa próximo, trabalhando duro, e
não agindo de forma inapropriada.
(C) Correta. Perceba que no trecho anterior João Grilo diz a Chicó que
arranjou um padre para o padeiro sem nem ter sido mandando. Ou seja, ele
trabalhou arduamente para o patrão.
(D) Incorreta. João Grilo deixa claro o quanto trabalha duro, por isso
estava tão chateado com o patrão, por não ter reconhecimento nem quando
estava doente.

\item (A) Incorreta. Embora o texto informe sobre um evento fictício, o objetivo
  da campanha é chamar os jovens, especialmente os que conhecem o anime,
  para a campanha de vacinação.
(B) Incorreta. O texto promove a campanha de vacinação, e não a vacina em
si.
(C) Correta. O texto é um anúncio de uma campanha de vacinação para
crianças e jovens a partir de 12 anos de idade.
(D) Incorreta. O texto não é um manual de instrução e sim uma campanha
publicitária.

\item (A) Correta. O evento divulgado é uma campanha de vacinação para crianças
e jovens com idade a partir de 12 anos.
(B) Incorreta. O evento trata de uma campanha de vacinação, direcionada à
saúde e bem-estar do público-alvo.
(C) Incorreta. O anime utilizado para promover o evento pode estar
relacionado a games e inovação, mas o objetivo do evento é promover
saúde e bem-estar.
(D) Incorreta. A campanha utiliza uma imagem familiar aos jovens para
promover a vacinação direcionada à saúde e bem-estar do público-alvo.
\end{enumerate}
