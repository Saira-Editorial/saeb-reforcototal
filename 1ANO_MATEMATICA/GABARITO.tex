\chapter{Respostas}
\pagestyle{plain}
\footnotesize

\pagecolor{gray!40}

\colorsec{Matemática – Módulo 1 – Treino}

\begin{enumerate}
\item
(A) Incorreta. O número apresentado não representa quantidade ou contagem.
(B) Incorreta. O número apresentado não apresenta necessariamente a ordem de matriculas feitas na escola.
(C) Incorreta. O número não representa nenhuma unidade como metro ou grama.
(D) Correta. Esse número é um código de identificação da aluna Júlia nos registros da escola.

\item
(A) Correta. Carlos tem a mesma quantidade de bolinhas que alguns amigos, porém todas são amarelas.
(B) Incorreta. Cristiano só tem bolinhas verdes, porém tem menos do que Carlos.
(C) Incorreta. Júlio tem bastante bolinhas azuis, porém em menor quantidade do que as amarelas de Carlos.
(D) Incorreta. Ricardo tem poucas bolinhas de cada cor, apesar de ter o mesmo número total de Carlos.

\item
(A)  Incorreta. O aluno pode ter considerado somente o zero, entendendo que não haveria outras ordens preenchidas.
(B)  Incorreta. O aluno pode ter se esquecido de considerar o zero e o cem.
(C)  Incorreta. O aluno pode ter se esquecido de considerar ou o zero ou o cem.
(D)  Correta. O aluno contou estes números: 0, 10, 20, 30, 40, 50, 60, 70, 80, 90 e 100.
\end{enumerate}

\colorsec{Matemática – Módulo 2 – Treino}

\begin{enumerate}
\item
(A)  Incorreta. O aluno pode ter subtraído ao invés de somar.
(B)  Incorreta. O aluno considerou somente as figurinhas de Benício.
(C)  Incorreta. O aluno considerou somente as figurinhas de Bernardo.
(D)  Correta. O aluno somou corretamente os dois valores: 25 + 32 = 57.

\item
(A) Incorreta. Adicionando-se 16 a 15, compõe-se o número 31, que é maior que o número de cestas da partida (30).
(B) Correta. Adicionando-se 16 a 14, compõe-se a exata quantidade de cestas do jogo, ou seja, 30.
(C) Incorreta. Adicionando-se 15 a 14, compõe-se o número 29, que é menor que o número de cestas da partida (30).
(D) Incorreta. Adicionando-se 15 a 13, compõe-se o número 28, que é menor que o número de cestas da partida (30).

\item
(A)  Incorreta. O aluno pode ter confundido o valor do lanche com o valor
  que Vinícius tinha.
(B)  Correta. O aluno acrescentou corretamente o valor do dinheiro que Vinícius tinha
  aos valores que ganhou e, então, retirou o valor do sorvete para
  descobrir o valor do lanche: ([(15 + 12 + 10) - 5] = 32).
(C)  Incorreta. O aluno pode ter se esquecido de retirar o valor do sorvete.
(D)  Incorreta. O aluno acrescentou o valor do sorvete ao invés de retirar.
\end{enumerate}

\colorsec{Matemática – Módulo 3 – Treino}

\begin{enumerate}
\item
(A) Incorreta. A menina de saia amarela é a mais baixa.
(B) Correta. A menina mascando goma tem a maior altura.
(C) Incorreta. O menino de boné é o terceiro mais alto (ou o segundo mais
baixo).
(D) Incorreta. O menino de tênis azul é o segundo mais alto.

\item
(A) Incorreta. A balança é utilizada para medir massa em quilogramas.
(B) Incorreta. O copo de medição é utilizado para medir volume em litros ou mililitros.
(C) Correta. A régua pode ser usada para medir a distância linear entre o tampo da mesa e o chão.
(D) Incorreta. A panela só pode ser utilizada para medir volume.

\item
(A) Incorreta. O aluno pode ter invertido o sentido de leitura do visor.
(B) Incorreta. O aluno entendeu que o ponteiro estava sobre o número 70.
(C) Correta. O aluno percebeu que o ponteiro está deslocado uma unidade
além do 70.
(D) Incorreta. O aluno deve ter pensado que cada marcação vale 2.
\end{enumerate}

\colorsec{Matemática – Módulo 4 – Treino}

\begin{enumerate}
\item
(A) Incorreta. O aluno pode ter se confundido com a primeira segunda-feira.
(B) Incorreta. O aluno pode ter se confundido com o domingo, entendendo que
a segunda-feira é o primeiro dia a aparecer no calendário.
(C) Correta. Seguindo-se a coluna da segunda-feira (a segunda da esquerda para a direita), a última linha dessa coluna é a que contém o número 27.
(D) Incorreta. O aluno pode ter confundido a segunda-feira com o último
dia do mês.

\item
a) Incorreta. O aluno entendeu que deveria seguir a sequência das letras.
b) Incorreta. O aluno não considerou o amanhecer e foi direto ao dia.
c) Incorreta. O aluno pode ter confundido o entardecer com o amanhecer.
d) Correta. A sequência mostra o amanhecer, seguido do dia claro, seguido
do entardecer e seguido da noite estrelada.

\item
a) Incorreta. O aluno só considerou os 15 minutos antes de
completar 7:00 horas.
b) Incorreta. O aluno esqueceu-se de considerar os 15 minutos antes de
completar 7:00 horas.
c) Correta. O aluno percebeu que o balão subiu no mesmo minuto em
que desceu, porém na hora posterior; logo o voo durou uma hora.
d) Incorreta. O aluno somou a hora aos 45 minutos.
\end{enumerate}

\colorsec{Matemática – Módulo 5 – Treino}

\begin{enumerate}
\item
a) Incorreta. O aluno confundiu a cédula de 10 reais com uma moeda de 10 centavos.
b) Incorreta. O aluno confundiu a cédula de 10 reais com uma moeda de 1 real.
c) Correta. A cédula representada é a de 10 reais.
d) Incorreta. O aluno confundiu a cédula de 10 reais com outra cédula, a de 100 reais.

\item
a) Correta. A composição resulta em 100 reais, valor suficiente para pagar pelo chapéu.
b) Incorreta. Apesar de a composição resultar em 75 reais, o valor do chapéu, não existe cédula de 25 reais.
c) Incorreta. A composição resulta em R\$ 30,00, valor insuficiente para pagar pelo chapéu.
d) Incorreta. A composição resulta em R\$ 15,00, valor insuficiente para pagar pelo chapéu.

\item
a) Incorreta. Ana tem R\$ 180,00, valor que não é suficiente para pagar a fatura de R\$ 182,00.
b) Incorreta. Não existe uma moeda de R\$ 2,00 para completar o valor necessário para pagar a fatura.
c) Correta. Ana tem R\$ 180,00 e, com mais uma cédula de R\$ 2,00, completaria o valor para pagar a fatura.
d) Incorreta. Com mais uma cédula de R\$ 5,00, Ana teria R\$ 185,00, valor mais que suficiente para pagar a fatura de R\$ 182,00.
\end{enumerate}

\colorsec{Matemática – Módulo 6 – Treino}

\begin{enumerate}
\item
a) Incorreta. Uma caneta é feita para escrever.
b) Incorreta. Gatos também dormem.
c) Incorreta. Todo navio foi feito para flutuar.
d) Correta. Peixes não podem cantar.

\item
a) Incorreta. É muito provável que faça calor no verão.
b) Correta. É pouco provável que faça frio no verão, ainda que seja
possível.
c) Incorreta. É muito provável que neve caia no inverno, principalmente
no hemisfério Norte do planeta.
d) Incorreta. É certeza que as árvores florescem na primavera.

\item
a) Incorreta. É sempre possível que chova em alguma lugar.
b) Incorreta. Sempre existe a chance de não chover, apesar do tempo fechado.
c) Incorreta. Como o tempo está bem fechado na cena retratada, é muito provável que chova.
d) Correta. O tempo bem fechado indica que a probabilidade de chuva é bem grande.
\end{enumerate}

\colorsec{Matemática – Módulo 7 – Treino}

\begin{enumerate}
\item

\item

\item
\end{enumerate}

\colorsec{Simulado 1}

\begin{enumerate}
\item

\item

\item

\item

\item

\item

\item

\item

\item

\item

\item

\item

\item

\item

\item
\end{enumerate}

\colorsec{Simulado 2}

\begin{enumerate}
\item

\item

\item

\item

\item

\item

\item

\item

\item

\item

\item

\item

\item

\item

\item
\end{enumerate}

\colorsec{Simulado 3}

\begin{enumerate}
\item

\item

\item

\item

\item

\item

\item

\item

\item

\item

\item

\item

\item

\item

\item
\end{enumerate}

\colorsec{Simulado 4}

\begin{enumerate}
\item

\item

\item

\item

\item

\item

\item

\item

\item

\item

\item

\item

\item

\item

\item
\end{enumerate}