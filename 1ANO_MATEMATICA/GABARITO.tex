\chapter{Respostas}
\pagestyle{plain}
\footnotesize

\pagecolor{gray!40}

\colorsec{Matemática – Módulo 1 – Treino}

\begin{enumerate}
\item
(A) Incorreta. O número apresentado não representa quantidade ou contagem.
(B) Incorreta. O número apresentado não apresenta necessariamente a ordem de matriculas feitas na escola.
(C) Incorreta. O número não representa nenhuma unidade como metro ou grama.
(D) Correta. Esse número é um código de identificação da aluna Júlia nos registros da escola.

\item
(A) Correta. Carlos tem a mesma quantidade de bolinhas que alguns amigos, porém todas são amarelas.
(B) Incorreta. Cristiano só tem bolinhas verdes, porém tem menos do que Carlos.
(C) Incorreta. Júlio tem bastante bolinhas azuis, porém em menor quantidade do que as amarelas de Carlos.
(D) Incorreta. Ricardo tem poucas bolinhas de cada cor, apesar de ter o mesmo número total de Carlos.

\item
(A)  Incorreta. O aluno pode ter considerado somente o zero, entendendo que não haveria outras ordens preenchidas.
(B)  Incorreta. O aluno pode ter se esquecido de considerar o zero e o cem.
(C)  Incorreta. O aluno pode ter se esquecido de considerar ou o zero ou o cem.
(D)  Correta. O aluno contou estes números: 0, 10, 20, 30, 40, 50, 60, 70, 80, 90 e 100.
\end{enumerate}

\colorsec{Matemática – Módulo 2 – Treino}

\begin{enumerate}
\item
(A)  Incorreta. O aluno pode ter subtraído ao invés de somar.
(B)  Incorreta. O aluno considerou somente as figurinhas de Benício.
(C)  Incorreta. O aluno considerou somente as figurinhas de Bernardo.
(D)  Correta. O aluno somou corretamente os dois valores: 25 + 32 = 57.

\item
(A) Incorreta. Adicionando-se 16 a 15, compõe-se o número 31, que é maior que o número de cestas da partida (30).
(B) Correta. Adicionando-se 16 a 14, compõe-se a exata quantidade de cestas do jogo, ou seja, 30.
(C) Incorreta. Adicionando-se 15 a 14, compõe-se o número 29, que é menor que o número de cestas da partida (30).
(D) Incorreta. Adicionando-se 15 a 13, compõe-se o número 28, que é menor que o número de cestas da partida (30).

\item
(A)  Incorreta. O aluno pode ter confundido o valor do lanche com o valor
  que Vinícius tinha.
(B)  Correta. O aluno acrescentou corretamente o valor do dinheiro que Vinícius tinha
  aos valores que ganhou e, então, retirou o valor do sorvete para
  descobrir o valor do lanche: ([(15 + 12 + 10) - 5] = 32).
(C)  Incorreta. O aluno pode ter se esquecido de retirar o valor do sorvete.
(D)  Incorreta. O aluno acrescentou o valor do sorvete ao invés de retirar.
\end{enumerate}

\colorsec{Matemática – Módulo 3 – Treino}

\begin{enumerate}
\item

\item

\item
\end{enumerate}

\colorsec{Matemática – Módulo 4 – Treino}

\begin{enumerate}
\item

\item

\item
\end{enumerate}

\colorsec{Matemática – Módulo 5 – Treino}

\begin{enumerate}
\item

\item

\item
\end{enumerate}

\colorsec{Matemática – Módulo 6 – Treino}

\begin{enumerate}
\item

\item

\item
\end{enumerate}

\colorsec{Matemática – Módulo 7 – Treino}

\begin{enumerate}
\item

\item

\item
\end{enumerate}

\colorsec{Simulado 1}

\begin{enumerate}
\item

\item

\item

\item

\item

\item

\item

\item

\item

\item

\item

\item

\item

\item

\item
\end{enumerate}

\colorsec{Simulado 2}

\begin{enumerate}
\item

\item

\item

\item

\item

\item

\item

\item

\item

\item

\item

\item

\item

\item

\item
\end{enumerate}

\colorsec{Simulado 3}

\begin{enumerate}
\item

\item

\item

\item

\item

\item

\item

\item

\item

\item

\item

\item

\item

\item

\item
\end{enumerate}

\colorsec{Simulado 4}

\begin{enumerate}
\item

\item

\item

\item

\item

\item

\item

\item

\item

\item

\item

\item

\item

\item

\item
\end{enumerate}